\subsubsection{Block Identification: \emph{How is a block found?}}\label{subsubsection: block identification}

When the CPU requests a memory address, the cache must \textbf{quickly determine} whether that \hl{address is stored inside it}, and if so, in \hl{which line}. This process depends on the \textbf{cache mapping type} that we studied in the previous section.

\begin{flushleft}
    \textcolor{Green3}{\faIcon{book} \textbf{General Method}}
\end{flushleft}
Regardless of mapping type, the process involves:
\begin{enumerate}
    \item \important{Index}. Determines \emph{where to look} (which set or which specific line in direct-mapped caches).
    \item \important{Tag}. Stored in the cache line; must match the tag bits extracted from the CPU's requested address.
    \item \important{Valid Bit}. Ensures the entry contains real, initialized data.
\end{enumerate}
If \textbf{index matches}, \textbf{tag matches}, and \textbf{valid bit $=$ 1}, we have a \emph{hit}. Otherwise a \emph{miss}.

\begin{flushleft}
    \textcolor{Green3}{\faIcon{tools} \textbf{How it works in each mapping type}}
\end{flushleft}
\begin{itemize}
    \item \important{Direct-Mapped Cache}
    \begin{enumerate}
        \item Use \textbf{index bits} from the address to locate exactly \textbf{one cache line}.
        \item Compare \textbf{tag bits} in that line to the requested address's tag.
        \item Check \textbf{valid bit}.
        \item \textbf{Hit} if both match; otherwise, \textbf{miss}.
    \end{enumerate}
    \item \important{Set-Associative Cache}
    \begin{enumerate}
        \item Use \textbf{index bits} to select the \textbf{set}. Inside that set, there are $n$ \textbf{lines} (ways).
        \item Compare $n$ \textbf{lines} of each way in that set in parallel.
        \item If one matches and \textbf{valid $=$ 1} $\rightarrow$ \textbf{hit}; otherwise, \textbf{miss}.
    \end{enumerate}
    \item \important{Fully Associative Cache}
    \begin{enumerate}
        \item \textbf{No index} bits; the whole cache is \textbf{one big set}.
        \item Compare the requested tag to \textbf{every tag} in the cache in parallel.
        \item \textbf{Hit} if any match with \textbf{valid $=$ 1}; otherwise \textbf{miss}.
    \end{enumerate}
\end{itemize}

\begin{table}[!htp]
    \centering
    \begin{tabular}{@{} l p{24em} @{}}
        \toprule
        Mapping Type & How Block is Found \\
        \midrule
        \textbf{Direct-Mapped}       & Use index to find 1 line, compare its tag.                        \\ [.3em]
        \textbf{Set-Associative}     & Use index to find 1 set, compare tags of $n$ ways in that set.    \\ [.3em]
        \textbf{Fully Associative}   & Compare tag with every line in the cache.                         \\
        \bottomrule
    \end{tabular}
\end{table}