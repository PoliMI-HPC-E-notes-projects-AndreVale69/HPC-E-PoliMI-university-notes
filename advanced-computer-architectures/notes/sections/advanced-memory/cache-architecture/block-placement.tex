\subsubsection{Block Placement: \emph{Where can a block be placed?}}\label{subsubsection: Block placement}

This question is about \hl{how memory addresses map to positions in the cache}. It defines the \textbf{mapping policy} between \textbf{main memory blocks} and \textbf{cache lines}.

\highspace
There are \textbf{three classic cache organizations}:
\begin{enumerate}
    \item \label{def: Direct-Mapped Cache} \definition{Direct-Mapped Cache}. It is the simplest and most intuitive. Each block from memory \textbf{can go in only one location} in the cache.
    \begin{equation}
        \text{Cache Index} = \left(\text{Block Address}\right) \mod \left(\text{Number of Cache Blocks}\right)
    \end{equation}
    This means:
    \begin{itemize}
        \item[\textcolor{Green3}{\faIcon{check-circle}}] Easy and \textbf{fast lookup} (just compute index).
        \item[\textcolor{Red2}{\faIcon{times-circle}}] But high risk of \textbf{conflict misses} (collision): two blocks that map to the same index will evict each other repeatedly.
        \begin{figure}[!htp]
            \centering
            \includegraphics[width=\textwidth]{img/collision-in-hashing-768.png}
            \caption{A great analogy is in a hashing environment, where a collision occurs if a key is the same. In our case, it's the same: same index, collision!\cite{GFG_CollisionResolutionImage_2025}}
        \end{figure}
    \end{itemize}
    \begin{examplebox}[: Direct-Mapped Cache]
        Cache has 8 blocks: indexes from 0 to 7. The memory block of 12 is calculated as follows:
        \begin{equation*}
            12 \mod 8 = 4
        \end{equation*}
        And it can only go in \textbf{cache block 4}.
        \begin{center}
            \begin{tabular}{@{} c c @{}}
                \toprule
                Memory Block & Cache Block \\
                \midrule
                4   & 4 \\ [.3em]
                12  & 4 \\ [.3em]
                20  & 4 \\
                \bottomrule
            \end{tabular}
        \end{center}
        So if we alternate access to blocks 12 and 20, we get repeated evictions, then poor performance.
    \end{examplebox}


    \item \label{def: Fully Associative Cache} \definition{Fully Associative Cache}. A block from memory \textbf{can be placed in any cache line}. There is \hl{no index in the address} (no needed), and the mapping rule is a \textbf{full \texttt{TAG} comparison across all cache lines}. This means:
    \begin{itemize}
        \item[\textcolor{Green3}{\faIcon{check-circle}}] \textbf{No conflict misses} (since any block can go anywhere).
        \item[\textcolor{Green3}{\faIcon{check-circle}}] Ideal flexibility.
        \item[\textcolor{Red2}{\faIcon{times-circle}}] \textbf{Expensive} hardware: must compare TAGs for \emph{all} cache lines. Because to do this, the hardware includes:
        \begin{itemize}
            \item $N$ \textbf{comparators}, each comparing the input tag with the tag stored in one of the $N$ cache lines.
            \item A \textbf{priority encoder} or logic to select the matching line (if any).
        \end{itemize}
        \item[\textcolor{Red2}{\faIcon{times-circle}}] Slower access.
    \end{itemize}


    \item \label{def: n-way set-associative cache} \definition{N-Way Set-Associative Cache}. A compromise between direct-mapped and fully associative.
    \begin{itemize}
        \item Cache is divided into \textbf{sets}.
        \item Each set contains $n$ \textbf{lines} (ways).
        \item A memory block maps to \textbf{exactly one set}, but can be placed in \textbf{any of the $n$ lines} within that set.
    \end{itemize}
    The mapping rule is:
    \begin{equation}
        \text{Set Index} = \left(\text{Block Address}\right) \mod \left(\text{Number of Sets}\right)
    \end{equation}
    This reduces conflict misses \textbf{while keeping lookup cost low} (only compare tags inside the set).

    \begin{examplebox}[: N-Way Set-Associative Cache]
        Imagine a 2-way set-associative cache with 4 sets (each set has 2 blocks). For example, the \textbf{block 12} goes into \textbf{Set 0}:
        \begin{equation*}
            12 \mod 4 = 0
        \end{equation*}
        In Set 0 it can go into \textbf{either Way 0 or Way 1}.
        \begin{center}
            \begin{tabular}{@{} c | c @{}}
                \toprule
                Memory Block & Set Index \\
                \midrule
                0   & 0 \\
                4   & 0 \\
                8   & 0 \\
                12  & 0 \\
                \bottomrule
            \end{tabular}
        \end{center}
        Each of these blocks competes for 2 slots in Set 0, so \textbf{fewer conflicts} than direct-mapped.
    \end{examplebox}
\end{enumerate}
Summary of Block Placement Policies:
\begin{itemize}
    \item \textbf{Direct-Mapped}
    \begin{table}[!htp]
        \centering
        \begin{adjustbox}{width={\textwidth},totalheight={\textheight},keepaspectratio}
            \begin{tabular}{@{} l l l l @{}}
                \toprule
                Mapping Rule & Flexibility & Hardware Cost & Risk of Conflict Miss \\
                \midrule
                $\text{Block Address} \mod \# \text{Blocks}$ & \textcolor{Red2}{\faIcon{circle}} Rigid (1 line)  & \textcolor{Green3}{\faIcon{check-circle}} Simple  & \textcolor{Red2}{\faIcon{circle}} High \\
                \bottomrule
            \end{tabular}
        \end{adjustbox}
    \end{table}


    \item \textbf{Fully Associative}
    \begin{table}[!htp]
        \centering
        \begin{adjustbox}{width={\textwidth},totalheight={\textheight},keepaspectratio}
            \begin{tabular}{@{} p{10em} l l l @{}}
                \toprule
                Mapping Rule & Flexibility & Hardware Cost & Risk of Conflict Miss \\
                \midrule
                Any block to any line (compare all) & \textcolor{Green3}{\faIcon{circle}} Full Freedom  & \textcolor{Red2}{\faIcon{circle}} Expensive  & \textcolor{Green3}{\faIcon{circle}} None \\
                \bottomrule
            \end{tabular}
        \end{adjustbox}
    \end{table}


    \item \textbf{N-Way Set-Associative}
    \begin{table}[!htp]
        \centering
        \begin{adjustbox}{width={\textwidth},totalheight={\textheight},keepaspectratio}
            \begin{tabular}{@{} l l l l @{}}
                \toprule
                Mapping Rule & Flexibility & Hardware Cost & Risk of Conflict Miss \\
                \midrule
                $\text{Block Address} \mod \# \text{Sets}$ & \textcolor{Yellow3}{\faIcon{circle}} Medium  & \textcolor{Yellow3}{\faIcon{circle}} Moderate  & \textcolor{Yellow3}{\faIcon{circle}} Controlled \\
                \bottomrule
            \end{tabular}
        \end{adjustbox}
    \end{table}
\end{itemize}