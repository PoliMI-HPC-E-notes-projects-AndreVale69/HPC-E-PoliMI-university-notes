\subsection{Cache Architecture}

Each \definition{Cache Block} (also called \definition{Cache Line}) is a row in the cache that stores part of memory.

\highspace
A typical cache line has \textbf{three fields}:
\begin{itemize}
    \item \definition{Valid Bit}: Indicates if the \textbf{block contains valid} (i.e., initialized) \textbf{data}.
    \item \definition{Tag}: The high-order bits (most significant bits, leftmost side) of the memory address, used to \textbf{identify the block}.
    \item \definition{Data}: The actual words/bytes of memory copied into the cache line.
\end{itemize}
At \hl{startup}, \textbf{all valid bits $=$ 0} (cache is empty).

\begin{examplebox}[: Visual Representation]
    \begin{center}
        \begin{tabular}{@{} c | c | c @{}}
            \toprule
            \texttt{V} & \texttt{TAG} & \texttt{DATA} \\
            \midrule
            1 & \texttt{0x001AE} & \texttt{[Word0, Word1, Word2...]} \\
            \bottomrule
        \end{tabular}
    \end{center}
    \begin{itemize}
        \item \texttt{V}: valid bit.
        \item \texttt{TAG}: identifies \emph{which memory block} this is.
        \item \texttt{DATA}: contains the content of that memory block (usually several words).
    \end{itemize}
\end{examplebox}

\highspace
\begin{flushleft}
    \textcolor{Green3}{\faIcon{key} \textbf{Why the TAG?}}
\end{flushleft}
Let's say the CPU requests a memory address like \texttt{0x1AE023}. The cache checks:
\begin{enumerate}
    \item \textbf{Index}: determines (extract) which row to look in.
    \item \textbf{Tag}: is this row the one holding \texttt{0x1AE}?
    \item \textbf{Valid}: is the content even initialized?
\end{enumerate}
If both Tag and Valid match, it's a \textbf{hit}.

\highspace
\begin{flushleft}
    \textcolor{Green3}{\faIcon{question-circle} \textbf{Four key cache design questions}}
\end{flushleft}
\begin{enumerate}[label=\textcolor{Green3}{\faIcon{question-circle}}]
    \item \definitionWithSpecificIndex{Block placement}{Cache - Block placement}{} - \textbf{\emph{Where can a block be placed?}} Page \pageref{subsubsection: Block placement}.

    \item \definitionWithSpecificIndex{Block identification}{Cache - Block identification}{} - \textbf{\emph{How is a block found?}} Page \pageref{subsubsection: block identification}.

    \item \definitionWithSpecificIndex{Replacement strategy}{Cache - Replacement strategy}{} - \textbf{\emph{Which block should be replaced?}} Page \pageref{subsubsection: Replacement strategy}.

    \item \definitionWithSpecificIndex{Write strategy}{Cache - Write strategy}{} - \textbf{\emph{What happens on a write?}} Page \pageref{subsubsection: write strategy}.
\end{enumerate}