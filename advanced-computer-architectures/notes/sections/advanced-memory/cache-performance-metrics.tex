\subsection{Cache Performance Metrics}

This section is about quantifying how effective a cache is, and how to improve it. Similar topics were covered on page \pageref{subsection: Memory Hierarchy}.

\highspace
The \definition{Average Memory Access Time (AMAT)} is the average time the CPU needs to access memory, considering both hits and misses. So, when accessing memory, we either get a \textbf{hit} or a \textbf{miss}:
\begin{equation*}
    \begin{array}{rcl}
        \text{AMAT} &=& (\text{Probability of Hit}) \cdot (\text{Time if Hit}) + \\ [.3em]
        && (\text{Probability of Miss}) \cdot (\text{Time if Miss})
    \end{array}
\end{equation*}
Which is written as:
\begin{equation}
    \text{AMAT} = \text{Hit Rate} \cdot \text{Hit Time} + \text{Miss Rate} \cdot \text{Miss Time}
\end{equation}
This is the \textbf{most general formula}, where:
\begin{itemize}
    \item \textbf{Hit Rate}: fraction of memory accesses that result in a hit.
    \item \textbf{Hit Time}: time to check cache and return data on a hit.
    \item \textbf{Miss Rate}: fraction of memory accesses that result in a miss.
    \item \definition{Miss Time}: time to
    \begin{enumerate}
        \item \textbf{Check} the \textbf{cache} (same time as a \emph{Hit Time}).
        \item \textbf{Fetch} the \textbf{data from main memory} (same time as a \emph{Miss Penalty})
    \end{enumerate}
    So it is reasonable to write the Miss Time as follows:
    \begin{equation}
        \text{Miss Time} = \text{Hit Time} + \text{Miss Penalty}
    \end{equation}
\end{itemize}
However, if the Miss Time is equal to the sum of the Hit Time and the Miss Penalty, then the most general formula can be \textbf{simplified}:
\begin{equation*}
    \begin{array}{rcl}
        \text{AMAT} &=& \text{Hit Rate} \cdot \text{Hit Time} + \text{Miss Rate} \cdot \text{Miss Time} \\ [.7em]
        %
        &=& \text{Hit Rate} \cdot \text{Hit Time} + \text{Miss Rate} \, \cdot \\ [.3em]
        &&  \left(\text{Hit Time} + \text{Miss Penalty}\right) \\ [.7em]
        %
        &=& \text{Hit Rate} \cdot \text{Hit Time} + \text{Miss Rate} \cdot \text{Hit Time} \, + \\ [.3em]
        &&  \text{Miss Rate} \cdot \text{Miss Penalty} \\ [.7em]
        %
        &=& \text{Hit Time} \cdot \underbrace{\left(\text{Hit Rate} + \text{Miss Rate}\right)}_{\% \, \text{Miss} \, + \, \% \, \text{Hit} \, = \, 100\% \, = \, 1} \, + \, \text{Miss Rate} \cdot \text{Miss Penalty}
    \end{array}
\end{equation*}
So the \textbf{simplified and most common} version is:
\begin{equation}
    \text{AMAT} = \text{Hit Time} + \text{Miss Rate} \times \text{Miss Penalty}
\end{equation}
Even if a cache \textbf{misses}, we still pay the \textbf{Hit Time}, because we have to \emph{check} the cache to know that it's a miss. So every access:
\begin{itemize}
    \item[\textcolor{Red2}{\faIcon{thumbs-down}}] Pays \textbf{Hit Time}.
    \item[\textcolor{Red2}{\faIcon{thumbs-down}}] Plus, if it's a miss, it pays \textbf{Miss Penalty}.
\end{itemize}
