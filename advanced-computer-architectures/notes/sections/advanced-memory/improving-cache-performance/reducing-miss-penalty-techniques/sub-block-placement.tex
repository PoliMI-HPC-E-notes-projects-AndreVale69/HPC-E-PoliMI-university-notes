\paragraph{Sub-block Placement}

Normally, when a cache miss occurs, the cache \textbf{fetches the entire block} (e.g., 64 bytes) from the next memory level. This \textbf{increases miss penalty} because:
\begin{itemize}
    \item The transfer time is proportional to block size.
    \item We might not even need all the words in that block.
\end{itemize}
\textbf{Idea:} Break each cache block into \textbf{smaller pieces (sub-blocks)} with \textbf{individual valid bits}. This way, on a miss, we can \textbf{load only the needed sub-block} instead of the entire block.


\highspace
\begin{flushleft}
    \textcolor{Green3}{\faIcon{tools} \textbf{How it works}}
\end{flushleft}
\begin{itemize}
    \item Each block is divided into \textbf{sub-blocks} (e.g., 4 words per block, becomes 4 sub-blocks).
    \item Each sub-block has its own \textbf{Valid Bit}.
    \item On a miss:
    \begin{itemize}
        \item Only the requested sub-block is fetched.
        \item Other sub-blocks in the same block are marked invalid until fetched.
    \end{itemize}
\end{itemize}

\highspace
\begin{flushleft}
    \textcolor{Green3}{\faIcon{check-circle} \textbf{Benefits}}
\end{flushleft}
\begin{itemize}
    \item[\textcolor{Green3}{\faIcon{check}}] \textbf{Lower Miss Penalty}: Only part of the block is transferred, then less latency.
    \item[\textcolor{Green3}{\faIcon{check}}] \textbf{Better Bandwidth Usage}: Avoids fetching unused data.
    \item[\textcolor{Green3}{\faIcon{check}}] Can be useful for \textbf{irregular access patterns} where spatial locality is weak.
\end{itemize}
It is useful when \textbf{memory bandwidth is limited} or when \textbf{access patterns are sparse} (e.g., random access over large arrays). It is also useful in systems where the \textbf{miss penalty is high} and spatial locality is low.

\highspace
\begin{flushleft}
    \textcolor{Red2}{\faIcon{exclamation-triangle} \textbf{Drawbacks}}
\end{flushleft}
\begin{itemize}
    \item[\textcolor{Red2}{\faIcon{times}}] \textbf{Lower Spatial Locality Exploitation}: Since we don't fetch the whole block, nearby addresses may miss later.
    \item[\textcolor{Red2}{\faIcon{times}}] \textbf{Extra Tag/Valid Bit Storage}: Need valid bits for each sub-block.
    \item[\textcolor{Red2}{\faIcon{times}}] \textbf{More Complex Control Logic}: Cache controller must track which sub-blocks are valid.
\end{itemize}

\highspace
Sub-block placement trades off \textbf{spatial locality} for \textbf{lower miss penalty} and better bandwidth usage. It's especially useful when \textbf{large block sizes} would otherwise cause excessive transfer delays.
