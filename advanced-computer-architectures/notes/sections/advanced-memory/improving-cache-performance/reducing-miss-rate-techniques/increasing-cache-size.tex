\subsubsection{Reducing Miss Rate Techniques}

\paragraph{Increasing Cache Size}

One intuitive way to reduce the number of cache misses is to simply make the cache \textbf{larger}. A bigger cache can store \textbf{more blocks}, which increases the chance that \textbf{future accesses will hit}, especially in programs with large working sets.

\highspace
\begin{flushleft}
    \textcolor{Green3}{\faIcon{\speedIcon} \textbf{Effect on Miss Rate}}
\end{flushleft}
\hl{Increasing cache size reduces \textbf{capacity misses}}. If a miss happens because a previously used block was evicted to make room, then increasing the cache may \textbf{prevent such eviction}. Miss rate \textbf{typically decreases} with increasing cache size, but with \textbf{diminishing returns}. Empirical studies show that doubling the cache size doesn't halve the miss rate.

\highspace
\begin{flushleft}
    \textcolor{Red2}{\faIcon{exclamation-triangle} \textbf{Trade-offs and Drawbacks}}
\end{flushleft}
While it's effective, this approach isn't free. Larger caches come with:
\begin{table}[!htp]
    \centering
    \begin{tabular}{@{} l p{21em} @{}}
        \toprule
        Factor & Effect \\
        \midrule
        \textbf{Hit Time}           & \textcolor{Red2}{\faIcon{times-circle}} Usually increases (larger cache, more complex logic and longer access path).  \\ [.5em]
        \textbf{Area}               & \textcolor{Red2}{\faIcon{times-circle}} Grows significantly (more memory cells, more tag storage).                    \\ [.5em]
        \textbf{Power Consumption}  & \textcolor{Red2}{\faIcon{times-circle}} Higher due to increased access energy.                                        \\ [.5em]
        \textbf{Cost}               & \textcolor{Red2}{\faIcon{times-circle}} More silicon area, higher production cost.                               \\
        \bottomrule
    \end{tabular}
\end{table}

\noindent
So, a larger cache \textbf{may reduce miss rate}, but could also \textbf{increase hit time} (page \hqpageref{def: Hit Time}), hurting the \emph{overall} AMAT.

\highspace
In practice:
\begin{itemize}
    \item \textbf{L1 caches} are kept \textbf{small} and fast (to preserve CPU lock frequency).
    \item \textbf{L2/L3 caches} are larger and slower, used to \textbf{absorb the remaining misses}.
\end{itemize}
Increasing cache size is an \textbf{effective way to reduce capacity misses}, but it \hl{comes with trade-offs in area, latency, power, and cost}. It \textbf{must be balanced} with the needs of the processor pipeline and application behavior.
