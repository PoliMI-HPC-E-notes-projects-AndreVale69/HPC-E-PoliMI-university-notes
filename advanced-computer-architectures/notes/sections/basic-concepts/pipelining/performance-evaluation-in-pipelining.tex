\subsubsection{Performance evaluation in pipelining}

As we have seen in the previous sections, the \textbf{pipelining increases the CPU instruction throughput} (number of instructions completed per unit of time) but doesn't reduce the latency (the execution time of a single instruction).

\highspace
The \textbf{increase in latency is a direct consequence of two problems}:
\begin{itemize}
    \item The \indexdefinition{imbalance among the pipeline stages}

    \item The \indexdefinition{overhead in the pipeline control}
\end{itemize}
This imbalance between the pipeline stages and the overhead are \underline{bad aspects}:
\begin{itemize}
    \item The \textbf{imbalance} reduces performance because the \textbf{clock can run no faster than the time needed for the slowest pipeline stage};

    \item The \textbf{overhead} arises from the \textbf{delay introduced by interstage registers and clock skew}.
\end{itemize}
Finally, \textbf{all instructions should be the same number of pipeline stages}. Each assumption and optimization shown previously works well in this case.

\begin{definitionbox}[: number of Clock Cycles, Clocks Per Instructions and MIPS formula]
    Given:
    \begin{itemize}
        \item The \textbf{Instruction Count per iteration} as $\texttt{IC}_{\texttt{per\_iter}}$

        \item The \textbf{number of Stall Cycles per iteration} as \texttt{\# Stall Cycles}

        \item The \textbf{length of the pipeline} is $x$
    \end{itemize}
    We can \textbf{calculate the} \indexdefinition{number of Clock Cycles} as the sum between the Instruction Count (how many stages there are in one instruction), the number of Stall Cycles inserted by the hardware technique (called insertion of stalls), plus the length of the pipeline $x$:
    \begin{equation}\label{eq: clock cycles per iteration}
        \texttt{\# Clock Cycles}_{\texttt{per\_iter}} = \texttt{IC}_{\texttt{per\_iter}} + \texttt{\# Stall Cycles}_{\texttt{per\_iter}} + x
    \end{equation}
    The \indexdefinition{Clocks Per Instruction} per iteration, $\texttt{CPI}_{\texttt{per\_iter}}$, is calculated with the rapport between the number of Clock Cycles per iteration (previous equation) divided by the Instruction Count per iteration:
    \begin{equation}\label{eq: clock per instruction per iteration}
        \begin{array}{rcl}
            \texttt{CPI}_{\texttt{per\_iter}} &=& \dfrac{
                \texttt{\# Clock Cycles}_{\texttt{per\_iter}}
            }{
                \texttt{IC}_{\texttt{per\_iter}}
            } \\ \\
            &=& \dfrac{
                \left(\texttt{IC}_{\texttt{per\_iter}} + \texttt{\# Stall Cycles}_{\texttt{per\_iter}} + x\right)
            }{
                \texttt{IC}_{\texttt{per\_iter}}
            }
        \end{array}
    \end{equation}
    Finally, the \indexdefinition{MIPS formula} per iteration is calculated with the rapport between the frequency of the clock ($\texttt{f}_{\texttt{clock}}$) divided by the multiply between the Instructions Per Clock (as the ratio $1 \div \texttt{CPI}$) and $10^{6}$ (1 million instructions):
    \begin{equation}\label{eq: MIPS per iteration}
        \texttt{MIPS}_{\texttt{per\_iter}}
        =
        \dfrac{
            \texttt{f}_{\texttt{clock}}
        }{
            \left(
                \texttt{CPI}_{\texttt{per\_iter}} \times 10^{6}
            \right)
        }
    \end{equation}
\end{definitionbox}

\noindent
We can asymptotically (\texttt{AS}) rewrite equations \ref{eq: clock cycles per iteration}, \ref{eq: clock per instruction per iteration} and \ref{eq: MIPS per iteration} as follows:
\begin{equation}
    \texttt{\# Clock Cycles}_{\texttt{AS}} = \texttt{IC}_{\texttt{AS}} + \texttt{\# Stall Cycles}_{\texttt{AS}} + x
\end{equation}
\vspace{1em}
\begin{equation}
    \begin{array}{rcl}
        \texttt{CPI}_{\texttt{AS}} &=& \lim_{n \rightarrow \infty} \dfrac{
            \texttt{\# Clock Cycles}_{\texttt{AS}}
        }{
            \texttt{IC}_{\texttt{AS}}
        } \\ \\
        &=& \lim_{n \rightarrow \infty} \dfrac{
            \left(
                \texttt{IC}_{\texttt{AS}} + \texttt{\# Stall Cycles}_{\texttt{AS}} + x
            \right)
        }{
            \texttt{IC}_{\texttt{AS}}
        }
    \end{array}
\end{equation}
\vspace{1em}
\begin{equation}
    \texttt{MIPS}_{\texttt{AS}} = \dfrac{
        \texttt{f}_{\texttt{clock}}
    }{
        \left(\texttt{CPI}_{\texttt{AS}} \times 10^{6}\right)
    }
\end{equation}
Note: the \textbf{ideal speedup, then Clock Per Instruction, should be equal to 1}. But stalls cause the pipeline performance to degrade from the ideal performance, so we have the \indexdefinition{Average Clock Per Instruction (CPI)}:
\begin{equation}
    \begin{array}{rcl}
        \texttt{AVG}\left(\texttt{CPI}\right) &=& \texttt{Ideal CPI} + \texttt{\# Stall Cycles}_{\texttt{per\_instruction}} \\ \\
        &=& 1 + \texttt{\# Stall Cycles}_{\texttt{per\_instruction}}
    \end{array}
\end{equation}
And obviously, the \indexdefinition{Pipeline Stall Cycles per Instruction} is:
\begin{equation}
    \texttt{PSCI} = \texttt{Structural Haz.} + \texttt{Data Haz.} + \texttt{Control Haz.} + \texttt{Memory Stalls}
\end{equation}