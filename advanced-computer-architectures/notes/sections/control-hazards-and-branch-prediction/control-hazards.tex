\subsection{Control Hazards}

In a pipelined architecture, one of the primary challenges in achieving high performance is dealing with \definition{Control Hazards}, also known as \textbf{branch hazards}. These \textbf{arise due to the presence of conditional branch instructions}, where the \textbf{processor must decide the next instruction to fetch before knowing whether the branch will be taken}.

\highspace
\begin{flushleft}
    \textcolor{Green3}{\faIcon{question-circle} \textbf{What really causes Control Hazards?}}
\end{flushleft}
To sustain the pipeline and \hl{avoid idle stages}, a processor needs to \hl{fetch one instruction per clock cycle}. However, with branch instructions, this becomes problematic because the \textbf{branch decision} (branch outcome) and the \textbf{branch target address} (BTA) are \textbf{not immediately available}. Specifically, \textbf{during the IF stage}, when the next instruction is fetched, the \textbf{processor still does not know whether the branch will be taken or not}, because this information typically becomes available later in the pipeline.

\highspace
This leads to uncertainty:
\begin{itemize}[label=\textcolor{Green3}{\faIcon{question-circle}}]
    \item \emph{Which instruction should be fetched after a branch?}
    \item \emph{Should it be sequential instruction (\texttt{PC + 4}) or the instruction at the BTA?}
\end{itemize}
\textbf{If the processor fetches the wrong instruction}, it might \textbf{need to discard or ``flush''} it later, \textcolor{Red2}{\textbf{wasting valuable cycles}}. Alternatively, the \textcolor{Red2}{\textbf{processor may stall}}, delaying the fetching of any instruction until the branch decision is known, which also hurts performance.

\highspace
The key issue is this: in a pipeline, the \textbf{instruction stream needs to continue}, but the \textbf{correct path is unclear} until the branch condition is evaluated. Thus:
\begin{enumerate}
    \item Either \textbf{wrong-path instructions are fetched} (requiring flushing later)
    \item Or \textbf{pipeline stalls} are introduced (causing delay and loss of ideal\break speedup)
\end{enumerate}

\highspace
\begin{takeawaysbox}[: Control Hazards]
    \begin{itemize}
        \item \important{Definition}: Pipeline hazard due to \textbf{uncertainty in branch outcome} during instruction fetch.
        \item \important{Cause}: The \textbf{branch condition is unresolved} when the next instruction must be fetched (IF stage).
        \item \important{Instructions Involved}: Conditional branches (\texttt{beq}, \texttt{bne}) and jumps, all \textbf{instructions modifying the PC}.
        \item \important{Pipeline Timing Conflict}: \textbf{BO and BTA known only in EX or later}, but instruction fetch \textbf{must occur every cycle}.
        \item \important{Main Problem}: \textbf{Cannot decide} whether to fetch \textbf{next sequential instruction} (\texttt{PC + 4}) or \textbf{BTA}.
        \item \important{Possible Outcomes}:
        \begin{itemize}
            \item \textbf{Stall}: Delay fetch until branch resolved.
            \item \textbf{Fetch wrong instruction} $\rightarrow$ flush.
        \end{itemize}
        \item \important{Performance Impact}: Loss of \textbf{ideal pipelining speedup}; \textbf{reduced throughput} due to stalls or wasted fetches.
        \item \important{Goal of Solutions}: \textbf{Mitigate stalls} and \textbf{improve fetch accuracy} through early evaluation or prediction.
    \end{itemize}
\end{takeawaysbox}