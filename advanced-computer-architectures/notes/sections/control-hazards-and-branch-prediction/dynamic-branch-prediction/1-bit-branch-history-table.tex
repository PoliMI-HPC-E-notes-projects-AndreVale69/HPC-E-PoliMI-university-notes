\subsubsection{1-bit Branch History Table}

In general, the \definition{Branch History Table (BHT)}, or \definition{Branch Prediction Buffer}, is a \textbf{dynamic hardware structure} that \textbf{predicts branch outcomes based on recent behavior}. The \hl{simplest version uses 1 bit per branch} to remember whether the branch was \textbf{recently taken or not taken}. For this reason, it is called a \definition{1-bit Branch History Table (1-bit BHT)}. It operates at runtime and uses a \textbf{Final State Machine (FSM) with 1-bit history}:
\begin{itemize}
    \item If last outcome was Taken $\Rightarrow$ predict Taken.
    \item If last outcome was Not Taken $\Rightarrow$ predict Not Taken.
\end{itemize}

\begin{figure}[!htp]
    \centering
    \includegraphics[width=.9\textwidth]{img/fsm-1-bit-bht.pdf}
\end{figure}

\begin{flushleft}
    \textcolor{Green3}{\faIcon{tools} \textbf{How it works}}
\end{flushleft}
\begin{itemize}
    \item \textbf{Each branch} instruction's address (or part of it) \textbf{indexes a table entry}.
    \item That \textbf{entry holds} a \textbf{single bit} (T/NT) representing the \textbf{last observed outcome}.
    \item On next encounter:
    \begin{itemize}
        \item Use the \textbf{bit to predict the outcome}.
        \item After actual branch resolution:
        \begin{itemize}
            \item \textcolor{Green3}{\textbf{Correct prediction}} $\Rightarrow$ keep bit unchanged
            \item \textcolor{Red2}{\textbf{Incorrect prediction}} $\Rightarrow$ flip bit
        \end{itemize}
    \end{itemize}
\end{itemize}

\highspace
\begin{flushleft}
    \textcolor{Green3}{\faIcon{book} \textbf{Indexing the Table}}
\end{flushleft}
\begin{itemize}
    \item Use $k$ lower bits (right side) of the branch's address as the index.
    \item[\textcolor{Red2}{\faIcon{exclamation-triangle}}] \textcolor{Red2}{\textbf{No tags}}: \textbf{any branch with the same low-order bits shares the entry (can cause interference)}.
    \item $2^{k}$ \textbf{entries} total.
\end{itemize}

\newpage

\begin{figure}[!htp]
    \centering
    \includegraphics[width=.8\textwidth]{img/1-bit-bht.pdf}
    \caption{Visual representation of the 1-bit Branch History Table.}
\end{figure}

\begin{examplebox}[: Accuracy Issue]
    Consider a loop that executes 10 iterations. The expected behavior of the branch is:
    \begin{center}
        \texttt{T T T T T T T T T NT}
    \end{center}
    Where the last is Not Taken because the code must exist from the loop and must continue (and not jump).
    
    There are \textbf{two mispredictions}:
    \begin{itemize}
        \item \textbf{At the end}: Since iteration 9 is marked as Taken, the 10th iteration is predicted to be Taken, since the BHT contains Taken. This throws a misprediction because the branch result is Not Taken.
        \item \textbf{On the next time loop starts}: since the last iteration is stored in the BHT as Not Taken, the BHT has to flip the bit on the next time loop starts, and a misprediction occurs.
    \end{itemize}
    As a result, with 100\% of 10 iterations, the BHT only catches 8 out of 10 iterations, the accuracy is 80\%.
\end{examplebox}

\highspace
\begin{flushleft}
    \textcolor{Red2}{\faIcon{times-circle} \textbf{Shortcomings}}
\end{flushleft}
\begin{itemize}[label=\textcolor{Red2}{\faIcon{times}}]
    \item \textbf{Flipping prediction after 1 misprediction causes instability}, especially in loops.
    \item \textbf{Conflict problem}: two different branches with same index overwrite each other's bit.
\end{itemize}

\highspace
\begin{flushleft}
    \textcolor{Green3}{\faIcon{check-circle} \textbf{Partial Solutions}}
\end{flushleft}
To \textbf{reduce interference}:
\begin{itemize}
    \item Increase table size (more $k$ bits).
    \item Use hashing to mix address bits better.
\end{itemize}
