\subsubsection{Branch Target Buffer}

The \definition{Branch Target Buffer (BTB)} is a \textbf{specialized cache used to store target address of taken branches}. The stored Predicted Target Address (PTA) allows the processor to fetch instructions from the target without delay when a branch is predicted taken. The PTA is typically stored in PC-relative format (offset from current PC).

\highspace
\begin{flushleft}
    \textcolor{Green3}{\faIcon{book} \textbf{Core Idea}}
\end{flushleft}
While the Branch History Table (BHT) predicts \emph{whether} a branch will be taken, the Branch Target Buffer (BTB) \textbf{predicts \emph{where} the program should go if the branch is taken}. The \textbf{BTB stores Predicted Target Addresses} (PTAs) for previously encountered branches and \textbf{enables fast redirection of control flow}.

\highspace
\begin{flushleft}
    \textcolor{Green3}{\faIcon{tools} \textbf{How Is the BTB Structured?}}
\end{flushleft}
\begin{itemize}
    \item The BTB is designed as a \textbf{direct-mapped cache}:
    \begin{itemize}
        \item The \textbf{address of the branch instruction is used to index} the BTB.
        \item \textbf{Tags} are \textbf{used} for associative lookup to \textbf{confirm correctness} (i.e., ensure the indexed entry really belongs to the current branch)
    \end{itemize}
    \item Components per \textbf{entry}:
    \begin{itemize}
        \item \important{Tags}: Identifies the branch instruction.
        \item \important{PTA}: The Predicted Target Address.
        \item Often combined with \textbf{T/NT bits} from a \textbf{Branch History Table} (1-bit or 2-bit) for branch outcome prediction.
    \end{itemize}
\end{itemize}
\begin{figure}[!htp]
    \centering
    \includegraphics[width=\textwidth]{img/btb.pdf}
    \caption{Branch Target Buffer without Branch Outcome Predictor.}
\end{figure}

\newpage

\begin{figure}[!htp]
    \centering
    \includegraphics[width=\textwidth]{img/btb-bht.pdf}
    \caption{Branch Target Buffer with Branch Outcome Predictor.}
\end{figure}

\highspace
\begin{flushleft}
    \textcolor{Green3}{\faIcon{microchip} \textbf{BTB in the Pipeline}}
\end{flushleft}
It is placed in the Instruction Fetch (IF) phase. During fetch:
\begin{itemize}
    \item The BTB is queried using the current PC (branch address).
    \item If \textbf{hit and BHT predicts Taken}, \hl{fetch from PTA}.
    \item If \textbf{miss or predict Not Taken}, \hl{continue} at \texttt{PC + 4}.
\end{itemize}

\begin{table}[!htp]
    \centering
    \begin{adjustbox}{width={\textwidth},totalheight={\textheight},keepaspectratio}
        \begin{tabular}{@{} l | l | l @{}}
            \toprule
            \textbf{Prediction} & \textbf{BTB use} & \textbf{Action} \\
            \midrule
            Predict Not Taken         & BTB \textbf{not used}  & Fetch from PC + 4                                       \\ [.5em]
            Predict Taken (BTB hit)   & BTB \textbf{used}      & Fetch from PTA stored in BTB                            \\ [.5em]
            Predict Taken (BTB miss)  & BTB \textbf{miss}      & Stall or default to \texttt{PC + 4}, then calculate BTA \\
            \bottomrule
        \end{tabular}
    \end{adjustbox}
    \caption{Summary of Behavior.}
\end{table}

\begin{flushleft}
    \textcolor{Green3}{\faIcon{check-circle} \textbf{Advantages}}
\end{flushleft}
\begin{itemize}[label=\textcolor{Green3}{\faIcon{check}}]
    \item \textbf{Eliminates delay} from calculating the \textbf{branch target address} manually.
    \item Enables speculative instruction fetch from correct target, \textbf{improving pipeline efficiency}.
\end{itemize}