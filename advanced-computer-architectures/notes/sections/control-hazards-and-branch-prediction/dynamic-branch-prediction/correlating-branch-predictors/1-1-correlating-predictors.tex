\paragraph{(1,1) Correlating Predictors}

Use the \textbf{result of the last executed branch} (global history, $m = 1$ bit) to \textbf{choose between two prediction tables}, each of which has 1-bit entries.
\begin{itemize}
    \item \important{1-bit Global History}: Stores last branch outcome (Taken \texttt{=} 1, Not Taken \texttt{=} 0).
    \item \important{2 BHTs (T1 \& T2)}: Each is a 1-bit predictor table, selected based on global history. We use a 1-bit Branch History Table technique.
    \item \important{Indexing}: Use PC low-order bits to index into the selected table.
\end{itemize}
Consider a pseudo code:
\begin{lstlisting}[language=c]
if (x > 0)  // Branch A
    ...
if (x > 0)  // Branch B
    ...
\end{lstlisting}
Let's say if \texttt{A} is true, \texttt{B} is usually true. The execution walkthrough:
\begin{itemize}
    \item Cycle 1: \important{First Execution of \texttt{Branch A}}
    \begin{enumerate}
        \item Global History: unknown or NT (0); because it doesn't track anything yet. We assume Not Taken (0).
        \item Use Table T2 (since \texttt{GH = 0}).
        \item Index into T2 with \texttt{Branch A}'s PC bits.
        \item[\textcolor{Green3}{\faIcon{magic}}] Predict: Not Taken!
        \item[\textcolor{Red2}{\faIcon{thumbs-down}}] Unfortunately, the Branch Outcome (BO) says Taken $\Rightarrow$ \textcolor{Red2}{\faIcon{times} \textbf{Mispredict}} $\Rightarrow$ update T2 entry to T.
        \item Update Global History \texttt{=} 1 (Taken).
    \end{enumerate}

    \item Cycle 2: \important{Now Executing Branch \texttt{Branch B}}
    \begin{enumerate}
        \item Global History \texttt{=} 1 (Taken) from \texttt{Branch A}.
        \item Use Table T1 (since \texttt{GH = 1}).
        \item Index with \texttt{Branch B}'s PC bits.
        \item[\textcolor{Green3}{\faIcon{magic}}] T1 says: ``Try to predict as Taken''.
        \item[\textcolor{Green3}{\faIcon{\speedIcon}}] The Branch Outcome (BO) says Taken $\Rightarrow$ \textcolor{Green3}{\faIcon{check} \textbf{Correct prediction}}.
        \item No update needed. The Global History doesn't need an update either, because it is already 1 (Taken).
    \end{enumerate}
\end{itemize}
Since \texttt{Branch A} was Taken, \texttt{Branch B} is likely to be Taken. This is a smart technique because normal predictors treat \texttt{Branch B} alone. Instead, the Correlating Branch Predictor uses context, and the last branch helps predict this one.

\highspace
\begin{adjustbox}{width={\textwidth},totalheight={\textheight},keepaspectratio}
    \begin{tabular}{@{} l | c | c | c | c | c @{}}
        \toprule
        Branch   & Global History & Table Used & Prediction & Outcome & Update        \\
        \midrule
        Branch A & 0              & T2         & NT         & T       & T2 entry to T \\ [.5em]
        Branch B & 1              & T1         & T          & T       & No change     \\
        \bottomrule
    \end{tabular}
\end{adjustbox}

\newpage

\begin{figure}[!htp]
    \centering
    \includegraphics[width=.9\textwidth]{img/1-1-correlating-predictors.pdf}
    \caption{Visual representation of the (1,1) Correlating Predictor.}
\end{figure}

\begin{table}[!htp]
    \centering
    \begin{tabular}{@{} l | l @{}}
        \toprule
        \textbf{Aspect} & \textbf{Description} \\
        \midrule
        $m = 1$         & Use 1-bit GHR (last branch result).                   \\ [.5em]
        $n = 1$         & 1-bit prediction, (T or NT).                          \\ [.5em]
        Tables          & 2 BHTs (for \texttt{GHR = 0} and \texttt{GHR = 1}).   \\ [.5em]
        Selection Logic & If last branch T, use Table T1; else T2.              \\
        \bottomrule
    \end{tabular}
    \caption{(1,1) Correlating Predictor.}
\end{table}


