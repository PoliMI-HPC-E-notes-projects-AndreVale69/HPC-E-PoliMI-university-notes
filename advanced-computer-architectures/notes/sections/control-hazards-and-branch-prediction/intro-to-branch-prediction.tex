\subsection{Intro to Branch Prediction}

In modern computer architectures, achieving high performance requires efficient instruction-level parallelism (ILP)\footnote{%
    \definition{Instruction-Level Parallelism (ILP)}: A measure of \textbf{how many of the operations in a computer program can be performed simultaneously}. High ILP enables multiple instructions to be executed in parallel within a single processor cycle, exploiting the parallelism inherent in sequential instruction streams through techniques like pipelining, superscalar execution, and out-of-order execution.
}. However, one of the \textbf{major obstacles to ILP is the occurrence of branch hazards}, which happen when the processor encounters a branch instruction (e.g., \texttt{if}, \texttt{for}, \texttt{while}) and cannot immediately determine which instruction to execute next. To mitigate the performance loss caused by these hazards, branch prediction is employed.

\highspace
\definition{Branch Prediction} is essentially a \textbf{speculative execution technique} where the \textbf{processor \emph{guesses} the outcome of a branch instruction}, whether the branch will be taken (control jumps to the branch target) or not taken (execution continues sequentially), \textbf{before the actual result is known}. Instead of stalling the pipeline and waiting for the branch condition to be resolved, the \textbf{processor proceeds based on the predicted outcome}. If the prediction:
\begin{itemize}
    \item[\textcolor{Green3}{\faIcon{check}}] Is \textcolor{Green3}{\textbf{correct}}, performance is preserved.
    \item[\textcolor{Red2}{\faIcon{times}}] Is \textcolor{Red2}{\textbf{wrong}} (a \definitionWithSpecificIndex{misprediction}{Misprediction}{}), the incorrectly fetched \textbf{instructions are flushed}, and \textbf{execution restarts at the correct address}, causing a \textbf{performance penalty}.
\end{itemize}

\highspace
\begin{flushleft}
    \textcolor{Green3}{\faIcon{list-ul} \textbf{Branch Prediction categories}}
\end{flushleft}
Branch prediction techniques are generally classified into two main categories:
\begin{itemize}
    \item \definition{Static Branch Prediction Techniques}. In this method, the \textbf{branch direction} (taken/untaken) is \textbf{decided at compile time} and \textbf{remains fixed during the program's execution}. Static prediction often relies on compiler heuristics or profiling data to guess likely outcomes. Since the behavior doesn't adapt to runtime changes, this \textbf{technique works best} when \textbf{branch outcomes} are \textbf{highly predictable} and \textbf{consistent} across executions.

    \item \definition{Dynamic Branch Prediction Techniques}. Unlike static methods, dynamic prediction uses \textbf{hardware mechanisms to observe past branch behavior at runtime} and make predictions accordingly. The \textbf{prediction adapts to actual program execution}, making it more effective for applications with \textbf{complex or data-dependent control flow}. This method can dynamically switch its guess depending on the \emph{branch history}.
\end{itemize}
It's important to note that in both static and dynamic techniques, the \textbf{processor must avoid updating its internal state} (registers, memory, etc.) \textbf{until the branch outcome is known with certainty}. This ensures speculative execution doesn't cause side effects in case of misprediction.

\highspace
Additionally, \textbf{hybrid approaches} are possible, where static and dynamic predictions are combined to optimize performance further.