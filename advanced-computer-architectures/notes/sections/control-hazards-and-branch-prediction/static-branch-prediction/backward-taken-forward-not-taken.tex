\subsubsection{Backward Taken Forward Not Taken (BTFNT)}

The \definition{Backward Taken Forward Not Taken (BTFNT)} strategy represents a \textbf{refinement of static prediction} that uses a simple yet effective heuristic: the \textbf{direction of the branch}, whether it \textbf{jumps backward or forward} in memory, can be \textbf{used to predict its outcome}.

\highspace
\begin{flushleft}
    \textcolor{Green3}{\faIcon{book} \textbf{Prediction Rule}}
\end{flushleft}
\begin{itemize}
    \item \important{Backward-going branches} (i.e., branches where the target address is lower than the current PC) are predicted as \textbf{taken}.
    \begin{itemize}
        \item These branches \textbf{often occur in loops}, where execution loops back to an earlier instruction (e.g., in \texttt{for}, \texttt{while}, or \texttt{do-while} constructs).
    \end{itemize}    
    \item \important{Forward-going branches} (i.e., target address is greater than the current PC) are predicted as \textbf{not taken}.
    \begin{itemize}
        \item These branches typically correspond to \textbf{\texttt{if-then-else} constructs}, where the \texttt{else} path is \textbf{less probable} and \textbf{control usually proceeds sequentially}.
    \end{itemize}
\end{itemize}

\highspace
\begin{flushleft}
    \textcolor{Green3}{\faIcon{question-circle} \textbf{Why does this work?}}
\end{flushleft}
The rationale behind BTFNT lies in \textbf{empirical observations}:
\begin{itemize}
    \item \textbf{Loops} tend to execute \textbf{multiple times}, hence \textbf{backward branches are mostly taken}.
    \item \textbf{Conditional statements} often have \textbf{rarely taken \texttt{else} paths}, hence \textbf{forward branches are mostly not taken}.
\end{itemize}

\highspace
\begin{flushleft}
    \textcolor{Green3}{\faIcon{check-circle} \textbf{Pros}} and \textcolor{Red2}{\faIcon{times-circle} \textbf{Cons}}
\end{flushleft}
\begin{itemize}
    \item[\textcolor{Green3}{\faIcon{check}}] Simple to implement because BTFNT requires just a comparison of the \textbf{target address vs the current PC}:
    \begin{itemize}
        \item \texttt{target address < PC} $\Rightarrow$ \textbf{predict taken}
        \item \texttt{target address > PC} $\Rightarrow$ \textbf{predict \underline{not} taken}
    \end{itemize}
    Also, better accuracy than uniform always-taken or always-not-taken, especially for mixed codebases.

    \item[\textcolor{Red2}{\faIcon{times}}] Not adaptive; fails for atypical control flows where direction doesn't align with expected behavior.
\end{itemize}
