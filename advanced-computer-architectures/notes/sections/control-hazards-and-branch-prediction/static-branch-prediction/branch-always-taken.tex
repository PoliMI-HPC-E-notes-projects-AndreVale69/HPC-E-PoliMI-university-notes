\subsubsection{Branch Always Taken}

This approach represents the dual case of the previous technique (page \pageref{subsubsection: Branch Always Not Taken}): it assumes that \textbf{every branch will be taken}, meaning the \textbf{control flow will jump to the branch target address} rather than continue sequentially. This method is especially \textbf{useful for backward branches}, which occur in \textbf{loops} such as \texttt{for}, \texttt{while}, and \texttt{do-while}, since these branches are typically \textbf{taken repeatedly during loop iterations}.

\highspace
\begin{flushleft}
    \textcolor{Red2}{\faIcon{question-circle} \textbf{Implementation Challenge}}
\end{flushleft}
Unlike the not-taken strategy, where the processor simply continues to \texttt{PC + 4}, the \textbf{taken strategy requires knowledge of the Branch Target Address (BTA)} at the Instruction Fetch (IF) stage. This is \hl{non-trivial} because:
\begin{itemize}
    \item[\textcolor{Red2}{\faIcon{question-circle}}] The \textbf{BTA depends on the branch instruction's target}, which typically requires decoding.
    \item[\textcolor{Green3}{\faIcon{check}}] To solve this, we introduce a \definition{Branch Target Buffer (BTB)}, a special hardware structure.
\end{itemize}

\highspace
\begin{flushleft}
    \textcolor{Green3}{\faIcon{question-circle} \textbf{What is BTB and why do we need it?}}
\end{flushleft}
The \definition{Branch Target Buffer (BTB)} is a \textbf{specialized cache} in the processor \textbf{designed to predict the target address} of a taken branch instruction \textbf{before the branch condition is actually resolved}. 

\highspace
In Branch Always Taken, we assume that the program will jump to a new address (the Branch Target Address, BTA). However, this \textbf{BTA is not immediately known during Instruction Fetch (IF)} because it typically requires decoding the branch instruction (Instruction Decode, ID). \hl{To avoid delays}, the \textbf{BTB remembers past branch target addresses}, allowing the processor to quickly predict where to jump when encountering a branch.

\highspace
\begin{flushleft}
    \textcolor{Green3}{\faIcon{tools} \textbf{How does BTB work? Quickest explanation}}
\end{flushleft}
\begin{itemize}
    \item \important{BTB Structure}, it is a kind of lookup table or cache where:
    \begin{itemize}
        \item \texttt{Key}: \textbf{address} of the \textbf{branch instruction} (the PC value where the branch resides)
        \item \texttt{Value}: Predicted Target Address (PTA), i.e., where to jump if the branch is taken
    \end{itemize}

    \item \important{BTB Lookup}: when fetching a branch instruction, the processor simultaneously queries the BTB via the branch PC.
    \begin{itemize}
        \item[\textcolor{Green3}{\faIcon{check}}] If a \textcolor{Green3}{\textbf{match is found}} (\emph{cache hit}), the \textbf{BTB immediately provides the Predicted Target Address} (PTA), and the \textbf{processor starts fetching from that address}, before knowing if the branch is actually taken.
        \item[\textcolor{Red2}{\faIcon{times}}] If a \textcolor{Red2}{\textbf{no match}} (\emph{cache miss}), the processor might \textbf{default to sequential execution} (\texttt{PC + 4}) or \textbf{wait for the BTA} to be calculated, which causes delay.
    \end{itemize}
\end{itemize}

\begin{examplebox}[: BTB and Branch Always Taken technique]
    Let's say:
    \begin{itemize}
        \item A loop branch at address \texttt{0x100} typically jumps to \texttt{0x80}.
        \item The BTB stores: \texttt{0x100} $\rightarrow$ \texttt{0x80} (key $\rightarrow$ value).
    \end{itemize}
    When the branch at \texttt{0x100} is fetched again:
    \begin{itemize}
        \item The BTB predicts the next instruction will be at \texttt{0x80} (taken).
        \item The processor \textbf{starts fetching from \texttt{0x80}}, \textbf{without waiting} to evaluate the branch condition.
    \end{itemize}
    If it turns out the \textbf{branch was not taken}, the processor \textbf{flushes the incorrect fetch} from \texttt{0x80} and resumes at \texttt{0x104}.
\end{examplebox}

\highspace
\begin{flushleft}
    \textcolor{Green3}{\faIcon{check-circle} \textbf{Correct Prediction Path}}
\end{flushleft}
If the \textbf{branch} is indeed \textbf{taken}, and the \textbf{BTB correctly supplies the BTA}, the processor proceeds \textbf{without penalty}. \textbf{Execution continues from the target address} just as expected.

\highspace
\begin{flushleft}
    \textcolor{Red2}{\faIcon{times-circle} \textbf{Misprediction Case}}
\end{flushleft}
If the \textbf{actual outcome} is \textbf{not taken}, the \textbf{prediction is incorrect}:
\begin{enumerate}
    \item The Instruction Fetched (IF) from the \textbf{target address} is \textbf{flushed} (\texttt{NOP}).
    \item The \textbf{processor must fetch the sequential instruction} at \texttt{PC + 4}.
    \item \textcolor{Red2}{\textbf{One-cycle penalty}} incurred, similar to the not-taken misprediction case.
\end{enumerate}

\highspace
\begin{flushleft}
    \textcolor{Green3}{\faIcon{\speedIcon} \textbf{When is this technique effective?}}
\end{flushleft}
This method is \textbf{well-suited for loop constructs}, where \textbf{branches typically go backward} and are \textbf{taken with high probability}. For example:
\begin{itemize}
    \item In a \texttt{do-while} loop, the branch is taken almost every time except the last iteration.
    \item Conversely, in forward branches like \texttt{if-then-else}, the branch is less likely to be taken, making this technique less effective.
\end{itemize}
This underscores that \textbf{branch direction} (forward or backward) \textbf{can influence the effectiveness of prediction strategies}.

\newpage

\begin{figure}[!htp]
    \centering
    \includegraphics[width=\textwidth]{img/branch-always-taken.pdf}
    \caption{Branch Always Taken techniques failed and the processor must flush instruction \texttt{i+1} and restart execution from the BTA.}
\end{figure}