\subsubsection{Delayed Branch}

The \definition{Delayed Branch} technique is a static scheduling approach where the \textbf{compiler} plays a central role in mitigating branch penalties. Unlike traditional branch prediction, which involves guessing the outcome of a branch, delayed branching \textbf{reorders instructions so that useful work is done regardless of the branch direction}.

\highspace
\begin{flushleft}
    \textcolor{Green3}{\faIcon{book} \textbf{Core Concept}}
\end{flushleft}
When a \textbf{branch instruction is executed}, it typically introduces a \textbf{delay before the processor can determine where to fetch the next instruction}. During this delay (known as the \definitionWithSpecificIndex{branch delay slot}{Branch Delay Slot}{}), rather than letting the pipeline sit idle or fetch incorrect instructions, the \textbf{compiler schedules an independent instruction to execute in that slot}.
\begin{itemize}
    \item The instruction in the \textbf{branch delay slot is always executed}, \textbf{regardless of whether the branch is taken or not}.
    \item This allows useful work to be completed during what would otherwise be a stall or pipeline bubble.
\end{itemize}

\begin{figure}[!htp]
    \centering
    \includegraphics[width=\textwidth]{img/delayed-branch-1.pdf}
    \caption{Example Scenario. In this case, the \texttt{add} instruction is scheduled after the branch, always executed, and does not affect the branch condition or outcomes.}
\end{figure}

\highspace
\begin{flushleft}
    \textcolor{Red2}{\faIcon{exclamation-triangle} \textbf{Compiler Responsibility}}
\end{flushleft}
A \textbf{critical task for the compiler} is \textbf{to find a valid and useful instruction} to place in the branch delay slot. The instruction must be:
\begin{itemize}
    \item \textbf{Independent} from the branch decision.
    \item \textbf{Safe to execute} whether the branch is taken or not.
\end{itemize}
To guide this, the compiler can choose an instruction:
\begin{enumerate}
    \item Section \ref{paragraph: From Before}, page \pageref{paragraph: From Before} - From \textbf{before} the branch.
    \item Section \ref{paragraph: From Target}, page \pageref{paragraph: From Target} - From the \textbf{target} of the branch.
    \item Section \ref{paragraph: From Fall-Through}, page \pageref{paragraph: From Fall-Through} - From the \textbf{fall-through path} (i.e., the sequential next instruction).
    \item Section \ref{}, page \pageref{} - From \textbf{after} the branch.
\end{enumerate}
We'll explore each of these four scheduling strategies step-by-step next.