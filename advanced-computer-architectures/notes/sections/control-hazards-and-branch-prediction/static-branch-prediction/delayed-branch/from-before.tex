\paragraph{From Before}\label{paragraph: From Before}

In the \definitionWithSpecificIndex{``From Before''}{Delayed Branch Scheduling: From Before}{} strategy of delayed branch scheduling, the \textbf{compiler selects an instruction that appears \emph{before} the branch} in program order and \textbf{moves it into the branch delay slot}. The selected \textbf{instruction must be independent of the branch decision} and safe to execute regardless of whether the branch is taken.

\highspace
\begin{flushleft}
    \textcolor{Green3}{\faIcon{book} \textbf{Key Characteristics}}
\end{flushleft}
\begin{itemize}
    \item The \textbf{instruction in the branch delay slot is always executed}.
    \item This instruction \textbf{will never be flushed}, since it is \textbf{guaranteed to execute} irrespective of the branch's outcome.
    \item \textbf{After the delay slot, execution continues normally}, either to the branch target or the fall-through instruction, depending on whether the branch is taken.
\end{itemize}

\begin{examplebox}[: From Before]
    Original code:
    \begin{lstlisting}[language=riscv]
add x1, x2, x3
beq x2, x4, L1
[delay slot]\end{lstlisting}
    After scheduling:
    \begin{lstlisting}[language=riscv]
beq x1, x2, x3
add x1, x2, x3      # delay slot filled\end{lstlisting}
    Here, the \texttt{add} is originally before the \texttt{beq} and has \textbf{no dependency} on the branch condition. It is safely moved into the delay slot.
\end{examplebox}

\begin{flushleft}
    \textcolor{Green3}{\faIcon{question-circle} \textbf{Pipeline Behavior}}
\end{flushleft}
\begin{itemize}
    \item \important{Branch Not Taken}
    \begin{itemize}
        \item The instruction in the delay slot is executed.
        \item Execution continues sequentially with the next instruction after the branch.
    \end{itemize}

    \item \important{Branch Taken}
    \begin{itemize}
        \item The delay slot instruction still executes.
        \item Execution jumps to the branch target after the delay slot.
    \end{itemize}
\end{itemize}
The \textbf{instruction moved to the delay slot is always executed}.

\highspace
\begin{flushleft}
    \textcolor{Green3}{\faIcon{check-circle} \textbf{Advantages}}
\end{flushleft}
\begin{itemize}[label=\textcolor{Green3}{\faIcon{check}}]
    \item \textcolor{Green3}{\textbf{No need for instruction flushing}}: the delay slot instruction is always valid.
    \item \textcolor{Green3}{\textbf{Efficiency}}: reuses existing instructions from earlier in the program to \textbf{hide the branch delay}.
\end{itemize}