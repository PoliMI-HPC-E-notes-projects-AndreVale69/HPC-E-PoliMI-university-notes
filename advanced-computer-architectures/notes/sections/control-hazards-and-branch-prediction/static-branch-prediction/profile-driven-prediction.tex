\subsubsection{Profile-Driven Prediction}

\definition{Profile-Driven Prediction} is a static prediction technique that \textbf{uses empirical data from previous program executions} to guide the prediction of branch outcomes. Rather than relying solely on heuristics or branch direction, this method \textbf{leverages profiling to derive probabilistic insights} about how branches behave under typical conditions.

\highspace
\begin{flushleft}
    \textcolor{Green3}{\faIcon{tools} \textbf{How does it work?}}
\end{flushleft}
\begin{enumerate}
    \item The \textbf{target application} is \textbf{executed multiple times beforehand}, \textbf{using diverse data sets} to simulate realistic execution scenarios.

    \item During these early runs, the \textbf{behavior of each branch instruction is recorded}. Specifically, how often it was taken or not taken.

    \item This \textbf{profiling produces statistics} for each branch, e.g., a pattern like:
    \begin{center}
      \texttt{T  T  T  T  T  T  T  T  NT  NT  NT}
    \end{center}
    ``Taken'' is most probable.

    \item Once the profiling is complete, the \textbf{compiler encodes a hint} directly into each branch instruction (e.g., in a dedicated \textbf{hint bit} in the instruction format):
    \begin{itemize}
      \item \texttt{1}: if the branch is \textbf{usually taken}.
      \item \texttt{0}: if the branch is \textbf{usually \underline{not} taken}.
    \end{itemize}
    This enables the \textbf{processor} to \textbf{consult the hint during execution} and predict accordingly, without requiring runtime monitoring.
\end{enumerate}

\highspace
\begin{flushleft}
  \textcolor{Green3}{\faIcon{check-circle} \textbf{Advantages}}
\end{flushleft}
\begin{itemize}[label=\textcolor{Green3}{\faIcon{check}}]
  \item Offers \textbf{higher accuracy than heuristics alone}, especially for \hl{applications with stable branch behavior}.
  \item \textbf{No runtime hardware cost}, since prediction decisions are guided by \textbf{static hints}.
\end{itemize}

\highspace
\begin{flushleft}
  \textcolor{Red2}{\faIcon{times-circle} \textbf{Limitations}}
\end{flushleft}
\begin{itemize}[label=\textcolor{Red2}{\faIcon{times}}]
  \item \textbf{Static nature}: Predictions don't adapt to runtime variability; changes in data patterns may invalidate the profile.
  \item Requires \textbf{extra compilation effort}: profiling and hint encoding add complexity to the build process.
  \item \textbf{Less effective} for programs with \textbf{input-dependent control flow}.
\end{itemize}
