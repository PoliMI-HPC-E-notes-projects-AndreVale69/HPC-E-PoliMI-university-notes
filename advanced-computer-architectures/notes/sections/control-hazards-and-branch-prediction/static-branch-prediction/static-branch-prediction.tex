\subsection{Static Branch Prediction}

Static branch prediction represents one of the \textbf{simplest approaches} to handling branch hazards. In this technique, the \textbf{prediction} regarding whether a branch will be taken or not is \textbf{made at compile time} and \textbf{remains unchanged throughout program execution}. This method \textbf{relies heavily on heuristics or compiler-generated hints}, which estimate the likely behavior of each branch without any consideration of the program's actual runtime behavior.

\highspace
\begin{flushleft}
    \textcolor{Green3}{\faIcon{check-circle} \textbf{When does static branch prediction work well?}}
\end{flushleft}
This approach is particularly effective in scenarios where the \textbf{branch behavior is stable and highly predictable}, such as in embedded or domain-specific applications. In such cases, the overhead and complexity of dynamic prediction mechanisms may not justify the potential benefits, making static prediction a practical alternative.

\highspace
\begin{flushleft}
    \textcolor{Red2}{\faIcon{exclamation-triangle} \textbf{RISC-V assumption}}
\end{flushleft}
A key architectural note here is the assumption that we are working with a \textbf{RISC-V processor}, which is \textbf{optimized for early branch evaluation during the Instruction Decode (ID) stage} (see more in section \ref{def: Early Branch Evaluation}, page \hqpageref{def: Early Branch Evaluation}). This means that in RISC-V, the decision to predict a branch direction occurs early in the pipeline, minimizing the potential for instruction fetch delays if the prediction is accurate.
