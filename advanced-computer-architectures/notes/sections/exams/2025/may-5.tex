\section{Exams}

\subsection{2025}

\subsubsection{Midterm - May 5}

\subsubsection*{Exercise 1.A - Software Pipelining}

\emph{Consider the following software pipelined loop \texttt{SP\_LOOP} and the corresponding start-up and finish-up code:}
\begin{lstlisting}
START_UP:   LD F0, 0 (R1)
            LD F2, 0 (R2)
            FADD F4, F0, F0
            FADD F6, F2, F2
            LD F0, 8 (R1)
            LD F2, 8 (R2)

SP_LOOP:    SD F4, 0 (R1)
            SD F6, 0 (R2)
            FADD F4, F0, F0
            FADD F6, F2, F2
            LD F0, 16 (R1)
            LD F2, 16 (R2)
            ADDUI R1, R1, 8
            ADDUI R2, R2, 8
            ADD R3, R2, R1
            BNE R3, R4, SP_LOOP

FINISH_UP:  SD F4, 8 (R1)
            SD F6, 8 (R2)
            FADD F4, F0, F0
            FADD F6, F2, F2
            SD F4, 16 (R1)
            SD F6, 16 (R2)
\end{lstlisting}
\emph{Reconstruct the original (non-pipelined) version of the code.}

\highspace
\textcolor{Green3}{\textbf{\emph{Answer:}}} \definition{Software Pipelining} is a loop optimization technique. Instead of processing iterations strictly sequentially, it overlaps instructions from multiple iterations. So in the pipelined version:
\begin{itemize}
    \item The loop body does \emph{parts} of \hl{multiple iterations at the same time}.
    \item Because of this, we need extra \textbf{start-up} code (to prime the pipeline) and \textbf{finish-up} code (to flush the remaining operations).
\end{itemize}
The original loop must do \textbf{one iteration at a time}. In the original \textbf{non-pipelined loop}, we:
\begin{itemize}
    \item Load the data for the current iteration.
    \item Do the computation.
    \item Store the results.
    \item Update pointers.
    \item Check whether you are done.
    \item Repeat.
\end{itemize}

\highspace
In software pipelining we always have:
\begin{enumerate}
    \item Start-up code
    \item Steady-state loop
    \item Finish-up code
\end{enumerate}
Because the pipeline overlaps \emph{parts of different iterations}. So we need:
\begin{itemize}
    \item Extra code at the \textbf{start} to \emph{prime} the pipeline with partial work.
    \item The \textbf{main loop} which does the steady overlapping work for all \emph{middle} iterations.
    \item Extra code at the \textbf{end} to \emph{drain} the pipeline, to finish the partial work left from the last overlapping iterations.
\end{itemize}
In other words, we need to reverse engineer the original code for this exercise.

\highspace
The software pipelined loop usually overlaps:
\begin{itemize}
    \item \emph{Load} for \textbf{next} iteration.
    \item \emph{Compute} for \textbf{current} iteration.
    \item \emph{Store} for \textbf{previous} iteration.
\end{itemize}
The goal is trace the dependency chain:
\begin{itemize}
    \item Which \texttt{LD} feeds which \texttt{FADD}.
    \item Which \texttt{FADD} feeds which \texttt{SD}.
\end{itemize}
So the key is to identify the loop-carried data flow and align addresses to match.

\highspace
\hl{General 5-step method:}
\begin{enumerate}
    \item \textbf{Mark what the pipelined loop does}. In our case:
    \begin{lstlisting}
SP_LOOP:
    SD F4, 0(R1)         # Store result of previous iteration?
    SD F6, 0(R2)

    FADD F4, F0, F0      # Compute for current iteration?
    FADD F6, F2, F2

    LD F0, 16(R1)        # Load for next iteration
    LD F2, 16(R2)

    ADDUI R1, R1, 8      # Increment base pointers
    ADDUI R2, R2, 8\end{lstlisting}

    \item \textbf{Figure out the purpose}
    \begin{itemize}
        \item \texttt{LD}: prefetch next input.
        \item \texttt{FADD}: do the math for the loaded input.
        \item \texttt{SD}: store the result \textbf{from previous iteration}.
    \end{itemize}

    \newpage

    \item \textbf{Sketch the timeline}. Think conceptually:
    \begin{table}[!htp]
        \centering
        \begin{tabular}{@{} l l @{}}
            \toprule
            Pipeline Stage & Operation \\
            \midrule
            1. Load     & Loads new data for \texttt{i + 1} \\ [.5em]
            2. Compute  & Uses old load for \texttt{i}      \\ [.5em]
            3. Store    & Saves result for \texttt{i - 1}   \\
            \bottomrule
        \end{tabular}
    \end{table}

    \item \textbf{Match start-up and finish-up}
    \begin{itemize}
        \item \textbf{Start-up}: primes the pipeline, so gets first \texttt{F0}, \texttt{F2} ready.
        \item \textbf{Loop}: repeats the middle chunk.
        \item \textbf{Finish-up}: drains the leftover results.
    \end{itemize}

    \item \textbf{Write the original loop}. For each iteration:
    \begin{itemize}
        \item Load \texttt{F0, F2} $\leftarrow$ data for $i$
        \item Compute \texttt{F4 = F0 + F0}, \texttt{F6 = F2 + F2} $\leftarrow$ process for $i$
        \item Store \texttt{F4, F6} $\leftarrow$ save result for $i$
        \item Increment pointers
        \item Check loop condition
    \end{itemize}
\end{enumerate}
So, here's the final solution:
\begin{lstlisting}
LOOP:   LD F0, 0 (R1)
        LD F2, 0 (R2)
        FADD F4, F0, F0
        FADD F6, F2, F2
        SD F4, 0 (R1)
        SD F6, 0 (R2)
        ADDUI R1, R1, 8
        ADDUI R2, R2, 8
        ADD R3, R2, R1
        BNE R3, R4, LOOP
\end{lstlisting}

\newpage

\subsubsection*{Exercise 1.B - Software Pipelining}

\emph{Give the same software pipelined loop of \textbf{Exercise 1.A}:}
\begin{lstlisting}
SP_LOOP:    SD F4, 0 (R1)
            SD F6, 0 (R2)
            FADD F4, F0, F0
            FADD F6, F2, F2
            LD F0, 16 (R1)
            LD F2, 16 (R2)
            ADDUI R1, R1, 8
            ADDUI R2, R2, 8
            ADD R3, R2, R1
            BNE R3, R4, SP_LOOP
\end{lstlisting}
Consider a \textbf{3-issue VLIW} machine with \textbf{fully pipelined functional units}:
\begin{itemize}
    \item \textbf{1 Memory Units with 3 cycles latency}
    \item \textbf{1 FP ALUs with 3 cycles latency}
    \item \textbf{1 Integer ALU with 1 cycle latency to next Int/FP \& 2 cycle latency to next Branch}
\end{itemize}
The branch is completed with 1 cycle delay slot (branch solved in ID stage). \textbf{No branch prediction}. In the Register File, it is possible to read and write at the same address at the same clock cycle.
\begin{enumerate}
    \item \emph{Considering one iteration of the \texttt{SP\_LOOP}, complete the following table by using the \textbf{list-based scheduling} (do NOT introduce any loop unrolling and modifications to loop indexes) on the 3-issue VLIW machine including the BRANCH DELAY SLOT. Please do not write in NOPs.}

    \textcolor{Green3}{\textbf{\emph{Answer:}}} In a \textbf{VLIW architecture} (Very Long Instruction Word), a single instruction word encodes \textbf{multiple operations} that execute \emph{in parallel}. The compiler does the heavy lifting: it schedules independent instructions side by side in the same very long instruction word. The hardware just \emph{fires} all operations in parallel, it does \textbf{not} dynamically check for dependencies at runtime (unlike out-of-order superscalar).

    \highspace
    \textcolor{Green3}{\faIcon{question-circle} \textbf{What does \emph{3-issue} mean?}} It is the issue width and is the number of operations the processor \emph{fetches, decodes, and executes} in parallel \textbf{per cycle}. So, with 3-issue, the processor can execute up to \textbf{3 instructions per clock cycles}. They must be \textbf{independent} enough for the compiler to pack them together.

    \highspace
    \textcolor{Green3}{\faIcon{question-circle} \textbf{What does \emph{fully pipelined functional units} mean?}} Each type of operation is done in its own \textbf{functional unit} (e.g., integer ALU, floating-point adder, load/store unit, etc.). \textbf{Fully pipelined} means multiple operations can be in \emph{different pipeline stages} of the same unit simultaneously. So, every clock cycle a new operation can start, even if a previous one isn't finished yet. For example, if an \texttt{FADD} unit is fully pipelined and the \texttt{FADD} takes 4 cycles to complete, it can accept \textbf{one new \texttt{FADD} per cycle}, the result pops our 4 cycles later.

    \highspace
    \textcolor{Green3}{\faIcon{question-circle} \textbf{What is \emph{list-based scheduling}?}} \definition{List-based scheduling} is a classic \textbf{compiler scheduling algorithm} used to assign operations to \emph{time slots} (cycles) and \emph{resources} (e.g., functional units) \textbf{while respecting dependencies}.
    \begin{itemize}
        \item It tries to pack \textbf{as many independent instructions as possible} into each cycle.
        \item It tries to \textbf{fill all slots} of a wide-issue machine (VLIW, superscalar).
        \item It tries to respect resource limits (e.g., 1 integer ALU, 1 FP adder).
    \end{itemize}
    It is called list-based because the compiler first builds a \textbf{dependency graph} (where the nodes are the operations and the edges are the dependencies) and then creates a \textbf{list} of \textbf{ready operations}. These are instructions \textbf{whose inputs are ready} and \textbf{whose predecessors have been completed}. Each cycle:
    \begin{itemize}
        \item It picks operations from the list.
        \item It assigns them to available slots/resources.
        \item It marks successors ready when dependencies clear.
    \end{itemize}    
    \textcolor{Red2}{\faIcon{exclamation-triangle} \textbf{What does \emph{In the Register File, it is possible to read and write at the same address in the same clock cycle} mean?}} The \textbf{register file} supports \emph{simultaneous read and write} to the \textbf{same register}. So, in one cycle:
    \begin{itemize}
        \item We might \textbf{read register} \texttt{F0} to feed an \texttt{FADD}.
        \item AND \textbf{write back a result} to \texttt{F0} in the same cycle.
    \end{itemize}
    The hardware guarantees the read gets the \textbf{old value}, the write updates it for \textbf{later cycles}.
    
    If the register file did not allow this, we'd have to schedule dependent instructions in separate cycles to avoid conflict (e.g., we couldn't read \texttt{F0} to feed \texttt{FADD} while overwriting \texttt{F0} with a new \texttt{LD}). Otherwise, if it is allowed, in same cycle we can:
    \begin{itemize}
        \item \texttt{LD F0, ...} (writes \texttt{F0})
        \item \texttt{FADD F4, F0, F0} (reads old \texttt{F0})
    \end{itemize}

    \highspace
    Here are some notes about the solution:
    \begin{itemize}
        \item We are using software pipelining, so the \texttt{SD} and \texttt{FADD} operators can start together because the \texttt{SD} operator stores the results of the previous iteration (or the initial loop, if it is the first iteration).

        \item The \texttt{ADDUI} operator starts when the first \texttt{SD} store finishes, thanks to the register file. In the same cycle, the load operation begins, as it loads the operands for the next iteration.

        \item Cycle 6 stalls because the \texttt{BNE} operator must wait for the result of the previous ADD and cannot start in cycle 6 because it reads old values.
    \end{itemize}

    \newpage

    \textcolor{Green3}{\faIcon{question-circle} \textbf{What should we do about our loop?}}
    \begin{enumerate}
        \item Build the \textbf{DAG} for one iteration. A Data Dependency Graph (DAG) for scheduling has nodes, which are operations or instructions, and edges, which are directed and show true dependencies (RAW). For example, an edge from \texttt{A} to \texttt{B} means that \texttt{B} must wait for \texttt{A} to finish.
        \item Draw a dashed line if there are WAR or WAW hazards. This means we can perform these two instructions in the same cycle if a FU free is available.
        \item In the table, write the instructions from start to finish.
    \end{enumerate}
    \begin{center}
        \begin{tikzpicture}[node distance=1cm, >=stealth]
            % Nodes
            \node[circle, draw] (I1) {$I_{1}$};
            \node[circle, draw, below=of I1] (I3) {$I_{3}$};
            \node[circle, draw, below=of I3] (I5) {$I_{5}$};
            \node[circle, draw, below=of I5] (I7) {$I_{7}$};

            \node[circle, draw, right=3cm of I1] (I2) {$I_{2}$};
            \node[circle, draw, below=of I2] (I4) {$I_{4}$};
            \node[circle, draw, below=of I4] (I6) {$I_{6}$};
            \node[circle, draw, below=of I6] (I8) {$I_{8}$};

            \node[circle, draw, below=1cm of $(I7)!0.5!(I8)$] (I9) {$I_{9}$};
            \node[circle, draw, below=of I9] (I10) {$I_{10}$};

            % Dashed Edges
            \draw[dashed, ->] (I1) -- (I3);
            \draw[dashed, ->] (I3) -- (I5);
            \draw[dashed, ->] (I5) -- (I7);
            \draw[dashed, ->] (I2) -- (I4);
            \draw[dashed, ->] (I4) -- (I6);
            \draw[dashed, ->] (I6) -- (I8);

            % Solid Edges
            \draw[->] (I7) -- (I9);
            \draw[->] (I8) -- (I9);
            \draw[->] (I9) -- (I10);

        \end{tikzpicture}
    \end{center}

    \begin{table}[!htp]
        \centering
        \begin{tabular}{@{} l | l | l | l @{}}
            \toprule
            & \textbf{Memory Unit} & \textbf{Floating Point Unit} & \textbf{Integer Unit} \\
            \midrule
            \textbf{C1}  & \texttt{SD F4, 0 (R1)} & \texttt{FADD F4, F0, F0} & \\ [.3em]
            \textbf{C2}  & \texttt{SD F6, 0 (R2)} & \texttt{FADD F6, F2, F2} & \\ [.3em]
            \textbf{C3}  & \texttt{LD F0, 16 (R1)} & & \texttt{ADDUI R1, R1, 8} \\ [.3em]
            \textbf{C4}  & \texttt{LD F2, 16 (R2)} & & \texttt{ADDUI R2, R2, 8} \\ [.3em]
            \textbf{C5}  & & & \texttt{ADD R3, R2, R1} \\ [.3em]
            \textbf{C6}  & & & \\ [.3em]
            \textbf{C7}  & & & \texttt{BNE R3, R4, SP\_LOOP} \\ [.3em]
            \textbf{C8}  & & & (branch delay) \\
            \bottomrule
        \end{tabular}
    \end{table}

    \newpage


    \item \emph{How long is the critical path for a single iteration?}

    \textcolor{Green3}{\textbf{\emph{Answer:}}} in \textbf{instruction scheduling}, the \textbf{critical path} is the longest chain of dependent operations that must be executed \emph{in sequence} because of true data dependencies (RAW). It determines the \textbf{minimum possible latency} to complete \emph{one full logical iteration}. The critical path length depends on the data dependencies and the assumed functional unit latencies, \textbf{not} the scheduling algorithm.

    In our case, there are 8 cycles for each iteration, so the \textbf{critical path is 8}.


    \item \emph{What performance did you achieve in CPI?}

    \textcolor{Green3}{\textbf{\emph{Answer:}}} The Clock Per Instruction (CPI) can be calculated using the following formula:
    \begin{equation*}
        \text{CPI} = \dfrac{\text{\# Clock Cycles}}{\text{IC}} = \dfrac{\text{\# Clock Cycles}}{\text{\# Instructions}} = \dfrac{8}{10} = 0.8
    \end{equation*}


    \item \emph{What performance did you achieve in FP ops per cycles?}
    
    \textcolor{Green3}{\textbf{\emph{Answer:}}} FP ops per cycle measures \textbf{how well we are utilizing the FP functional units}. Here, there are 2 floating-point additions (\texttt{FADD F4}, \texttt{FADD F6}):
    \begin{equation*}
        \text{FP ops / cycle} = \dfrac{\text{\# Floating-Point ops}}{\text{\# Clock Cycles}} = \dfrac{2}{8} = 0.25
    \end{equation*}


    \item \emph{How much is the code efficiency?}
    
    \textcolor{Green3}{\textbf{\emph{Answer:}}} When asked for \textbf{code efficiency} in this VLIW/ILP context, they usually mean:
    \begin{equation*}
        \text{Efficiency} = \frac{\text{Achieved IPC}}{\text{Maximum Theoretical IPC}}
    \end{equation*}
    So we measure \textbf{how much of the machine's peak parallelism we actually use}. The achieved Instruction Per Cycle (IPC) is:
    \begin{equation*}
        \text{Achieved IPC} = \dfrac{\text{\# Instructions}}{\text{\# Clock Cycles}} = \dfrac{10}{8} = 1.25
    \end{equation*}
    Here, VLIW is 3-issue, so it can execute up to \textbf{3 ops per cycle} if there's no data or resource limit ($\text{Peak IPC} = 3$). So, code efficiency is:
    \begin{equation*}
        \text{Efficiency} = \frac{1.25}{3} = 0.4167 \approx 41.67\% \approx 42\%
    \end{equation*}
    \hl{We're achieving about 41.67\%} ($\approx 42 \%$) of the maximum possible execution parallelism.
\end{enumerate}

\newpage

\subsubsection*{Exercise 2.A - Dependency Analysis}

\emph{Consider the same software pipelined loop of \textbf{Exercise 1.A} containing multiple types of intra-loop dependences. Complete the following table by inserting all types of true-data-dependences, anti-
dependences and output dependences for each instruction:}

\highspace
\textcolor{Green3}{\textbf{\emph{Answer:}}}

\begin{table}[!htp]
    \centering
    \begin{tabular}{@{} l | l | p{20em} @{}}
        \toprule
        \texttt{I\#} & \textbf{Type of instruction}   & \textbf{Analysis of dependences:} \begin{enumerate}
            \item True data dependence with \texttt{I\#} for \texttt{\$Fx}
            \item Anti-dependence with \texttt{I\#} for \texttt{\$Fy}
            \item Output-dependence with \texttt{I\#} for \texttt{\$Fz}
        \end{enumerate} \\
        \midrule
        I1  & \texttt{SD F4, 0 (R1)} & None \\ [.3em]
        I2  & \texttt{SD F6, 0 (R2)} & None \\ [.3em]
        I3  & \texttt{FADD F4, F0, F0} & Anti-dependence with \texttt{I1} for \texttt{F4} \\ [.3em]
        I4  & \texttt{FADD F6, F2, F2} & Anti-dependence with \texttt{I2} for \texttt{F6} \\ [.3em]
        I5  & \texttt{LD F0, 16 (R1)} & Anti-dependence with \texttt{I3} for \texttt{F0} \\ [.3em]
        I6  & \texttt{LD F2, 16 (R2)} & Anti-dependence with \texttt{I4} for \texttt{F2} \\ [.3em]
        I7  & \texttt{ADDUI R1, R1, 8} & Anti-dependence with \texttt{I1} for \texttt{R1} \newline Anti-dependence with \texttt{I5} for \texttt{R1} \\ [.3em]
        I8  & \texttt{ADDUI R2, R2, 8} & Anti-dependence with \texttt{I2} for \texttt{R2} \newline Anti-dependence with \texttt{I6} for \texttt{R2} \\ [.3em]
        I9  & \texttt{ADD R3, R2, R1} & True data dependence with \texttt{I7} for \texttt{R1} \newline True data dependence with \texttt{I8} for \texttt{R2} \\ [.3em]
        I10 & \texttt{BNE R3, R4, SP\_LOOP} & True data dependence with \texttt{I9} for \texttt{R3} \\
        \bottomrule
    \end{tabular}
\end{table}

\newpage

\subsubsection*{Exercise 2.B - Tomasulo}

\emph{Consider the same assembly code to be executed on a CPU with dynamic scheduling based on \textbf{Tomasulo algorithm} with all cache HITS, a single Common Data Bus and:}
\begin{itemize}
    \item \emph{2 RESERVATION STATIONS (\textbf{RS1}, \textbf{RS2}) with 2 LOAD/STORE units (\textbf{LDU1}, \textbf{LDU2}) with latency \textbf{4}}
    \item \emph{2 RESERVATION STATIONS (\textbf{RS3}, \textbf{RS4}) with 1 FP unit (\textbf{FPU1}) with latency \textbf{4}}
    \item \emph{2 RESERVATION STATIONS (\textbf{RS5}, \textbf{RS6}) with 2 INT\_ALU/BR units (\textbf{ALU1}, \textbf{ALU2}) with latency \textbf{2}}
\end{itemize}
\emph{Please complete the following table:}

\highspace
\textcolor{Green3}{\textbf{\emph{Answer:}}} Before solving the Tomasulo, we need to remember how it is composed (see page \pageref{fig: tomasulo fpu}).

\begin{table}[!htp]
    \centering
    \begin{adjustbox}{width={\textwidth},totalheight={\textheight},keepaspectratio}
    \begin{tabular}{@{} l c c c c c c @{}}
        \toprule
        \textbf{Instruction} & \textbf{Issue} & \textbf{Start Exec} & \textbf{Write Res} & \textbf{Hazard} & \textbf{RSi} & \textbf{Unit} \\
        \midrule
        \texttt{I1: SD F4, 0 (R1)}      & \hl{1} & 2 & 6 & None  & \hl{\texttt{RS1}}   & \texttt{LDU1}  \\ [.5em]
        \texttt{I2: SD F6, 0 (R2)}      & 2 & 3 & 7 & None  & \texttt{RS2}   & \texttt{LDU2}  \\ [.5em]
        \texttt{I3: FADD F4, F0, F0}    &   &   &   &       &       &       \\ [.5em]
        \texttt{I4: FADD F6, F2, F2}    &   &   &   &       &       &       \\ [.5em]
        \texttt{I5: LD F0, 16 (R1)}     &   &   &   &       &       &       \\ [.5em]
        \texttt{I6: LD F2, 16 (R2)}     &   &   &   &       &       &       \\ [.5em]
        \texttt{I7: ADDUI R1, R1, 8}    &   &   &   &       &       &       \\ [.5em]
        \texttt{I8: ADDUI R2, R2, 8}    &   &   &   &       &       &       \\ [.5em]
        \texttt{I9: ADD R3, R2, R1}     &   &   &   &       &       &       \\ [.5em]
        \texttt{I10: BNE R3, R4, LOOP}   &   &   &   &       &       &       \\
        \bottomrule
    \end{tabular}
    \end{adjustbox}
\end{table}

% \begin{minipage}{0.45\textwidth}
%     \centering
%     \begin{tabular}{@{} l | l l l l @{}}
%         \toprule
%         & \texttt{Vj} & \texttt{Qj} & \texttt{Vk} & \texttt{Qk} \\
%         \midrule
%         RS1 & & & \\ [.3em]
%         RS2 & & & \\ [.3em]
%         \cmidrule{1-5}
%         LDU1 & & & \\ [.3em]
%         LDU2 & & & \\
%         \bottomrule
%     \end{tabular}
% \end{minipage}
% \hfill
% \begin{minipage}{0.45\textwidth}
%     \centering
%     \begin{tabular}{@{} l | l l l l @{}}
%         \toprule
%         & \texttt{Vj} & \texttt{Qj} & \texttt{Vk} & \texttt{Qk} \\
%         \midrule
%         RS3 & & & \\ [.3em]
%         RS4 & & & \\ [.3em]
%         \cmidrule{1-5}
%         FPU1 & & & \\
%         \bottomrule
%     \end{tabular}
% \end{minipage}
% 
% \begin{center}
%     \begin{tabular}{@{} l | l l l l @{}}
%         \toprule
%         & \texttt{Vj} & \texttt{Qj} & \texttt{Vk} & \texttt{Qk} \\
%         \midrule
%         RS5 & & & \\ [.3em]
%         RS6 & & & \\ [.3em]
%         \cmidrule{1-5}
%         ALU1 & & & \\ [.3em]
%         ALU2 & & & \\
%         \bottomrule
%     \end{tabular}
% \end{center}

\begin{enumerate}
    \item \textbf{Cycle \theenumi}:
    
    \begin{minipage}{0.45\textwidth}
        \centering
        \begin{tabular}{@{} l | l l l l @{}}
            \toprule
                & \texttt{Vj} & \texttt{Qj} & \texttt{Vk} & \texttt{Qk} \\
            \midrule
            \texttt{RS1} & \hl{\texttt{F4}} & & \hl{\texttt{-}} & \\ [.3em]
            \texttt{RS2} & & & & \\ [.3em]
            \cmidrule{1-5}
            \texttt{LDU1} & & & & \\ [.3em]
            \texttt{LDU2} & & & & \\
            \bottomrule
        \end{tabular}
    \end{minipage}
    \hfill
    \begin{minipage}{0.45\textwidth}
        \centering
        \begin{tabular}{@{} l | l l l l @{}}
            \toprule
            & \texttt{Vj} & \texttt{Qj} & \texttt{Vk} & \texttt{Qk} \\
            \midrule
            \texttt{RS3} & & & & \\ [.3em]
            \texttt{RS4} & & & & \\ [.3em]
            \cmidrule{1-5}
            \texttt{FPU1} & & & & \\
            \bottomrule
        \end{tabular}
    \end{minipage}

    \begin{center}
        \begin{tabular}{@{} l | l l l l @{}}
            \toprule
            & \texttt{Vj} & \texttt{Qj} & \texttt{Vk} & \texttt{Qk} \\
            \midrule
            \texttt{RS5} & & & & \\ [.3em]
            \texttt{RS6} & & & & \\ [.3em]
            \cmidrule{1-5}
            \texttt{ALU1} & & & & \\ [.3em]
            \texttt{ALU2} & & & & \\
            \bottomrule
        \end{tabular}
    \end{center}
    \newpage
    


    %%%%%%%%%%%%%%%%%%%%%%%%%%%%%%%%%%%%%%%%%%%%%%%%%%%%%%%%%%%%%%%%%%%%%%%%%%%%%%%%%%%%%%%%%%%%%%%%%%%%%%%%
    \item \textbf{Cycle \theenumi}:
    
    \begin{table}[!htp]
        \centering
        \begin{adjustbox}{width={\textwidth},totalheight={\textheight},keepaspectratio}
        \begin{tabular}{@{} l c c c c c c @{}}
            \toprule
            \textbf{Instruction} & \textbf{Issue} & \textbf{Start Exec} & \textbf{Write Res} & \textbf{Hazard} & \textbf{RSi} & \textbf{Unit} \\
            \midrule
            \texttt{I1: SD F4, 0 (R1)}      & 1 & \hl{2} & 6 & None  & \texttt{RS1}   & \hl{\texttt{LDU1}}  \\ [.5em]
            \texttt{I2: SD F6, 0 (R2)}      & \hl{2} & 3 & 7 & None  & \hl{\texttt{RS2}}   & \texttt{LDU2}  \\ [.5em]
            \texttt{I3: FADD F4, F0, F0}    &   &   &   &       &       &       \\ [.5em]
            \texttt{I4: FADD F6, F2, F2}    &   &   &   &       &       &       \\ [.5em]
            \texttt{I5: LD F0, 16 (R1)}     &   &   &   &       &       &       \\ [.5em]
            \texttt{I6: LD F2, 16 (R2)}     &   &   &   &       &       &       \\ [.5em]
            \texttt{I7: ADDUI R1, R1, 8}    &   &   &   &       &       &       \\ [.5em]
            \texttt{I8: ADDUI R2, R2, 8}    &   &   &   &       &       &       \\ [.5em]
            \texttt{I9: ADD R3, R2, R1}     &   &   &   &       &       &       \\ [.5em]
            \texttt{I10: BNE R3, R4, LOOP}   &   &   &   &       &       &       \\
            \bottomrule
        \end{tabular}
        \end{adjustbox}
    \end{table}
    
    \begin{minipage}{0.45\textwidth}
        \centering
        \begin{tabular}{@{} l | l l l l @{}}
            \toprule
                & \texttt{Vj} & \texttt{Qj} & \texttt{Vk} & \texttt{Qk} \\
            \midrule
            \texttt{RS1} & \texttt{F4} & & \texttt{-} & \\ [.3em]
            \texttt{RS2} & \hl{\texttt{F6}} & & \hl{\texttt{-}} & \\
            \cmidrule{1-5}
            \texttt{LDU1} & \hl{\texttt{F4}} & & \hl{\texttt{-}} & \\ [.3em]
            \texttt{LDU2} & & & & \\
            \bottomrule
        \end{tabular}
    \end{minipage}
    \hfill
    \begin{minipage}{0.45\textwidth}
        \centering
        \begin{tabular}{@{} l | l l l l @{}}
            \toprule
            & \texttt{Vj} & \texttt{Qj} & \texttt{Vk} & \texttt{Qk} \\
            \midrule
            \texttt{RS3} & & & & \\ [.3em]
            \texttt{RS4} & & & & \\
            \cmidrule{1-5}
            \texttt{FPU1} & & & & \\
            \bottomrule
        \end{tabular}
    \end{minipage}

    \begin{minipage}{0.45\textwidth}
        \centering
        \begin{tabular}{@{} l | l l l l @{}}
            \toprule
            & \texttt{Vj} & \texttt{Qj} & \texttt{Vk} & \texttt{Qk} \\
            \midrule
            \texttt{RS5} & & & & \\ [.3em]
            \texttt{RS6} & & & & \\
            \cmidrule{1-5}
            \texttt{ALU1} & & & & \\ [.3em]
            \texttt{ALU2} & & & & \\
            \bottomrule
        \end{tabular}
    \end{minipage}
    \hfill
    \begin{minipage}{0.45\textwidth}
        \centering
        \begin{tabular}{@{} l c @{}}
            \toprule
            Unit            & Remaining cycles \\
            \midrule
            \texttt{LDU1}   & 4 \\ [.3em]
            \texttt{LDU2}   & \\ [.3em]
            \texttt{FPU1}   & \\ [.3em]
            \texttt{ALU1}   & \\ [.3em]
            \texttt{ALU2}   & \\
            \bottomrule
        \end{tabular}
    \end{minipage}
    \newpage




    %%%%%%%%%%%%%%%%%%%%%%%%%%%%%%%%%%%%%%%%%%%%%%%%%%%%%%%%%%%%%%%%%%%%%%%%%%%%%%%%%%%%%%%%%%%%%%%%%%%%%%%%
    \item \textbf{Cycle \theenumi}: Here, we have a anti-dependence hazard (WAR) because the third instruction writes the sum to register \texttt{F4} and the first instruction reads \texttt{F4}. In Tomasulo, though, we don't care about this because the issue stage stalls if and only if there are no reservation stations, the Load/Store queue is full (for memory ops), or for speculative instructions (for prediction, which is not our case). Therefore, we issue and start execution. At the end of execution, we will determine whether to write the result or wait.
    
    \begin{table}[!htp]
        \centering
        \begin{adjustbox}{width={\textwidth},totalheight={\textheight},keepaspectratio}
        \begin{tabular}{@{} l c c c c c c @{}}
            \toprule
            \textbf{Instruction} & \textbf{Issue} & \textbf{Start Exec} & \textbf{Write Res} & \textbf{Hazard} & \textbf{RSi} & \textbf{Unit} \\
            \midrule
            \texttt{I1: SD F4, 0 (R1)}      & 1 & 2 & 6 & None  & \texttt{RS1}   & \texttt{LDU1}  \\ [.5em]
            \texttt{I2: SD F6, 0 (R2)}      & 2 & \hl{3} & 7 & None  & \texttt{RS2}   & \hl{\texttt{LDU2}}  \\ [.5em]
            \texttt{I3: FADD F4, F0, F0}    & \hl{3} &   &   & \hl{\texttt{WAR I1, F4}} & \hl{\texttt{RS3}} &       \\ [.5em]
            \texttt{I4: FADD F6, F2, F2}    &   &   &   &       &       &       \\ [.5em]
            \texttt{I5: LD F0, 16 (R1)}     &   &   &   &       &       &       \\ [.5em]
            \texttt{I6: LD F2, 16 (R2)}     &   &   &   &       &       &       \\ [.5em]
            \texttt{I7: ADDUI R1, R1, 8}    &   &   &   &       &       &       \\ [.5em]
            \texttt{I8: ADDUI R2, R2, 8}    &   &   &   &       &       &       \\ [.5em]
            \texttt{I9: ADD R3, R2, R1}     &   &   &   &       &       &       \\ [.5em]
            \texttt{I10: BNE R3, R4, LOOP}   &   &   &   &       &       &       \\
            \bottomrule
        \end{tabular}
        \end{adjustbox}
    \end{table}
    
    \begin{minipage}{0.45\textwidth}
        \centering
        \begin{tabular}{@{} l | l l l l @{}}
            \toprule
                & \texttt{Vj} & \texttt{Qj} & \texttt{Vk} & \texttt{Qk} \\
            \midrule
            \texttt{RS1} & \texttt{F4} & & \texttt{-} & \\ [.3em]
            \texttt{RS2} & \texttt{F6} & & \texttt{-} & \\
            \cmidrule{1-5}
            \texttt{LDU1} & \texttt{F4} & & \texttt{-} & \\ [.3em]
            \texttt{LDU2} & \hl{\texttt{F6}} & & \hl{\texttt{-}} & \\
            \bottomrule
        \end{tabular}
    \end{minipage}
    \hfill
    \begin{minipage}{0.45\textwidth}
        \centering
        \begin{tabular}{@{} l | l l l l @{}}
            \toprule
            & \texttt{Vj} & \texttt{Qj} & \texttt{Vk} & \texttt{Qk} \\
            \midrule
            \texttt{RS3} & \hl{\texttt{F0}} & & \hl{\texttt{F0}} & \\ [.3em]
            \texttt{RS4} & & & & \\
            \cmidrule{1-5}
            \texttt{FPU1} & & & & \\
            \bottomrule
        \end{tabular}
    \end{minipage}

    \begin{minipage}{0.45\textwidth}
        \centering
        \begin{tabular}{@{} l | l l l l @{}}
            \toprule
            & \texttt{Vj} & \texttt{Qj} & \texttt{Vk} & \texttt{Qk} \\
            \midrule
            \texttt{RS5} & & & & \\ [.3em]
            \texttt{RS6} & & & & \\
            \cmidrule{1-5}
            \texttt{ALU1} & & & & \\ [.3em]
            \texttt{ALU2} & & & & \\
            \bottomrule
        \end{tabular}
    \end{minipage}
    \hfill
    \begin{minipage}{0.45\textwidth}
        \centering
        \begin{tabular}{@{} l c @{}}
            \toprule
            Unit            & Remaining cycles \\
            \midrule
            \texttt{LDU1}   & 3 \\ [.3em]
            \texttt{LDU2}   & 4 \\ [.3em]
            \texttt{FPU1}   & \\ [.3em]
            \texttt{ALU1}   & \\ [.3em]
            \texttt{ALU2}   & \\
            \bottomrule
        \end{tabular}
    \end{minipage}
    \newpage




    %%%%%%%%%%%%%%%%%%%%%%%%%%%%%%%%%%%%%%%%%%%%%%%%%%%%%%%%%%%%%%%%%%%%%%%%%%%%%%%%%%%%%%%%%%%%%%%%%%%%%%%%
    \item \textbf{Cycle \theenumi}: It's the same situation as before with the hazard (WAR), but now it's \texttt{I4} with \texttt{I2}.
    
    \begin{table}[!htp]
        \centering
        \begin{adjustbox}{width={\textwidth},totalheight={\textheight},keepaspectratio}
        \begin{tabular}{@{} l c c c c c c @{}}
            \toprule
            \textbf{Instruction} & \textbf{Issue} & \textbf{Start Exec} & \textbf{Write Res} & \textbf{Hazard} & \textbf{RSi} & \textbf{Unit} \\
            \midrule
            \texttt{I1: SD F4, 0 (R1)}      & 1 & 2 & 6 & None  & \texttt{RS1}   & \texttt{LDU1}  \\ [.5em]
            \texttt{I2: SD F6, 0 (R2)}      & 2 & 3 & 7 & None  & \texttt{RS2}   & \texttt{LDU2}  \\ [.5em]
            \texttt{I3: FADD F4, F0, F0}    & 3 & \hl{4} &   & \texttt{WAR I1, F4}  & \texttt{RS3} & \hl{\texttt{FPU1}} \\ [.5em]
            \texttt{I4: FADD F6, F2, F2}    & \hl{4} &   &   & \hl{\texttt{WAR I2, F6}} & \hl{\texttt{RS4}} &       \\ [.5em]
            \texttt{I5: LD F0, 16 (R1)}     &   &   &   &       &       &       \\ [.5em]
            \texttt{I6: LD F2, 16 (R2)}     &   &   &   &       &       &       \\ [.5em]
            \texttt{I7: ADDUI R1, R1, 8}    &   &   &   &       &       &       \\ [.5em]
            \texttt{I8: ADDUI R2, R2, 8}    &   &   &   &       &       &       \\ [.5em]
            \texttt{I9: ADD R3, R2, R1}     &   &   &   &       &       &       \\ [.5em]
            \texttt{I10: BNE R3, R4, LOOP}   &   &   &   &       &       &       \\
            \bottomrule
        \end{tabular}
        \end{adjustbox}
    \end{table}
    
    \begin{minipage}{0.45\textwidth}
        \centering
        \begin{tabular}{@{} l | l l l l @{}}
            \toprule
                & \texttt{Vj} & \texttt{Qj} & \texttt{Vk} & \texttt{Qk} \\
            \midrule
            \texttt{RS1} & \texttt{F4} & & \texttt{-} & \\ [.3em]
            \texttt{RS2} & \texttt{F6} & & \texttt{-} & \\
            \cmidrule{1-5}
            \texttt{LDU1} & \texttt{F4} & & \texttt{-} & \\ [.3em]
            \texttt{LDU2} & \texttt{F6} & & \texttt{-} & \\
            \bottomrule
        \end{tabular}
    \end{minipage}
    \hfill
    \begin{minipage}{0.45\textwidth}
        \centering
        \begin{tabular}{@{} l | l l l l @{}}
            \toprule
            & \texttt{Vj} & \texttt{Qj} & \texttt{Vk} & \texttt{Qk} \\
            \midrule
            \texttt{RS3} & \texttt{F0} & & \texttt{F0} & \\ [.3em]
            \texttt{RS4} & \hl{\texttt{F2}} & & \hl{\texttt{F2}} & \\
            \cmidrule{1-5}
            \texttt{FPU1} & \hl{\texttt{F0}} & & \hl{\texttt{F0}} & \\
            \bottomrule
        \end{tabular}
    \end{minipage}

    \begin{minipage}{0.45\textwidth}
        \centering
        \begin{tabular}{@{} l | l l l l @{}}
            \toprule
            & \texttt{Vj} & \texttt{Qj} & \texttt{Vk} & \texttt{Qk} \\
            \midrule
            \texttt{RS5} & & & & \\ [.3em]
            \texttt{RS6} & & & & \\
            \cmidrule{1-5}
            \texttt{ALU1} & & & & \\ [.3em]
            \texttt{ALU2} & & & & \\
            \bottomrule
        \end{tabular}
    \end{minipage}
    \hfill
    \begin{minipage}{0.45\textwidth}
        \centering
        \begin{tabular}{@{} l c @{}}
            \toprule
            Unit            & Remaining cycles \\
            \midrule
            \texttt{LDU1}   & 2 \\ [.3em]
            \texttt{LDU2}   & 3 \\ [.3em]
            \texttt{FPU1}   & 4 \\ [.3em]
            \texttt{ALU1}   & \\ [.3em]
            \texttt{ALU2}   & \\
            \bottomrule
        \end{tabular}
    \end{minipage}
    \newpage






    %%%%%%%%%%%%%%%%%%%%%%%%%%%%%%%%%%%%%%%%%%%%%%%%%%%%%%%%%%%%%%%%%%%%%%%%%%%%%%%%%%%%%%%%%%%%%%%%%%%%%%%%
    \item \textbf{Cycle \theenumi}: Instruction 4 cannot begin because there are no available functional units. Therefore, it remains idle at reservation station 4. Instructions 5 and 6 cannot start either because there are no load/store reservation units available. \texttt{ADDUI} operations cannot start because instructions 5 and 6 need to read the \texttt{R1} and \texttt{R2} registers (WAR hazard).
    
    \begin{table}[!htp]
        \centering
        \begin{adjustbox}{width={\textwidth},totalheight={\textheight},keepaspectratio}
        \begin{tabular}{@{} l c c c c c c @{}}
            \toprule
            \textbf{Instruction} & \textbf{Issue} & \textbf{Start Exec} & \textbf{Write Res} & \textbf{Hazard} & \textbf{RSi} & \textbf{Unit} \\
            \midrule
            \texttt{I1: SD F4, 0 (R1)}      & 1 & 2 & 6 & None  & \texttt{RS1}   & \texttt{LDU1}  \\ [.5em]
            \texttt{I2: SD F6, 0 (R2)}      & 2 & 3 & 7 & None  & \texttt{RS2}   & \texttt{LDU2}  \\ [.5em]
            \texttt{I3: FADD F4, F0, F0}    & 3 & 4 &   & \texttt{WAR I1, F4}  & \texttt{RS3} & \texttt{FPU1} \\ [.5em]
            \texttt{I4: FADD F6, F2, F2}    & 4 &   &   & \texttt{WAR I2, F6} & \texttt{RS4} &       \\ [.5em]
            \texttt{I5: LD F0, 16 (R1)}     &   &   &   & \hl{\texttt{WAR I3, F0}} &       &       \\ [.5em]
            \texttt{I6: LD F2, 16 (R2)}     &   &   &   & \hl{\texttt{WAR I4, F2}} &       &       \\ [.5em]
            \texttt{I7: ADDUI R1, R1, 8}    &   &   &   & \hl{\texttt{WAR I5, R1}} &       &       \\ [.5em]
            \texttt{I8: ADDUI R2, R2, 8}    &   &   &   & \hl{\texttt{WAR I6, R2}} &       &       \\ [.5em]
            \texttt{I9: ADD R3, R2, R1}     &   &   &   & \hl{\texttt{RAW R1, R2}} &       &       \\ [.5em]
            \texttt{I10: BNE R3, R4, LOOP}   &   &   &   & \hl{\texttt{RAW I9, R3}} &       &       \\
            \bottomrule
        \end{tabular}
        \end{adjustbox}
    \end{table}
    
    \begin{minipage}{0.45\textwidth}
        \centering
        \begin{tabular}{@{} l | l l l l @{}}
            \toprule
                & \texttt{Vj} & \texttt{Qj} & \texttt{Vk} & \texttt{Qk} \\
            \midrule
            \texttt{RS1} & \texttt{F4} & & \texttt{-} & \\ [.3em]
            \texttt{RS2} & \texttt{F6} & & \texttt{-} & \\
            \cmidrule{1-5}
            \texttt{LDU1} & \texttt{F4} & & \texttt{-} & \\ [.3em]
            \texttt{LDU2} & \texttt{F6} & & \texttt{-} & \\
            \bottomrule
        \end{tabular}
    \end{minipage}
    \hfill
    \begin{minipage}{0.45\textwidth}
        \centering
        \begin{tabular}{@{} l | l l l l @{}}
            \toprule
            & \texttt{Vj} & \texttt{Qj} & \texttt{Vk} & \texttt{Qk} \\
            \midrule
            \texttt{RS3} & \texttt{F0} & & \texttt{F0} & \\ [.3em]
            \texttt{RS4} & \texttt{F2} & & \texttt{F2} & \\
            \cmidrule{1-5}
            \texttt{FPU1} & \texttt{F0} & & \texttt{F0} & \\
            \bottomrule
        \end{tabular}
    \end{minipage}

    \begin{minipage}{0.45\textwidth}
        \centering
        \begin{tabular}{@{} l | l l l l @{}}
            \toprule
            & \texttt{Vj} & \texttt{Qj} & \texttt{Vk} & \texttt{Qk} \\
            \midrule
            \texttt{RS5} & & & & \\ [.3em]
            \texttt{RS6} & & & & \\
            \cmidrule{1-5}
            \texttt{ALU1} & & & & \\ [.3em]
            \texttt{ALU2} & & & & \\
            \bottomrule
        \end{tabular}
    \end{minipage}
    \hfill
    \begin{minipage}{0.45\textwidth}
        \centering
        \begin{tabular}{@{} l c @{}}
            \toprule
            Unit            & Remaining cycles \\
            \midrule
            \texttt{LDU1}   & 1 \\ [.3em]
            \texttt{LDU2}   & 2 \\ [.3em]
            \texttt{FPU1}   & 3 \\ [.3em]
            \texttt{ALU1}   & \\ [.3em]
            \texttt{ALU2}   & \\
            \bottomrule
        \end{tabular}
    \end{minipage}
    \newpage





    %%%%%%%%%%%%%%%%%%%%%%%%%%%%%%%%%%%%%%%%%%%%%%%%%%%%%%%%%%%%%%%%%%%%%%%%%%%%%%%%%%%%%%%%%%%%%%%%%%%%%%%%
    \item \textbf{Cycle \theenumi}:
    
    \begin{table}[!htp]
        \centering
        \begin{adjustbox}{width={\textwidth},totalheight={\textheight},keepaspectratio}
        \begin{tabular}{@{} l c c c c c c @{}}
            \toprule
            \textbf{Instruction} & \textbf{Issue} & \textbf{Start Exec} & \textbf{Write Res} & \textbf{Hazard} & \textbf{RSi} & \textbf{Unit} \\
            \midrule
            \texttt{I1: SD F4, 0 (R1)}      & 1 & 2 & \hl{6} & None  & \texttt{RS1}   & \texttt{LDU1}  \\ [.5em]
            \texttt{I2: SD F6, 0 (R2)}      & 2 & 3 & 7 & None  & \texttt{RS2}   & \texttt{LDU2}  \\ [.5em]
            \texttt{I3: FADD F4, F0, F0}    & 3 & 4 &   & \texttt{WAR I1, F4}  & \texttt{RS3} & \texttt{FPU1} \\ [.5em]
            \texttt{I4: FADD F6, F2, F2}    & 4 &   &   & \texttt{WAR I2, F6} & \texttt{RS4} &       \\ [.5em]
            \texttt{I5: LD F0, 16 (R1)}     &   &   &   & \texttt{WAR I3, F0} &       &       \\ [.5em]
            \texttt{I6: LD F2, 16 (R2)}     &   &   &   & \texttt{WAR I4, F2} &       &       \\ [.5em]
            \texttt{I7: ADDUI R1, R1, 8}    &   &   &   & \texttt{WAR I5, R1} &       &       \\ [.5em]
            \texttt{I8: ADDUI R2, R2, 8}    &   &   &   & \texttt{WAR I6, R2} &       &       \\ [.5em]
            \texttt{I9: ADD R3, R2, R1}     &   &   &   & \texttt{RAW R1, R2} &       &       \\ [.5em]
            \texttt{I10: BNE R3, R4, LOOP}   &   &   &   & \texttt{RAW I9, R3} &       &       \\
            \bottomrule
        \end{tabular}
        \end{adjustbox}
    \end{table}
    
    \begin{minipage}{0.45\textwidth}
        \centering
        \begin{tabular}{@{} l | l l l l @{}}
            \toprule
                & \texttt{Vj} & \texttt{Qj} & \texttt{Vk} & \texttt{Qk} \\
            \midrule
            \texttt{RS1} & \texttt{F4} & & \texttt{-} & \\ [.3em]
            \texttt{RS2} & \texttt{F6} & & \texttt{-} & \\
            \cmidrule{1-5}
            \texttt{LDU1} & \texttt{F4} & & \texttt{-} & \\ [.3em]
            \texttt{LDU2} & \texttt{F6} & & \texttt{-} & \\
            \bottomrule
        \end{tabular}
    \end{minipage}
    \hfill
    \begin{minipage}{0.45\textwidth}
        \centering
        \begin{tabular}{@{} l | l l l l @{}}
            \toprule
            & \texttt{Vj} & \texttt{Qj} & \texttt{Vk} & \texttt{Qk} \\
            \midrule
            \texttt{RS3} & \texttt{F0} & & \texttt{F0} & \\ [.3em]
            \texttt{RS4} & \texttt{F2} & & \texttt{F2} & \\
            \cmidrule{1-5}
            \texttt{FPU1} & \texttt{F0} & & \texttt{F0} & \\
            \bottomrule
        \end{tabular}
    \end{minipage}

    \begin{minipage}{0.45\textwidth}
        \centering
        \begin{tabular}{@{} l | l l l l @{}}
            \toprule
            & \texttt{Vj} & \texttt{Qj} & \texttt{Vk} & \texttt{Qk} \\
            \midrule
            \texttt{RS5} & & & & \\ [.3em]
            \texttt{RS6} & & & & \\
            \cmidrule{1-5}
            \texttt{ALU1} & & & & \\ [.3em]
            \texttt{ALU2} & & & & \\
            \bottomrule
        \end{tabular}
    \end{minipage}
    \hfill
    \begin{minipage}{0.45\textwidth}
        \centering
        \begin{tabular}{@{} l c @{}}
            \toprule
            Unit            & Remaining cycles \\
            \midrule
            \texttt{LDU1}   & 0 \\ [.3em]
            \texttt{LDU2}   & 1 \\ [.3em]
            \texttt{FPU1}   & 2 \\ [.3em]
            \texttt{ALU1}   & \\ [.3em]
            \texttt{ALU2}   & \\
            \bottomrule
        \end{tabular}
    \end{minipage}
    \newpage





    %%%%%%%%%%%%%%%%%%%%%%%%%%%%%%%%%%%%%%%%%%%%%%%%%%%%%%%%%%%%%%%%%%%%%%%%%%%%%%%%%%%%%%%%%%%%%%%%%%%%%%%%
    \item \textbf{Cycle \theenumi}:
    
    \begin{table}[!htp]
        \centering
        \begin{adjustbox}{width={\textwidth},totalheight={\textheight},keepaspectratio}
        \begin{tabular}{@{} l c c c c c c @{}}
            \toprule
            \textbf{Instruction} & \textbf{Issue} & \textbf{Start Exec} & \textbf{Write Res} & \textbf{Hazard} & \textbf{RSi} & \textbf{Unit} \\
            \midrule
            \texttt{I1: SD F4, 0 (R1)}      & 1 & 2 & 6 & None  & \texttt{RS1}   & \texttt{LDU1}  \\ [.5em]
            \texttt{I2: SD F6, 0 (R2)}      & 2 & 3 & \hl{7} & None  & \texttt{RS2}   & \texttt{LDU2}  \\ [.5em]
            \texttt{I3: FADD F4, F0, F0}    & 3 & 4 &   & \texttt{WAR I1, F4}  & \texttt{RS3} & \texttt{FPU1} \\ [.5em]
            \texttt{I4: FADD F6, F2, F2}    & 4 &   &   & \texttt{WAR I2, F6} & \texttt{RS4} &       \\ [.5em]
            \texttt{I5: LD F0, 16 (R1)}     & \hl{7} &   &   & \texttt{WAR I3, F0} & \hl{\texttt{RS1}} &       \\ [.5em]
            \texttt{I6: LD F2, 16 (R2)}     &   &   &   & \texttt{WAR I4, F2} &       &       \\ [.5em]
            \texttt{I7: ADDUI R1, R1, 8}    &   &   &   & \texttt{WAR I5, R1} &       &       \\ [.5em]
            \texttt{I8: ADDUI R2, R2, 8}    &   &   &   & \texttt{WAR I6, R2} &       &       \\ [.5em]
            \texttt{I9: ADD R3, R2, R1}     &   &   &   & \texttt{RAW R1, R2} &       &       \\ [.5em]
            \texttt{I10: BNE R3, R4, LOOP}   &   &   &   & \texttt{RAW I9, R3} &       &       \\
            \bottomrule
        \end{tabular}
        \end{adjustbox}
    \end{table}
    
    \begin{minipage}{0.45\textwidth}
        \centering
        \begin{tabular}{@{} l | l l l l @{}}
            \toprule
                & \texttt{Vj} & \texttt{Qj} & \texttt{Vk} & \texttt{Qk} \\
            \midrule
            \texttt{RS1} & \hl{\texttt{16}} & & \hl{\texttt{R1}} & \\ [.3em]
            \texttt{RS2} & \texttt{F6} & & \texttt{-} & \\
            \cmidrule{1-5}
            \texttt{LDU1} & & & & \\ [.3em]
            \texttt{LDU2} & \texttt{F6} & & \texttt{-} & \\
            \bottomrule
        \end{tabular}
    \end{minipage}
    \hfill
    \begin{minipage}{0.45\textwidth}
        \centering
        \begin{tabular}{@{} l | l l l l @{}}
            \toprule
            & \texttt{Vj} & \texttt{Qj} & \texttt{Vk} & \texttt{Qk} \\
            \midrule
            \texttt{RS3} & \texttt{F0} & & \texttt{F0} & \\ [.3em]
            \texttt{RS4} & \texttt{F2} & & \texttt{F2} & \\
            \cmidrule{1-5}
            \texttt{FPU1} & \texttt{F0} & & \texttt{F0} & \\
            \bottomrule
        \end{tabular}
    \end{minipage}

    \begin{minipage}{0.45\textwidth}
        \centering
        \begin{tabular}{@{} l | l l l l @{}}
            \toprule
            & \texttt{Vj} & \texttt{Qj} & \texttt{Vk} & \texttt{Qk} \\
            \midrule
            \texttt{RS5} & & & & \\ [.3em]
            \texttt{RS6} & & & & \\
            \cmidrule{1-5}
            \texttt{ALU1} & & & & \\ [.3em]
            \texttt{ALU2} & & & & \\
            \bottomrule
        \end{tabular}
    \end{minipage}
    \hfill
    \begin{minipage}{0.45\textwidth}
        \centering
        \begin{tabular}{@{} l c @{}}
            \toprule
            Unit            & Remaining cycles \\
            \midrule
            \texttt{LDU1}   & \\ [.3em]
            \texttt{LDU2}   & 0 \\ [.3em]
            \texttt{FPU1}   & 1 \\ [.3em]
            \texttt{ALU1}   & \\ [.3em]
            \texttt{ALU2}   & \\
            \bottomrule
        \end{tabular}
    \end{minipage}
    \newpage





    %%%%%%%%%%%%%%%%%%%%%%%%%%%%%%%%%%%%%%%%%%%%%%%%%%%%%%%%%%%%%%%%%%%%%%%%%%%%%%%%%%%%%%%%%%%%%%%%%%%%%%%%
    \item \textbf{Cycle \theenumi}:
    
    \begin{table}[!htp]
        \centering
        \begin{adjustbox}{width={\textwidth},totalheight={\textheight},keepaspectratio}
        \begin{tabular}{@{} l c c c c c c @{}}
            \toprule
            \textbf{Instruction} & \textbf{Issue} & \textbf{Start Exec} & \textbf{Write Res} & \textbf{Hazard} & \textbf{RSi} & \textbf{Unit} \\
            \midrule
            \texttt{I1: SD F4, 0 (R1)}      & 1 & 2 & 6 & None  & \texttt{RS1}   & \texttt{LDU1}  \\ [.5em]
            \texttt{I2: SD F6, 0 (R2)}      & 2 & 3 & 7 & None  & \texttt{RS2}   & \texttt{LDU2}  \\ [.5em]
            \texttt{I3: FADD F4, F0, F0}    & 3 & 4 & \hl{8} & \texttt{WAR I1, F4}  & \texttt{RS3} & \texttt{FPU1} \\ [.5em]
            \texttt{I4: FADD F6, F2, F2}    & 4 &   &   & \texttt{WAR I2, F6} & \texttt{RS4} &       \\ [.5em]
            \texttt{I5: LD F0, 16 (R1)}     & 7 & \hl{8} &   & \texttt{WAR I3, F0} & \texttt{RS1} & \hl{\texttt{LDU1}} \\ [.5em]
            \texttt{I6: LD F2, 16 (R2)}     & \hl{8} &   &   & \texttt{WAR I4, F2} & \hl{\texttt{RS2}} &       \\ [.5em]
            \texttt{I7: ADDUI R1, R1, 8}    &   &   &   & \texttt{WAR I5, R1} &       &       \\ [.5em]
            \texttt{I8: ADDUI R2, R2, 8}    &   &   &   & \texttt{WAR I6, R2} &       &       \\ [.5em]
            \texttt{I9: ADD R3, R2, R1}     &   &   &   & \texttt{RAW R1, R2} &       &       \\ [.5em]
            \texttt{I10: BNE R3, R4, LOOP}   &   &   &   & \texttt{RAW I9, R3} &       &       \\
            \bottomrule
        \end{tabular}
        \end{adjustbox}
    \end{table}
    
    \begin{minipage}{0.45\textwidth}
        \centering
        \begin{tabular}{@{} l | l l l l @{}}
            \toprule
                & \texttt{Vj} & \texttt{Qj} & \texttt{Vk} & \texttt{Qk} \\
            \midrule
            \texttt{RS1} & \texttt{16} & & \texttt{R1} & \\ [.3em]
            \texttt{RS2} & \hl{\texttt{16}} & & \hl{\texttt{R2}} & \\
            \cmidrule{1-5}
            \texttt{LDU1} & \hl{\texttt{16}} & & \hl{\texttt{R1}} & \\ [.3em]
            \texttt{LDU2} & & & & \\
            \bottomrule
        \end{tabular}
    \end{minipage}
    \hfill
    \begin{minipage}{0.45\textwidth}
        \centering
        \begin{tabular}{@{} l | l l l l @{}}
            \toprule
            & \texttt{Vj} & \texttt{Qj} & \texttt{Vk} & \texttt{Qk} \\
            \midrule
            \texttt{RS3} & \texttt{F0} & & \texttt{F0} & \\ [.3em]
            \texttt{RS4} & \texttt{F2} & & \texttt{F2} & \\
            \cmidrule{1-5}
            \texttt{FPU1} & \texttt{F0} & & \texttt{F0} & \\
            \bottomrule
        \end{tabular}
    \end{minipage}

    \begin{minipage}{0.45\textwidth}
        \centering
        \begin{tabular}{@{} l | l l l l @{}}
            \toprule
            & \texttt{Vj} & \texttt{Qj} & \texttt{Vk} & \texttt{Qk} \\
            \midrule
            \texttt{RS5} & & & & \\ [.3em]
            \texttt{RS6} & & & & \\
            \cmidrule{1-5}
            \texttt{ALU1} & & & & \\ [.3em]
            \texttt{ALU2} & & & & \\
            \bottomrule
        \end{tabular}
    \end{minipage}
    \hfill
    \begin{minipage}{0.45\textwidth}
        \centering
        \begin{tabular}{@{} l c @{}}
            \toprule
            Unit            & Remaining cycles \\
            \midrule
            \texttt{LDU1}   & 4 \\ [.3em]
            \texttt{LDU2}   & \\ [.3em]
            \texttt{FPU1}   & 0 \\ [.3em]
            \texttt{ALU1}   & \\ [.3em]
            \texttt{ALU2}   & \\
            \bottomrule
        \end{tabular}
    \end{minipage}
    \newpage





    %%%%%%%%%%%%%%%%%%%%%%%%%%%%%%%%%%%%%%%%%%%%%%%%%%%%%%%%%%%%%%%%%%%%%%%%%%%%%%%%%%%%%%%%%%%%%%%%%%%%%%%%
    \item \textbf{Cycle \theenumi}:
    
    \begin{table}[!htp]
        \centering
        \begin{adjustbox}{width={\textwidth},totalheight={\textheight},keepaspectratio}
        \begin{tabular}{@{} l c c c c c c @{}}
            \toprule
            \textbf{Instruction} & \textbf{Issue} & \textbf{Start Exec} & \textbf{Write Res} & \textbf{Hazard} & \textbf{RSi} & \textbf{Unit} \\
            \midrule
            \texttt{I1: SD F4, 0 (R1)}      & 1 & 2 & 6 & None  & \texttt{RS1}   & \texttt{LDU1}  \\ [.5em]
            \texttt{I2: SD F6, 0 (R2)}      & 2 & 3 & 7 & None  & \texttt{RS2}   & \texttt{LDU2}  \\ [.5em]
            \texttt{I3: FADD F4, F0, F0}    & 3 & 4 & 8 & \texttt{WAR I1, F4}  & \texttt{RS3} & \texttt{FPU1} \\ [.5em]
            \texttt{I4: FADD F6, F2, F2}    & 4 & \hl{9} &   & \texttt{WAR I2, F6} & \texttt{RS4} & \hl{\texttt{FPU1}} \\ [.5em]
            \texttt{I5: LD F0, 16 (R1)}     & 7 & 8 &   & \texttt{WAR I3, F0} & \texttt{RS1} & \texttt{LDU1} \\ [.5em]
            \texttt{I6: LD F2, 16 (R2)}     & 8 & \hl{9} &   & \texttt{WAR I4, F2} & \texttt{RS2} & \hl{\texttt{LDU2}} \\ [.5em]
            \texttt{I7: ADDUI R1, R1, 8}    & \hl{9} &   &   & \texttt{WAR I5, R1} & \hl{\texttt{RS5}} &       \\ [.5em]
            \texttt{I8: ADDUI R2, R2, 8}    &   &   &   & \texttt{WAR I6, R2} &       &       \\ [.5em]
            \texttt{I9: ADD R3, R2, R1}     &   &   &   & \texttt{RAW R1, R2} &       &       \\ [.5em]
            \texttt{I10: BNE R3, R4, LOOP}   &   &   &   & \texttt{RAW I9, R3} &       &       \\
            \bottomrule
        \end{tabular}
        \end{adjustbox}
    \end{table}
    
    \begin{minipage}{0.45\textwidth}
        \centering
        \begin{tabular}{@{} l | l l l l @{}}
            \toprule
                & \texttt{Vj} & \texttt{Qj} & \texttt{Vk} & \texttt{Qk} \\
            \midrule
            \texttt{RS1} & \texttt{16} & & \texttt{R1} & \\ [.3em]
            \texttt{RS2} & \texttt{16} & & \texttt{R2} & \\
            \cmidrule{1-5}
            \texttt{LDU1} & \texttt{16} & & \texttt{R1} & \\ [.3em]
            \texttt{LDU2} & \hl{\texttt{16}} & & \hl{\texttt{R2}} & \\
            \bottomrule
        \end{tabular}
    \end{minipage}
    \hfill
    \begin{minipage}{0.45\textwidth}
        \centering
        \begin{tabular}{@{} l | l l l l @{}}
            \toprule
            & \texttt{Vj} & \texttt{Qj} & \texttt{Vk} & \texttt{Qk} \\
            \midrule
            \texttt{RS3} & & & & \\ [.3em]
            \texttt{RS4} & \texttt{F2} & & \texttt{F2} & \\
            \cmidrule{1-5}
            \texttt{FPU1} & \hl{\texttt{F2}} & & \hl{\texttt{F2}} & \\
            \bottomrule
        \end{tabular}
    \end{minipage}

    \begin{minipage}{0.45\textwidth}
        \centering
        \begin{tabular}{@{} l | l l l l @{}}
            \toprule
            & \texttt{Vj} & \texttt{Qj} & \texttt{Vk} & \texttt{Qk} \\
            \midrule
            \texttt{RS5} & \hl{\texttt{R1}} & & \hl{\texttt{8}} & \\ [.3em]
            \texttt{RS6} & & & & \\
            \cmidrule{1-5}
            \texttt{ALU1} & & & & \\ [.3em]
            \texttt{ALU2} & & & & \\
            \bottomrule
        \end{tabular}
    \end{minipage}
    \hfill
    \begin{minipage}{0.45\textwidth}
        \centering
        \begin{tabular}{@{} l c @{}}
            \toprule
            Unit            & Remaining cycles \\
            \midrule
            \texttt{LDU1}   & 3 \\ [.3em]
            \texttt{LDU2}   & 4 \\ [.3em]
            \texttt{FPU1}   & 4 \\ [.3em]
            \texttt{ALU1}   & \\ [.3em]
            \texttt{ALU2}   & \\
            \bottomrule
        \end{tabular}
    \end{minipage}
    \newpage





    %%%%%%%%%%%%%%%%%%%%%%%%%%%%%%%%%%%%%%%%%%%%%%%%%%%%%%%%%%%%%%%%%%%%%%%%%%%%%%%%%%%%%%%%%%%%%%%%%%%%%%%%
    \item \textbf{Cycle \theenumi}:
    
    \begin{table}[!htp]
        \centering
        \begin{adjustbox}{width={\textwidth},totalheight={\textheight},keepaspectratio}
        \begin{tabular}{@{} l c c c c c c @{}}
            \toprule
            \textbf{Instruction} & \textbf{Issue} & \textbf{Start Exec} & \textbf{Write Res} & \textbf{Hazard} & \textbf{RSi} & \textbf{Unit} \\
            \midrule
            \texttt{I1: SD F4, 0 (R1)}      & 1 & 2 & 6 & None  & \texttt{RS1}   & \texttt{LDU1}  \\ [.5em]
            \texttt{I2: SD F6, 0 (R2)}      & 2 & 3 & 7 & None  & \texttt{RS2}   & \texttt{LDU2}  \\ [.5em]
            \texttt{I3: FADD F4, F0, F0}    & 3 & 4 & 8 & \texttt{WAR I1, F4}  & \texttt{RS3} & \texttt{FPU1} \\ [.5em]
            \texttt{I4: FADD F6, F2, F2}    & 4 & 9 &   & \texttt{WAR I2, F6} & \texttt{RS4} & \texttt{FPU1} \\ [.5em]
            \texttt{I5: LD F0, 16 (R1)}     & 7 & 8 &   & \texttt{WAR I3, F0} & \texttt{RS1} & \texttt{LDU1} \\ [.5em]
            \texttt{I6: LD F2, 16 (R2)}     & 8 & 9 &   & \texttt{WAR I4, F2} & \texttt{RS2} & \texttt{LDU2} \\ [.5em]
            \texttt{I7: ADDUI R1, R1, 8}    & 9 & \hl{10} &   & \texttt{WAR I5, R1} & \texttt{RS5} & \hl{\texttt{ALU1}} \\ [.5em]
            \texttt{I8: ADDUI R2, R2, 8}    & \hl{10} &   &   & \texttt{WAR I6, R2} & \hl{\texttt{RS6}} &       \\ [.5em]
            \texttt{I9: ADD R3, R2, R1}     &   &   &   & \texttt{RAW R1, R2} &       &       \\ [.5em]
            \texttt{I10: BNE R3, R4, LOOP}   &   &   &   & \texttt{RAW I9, R3} &       &       \\
            \bottomrule
        \end{tabular}
        \end{adjustbox}
    \end{table}
    
    \begin{minipage}{0.45\textwidth}
        \centering
        \begin{tabular}{@{} l | l l l l @{}}
            \toprule
                & \texttt{Vj} & \texttt{Qj} & \texttt{Vk} & \texttt{Qk} \\
            \midrule
            \texttt{RS1} & \texttt{16} & & \texttt{R1} & \\ [.3em]
            \texttt{RS2} & \texttt{16} & & \texttt{R2} & \\
            \cmidrule{1-5}
            \texttt{LDU1} & \texttt{16} & & \texttt{R1} & \\ [.3em]
            \texttt{LDU2} & \texttt{16} & & \texttt{R2} & \\
            \bottomrule
        \end{tabular}
    \end{minipage}
    \hfill
    \begin{minipage}{0.45\textwidth}
        \centering
        \begin{tabular}{@{} l | l l l l @{}}
            \toprule
            & \texttt{Vj} & \texttt{Qj} & \texttt{Vk} & \texttt{Qk} \\
            \midrule
            \texttt{RS3} & & & & \\ [.3em]
            \texttt{RS4} & \texttt{F2} & & \texttt{F2} & \\
            \cmidrule{1-5}
            \texttt{FPU1} & \texttt{F2} & & \texttt{F2} & \\
            \bottomrule
        \end{tabular}
    \end{minipage}

    \begin{minipage}{0.45\textwidth}
        \centering
        \begin{tabular}{@{} l | l l l l @{}}
            \toprule
            & \texttt{Vj} & \texttt{Qj} & \texttt{Vk} & \texttt{Qk} \\
            \midrule
            \texttt{RS5} & \texttt{R1} & & \texttt{8} & \\ [.3em]
            \texttt{RS6} & \hl{\texttt{R2}} & & \hl{\texttt{8}} & \\
            \cmidrule{1-5}
            \texttt{ALU1} & \hl{\texttt{R1}} & & \hl{\texttt{8}} & \\ [.3em]
            \texttt{ALU2} & & & & \\
            \bottomrule
        \end{tabular}
    \end{minipage}
    \hfill
    \begin{minipage}{0.45\textwidth}
        \centering
        \begin{tabular}{@{} l c @{}}
            \toprule
            Unit            & Remaining cycles \\
            \midrule
            \texttt{LDU1}   & 2 \\ [.3em]
            \texttt{LDU2}   & 3 \\ [.3em]
            \texttt{FPU1}   & 3 \\ [.3em]
            \texttt{ALU1}   & 2 \\ [.3em]
            \texttt{ALU2}   & \\
            \bottomrule
        \end{tabular}
    \end{minipage}
    \newpage





    %%%%%%%%%%%%%%%%%%%%%%%%%%%%%%%%%%%%%%%%%%%%%%%%%%%%%%%%%%%%%%%%%%%%%%%%%%%%%%%%%%%%%%%%%%%%%%%%%%%%%%%%
    \item \textbf{Cycle \theenumi}:
    
    \begin{table}[!htp]
        \centering
        \begin{adjustbox}{width={\textwidth},totalheight={\textheight},keepaspectratio}
        \begin{tabular}{@{} l c c c c c c @{}}
            \toprule
            \textbf{Instruction} & \textbf{Issue} & \textbf{Start Exec} & \textbf{Write Res} & \textbf{Hazard} & \textbf{RSi} & \textbf{Unit} \\
            \midrule
            \texttt{I1: SD F4, 0 (R1)}      & 1 & 2 & 6 & None  & \texttt{RS1}   & \texttt{LDU1}  \\ [.5em]
            \texttt{I2: SD F6, 0 (R2)}      & 2 & 3 & 7 & None  & \texttt{RS2}   & \texttt{LDU2}  \\ [.5em]
            \texttt{I3: FADD F4, F0, F0}    & 3 & 4 & 8 & \texttt{WAR I1, F4}  & \texttt{RS3} & \texttt{FPU1} \\ [.5em]
            \texttt{I4: FADD F6, F2, F2}    & 4 & 9 &   & \texttt{WAR I2, F6} & \texttt{RS4} & \texttt{FPU1} \\ [.5em]
            \texttt{I5: LD F0, 16 (R1)}     & 7 & 8 &   & \texttt{WAR I3, F0} & \texttt{RS1} & \texttt{LDU1} \\ [.5em]
            \texttt{I6: LD F2, 16 (R2)}     & 8 & 9 &   & \texttt{WAR I4, F2} & \texttt{RS2} & \texttt{LDU2} \\ [.5em]
            \texttt{I7: ADDUI R1, R1, 8}    & 9 & 10 &   & \texttt{WAR I5, R1} & \texttt{RS5} & \texttt{ALU1} \\ [.5em]
            \texttt{I8: ADDUI R2, R2, 8}    & 10 & \hl{11} &   & \texttt{WAR I6, R2} & \texttt{RS6} & \hl{\texttt{ALU2}} \\ [.5em]
            \texttt{I9: ADD R3, R2, R1}     &   &   &   & \texttt{RAW R1, R2} &       &       \\ [.5em]
            \texttt{I10: BNE R3, R4, LOOP}   &   &   &   & \texttt{RAW I9, R3} &       &       \\
            \bottomrule
        \end{tabular}
        \end{adjustbox}
    \end{table}
    
    \begin{minipage}{0.45\textwidth}
        \centering
        \begin{tabular}{@{} l | l l l l @{}}
            \toprule
                & \texttt{Vj} & \texttt{Qj} & \texttt{Vk} & \texttt{Qk} \\
            \midrule
            \texttt{RS1} & \texttt{16} & & \texttt{R1} & \\ [.3em]
            \texttt{RS2} & \texttt{16} & & \texttt{R2} & \\
            \cmidrule{1-5}
            \texttt{LDU1} & \texttt{16} & & \texttt{R1} & \\ [.3em]
            \texttt{LDU2} & \texttt{16} & & \texttt{R2} & \\
            \bottomrule
        \end{tabular}
    \end{minipage}
    \hfill
    \begin{minipage}{0.45\textwidth}
        \centering
        \begin{tabular}{@{} l | l l l l @{}}
            \toprule
            & \texttt{Vj} & \texttt{Qj} & \texttt{Vk} & \texttt{Qk} \\
            \midrule
            \texttt{RS3} & & & & \\ [.3em]
            \texttt{RS4} & \texttt{F2} & & \texttt{F2} & \\
            \cmidrule{1-5}
            \texttt{FPU1} & \texttt{F2} & & \texttt{F2} & \\
            \bottomrule
        \end{tabular}
    \end{minipage}

    \begin{minipage}{0.45\textwidth}
        \centering
        \begin{tabular}{@{} l | l l l l @{}}
            \toprule
            & \texttt{Vj} & \texttt{Qj} & \texttt{Vk} & \texttt{Qk} \\
            \midrule
            \texttt{RS5} & \texttt{R1} & & \texttt{8} & \\ [.3em]
            \texttt{RS6} & \texttt{R2} & & \texttt{8} & \\
            \cmidrule{1-5}
            \texttt{ALU1} & \texttt{R1} & & \texttt{8} & \\ [.3em]
            \texttt{ALU2} & \hl{\texttt{R2}} & & \hl{\texttt{8}} & \\
            \bottomrule
        \end{tabular}
    \end{minipage}
    \hfill
    \begin{minipage}{0.45\textwidth}
        \centering
        \begin{tabular}{@{} l c @{}}
            \toprule
            Unit            & Remaining cycles \\
            \midrule
            \texttt{LDU1}   & 1 \\ [.3em]
            \texttt{LDU2}   & 2 \\ [.3em]
            \texttt{FPU1}   & 2 \\ [.3em]
            \texttt{ALU1}   & 1 \\ [.3em]
            \texttt{ALU2}   & 2 \\
            \bottomrule
        \end{tabular}
    \end{minipage}
    \newpage





    %%%%%%%%%%%%%%%%%%%%%%%%%%%%%%%%%%%%%%%%%%%%%%%%%%%%%%%%%%%%%%%%%%%%%%%%%%%%%%%%%%%%%%%%%%%%%%%%%%%%%%%%
    \item \textbf{Cycle \theenumi}: ``\emph{The instruction 7 has finished executing, and it wants to write the result back. Why can't it do that?}'' Because instruction 5 has higher priority, being the first in chronological order, and the common data bus can host only one value at a time.
    
    \begin{table}[!htp]
        \centering
        \begin{adjustbox}{width={\textwidth},totalheight={\textheight},keepaspectratio}
        \begin{tabular}{@{} l c c c c c c @{}}
            \toprule
            \textbf{Instruction} & \textbf{Issue} & \textbf{Start Exec} & \textbf{Write Res} & \textbf{Hazard} & \textbf{RSi} & \textbf{Unit} \\
            \midrule
            \texttt{I1: SD F4, 0 (R1)}      & 1 & 2 & 6 & None  & \texttt{RS1}   & \texttt{LDU1}  \\ [.5em]
            \texttt{I2: SD F6, 0 (R2)}      & 2 & 3 & 7 & None  & \texttt{RS2}   & \texttt{LDU2}  \\ [.5em]
            \texttt{I3: FADD F4, F0, F0}    & 3 & 4 & 8 & \texttt{WAR I1, F4}  & \texttt{RS3} & \texttt{FPU1} \\ [.5em]
            \texttt{I4: FADD F6, F2, F2}    & 4 & 9 &   & \texttt{WAR I2, F6} & \texttt{RS4} & \texttt{FPU1} \\ [.5em]
            \texttt{I5: LD F0, 16 (R1)}     & 7 & 8 & \hl{12} & \texttt{WAR I3, F0} & \texttt{RS1} & \texttt{LDU1} \\ [.5em]
            \texttt{I6: LD F2, 16 (R2)}     & 8 & 9 &   & \texttt{WAR I4, F2} & \texttt{RS2} & \texttt{LDU2} \\ [.5em]
            \texttt{I7: ADDUI R1, R1, 8}    & 9 & 10 &   & \texttt{WAR I5, R1} & \texttt{RS5} & \texttt{ALU1} \\ [.5em]
            \texttt{I8: ADDUI R2, R2, 8}    & 10 & 11 &   & \texttt{WAR I6, R2} & \texttt{RS6} & \texttt{ALU2} \\ [.5em]
            \texttt{I9: ADD R3, R2, R1}     &   &   &   & \texttt{RAW R1, R2} &       &       \\ [.5em]
            \texttt{I10: BNE R3, R4, LOOP}   &   &   &   & \texttt{RAW I9, R3} &       &       \\
            \bottomrule
        \end{tabular}
        \end{adjustbox}
    \end{table}
    
    \begin{minipage}{0.45\textwidth}
        \centering
        \begin{tabular}{@{} l | l l l l @{}}
            \toprule
                & \texttt{Vj} & \texttt{Qj} & \texttt{Vk} & \texttt{Qk} \\
            \midrule
            \texttt{RS1} & \texttt{16} & & \texttt{R1} & \\ [.3em]
            \texttt{RS2} & \texttt{16} & & \texttt{R2} & \\
            \cmidrule{1-5}
            \texttt{LDU1} & \texttt{16} & & \texttt{R1} & \\ [.3em]
            \texttt{LDU2} & \texttt{16} & & \texttt{R2} & \\
            \bottomrule
        \end{tabular}
    \end{minipage}
    \hfill
    \begin{minipage}{0.45\textwidth}
        \centering
        \begin{tabular}{@{} l | l l l l @{}}
            \toprule
            & \texttt{Vj} & \texttt{Qj} & \texttt{Vk} & \texttt{Qk} \\
            \midrule
            \texttt{RS3} & & & & \\ [.3em]
            \texttt{RS4} & \texttt{F2} & & \texttt{F2} & \\
            \cmidrule{1-5}
            \texttt{FPU1} & \texttt{F2} & & \texttt{F2} & \\
            \bottomrule
        \end{tabular}
    \end{minipage}

    \begin{minipage}{0.45\textwidth}
        \centering
        \begin{tabular}{@{} l | l l l l @{}}
            \toprule
            & \texttt{Vj} & \texttt{Qj} & \texttt{Vk} & \texttt{Qk} \\
            \midrule
            \texttt{RS5} & \texttt{R1} & & \texttt{8} & \\ [.3em]
            \texttt{RS6} & \texttt{R2} & & \texttt{8} & \\
            \cmidrule{1-5}
            \texttt{ALU1} & \texttt{R1} & & \texttt{8} & \\ [.3em]
            \texttt{ALU2} & \texttt{R2} & & \texttt{8} & \\
            \bottomrule
        \end{tabular}
    \end{minipage}
    \hfill
    \begin{minipage}{0.45\textwidth}
        \centering
        \begin{tabular}{@{} l c @{}}
            \toprule
            Unit            & Remaining cycles \\
            \midrule
            \texttt{LDU1}   & 0 \\ [.3em]
            \texttt{LDU2}   & 1 \\ [.3em]
            \texttt{FPU1}   & 1 \\ [.3em]
            \texttt{ALU1}   & 0 \\ [.3em]
            \texttt{ALU2}   & 1 \\
            \bottomrule
        \end{tabular}
    \end{minipage}
    \newpage





    %%%%%%%%%%%%%%%%%%%%%%%%%%%%%%%%%%%%%%%%%%%%%%%%%%%%%%%%%%%%%%%%%%%%%%%%%%%%%%%%%%%%%%%%%%%%%%%%%%%%%%%%
    \item \textbf{Cycle \theenumi}: Each functional unit has finished executing and is ready to write back the result. Again, since there is only one common data bus, we issue each instruction in chronological order. In this case, it is instruction 4.
    
    \begin{table}[!htp]
        \centering
        \begin{adjustbox}{width={\textwidth},totalheight={\textheight},keepaspectratio}
        \begin{tabular}{@{} l c c c c c c @{}}
            \toprule
            \textbf{Instruction} & \textbf{Issue} & \textbf{Start Exec} & \textbf{Write Res} & \textbf{Hazard} & \textbf{RSi} & \textbf{Unit} \\
            \midrule
            \texttt{I1: SD F4, 0 (R1)}      & 1 & 2 & 6 & None  & \texttt{RS1}   & \texttt{LDU1}  \\ [.5em]
            \texttt{I2: SD F6, 0 (R2)}      & 2 & 3 & 7 & None  & \texttt{RS2}   & \texttt{LDU2}  \\ [.5em]
            \texttt{I3: FADD F4, F0, F0}    & 3 & 4 & 8 & \texttt{WAR I1, F4}  & \texttt{RS3} & \texttt{FPU1} \\ [.5em]
            \texttt{I4: FADD F6, F2, F2}    & 4 & 9 & \hl{13} & \texttt{WAR I2, F6} & \texttt{RS4} & \texttt{FPU1} \\ [.5em]
            \texttt{I5: LD F0, 16 (R1)}     & 7 & 8 & 12 & \texttt{WAR I3, F0} & \texttt{RS1} & \texttt{LDU1} \\ [.5em]
            \texttt{I6: LD F2, 16 (R2)}     & 8 & 9 &   & \texttt{WAR I4, F2} & \texttt{RS2} & \texttt{LDU2} \\ [.5em]
            \texttt{I7: ADDUI R1, R1, 8}    & 9 & 10 &   & \texttt{WAR I5, R1} & \texttt{RS5} & \texttt{ALU1} \\ [.5em]
            \texttt{I8: ADDUI R2, R2, 8}    & 10 & 11 &   & \texttt{WAR I6, R2} & \texttt{RS6} & \texttt{ALU2} \\ [.5em]
            \texttt{I9: ADD R3, R2, R1}     &   &   &   & \texttt{RAW R1, R2} &       &       \\ [.5em]
            \texttt{I10: BNE R3, R4, LOOP}   &   &   &   & \texttt{RAW I9, R3} &       &       \\
            \bottomrule
        \end{tabular}
        \end{adjustbox}
    \end{table}
    
    \begin{minipage}{0.45\textwidth}
        \centering
        \begin{tabular}{@{} l | l l l l @{}}
            \toprule
                & \texttt{Vj} & \texttt{Qj} & \texttt{Vk} & \texttt{Qk} \\
            \midrule
            \texttt{RS1} & & & & \\ [.3em]
            \texttt{RS2} & \texttt{16} & & \texttt{R2} & \\
            \cmidrule{1-5}
            \texttt{LDU1} & & & & \\ [.3em]
            \texttt{LDU2} & \texttt{16} & & \texttt{R2} & \\
            \bottomrule
        \end{tabular}
    \end{minipage}
    \hfill
    \begin{minipage}{0.45\textwidth}
        \centering
        \begin{tabular}{@{} l | l l l l @{}}
            \toprule
            & \texttt{Vj} & \texttt{Qj} & \texttt{Vk} & \texttt{Qk} \\
            \midrule
            \texttt{RS3} & & & & \\ [.3em]
            \texttt{RS4} & \texttt{F2} & & \texttt{F2} & \\
            \cmidrule{1-5}
            \texttt{FPU1} & \texttt{F2} & & \texttt{F2} & \\
            \bottomrule
        \end{tabular}
    \end{minipage}

    \begin{minipage}{0.45\textwidth}
        \centering
        \begin{tabular}{@{} l | l l l l @{}}
            \toprule
            & \texttt{Vj} & \texttt{Qj} & \texttt{Vk} & \texttt{Qk} \\
            \midrule
            \texttt{RS5} & \texttt{R1} & & \texttt{8} & \\ [.3em]
            \texttt{RS6} & \texttt{R2} & & \texttt{8} & \\
            \cmidrule{1-5}
            \texttt{ALU1} & \texttt{R1} & & \texttt{8} & \\ [.3em]
            \texttt{ALU2} & \texttt{R2} & & \texttt{8} & \\
            \bottomrule
        \end{tabular}
    \end{minipage}
    \hfill
    \begin{minipage}{0.45\textwidth}
        \centering
        \begin{tabular}{@{} l c @{}}
            \toprule
            Unit            & Remaining cycles \\
            \midrule
            \texttt{LDU1}   & \\ [.3em]
            \texttt{LDU2}   & 0 \\ [.3em]
            \texttt{FPU1}   & 0 \\ [.3em]
            \texttt{ALU1}   & 0 ($-1$) \\ [.3em]
            \texttt{ALU2}   & 0 \\
            \bottomrule
        \end{tabular}
    \end{minipage}
    \newpage





    %%%%%%%%%%%%%%%%%%%%%%%%%%%%%%%%%%%%%%%%%%%%%%%%%%%%%%%%%%%%%%%%%%%%%%%%%%%%%%%%%%%%%%%%%%%%%%%%%%%%%%%%
    \item \textbf{Cycle \theenumi}:
    
    \begin{table}[!htp]
        \centering
        \begin{adjustbox}{width={\textwidth},totalheight={\textheight},keepaspectratio}
        \begin{tabular}{@{} l c c c c c c @{}}
            \toprule
            \textbf{Instruction} & \textbf{Issue} & \textbf{Start Exec} & \textbf{Write Res} & \textbf{Hazard} & \textbf{RSi} & \textbf{Unit} \\
            \midrule
            \texttt{I1: SD F4, 0 (R1)}      & 1 & 2 & 6 & None  & \texttt{RS1}   & \texttt{LDU1}  \\ [.5em]
            \texttt{I2: SD F6, 0 (R2)}      & 2 & 3 & 7 & None  & \texttt{RS2}   & \texttt{LDU2}  \\ [.5em]
            \texttt{I3: FADD F4, F0, F0}    & 3 & 4 & 8 & \texttt{WAR I1, F4}  & \texttt{RS3} & \texttt{FPU1} \\ [.5em]
            \texttt{I4: FADD F6, F2, F2}    & 4 & 9 & 13 & \texttt{WAR I2, F6} & \texttt{RS4} & \texttt{FPU1} \\ [.5em]
            \texttt{I5: LD F0, 16 (R1)}     & 7 & 8 & 12 & \texttt{WAR I3, F0} & \texttt{RS1} & \texttt{LDU1} \\ [.5em]
            \texttt{I6: LD F2, 16 (R2)}     & 8 & 9 & \hl{14} & \texttt{WAR I4, F2} & \texttt{RS2} & \texttt{LDU2} \\ [.5em]
            \texttt{I7: ADDUI R1, R1, 8}    & 9 & 10 &   & \texttt{WAR I5, R1} & \texttt{RS5} & \texttt{ALU1} \\ [.5em]
            \texttt{I8: ADDUI R2, R2, 8}    & 10 & 11 &   & \texttt{WAR I6, R2} & \texttt{RS6} & \texttt{ALU2} \\ [.5em]
            \texttt{I9: ADD R3, R2, R1}     &   &   &   & \texttt{RAW R1, R2} &       &       \\ [.5em]
            \texttt{I10: BNE R3, R4, LOOP}   &   &   &   & \texttt{RAW I9, R3} &       &       \\
            \bottomrule
        \end{tabular}
        \end{adjustbox}
    \end{table}
    
    \begin{minipage}{0.45\textwidth}
        \centering
        \begin{tabular}{@{} l | l l l l @{}}
            \toprule
                & \texttt{Vj} & \texttt{Qj} & \texttt{Vk} & \texttt{Qk} \\
            \midrule
            \texttt{RS1} & & & & \\ [.3em]
            \texttt{RS2} & \texttt{16} & & \texttt{R2} & \\
            \cmidrule{1-5}
            \texttt{LDU1} & & & & \\ [.3em]
            \texttt{LDU2} & \texttt{16} & & \texttt{R2} & \\
            \bottomrule
        \end{tabular}
    \end{minipage}
    \hfill
    \begin{minipage}{0.45\textwidth}
        \centering
        \begin{tabular}{@{} l | l l l l @{}}
            \toprule
            & \texttt{Vj} & \texttt{Qj} & \texttt{Vk} & \texttt{Qk} \\
            \midrule
            \texttt{RS3} & & & & \\ [.3em]
            \texttt{RS4} & & & & \\
            \cmidrule{1-5}
            \texttt{FPU1} & & & & \\
            \bottomrule
        \end{tabular}
    \end{minipage}

    \begin{minipage}{0.45\textwidth}
        \centering
        \begin{tabular}{@{} l | l l l l @{}}
            \toprule
            & \texttt{Vj} & \texttt{Qj} & \texttt{Vk} & \texttt{Qk} \\
            \midrule
            \texttt{RS5} & \texttt{R1} & & \texttt{8} & \\ [.3em]
            \texttt{RS6} & \texttt{R2} & & \texttt{8} & \\
            \cmidrule{1-5}
            \texttt{ALU1} & \texttt{R1} & & \texttt{8} & \\ [.3em]
            \texttt{ALU2} & \texttt{R2} & & \texttt{8} & \\
            \bottomrule
        \end{tabular}
    \end{minipage}
    \hfill
    \begin{minipage}{0.45\textwidth}
        \centering
        \begin{tabular}{@{} l c @{}}
            \toprule
            Unit            & Remaining cycles \\
            \midrule
            \texttt{LDU1}   & \\ [.3em]
            \texttt{LDU2}   & 0 ($-1$) \\ [.3em]
            \texttt{FPU1}   & \\ [.3em]
            \texttt{ALU1}   & 0 ($-2$) \\ [.3em]
            \texttt{ALU2}   & 0 ($-1$) \\
            \bottomrule
        \end{tabular}
    \end{minipage}
    \newpage





    %%%%%%%%%%%%%%%%%%%%%%%%%%%%%%%%%%%%%%%%%%%%%%%%%%%%%%%%%%%%%%%%%%%%%%%%%%%%%%%%%%%%%%%%%%%%%%%%%%%%%%%%
    \item \textbf{Cycle \theenumi}:
    
    \begin{table}[!htp]
        \centering
        \begin{adjustbox}{width={\textwidth},totalheight={\textheight},keepaspectratio}
        \begin{tabular}{@{} l c c c c c c @{}}
            \toprule
            \textbf{Instruction} & \textbf{Issue} & \textbf{Start Exec} & \textbf{Write Res} & \textbf{Hazard} & \textbf{RSi} & \textbf{Unit} \\
            \midrule
            \texttt{I1: SD F4, 0 (R1)}      & 1 & 2 & 6 & None  & \texttt{RS1}   & \texttt{LDU1}  \\ [.5em]
            \texttt{I2: SD F6, 0 (R2)}      & 2 & 3 & 7 & None  & \texttt{RS2}   & \texttt{LDU2}  \\ [.5em]
            \texttt{I3: FADD F4, F0, F0}    & 3 & 4 & 8 & \texttt{WAR I1, F4}  & \texttt{RS3} & \texttt{FPU1} \\ [.5em]
            \texttt{I4: FADD F6, F2, F2}    & 4 & 9 & 13 & \texttt{WAR I2, F6} & \texttt{RS4} & \texttt{FPU1} \\ [.5em]
            \texttt{I5: LD F0, 16 (R1)}     & 7 & 8 & 12 & \texttt{WAR I3, F0} & \texttt{RS1} & \texttt{LDU1} \\ [.5em]
            \texttt{I6: LD F2, 16 (R2)}     & 8 & 9 & 14 & \texttt{WAR I4, F2} & \texttt{RS2} & \texttt{LDU2} \\ [.5em]
            \texttt{I7: ADDUI R1, R1, 8}    & 9 & 10 & \hl{15} & \texttt{WAR I5, R1} & \texttt{RS5} & \texttt{ALU1} \\ [.5em]
            \texttt{I8: ADDUI R2, R2, 8}    & 10 & 11 &   & \texttt{WAR I6, R2} & \texttt{RS6} & \texttt{ALU2} \\ [.5em]
            \texttt{I9: ADD R3, R2, R1}     &   &   &   & \texttt{RAW R1, R2} &       &       \\ [.5em]
            \texttt{I10: BNE R3, R4, LOOP}   &   &   &   & \texttt{RAW I9, R3} &       &       \\
            \bottomrule
        \end{tabular}
        \end{adjustbox}
    \end{table}
    
    \begin{minipage}{0.45\textwidth}
        \centering
        \begin{tabular}{@{} l | l l l l @{}}
            \toprule
                & \texttt{Vj} & \texttt{Qj} & \texttt{Vk} & \texttt{Qk} \\
            \midrule
            \texttt{RS1} & & & & \\ [.3em]
            \texttt{RS2} & & & & \\
            \cmidrule{1-5}
            \texttt{LDU1} & & & & \\ [.3em]
            \texttt{LDU2} & & & & \\
            \bottomrule
        \end{tabular}
    \end{minipage}
    \hfill
    \begin{minipage}{0.45\textwidth}
        \centering
        \begin{tabular}{@{} l | l l l l @{}}
            \toprule
            & \texttt{Vj} & \texttt{Qj} & \texttt{Vk} & \texttt{Qk} \\
            \midrule
            \texttt{RS3} & & & & \\ [.3em]
            \texttt{RS4} & & & & \\
            \cmidrule{1-5}
            \texttt{FPU1} & & & & \\
            \bottomrule
        \end{tabular}
    \end{minipage}

    \begin{minipage}{0.45\textwidth}
        \centering
        \begin{tabular}{@{} l | l l l l @{}}
            \toprule
            & \texttt{Vj} & \texttt{Qj} & \texttt{Vk} & \texttt{Qk} \\
            \midrule
            \texttt{RS5} & \texttt{R1} & & \texttt{8} & \\ [.3em]
            \texttt{RS6} & \texttt{R2} & & \texttt{8} & \\
            \cmidrule{1-5}
            \texttt{ALU1} & \texttt{R1} & & \texttt{8} & \\ [.3em]
            \texttt{ALU2} & \texttt{R2} & & \texttt{8} & \\
            \bottomrule
        \end{tabular}
    \end{minipage}
    \hfill
    \begin{minipage}{0.45\textwidth}
        \centering
        \begin{tabular}{@{} l c @{}}
            \toprule
            Unit            & Remaining cycles \\
            \midrule
            \texttt{LDU1}   & \\ [.3em]
            \texttt{LDU2}   & \\ [.3em]
            \texttt{FPU1}   & \\ [.3em]
            \texttt{ALU1}   & 0 ($-3$) \\ [.3em]
            \texttt{ALU2}   & 0 ($-2$) \\
            \bottomrule
        \end{tabular}
    \end{minipage}
    \newpage





    %%%%%%%%%%%%%%%%%%%%%%%%%%%%%%%%%%%%%%%%%%%%%%%%%%%%%%%%%%%%%%%%%%%%%%%%%%%%%%%%%%%%%%%%%%%%%%%%%%%%%%%%
    \item \textbf{Cycle \theenumi}:
    
    \begin{table}[!htp]
        \centering
        \begin{adjustbox}{width={\textwidth},totalheight={\textheight},keepaspectratio}
        \begin{tabular}{@{} l c c c c c c @{}}
            \toprule
            \textbf{Instruction} & \textbf{Issue} & \textbf{Start Exec} & \textbf{Write Res} & \textbf{Hazard} & \textbf{RSi} & \textbf{Unit} \\
            \midrule
            \texttt{I1: SD F4, 0 (R1)}      & 1 & 2 & 6 & None  & \texttt{RS1}   & \texttt{LDU1}  \\ [.5em]
            \texttt{I2: SD F6, 0 (R2)}      & 2 & 3 & 7 & None  & \texttt{RS2}   & \texttt{LDU2}  \\ [.5em]
            \texttt{I3: FADD F4, F0, F0}    & 3 & 4 & 8 & \texttt{WAR I1, F4}  & \texttt{RS3} & \texttt{FPU1} \\ [.5em]
            \texttt{I4: FADD F6, F2, F2}    & 4 & 9 & 13 & \texttt{WAR I2, F6} & \texttt{RS4} & \texttt{FPU1} \\ [.5em]
            \texttt{I5: LD F0, 16 (R1)}     & 7 & 8 & 12 & \texttt{WAR I3, F0} & \texttt{RS1} & \texttt{LDU1} \\ [.5em]
            \texttt{I6: LD F2, 16 (R2)}     & 8 & 9 & 14 & \texttt{WAR I4, F2} & \texttt{RS2} & \texttt{LDU2} \\ [.5em]
            \texttt{I7: ADDUI R1, R1, 8}    & 9 & 10 & 15 & \texttt{WAR I5, R1} & \texttt{RS5} & \texttt{ALU1} \\ [.5em]
            \texttt{I8: ADDUI R2, R2, 8}    & 10 & 11 & \hl{16} & \texttt{WAR I6, R2} & \texttt{RS6} & \texttt{ALU2} \\ [.5em]
            \texttt{I9: ADD R3, R2, R1}     & \hl{16} &   &   & \texttt{RAW R1, R2} & \hl{\texttt{RS5}} &       \\ [.5em]
            \texttt{I10: BNE R3, R4, LOOP}   &   &   &   & \texttt{RAW I9, R3} &       &       \\
            \bottomrule
        \end{tabular}
        \end{adjustbox}
    \end{table}
    
    \begin{minipage}{0.45\textwidth}
        \centering
        \begin{tabular}{@{} l | l l l l @{}}
            \toprule
                & \texttt{Vj} & \texttt{Qj} & \texttt{Vk} & \texttt{Qk} \\
            \midrule
            \texttt{RS1} & & & & \\ [.3em]
            \texttt{RS2} & & & & \\
            \cmidrule{1-5}
            \texttt{LDU1} & & & & \\ [.3em]
            \texttt{LDU2} & & & & \\
            \bottomrule
        \end{tabular}
    \end{minipage}
    \hfill
    \begin{minipage}{0.45\textwidth}
        \centering
        \begin{tabular}{@{} l | l l l l @{}}
            \toprule
            & \texttt{Vj} & \texttt{Qj} & \texttt{Vk} & \texttt{Qk} \\
            \midrule
            \texttt{RS3} & & & & \\ [.3em]
            \texttt{RS4} & & & & \\
            \cmidrule{1-5}
            \texttt{FPU1} & & & & \\
            \bottomrule
        \end{tabular}
    \end{minipage}

    \begin{minipage}{0.45\textwidth}
        \centering
        \begin{tabular}{@{} l | l l l l @{}}
            \toprule
            & \texttt{Vj} & \texttt{Qj} & \texttt{Vk} & \texttt{Qk} \\
            \midrule
            \texttt{RS5} & \hl{\texttt{R2}} & & \hl{\texttt{R1}} & \\ [.3em]
            \texttt{RS6} & \texttt{R2} & & \texttt{8} & \\
            \cmidrule{1-5}
            \texttt{ALU1} & & & & \\ [.3em]
            \texttt{ALU2} & \texttt{R2} & & \texttt{8} & \\
            \bottomrule
        \end{tabular}
    \end{minipage}
    \hfill
    \begin{minipage}{0.45\textwidth}
        \centering
        \begin{tabular}{@{} l c @{}}
            \toprule
            Unit            & Remaining cycles \\
            \midrule
            \texttt{LDU1}   & \\ [.3em]
            \texttt{LDU2}   & \\ [.3em]
            \texttt{FPU1}   & \\ [.3em]
            \texttt{ALU1}   & \\ [.3em]
            \texttt{ALU2}   & 0 ($-3$) \\
            \bottomrule
        \end{tabular}
    \end{minipage}
    \newpage
    




    %%%%%%%%%%%%%%%%%%%%%%%%%%%%%%%%%%%%%%%%%%%%%%%%%%%%%%%%%%%%%%%%%%%%%%%%%%%%%%%%%%%%%%%%%%%%%%%%%%%%%%%%
    \item \textbf{Cycle \theenumi}: This is the power of Tomasulo and its tag. This algorithm doesn't block the tenth instruction because it lacks the operand \texttt{R3}. Instead, it says, ``\emph{You cannot retrieve the value of register \texttt{R3}. No problem. Meanwhile, you can issue the instruction. When the \texttt{R3} value is ready, it will be sent to the Common Data Bus.}''. Thus, instruction 10 waits on the Common Data Bus for the value of register \texttt{R3}. ``\emph{But how does it know when the value is ready?}'' Simple. Use the tag logic. The value will be sent on the Common Data Bus with the key \texttt{ALU1}. Instruction 10 writes the tag ``\texttt{ALU1}'' to the reservation station, and when the Common Data Bus sends that tag, it takes the value.
    
    \begin{table}[!htp]
        \centering
        \begin{adjustbox}{width={\textwidth},totalheight={\textheight},keepaspectratio}
        \begin{tabular}{@{} l c c c c c c @{}}
            \toprule
            \textbf{Instruction} & \textbf{Issue} & \textbf{Start Exec} & \textbf{Write Res} & \textbf{Hazard} & \textbf{RSi} & \textbf{Unit} \\
            \midrule
            \texttt{I1: SD F4, 0 (R1)}      & 1 & 2 & 6 & None  & \texttt{RS1}   & \texttt{LDU1}  \\ [.5em]
            \texttt{I2: SD F6, 0 (R2)}      & 2 & 3 & 7 & None  & \texttt{RS2}   & \texttt{LDU2}  \\ [.5em]
            \texttt{I3: FADD F4, F0, F0}    & 3 & 4 & 8 & \texttt{WAR I1, F4}  & \texttt{RS3} & \texttt{FPU1} \\ [.5em]
            \texttt{I4: FADD F6, F2, F2}    & 4 & 9 & 13 & \texttt{WAR I2, F6} & \texttt{RS4} & \texttt{FPU1} \\ [.5em]
            \texttt{I5: LD F0, 16 (R1)}     & 7 & 8 & 12 & \texttt{WAR I3, F0} & \texttt{RS1} & \texttt{LDU1} \\ [.5em]
            \texttt{I6: LD F2, 16 (R2)}     & 8 & 9 & 14 & \texttt{WAR I4, F2} & \texttt{RS2} & \texttt{LDU2} \\ [.5em]
            \texttt{I7: ADDUI R1, R1, 8}    & 9 & 10 & 15 & \texttt{WAR I5, R1} & \texttt{RS5} & \texttt{ALU1} \\ [.5em]
            \texttt{I8: ADDUI R2, R2, 8}    & 10 & 11 & 16 & \texttt{WAR I6, R2} & \texttt{RS6} & \texttt{ALU2} \\ [.5em]
            \texttt{I9: ADD R3, R2, R1}     & 16 & \hl{17} &   & \texttt{RAW R1, R2} & \texttt{RS5} & \hl{\texttt{ALU1}} \\ [.5em]
            \texttt{I10: BNE R3, R4, LOOP}   & \hl{17} &   &   & \texttt{RAW I9, R3} & \hl{\texttt{RS6}} &       \\
            \bottomrule
        \end{tabular}
        \end{adjustbox}
    \end{table}
    
    \begin{minipage}{0.45\textwidth}
        \centering
        \begin{tabular}{@{} l | l l l l @{}}
            \toprule
                & \texttt{Vj} & \texttt{Qj} & \texttt{Vk} & \texttt{Qk} \\
            \midrule
            \texttt{RS1} & & & & \\ [.3em]
            \texttt{RS2} & & & & \\
            \cmidrule{1-5}
            \texttt{LDU1} & & & & \\ [.3em]
            \texttt{LDU2} & & & & \\
            \bottomrule
        \end{tabular}
    \end{minipage}
    \hfill
    \begin{minipage}{0.45\textwidth}
        \centering
        \begin{tabular}{@{} l | l l l l @{}}
            \toprule
            & \texttt{Vj} & \texttt{Qj} & \texttt{Vk} & \texttt{Qk} \\
            \midrule
            \texttt{RS3} & & & & \\ [.3em]
            \texttt{RS4} & & & & \\
            \cmidrule{1-5}
            \texttt{FPU1} & & & & \\
            \bottomrule
        \end{tabular}
    \end{minipage}

    \begin{minipage}{0.45\textwidth}
        \centering
        \begin{tabular}{@{} l | l l l l @{}}
            \toprule
            & \texttt{Vj} & \texttt{Qj} & \texttt{Vk} & \texttt{Qk} \\
            \midrule
            \texttt{RS5} & \texttt{R2} & & \texttt{R1} & \\ [.3em]
            \texttt{RS6} & & \hl{\texttt{ALU1}} & \hl{\texttt{R4}} & \\
            \cmidrule{1-5}
            \texttt{ALU1} & \hl{\texttt{R2}} & & \hl{\texttt{R1}} & \\ [.3em]
            \texttt{ALU2} & & & & \\
            \bottomrule
        \end{tabular}
    \end{minipage}
    \hfill
    \begin{minipage}{0.45\textwidth}
        \centering
        \begin{tabular}{@{} l c @{}}
            \toprule
            Unit            & Remaining cycles \\
            \midrule
            \texttt{LDU1}   & \\ [.3em]
            \texttt{LDU2}   & \\ [.3em]
            \texttt{FPU1}   & \\ [.3em]
            \texttt{ALU1}   & 2 \\ [.3em]
            \texttt{ALU2}   & \\
            \bottomrule
        \end{tabular}
    \end{minipage}
    \newpage





    %%%%%%%%%%%%%%%%%%%%%%%%%%%%%%%%%%%%%%%%%%%%%%%%%%%%%%%%%%%%%%%%%%%%%%%%%%%%%%%%%%%%%%%%%%%%%%%%%%%%%%%%
    \setcounter{enumi}{18}
    \item \textbf{Cycle \theenumi}: We skip cycle 18 because no operations have been performed. In instruction 10, the reservation station reads the value from the common data bus and stores it in memory. It is now ready to be executed!
    
    \begin{table}[!htp]
        \centering
        \begin{adjustbox}{width={\textwidth},totalheight={\textheight},keepaspectratio}
        \begin{tabular}{@{} l c c c c c c @{}}
            \toprule
            \textbf{Instruction} & \textbf{Issue} & \textbf{Start Exec} & \textbf{Write Res} & \textbf{Hazard} & \textbf{RSi} & \textbf{Unit} \\
            \midrule
            \texttt{I1: SD F4, 0 (R1)}      & 1 & 2 & 6 & None  & \texttt{RS1}   & \texttt{LDU1}  \\ [.5em]
            \texttt{I2: SD F6, 0 (R2)}      & 2 & 3 & 7 & None  & \texttt{RS2}   & \texttt{LDU2}  \\ [.5em]
            \texttt{I3: FADD F4, F0, F0}    & 3 & 4 & 8 & \texttt{WAR I1, F4}  & \texttt{RS3} & \texttt{FPU1} \\ [.5em]
            \texttt{I4: FADD F6, F2, F2}    & 4 & 9 & 13 & \texttt{WAR I2, F6} & \texttt{RS4} & \texttt{FPU1} \\ [.5em]
            \texttt{I5: LD F0, 16 (R1)}     & 7 & 8 & 12 & \texttt{WAR I3, F0} & \texttt{RS1} & \texttt{LDU1} \\ [.5em]
            \texttt{I6: LD F2, 16 (R2)}     & 8 & 9 & 14 & \texttt{WAR I4, F2} & \texttt{RS2} & \texttt{LDU2} \\ [.5em]
            \texttt{I7: ADDUI R1, R1, 8}    & 9 & 10 & 15 & \texttt{WAR I5, R1} & \texttt{RS5} & \texttt{ALU1} \\ [.5em]
            \texttt{I8: ADDUI R2, R2, 8}    & 10 & 11 & 16 & \texttt{WAR I6, R2} & \texttt{RS6} & \texttt{ALU2} \\ [.5em]
            \texttt{I9: ADD R3, R2, R1}     & 16 & 17 & \hl{19} & \texttt{RAW R1, R2} & \texttt{RS5} & \texttt{ALU1} \\ [.5em]
            \texttt{I10: BNE R3, R4, LOOP}   & 17 &   &   & \texttt{RAW I9, R3} & \texttt{RS6} &       \\
            \bottomrule
        \end{tabular}
        \end{adjustbox}
    \end{table}
    
    \begin{minipage}{0.45\textwidth}
        \centering
        \begin{tabular}{@{} l | l l l l @{}}
            \toprule
                & \texttt{Vj} & \texttt{Qj} & \texttt{Vk} & \texttt{Qk} \\
            \midrule
            \texttt{RS1} & & & & \\ [.3em]
            \texttt{RS2} & & & & \\
            \cmidrule{1-5}
            \texttt{LDU1} & & & & \\ [.3em]
            \texttt{LDU2} & & & & \\
            \bottomrule
        \end{tabular}
    \end{minipage}
    \hfill
    \begin{minipage}{0.45\textwidth}
        \centering
        \begin{tabular}{@{} l | l l l l @{}}
            \toprule
            & \texttt{Vj} & \texttt{Qj} & \texttt{Vk} & \texttt{Qk} \\
            \midrule
            \texttt{RS3} & & & & \\ [.3em]
            \texttt{RS4} & & & & \\
            \cmidrule{1-5}
            \texttt{FPU1} & & & & \\
            \bottomrule
        \end{tabular}
    \end{minipage}

    \begin{minipage}{0.45\textwidth}
        \centering
        \begin{tabular}{@{} l | l l l l @{}}
            \toprule
            & \texttt{Vj} & \texttt{Qj} & \texttt{Vk} & \texttt{Qk} \\
            \midrule
            \texttt{RS5} & \texttt{R2} & & \texttt{R1} & \\ [.3em]
            \texttt{RS6} & \hl{\texttt{R3}} & & \texttt{R4} & \\
            \cmidrule{1-5}
            \texttt{ALU1} & \texttt{R2} & & \texttt{R1} & \\ [.3em]
            \texttt{ALU2} & & & & \\
            \bottomrule
        \end{tabular}
    \end{minipage}
    \hfill
    \begin{minipage}{0.45\textwidth}
        \centering
        \begin{tabular}{@{} l c @{}}
            \toprule
            Unit            & Remaining cycles \\
            \midrule
            \texttt{LDU1}   & \\ [.3em]
            \texttt{LDU2}   & \\ [.3em]
            \texttt{FPU1}   & \\ [.3em]
            \texttt{ALU1}   & 0 \\ [.3em]
            \texttt{ALU2}   & \\
            \bottomrule
        \end{tabular}
    \end{minipage}
    \newpage
    




    %%%%%%%%%%%%%%%%%%%%%%%%%%%%%%%%%%%%%%%%%%%%%%%%%%%%%%%%%%%%%%%%%%%%%%%%%%%%%%%%%%%%%%%%%%%%%%%%%%%%%%%%
    \item \textbf{Cycle \theenumi}:
    
    \begin{table}[!htp]
        \centering
        \begin{adjustbox}{width={\textwidth},totalheight={\textheight},keepaspectratio}
        \begin{tabular}{@{} l c c c c c c @{}}
            \toprule
            \textbf{Instruction} & \textbf{Issue} & \textbf{Start Exec} & \textbf{Write Res} & \textbf{Hazard} & \textbf{RSi} & \textbf{Unit} \\
            \midrule
            \texttt{I1: SD F4, 0 (R1)}      & 1 & 2 & 6 & None  & \texttt{RS1}   & \texttt{LDU1}  \\ [.5em]
            \texttt{I2: SD F6, 0 (R2)}      & 2 & 3 & 7 & None  & \texttt{RS2}   & \texttt{LDU2}  \\ [.5em]
            \texttt{I3: FADD F4, F0, F0}    & 3 & 4 & 8 & \texttt{WAR I1, F4}  & \texttt{RS3} & \texttt{FPU1} \\ [.5em]
            \texttt{I4: FADD F6, F2, F2}    & 4 & 9 & 13 & \texttt{WAR I2, F6} & \texttt{RS4} & \texttt{FPU1} \\ [.5em]
            \texttt{I5: LD F0, 16 (R1)}     & 7 & 8 & 12 & \texttt{WAR I3, F0} & \texttt{RS1} & \texttt{LDU1} \\ [.5em]
            \texttt{I6: LD F2, 16 (R2)}     & 8 & 9 & 14 & \texttt{WAR I4, F2} & \texttt{RS2} & \texttt{LDU2} \\ [.5em]
            \texttt{I7: ADDUI R1, R1, 8}    & 9 & 10 & 15 & \texttt{WAR I5, R1} & \texttt{RS5} & \texttt{ALU1} \\ [.5em]
            \texttt{I8: ADDUI R2, R2, 8}    & 10 & 11 & 16 & \texttt{WAR I6, R2} & \texttt{RS6} & \texttt{ALU2} \\ [.5em]
            \texttt{I9: ADD R3, R2, R1}     & 16 & 17 & 19 & \texttt{RAW R1, R2} & \texttt{RS5} & \texttt{ALU1} \\ [.5em]
            \texttt{I10: BNE R3, R4, LOOP}   & 17 & \hl{20} &   & \texttt{RAW I9, R3} & \texttt{RS6} & \hl{\texttt{ALU2}} \\
            \bottomrule
        \end{tabular}
        \end{adjustbox}
    \end{table}
    
    \begin{minipage}{0.45\textwidth}
        \centering
        \begin{tabular}{@{} l | l l l l @{}}
            \toprule
                & \texttt{Vj} & \texttt{Qj} & \texttt{Vk} & \texttt{Qk} \\
            \midrule
            \texttt{RS1} & & & & \\ [.3em]
            \texttt{RS2} & & & & \\
            \cmidrule{1-5}
            \texttt{LDU1} & & & & \\ [.3em]
            \texttt{LDU2} & & & & \\
            \bottomrule
        \end{tabular}
    \end{minipage}
    \hfill
    \begin{minipage}{0.45\textwidth}
        \centering
        \begin{tabular}{@{} l | l l l l @{}}
            \toprule
            & \texttt{Vj} & \texttt{Qj} & \texttt{Vk} & \texttt{Qk} \\
            \midrule
            \texttt{RS3} & & & & \\ [.3em]
            \texttt{RS4} & & & & \\
            \cmidrule{1-5}
            \texttt{FPU1} & & & & \\
            \bottomrule
        \end{tabular}
    \end{minipage}

    \begin{minipage}{0.45\textwidth}
        \centering
        \begin{tabular}{@{} l | l l l l @{}}
            \toprule
            & \texttt{Vj} & \texttt{Qj} & \texttt{Vk} & \texttt{Qk} \\
            \midrule
            \texttt{RS5} & & & & \\ [.3em]
            \texttt{RS6} & \texttt{R3} & & \texttt{R4} & \\
            \cmidrule{1-5}
            \texttt{ALU1} & & & & \\ [.3em]
            \texttt{ALU2} & \hl{\texttt{R3}} & & \hl{\texttt{R4}} & \\
            \bottomrule
        \end{tabular}
    \end{minipage}
    \hfill
    \begin{minipage}{0.45\textwidth}
        \centering
        \begin{tabular}{@{} l c @{}}
            \toprule
            Unit            & Remaining cycles \\
            \midrule
            \texttt{LDU1}   & \\ [.3em]
            \texttt{LDU2}   & \\ [.3em]
            \texttt{FPU1}   & \\ [.3em]
            \texttt{ALU1}   & \\ [.3em]
            \texttt{ALU2}   & 2 \\
            \bottomrule
        \end{tabular}
    \end{minipage}
    \newpage
        




    %%%%%%%%%%%%%%%%%%%%%%%%%%%%%%%%%%%%%%%%%%%%%%%%%%%%%%%%%%%%%%%%%%%%%%%%%%%%%%%%%%%%%%%%%%%%%%%%%%%%%%%%
    \setcounter{enumi}{21}
    \item \textbf{Cycle \theenumi}:
    
    \begin{table}[!htp]
        \centering
        \begin{adjustbox}{width={\textwidth},totalheight={\textheight},keepaspectratio}
        \begin{tabular}{@{} l c c c c c c @{}}
            \toprule
            \textbf{Instruction} & \textbf{Issue} & \textbf{Start Exec} & \textbf{Write Res} & \textbf{Hazard} & \textbf{RSi} & \textbf{Unit} \\
            \midrule
            \texttt{I1: SD F4, 0 (R1)}      & 1 & 2 & 6 & None  & \texttt{RS1}   & \texttt{LDU1}  \\ [.5em]
            \texttt{I2: SD F6, 0 (R2)}      & 2 & 3 & 7 & None  & \texttt{RS2}   & \texttt{LDU2}  \\ [.5em]
            \texttt{I3: FADD F4, F0, F0}    & 3 & 4 & 8 & \texttt{WAR I1, F4}  & \texttt{RS3} & \texttt{FPU1} \\ [.5em]
            \texttt{I4: FADD F6, F2, F2}    & 4 & 9 & 13 & \texttt{WAR I2, F6} & \texttt{RS4} & \texttt{FPU1} \\ [.5em]
            \texttt{I5: LD F0, 16 (R1)}     & 7 & 8 & 12 & \texttt{WAR I3, F0} & \texttt{RS1} & \texttt{LDU1} \\ [.5em]
            \texttt{I6: LD F2, 16 (R2)}     & 8 & 9 & 14 & \texttt{WAR I4, F2} & \texttt{RS2} & \texttt{LDU2} \\ [.5em]
            \texttt{I7: ADDUI R1, R1, 8}    & 9 & 10 & 15 & \texttt{WAR I5, R1} & \texttt{RS5} & \texttt{ALU1} \\ [.5em]
            \texttt{I8: ADDUI R2, R2, 8}    & 10 & 11 & 16 & \texttt{WAR I6, R2} & \texttt{RS6} & \texttt{ALU2} \\ [.5em]
            \texttt{I9: ADD R3, R2, R1}     & 16 & 17 & 19 & \texttt{RAW R1, R2} & \texttt{RS5} & \texttt{ALU1} \\ [.5em]
            \texttt{I10: BNE R3, R4, LOOP}   & 17 & 20 & \hl{22} & \texttt{RAW I9, R3} & \texttt{RS6} & \texttt{ALU2} \\
            \bottomrule
        \end{tabular}
        \end{adjustbox}
    \end{table}
    
    \begin{minipage}{0.45\textwidth}
        \centering
        \begin{tabular}{@{} l | l l l l @{}}
            \toprule
                & \texttt{Vj} & \texttt{Qj} & \texttt{Vk} & \texttt{Qk} \\
            \midrule
            \texttt{RS1} & & & & \\ [.3em]
            \texttt{RS2} & & & & \\
            \cmidrule{1-5}
            \texttt{LDU1} & & & & \\ [.3em]
            \texttt{LDU2} & & & & \\
            \bottomrule
        \end{tabular}
    \end{minipage}
    \hfill
    \begin{minipage}{0.45\textwidth}
        \centering
        \begin{tabular}{@{} l | l l l l @{}}
            \toprule
            & \texttt{Vj} & \texttt{Qj} & \texttt{Vk} & \texttt{Qk} \\
            \midrule
            \texttt{RS3} & & & & \\ [.3em]
            \texttt{RS4} & & & & \\
            \cmidrule{1-5}
            \texttt{FPU1} & & & & \\
            \bottomrule
        \end{tabular}
    \end{minipage}

    \begin{minipage}{0.45\textwidth}
        \centering
        \begin{tabular}{@{} l | l l l l @{}}
            \toprule
            & \texttt{Vj} & \texttt{Qj} & \texttt{Vk} & \texttt{Qk} \\
            \midrule
            \texttt{RS5} & & & & \\ [.3em]
            \texttt{RS6} & \texttt{R3} & & \texttt{R4} & \\
            \cmidrule{1-5}
            \texttt{ALU1} & & & & \\ [.3em]
            \texttt{ALU2} & \texttt{R3} & & \texttt{R4} & \\
            \bottomrule
        \end{tabular}
    \end{minipage}
    \hfill
    \begin{minipage}{0.45\textwidth}
        \centering
        \begin{tabular}{@{} l c @{}}
            \toprule
            Unit            & Remaining cycles \\
            \midrule
            \texttt{LDU1}   & \\ [.3em]
            \texttt{LDU2}   & \\ [.3em]
            \texttt{FPU1}   & \\ [.3em]
            \texttt{ALU1}   & \\ [.3em]
            \texttt{ALU2}   & 0 \\
            \bottomrule
        \end{tabular}
    \end{minipage}
\end{enumerate}
\emph{Calculate the CPI.}

\highspace
\textcolor{Green3}{\textbf{\emph{Answer:}}}
\begin{equation*}
    \text{CPI} = \dfrac{\text{\# Clock Cycles}}{\text{\# Instructions}} = \dfrac{22}{10} = 2.2
\end{equation*}

\newpage

\subsubsection*{Question 3: Loop Unrolling}

\emph{Let's consider the following loop code where registers \textbf{R1} and \textbf{R2} are initialized to \textbf{0} and \textbf{R3} is initialized to \textbf{400}:}
\begin{lstlisting}
LOOP:   LD F2, 0 (R1)
        LD F4, 0 (R2)
        FADD F6, F2, F4
        SD F6, 0 (R1)
        ADDUI R1, R1, 4
        ADDUI R2, R2, 4
        BNE R1, R3, LOOP
\end{lstlisting}
\emph{Answer to the following questions:}
\begin{itemize}
    \item \emph{How many iterations of the loop?}

    \textcolor{Green3}{\textbf{\emph{Answer:}}} 100 iterations. This is because register \texttt{R1} is initialized to \texttt{0} and is incremented by \texttt{4} each iteration until it reaches \texttt{400}:
    \begin{equation*}
        \text{Iterations} = \dfrac{400}{4} = 100
    \end{equation*}
    So the loop executes \texttt{100} iterations before \texttt{R1} equals \texttt{R3} and the loop exits. Note that the \texttt{SD F6, 0(R1)} instruction stores \texttt{F6} back into memory at the address in \texttt{R1}; it does not write to \texttt{R1}. Therefore, only the \texttt{ADDUI} operator modifies the \texttt{R1} register.


    \item \emph{How many instructions per iteration?}

    \textcolor{Green3}{\textbf{\emph{Answer:}}} The number of instructions per iteration equals the number of instructions in the loop block, which is 7.
    
    
    \item \emph{How many instructions due to the loop overhead per iteration?}

    \textcolor{Green3}{\textbf{\emph{Answer:}}} When we write a loop in assembly, \textbf{some instructions do useful work}, and \textbf{some instructions are only there to control the loop}.
    \begin{itemize}
        \item \textbf{Useful work}: instructions that do the actual computation we want (load data, do the calculation, store result).
        \item \textbf{Overhead}: instructions that \emph{only} exist to keep the loop going; updating counters, changing pointers, checking conditions, branching.
    \end{itemize}
    In our case, we have 4 useful operations (2 loads, 1 add, and 1 store), and \textbf{3} that \emph{only} exist to make the loop repeat: incrementing pointers/counters (\texttt{ADDUI R1}, \texttt{ADDUI R2}) and testing the loop condition (\texttt{BNE}). Loop overhead is extra work we must do every iteration to \textbf{make the loop repeat}, but it does not contribute to the real useful computation.
    
    
    \item \emph{How many instructions are executed globally?}

    \textcolor{Green3}{\textbf{\emph{Answer:}}} There are 7 instructions for each iteration. Since there are 100 iterations, \textbf{700 instructions} ($7 \times 100$) are executed globally.
    
    
    \item \emph{How many branch instructions are executed globally?}

    \textcolor{Green3}{\textbf{\emph{Answer:}}} There is 1 branch instruction for each iteration, so 100 branch instructions ($1 \times 100$) are executed globally.
\end{itemize}
\textbf{Let's consider a loop unrolling version of the code with unrolling factor 2.}
\begin{itemize}
    \item \emph{How many iterations of the unrolled loop?}

    \textcolor{Green3}{\textbf{\emph{Answer:}}} Instead of repeating a small body many times, we manually replicate the body multiple times inside the loop; so each iteration does more useful work, but we run the loop fewer times (loop unrolling).


    If we unroll by a factor of 2, the loop body is rewritten to do 2 operations per iteration:
    \begin{lstlisting}
LOOP_UNROLLING_2:   LD F2, 0(R1)
                    LD F4, 0(R2)
                    FADD F6, F2, F4
                    SD F6, 0(R1)

                    LD F8, 4(R1)
                    LD F10, 4(R2)
                    FADD F12, F8, F10
                    SD F12, 4(R1)

                    ADDUI R1, R1, 8
                    ADDUI R2, R2, 8
                    BNE R1, R3, LOOP_UNROLLING_2\end{lstlisting}
    So each iteration now does twice as much useful work and we only need half as many loop iterations:
    \begin{equation*}
        \text{New Iteration Count} = \dfrac{\text{Original Iteration Count}}{\text{Unrolling Factor}} = \dfrac{100}{2} = 50
    \end{equation*}
    
    
    \item \emph{How many instructions per iteration?}

    \textcolor{Green3}{\textbf{\emph{Answer:}}} The instructions are now 11.
    
    
    \item \emph{How many instructions due to the loop overhead per iteration?}

    \textcolor{Green3}{\textbf{\emph{Answer:}}} There are always 3 because we have unrolled useful work, not overhead work.
    
    
    \item \emph{How many instructions are executed globally?}

    \textcolor{Green3}{\textbf{\emph{Answer:}}} There are 11 instructions for each iteration. We repeat the cycle 50 times, so a total of 550 instructions are executed.
    
    
    \item \emph{How many branch instructions are executed globally?}

    \textcolor{Green3}{\textbf{\emph{Answer:}}} There is only one branch instruction per cycle and 50 iterations, so a total of 50 branch instructions are executed globally.
    
    
    \item \emph{How many more registers are needed due to register renaming?}

    \textcolor{Green3}{\textbf{\emph{Answer:}}} Original loop has 3 FP registers (\texttt{F2} holds first operand, \texttt{F4} holds second operand, and \texttt{F6} holds result). When we unroll the loop by factor 2, we do two sums at the same time. If we use the same registers, we overwrite the result of the first sum before we store it. So we need new registers for the second pair.

    \newpage

    \noindent
    Therefore, we can conclude that:
    \begin{itemize}
        \item For the first sum: \texttt{F2}, \texttt{F4}, \texttt{F6}.
        \item For the second sum: \texttt{F8}, \texttt{F10}, \texttt{F12}.
    \end{itemize}
    So we need 3 more FP registers. In general, with an unrolling factor $X$, we need:
    \begin{equation*}
        \left(X-1\right) \times \left(\text{number of live registers per body}\right) \quad \text{more registers}
    \end{equation*}    
    
    
    \item \emph{What are the registers that need to be renamed?}

    \textcolor{Green3}{\textbf{\emph{Answer:}}} As mentioned earlier, the registers that need to be renamed are \texttt{F2}, \texttt{F4}, and \texttt{F6}.
    
    
    \item \emph{How much is the global instruction count decrease?}

    \textcolor{Green3}{\textbf{\emph{Answer:}}} Before unrolling, the global instructions to execute was 700, and now 550. So we have 150 instructions less than before, that correspond to:
    \begin{equation*}
        700 : 100 = 150 : x \quad \Rightarrow \quad \dfrac{150 \cdot 100}{700} = 21.428571429 = 21.43 \%
    \end{equation*}
\end{itemize}
\textbf{Let's consider a loop unrolling version of the code with unrolling factor 4.}
\begin{itemize}
    \item \emph{How many iterations of the unrolled loop?}

    \textcolor{Green3}{\textbf{\emph{Answer:}}} If we unroll by a factor of 4, the loop body is rewritten to do 4 operations per iteration:

    \begin{lstlisting}
LOOP_UNROLLING_4:   LD F2, 0(R1)
                    LD F4, 0(R2)
                    FADD F6, F2, F4
                    SD F6, 0(R1)

                    LD F8, 4(R1)
                    LD F10, 4(R2)
                    FADD F12, F8, F10
                    SD F12, 4(R1)

                    LD F14, 8(R1)
                    LD F16, 8(R2)
                    FADD F18, F14, F16
                    SD F18, 8(R1)

                    LD F20, 12(R1)
                    LD F22, 12(R2)
                    FADD F24, F20, F22
                    SD F24, 12(R1)

                    ADDUI R1, R1, 16
                    ADDUI R2, R2, 16
                    BNE R1, R3, LOOP_UNROLLING_4\end{lstlisting}
    \begin{equation*}
        \text{New Iteration Count} = \dfrac{\text{Original Iteration Count}}{\text{Unrolling Factor}} = \dfrac{100}{4} = 25
    \end{equation*}


    \item \emph{How many instructions per iteration?}

    \textcolor{Green3}{\textbf{\emph{Answer:}}} The instructions are now 19.


    \item \emph{How many instructions due to the loop overhead per iteration?}

    \textcolor{Green3}{\textbf{\emph{Answer:}}} There are always 3 instructions due to the loop overhead per iteration.


    \item \emph{How many instructions are executed globally?}

    \textcolor{Green3}{\textbf{\emph{Answer:}}}
    \begin{equation*}
        19 \text{ instructions} \cdot 25 \text{ iterations} = 475
    \end{equation*}


    \item \emph{How many branches are executed?}

    \textcolor{Green3}{\textbf{\emph{Answer:}}} There is always 1 branch for each iteration, so a total of 25 branches are executed.


    \item \emph{How many more registers are needed due to register renaming?}

    \textcolor{Green3}{\textbf{\emph{Answer:}}}
    \begin{equation*}
        \left(4 - 1\right) \cdot 3 = 9 \text{ more registers}
    \end{equation*}


    \item \emph{What are the registers that need to be renamed?}

    \textcolor{Green3}{\textbf{\emph{Answer:}}} It's always the three registers \texttt{F2}, \texttt{F4}, and \texttt{F6}, but three times.


    \item \emph{How much is the global instruction count decrease?}

    \textcolor{Green3}{\textbf{\emph{Answer:}}}
    \begin{equation*}
        700 : 100 = \left(700-475\right) : x \quad \Rightarrow \quad \dfrac{225 \cdot 100}{700} = 32.142857143 = 32.14 \%
    \end{equation*}


    \item \emph{What is the best unrolling factor between 2 and 4? Explain why.}

    \textcolor{Green3}{\textbf{\emph{Answer:}}} Factor 4 is preferable to Factor 2 because it reduces loop control overhead per useful operation even further. The number of iterations is halved again, from 50 to 25, and the overhead instructions per sum drop from $1.5$ to $0.75$.
    
    However, a higher unrolling factor requires more register names due to overlapping live values. In this case, factor 4 increases the demand for floating-point registers from 3 to 12. This increase must be handled by the hardware's register renaming; otherwise, spilling will occur. Therefore, factor 4 is preferable if the register file and rename logic are large enough to store all live values in registers without spilling to memory. Otherwise, factor 2 may be safer.
\end{itemize}

\newpage

\subsubsection*{Quiz 4 - Branch Prediction}

\emph{During the execution of a program, the number of mispredictions depends on the initialization and the size of the Branch History Table (BHT). True or False? Motivate.}

\highspace
\textcolor{Green3}{\textbf{\emph{Answer:}}} A BHT stores the \emph{history} (past behavior) of branches, usually with \emph{n}-bit saturating counters (\autopageref{def:branch-history-table}). The size of the BHT determines \textbf{how many unique branches can be tracked independently}
\begin{itemize}
    \item Small BHT $\rightarrow$ more aliasing (\autopageref{def:aliasing}) $\rightarrow$ different branches share the same entry $\rightarrow$ more chance of misprediction.
    \item Large BHT $\rightarrow$ fewer collisions $\rightarrow$ better prediction accuracy.
\end{itemize}
At program start, the BHT entries are \textbf{initialized} (often all taken or all not take, or all weakly taken). If the actual branch pattern is opposite to the initialization, we'll have \textbf{mispredictions} until the counters adapt. So initialization affects startup behavior (how many initial mispredictions we get) and size of BHT affects aliasing (whether unrelated branches interfere with each other, more or fewer mispredictions overall).

\highspace
The correct answer is: \textbf{True}, because the number of mispredictions depends on how the BHT entries are initialized and on the size of the BHT, which determines how many branches can be tracked without aliasing.

\longline

\subsubsection*{Quiz 5 - Branch Prediction}

\emph{Consider the following assembly code executed by a processor with 1-entry 1-bit Branch History Table initialized as Taken}:
\begin{lstlisting}
        ADDI R1, R0, 0
        ADDI R2, R0, 100

LOOP:   BEQ R1, R2, DONE
        ADD R5, R5, R4
        ADDI R1, R1, 1
        J LOOP

DONE:
\end{lstlisting}
\emph{Assuming \textbf{only} the branch instruction accesses the BHT:}
\begin{itemize}
    \item \emph{How many accesses to the Branch History Table?}
    \item \emph{How many mispredictions?}
\end{itemize}
\emph{\textbf{(Single Answer)}}
\begin{description}
    \item[\textbf{Answer 1:}] 101 accesses, 2 mispredictions
    \item[\textbf{Answer 2:}] 100 accesses, 1 misprediction
    \item[\textbf{Answer 3:}] 99 accesses, 1 misprediction
    \item[\textbf{Answer 4:}] 201 accesses, 2 mispredictions
\end{description}
\textcolor{Green3}{\textbf{\emph{Answer:}}} The correct answer is \textbf{Answer 1: 101 accesses, 2 mispredictions}. Let's explain why. See also: \autopageref{sec:1-bit-bht}.
\begin{itemize}
    \item \textbf{How many BHT accesses?} Only the \textbf{conditional branch} \texttt{BEQ R1, R2, DONE} accesses the BHT (the \texttt{J LOOP} does \textbf{not}, by assumption).
    \begin{itemize}
        \item \code{R1} starts at \code{0}, \code{R2 = 100}.
        \item The branch is executed once per loop check for $\code{R1 = 0,1,2,\ldots,100}$.
        \begin{itemize}
            \item For $\code{R1 = 0 \ldots 99}$: branch outcome is \textbf{Not Taken}.
            \item For $\code{R1 = 100}$: branch outcome is \textbf{Taken} and we exit the loop.
        \end{itemize}
    \end{itemize}
    So the number of executions (thus BHT accesses) of the branch is:
    \begin{equation*}
        100 \text{ (for R1 = 0 to 99)} + 1 \text{ (for R1 = 100)} = 101 \text{ accesses}
    \end{equation*}


    \item \textbf{How many mispredictions with a 1-entry, 1-bit BHT initialized as Taken?} A \textbf{1-bit BHT} remembers the \textbf{last outcome} (Taken or Not Taken) of the branch. If correct it stays, if wrong it flips. The initial state is \textbf{Predict Taken}. Then, let's walk through the loop:
    \begin{enumerate}
        \item \textbf{First time} branch is encountered \code{R1 = 0}:
        \begin{itemize}
            \item[\textcolor{Green3}{\faIcon{magic}}] Prediction: \textbf{Taken} (initial state, written on the exercise text).
            \item[\textcolor{Green3}{\faIcon{check}}] Actual outcome: \textbf{Not Taken} (since \code{R1 != R2})
            \item[\textcolor{Red3}{\faIcon{times}}] \textbf{Misprediction!} Update BHT to \textbf{Predict Not Taken} (flip the bit).
        \end{itemize}

        \item From \code{R1 = 1} to \code{R1 = 99} (99 iterations):
        \begin{itemize}
            \item[\textcolor{Green3}{\faIcon{magic}}] Prediction: \textbf{Not Taken} (from last update).
            \item[\textcolor{Green3}{\faIcon{check}}] Actual outcome: \textbf{Not Taken} (since \code{R1 != R2})
            \item[\textcolor{Green3}{\faIcon{check}}] \textbf{Correct prediction!} BHT remains \textbf{Predict Not Taken}.
        \end{itemize}

        \item \textbf{Exit check} when \code{R1 = 100}:
        \begin{itemize}
            \item[\textcolor{Green3}{\faIcon{magic}}] Prediction: \textbf{Not Taken} (from last update).
            \item[\textcolor{Green3}{\faIcon{check}}] Actual outcome: \textbf{Taken} (since \code{R1 == R2})
            \item[\textcolor{Red3}{\faIcon{times}}] \textbf{Misprediction!} Update BHT to \textbf{Predict Taken} (flip the bit).
        \end{itemize}
    \end{enumerate}
    So in total, we have \textbf{2 mispredictions} (first encounter and exit check).
\end{itemize}
The answer is as expected, as it aligns with the \textbf{traditional limitation of 1-bit predictors}. They often overlook a ``mostly one direction'' pattern when the outcome changes. In this case, the outcome is first \emph{Not Taken} and then \emph{Taken}. This results in two misses in this specific scenario: one at the beginning due to poor initialization and one at the final change.

\newpage

\subsubsection*{Quiz 6 - Superscalar Processors}

\emph{Considering superscalar processors, answer \textbf{TRUE} or \textbf{FALSE} to the following questions, \textbf{motivating your answers}.}

\begin{itemize}
    \item \emph{Doubling the number of issues reduces the CPI metric.}

    \textcolor{Green3}{\textbf{\emph{Answer:}}} \textbf{True}. In superscalar processors, the CPI (Cycles Per Instruction) is ideally:
    \begin{equation*}
        \text{CPI}_{\text{ideal}} = \dfrac{1}{\text{issue width}}
    \end{equation*}
    So if we double the issue width, theoretical CPI reduces. This is true only under the assumption of perfect conditions: infinite ILP, no hazards, no cache misses, no branch mispredictions, perfect decode, dispatch, execution bandwidth.

    
    \item \emph{Out-of-order execution can be generated by data cache misses.}

    \textcolor{Green3}{\textbf{\emph{Answer:}}} \textbf{True}. Cache misses introduce stalls for dependent instructions; out-of-order execution allows independent instructions to bypass stalled loads, maximizing utilization of execution units. In other words, data cache misses do \textbf{not prevent} execution; in an out-of-order superscalar processor, they often \textbf{enable visible out-of-order behavior} by stalling one instruction and allowing others to bypass it.

    \begin{deepeningbox}[: Motivation]
        In a \textbf{superscalar processor} (\autopageref{subsubsection: Superscalar Processors}), instructions are fetched and decoded in program order, but, if the microarchitecture supports it, they may \textbf{execute out of order} as long as \textbf{data and control dependencies are respected}.

        \highspace
        A \textbf{data cache miss} (\autopageref{def:data-cache-miss}, e.g., a load that misses in L1/L2) introduces a \textbf{long and unpredictable latency}. When this happens:
        \begin{itemize}
            \item The instruction that triggered the miss \textbf{cannot complete} (it is stalled waiting for data from lower levels of the memory hierarchy).
            \item Independent instructions that are \textbf{already in the instruction window} / \textbf{reservation stations} and \textbf{do not depend on the missing load} can still be: issued, executed, and possibly completed \textbf{before} the stalled load completes.
        \end{itemize}
        This behavior is precisely \textbf{out-of-order execution}: instructions that appear \emph{later} in program order are executed \emph{earlier} in time.

        \highspace
        From a hardware point of view:
        \begin{itemize}
            \item The cache miss \textbf{creates a stall in one dependency chain}.
            \item The \textbf{dynamic scheduler} (e.g., Tomasulo logic with reservation stations and register renaming) exploits \textbf{instruction-level parallelism (ILP)} by continuing execution on other ready instructions.
        \end{itemize}
        Therefore, although the cache miss itself is not a scheduling mechanism, it is a \textbf{triggering event} that \textbf{induces out-of-order execution} by freeing the scheduler to execute independent instructions while the miss is serviced.
    \end{deepeningbox}


    \item \emph{The introduction of dynamic branch predictors improves the CPI metric.}

    \textcolor{Green3}{\textbf{\emph{Answer:}}} \textbf{True}. Dynamic branch predictors reduce the number of branch mispredictions by learning from past branch behavior. Fewer mispredictions mean fewer pipeline flushes and stalls, which directly lowers the average Cycles Per Instruction (CPI) and improves overall processor performance.

    \begin{deepeningbox}[: Motivation]
        The \textbf{Cycles Per Instruction (CPI)} metric captures both the ideal execution rate and the penalties due to pipeline hazards, including \textbf{control hazards} caused by branches (\autopageref{sec:control-hazards}). In a pipelined or superscalar processor:
        \begin{itemize}
            \item A \textbf{branch misprediction} causes:
            \begin{itemize}
                \item Pipeline flushing (all instructions after the mispredicted branch are discarded).
                \item Squashing of wrong-path instructions (i.e., instructions that were fetched and possibly executed based on the incorrect prediction are invalidated).
                \item And a \textbf{stall penalty} proportional to pipeline depth and issue width.
            \end{itemize}
            \item This penalty directly \textbf{increases CPI}, since extra cycles are spent without retiring useful instructions.
        \end{itemize}
        A \textbf{dynamic branch predictor} (e.g., 1-bit, 2-bit saturaing counters, local/global history predictors):
        \begin{itemize}
            \item Learns branch behavior at runtime.
            \item Increases \textbf{prediction accuracy} compared to static schemes (e.g., always taken / always not taken).
            \item \textbf{Reduces the misprediction rate}.
        \end{itemize}
        Since CPI can be expressed as:
        \begin{equation*}
            \text{CPI} = \text{CPI}_{\text{ideal}} + \text{Branch frequency} \times \text{Misp. rate} \times \text{Misp. penalty}
        \end{equation*}
        Improving prediction accuracy \textbf{reduces the branch penalty term}, thus \textbf{lowering CPI}. Therefore, introducing \textbf{dynamic branch predictors} improves (i.e., reduces) the CPI metric, assuming: branches are present (which is almost always true), and the predictor does not introduce prohibitive overheads.
    \end{deepeningbox}
\end{itemize}