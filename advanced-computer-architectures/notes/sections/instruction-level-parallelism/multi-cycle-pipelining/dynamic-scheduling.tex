\subsection{Dynamic Scheduling}

As we've seen, static in-order pipelines are limited in their ability to exploit ILP because they stall the entire pipeline when a single instruction is blocked. \definition{Dynamic Scheduling} solves this by allowing \textbf{instructions to be issued, executed, and even completed out of order}, as long as doing so does not violate program correctness.

\highspace
\begin{flushleft}
    \textcolor{Red2}{\faIcon{question-circle} \textbf{The Need for Dynamic Scheduling}}
\end{flushleft}
Consider the following instruction sequence:
\begin{lstlisting}[language=riscv, mathescape=true]
I1:  F0  $\leftarrow$ F2 / F4   # long-latency divide
I2:  F10 $\leftarrow$ F0 + F8   # depends on F0
I3:  F12 $\leftarrow$ F8 - F14  # independent of I1 and I2
\end{lstlisting}
In a naive in-order pipeline:
\begin{itemize}
    \item \texttt{I2} stalls waiting for \texttt{F0}, and
    \item \texttt{I3} stalls behind \texttt{I2}, even though it's independent.
\end{itemize}
This results in lost parallelism. In contrast, \textbf{dynamic scheduling} would:
\begin{itemize}
    \item Allow \texttt{I1} to begin and proceed through the divider.
    \item Stall \texttt{I2} because it depends on \texttt{F0}.
    \item Allow \texttt{I3} to \textbf{proceed and complete} immediately, despite the stall.
\end{itemize}
This is possible because the processor can \textbf{track operand availability} and issue instructions \textbf{as soon as dependencies are satisfied}, not based solely on their program order.

\highspace
\begin{flushleft}
    \textcolor{Green3}{\faIcon{tools} \textbf{How Dynamic Scheduling Works}}
\end{flushleft}
\textbf{Instructions are issued in order} but may \textbf{execute and complete out of order}, depending on operand readiness and unit availability. The processor uses dedicated \textbf{hardware structures} to manage this:
\begin{itemize}
    \item Reservation Stations (or Issue Queues)
    \item Reorder Buffer (ROB)
    \item Register Renaming Tables
    \item Common Data Bus (CDB) for broadcasting results
\end{itemize}
In a dynamically scheduled pipeline, stages are typically:
\begin{itemize}
    \item \textbf{Fetch (IF)}: Get instruction from memory.
    \item \textbf{Decode (ID)}: Determine opcode, operands, destination.
    \item \textbf{Issue}: Place instruction into reservation station if operands aren't ready.
    \item \textbf{Execute (EX)}: Start when all operands are available.
    \item \textbf{Write Result}: Write result to a temporary buffer or broadcast to waiting instructions.
    \item \textbf{Commit}: Update architectural registers in order.
\end{itemize}

\highspace
\begin{flushleft}
    \textcolor{Green3}{\faIcon{check-circle} \textbf{Benefits}}
\end{flushleft}
\begin{itemize}[label=\textcolor{Green3}{\faIcon{check}}]
    \item \textbf{Higher ILP}: Instructions don't wait unnecessarily.
    \item \textbf{Resource Utilization}: Keeps functional units busy.
    \item \textbf{Latency Hiding}: Tolerates cache misses, long FP ops.
    \item \textbf{Exploits Independence}: Independent ops no longer block one another.
\end{itemize}

\highspace
\begin{flushleft}
    \textcolor{Red2}{\faIcon{exclamation-triangle} \textbf{Challenges Introduced}}
\end{flushleft}
While powerful, dynamic scheduling is \textbf{complex}:
\begin{itemize}[label=\textcolor{Red2}{\faIcon{times}}]
    \item \textbf{WAR and WAW Hazards}. With out-of-order execution, later instructions might write before earlier ones. Solution: use register renaming to remove name dependencies.
    \item \textbf{Imprecise Exceptions}. If a later instruction causes an exception, but earlier ones have already modified state, it becomes hard to roll back. Solution: use a Reorder Buffer (ROB) that holds results temporarily and commits them in program order.
    \item \textbf{Hardware Cost and Complexity}. Additional logic is needed for:
    \begin{itemize}
        \item Dependency tracking
        \item Wakeup and select logic
        \item Common data bus broadcasting
        \item Instruction window buffering
    \end{itemize}
\end{itemize}