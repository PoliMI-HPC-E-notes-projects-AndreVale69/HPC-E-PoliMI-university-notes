\subsection{Multi-Cycle Pipelining}\label{subsection: Multi-Cycle Pipelining}

As processor microarchitectures evolved to support more complex instructions and higher performance demands, the basic model of a uniform single-cycle pipeline became insufficient. In practice, \textbf{many instructions}, especially those involving floating-point operations, memory access, or division, \textbf{require more than one clock cycle} to complete their execution or memory stages.

\highspace
This leads to the development of \textbf{multi-cycle pipelines}, where \textbf{individual stages} (particularly EX and MEM) may \textbf{last for multiple cycles}, depending on the instruction type and runtime events. In such architectures, the ability to manage instruction progress intelligently becomes central to maintaining high throughput and correctness.

\highspace
\begin{flushleft}
    \textcolor{Green3}{\faIcon{book} \textbf{Motivation and Assumptions}}
\end{flushleft}
In a classical 5-stage pipeline (IF, ID, EX, MEM, WB), all \textbf{stages are assumed to complete in one clock cycle}. However, this assumption doesn't hold in realistic systems:
\begin{itemize}
    \item \important{Integer instructions} may complete in 1-2 cycles.
    \item \important{Floating-point operations}, like multiplication or division, can take 3 to $10+$ cycles.
    \item \important{Memory access} times are unpredictable due to cache hits and misses, which can add variable delays.
    \item \important{Instruction fetch} may also stall due to instruction cache misses or branch resolution delays.
\end{itemize}
To accommodate these characteristics, \textbf{processors adopt multi-cycle pipe}-\break \textbf{lines}, where:
\begin{itemize}
    \item Execution latency varies by operation type.
    \item Memory access can take multiple cycles.
    \item Functional units are not necessarily pipelined, particularly for floating-point operations.
\end{itemize}
\textbf{Basic assumptions} in this model:
\begin{enumerate}
    \item The processor is \textbf{single-issue} (one instruction issued per cycle).
    \item \textbf{Instructions} are typically \textbf{issued in-order} (fetched and passed into the pipeline in the order that they appear in the program).
    \item Execution and memory stages may involve \textbf{multiple functional units with variable latencies}.
    \item \textbf{Write-back} is often \textbf{delayed or synchronized} to ensure consistent state updates and avoid hazards.
\end{enumerate}