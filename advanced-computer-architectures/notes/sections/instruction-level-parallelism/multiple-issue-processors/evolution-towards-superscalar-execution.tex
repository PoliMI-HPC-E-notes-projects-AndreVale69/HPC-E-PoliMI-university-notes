\subsubsection{Evolution Towards Superscalar Execution}

The transition from simple scalar pipelines to high-performance superscalar processors is not abrupt. It is the result of a \textbf{progressive refinement} of microarchitectural techniques aimed at exposing and exploiting more Instruction-Level Parallelism (ILP). This section traces the key evolutionary steps that bridge the gap between single-issue scalar designs and fully dynamic multiple-issue architectures.

\highspace
\begin{flushleft}
    \textcolor{Red2}{\faIcon{hourglass-half} \textbf{Step 1: Single-Issue, In-Order Execution}}
\end{flushleft}
This is the traditional scalar baseline and was explained in the earlier sections (section \ref{subsubsection: Multi-Cycle In-Order Pipeline}, page \pageref{subsubsection: Multi-Cycle In-Order Pipeline}).
\begin{itemize}
    \item Only \textbf{one instruction} is \textbf{issued} and \textbf{committed} per clock cycle.
    \item All instructions are fetched, decoded, executed, and written back in \textbf{strict program order}.
    \item Hazards (data, structural, control) cause pipeline stalls that affect all subsequent instructions.
    \item CPI $\ge 1$, IPC $\le 1$.
\end{itemize}
This model is \textcolor{Green3}{\textbf{simple and ensures precise state at all times}}, but it is \textcolor{Red2}{\textbf{severely limited in ILP exploitation}}.

\highspace
\begin{flushleft}
    \textcolor{Red2}{\faIcon{hourglass-half} \textbf{Step 2: Single-Issue, Out-of-Order Execution}}
\end{flushleft}
To overcome unnecessary stalls caused by instruction dependencies, processors began executing instructions \textbf{out of order}, while still \textbf{issuing only one instruction per cycle} (section \ref{subsubsection: Multi-Cycle Out-of-Order Pipeline}, page \pageref{subsubsection: Multi-Cycle Out-of-Order Pipeline}).
\begin{itemize}
    \item Instructions are fetched and issued \textbf{in program order}.
    \item But \textbf{independent instructions} are allowed to \textbf{execute and complete out of order}, as soon as their operands are ready and a functional unit is available.
    \item Techniques like \textbf{dynamic scheduling} (e.g., Tomasulo's algorithm) are used to manage dependencies and operand availability.
    \item A \textbf{commit stage ensures in-order architectural updates}, preserving program correctness and exception handling.
\end{itemize}
This \textcolor{Green3}{\textbf{significantly improves ILP}}, but \textcolor{Red2}{\textbf{throughput is still constrained by the single-issue limit}}.

\newpage

\begin{flushleft}
    \textcolor{Green3}{\faIcon{chart-line} \textbf{Step 3: Multiple-Issue, In-Order Execution (Dual-Issue Pipeline)}}
\end{flushleft}
This step involves fetching, decoding, and executing \textbf{more than one instruction per cycle}, but \textbf{in program order}.
\begin{itemize}
    \item Typical example: \textbf{dual-issue pipelines} (e.g., MIPS dual-issue architecture, section \ref{subsubsection: Introduction to Multiple-Issue Pipelines}, page \pageref{subsubsection: Introduction to Multiple-Issue Pipelines}).
    \item The \textbf{hardware allows the issue of up to two instructions per clock}, provided they are independent and compatible (e.g., one ALU $+$ one Load/Store).
    \item Requires hardware additions such as:
    \begin{itemize}
        \item Multiple functional units,
        \item Multi-ported register file,
        \item Hazard detection logic across simultaneously issued instructions.
    \end{itemize}
\end{itemize}
This model \textcolor{Green3}{\textbf{increases IPC (ideal $\mathbf{IPC} = 2$)}}, but still suffers from limitations:
\begin{itemize}[label=\textcolor{Red2}{\faIcon{times}}]
    \item \textcolor{Red2}{\textbf{Dependent instructions must wait}}, even if others could proceed.
    \item \textcolor{Red2}{\textbf{Static scheduling}} (compiler) or simple hardware interlocks determine issue feasibility.
\end{itemize}

\highspace
\begin{flushleft}
    \textcolor{Green3}{\faIcon{\speedIcon} \textbf{Step 4: Multiple-Issue, Out-of-Order Execution}}
\end{flushleft}
This is the \textbf{most flexible and powerful configuration}, forming the \textbf{basis of superscalar processors}.
\begin{itemize}
    \item Fetches and decodes \textbf{multiple instructions per cycle}.
    \item Uses \textbf{dynamic scheduling logic} to decide which subset can be issued and executed out of order.
    \item Independent \textbf{instructions proceed as soon as their operands are ready}, regardless of program order.
    \item Results are \textbf{committed in order} to maintain a precise architectural state.
    \item This model requires \textbf{complex hardware}:
    \begin{itemize}
        \item Reservation stations
        \item Register renaming
        \item Reorder buffer (ROB)
        \item Instruction window/wakeup-select logic
    \end{itemize}
\end{itemize}
This architecture provides the \textcolor{Green3}{\textbf{highest ILP}}, as it combines the breadth of multiple issue with the flexibility of out-of-order execution.

\newpage

\noindent
The transition toward superscalar execution is a gradual process, where each step builds upon the previous to \textbf{mitigate limitations}, \textbf{maximize resource utilization}, and \textbf{exploit greater ILP}. Superscalar architectures represent the culmination of this evolution, dynamically scheduling and executing multiple instructions in parallel, out of order, while maintaining program correctness and exception safety.

\begin{table}[!htp]
    \centering
    \begin{tabular}{@{} c | l | l | l | l @{}}
        \toprule
        \textbf{Step} & \textbf{Issue} & \textbf{Execution} & \textbf{Commit} & \textbf{Example} \\
        \midrule
        1             & In-order       & In-order           & In-order        & Scalar pipeline \\ [.3em]
        2             & In-order       & Out-of-order       & In-order        & Dynamic single-issue \\ [.3em]
        3             & In-order       & In-order           & In-order        & Dual-issue pipeline \\ [.3em]
        4             & In-order       & Out-of-order       & In-order        & Superscalar processor \\
        \bottomrule
    \end{tabular}
    \caption{Evolution towards superscalar execution.}
\end{table}