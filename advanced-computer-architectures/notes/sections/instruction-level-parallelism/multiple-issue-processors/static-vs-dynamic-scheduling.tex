\subsubsection{Static vs Dynamic Scheduling}

Instruction-level parallelism (ILP) can be \textbf{exploited} either \textbf{at compile time} by the compiler or \textbf{at runtime} by the hardware. These two approaches give rise to two distinct classes of architectures:
\begin{itemize}
    \item \important{Statically scheduled processors}, such as VLIW (Very Long Instruction Word) architectures.
    \item \important{Dynamically scheduled processors}, such as superscalar architectures.
\end{itemize}
This section compares these two philosophies, focusing on their mechanisms, advantages, drawbacks, and the contexts in which each is most appropriate.

\begin{table}[!htp]
    \centering
    \begin{tabular}{@{} l | l | l @{}}
        \toprule
        \textbf{Scheduling Type} & \textbf{Performed By} & \textbf{When?} \\
        \midrule
        Static  & Compiler & At compile time \\ [.3em]
        Dynamic & Hardware & At runtime \\
        \bottomrule
    \end{tabular}
    \caption{Key difference: who does the scheduling?}
\end{table}

\highspace
\begin{flushleft}
    \textcolor{Green3}{\faIcon{question-circle} \textbf{What is VLIW (Static Scheduling)?}}
\end{flushleft}
\definition{VLIW (Very Long Instruction Word)} is a \textbf{statically scheduled}, \textbf{multiple issue processor architecture} in which the \hl{compiler selects and packs multiple independent operations into a single, wide instruction word} that is \textbf{executed in parallel} by the processor.

\highspace
The \textbf{instruction word} in a VLIW architecture consists of \textbf{multiple operations} (e.g., an ALU op, a memory op, and a floating-point op) that are \textbf{intended to be executed simultaneously}. The \textbf{compiler is responsible} for:
\begin{itemize}
    \item Detecting ILP in the program.
    \item Scheduling independent instructions to avoid hazards.
    \item Filling empty slots with \texttt{NOP}s when no instruction fits.
\end{itemize}
In contrast to superscalar processors (which discover ILP dynamically at runtime), \textbf{VLIW processors rely entirely on compile-time scheduling}.
\begin{itemize}
    \item[\textcolor{Green3}{\faIcon{check}}] \textcolor{Green3}{\textbf{Simple hardware}}: no need for out-of-order logic, dependency checks at runtime, or renaming hardware.

    \item[\textcolor{Green3}{\faIcon{check}}] \textcolor{Green3}{\textbf{Predictable performance}}: useful in embedded or real-time systems.

    \item[\textcolor{Green3}{\faIcon{check}}] \textcolor{Green3}{\textbf{Energy efficient}}: avoids complex runtime scheduling logic.

    \item[\textcolor{Red2}{\faIcon{times}}] \textcolor{Red2}{\textbf{Compiler must do all the work}}: requires sophisticated analysis and scheduling. So the performance depends on the quality of the compiler and the amount of visible ILP.
    
    \item[\textcolor{Red2}{\faIcon{times}}] \textcolor{Red2}{\textbf{Binary compatibility issues (limited portability)}}: compiled code often tied to a specific machine configuration (e.g., number of functional units). Cannot adapt to unpredictable latencies at runtime (e.g., cache misses, branch misprediction).
    
    \item[\textcolor{Red2}{\faIcon{times}}] \textcolor{Red2}{\textbf{Wasted instruction slots}}: when insufficient ILP is found, unused slots become \texttt{NOP}s, reducing efficiency.
\end{itemize}

\highspace
\begin{flushleft}
    \textcolor{Green3}{\faIcon{question-circle} \textbf{Superscalar: Dynamic Scheduling}}
\end{flushleft}
In dynamically scheduled architectures:
\begin{itemize}
    \item The compiler generates sequential code (as usual, no additional effort).
    \item The processor's \textbf{hardware detects ILP at runtime}, using structures like reservation stations, reorder buffers, and register renaming.
    \item The processor decides \textbf{which instructions to issue and execute} based on operand availability and resource status.
    \item[\textcolor{Green3}{\faIcon{check}}] Automatically adapts to \textcolor{Green3}{\textbf{unpredictable latencies}} and instruction dependencies.
    \item[\textcolor{Green3}{\faIcon{check}}] \textcolor{Green3}{\textbf{Improved performance portability}}: no need to recompile code for each variant.
    \item[\textcolor{Green3}{\faIcon{check}}] Better at \textcolor{Green3}{\textbf{exploiting ILP in general-purpose programs}}.
    \item[\textcolor{Red2}{\faIcon{times}}] \textcolor{Red2}{\textbf{Higher hardware complexity}}, area, and power consumption.
    \item[\textcolor{Red2}{\faIcon{times}}] \textcolor{Red2}{\textbf{Scheduling logic becomes a bottleneck}} at wider issue widths.
    \item[\textcolor{Red2}{\faIcon{times}}] Greater difficulty in verifying and validating timing behavior.
\end{itemize}

\highspace
\begin{table}[!htp]
    \centering
    \begin{adjustbox}{width={\textwidth},totalheight={\textheight},keepaspectratio}
        \begin{tabular}{@{} l | l | l @{}}
            \toprule
            \textbf{Feature}                & \textbf{Static (VLIW)}        & \textbf{Dynamic (Superscalar)}            \\
            \midrule
            Scheduling responsibility       & Compiler                      & Hardware                                  \\ [.5em]
            Instruction issue               & Fixed and pre-planned         & Determined at runtime                     \\ [.5em]
            Flexibility at runtime          & Low                           & High                                      \\ [.5em]
            Hardware complexity             & Low                           & High                                      \\ [.5em]
            Compiler complexity             & High                          & Moderate                                  \\ [.5em]
            Portability of compiled code    & Low (machine-dependent)       & High                                      \\ [.5em]
            Latency tolerance               & Poor (fixed schedule)         & Good (adaptive execution)                 \\ [.5em]
            ILP exploitation                & Only what compiler exposes    & Also includes dynamic/hidden parallelism  \\
            \bottomrule
        \end{tabular}
    \end{adjustbox}
    \caption{Static vs Dynamic Scheduling.}
\end{table}

\newpage

\noindent
Static and dynamic scheduling represent two fundamentally different approaches to exploiting ILP.
\begin{itemize}
    \item \important{VLIW} is ideal for \textbf{predictable workloads}, embedded systems, or domain specific processors, where simplicity and determinism matter.
    \item \important{Superscalar processors} excel in \textbf{general-purpose computing}, where \textbf{dynamic behavior and runtime variability} make hardware-managed scheduling more effective.
\end{itemize}
Ultimately, both models aim to improve throughput, but the trade-off between hardware complexity and compiler sophistication defines their respective domains of success.