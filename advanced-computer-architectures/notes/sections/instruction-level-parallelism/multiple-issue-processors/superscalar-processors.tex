\subsubsection{Superscalar Processors}\label{subsubsection: Superscalar Processors}

The culmination of the architectural evolution toward higher ILP is the \textbf{superscalar processor}, a class of processors capable of \textbf{issuing, executing, and committing multiple instructions per clock cycle}, \textbf{dynamically and out of order}. Superscalar architectures aim to exploit maximum parallelism hidden within sequential instruction streams, while preserving the illusion of sequential execution.

\highspace
\begin{definitionbox}[: Superscalar Processor]
    A \definition{Superscalar Processor} is a dynamically scheduled, multiple-issue architecture capable of issuing and executing \textbf{several instructions per clock cycle}, using complex hardware mechanisms to \textbf{detect and exploit instruction-level parallelism at runtime}.
\end{definitionbox}

\noindent
Unlike static VLIW processors\footnote{%
    \definition{VLIW (Very Long Instruction Word) Processor}: A type of multiple-issue processor where the compiler statically schedules multiple operations into a single wide instruction word. Each operation in the bundle is executed in parallel, assuming they are independent. VLIW architectures rely on simple hardware and place the burden of dependency checking and scheduling on the compiler rather than the processor.
}, where instruction parallelism is exposed at compile time, \textbf{superscalar designs rely on hardware to discover and schedule parallel instructions on the fly}.

\highspace
\begin{flushleft}
    \textcolor{Green3}{\faIcon{book} \textbf{Key Characteristics}}
\end{flushleft}
\begin{enumerate}\index{Multiple-Issue Width}
    \item \important{Multiple-Issue Width}, also called issue width, is the \textbf{maximum number of instructions} a processor can fetch, decode, issue, and begin executing in a \textbf{single clock cycle}. It defines the \textbf{instruction throughput potential} of a multiple-issue processor.
    \begin{itemize}
        \item Therefore, this value is given by the superscalar processor, but obviously it is the maximum and it cannot be always reached (hazards, etc.).
        \item The maximum number of instructions per single clock cycle \textbf{affects the theoretical IPC limit}:
        \begin{equation*}
            \text{CPI}_{\text{ideal}} = \frac{1}{\text{issue width}}
        \end{equation*}
    \end{itemize}
 
    \item \important{Dynamic Scheduling}
    \begin{itemize}
        \item \textbf{Hardware analyzes instruction dependencies in real time}.
        \item \textbf{Independent instructions} are allowed to proceed \textbf{out of order}.
    \end{itemize}
 
    \item \important{Out-of-Order Execution}
    \begin{itemize}
        \item Instructions \textbf{execute as soon as operands and resources are ready}.
        \item Improves pipeline utilization and hides latencies.
    \end{itemize}
 
    \item \important{In-Order Commit}
    \begin{itemize}
        \item Despite out-of-order execution, the processor updates architectural state in program order.
        \item Ensures \textbf{precise exceptions} and consistent state.
    \end{itemize}
\end{enumerate}

\highspace
\begin{flushleft}
    \textcolor{Green3}{\faIcon{microchip} \textbf{Core Hardware Components}}
\end{flushleft}
To support superscalar execution, the \textbf{architecture must include}:
\begin{itemize}
    \item \textbf{Multiple Functional Units}: ALUs, FPUs, load/store units, enough to support parallel execution.
  
    \item \textbf{Register Renaming}: Eliminates WAR and WAW hazards by mapping architectural registers to a larger set of physical registers.
  
    \item \textbf{Reservation Stations / Issue Queues}: Buffer instructions waiting for operands or functional units.
  
    \item \textbf{Reorder Buffer (ROB)}: Holds results of completed instructions until they are safe to commit. Ensures correct program order and precise exceptions.
  
    \item \textbf{Common Data Bus (CDB)}: Broadcasts results to dependent instructions.
  
    \item \textbf{Instruction Window}: Sliding window of in-flight instructions from\break which the scheduler selects those ready to issue.
\end{itemize}

\highspace
\begin{flushleft}
    \textcolor{Green3}{\faIcon{check-circle} \textbf{Benefits}}
\end{flushleft}
\begin{itemize}[label=\textcolor{Green3}{\faIcon{check-circle}}]
    \item \textcolor{Green3}{\textbf{High ILP}}: Multiple independent instructions can be executed in parallel.
    \item \textcolor{Green3}{\textbf{Dynamic Parallelism}}: No need for compiler to expose ILP, hardware finds it at runtime.
    \item \textcolor{Green3}{\textbf{Latency Hiding}}: Long operations (e.g., cache misses, FP div) can be overlapped with other instructions.
    \item \textcolor{Green3}{\textbf{General Purpose}}: Works well with a wide range of programs, even without manual tuning.
\end{itemize}

\highspace
\begin{flushleft}
    \textcolor{Red2}{\faIcon{times-circle} \textbf{Challenges}}
\end{flushleft}
\begin{itemize}[label=\textcolor{Red2}{\faIcon{times-circle}}]
    \item \textcolor{Red2}{\textbf{Complex Hardware}}: Scheduling, renaming, hazard detection logic increases \textbf{area and power}.
    \item \textcolor{Red2}{\textbf{Issue Logic Scalability}}: The complexity of deciding which $N$ instructions to issue out of $M$ in-flight grows quickly.
    \item \textcolor{Red2}{\textbf{Branch and Memory Dependencies}}: Control and memory dependencies still limit achievable ILP.
    \item \textcolor{Red2}{\textbf{Diminishing Returns}}: After 3-4 issue width, real programs rarely expose enough parallelism to fully utilize the issue bandwidth.
\end{itemize}

\highspace
Superscalar processors represent a powerful solution to the ILP problem, combining \textbf{multiple-issue capability} with \textbf{hardware-driven dynamic scheduling}. They dominate the high-performance general-purpose processor space (e.g., x86 and ARM cores), though their complexity and scalability limitations have motivated complementary techniques such as multithreading, vectorization, and heterogeneous architectures.