\subsubsection{Tomasulo's Algorithm}

Tomasulo's algorithm is a classic dynamic scheduling method that \textbf{naturally incorporates implicit register renaming}. It is designed to \textbf{overlap instruction execution} even when instructions involve true or false dependencies, by dynamically resolving hazards at runtime.

\highspace
In this section, we focus specifically on \textbf{how Tomasulo's algorithm deals with a single loop}, showing the issues of data hazards and motivating why \textbf{renaming}, even if implicit, is critical. Furthermore, we show three techniques, included register renaming, to reach the best optimization. We also show three techniques, including register renaming, to achieve the best optimization.

\highspace
Note that these topics are already covered in the Tomasulo section (\ref{ref: implicit register renaming in Tomasulo algorithm}, page \hqpageref{ref: implicit register renaming in Tomasulo algorithm}).

\highspace
\begin{flushleft}
    \textcolor{Green3}{\faIcon{question-circle} \textbf{Why Renaming?}}
\end{flushleft}
Consider the following floating-point loop:
\begin{lstlisting}[language=unknown, mathescape=true]
Loop:  LD    F0, 0(R1)      ; Load word into F0 from address 0+R1
       MULTD F4, F0, F2     ; Multiply F0 and F2, result into F4
       SD    F4, 0(R1)      ; Store word from F4 to address 0+R1
       SUBI  R1, R1, #8     ; Decrement R1 by 8
       BNEZ  R1, Loop       ; Branch to Loop if R1 $\textcolor{codegreen}{\ne}$ 0
\end{lstlisting}
Each loop iteration consists of five instructions. We assume branch prediction correctly predicts the loop branch as taken. \emph{But are there dependencies?} Yes, \textbf{several hazards arise}:
\begin{enumerate}
    \item RAW (Read After Write) on \texttt{F0} and \texttt{F4}
    \item WAW (Write After Write) on \texttt{F0} and \texttt{F4} across iterations
    \item WAR (Write After Read) on \texttt{R1} due to address updates and branching
\end{enumerate}
These \textbf{hazards limit how aggressively we can overlap loop iterations}. Let's break down the hazard types across consecutive iterations:
\begin{itemize}
    \item \textbf{Within a single iteration}:
    \begin{itemize}
        \item \texttt{MULTD} depends on the result of \texttt{LD} (needs \texttt{F0}).
        \item \texttt{SD} depends on the result of \texttt{MULTD} (needs \texttt{F4}).
        \item \texttt{BNEZ} depends on \texttt{SUBI} to determine the loop condition.
    \end{itemize}
    
    \item \textbf{Across iterations}:
    \begin{itemize}
        \item \texttt{LD} of next iteration \textbf{writes} \texttt{F0}, creating a \textbf{WAW hazard} with the previous \texttt{LD}.
        \item \texttt{MULTD} of next iteration \textbf{writes} \texttt{F4}, also leading to \textbf{WAW hazard}.
        \item \texttt{SUBI} and \texttt{BNEZ} reuse \texttt{R1}, creating \textbf{WAR hazards}.
    \end{itemize}
\end{itemize}
\textbf{Without renaming}, these hazards \textbf{would serialize iterations}, nullifying any gain from speculative execution.

\highspace
\begin{flushleft}
    \textcolor{Green3}{\faIcon{book} \textbf{Tomasulo's Contribution: Implicit Register Renaming}}
\end{flushleft}
Tomasulo's algorithm automatically resolves these hazards by \textbf{renaming registers dynamically}:
\begin{itemize}
    \item It uses \textbf{reservation stations} that hold \textbf{tags} instead of register names.
    \item When an instruction needs a value, it can either:
    \begin{itemize}
        \item Use the value directly if it's ready;
        \item Or wait for a tag associated with the producing reservation station.
    \end{itemize}
    \item WAW and WAR hazards are naturally eliminated because \textbf{register\break names are no longer the synchronization point}, tags are!
\end{itemize}
Thus, \textbf{each iteration of the loop can proceed independently} as soon as its operands are available, even if they involve the same architectural register names.

\highspace
\begin{flushleft}
    \textcolor{Green3}{\faIcon{\speedIcon} \textbf{Loop Unrolling combined with Register Renaming}}
\end{flushleft}
While Tomasulo's algorithm handles implicit renaming at runtime, an alternative strategy to reduce the impact of dependencies, particularly across loop iterations, is to act \textbf{at compile time}, through a technique called \definition{Loop Unrolling} combined with \textbf{Register Renaming}.

\highspace
\definition{Loop Unrolling} is a \textbf{compiler optimization technique} (or sometimes a manual optimization) that \textbf{expands the loop body} by replicating its operations multiple times in sequence, reducing the number of iterations and minimizing the overhead of branch instructions. Instead of executing a small amount of work many times, we ``\textbf{\emph{stretch out}}'' multiple loop iterations into a single, larger block of straight-line code.

\highspace
In our context, Loop Unrolling aims to \textbf{expose more parallelism statically} and to \textbf{reduce the loop overhead} by executing multiple iterations at once. Register Renaming in this context is crucial: \textbf{it eliminates artificial WAW and WAR dependencies} that would otherwise serialize execution.

\highspace
Let's revisit the same loop we discussed before:
\begin{lstlisting}[language=unknown]
Loop:  LD    F0, 0(R1)
       MULTD F4, F0, F2
       SD    F4, 0(R1)
       SUBI  R1, R1, #8
       BNEZ  R1, Loop
\end{lstlisting}
If we simply replicate this loop multiple times without renaming registers, WAW and WAR hazards would prevent full exploitation of the increased instruction stream. \hl{To eliminate such hazards}, \textbf{we must rename the registers} properly \hl{during unrolling}.

\newpage

\noindent
Here's the unrolled version of the loop, unrolled four times, with \textbf{explicit register renaming} to avoid hazards:
\begin{lstlisting}[language=unknown]
Loop:  LD    F0, 0(R1)
       MULTD F4, F0, F2
       SD    F4, 0(R1)

       LD    F6, -8(R1)
       MULTD F8, F6, F2
       SD    F8, -8(R1)

       LD    F10, -16(R1)
       MULTD F12, F10, F2
       SD    F12, -16(R1)

       LD    F14, -24(R1)
       MULTD F16, F14, F2
       SD    F16, -24(R1)

       SUBI  R1, R1, #32
       BNEZ  R1, Loop
\end{lstlisting}
\begin{itemize}
    \item Observations
    \begin{itemize}
        \item New registers are introduced: \texttt{F6}, \texttt{F8}, \texttt{F10}, \texttt{F12}, \texttt{F14}, \texttt{F16}.
        \item \textbf{No two iterations reuse the same destination register}.
        \item \textbf{WAW hazards} between iterations are eliminated.
        \item \textbf{More operations are ready to issue independently}.
        \item Original loop: 5 instructions per iteration. Unrolled loop: 14 instructions for 4 iterations, so 3.5 instructions per iteration on average. Thus, unrolling not only \textbf{exposes parallelism} but also \textbf{increases code density efficiency}.
    \end{itemize}
    \item[\textcolor{Green3}{\faIcon{check-circle}}] \textcolor{Green3}{\textbf{Benefits}}
    \begin{itemize}[label=\textcolor{Green3}{\faIcon{check}}]
        \item \textbf{Higher ILP}: Each iteration's operations are now largely independent.
        \item \textbf{Reduced Loop Overhead}: The branch and counter updates are performed once every 4 iterations, decreasing control dependencies.
        \item \textbf{Efficient Use of Hardware}: Multiple functional units can work simultaneously, better exploiting available execution bandwidth.
        \item \textbf{Preparation for Further Optimizations}: The code can now be \emph{rescheduled} to minimize stalls (we'll see this in future sections).
    \end{itemize}
\end{itemize}
Finally, Tomasulo's algorithm \textbf{does not perform loop unrolling} by itself. Loop Unrolling is a \textbf{compiler optimization}. However, thanks to \textbf{dynamic register renaming and scheduling}, Tomasulo can \textbf{overlap multiple loop iterations even without unrolling}. \hl{This behavior mimics some of benefits of loop unrolling at runtime}, but it doesn't physically unroll the code or replicate instructions.

\newpage

\begin{flushleft}
    \textcolor{Green3}{\faIcon{bolt} \textbf{Even more performance: Code Scheduling with Register Renaming}}
\end{flushleft}
Once we have \textbf{unrolled a loop} and \textbf{renamed registers} to avoid false dependencies, the next logical optimization is \textbf{code scheduling}. The goal is now to \textbf{rearrange the unrolled instructions to minimize stalls}, especially those caused by \textbf{true RAW (Read After Write)} dependencies or \textbf{long execution latencies} (e.g., memory loads or floating-point operations). Thus, \textbf{even with register renaming}, \textbf{the order in which instructions are issued} still matters to achieve \textbf{maximum parallelism}.

\highspace
\definition{Code Scheduling} is the process of \textbf{reordering the instructions} in a program (especially after loop unrolling and register renaming) \textbf{to minimize and maximize parallelism} during execution. It is a \textbf{compiler-level} or sometimes \textbf{hardware-level} optimization. The \hl{goal is to feed the CPU continuously with ready-to-execution instructions}.

\highspace
Let's look at the previous code again, and by unrolling the loop by a factor of 4 and renaming the registers correctly, we had this sequence:
\begin{lstlisting}[language=unknown]
LD    F0, 0(R1)
MULTD F4, F0, F2
SD    F4, 0(R1)
LD    F6, -8(R1)
MULTD F8, F6, F2
SD    F8, -8(R1)
LD    F10, -16(R1)
MULTD F12, F10, F2
SD    F12, -16(R1)
LD    F14, -24(R1)
MULTD F16, F14, F2
SD    F16, -24(R1)
SUBI  R1, R1, #32
BNEZ  R1, Loop
\end{lstlisting}
\textcolor{Red2}{\faIcon{exclamation-triangle} \textbf{Problem}}: if we execute this exactly in order, some instructions will have to wait unnecessarily for their operands because of true dependencies (particularly between \texttt{LD} and \texttt{MULTD}). So the \textbf{key idea} of code scheduling is:
\begin{itemize}
    \item \textbf{Group independent instructions together} as much as possible.
    \item \textbf{Delay dependent instructions} just enough to satisfy their true data dependencies.
    \item \textbf{Hide long latencies} (e.g., waiting for a memory load) by \textbf{doing useful work in parallel}.
\end{itemize}
After careful rescheduling, the reordered code looks like:
\begin{lstlisting}[language=unknown]
LD    F0, 0(R1)
LD    F6, -8(R1)
LD    F10, -16(R1)
LD    F14, -24(R1)
MULTD F4, F0, F2
MULTD F8, F6, F2
MULTD F12, F10, F2
MULTD F16, F14, F2
SD    F4, 0(R1)
SD    F8, -8(R1)
SD    F12, -16(R1)
SUBI  R1, R1, #32
BNEZ  R1, Loop
SD    F16, -24(R1)   ; In branch delay slot
\end{lstlisting}
The changes are:
\begin{itemize}
    \item All \textbf{loads} are moved upfront, memory latency can be hidden while computing.
    \item \textbf{Multiplications} (dependent on loads) are grouped next.
    \item \textbf{Stores} happen \textbf{only after the corresponding multiply is finished}, respecting the RAW dependency.
    \item \texttt{SUBI} and \texttt{BNEZ} are moved right before the branch to reduce branching penalties.
    \item The \textbf{last store} (\texttt{SD F16}) is placed in the \textbf{branch delay slot}, making clever use of a wasted cycle.
\end{itemize}
And the benefits are:
\begin{itemize}[label=\textcolor{Green3}{\faIcon{check}}]
    \item \textcolor{Green3}{\textbf{Load latency hidden}}. Loads start early while compute units are idle.
    \item \textcolor{Green3}{\textbf{Maximal Floating Point unit usage}}. Multiple \texttt{MULTD}s can be issued in parallel.
    \item \textcolor{Green3}{\textbf{Minimal pipeline stalls}}. Operands ready when needed.
    \item \textcolor{Green3}{\textbf{Reduced branch penalties}}. Good use of branch delay slot.
\end{itemize}

\highspace
\begin{table}[!htp]
    \centering
    \begin{adjustbox}{width={\textwidth},totalheight={\textheight},keepaspectratio}
        \begin{tabular}{@{} l p{24em} @{}}
            \toprule
            Technique & How Tomasulo handles it \\
            \midrule
            \textbf{Register Renaming}   & \textcolor{Green3}{\faIcon{check} \textbf{Yes}}, dynamic, using RSs and tags. \\ [.3em]
            \textbf{Code Scheduling}     & \textcolor{Green3}{\faIcon{check} \textbf{Yes}}, dynamic, based on operand availability. \\ [.3em]
            \textbf{Loop Unrolling}      & \textcolor{Red2}{\faIcon{times} \textbf{No}}, but it can dynamically overlap iterations thanks to renaming $+$ scheduling. \\
            \bottomrule
        \end{tabular}
    \end{adjustbox}
    \caption{Does Tomasulo's algorithm adopt Loop Unrolling, Register Renaming, and Code Scheduling?}
\end{table}