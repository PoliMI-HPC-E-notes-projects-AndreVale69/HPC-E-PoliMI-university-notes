\subsubsection{Control Dependencies}

While data and name dependencies arise from how instructions read and write operands, \definition{Control Dependencies} stem from \textbf{the flow of control in the program}, that is, the \textbf{presence of branches and conditional execution}.

\highspace
Control dependencies are fundamentally about \textbf{deciding whether an instruction should execute at all}, based on the \textbf{result of} a preceding \textbf{branch} or \textbf{conditional instruction}.

\highspace
Formally, an instruction $I_{j}$ is \textbf{control-dependent} on a branch instruction $I_{b}$ if:
\begin{itemize}
    \item $I_{j}$ must only execute \textbf{if a particular outcome} of $I_{b}$ is taken.
    \item But the \textbf{decision} made by $I_{b}$ (e.g., whether to branch or not) is \textbf{not yet known} when $I_{j}$ enters the pipeline.
\end{itemize}
This introduces uncertainty: \emph{should $I_{j}$ be fetched and executed, or not?}

\highspace
\begin{examplebox}[: Control Dependencies]
    \begin{lstlisting}[language=c]
if (x > 0)
    A; // Instruction A is control dependent on the condition (x > 0)\end{lstlisting}
    In assembly:
    \begin{lstlisting}[language=riscv, mathescape=true]
I1:  bgtz  r1, LABEL   # branch if r1 > 0
I2:  ...               # instruction before LABEL
I3:  LABEL: A          # instruction A\end{lstlisting}
    \begin{itemize}
        \item \texttt{A} should only execute \textbf{if the branch is taken}.
        \item But we don't know whether the branch is taken until the condition is resolved, which happens later in the pipeline.
    \end{itemize}
\end{examplebox}

\highspace
\begin{flushleft}
    \textcolor{Red2}{\faIcon{question-circle} \textbf{Why Control Dependencies matter for ILP}}
\end{flushleft}
Control dependencies \textbf{limit instruction parallelism}:
\begin{itemize}
    \item We cannot freely reorder or speculate on the instructions following a branch.
    \item Waiting for the branch outcome introduces \textbf{stalls} in the pipeline.
\end{itemize}
Thus, exploiting ILP requires \textbf{breaking or relaxing control dependencies}, without violating program semantics.

\highspace
\begin{flushleft}
    \textcolor{Green3}{\faIcon{check-circle} \textbf{Control Dependencies Solution}}
\end{flushleft}
We have dedicated an entire section to this topic, see \ref{section: Control Hazards and Branch Prediction}, page \pageref{section: Control Hazards and Branch Prediction}.