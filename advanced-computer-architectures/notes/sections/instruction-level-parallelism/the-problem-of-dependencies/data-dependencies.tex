\subsubsection{Data Dependencies}

\definition{Data Dependencies}, also called \definition{True Data Dependencies}, are the most fundamental type of instruction dependencies in a program. They \textbf{express the real flow of data from one instruction to another} and are dictated by the \emph{semantics} of the program. These \textbf{dependencies must be strictly preserved} during any reordering or parallel execution of instructions, \textbf{otherwise the correctness of the program is compromised}.

\highspace
Formally, we say there is a data dependencies from instruction $I_{i}$ to instruction $I_{j}$ (where $I_{j}$ follows $I_{i}$ in program order), \textbf{if $I_{j}$ reads a value that is produced by $I_{i}$}. In other words, $I_{j}$ needs the output of $I_{i}$ as its input. This is known as a \definition{Read After Write (RAW)} hazard in pipeline terminology.

\highspace
\begin{flushleft}
    \textcolor{Green3}{\faIcon{question-circle} \textbf{Why is it called ``\emph{true}''?}}
\end{flushleft}
This type of dependence is ``\emph{true}'' because the \textbf{second instruction cannot proceed correctly until the first one completes its write operation}. It reflects an \hl{actual requirement for program correctness}.

\highspace
\begin{examplebox}[: RAW Hazard]
    \begin{lstlisting}[language=riscv, mathescape=true]
I1:   r3 $\leftarrow$ r1 + r2   # produces a value in r3
I2:   r4 $\leftarrow$ r3 + r5   # consumes the value from r3\end{lstlisting}
    Here, \texttt{I2} is data-dependent on \texttt{I1} because it reads from register \texttt{r3}, which is written by \texttt{I1}. The instructions must execute in order:
    \begin{itemize}
        \item \texttt{I1} must execute and complete its write to \texttt{r3} before.
        \item \textsl{I2} reads \texttt{r3} to perform its own computation.
    \end{itemize}
    If this order is violated, e.g., \texttt{I2} executes before \texttt{I1} finishes, then \texttt{I2} will read an incorrect or undefined value.
\end{examplebox}

\highspace
\begin{flushleft}
    \textcolor{Red2}{\faIcon{exclamation-triangle} \textbf{Why Data dependencies Matter for ILP}}
\end{flushleft}
Data dependencies define which \textbf{instructions must not be executed in parallel}, because doing so would result in violating program semantics.
\begin{itemize}
    \item In a \textbf{pipelined processor}, data dependencies may cause \textbf{pipeline stalls}.
    \item In \textbf{out-of-order processors}, special mechanisms (like reservation stations and the reorder buffer) track and resolve data dependencies to allow other independent instructions to proceed while dependent ones wait for operands.
\end{itemize}