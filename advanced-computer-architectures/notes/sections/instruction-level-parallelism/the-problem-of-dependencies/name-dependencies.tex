\subsubsection{Name Dependencies}\label{subsubsection: Name Dependencies}

Unlike true data dependencies, \definition{Name Dependencies} arise when \textbf{two instructions use the same register or memory location}, but \textbf{there is no actual flow of data between them}. These dependencies are called \definition{False Dependencies} or \definition{Pseudo-Dependencies} because they are \textbf{not required} for program correctness from a data perspective, \textbf{but still impose constraints} on instruction scheduling.

\highspace
These constraints are due to the reuse of names (i.e., identifiers like register names), not due to real dependencies in the data values. They may still cause hazards in a pipeline and need to be addressed, especially when trying to execute instructions in parallel or out of order.

\highspace
\begin{flushleft}
    \textcolor{Green3}{\faIcon{stream} \textbf{Types of Name Dependencies}}
\end{flushleft}
There are two main types:
\begin{itemize}
    \item \definition{Anti-Dependence (Write After Read - WAR)}. An \textbf{anti-dependence} occurs when:
    \begin{itemize}
        \item A first instruction \textbf{reads} from a location (register/memory);
        \item A second instruction \textbf{writes} to that same location, after it.
    \end{itemize}
    This introduces a WAR hazard: if the \textbf{second instruction is executed too early} (before the first instruction finishes reading), \textbf{it might overwrite the value before the first instruction uses it}.

    \begin{examplebox}[: WAR Hazard]
        \begin{lstlisting}[language=riscv, mathescape=true]
I1:   r3 $\leftarrow$ r1 + r2   # reads r1 and r2
I2:   r1 $\leftarrow$ r4 + r5   # writes to r1\end{lstlisting}
        In this case:
        \begin{itemize}
            \item \texttt{I1} reads from \texttt{r1}
            \item \texttt{I2} writes to \texttt{r1}
        \end{itemize}
        There is no data flow between the two (i.e., \texttt{I1} doesn't use the result of \texttt{I2}, and vice versa), but \hl{if \texttt{I2} executes before \texttt{I1} finishes}, the read in \textbf{I1} \hl{may get a corrupted value}.
    \end{examplebox}


    \item \definition{Output Dependence (Write After Write - WAW)}. An \textbf{output dependence} occurs when \textbf{two instructions write to the same location} (register or memory). This results in a WAW hazard: executing the \textbf{second instruction first may overwrite the location}, changing the final value from what the program originally intended.

    \begin{examplebox}[: WAW Hazard]
        \begin{lstlisting}[language=riscv, mathescape=true]
I1:   r3 $\leftarrow$ r1 + r2   # writes to r3
I2:   r3 $\leftarrow$ r4 + r5   # writes to r3\end{lstlisting}
        There's no direct data flow between the two, but the \textbf{ordering matters}. If \texttt{I2} is supposed to overwrite \texttt{r3} after \texttt{I1}, reversing the order would result in \texttt{I1}'s result being incorrectly seen as the final value in \texttt{r3}.
    \end{examplebox}
\end{itemize}

\highspace
\begin{flushleft}
    \textcolor{Green3}{\faIcon{check-circle} \textbf{Resolving Name Dependencies: Register Renaming}}
    \label{def: Register Renaming}
\end{flushleft}
The key idea in dealing with name dependencies is to \textbf{recognize that they are not real} and can be \textbf{eliminated if we avoid the reuse of names}. The technique used to \hl{eliminate these artificial constraints} is called \definition{Register Renaming}. The idea is simple:
\begin{itemize}
    \item If two instructions refer to the same register but don't actually share data, assign them \textbf{different physical registers}.
\end{itemize}
This is only possible when the underlying hardware (or compiler) provides more physical registers than the number of logical registers visible in the ISA.

\highspace
\begin{examplebox}[: Resolving WAR and WAW]
    Original code (WAR):
    \begin{lstlisting}[language=riscv, mathescape=true]
I1:   r3 $\leftarrow$ r1 + r2
I2:   r1 $\leftarrow$ r4 + r5\end{lstlisting}
    Renamed:
    \begin{lstlisting}[language=riscv, mathescape=true]
I1:   r3 $\leftarrow$ r1 + r2
I2:   r9 $\leftarrow$ r4 + r5    # write to r9 instead of r1\end{lstlisting}
    Now, there is no conflict, \texttt{I2} can proceed independently of \texttt{I1}.

    Original code (WAW):
    \begin{lstlisting}[language=riscv, mathescape=true]
I1:   r3 $\leftarrow$ r1 + r2
I2:   r3 $\leftarrow$ r4 + r5\end{lstlisting}
    Renamed:
    \begin{lstlisting}[language=riscv, mathescape=true]
I1:   r3 $\leftarrow$ r1 + r2
I2:   r9 $\leftarrow$ r4 + r5    # write to a new register\end{lstlisting}
\end{examplebox}

\newpage

\begin{flushleft}
    \textcolor{Green3}{\faIcon{balance-scale} \textbf{Hardware vs. Software Register Renaming}}
\end{flushleft}
\begin{itemize}
    \item \definitionWithSpecificIndex{Hardware (Dynamic Renaming)}{Dynamic Renaming - Hardware-side}{}. \textbf{Performed at runtime} by structures such as the Register Alias Table (RAT), typically in out-of-order superscalar processors.
    \begin{itemize}
        \item[\textcolor{Green3}{\faIcon{check-circle}}] Flexible
        \item[\textcolor{Red2}{\faIcon{times-circle}}] Adds hardware complexity
    \end{itemize}
    
    
    \item \definitionWithSpecificIndex{Software (Static Renaming)}{Static Renaming - Software-side}{}. \textbf{Performed at compile time} by the compiler, particularly for VLIW or statically scheduled processors.
    \begin{itemize}
        \item[\textcolor{Green3}{\faIcon{check-circle}}] Simpler in hardware
        \item[\textcolor{Red2}{\faIcon{times-circle}}] Puts more pressure on compiler technology
    \end{itemize}
\end{itemize}