\section{Instruction Level Parallelism}

\subsection{The problem of dependencies}\label{subsection: The problem of dependencies}

\definition{Instruction-Level Parallelism (ILP)} is a foundational concept in modern processor design that aims to \textbf{improve performance by executing multiple instructions simultaneously within a single processor core}. The fundamental \textbf{premise of ILP is that many instructions within a program can be executed independently}, and thus, \textbf{can be overlapped in time}. This section explores the principles of pipelining as a means to exploit ILP, emphasizing its benefits, ideal performance metrics, and the inherent limitations.

\highspace
However, \textbf{instruction dependencies} represent critical \hl{constraints} on this parallelism. Understanding these dependencies is essential for analyzing the potential parallelism in a program and for designing hardware or compilers that can exploit ILP safely and efficiently.

\highspace
\textbf{Instruction dependencies determine which instructions can be executed simultaneously} and which ones must respect a specific order of execution. These dependencies are classified into three broad categories: \textbf{data dependencies}, \textbf{name dependencies}, and \textbf{control dependencies}.

\highspace
\begin{flushleft}
    \textcolor{Green3}{\faIcon{check-circle} \textbf{Correct Program Behavior}}
\end{flushleft}
For correct program behavior, \textbf{two program properties must always be preserved} during instruction scheduling:
\begin{enumerate}
    \item \important{Data Flow}. The correct \textbf{values} must be \textbf{produced and consumed in the proper order}.
    \item \important{Exception Behavior}. Reordering must \textbf{not alter the way exceptions are raised and handled} in the program.
\end{enumerate}
While \textbf{dependencies are intrinsic to the program semantics}, \textbf{hazards are an architectural artifact of the pipeline implementation}.