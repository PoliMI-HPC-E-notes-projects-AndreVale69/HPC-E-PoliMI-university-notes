\subsection{Tomasulo's Algorithm}\label{subsection: tomasulo algorithm}

\subsubsection{Introduction}

\definition{Tomasulo's Algorithm} represents a pivotal innovation in the domain of dynamic scheduling and out-of-order execution within high-performance computing. Developed in 1967 at IBM for the \href{https://en.wikipedia.org/wiki/IBM_System/360_Model_91}{IBM System/360 Model 91}, it was introduced as a means to exploit \textbf{Instruction-Level Parallelism (ILP)} in the absence of compiler support or source-level reordering. The essential goal was to \textbf{overcome pipeline stalls} due to data hazards, particularly \textbf{Write After Write (WAW)} and \textbf{Write After Read (WAR)} hazards, both of which are challenging to resolve through simple pipeline control mechanisms.

\highspace
\begin{flushleft}
    \textcolor{Green3}{\faIcon{heartbeat} \textbf{Core Idea}}
\end{flushleft}
Tomasulo's algorithm enables \textbf{instructions to execute out of program order}, yet maintains \textbf{data correctness} via hardware-level mechanisms. Central to this approach is the concept of \textbf{implicit register renaming}, which dynamically assigns storage locations (reservation stations) to values rather than using architectural register names directly. This mechanisms ensures that no two instructions mistakenly read or overwrite the same register unless there is a true data dependency (RAW, Read After Write).

\highspace
\begin{flushleft}
    \textcolor{Green3}{\faIcon{book} \textbf{New features introduced: a kind of Dynamic Scheduling 2.0}}
\end{flushleft}
\begin{itemize}
    \item \important{Implicit Register Renaming}: avoids WAR and WAW hazards by assigning intermediate results to reservation stations rather than architecture registers.
    \item \important{Dynamic Scheduling}: unlike static instruction scheduling (done at compile time), Tomasulo's algorithm uses runtime analysis to decide the order of instruction execution (yes, like the Scoreboard algorithm, but smarter).
    \item \important{Out-of-Order Execution}: instructions can be issued and begin execution as soon as operands are ready, independently of program order, provided that dependencies are resolved.
    \item \important{Common Data Bus (CDB)}: results are broadcast to all units waiting for them, further enabling parallelism.
\end{itemize}

\highspace
\begin{flushleft}
    \textcolor{Green3}{\faIcon{landmark} \textbf{Historical Significance}}
\end{flushleft}
Tomasulo's work appeared just three years after the CDC 6600 Scoreboarding mechanism (Seymour Cray's design), which was the first dynamic scheduling mechanisms. Unlike the scoreboard, Tomasulo's algorithm \textbf{distributes control} among the \textbf{functional units} via \textbf{reservation stations}, offering more scalable and parallel data communication through the CDB.

\newpage

\noindent
It served as the architectural blueprint for later processors such as:
\begin{multicols}{2}
    \begin{itemize}
        \item \href{https://en.wikipedia.org/wiki/Alpha_21264}{Alpha 21264}
        \begin{center}
            \qrcode{https://en.wikipedia.org/wiki/Alpha_21264}
        \end{center}
        \item \href{https://en.wikipedia.org/wiki/PA-8000}{HP PA-8000}
        \begin{center}
            \qrcode{https://en.wikipedia.org/wiki/PA-8000}
        \end{center}
        \item \href{https://en.wikipedia.org/wiki/R10000}{MIPS R10000}
        \begin{center}
            \qrcode{https://en.wikipedia.org/wiki/R10000}
        \end{center}
        \item \href{https://en.wikipedia.org/wiki/Pentium_II}{Intel Pentium II}
        \begin{center}
            \qrcode{https://en.wikipedia.org/wiki/Pentium_II}
        \end{center}
        \item \href{https://en.wikipedia.org/wiki/PowerPC_600#PowerPC_604}{PowerPC 604}
        \begin{center}
            \qrcode{https://en.wikipedia.org/wiki/PowerPC_600#PowerPC_604}
        \end{center}
    \end{itemize}
\end{multicols}

\highspace
\begin{flushleft}
    \textcolor{Green3}{\faIcon{balance-scale} \textbf{Tomasulo vs. Scoreboarding: What's the difference?}}
\end{flushleft}
Both \textbf{Tomasulo's Algorithm} and the \textbf{Scoreboard Algorithm} are techniques for \textbf{dynamic instruction scheduling}, meaning they allow instructions to be executed \textbf{out of program order} while still preserving correctness. But Tomasulo goes a step further in terms of efficiency and cleverness:

\begin{table}[!htp]
    \centering
    \begin{adjustbox}{width={\textwidth},totalheight={\textheight},keepaspectratio}
        \begin{tabular}{@{} l | l | l @{}}
            \toprule
            Feature                             & Scoreboard (CDC 6600)                                                 & Tomasulo (IBM 360/91) \\
            \midrule
            \textbf{Register Renaming}          & \textcolor{Red2}{\faIcon{times} \textbf{No}}                          & \textcolor{Green3}{\faIcon{check} \textbf{Yes} (implicit via reservation stations)}   \\ [.5em]
            \textbf{WAR/WAW Hazards Handling}   & \textcolor{Red2}{\faIcon{times} \textbf{Needs to stall}}              & \textcolor{Green3}{\faIcon{check} \textbf{Avoided by renaming}}                       \\ [.5em]
            \textbf{Data Communication}         & \textcolor{Green3}{\faIcon{check} \textbf{Writes to Register File}}   & \textcolor{Green3}{\faIcon{check} \textbf{Used CDB to forward directly}}              \\ [.5em]
            \textbf{Control}                    & Centralized scoreboard                                                & Distributed (each FU has its own RS)                                                  \\ [.5em]
            \textbf{Execution Start Condition}  & Wait for operand \emph{registers}                                     & Wait for operand \emph{values} or \emph{tags}                                         \\
            \bottomrule
        \end{tabular}
    \end{adjustbox}
    \caption{Tomasulo vs. Scoreboarding.}
\end{table}