\subsubsection{Register Renaming}\label{subsubsection: Register Renaming}

The Register Renaming techniques have already been discussed in the previous sections. In section \ref{def: Register Renaming} (page \pageref{def: Register Renaming}) we gave a very short presentation dedicated to understand how to solve name dependencies, and in section \ref{subsubsection: static vs implicit register renaming} (page \pageref{subsubsection: static vs implicit register renaming}) we have seen what is the difference between static register renaming and implicit register renaming. In this section we will go deeper and understand how to apply this technique to the tomasulo algorithm.

\paragraph{Introduction}

In modern out-of-order execution processors, one of the most significant \textbf{bottlenecks} to exploiting Instruction-Level Parallelism (ILP) lies in \textbf{false data dependencies}. Among these, \textbf{WAR (Write After Read)} and \textbf{WAW (Write After Write)} hazards can severely restrict the freedom to reorder instructions during execution. These are not \textbf{true dependencies} (like RAW, Read After Write), but are constraints imposed by the \textbf{limited number of registers} visible in the Instruction Set Architecture (ISA).

\highspace
To tackle these issues, \definition{Register Renaming} is employed, a mechanism that dynamically or statically \textbf{replaces architectural registers with a larger set of physical registers}, breaking these false dependencies and enabling more parallel execution.

\highspace
\begin{flushleft}
    \textcolor{Green3}{\faIcon{stream} \textbf{Types of Register Renaming}}
\end{flushleft}
There are \textbf{two primary approaches} to register renaming:
\begin{itemize}
    \item \important{Static Register Renaming}, handled at \textbf{compile-time}, often via techniques like \textbf{loop unrolling} and aggressive register allocation. This relies on \textbf{ISA-visible registers} and requires the compiler to explicitly assign new register names to avoid reuse conflicts.

    \item \important{Implicit (Dynamic) Register Renaming}, which is employed in hardware at \textbf{runtime}, often as part of advanced scheduling mechanisms such as \textbf{Tomasulo's algorithm}. This method uses structures like \textbf{reservation stations} and \textbf{reorder buffers} to rename registers implicitly and track dependencies through tags instead of register names.
\end{itemize}

\highspace
\begin{flushleft}
    \textcolor{Green3}{\faIcon{question-circle} \textbf{Why Register Renaming is important?}}
\end{flushleft}
Without register renaming, WAR and WAW hazards can stall instruction issue unnecessarily:
\begin{itemize}
    \item A WAR hazard arises when an instruction wants to write to a register that a previous instruction still needs to read.
    \item A WAW hazard occurs when two instructions write to the same register, and their order must be preserved to maintain correctness.
\end{itemize}
By decoupling the programmer-visible registers from the physical storage used in the microarchitecture, renaming allows:
\begin{itemize}[label=\textcolor{Green3}{\faIcon{check}}]
    \item \textbf{Multiple instructions} writing to the ``same'' architectural register to execute out of order.
    \item \textbf{Concurrent execution} of independent iterations of loops.
    \item \textbf{Elimination of false dependencies}, improving pipeline utilization and throughput.
\end{itemize}

\highspace
Register renaming is a fundamental enabler of modern ILP exploitation. Whether implemented statically via compiler transformations or dynamically through hardware mechanisms like Tomasulo's algorithm, it allows us to \textbf{overlap instruction execution} aggressively and avoid performance penalties due to naming limitations of the architectural register file.