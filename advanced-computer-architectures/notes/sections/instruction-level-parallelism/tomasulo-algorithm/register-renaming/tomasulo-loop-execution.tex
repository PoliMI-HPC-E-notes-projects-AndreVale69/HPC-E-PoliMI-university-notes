\paragraph{Tomasulo Loop Execution}

We now analyze a \textbf{concrete example} of how Tomasulo's algorithm \textbf{dynamically overlaps multiple loops iterations}, using \textbf{register renaming} and \textbf{out-of-order execution}.

\highspace
The loop we consider is:
\begin{lstlisting}[language=unknown]
Loop:
        LD    F0, 0(R1)
        MULTD F4, F0, F2
        SD    F4, 0(R1)
        SUBI  R1, R1, #8
        BNEZ  R1, Loop
\end{lstlisting}

\highspace
\begin{flushleft}
    \textcolor{Green3}{\faIcon{tools} \textbf{Setup Assumptions}}
\end{flushleft}
\begin{itemize}
    \item \textbf{Cache miss} on the \textbf{first} \texttt{LD} $\rightarrow$ 8 clock cycles.
    \item \textbf{Cache hit} on the \textbf{second} \texttt{LD} $\rightarrow$ 1 clock cycles.
    \item \texttt{FP MULTD} operation \textbf{latency} $\rightarrow$ 4 clock cycles.
    \item \textbf{Branch Prediction} $\rightarrow$ branch assumed \textbf{taken} (so iterations continue).
    \item Only \texttt{SUBI} and \texttt{BNEZ} clock times are shown explicitly for clarity.
    \item 5 instructions per loop iteration.
\end{itemize}
Thus, \textbf{memory accesses are slow}, and \textbf{floating-point operations are moderately slow} compared to integer operations.

\highspace
\begin{flushleft}
    \textcolor{Red2}{\faIcon{exclamation-triangle} \textbf{Hazards}}
\end{flushleft}
\begin{itemize}
    \item[\textcolor{Red2}{\faIcon{times}}] \textbf{RAW hazards} on \texttt{F0}, \texttt{F4}, \texttt{R1}.
    \item[\textcolor{Green3}{\faIcon{check}}] (\textcolor{Green3}{\textbf{eliminated dynamically}}, thanks to Tomasulo's implicit register renaming) \textbf{WAW hazards} mainly on \texttt{R1}.
    \item[\textcolor{Green3}{\faIcon{check}}] (\textcolor{Green3}{\textbf{eliminated dynamically}}, thanks to Tomasulo's implicit register renaming) \textbf{WAW hazards} on \texttt{F0} and \texttt{F4} across iterations.
\end{itemize}

\highspace
\begin{flushleft}
    \textcolor{Green3}{\faIcon{book} \textbf{Structures in Tomasulo's algorithm}}
\end{flushleft}
\begin{itemize}
    \item \textbf{Instruction Status Table} (or Instruction Queue, or Tracking Table), tracks the status of each issued instruction through its three major stages: issue, execute and write result. Allow us to monitor the instruction flow cycle-by-cycle.
    
    \item \textbf{Reservation Stations Table}, temporarily hold instructions waiting to execute, \textbf{along with operand values or operand tags} if values are not yet available. Performs dynamic register renaming and resolve dependencies internally before issuing instructions to functional units.

    \item \textbf{Register Result Status Table}, tells \textbf{which Reservation Station (RS) is responsible for generating the current value of each architectural register}. Allows forwarding and prevents WAR/WAW hazards, by making reads wait for the right instruction to finish.
    
    \item \textbf{Load and Store Buffers (Memory Buffers)}, manage memory operations separately, since memory addresses take time to compute and loads/stores have different latencies. Supports out-of-order memory access while ensuring correct memory ordering when necessary.
\end{itemize}

\begin{enumerate}
    %
    % Cycle 1
    %
    \item \textbf{Cycle \theenumi}
    \begin{itemize}
        \item The \textbf{first instruction} from iteration 1, \texttt{LD F0, 0(R1)}, is issued. It is assigned to Load Buffer 1 (\texttt{Load1}), marked as busy. The memory address to load from is calculated: \texttt{R1 = 80}, so the load address is \texttt{80}.
        \item \texttt{F0} is now associated with \texttt{Load1} in the register status table. This means:
        \begin{itemize}
            \item Any instruction that wants to read \texttt{F0} must wait for \texttt{Load1} to complete and broadcast its result.
            \item Tomasulo does not write to \texttt{F0} immediately, tag \texttt{Load1} is what matters now.
        \end{itemize}
        \item The other instructions (\texttt{MULTD}, \texttt{SD}, etc.) are \textbf{not issued yet} in this cycle. They are waiting because their source operands depend on \texttt{F0} (the load result), which is \textbf{not ready} yet. Only the first load is active at this point.
    \end{itemize}
    At cycle \theenumi, we \textbf{prepare the first data} to flow through Tomasulo's architecture: we \textbf{rename} \texttt{F0} dynamically, \textbf{assign} it to a load buffer, and \textbf{start the memory access} (which will take 8 cycles because of cache miss). All following instructions \textbf{must track this tag} (\texttt{Load1}) to know when \texttt{F0} will be ready.
    \begin{table}[!htp]
        \centering
        \begin{tabular}{@{} c l | c c c @{}}
            \toprule
            Iter.       & Instruction                   & Issue & Exec Comp & Write Res \\
            \midrule
            1           & \texttt{LD    F0, 0, R1}      & \hl{1}&           &           \\ [.3em]
            1           & \texttt{MULTD F4, F0, F2}     &       &           &           \\ [.3em]
            1           & \texttt{SD    F4, 0(R1)}      &       &           &           \\ [.3em]
            2           & \texttt{LD    F0, 0, R1}      &       &           &           \\ [.3em]
            2           & \texttt{MULTD F4, F0, F2}     &       &           &           \\ [.3em]
            2           & \texttt{SD    F4, 0(R1)}      &       &           &           \\
            \bottomrule
        \end{tabular}
        \caption*{Instruction status.}
    \end{table}

    \newpage

    \begin{table}[!htp]
        \centering
        \begin{tabular}{@{} l | c c c @{}}
            \toprule
                                & Busy  & Addr   & FU   \\
            \midrule
            \texttt{Load1}      & \hl{Yes}  & \hl{80}   & \textbackslash    \\ [.3em]
            \texttt{Load2}      & No        &           & \textbackslash    \\ [.3em]
            \texttt{Load3}      & No        &           & \textbackslash    \\
            \cmidrule{1-4}
            \texttt{Store1}     & No        &           &                   \\ [.3em]
            \texttt{Store2}     & No        &           &                   \\ [.3em]
            \texttt{Store3}     & No        &           &                   \\
            \bottomrule
        \end{tabular}
        \caption*{Load and Store Buffers.}
    \end{table}

    \begin{table}[!htp]
        \centering
        \begin{tabular}{@{} l l | c c c c c c @{}}
            \toprule
            Time        & Name              & Busy      & \texttt{Op}       & \texttt{Vj}       & \texttt{Vk}       & \texttt{Qj}       & \texttt{Qk}       \\
            \midrule
                        & \texttt{Add1}     & No        &                   &                   &                   &                   &                   \\ [.3em]
                        & \texttt{Add2}     & No        &                   &                   &                   &                   &                   \\ [.3em]
                        & \texttt{Add3}     & No        &                   &                   &                   &                   &                   \\ [.3em]
                        & \texttt{Mult1}    & No        &                   &                   &                   &                   &                   \\ [.3em]
                        & \texttt{Mult2}    & No        &                   &                   &                   &                   &                   \\
            \bottomrule
        \end{tabular}
        \caption*{Reservation Stations.}
    \end{table}

    \begin{table}[!htp]
        \centering
        \begin{tabular}{@{} c | c | c c c c c c c | c | c @{}}
            \toprule
            Clock       & \texttt{R1}       & \texttt{F0}           & \texttt{F2}   & \texttt{F4}   & \texttt{F6}       & \texttt{F8}   & \texttt{F10}  & \texttt{F12}  & $\dots$   & \texttt{F30}  \\
            \midrule
            \theenumi   & \hl{80}           & \hl{\texttt{Load1}}   &               &               &                   &               &               &               &           &               \\
            \bottomrule
        \end{tabular}
        \caption*{Register result status.}
    \end{table}

    \newpage


    %%%%%%%%%%%%%%%%%%%%%%%%%%%%%%%%%
    % Cycle 2
    %%%%%%%%%%%%%%%%%%%%%%%%%%%%%%%%%
    \item \textbf{Cycle \theenumi}
    \begin{itemize}
        \item Instruction \texttt{MULTD F4, F0, F2} (from iteration 1) is \textbf{issued}. It is assigned to \texttt{Mult1} (a multiplication Reservation Station). Operands:
        \begin{itemize}
            \item \texttt{F2} is ready, its value is available $\rightarrow$ \texttt{Vj = R(F2)}
            \item \texttt{F0} is \emph{not} ready, \textbf{waiting on} \texttt{Load1} to produce it $\rightarrow$ \texttt{Qj = Load1}.
        \end{itemize}
        
        \item \texttt{F0} is still pending, being loaded by \texttt{Load1}. \texttt{F4} will eventually be produced by the \texttt{MULTD} operation (currently waiting for \texttt{F0} to complete).
    \end{itemize}
    \begin{table}[!htp]
        \centering
        \begin{tabular}{@{} c l | c c c @{}}
            \toprule
            Iter.       & Instruction                   & Issue & Exec Comp & Write Res \\
            \midrule
            1           & \texttt{LD    F0, 0, R1}      & 1     &           &           \\ [.3em]
            1           & \texttt{MULTD F4, F0, F2}     & \hl{2}&           &           \\ [.3em]
            1           & \texttt{SD    F4, 0(R1)}      &       &           &           \\ [.3em]
            2           & \texttt{LD    F0, 0, R1}      &       &           &           \\ [.3em]
            2           & \texttt{MULTD F4, F0, F2}     &       &           &           \\ [.3em]
            2           & \texttt{SD    F4, 0(R1)}      &       &           &           \\
            \bottomrule
        \end{tabular}
        \caption*{Instruction status.}
    \end{table}

    \begin{table}[!htp]
        \centering
        \begin{tabular}{@{} l | c c c @{}}
            \toprule
                                & Busy      & Addr      & FU   \\
            \midrule
            \texttt{Load1}      & Yes       & 80        & \textbackslash    \\ [.3em]
            \texttt{Load2}      & No        &           & \textbackslash    \\ [.3em]
            \texttt{Load3}      & No        &           & \textbackslash    \\
            \cmidrule{1-4}
            \texttt{Store1}     & No        &           &                   \\ [.3em]
            \texttt{Store2}     & No        &           &                   \\ [.3em]
            \texttt{Store3}     & No        &           &                   \\
            \bottomrule
        \end{tabular}
        \caption*{Load and Store Buffers.}
    \end{table}

    \begin{table}[!htp]
        \centering
        \begin{tabular}{@{} l l | c c c c c c @{}}
            \toprule
            Time        & Name              & Busy      & \texttt{Op}           & \texttt{Vj}       & \texttt{Vk}           & \texttt{Qj}           & \texttt{Qk}       \\
            \midrule
                        & \texttt{Add1}     & No        &                       &                   &                       &                       &                   \\ [.3em]
                        & \texttt{Add2}     & No        &                       &                   &                       &                       &                   \\ [.3em]
                        & \texttt{Add3}     & No        &                       &                   &                       &                       &                   \\ [.3em]
                        & \texttt{Mult1}    & \hl{Yes}  & \hl{\texttt{MULTD}}   &                   & \hl{\texttt{R(F2)}}   & \hl{\texttt{Load1}}   &                   \\ [.3em]
                        & \texttt{Mult2}    & No        &                       &                   &                       &                       &                   \\
            \bottomrule
        \end{tabular}
        \caption*{Reservation Stations.}
    \end{table}

    \newpage

    \begin{table}[!htp]
        \centering
        \begin{tabular}{@{} c | c | c c c c c c c | c | c @{}}
            \toprule
            Clock       & \texttt{R1}       & \texttt{F0}           & \texttt{F2}   & \texttt{F4}           & \texttt{F6}       & \texttt{F8}   & \texttt{F10}  & \texttt{F12}  & $\dots$   & \texttt{F30}  \\
            \midrule
            \theenumi   & 80                & \texttt{Load1}        &               & \hl{\texttt{Mult1}}   &                   &               &               &               &           &               \\
            \bottomrule
        \end{tabular}
        \caption*{Register result status.}
    \end{table}

    \newpage


    %%%%%%%%%%%%%%%%%%%%%%%%%%%%%%%%%
    % Cycle 3
    %%%%%%%%%%%%%%%%%%%%%%%%%%%%%%%%%
    \item \textbf{Cycle \theenumi}
    \begin{itemize}
        \item Instruction \texttt{SD F4, 0(R1)} (store) from iteration 1 is \textbf{issued}. It is assigned to \texttt{Store1} (store buffer). The store buffer must wait for \texttt{F4} to be computed by \texttt{Mult1}. We record the tag of the functional unit that will produce the needed value \texttt{F4}.
    \end{itemize}
    \begin{table}[!htp]
        \centering
        \begin{tabular}{@{} c l | c c c @{}}
            \toprule
            Iter.       & Instruction                   & Issue & Exec Comp & Write Res \\
            \midrule
            1           & \texttt{LD    F0, 0, R1}      & 1     &           &           \\ [.3em]
            1           & \texttt{MULTD F4, F0, F2}     & 2     &           &           \\ [.3em]
            1           & \texttt{SD    F4, 0(R1)}      & \hl{3}&           &           \\ [.3em]
            2           & \texttt{LD    F0, 0, R1}      &       &           &           \\ [.3em]
            2           & \texttt{MULTD F4, F0, F2}     &       &           &           \\ [.3em]
            2           & \texttt{SD    F4, 0(R1)}      &       &           &           \\
            \bottomrule
        \end{tabular}
        \caption*{Instruction status.}
    \end{table}

    \begin{table}[!htp]
        \centering
        \begin{tabular}{@{} l | c c c @{}}
            \toprule
                                & Busy      & Addr      & FU   \\
            \midrule
            \texttt{Load1}      & Yes       & 80        & \textbackslash        \\ [.3em]
            \texttt{Load2}      & No        &           & \textbackslash        \\ [.3em]
            \texttt{Load3}      & No        &           & \textbackslash        \\
            \cmidrule{1-4}
            \texttt{Store1}     & \hl{Yes}  & \hl{80}   & \hl{\texttt{Mult1}}   \\ [.3em]
            \texttt{Store2}     & No        &           &                       \\ [.3em]
            \texttt{Store3}     & No        &           &                       \\
            \bottomrule
        \end{tabular}
        \caption*{Load and Store Buffers.}
    \end{table}

    \begin{table}[!htp]
        \centering
        \begin{tabular}{@{} l l | c c c c c c @{}}
            \toprule
            Time        & Name              & Busy      & \texttt{Op}           & \texttt{Vj}       & \texttt{Vk}           & \texttt{Qj}           & \texttt{Qk}       \\
            \midrule
                        & \texttt{Add1}     & No        &                       &                   &                       &                       &                   \\ [.3em]
                        & \texttt{Add2}     & No        &                       &                   &                       &                       &                   \\ [.3em]
                        & \texttt{Add3}     & No        &                       &                   &                       &                       &                   \\ [.3em]
                        & \texttt{Mult1}    & Yes       & \texttt{MULTD}        &                   & \texttt{R(F2)}        & \texttt{Load1}        &                   \\ [.3em]
                        & \texttt{Mult2}    & No        &                       &                   &                       &                       &                   \\
            \bottomrule
        \end{tabular}
        \caption*{Reservation Stations.}
    \end{table}

    \begin{table}[!htp]
        \centering
        \begin{tabular}{@{} c | c | c c c c c c c | c | c @{}}
            \toprule
            Clock       & \texttt{R1}       & \texttt{F0}           & \texttt{F2}   & \texttt{F4}           & \texttt{F6}       & \texttt{F8}   & \texttt{F10}  & \texttt{F12}  & $\dots$   & \texttt{F30}  \\
            \midrule
            \theenumi   & 80                & \texttt{Load1}        &               & \texttt{Mult1}        &                   &               &               &               &           &               \\
            \bottomrule
        \end{tabular}
        \caption*{Register result status.}
    \end{table}

    \newpage
    \setcounter{enumi}{5}


    %%%%%%%%%%%%%%%%%%%%%%%%%%%%%%%%%
    % Cycle 6
    %%%%%%%%%%%%%%%%%%%%%%%%%%%%%%%%%
    \item \textbf{Cycle \theenumi}
    \begin{itemize}
        \item We have skipped cycle 4 and 5 because we are only waiting for \texttt{LD}, \texttt{MULTD} or \texttt{SD} updates. In cycle 5, register \texttt{R1} becomes \texttt{72} due to the operation \texttt{SUBI R1, R1, \#8}.

        
        \item Instruction \texttt{LD F0, 0(R1)} from iteration 2 is now issued into \texttt{Load2} buffer. This new load has a cache hit (remember: only the first load had a cache miss).
        

        \item \texttt{F0} is again associated with \texttt{Load2} (the second load). The previous \texttt{F0} result (from \texttt{Load1}) is \textbf{overwritten} by the new \texttt{F0} pending load. Even though the logical register \texttt{F0} is overwritten, the data dependency graph remains correct, because \textbf{Tomasulo tracks data flow by tags}, not register names. Thus:
        \begin{itemize}
            \item \texttt{MULTD F4, F0, F2} (iteration 1): waits for \texttt{Load1} to complete (old \texttt{F0})
            \item \texttt{MULTD F4, F0, F2} (iteration 2): will wait for \texttt{Load2} to complete (new \texttt{F0})
        \end{itemize}
        
        \textcolor{Green3}{\faIcon{check-circle}} \textbf{WAW hazard avoided} by dynamic renaming. New instructions get the new \texttt{F0} and old instructions already locked the old value.
    \end{itemize}
    \begin{table}[!htp]
        \centering
        \begin{tabular}{@{} c l | c c c @{}}
            \toprule
            Iter.       & Instruction                   & Issue & Exec Comp & Write Res \\
            \midrule
            1           & \texttt{LD    F0, 0, R1}      & 1     &           &           \\ [.3em]
            1           & \texttt{MULTD F4, F0, F2}     & 2     &           &           \\ [.3em]
            1           & \texttt{SD    F4, 0(R1)}      & 3     &           &           \\ [.3em]
            2           & \texttt{LD    F0, 0, R1}      & \hl{6}&           &           \\ [.3em]
            2           & \texttt{MULTD F4, F0, F2}     &       &           &           \\ [.3em]
            2           & \texttt{SD    F4, 0(R1)}      &       &           &           \\
            \bottomrule
        \end{tabular}
        \caption*{Instruction status.}
    \end{table}

    \begin{table}[!htp]
        \centering
        \begin{tabular}{@{} l | c c c @{}}
            \toprule
                                & Busy      & Addr      & FU   \\
            \midrule
            \texttt{Load1}      & Yes       & 80        & \textbackslash        \\ [.3em]
            \texttt{Load2}      & \hl{Yes}  & \hl{72}   & \textbackslash        \\ [.3em]
            \texttt{Load3}      & No        &           & \textbackslash        \\
            \cmidrule{1-4}
            \texttt{Store1}     & Yes       & 80        & \texttt{Mult1}        \\ [.3em]
            \texttt{Store2}     & No        &           &                       \\ [.3em]
            \texttt{Store3}     & No        &           &                       \\
            \bottomrule
        \end{tabular}
        \caption*{Load and Store Buffers.}
    \end{table}

    \newpage

    \begin{table}[!htp]
        \centering
        \begin{tabular}{@{} l l | c c c c c c @{}}
            \toprule
            Time        & Name              & Busy      & \texttt{Op}           & \texttt{Vj}       & \texttt{Vk}           & \texttt{Qj}           & \texttt{Qk}       \\
            \midrule
                        & \texttt{Add1}     & No        &                       &                   &                       &                       &                   \\ [.3em]
                        & \texttt{Add2}     & No        &                       &                   &                       &                       &                   \\ [.3em]
                        & \texttt{Add3}     & No        &                       &                   &                       &                       &                   \\ [.3em]
                        & \texttt{Mult1}    & Yes       & \texttt{MULTD}        &                   & \texttt{R(F2)}        & \texttt{Load1}        &                   \\ [.3em]
                        & \texttt{Mult2}    & No        &                       &                   &                       &                       &                   \\
            \bottomrule
        \end{tabular}
        \caption*{Reservation Stations.}
    \end{table}

    \begin{table}[!htp]
        \centering
        \begin{tabular}{@{} c | c | c c c c c c c | c | c @{}}
            \toprule
            Clock       & \texttt{R1}       & \texttt{F0}           & \texttt{F2}   & \texttt{F4}           & \texttt{F6}       & \texttt{F8}   & \texttt{F10}  & \texttt{F12}  & $\dots$   & \texttt{F30}  \\
            \midrule
            \theenumi   & 72                & \hl{\texttt{Load2}}   &               & \texttt{Mult1}        &                   &               &               &               &           &               \\
            \bottomrule
        \end{tabular}
        \caption*{Register result status.}
    \end{table}

    \newpage


    %%%%%%%%%%%%%%%%%%%%%%%%%%%%%%%%%
    % Cycle 7
    %%%%%%%%%%%%%%%%%%%%%%%%%%%%%%%%%
    \item \textbf{Cycle \theenumi}
    \begin{itemize}
        \item \textbf{Even though the second load is ready} (cache hit, 1 cycle latency, and issued in cycle 6), it must \textbf{wait for the first load to complete} because they share the same architectural address and could have aliasing or store/load order dependencies. So unless advanced speculation is used, \textbf{loads and stores to the same address must be done in program order}.
        
        This is called \definition{Memory Disambiguation}. In out-of-order processors, memory disambiguation is the logic that \textbf{decides whether two loads/stores/loads can safely reorder}. Tomasulo in its original form \textbf{does not perform aggressive disambiguation}.


        \item \texttt{MULTD F4, F0, F2} (iteration 2) is issued.
    \end{itemize}
    \begin{table}[!htp]
        \centering
        \begin{tabular}{@{} c l | c c c @{}}
            \toprule
            Iter.       & Instruction                   & Issue & Exec Comp & Write Res \\
            \midrule
            1           & \texttt{LD    F0, 0, R1}      & 1     &           &           \\ [.3em]
            1           & \texttt{MULTD F4, F0, F2}     & 2     &           &           \\ [.3em]
            1           & \texttt{SD    F4, 0(R1)}      & 3     &           &           \\ [.3em]
            2           & \texttt{LD    F0, 0, R1}      & 6     &           &           \\ [.3em]
            2           & \texttt{MULTD F4, F0, F2}     & \hl{7}&           &           \\ [.3em]
            2           & \texttt{SD    F4, 0(R1)}      &       &           &           \\
            \bottomrule
        \end{tabular}
        \caption*{Instruction status.}
    \end{table}

    \begin{table}[!htp]
        \centering
        \begin{tabular}{@{} l | c c c @{}}
            \toprule
                                & Busy      & Addr      & FU   \\
            \midrule
            \texttt{Load1}      & Yes       & 80        & \textbackslash        \\ [.3em]
            \texttt{Load2}      & Yes       & 72        & \textbackslash        \\ [.3em]
            \texttt{Load3}      & No        &           & \textbackslash        \\
            \cmidrule{1-4}
            \texttt{Store1}     & Yes       & 80        & \texttt{Mult1}        \\ [.3em]
            \texttt{Store2}     & No        &           &                       \\ [.3em]
            \texttt{Store3}     & No        &           &                       \\
            \bottomrule
        \end{tabular}
        \caption*{Load and Store Buffers.}
    \end{table}

    \begin{table}[!htp]
        \centering
        \begin{tabular}{@{} l l | c c c c c c @{}}
            \toprule
            Time        & Name              & Busy      & \texttt{Op}           & \texttt{Vj}       & \texttt{Vk}           & \texttt{Qj}           & \texttt{Qk}       \\
            \midrule
                        & \texttt{Add1}     & No        &                       &                   &                       &                       &                   \\ [.3em]
                        & \texttt{Add2}     & No        &                       &                   &                       &                       &                   \\ [.3em]
                        & \texttt{Add3}     & No        &                       &                   &                       &                       &                   \\ [.3em]
                        & \texttt{Mult1}    & Yes       & \texttt{MULTD}        &                   & \texttt{R(F2)}        & \texttt{Load1}        &                   \\ [.3em]
                        & \texttt{Mult2}    & \hl{Yes}  & \hl{\texttt{MULTD}}   &                   & \hl{\texttt{R(F2)}}   & \hl{\texttt{Load2}}   &                   \\
            \bottomrule
        \end{tabular}
        \caption*{Reservation Stations.}
    \end{table}

    \newpage

    \begin{table}[!htp]
        \centering
        \begin{tabular}{@{} c | c | c c c c c c c | c | c @{}}
            \toprule
            Clock       & \texttt{R1}       & \texttt{F0}           & \texttt{F2}   & \texttt{F4}           & \texttt{F6}       & \texttt{F8}   & \texttt{F10}  & \texttt{F12}  & $\dots$   & \texttt{F30}  \\
            \midrule
            \theenumi   & 72                & \texttt{Load2}        &               & \hl{\texttt{Mult2}}   &                   &               &               &               &           &               \\
            \bottomrule
        \end{tabular}
        \caption*{Register result status.}
    \end{table}

    \newpage


    %%%%%%%%%%%%%%%%%%%%%%%%%%%%%%%%%
    % Cycle 8
    %%%%%%%%%%%%%%%%%%%%%%%%%%%%%%%%%
    \item \textbf{Cycle \theenumi}
    \begin{itemize}
        \item \texttt{SD F4, 0(R1)} (iteration 2) is issued.  It is assigned to \texttt{Store2} (store buffer). The store buffer must wait for \texttt{F4} to be computed by \texttt{Mult2}. We record the tag of the functional unit that will produce the needed value \texttt{F4}.

        \item \textbf{WAW (Write After Write) hazards on registers \texttt{F0} and \texttt{F4} have been resolved}, meaning the first and second iterations are now \textbf{fully overlapped}.
    \end{itemize}
    \begin{table}[!htp]
        \centering
        \begin{tabular}{@{} c l | c c c @{}}
            \toprule
            Iter.       & Instruction                   & Issue & Exec Comp & Write Res \\
            \midrule
            1           & \texttt{LD    F0, 0, R1}      & 1     &           &           \\ [.3em]
            1           & \texttt{MULTD F4, F0, F2}     & 2     &           &           \\ [.3em]
            1           & \texttt{SD    F4, 0(R1)}      & 3     &           &           \\ [.3em]
            2           & \texttt{LD    F0, 0, R1}      & 6     &           &           \\ [.3em]
            2           & \texttt{MULTD F4, F0, F2}     & 7     &           &           \\ [.3em]
            2           & \texttt{SD    F4, 0(R1)}      & \hl{8}&           &           \\
            \bottomrule
        \end{tabular}
        \caption*{Instruction status.}
    \end{table}

    \begin{table}[!htp]
        \centering
        \begin{tabular}{@{} l | c c c @{}}
            \toprule
                                & Busy      & Addr      & FU   \\
            \midrule
            \texttt{Load1}      & Yes       & 80        & \textbackslash        \\ [.3em]
            \texttt{Load2}      & Yes       & 72        & \textbackslash        \\ [.3em]
            \texttt{Load3}      & No        &           & \textbackslash        \\
            \cmidrule{1-4}
            \texttt{Store1}     & Yes       & 80        & \texttt{Mult1}        \\ [.3em]
            \texttt{Store2}     & \hl{Yes}  & \hl{72}   & \hl{\texttt{Mult2}}   \\ [.3em]
            \texttt{Store3}     & No        &           &                       \\
            \bottomrule
        \end{tabular}
        \caption*{Load and Store Buffers.}
    \end{table}

    \begin{table}[!htp]
        \centering
        \begin{tabular}{@{} l l | c c c c c c @{}}
            \toprule
            Time        & Name              & Busy      & \texttt{Op}           & \texttt{Vj}       & \texttt{Vk}           & \texttt{Qj}           & \texttt{Qk}       \\
            \midrule
                        & \texttt{Add1}     & No        &                       &                   &                       &                       &                   \\ [.3em]
                        & \texttt{Add2}     & No        &                       &                   &                       &                       &                   \\ [.3em]
                        & \texttt{Add3}     & No        &                       &                   &                       &                       &                   \\ [.3em]
                        & \texttt{Mult1}    & Yes       & \texttt{MULTD}        &                   & \texttt{R(F2)}        & \texttt{Load1}        &                   \\ [.3em]
                        & \texttt{Mult2}    & Yes       & \texttt{MULTD}        &                   & \texttt{R(F2)}        & \texttt{Load2}        &                   \\
            \bottomrule
        \end{tabular}
        \caption*{Reservation Stations.}
    \end{table}

    \newpage

    \begin{table}[!htp]
        \centering
        \begin{tabular}{@{} c | c | c c c c c c c | c | c @{}}
            \toprule
            Clock       & \texttt{R1}       & \texttt{F0}           & \texttt{F2}   & \texttt{F4}           & \texttt{F6}       & \texttt{F8}   & \texttt{F10}  & \texttt{F12}  & $\dots$   & \texttt{F30}  \\
            \midrule
            \theenumi   & 72                & \texttt{Load2}        &               & \texttt{Mult2}        &                   &               &               &               &           &               \\
            \bottomrule
        \end{tabular}
        \caption*{Register result status.}
    \end{table}

    \newpage


    %%%%%%%%%%%%%%%%%%%%%%%%%%%%%%%%%
    % Cycle 9
    %%%%%%%%%%%%%%%%%%%%%%%%%%%%%%%%%
    \item \textbf{Cycle \theenumi}
    \begin{itemize}
        \item \texttt{LD F0, 0(R1)} (iteration 1, address 80, cache miss) finally completes and writes back in the next cycle (1 cycle latency). Now the \texttt{LD F0, 0(R1)} from iteration 2 can start its execution.


        \item Even though we don't track the \texttt{SUBI} instruction, it is issued in this cycle (iteration 2).
    \end{itemize}
    \begin{table}[!htp]
        \centering
        \begin{tabular}{@{} c l | c c c @{}}
            \toprule
            Iter.       & Instruction                   & Issue & Exec Comp & Write Res \\
            \midrule
            1           & \texttt{LD    F0, 0, R1}      & 1     & \hl{9}    &           \\ [.3em]
            1           & \texttt{MULTD F4, F0, F2}     & 2     &           &           \\ [.3em]
            1           & \texttt{SD    F4, 0(R1)}      & 3     &           &           \\ [.3em]
            2           & \texttt{LD    F0, 0, R1}      & 6     &           &           \\ [.3em]
            2           & \texttt{MULTD F4, F0, F2}     & 7     &           &           \\ [.3em]
            2           & \texttt{SD    F4, 0(R1)}      & 8     &           &           \\
            \bottomrule
        \end{tabular}
        \caption*{Instruction status.}
    \end{table}

    \begin{table}[!htp]
        \centering
        \begin{tabular}{@{} l | c c c @{}}
            \toprule
                                & Busy      & Addr      & FU   \\
            \midrule
            \texttt{Load1}      & Yes       & 80        & \textbackslash        \\ [.3em]
            \texttt{Load2}      & Yes       & 72        & \textbackslash        \\ [.3em]
            \texttt{Load3}      & No        &           & \textbackslash        \\
            \cmidrule{1-4}
            \texttt{Store1}     & Yes       & 80        & \texttt{Mult1}        \\ [.3em]
            \texttt{Store2}     & Yes       & 72        & \texttt{Mult2}        \\ [.3em]
            \texttt{Store3}     & No        &           &                       \\
            \bottomrule
        \end{tabular}
        \caption*{Load and Store Buffers.}
    \end{table}

    \begin{table}[!htp]
        \centering
        \begin{tabular}{@{} l l | c c c c c c @{}}
            \toprule
            Time        & Name              & Busy      & \texttt{Op}           & \texttt{Vj}       & \texttt{Vk}           & \texttt{Qj}           & \texttt{Qk}       \\
            \midrule
                        & \texttt{Add1}     & No        &                       &                   &                       &                       &                   \\ [.3em]
                        & \texttt{Add2}     & No        &                       &                   &                       &                       &                   \\ [.3em]
                        & \texttt{Add3}     & No        &                       &                   &                       &                       &                   \\ [.3em]
                        & \texttt{Mult1}    & Yes       & \texttt{MULTD}        &                   & \texttt{R(F2)}        & \texttt{Load1}        &                   \\ [.3em]
                        & \texttt{Mult2}    & Yes       & \texttt{MULTD}        &                   & \texttt{R(F2)}        & \texttt{Load2}        &                   \\
            \bottomrule
        \end{tabular}
        \caption*{Reservation Stations.}
    \end{table}

    \newpage

    \begin{table}[!htp]
        \centering
        \begin{tabular}{@{} c | c | c c c c c c c | c | c @{}}
            \toprule
            Clock       & \texttt{R1}       & \texttt{F0}           & \texttt{F2}   & \texttt{F4}           & \texttt{F6}       & \texttt{F8}   & \texttt{F10}  & \texttt{F12}  & $\dots$   & \texttt{F30}  \\
            \midrule
            \theenumi   & 72                & \texttt{Load2}        &               & \texttt{Mult2}        &                   &               &               &               &           &               \\
            \bottomrule
        \end{tabular}
        \caption*{Register result status.}
    \end{table}

    \newpage


    %%%%%%%%%%%%%%%%%%%%%%%%%%%%%%%%%
    % Cycle 10
    %%%%%%%%%%%%%%%%%%%%%%%%%%%%%%%%%
    \item \textbf{Cycle \theenumi}
    \begin{itemize}
        \item \texttt{LD F0, 0(R1)} (iteration 1) now writes back (1 cycle latency). It also sends the result through the CDB. The value written can be represented as \texttt{M[80]}, indicating the location of memory at address \texttt{80}.

        \texttt{Mult1} was listening in the CDB for the \texttt{Load1} value; now it receives the update, so it sets the \texttt{Qj} entry to zero and updates its \texttt{Vj} value with the value in memory at address \texttt{80} (\texttt{M[80]}).
        
        \item The second load (\texttt{Load2}) completes its execution in 1 cycle thanks to the cache hit. In the next cycle, the result is written back. This instruction was blocked by the previous load to avoid too aggressive memory disambiguation.

        \item Even though we don't track the \texttt{BNEZ} instruction, it is issued in this cycle (iteration 2). This updates the value in register \texttt{R1} from \texttt{72} to \texttt{64} (minus \texttt{8}).
    \end{itemize}
    \begin{table}[!htp]
        \centering
        \begin{tabular}{@{} c l | c c c @{}}
            \toprule
            Iter.       & Instruction                   & Issue & Exec Comp & Write Res \\
            \midrule
            1           & \texttt{LD    F0, 0, R1}      & 1     & 9         & \hl{10}   \\ [.3em]
            1           & \texttt{MULTD F4, F0, F2}     & 2     &           &           \\ [.3em]
            1           & \texttt{SD    F4, 0(R1)}      & 3     &           &           \\ [.3em]
            2           & \texttt{LD    F0, 0, R1}      & 6     & \hl{10}   &           \\ [.3em]
            2           & \texttt{MULTD F4, F0, F2}     & 7     &           &           \\ [.3em]
            2           & \texttt{SD    F4, 0(R1)}      & 8     &           &           \\
            \bottomrule
        \end{tabular}
        \caption*{Instruction status.}
    \end{table}

    \begin{table}[!htp]
        \centering
        \begin{tabular}{@{} l | c c c @{}}
            \toprule
                                & Busy      & Addr      & FU   \\
            \midrule
            \texttt{Load1}      & \hl{No}   &           & \textbackslash        \\ [.3em]
            \texttt{Load2}      & Yes       & 72        & \textbackslash        \\ [.3em]
            \texttt{Load3}      & No        &           & \textbackslash        \\
            \cmidrule{1-4}
            \texttt{Store1}     & Yes       & 80        & \texttt{Mult1}        \\ [.3em]
            \texttt{Store2}     & Yes       & 72        & \texttt{Mult2}        \\ [.3em]
            \texttt{Store3}     & No        &           &                       \\
            \bottomrule
        \end{tabular}
        \caption*{Load and Store Buffers.}
    \end{table}

    \newpage

    \begin{table}[!htp]
        \centering
        \begin{tabular}{@{} l l | c c c c c c @{}}
            \toprule
            Time        & Name              & Busy      & \texttt{Op}           & \texttt{Vj}           & \texttt{Vk}           & \texttt{Qj}           & \texttt{Qk}       \\
            \midrule
                        & \texttt{Add1}     & No        &                       &                       &                       &                       &                   \\ [.3em]
                        & \texttt{Add2}     & No        &                       &                       &                       &                       &                   \\ [.3em]
                        & \texttt{Add3}     & No        &                       &                       &                       &                       &                   \\ [.3em]
            \hl{4}      & \texttt{Mult1}    & Yes       & \texttt{MULTD}        & \hl{\texttt{M[80]}}   & \texttt{R(F2)}        &                       &                   \\ [.3em]
                        & \texttt{Mult2}    & Yes       & \texttt{MULTD}        &                       & \texttt{R(F2)}        & \texttt{Load2}        &                   \\
            \bottomrule
        \end{tabular}
        \caption*{Reservation Stations.}
    \end{table}

    \begin{table}[!htp]
        \centering
        \begin{tabular}{@{} c | c | c c c c c c c | c | c @{}}
            \toprule
            Clock       & \texttt{R1}       & \texttt{F0}           & \texttt{F2}   & \texttt{F4}           & \texttt{F6}       & \texttt{F8}   & \texttt{F10}  & \texttt{F12}  & $\dots$   & \texttt{F30}  \\
            \midrule
            \theenumi   & 64                & \texttt{Load2}        &               & \texttt{Mult2}        &                   &               &               &               &           &               \\
            \bottomrule
        \end{tabular}
        \caption*{Register result status.}
    \end{table}

    \newpage


    %%%%%%%%%%%%%%%%%%%%%%%%%%%%%%%%%
    % Cycle 11
    %%%%%%%%%%%%%%%%%%%%%%%%%%%%%%%%%
    \item \textbf{Cycle \theenumi}
    \begin{itemize}
        \item \texttt{LD F0, 0(R1)} (iteration 2) now writes back (1 cycle latency). It also sends the result through the CDB. The value written can be represented as \texttt{M[72]}, indicating the location of memory at address \texttt{72}.

        \texttt{Mult2} was listening in the CDB for the \texttt{Load2} value; now it receives the update, so it sets the \texttt{Qj} entry to zero and updates its \texttt{Vj} value with the value in memory at address \texttt{72} (\texttt{M[72]}).
        
        \item The third load (\texttt{Load3}) starts its execution because the cycle continues to execute. So we are now at iteration number 3, at the first load command: \texttt{LD F0, 0(R1)}. So we add the entry \texttt{Load3} to the table \emph{Load and Store Buffers}.

        \item \texttt{Mult1} continues its execution, it has 3 cycles left to complete.
    \end{itemize}
    \begin{table}[!htp]
        \centering
        \begin{tabular}{@{} c l | c c c @{}}
            \toprule
            Iter.       & Instruction                   & Issue & Exec Comp & Write Res \\
            \midrule
            1           & \texttt{LD    F0, 0, R1}      & 1     & 9         & 10        \\ [.3em]
            1           & \texttt{MULTD F4, F0, F2}     & 2     &           &           \\ [.3em]
            1           & \texttt{SD    F4, 0(R1)}      & 3     &           &           \\ [.3em]
            2           & \texttt{LD    F0, 0, R1}      & 6     & 10        & \hl{11}   \\ [.3em]
            2           & \texttt{MULTD F4, F0, F2}     & 7     &           &           \\ [.3em]
            2           & \texttt{SD    F4, 0(R1)}      & 8     &           &           \\
            \bottomrule
        \end{tabular}
        \caption*{Instruction status.}
    \end{table}

    \begin{table}[!htp]
        \centering
        \begin{tabular}{@{} l | c c c @{}}
            \toprule
                                & Busy      & Addr      & FU   \\
            \midrule
            \texttt{Load1}      & No        &           & \textbackslash        \\ [.3em]
            \texttt{Load2}      & \hl{No}   &           & \textbackslash        \\ [.3em]
            \texttt{Load3}      & \hl{Yes}  & \hl{64}   & \textbackslash        \\
            \cmidrule{1-4}
            \texttt{Store1}     & Yes       & 80        & \texttt{Mult1}        \\ [.3em]
            \texttt{Store2}     & Yes       & 72        & \texttt{Mult2}        \\ [.3em]
            \texttt{Store3}     & No        &           &                       \\
            \bottomrule
        \end{tabular}
        \caption*{Load and Store Buffers.}
    \end{table}

    \newpage

    \begin{table}[!htp]
        \centering
        \begin{tabular}{@{} l l | c c c c c c @{}}
            \toprule
            Time        & Name              & Busy      & \texttt{Op}           & \texttt{Vj}           & \texttt{Vk}           & \texttt{Qj}           & \texttt{Qk}       \\
            \midrule
                        & \texttt{Add1}     & No        &                       &                       &                       &                       &                   \\ [.3em]
                        & \texttt{Add2}     & No        &                       &                       &                       &                       &                   \\ [.3em]
                        & \texttt{Add3}     & No        &                       &                       &                       &                       &                   \\ [.3em]
            \hl{3}      & \texttt{Mult1}    & Yes       & \texttt{MULTD}        & \texttt{M[80]}        & \texttt{R(F2)}        &                       &                   \\ [.3em]
            \hl{4}      & \texttt{Mult2}    & Yes       & \texttt{MULTD}        & \hl{\texttt{M[72]}}   & \texttt{R(F2)}        &                       &                   \\
            \bottomrule
        \end{tabular}
        \caption*{Reservation Stations.}
    \end{table}

    \begin{table}[!htp]
        \centering
        \begin{tabular}{@{} c | c | c c c c c c c | c | c @{}}
            \toprule
            Clock       & \texttt{R1}       & \texttt{F0}           & \texttt{F2}   & \texttt{F4}           & \texttt{F6}       & \texttt{F8}   & \texttt{F10}  & \texttt{F12}  & $\dots$   & \texttt{F30}  \\
            \midrule
            \theenumi   & 64                & \hl{\texttt{Load3}}   &               & \texttt{Mult2}        &                   &               &               &               &           &               \\
            \bottomrule
        \end{tabular}
        \caption*{Register result status.}
    \end{table}

    \newpage


    %%%%%%%%%%%%%%%%%%%%%%%%%%%%%%%%%
    % Cycle 12
    %%%%%%%%%%%%%%%%%%%%%%%%%%%%%%%%%
    \item \textbf{Cycle \theenumi}
    \begin{itemize}
        \item \texttt{MULTD F4, F0, F2} (iteration 3) cannot be issued, because in Tomasulo's algorithm, to issue a new instruction there must be a free Reservation Station available for that operation type (here, floating-point multiply). If no RS is free, the instruction cannot be issued and must wait.

        Since both \texttt{Mult1} and \texttt{Mult2} are occupied executing previous multiplies, the third multiply (from iteration 3) cannot issue yet.
        

        \item \texttt{Mult1} continues its execution, it has 2 cycles left to complete.


        \item \texttt{Mult2} continues its execution, it has 3 cycles left to complete.
    \end{itemize}
    \begin{table}[!htp]
        \centering
        \begin{tabular}{@{} c l | c c c @{}}
            \toprule
            Iter.       & Instruction                   & Issue & Exec Comp & Write Res \\
            \midrule
            1           & \texttt{LD    F0, 0, R1}      & 1     & 9         & 10        \\ [.3em]
            1           & \texttt{MULTD F4, F0, F2}     & 2     &           &           \\ [.3em]
            1           & \texttt{SD    F4, 0(R1)}      & 3     &           &           \\ [.3em]
            2           & \texttt{LD    F0, 0, R1}      & 6     & 10        & 11        \\ [.3em]
            2           & \texttt{MULTD F4, F0, F2}     & 7     &           &           \\ [.3em]
            2           & \texttt{SD    F4, 0(R1)}      & 8     &           &           \\
            \bottomrule
        \end{tabular}
        \caption*{Instruction status.}
    \end{table}

    \begin{table}[!htp]
        \centering
        \begin{tabular}{@{} l | c c c @{}}
            \toprule
                                & Busy      & Addr      & FU   \\
            \midrule
            \texttt{Load1}      & No        &           & \textbackslash        \\ [.3em]
            \texttt{Load2}      & No        &           & \textbackslash        \\ [.3em]
            \texttt{Load3}      & Yes       & 64        & \textbackslash        \\
            \cmidrule{1-4}
            \texttt{Store1}     & Yes       & 80        & \texttt{Mult1}        \\ [.3em]
            \texttt{Store2}     & Yes       & 72        & \texttt{Mult2}        \\ [.3em]
            \texttt{Store3}     & No        &           &                       \\
            \bottomrule
        \end{tabular}
        \caption*{Load and Store Buffers.}
    \end{table}

    \begin{table}[!htp]
        \centering
        \begin{tabular}{@{} l l | c c c c c c @{}}
            \toprule
            Time        & Name              & Busy      & \texttt{Op}           & \texttt{Vj}           & \texttt{Vk}           & \texttt{Qj}           & \texttt{Qk}       \\
            \midrule
                        & \texttt{Add1}     & No        &                       &                       &                       &                       &                   \\ [.3em]
                        & \texttt{Add2}     & No        &                       &                       &                       &                       &                   \\ [.3em]
                        & \texttt{Add3}     & No        &                       &                       &                       &                       &                   \\ [.3em]
            \hl{2}      & \texttt{Mult1}    & Yes       & \texttt{MULTD}        & \texttt{M[80]}        & \texttt{R(F2)}        &                       &                   \\ [.3em]
            \hl{3}      & \texttt{Mult2}    & Yes       & \texttt{MULTD}        & \texttt{M[72]}        & \texttt{R(F2)}        &                       &                   \\
            \bottomrule
        \end{tabular}
        \caption*{Reservation Stations.}
    \end{table}

    \newpage

    \begin{table}[!htp]
        \centering
        \begin{tabular}{@{} c | c | c c c c c c c | c | c @{}}
            \toprule
            Clock       & \texttt{R1}       & \texttt{F0}           & \texttt{F2}   & \texttt{F4}           & \texttt{F6}       & \texttt{F8}   & \texttt{F10}  & \texttt{F12}  & $\dots$   & \texttt{F30}  \\
            \midrule
            \theenumi   & 64                & \texttt{Load3}        &               & \texttt{Mult2}        &                   &               &               &               &           &               \\
            \bottomrule
        \end{tabular}
        \caption*{Register result status.}
    \end{table}

    \newpage


    %%%%%%%%%%%%%%%%%%%%%%%%%%%%%%%%%
    % Cycle 13
    %%%%%%%%%%%%%%%%%%%%%%%%%%%%%%%%%
    \item \textbf{Cycle \theenumi}
    \begin{itemize}
        \item \texttt{LD F0, 0(R1)} completes its execution and writes the result back to the CDB (\texttt{M[64]}).
        

        \item \texttt{Mult1} continues its execution, it has 1 cycle left to complete.


        \item \texttt{Mult2} continues its execution, it has 2 cycles left to complete.
    \end{itemize}
    \begin{table}[!htp]
        \centering
        \begin{tabular}{@{} c l | c c c @{}}
            \toprule
            Iter.       & Instruction                   & Issue & Exec Comp & Write Res \\
            \midrule
            1           & \texttt{LD    F0, 0, R1}      & 1     & 9         & 10        \\ [.3em]
            1           & \texttt{MULTD F4, F0, F2}     & 2     &           &           \\ [.3em]
            1           & \texttt{SD    F4, 0(R1)}      & 3     &           &           \\ [.3em]
            2           & \texttt{LD    F0, 0, R1}      & 6     & 10        & 11        \\ [.3em]
            2           & \texttt{MULTD F4, F0, F2}     & 7     &           &           \\ [.3em]
            2           & \texttt{SD    F4, 0(R1)}      & 8     &           &           \\
            \bottomrule
        \end{tabular}
        \caption*{Instruction status.}
    \end{table}

    \begin{table}[!htp]
        \centering
        \begin{tabular}{@{} l | c c c @{}}
            \toprule
                                & Busy      & Addr      & FU   \\
            \midrule
            \texttt{Load1}      & No        &           & \textbackslash        \\ [.3em]
            \texttt{Load2}      & No        &           & \textbackslash        \\ [.3em]
            \texttt{Load3}      & \hl{No}   &           & \textbackslash        \\
            \cmidrule{1-4}
            \texttt{Store1}     & Yes       & 80        & \texttt{Mult1}        \\ [.3em]
            \texttt{Store2}     & Yes       & 72        & \texttt{Mult2}        \\ [.3em]
            \texttt{Store3}     & No        &           &                       \\
            \bottomrule
        \end{tabular}
        \caption*{Load and Store Buffers.}
    \end{table}

    \begin{table}[!htp]
        \centering
        \begin{tabular}{@{} l l | c c c c c c @{}}
            \toprule
            Time        & Name              & Busy      & \texttt{Op}           & \texttt{Vj}           & \texttt{Vk}           & \texttt{Qj}           & \texttt{Qk}       \\
            \midrule
                        & \texttt{Add1}     & No        &                       &                       &                       &                       &                   \\ [.3em]
                        & \texttt{Add2}     & No        &                       &                       &                       &                       &                   \\ [.3em]
                        & \texttt{Add3}     & No        &                       &                       &                       &                       &                   \\ [.3em]
            \hl{1}      & \texttt{Mult1}    & Yes       & \texttt{MULTD}        & \texttt{M[80]}        & \texttt{R(F2)}        &                       &                   \\ [.3em]
            \hl{2}      & \texttt{Mult2}    & Yes       & \texttt{MULTD}        & \texttt{M[72]}        & \texttt{R(F2)}        &                       &                   \\
            \bottomrule
        \end{tabular}
        \caption*{Reservation Stations.}
    \end{table}

    \begin{table}[!htp]
        \centering
        \begin{tabular}{@{} c | c | c c c c c c c | c | c @{}}
            \toprule
            Clock       & \texttt{R1}       & \texttt{F0}           & \texttt{F2}   & \texttt{F4}           & \texttt{F6}       & \texttt{F8}   & \texttt{F10}  & \texttt{F12}  & $\dots$   & \texttt{F30}  \\
            \midrule
            \theenumi   & 64                & \hl{\texttt{M[64]}}   &               & \texttt{Mult2}        &                   &               &               &               &           &               \\
            \bottomrule
        \end{tabular}
        \caption*{Register result status.}
    \end{table}

    \newpage


    %%%%%%%%%%%%%%%%%%%%%%%%%%%%%%%%%
    % Cycle 14
    %%%%%%%%%%%%%%%%%%%%%%%%%%%%%%%%%
    \item \textbf{Cycle \theenumi}
    \begin{itemize}
        \item \texttt{Mult1} finishes its execution at the end of this cycle! This will allow the \texttt{Store1} to write the result in memory at the address \texttt{80}, and also to the \texttt{MULTD} instruction in the iteration 3 to be issued.
        

        \item \texttt{Mult2} continues its execution, it has 1 cycle left to complete.
    \end{itemize}
    \begin{table}[!htp]
        \centering
        \begin{tabular}{@{} c l | c c c @{}}
            \toprule
            Iter.       & Instruction                   & Issue & Exec Comp & Write Res \\
            \midrule
            1           & \texttt{LD    F0, 0, R1}      & 1     & 9         & 10        \\ [.3em]
            1           & \texttt{MULTD F4, F0, F2}     & 2     & \hl{14}   &           \\ [.3em]
            1           & \texttt{SD    F4, 0(R1)}      & 3     &           &           \\ [.3em]
            2           & \texttt{LD    F0, 0, R1}      & 6     & 10        & 11        \\ [.3em]
            2           & \texttt{MULTD F4, F0, F2}     & 7     &           &           \\ [.3em]
            2           & \texttt{SD    F4, 0(R1)}      & 8     &           &           \\
            \bottomrule
        \end{tabular}
        \caption*{Instruction status.}
    \end{table}

    \begin{table}[!htp]
        \centering
        \begin{tabular}{@{} l | c c c @{}}
            \toprule
                                & Busy      & Addr      & FU   \\
            \midrule
            \texttt{Load1}      & No        &           & \textbackslash        \\ [.3em]
            \texttt{Load2}      & No        &           & \textbackslash        \\ [.3em]
            \texttt{Load3}      & No        &           & \textbackslash        \\
            \cmidrule{1-4}
            \texttt{Store1}     & Yes       & 80        & \texttt{Mult1}        \\ [.3em]
            \texttt{Store2}     & Yes       & 72        & \texttt{Mult2}        \\ [.3em]
            \texttt{Store3}     & No        &           &                       \\
            \bottomrule
        \end{tabular}
        \caption*{Load and Store Buffers.}
    \end{table}

    \begin{table}[!htp]
        \centering
        \begin{tabular}{@{} l l | c c c c c c @{}}
            \toprule
            Time        & Name              & Busy      & \texttt{Op}           & \texttt{Vj}           & \texttt{Vk}           & \texttt{Qj}           & \texttt{Qk}       \\
            \midrule
                        & \texttt{Add1}     & No        &                       &                       &                       &                       &                   \\ [.3em]
                        & \texttt{Add2}     & No        &                       &                       &                       &                       &                   \\ [.3em]
                        & \texttt{Add3}     & No        &                       &                       &                       &                       &                   \\ [.3em]
            \hl{0}      & \texttt{Mult1}    & Yes       & \texttt{MULTD}        & \texttt{M[80]}        & \texttt{R(F2)}        &                       &                   \\ [.3em]
            \hl{1}      & \texttt{Mult2}    & Yes       & \texttt{MULTD}        & \texttt{M[72]}        & \texttt{R(F2)}        &                       &                   \\
            \bottomrule
        \end{tabular}
        \caption*{Reservation Stations.}
    \end{table}

    \begin{table}[!htp]
        \centering
        \begin{tabular}{@{} c | c | c c c c c c c | c | c @{}}
            \toprule
            Clock       & \texttt{R1}       & \texttt{F0}           & \texttt{F2}   & \texttt{F4}           & \texttt{F6}       & \texttt{F8}   & \texttt{F10}  & \texttt{F12}  & $\dots$   & \texttt{F30}  \\
            \midrule
            \theenumi   & 64                & \texttt{M[64]}        &               & \texttt{Mult2}        &                   &               &               &               &           &               \\
            \bottomrule
        \end{tabular}
        \caption*{Register result status.}
    \end{table}

    \newpage


    %%%%%%%%%%%%%%%%%%%%%%%%%%%%%%%%%
    % Cycle 15
    %%%%%%%%%%%%%%%%%%%%%%%%%%%%%%%%%
    \item \textbf{Cycle \theenumi}
    \begin{itemize}
        \item \texttt{MULTD} (iteration 1) writes the result to the CDB; this triggers the \texttt{Store1} listener, which waits for the result to be stored in memory (the result of the load, \texttt{M[80]}, multiplied by the value of register \texttt{F2}), and the \texttt{MULTD} of iteration 3, which waits for a free reservation station to be issued.
        

        \item \texttt{Mult2} finishes its execution at the end of this cycle! This will allow the \texttt{Store2} to write the result in memory at the address \texttt{72}.
    \end{itemize}
    \begin{table}[!htp]
        \centering
        \begin{tabular}{@{} c l | c c c @{}}
            \toprule
            Iter.       & Instruction                   & Issue & Exec Comp & Write Res \\
            \midrule
            1           & \texttt{LD    F0, 0, R1}      & 1     & 9         & 10        \\ [.3em]
            1           & \texttt{MULTD F4, F0, F2}     & 2     & 14        & \hl{15}   \\ [.3em]
            1           & \texttt{SD    F4, 0(R1)}      & 3     &           &           \\ [.3em]
            2           & \texttt{LD    F0, 0, R1}      & 6     & 10        & 11        \\ [.3em]
            2           & \texttt{MULTD F4, F0, F2}     & 7     & \hl{15}   &           \\ [.3em]
            2           & \texttt{SD    F4, 0(R1)}      & 8     &           &           \\
            \bottomrule
        \end{tabular}
        \caption*{Instruction status.}
    \end{table}

    \begin{table}[!htp]
        \centering
        \begin{tabular}{@{} l | c c c @{}}
            \toprule
                                & Busy      & Addr      & FU   \\
            \midrule
            \texttt{Load1}      & No        &           & \textbackslash                    \\ [.3em]
            \texttt{Load2}      & No        &           & \textbackslash                    \\ [.3em]
            \texttt{Load3}      & No        &           & \textbackslash                    \\
            \cmidrule{1-4}
            \texttt{Store1}     & Yes       & 80        & \hl{\texttt{M[80] $\times$ F2}}   \\ [.3em]
            \texttt{Store2}     & Yes       & 72        & \texttt{Mult2}                    \\ [.3em]
            \texttt{Store3}     & No        &           &                                   \\
            \bottomrule
        \end{tabular}
        \caption*{Load and Store Buffers.}
    \end{table}

    \begin{table}[!htp]
        \centering
        \begin{tabular}{@{} l l | c c c c c c @{}}
            \toprule
            Time        & Name              & Busy      & \texttt{Op}           & \texttt{Vj}           & \texttt{Vk}           & \texttt{Qj}           & \texttt{Qk}       \\
            \midrule
                        & \texttt{Add1}     & No        &                       &                       &                       &                       &                   \\ [.3em]
                        & \texttt{Add2}     & No        &                       &                       &                       &                       &                   \\ [.3em]
                        & \texttt{Add3}     & No        &                       &                       &                       &                       &                   \\ [.3em]
            \hl{4}      & \texttt{Mult1}    & \hl{Yes}  & \hl{\texttt{MULTD}}   & \hl{\texttt{M[64]}}   & \hl{\texttt{R(F2)}}   &                       &                   \\ [.3em]
            \hl{0}      & \texttt{Mult2}    & Yes       & \texttt{MULTD}        & \texttt{M[72]}        & \texttt{R(F2)}        &                       &                   \\
            \bottomrule
        \end{tabular}
        \caption*{Reservation Stations.}
    \end{table}

    \newpage

    \begin{table}[!htp]
        \centering
        \begin{tabular}{@{} c | c | c c c c c c c | c | c @{}}
            \toprule
            Clock       & \texttt{R1}       & \texttt{F0}           & \texttt{F2}   & \texttt{F4}           & \texttt{F6}       & \texttt{F8}   & \texttt{F10}  & \texttt{F12}  & $\dots$   & \texttt{F30}  \\
            \midrule
            \theenumi   & 64                & \texttt{M[64]}        &               & \hl{\texttt{Mult1}}   &                   &               &               &               &           &               \\
            \bottomrule
        \end{tabular}
        \caption*{Register result status.}
    \end{table}

    \newpage


    %%%%%%%%%%%%%%%%%%%%%%%%%%%%%%%%%
    % Cycle 16
    %%%%%%%%%%%%%%%%%%%%%%%%%%%%%%%%%
    \item \textbf{Cycle \theenumi}
    \begin{itemize}
        \item \texttt{MULTD} (iteration 2) writes the result to the CDB; this triggers the \texttt{Store2} listener, which waits for the result to be stored in memory (the result of the load, \texttt{M[72]}, multiplied by the value of register \texttt{F2}).
    \end{itemize}
    \begin{table}[!htp]
        \centering
        \begin{tabular}{@{} c l | c c c @{}}
            \toprule
            Iter.       & Instruction                   & Issue & Exec Comp & Write Res \\
            \midrule
            1           & \texttt{LD    F0, 0, R1}      & 1     & 9         & 10        \\ [.3em]
            1           & \texttt{MULTD F4, F0, F2}     & 2     & 14        & 15        \\ [.3em]
            1           & \texttt{SD    F4, 0(R1)}      & 3     & \hl{16}   &           \\ [.3em]
            2           & \texttt{LD    F0, 0, R1}      & 6     & 10        & 11        \\ [.3em]
            2           & \texttt{MULTD F4, F0, F2}     & 7     & 15        & \hl{16}   \\ [.3em]
            2           & \texttt{SD    F4, 0(R1)}      & 8     &           &           \\
            \bottomrule
        \end{tabular}
        \caption*{Instruction status.}
    \end{table}

    \begin{table}[!htp]
        \centering
        \begin{tabular}{@{} l | c c c @{}}
            \toprule
                                & Busy      & Addr      & FU   \\
            \midrule
            \texttt{Load1}      & No        &           & \textbackslash                    \\ [.3em]
            \texttt{Load2}      & No        &           & \textbackslash                    \\ [.3em]
            \texttt{Load3}      & No        &           & \textbackslash                    \\
            \cmidrule{1-4}
            \texttt{Store1}     & Yes       & 80        & \hl{\texttt{M[80] $\times$ F2}}   \\ [.3em]
            \texttt{Store2}     & Yes       & 72        & \hl{\texttt{M[72] $\times$ F2}}   \\ [.3em]
            \texttt{Store3}     & \hl{Yes}  & \hl{64}   & \hl{\texttt{Mult1}}               \\
            \bottomrule
        \end{tabular}
        \caption*{Load and Store Buffers.}
    \end{table}

    \begin{table}[!htp]
        \centering
        \begin{tabular}{@{} l l | c c c c c c @{}}
            \toprule
            Time        & Name              & Busy      & \texttt{Op}           & \texttt{Vj}           & \texttt{Vk}           & \texttt{Qj}           & \texttt{Qk}       \\
            \midrule
                        & \texttt{Add1}     & No        &                       &                       &                       &                       &                   \\ [.3em]
                        & \texttt{Add2}     & No        &                       &                       &                       &                       &                   \\ [.3em]
                        & \texttt{Add3}     & No        &                       &                       &                       &                       &                   \\ [.3em]
            \hl{3}      & \texttt{Mult1}    & Yes       & \texttt{MULTD}        & \texttt{M[64]}        & \texttt{R(F2)}        &                       &                   \\ [.3em]
                        & \texttt{Mult2}    & \hl{No}   &                       &                       &                       &                       &                   \\
            \bottomrule
        \end{tabular}
        \caption*{Reservation Stations.}
    \end{table}

    \begin{table}[!htp]
        \centering
        \begin{tabular}{@{} c | c | c c c c c c c | c | c @{}}
            \toprule
            Clock       & \texttt{R1}       & \texttt{F0}           & \texttt{F2}   & \texttt{F4}           & \texttt{F6}       & \texttt{F8}   & \texttt{F10}  & \texttt{F12}  & $\dots$   & \texttt{F30}  \\
            \midrule
            \theenumi   & 64                & \texttt{M[64]}        &               & \texttt{Mult1}        &                   &               &               &               &           &               \\
            \bottomrule
        \end{tabular}
        \caption*{Register result status.}
    \end{table}

    \newpage


    %%%%%%%%%%%%%%%%%%%%%%%%%%%%%%%%%
    % Cycle 17
    %%%%%%%%%%%%%%%%%%%%%%%%%%%%%%%%%
    \item \textbf{Cycle \theenumi}
    \begin{itemize}
        \item \texttt{SD} (iteration 1) writes the result of the expression \texttt{M[80] $\times$ F2} to memory at address \texttt{80}.
        \item Since the \texttt{SD} instruction (iteration 1) has finished its execution, the second store (iteration 2) can start its execution and send the result to memory between \texttt{M[72]} and \texttt{F2}.
    \end{itemize}
    \begin{table}[!htp]
        \centering
        \begin{tabular}{@{} c l | c c c @{}}
            \toprule
            Iter.       & Instruction                   & Issue & Exec Comp & Write Res \\
            \midrule
            1           & \texttt{LD    F0, 0, R1}      & 1     & 9         & 10        \\ [.3em]
            1           & \texttt{MULTD F4, F0, F2}     & 2     & 14        & 15        \\ [.3em]
            1           & \texttt{SD    F4, 0(R1)}      & 3     & 16        & \hl{17}   \\ [.3em]
            2           & \texttt{LD    F0, 0, R1}      & 6     & 10        & 11        \\ [.3em]
            2           & \texttt{MULTD F4, F0, F2}     & 7     & 15        & 16        \\ [.3em]
            2           & \texttt{SD    F4, 0(R1)}      & 8     & \hl{17}   &           \\
            \bottomrule
        \end{tabular}
        \caption*{Instruction status.}
    \end{table}

    \begin{table}[!htp]
        \centering
        \begin{tabular}{@{} l | c c c @{}}
            \toprule
                                & Busy      & Addr      & FU   \\
            \midrule
            \texttt{Load1}      & No        &           & \textbackslash                    \\ [.3em]
            \texttt{Load2}      & No        &           & \textbackslash                    \\ [.3em]
            \texttt{Load3}      & No        &           & \textbackslash                    \\
            \cmidrule{1-4}
            \texttt{Store1}     & \hl{No}   &           &                                   \\ [.3em]
            \texttt{Store2}     & Yes       & 72        & \texttt{M[72] $\times$ F2}        \\ [.3em]
            \texttt{Store3}     & Yes       & 64        & \texttt{Mult1}                    \\
            \bottomrule
        \end{tabular}
        \caption*{Load and Store Buffers.}
    \end{table}

    \begin{table}[!htp]
        \centering
        \begin{tabular}{@{} l l | c c c c c c @{}}
            \toprule
            Time        & Name              & Busy      & \texttt{Op}           & \texttt{Vj}           & \texttt{Vk}           & \texttt{Qj}           & \texttt{Qk}       \\
            \midrule
                        & \texttt{Add1}     & No        &                       &                       &                       &                       &                   \\ [.3em]
                        & \texttt{Add2}     & No        &                       &                       &                       &                       &                   \\ [.3em]
                        & \texttt{Add3}     & No        &                       &                       &                       &                       &                   \\ [.3em]
            \hl{2}      & \texttt{Mult1}    & Yes       & \texttt{MULTD}        & \texttt{M[64]}        & \texttt{R(F2)}        &                       &                   \\ [.3em]
                        & \texttt{Mult2}    & No        &                       &                       &                       &                       &                   \\
            \bottomrule
        \end{tabular}
        \caption*{Reservation Stations.}
    \end{table}

    \newpage

    \begin{table}[!htp]
        \centering
        \begin{tabular}{@{} c | c | c c c c c c c | c | c @{}}
            \toprule
            Clock       & \texttt{R1}       & \texttt{F0}           & \texttt{F2}   & \texttt{F4}           & \texttt{F6}       & \texttt{F8}   & \texttt{F10}  & \texttt{F12}  & $\dots$   & \texttt{F30}  \\
            \midrule
            \theenumi   & 64                & \texttt{M[64]}        &               & \texttt{Mult1}        &                   &               &               &               &           &               \\
            \bottomrule
        \end{tabular}
        \caption*{Register result status.}
    \end{table}

    \newpage


    %%%%%%%%%%%%%%%%%%%%%%%%%%%%%%%%%
    % Cycle 18
    %%%%%%%%%%%%%%%%%%%%%%%%%%%%%%%%%
    \item \textbf{Cycle \theenumi}
    \begin{itemize}
        \item \texttt{SD} (iteration 2) writes the result of the expression \texttt{M[72] $\times$ F2} to memory at address \texttt{72}.
    \end{itemize}
    \begin{table}[!htp]
        \centering
        \begin{tabular}{@{} c l | c c c @{}}
            \toprule
            Iter.       & Instruction                   & Issue & Exec Comp & Write Res \\
            \midrule
            1           & \texttt{LD    F0, 0, R1}      & 1     & 9         & 10        \\ [.3em]
            1           & \texttt{MULTD F4, F0, F2}     & 2     & 14        & 15        \\ [.3em]
            1           & \texttt{SD    F4, 0(R1)}      & 3     & 16        & 17        \\ [.3em]
            2           & \texttt{LD    F0, 0, R1}      & 6     & 10        & 11        \\ [.3em]
            2           & \texttt{MULTD F4, F0, F2}     & 7     & 15        & 16        \\ [.3em]
            2           & \texttt{SD    F4, 0(R1)}      & 8     & 17        & \hl{18}   \\
            \bottomrule
        \end{tabular}
        \caption*{Instruction status.}
    \end{table}

    \begin{table}[!htp]
        \centering
        \begin{tabular}{@{} l | c c c @{}}
            \toprule
                                & Busy      & Addr      & FU   \\
            \midrule
            \texttt{Load1}      & No        &           & \textbackslash                    \\ [.3em]
            \texttt{Load2}      & No        &           & \textbackslash                    \\ [.3em]
            \texttt{Load3}      & No        &           & \textbackslash                    \\
            \cmidrule{1-4}
            \texttt{Store1}     & No        &           &                                   \\ [.3em]
            \texttt{Store2}     & \hl{No}   &           &                                   \\ [.3em]
            \texttt{Store3}     & Yes       & 64        & \texttt{Mult1}                    \\
            \bottomrule
        \end{tabular}
        \caption*{Load and Store Buffers.}
    \end{table}

    \begin{table}[!htp]
        \centering
        \begin{tabular}{@{} l l | c c c c c c @{}}
            \toprule
            Time        & Name              & Busy      & \texttt{Op}           & \texttt{Vj}           & \texttt{Vk}           & \texttt{Qj}           & \texttt{Qk}       \\
            \midrule
                        & \texttt{Add1}     & No        &                       &                       &                       &                       &                   \\ [.3em]
                        & \texttt{Add2}     & No        &                       &                       &                       &                       &                   \\ [.3em]
                        & \texttt{Add3}     & No        &                       &                       &                       &                       &                   \\ [.3em]
            \hl{1}      & \texttt{Mult1}    & Yes       & \texttt{MULTD}        & \texttt{M[64]}        & \texttt{R(F2)}        &                       &                   \\ [.3em]
                        & \texttt{Mult2}    & No        &                       &                       &                       &                       &                   \\
            \bottomrule
        \end{tabular}
        \caption*{Reservation Stations.}
    \end{table}

    \begin{table}[!htp]
        \centering
        \begin{tabular}{@{} c | c | c c c c c c c | c | c @{}}
            \toprule
            Clock       & \texttt{R1}       & \texttt{F0}           & \texttt{F2}   & \texttt{F4}           & \texttt{F6}       & \texttt{F8}   & \texttt{F10}  & \texttt{F12}  & $\dots$   & \texttt{F30}  \\
            \midrule
            \theenumi   & 64                & \texttt{M[64]}        &               & \texttt{Mult1}        &                   &               &               &               &           &               \\
            \bottomrule
        \end{tabular}
        \caption*{Register result status.}
    \end{table}
\end{enumerate}