\subsubsection{Reorder Buffer (ROB)}

\paragraph{Hardware-based Speculation}

Speculation in hardware is a foundational technique in high-performance processors, used to execute instructions \textbf{before it is know whether they are needed}. This is especially relevant when instructions are \textbf{control-dependent} on unresolved branches.

\highspace
In particular, hardware-based speculation enables:
\begin{itemize}
    \item \textbf{Speculative instruction execution} before knowing branch outcomes.
    \item \textbf{Rollback} in case a mispredicted path was taken.
\end{itemize}
This approach increases Instruction-Level Parallelism (ILP), allowing more instructions to be in-flight, even if some turn out to be unnecessary.

\highspace
\begin{flushleft}
    \textcolor{Green3}{\faIcon{question-circle} \textbf{Why speculation needs the ROB}}
\end{flushleft}
Executing instructions speculatively introduces a challenge: ``\emph{how do we prevent side effects (e.g., register or memory updates) from mispredicted instructions?}''. The \textbf{solution}: \textbf{defer the architectural state update} until it's safe to commit.

\highspace
This is exactly what the \definition{Reorder Buffer (ROB)} is for:
\begin{itemize}[label=\textcolor{Green3}{\faIcon{check}}]
    \item It holds results of instructions that have \textbf{finished execution} but are \textbf{not yet committed}.
    \item It allows \textbf{instruction results to be written in-order}, preserving program semantics.
    \item In case of a misprediction, the ROB enables the processor to \textbf{flush invalid speculative results} quickly and precisely.
\end{itemize}
But the ROB is \textbf{critical} for:
\begin{itemize}[label=\textcolor{Red2}{\faIcon{exclamation-triangle}}]
    \item Decoupling \textbf{execution completion} from \textbf{state update (commit)}.
    \item Supporting \textbf{precise exceptions} (so the CPU can cleanly stop at the last correct instruction).
    \item Managing \textbf{speculative and non-speculative instruction tracking}.
\end{itemize}