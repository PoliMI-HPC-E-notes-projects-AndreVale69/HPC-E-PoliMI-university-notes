\paragraph{ROB as a Data Communication Mechanism}

In Tomasulo's original algorithm, \textbf{Reservation Stations (RSs)} handled register renaming and operand forwarding. However, with the introduction of the ROB, it \textbf{replaces the renaming and forwarding function of RSs}. Before we explain how, we need to understand some \emph{basic} concepts of ROB.

\highspace
\begin{flushleft}
    \textcolor{Green3}{\faIcon{question-circle} \textbf{What are ROB numbers?}}
\end{flushleft}
A \definition{ROB number} is simply an \textbf{index} (or tag, in Tomasulo's algorithm) \textbf{identifying} a specific \textbf{entry in the Reorder Buffer}. Each instruction that is issued receives a \underline{unique} ROB number that corresponds to its place in the ROB, its \textbf{slot identifier} (slot ID).

\highspace
The ROB number is then used in two key ways:
\begin{enumerate}
    \item \important{As a destination tag (Register Renaming)}. When an \textbf{instruction} is issued and it \textbf{will write to a register}:
    \begin{itemize}
        \item The destination register is \textbf{not renamed to another physical register} (like in a pure register renaming scheme).
        \item Instead, the \textbf{architectural register is mapped to the instruction's ROB number}.
    \end{itemize}
    It is the \emph{identical} logic of tag in Tomasulo algorithm.

    \item \important{For Operand Dependency Resolution}. Later \textbf{instructions that depend on the result} of the ROB number do not need to stall:
    \begin{itemize}
        \item They \textbf{record the ROB number} as the operand source.
        \item Once the ROB number becomes ``ready'' (i.e., the value is written to the ROB), dependent instructions can \textbf{read the value from the ROB number} and proceed to execution.
    \end{itemize}
    In tomasulo algorithm this feature is provided by the CDB, where each instruction listens to that because it waits for the result of the tag.

    As happens in tomasulo algorithm, this mechanism:
    \begin{itemize}[label=\textcolor{Green3}{\faIcon{check}}]
        \item Avoids \textbf{WAR and WAW hazards}.
        \item Enables \textbf{data forwarding} even before instructions commit.
    \end{itemize}
\end{enumerate}

\begin{table}[!htp]
    \centering
    \begin{tabular}{@{} l | l @{}}
        \toprule
        Tomasulo & ROB-based Tomasulo \\
        \midrule
        Tags $=$ RS entry IDs       & Tags $=$ ROB entry numbers            \\ [.3em]
        Values broadcast via CDB    & Same, but identified by ROB number    \\ [.3em]
        RS buffers result locally   & ROB stores result globally            \\ [.3em]
        Commit on write-back        & Commit delayed and via ROB head       \\
        \bottomrule
    \end{tabular}
    \caption{Quick comparison between Tomasulo with and without ROB.}
\end{table}

\highspace
\begin{flushleft}
    \textcolor{Green3}{\faIcon{tools} \textbf{Updated Role of Reservation Stations}}
\end{flushleft}
The ROB not only buffers instruction results but also \textbf{propagates those results to dependent instructions} as soon as they are ready, even if not yet committed. So, with the ROB managing renaming and data forwarding:
\begin{itemize}
    \item[\textcolor{Red2}{\faIcon{times}}] \hl{Reservation Stations} are \textbf{no longer responsible for naming/tagging}.
    \item[\textcolor{Green3}{\faIcon{check}}] Their role is now focused on:
    \begin{itemize}
        \item \textbf{Buffering decoded instructions} before they're issued to the Functional Units (FUs);
        \item \textbf{Holding operand values} (or ROB tags) temporarily;
        \item Helping reduce \textbf{structural hazards}.
    \end{itemize}
\end{itemize}
\hl{Reservation Stations} now act like \textbf{staging areas}, not tracking results or resolving dependencies directly.

\begin{table}[!htp]
    \centering
    \begin{tabular}{@{} l | l | l @{}}
        \toprule
        Function                        & Tomasulo & ROB-based Tomasulo \\
        \midrule
        Register Renaming               & RS       & ROB                \\ [.3em]
        Data forwarding                 & RS       & ROB                \\ [.3em]
        Instruction buffering           & RS       & RS (same)          \\ [.3em]
        Operand availability tracking   & RS       & RS (via ROB tags)  \\ [.3em]
        \bottomrule
    \end{tabular}
    \caption{Summary of architectural changes.}
\end{table}

\noindent
In modern speculative Tomasulo architectures, the ROB becomes the central structure for both result tracking and inter-instruction communication. RSs are demoted to lightweight instruction and operand buffers.