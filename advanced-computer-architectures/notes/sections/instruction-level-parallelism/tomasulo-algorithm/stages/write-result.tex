\paragraph{Stage 3: Write Result}

This is the \textbf{moment the instruction finishes its computation} in the Functional Unit (FU), and the result is made \textbf{globally visible} by broadcasting it over the \textbf{Common Data Bus (CDB)}. It's the stage where:
\begin{itemize}
    \item The \textbf{value computed by the FU} is sent to:
    \begin{itemize}
        \item All \textbf{Reservation Stations (RSs)} waiting for it;
        \item The \textbf{Register File (RF)} (if still tagged with RS).
    \end{itemize}
    \item The instruction is now officially ``\textbf{done}''.
\end{itemize}

\highspace
\begin{flushleft}
    \textcolor{Green3}{\faIcon{tools} \textbf{What happens step-by-step}}
\end{flushleft}
Let's say we have this instruction in Reservation Station 3 (\texttt{RS3}):
\begin{lstlisting}[language=unknown]
MUL.D F6, F2, F4
\end{lstlisting}
\texttt{RS3} starts execution in the \texttt{MUL} unit, takes several cycles, and finally finishes. Now we begin:
\begin{enumerate}
    \item \important{Broadcast on CDB}. The Functional Unit (FU) places the result on the \textbf{Common Data Bus (CDB) along with its tag (\texttt{RS3})}:
    \begin{equation*}
        \texttt{CDB} \leftarrow \left\langle \texttt{tag} = \texttt{RS3}, \texttt{value} = \texttt{result} \right\rangle 
    \end{equation*}
    This is the Tomasulo version of a ``public announcement'': ``\emph{\texttt{RS3} has finished computing a result. Anyone waiting for this, come get it!}''.
    
    \item \important{Reservation Stations listen for Tags}. Every Reservation Station (RS) in the system checks: ``\emph{does my \texttt{Qj} or \texttt{Qk} match the tag on the CDB?}''. If yes:
    \begin{itemize}
        \item The Reservation Station (RS) grabs the \textbf{value}.
        \item Stores it in \texttt{Vj} or \texttt{Vk}.
        \item Clears the tag:
        \begin{equation*}
            \texttt{Qj} \leftarrow 0 \hspace{1em} \lor \hspace{1em} \texttt{Qk} \leftarrow 0 
        \end{equation*}
        Meaning: operand is now ready.
    \end{itemize}
    This process allows many instructions to \textbf{simultaneously wake up} when the result they needed finally becomes available.
    
    \item \important{Register File Update}. The \textbf{Register File} checks: ``\emph{is there any register whose \texttt{Qi = RS3}?}''. If yes:
    \begin{itemize}
        \item It writes the result into that register.
        \item Clears the tag field (\texttt{Qi = 0}), this means ``\texttt{F6} is now valid and available''.
    \end{itemize}
    If the register has already been renamed again (e.g., \texttt{Qi = RS4}), we \textbf{don't write} to it, this avoid \textbf{WAW hazards}.

    \item \important{Free the RS and FU}. The instruction in \texttt{RS3} is now complete, \texttt{RS3} is marked free, and the Functional Unit (FU) becomes available again for other instructions.
\end{enumerate}

\begin{examplebox}[: why don't we write to a register when it has been renamed?]
    Suppose we have two instructions writing to the \textbf{same register}:
    \begin{lstlisting}[language=unknown]
ADD.D  F6, F2, F4    ; instruction A
MUL.D  F6, F6, F8    ; instruction B\end{lstlisting}
    Instruction 1 is issued first, but instruction 2 also \textbf{writes to \texttt{F6}} (again). This is a \textbf{Write After Write (WAW)} situation.
    \begin{enumerate}
        \item Instruction 1 (\texttt{ADD.D}) is issued $\rightarrow$ assigned to \texttt{RS1}
        \begin{itemize}
            \item Tomasulo sets: \texttt{F6.Qi} $\leftarrow$ \texttt{RS1} (``\emph{\texttt{F6} will come from \texttt{RS1}}'').
        \end{itemize}
        \item Instruction 2 (\texttt{MUL.D}) is issued \textbf{immediately after} $\rightarrow$ assigned to \texttt{RS2}
        \begin{itemize}
            \item Tomasulo updates again: \texttt{F6.Qi} $\leftarrow$ \texttt{RS2} (``\emph{\texttt{F6} will now come from \texttt{RS2}}'').
        \end{itemize}
    \end{enumerate}
    Now we have:
    \begin{itemize}
        \item Two instructions writing to \texttt{F6}.
        \item But \textbf{only the result of \texttt{RS2} should actually end up in \texttt{F6}}.
        \item \textbf{\texttt{RS1}'s result is now obsolete}, because it was overwritten by a newer instruction.
    \end{itemize}
    \textbf{When \texttt{RS1} finishes (WRITE STAGE)}, \texttt{RS1} broadcasts its result on the CDB:
    \begin{equation*}
        \left\langle \texttt{Tag} = \texttt{RS1}, \texttt{Value} = \dots \right\rangle
    \end{equation*}
    Tomasulo checks the Register File: ``\emph{does any register have \texttt{Qi = RS1}?}'':
    \begin{itemize}
        \item[\textcolor{Green3}{\faIcon{check-circle}}] If \textcolor{Green3}{\textbf{yes}}, update that register.
        \item[\textcolor{Red2}{\faIcon{times-circle}}] If \textcolor{Red2}{\textbf{no}} (this case), \texttt{F6.Qi = RS2}, not \texttt{RS1} anymore.
    \end{itemize}
    \textbf{So we skip the write}. Because if we allowed \texttt{RS1} to write to \texttt{F6} now, it would \textbf{overwrite the result of \texttt{RS2}}, which hasn't been computed yet, and then we'd break correctness.

    The is how \textbf{Tomasulo avoids WAW hazards}: only write to a register if it's still waiting for our result. If someone newer came along and renamed the register again, we're out. Our value is no longer needed in the register file.
\end{examplebox}

\begin{itemize}
    \item CDB allows \textbf{many instructions} to grab the result \textbf{at the same time}.
    \item Values are \textbf{forwarded before being stored} (no need to wait for register write-back).
    \item \textbf{False dependencies (WAR/WAW)} are avoided automatically.
    \item \textbf{RAW dependencies} are resolved \textbf{dynamically}, as instructions ``listen'' for the values they need.
\end{itemize}