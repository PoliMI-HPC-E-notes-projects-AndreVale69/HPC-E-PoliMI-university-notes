\subsection{Classic Multiprocessor Topologies}

A \textbf{topology} is the way processors and memories are \textbf{interconnected} in a multiprocessor. It defines:
\begin{itemize}
    \item \textbf{How processors communicate} (direct neighbors vs. global links).
    \item \textbf{How scalable the system is} (bandwidth, latency).
    \item \textbf{How much it costs} (wiring, switches).
\end{itemize}
``Classic'' refers to the \textbf{fundamental interconnection structures} studied since the 70s-90s and still forming the basis of today's multicore \textbf{Networks-on-Chip (NoCs)} and HPC interconnects.

\highspace
The five canonical interconnection structures used to classify and reason about multiprocessor networks are as follows:
\begin{enumerate}
    \item \definition{Single Bus}: All processors connected to one shared bus (one link).
    \begin{itemize}
        \item \important{Total Bandwidth}
        \begin{equation}
            \text{BW}_{\text{total}} = b
        \end{equation}
        \item \important{Bisection Bandwidth}
        \begin{equation}
            \text{BW}_{\text{bisect}} = b
        \end{equation}
    \end{itemize}
    For example, with 4 processors and $b = 1$ GB/s, both total and bisection bandwidth is 1 GB/s.
    \begin{itemize}
        \item[\textcolor{Red2}{\faIcon{times}}] Very low scalability.
        \item[\textcolor{Red2}{\faIcon{times}}] Saturates quickly.
    \end{itemize}
    \begin{figure}[!htp]
        \centering
        \includegraphics[width=.6\textwidth]{img/single-bus.pdf}
        \caption{Single Bus Architecture.\cite{course-slides-polimi}}
    \end{figure}

    \newpage

    \item \definition{Ring}: Each processor connected to two neighbors in a loop.
    \begin{itemize}
        \item \important{Total Bandwidth}
        \begin{equation}
            \text{BW}_{\text{total}} = P \cdot b
        \end{equation}
        \item \important{Bisection Bandwidth}
        \begin{equation}
            \text{BW}_{\text{bisect}} = 2b
        \end{equation}
    \end{itemize}
    For example 8-node ring, $b = 1$ GB/s, total BW $=8$ GB/s, bisection BW $=$ 2 GB/s. 
    \begin{itemize}
        \item[\textcolor{Green3}{\faIcon{check}}] Good best-case parallelism.
        \item[\textcolor{Red2}{\faIcon{times}}] Worst-case is only $2\times$ bus.
    \end{itemize}

    \item \definition{Crossbar (Fully Connected Switch Network)}: Each processor has a dedicated bidirectional link to every other.
    \begin{itemize}
        \item \important{Total Bandwidth}
        \begin{equation}
            \text{BW}_{\text{total}} = \dfrac{P \left(P-1\right)}{2} \cdot b
        \end{equation}
        \item \important{Bisection Bandwidth}
        \begin{equation}
            \text{BW}_{\text{bisect}} = \left(\dfrac{P}{2}\right)^{2} \cdot b
        \end{equation}
    \end{itemize}
    For example, $P = 4$, $b = 1$, total BW is 6 and bisection BW is 4.
    \begin{itemize}
        \item[\textcolor{Green3}{\faIcon{check}}] Non-blocking.
        \item[\textcolor{Green3}{\faIcon{check}}] High performance.
        \item[\textcolor{Red2}{\faIcon{times}}] Very high cost (scales $P^2$).
    \end{itemize}
    
    \item \definition{2-D Mesh}: Processors arranged in an $\sqrt{P} \times \sqrt{P}$ grid.
    \begin{itemize}
        \item \important{Total Bandwidth}
        \begin{equation}
            \text{BW}_{\text{total}} = 2N(N-1) \cdot b \quad \text{with } N=\sqrt{P}
        \end{equation}
        \item \important{Bisection Bandwidth}
        \begin{equation}
            \text{BW}_{\text{bisect}} = N \cdot b
        \end{equation}
    \end{itemize}
    For example, $P = 16$ ($N =4$), $b = 1$, total BW is 24 and bisection BW is 4.
    \begin{itemize}
        \item[\textcolor{Green3}{\faIcon{check}}] Good scalability.
        \item[\textcolor{Green3}{\faIcon{check}}] Cost-effective.
        \item[\textcolor{Red2}{\faIcon{times}}] Multiple hops increase latency.
    \end{itemize}
    \newpage
    \begin{figure}[!htp]
        \centering
        \includegraphics[width=.3\textwidth]{img/2d-mesh.pdf}
        \caption{2-D Mesh Architecture.\cite{course-slides-polimi}}
    \end{figure}
    
    \item \definition{Hypercube (N-cube)}: Boolean N-cube, with $P = 2^{N}$ nodes. Each node connects to $N$ neighbors.
    \begin{itemize}
        \item \important{Total Bandwidth}
        \begin{equation}
            \text{BW}_{\text{total}} = \dfrac{N \cdot P}{2} \cdot b
        \end{equation}
        \item \important{Bisection Bandwidth}
        \begin{equation}
            \text{BW}_{\text{bisect}} = 2(N-1) \cdot b
        \end{equation}
    \end{itemize}
    For example:
    \begin{itemize}
        \item $P = 4$ ($N=2$): $\text{total} = 4b$, $\text{bisection} = 2b$.
        \item $P = 8$ ($N=3$): $\text{total} = 12b$, $\text{bisection} = 4b$.
    \end{itemize}
    \begin{itemize}
        \item[\textcolor{Green3}{\faIcon{check}}] Very good connectivity.
        \item[\textcolor{Green3}{\faIcon{check}}] Logarithmic diameter.
        \item[\textcolor{Red2}{\faIcon{times}}] Wiring complexity grows.
    \end{itemize}
    \begin{figure}[!htp]
        \centering
        \includegraphics[width=.3\textwidth]{img/hypercube.pdf}
        \caption{Hypercube Architecture.\cite{course-slides-polimi}}
    \end{figure}
\end{enumerate}