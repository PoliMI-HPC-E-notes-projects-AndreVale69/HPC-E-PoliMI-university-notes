\subsection{From single bus to interconnection networks}

The \textbf{bus} was sufficient for early multiprocessors (workstation, small servers). For scaling (tens/hundreds of processors), the \textbf{interconnection network} is mandatory. The ``sweet spot'' explains why small SMPs (e.g., 4-16 cores) dominated desktops/servers, while large-scale HPC systems adopt meshes, tori, and fat trees.

\highspace
\begin{flushleft}
    \textcolor{Green3}{\faIcon{bus-alt} \textbf{Single Bus Multiprocessors}}
\end{flushleft}
In \definition{Single Bus Multiprocessors}, all processors and memory modules are connected via a \textbf{single bus}. Therefore, every memory access uses the bus, and \textbf{all processors compete for the same shared medium}. The \textbf{main limitation is \underline{saturation}}. Commercial systems maxed out at $\approx 36$ processors before the bus saturated. Beyond that point, performance \emph{flattens} because adding CPUs doesn't help; the bus is the bottleneck.

\highspace
\begin{flushleft}
    \textcolor{Green3}{\faIcon{network-wired} \textbf{Network-Connected Multiprocessors}}
\end{flushleft}
In \definition{Network-Connected Multiprocessors}, each processor has \textbf{its onw local memory}. The \textbf{interconnection network} is used \emph{only} for \textbf{inter-processor communication}, not for every single load/store. The main \textbf{advantage} is that it removes the shared bus from the critical path and can \textbf{scale to many more processors}. We will examine some of these topologies, such as rings, meshes, hypercubes, and crossbars.

\highspace
\begin{flushleft}
    \textcolor{Red2}{\faIcon{balance-scale} \textbf{Cost/Performance Trade-offs}}
\end{flushleft}
\begin{itemize}
    \item \important{Bus-based MPs}
    \begin{itemize}
        \item Cheap for small $N$ (2-8 CPUs).
        \item Linear performance growth until the bus saturates.
        \item \textbf{Sweet spot}: 8-16 processors, after that, performance plateaus regardless of added CPUs.
    \end{itemize}
    \item \important{Network-based MPs}
    \begin{itemize}
        \item Initially costlier (switches, links).
        \item Scale better: performance per processor stays roughly constant as $N$ grows.
        \item Always win once the bus reaches its plateau.
    \end{itemize}
\end{itemize}