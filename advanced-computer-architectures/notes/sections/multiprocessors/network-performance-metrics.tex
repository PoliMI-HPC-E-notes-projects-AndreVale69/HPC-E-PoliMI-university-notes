\subsection{Network performance metrics}

When comparing interconnection networks (bus, ring, mesh, hypercube, etc.), we need \textbf{quantitative metrics} to evaluate scalability. The two canonical ones are:
\begin{enumerate}
    \item \definition{Total Network Bandwidth} (Best Case): the \textbf{sum of the bandwidths of all links} in the network.
    \begin{equation}
        \text{Total BW} = M \times b
    \end{equation}
    Where:
    \begin{itemize}
        \item $M$ is the number of links.
        \item $b$ is the bandwidth of one link.
    \end{itemize}
    For example, the total bandwidth of a single bus is equal to the bandwidth of one link. Total Network Bandwidth represents the \hl{maximum aggregate throughput that the network could deliver} if all links were used simultaneously and were perfectly balanced.
    
    \item \definition{Bisection Bandwidth} (Worst Case): the \textbf{minimum total bandwidth of links that must be cut} to split the network into two equal halves (each with half the nodes).
    \begin{equation}
        \text{Bisection BW} = \min \displaystyle\sum b_{\text{crossing links}}
    \end{equation}
    It is the \textbf{worst-case metric} because, for \textbf{asymmetric} topologies, we have to \textbf{try all possible cuts and select the smallest one}. For example:
    \begin{itemize}
        \item \textbf{Single bus}: splitting cuts just the bus $\rightarrow$ $b$.
        \item \textbf{Ring}: any cut crosses 2 links $\rightarrow$ $2b$.
        \item \textbf{Crossbar} ($P$ nodes): $\left(\dfrac{P}{2}\right)^2 \times b$.
        \item \textbf{2D Mesh} ($P = N^{2}$ nodes): $N \times b$.
        \item \textbf{Hypercube} ($P = 2^{N}$): $2\left(N-1\right) \times b$.
    \end{itemize}
    It measures \emph{worst-case throughput} when half the processors communicate with the other half.
\end{enumerate}

\highspace
\begin{flushleft}
    \textcolor{Green3}{\faIcon{balance-scale} \textbf{Best vs. Worst Case}}
\end{flushleft}
\begin{itemize}
    \item[\textcolor{Green3}{\faIcon{check}}] \textbf{Total bandwidth (best case)}: optimistic, assumes all links are equally used.
    \item[\textcolor{Red2}{\faIcon{times}}] \textbf{Bisection bandwidth (worst case)}: pessimistic, captures global contention.
\end{itemize}
In practice, a bus topology is sufficient for \textbf{small SMPs} (it is low cost, but has low bisection bandwidth). For \textbf{large-scale systems}, however, one must choose a topology with a higher bisection bandwidth, such as meshes, hypercubes, or fat trees, to avoid bottlenecks.