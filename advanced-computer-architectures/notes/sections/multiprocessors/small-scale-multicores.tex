\subsection{Small-scale multicores}

\definition{Small-scale multicore} processors are multicore CPUs with a \textbf{relatively small number of cores} (typically up to 8, sometimes 16). They implement a \definition{Symmetric Multiprocessor (SMP)} \textbf{model on a single chip}:
\begin{itemize}
    \item All cores are identical (symmetric).
    \item All cores have \textbf{uniform access} to a centralized main memory (UMA).
    \item The OS and software see them as a shared-memory machine.
\end{itemize}
Most commercial multicore processors today, including those from Intel, AMD, and ARM, as well as System on a Chip (SoC) processors, are Small-scale Symmetric Multiprocessing (SMP) systems and have these features:
\begin{itemize}
    \item Less than 8 cores (typical laptops, desktops).
    \item Centralized shared memory (UMA model).
    \item One shared L3 cache and private L1 and L2 per core.
\end{itemize}
Each processor core is identical and symmetric: any core can execute any thread, and all have equal access to memory.

\highspace
\begin{flushleft}
    \textcolor{Green3}{\faIcon{microchip} \textbf{Cache hierarchy organization}}
\end{flushleft}
\begin{itemize}
    \item \textbf{Private L1 and L2 caches}. Very small, per-core and ultra-low latency. Store local working sets and reduce contention\footnote{%
        \definition{Contention} means that \textbf{two or more processors (or components) try to use the same resource at the same time}. Since the resource is finite (bus, memory bank, cache port, interconnect link), they cannot all proceed simultaneously, but some must \textbf{wait}. This waiting increases \textbf{latency} and reduces \textbf{throughput}.
    }.
    \item \textbf{Shared L3 cache}. A larger, slower cache accessible by all cores. Reduces memory traffic to DRAM and acts as a buffer for inter-core communication: if Core A writes to shared data, Core B can see it via L3 (with coherence enforced).
\end{itemize}
\textcolor{Green3}{\faIcon{book} \textbf{Cache coherence: Snooping.}} Since multiple caches may hold copies of the same memory block, coherence is needed. One common protocol, which will be discussed later, is called the snooping protocol. \textbf{Snooping protocols} are typically used for small-scale SMP:
\begin{itemize}
    \item Each cache controller ``snoops'' (monitors) the \textbf{shared bus or interconnect}.
    \item When one core issues a read/write, others observe it and update/invalidate their cache lines accordingly.
    \item Works well for \textbf{small numbers of cores} ($\le 8$), where bus-based broadcasting is affordable.
\end{itemize}

\highspace
\begin{flushleft}
    \textcolor{Green3}{\faIcon{\speedIcon} \textbf{Scaling beyond small-scale: banked caches \& NoCs}}
\end{flushleft}
As the number of cores grows ($> 8$, toward manycores), a \textbf{single shared L3} or memory controller becomes a bottleneck. To scale, designers use \textbf{banked caches}:
\begin{itemize}
    \item L3 is partitioned into several \textbf{banks}, each connected to different parts of the chip.
    \item Improves \textbf{aggregate bandwidth}, since multiple banks can serve requests in parallel.
\end{itemize}
The interconnect evolves into a \textbf{Network-on-Chip (NoC)}. Instead of a single bus, cores and cache banks are linked by mesh, ring, or crossbar topologies. NoC allows scalable communication without saturating a single bus. These types of protocols will be covered in future sections.