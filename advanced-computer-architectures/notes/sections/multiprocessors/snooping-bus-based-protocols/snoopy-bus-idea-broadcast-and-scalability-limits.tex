\subsection{Snooping (bus-based) Protocols}

\subsubsection{Snoopy bus idea, broadcast, and scalability limits}

In early multiprocessors, all CPUs and memory are connected via a \textbf{single bus}. Every \textbf{cache controller} ``snoops'' (listens) to the bus: if another processor issues a \textbf{read or write request} for a block that the cache holds, the snooper reacts appropriately (invalidate, supply data, etc.). In other words: caches \textbf{cooperate} by monitoring the bus traffic, so that all copies of a memory block can be kept coherent. This is why it's called a \definition{Snooping Protocol}, because each cache controller acts like a ``spy'' on the bus.

\highspace
\begin{flushleft}
    \textcolor{Green3}{\faIcon{network-wired} \textbf{Broadcast mechanism}}
\end{flushleft}
The bus is a \textbf{broadcast medium} because any request placed on it is seen by \textbf{all processors} simultaneously. For example, when the processor \code{P1} writes to address \code{X}, the bus carries the request, and all other caches check if they have \code{X}, they \textbf{invalidate} or \textbf{update} depending on the protocol. This \textbf{broadcast simplicity} make snooping straightforward:
\begin{itemize}
    \item \textbf{Only one bus transaction} is needed for all caches to know what's happening.
    \item No separate ``directory'' or centralized record is required.
\end{itemize}

\highspace
\begin{flushleft}
    \textcolor{Red2}{\faIcon{exclamation-triangle} \textbf{Scalability limits}}
\end{flushleft}
Snooping is elegant but doesn't scale well:
\begin{enumerate}
    \item \textbf{Bus bandwidth bottleneck}. Every coherence transaction is broadcast to \textbf{all processors}. With more processors, bus traffic grows quickly, saturating available bandwidth.
    \item \textbf{Electrical limits}. A single shared bus has \textbf{limited speed} and \textbf{fan-out}\footnote{%
        \definition{Fan-out} is the \textbf{number of loads on the bus}. It is the \textbf{electrical loading} on the bus. In cache context, each cache controller (and memory module) connected to the shared bus is a \textbf{receiver}. The more devices we attach, the \textbf{greater the capacitance} and resistance that the driving signal must overcome. This increases the \textbf{propagation delay} of the bus: the time it takes for a transition at point A to be seen cleanly at the farthest point B.
    } (signal must reach all caches). Adding more processors increases capacitance, then slower bus cycles.
    \item \textbf{Serialization}. Bus provides a single global order (all requests are serialized). This is good for correctness but bad for performance since no parallel coherence actions are possible.
\end{enumerate}
In practice, snooping works well for \textbf{small-scale simultaneously multiprocessors (SMPs)}, like 2-8 CPUs. For larger systems (tens or hundreds of cores), it becomes inefficient; that's why \textbf{directory protocols} (covered later) are needed.

\highspace
In summary, snooping leverages the \textbf{shared bus as a natural broadcast medium}. Each cache controller \textbf{snoops} to maintain coherence. It's \textbf{simple and effective}, but \textbf{not scalable}: bus traffic and electrical limits make it suitable only for small multiprocessors.