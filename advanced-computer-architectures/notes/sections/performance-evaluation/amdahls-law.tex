\subsection{Amdahl's Law}

\definition{Amdahl's Law} describes the \textbf{limits} of performance improvement when only \textbf{part of a system} is enhanced. Key principles are:
\begin{itemize}
    \item \textbf{Speedup} of a program is \textbf{limited} by the part of the program that \textbf{cannot be improved}.
    \item \textbf{Making the common case fast} gives the best returns.
\end{itemize}
Thus, \textbf{no matter how much we optimize a part of a system}, the total speedup is \textbf{bounded}.

\highspace
\begin{flushleft}
    \textcolor{Green3}{\faIcon{book} \textbf{Formal Definition}}
\end{flushleft}
Let:
\begin{itemize}
    \item $F$: fraction of execution time \textbf{affected} by the enhancement.
    \item $S$: speedup \textbf{factor} for the enhanced part.
\end{itemize}
Then:
\begin{itemize}
    \item \textbf{New Execution Time}:
    \begin{equation}
        \text{ExTime(w/ E)} = \left( (1-F) + \dfrac{F}{S} \right) \times \text{ExTime(w/o E)}
    \end{equation}
    Where:
    \begin{itemize}
        \item ``w/ E'' $=$ ``\emph{with Enhancement}''
        \item ``w/o E'' $=$ ``\emph{without Enhancement}''
    \end{itemize}
    \item \textbf{Overall Speedup}:
    \begin{equation}\label{eq:amdahls-law}
        \text{Speedup(E)} = \frac{1}{(1-F) + \frac{F}{S}}
    \end{equation}
    Where:
    \begin{itemize}
        \item $\left(1-F\right)$ is a portion of execution time \textbf{not improved}.
        \item $\frac{F}{S}$ is the improved portion, now \textbf{accelerated}.
    \end{itemize}
\end{itemize}
In many books, we also see:
\begin{itemize}
    \item \textbf{FractionE}: fraction of computation time accelerated.
    \item \textbf{SpeedupE}: speedup factor for the enhanced portion.
\end{itemize}
Thus:
\begin{equation}
    \text{Speedup}_{\text{overall}} = \frac{1}{(1 - \text{FractionE}) + \dfrac{\text{FractionE}}{\text{SpeedupE}}}
\end{equation}

\begin{examplebox}[: Amdahl's Law]
    Let us consider an enhancement for a CPU resulting 10 time faster on computation than the original one but the original CPU is busy with computation only 40\% of the time. What is the overall speedup gained by introducing the enhancement?

    \highspace
    Given:
    \begin{itemize}
        \item FractionE: 0.4 (40\% of the time can be accelerated).
        \item SpeedupE: 10 (enhancement is 10x faster)
    \end{itemize}
    Apply Amdahl's Law:
    \begin{equation*}
        \begin{array}{rcl}
            \text{Speedup}_{\text{overall}} &=& \dfrac{1}{(1-0.4) + \dfrac{0.4}{10}} \\ [1.5em]
            &=& \dfrac{1}{0.6 + 0.04} \\ [1.5em]
            &=& \dfrac{1}{0.64} \\ [1.5em]
            &=& 1.56
        \end{array}
    \end{equation*}
    Thus, even if the improvement is 10x, the overall speedup is only about 1.56x.
\end{examplebox}

\highspace
If we \textbf{only speed up a small part} (small $F$), the \textbf{impact is small}. To achieve \textbf{large speedup}, we must:
\begin{itemize}
    \item Speed up a \textbf{large fraction} of the execution time.
    \item Or make the enhancement extremely fast ($S \rightarrow \infty$).
\end{itemize}
To reach the maximum theoretical speedup, if $S \to \infty$ (perfect acceleration), then:
\begin{equation}
    \text{Speedup}_{\max} = \frac{1}{1-F}
\end{equation}
For example if only 40\% can be improved, even with infinite speedup, the maximum is:
\begin{equation*}
    \text{Speedup}_{\max} = \frac{1}{1-0.4} = \frac{1}{0.6} \approx 1.67
\end{equation*}
This shows \textbf{the real bottleneck}: the non-improved part.