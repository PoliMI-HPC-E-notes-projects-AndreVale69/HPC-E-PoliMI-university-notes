\section{Performance Evaluation}

All formulas and metrics can be found in the Index section, page \hqpageref{index}, under the word F (Formula).

\subsection{Basic Concepts and Performance Metrics}

Performance can be considered from two key viewpoints:
\begin{itemize}
  \item \textbf{Purchasing Perspective}. Given multiple machines, the goal is to determine:
  \begin{itemize}
    \item \emph{Which machine has the \textbf{best performance}.}
    \item \emph{Which machine has the \textbf{lowest cost}.}
    \item \emph{Which machine offers the \textbf{best performance-to-cost ratio}.}
  \end{itemize}
  
  \item \textbf{Design Perspective}. When facing different architectural or technological options, the aim is to identify:
  \begin{itemize}
    \item \emph{Which design achieves the \textbf{greatest performance improvement}.}
    \item \emph{Which design \textbf{minimizes cost}.}
    \item \emph{Which design \textbf{optimize the performance-to-cost balance}.}
  \end{itemize}
\end{itemize}
\textbf{Both perspective} require a \textbf{basic for comparison} and \textbf{clear performance metrics}. The main goal is to understand how architectural choices impact both \textbf{performance} and \textbf{cost}.

\highspace
\begin{flushleft}
  \textcolor{Green3}{\faIcon{bookmark} \textbf{Two Fundamental Notions of Performance}}
\end{flushleft}
\begin{enumerate}
  \item \definition{Execution Time (Latency, Response Time)}: refers to the \textbf{time needed to complete a task}.
  
  \hl{Lower} execution time $=$ \hl{better}.

  
  \item \definition{Throughput (Bandwidth)}: measures the \textbf{number of tasks completed per unit of time} (e.g., jobs per hour).
  
  \hl{Higher} throughput $=$ \hl{better}.
\end{enumerate}
Response time and throughput often \textbf{oppose} each other. Optimizing for one can degrade the other.

\begin{examplebox}[: Planes Comparison, Concorde vs Boeing 747]
  Concorde is a supersonic airliner.
  \begin{itemize}
    \item \textbf{Speed}: Concorde is faster (1350 mph vs 610 mph).
    \item \textbf{Execution Time}: Concorde takes less time to fly Washington DC to Paris (3h vs 6.5h).
    \item \textbf{Throughput}: Boeing transports more passengers per mile (higher ``passenger-miles per hour'').
  \end{itemize}
\end{examplebox}

\newpage

\begin{flushleft}
  \textcolor{Green3}{\faIcon{book} \textbf{Basic Definitions and Calculations}}
\end{flushleft}
\begin{itemize}
  \item \definitionWithSpecificIndex{Performance Formula}{Formula: Performance}{}
  \begin{equation}\label{eq: performance}
    \text{Performance} = \dfrac{1}{\text{Execution Time}}
  \end{equation}

  \item \definitionWithSpecificIndex{Relative Performance}{Formula: Relative Performance}{}. If $X$ is $n$\% faster than $Y$, then:
  \begin{equation}\label{eq: performance X}
    \text{Performance}\left(X\right) = \left(
      1 + \dfrac{n}{100}
    \right) \times \text{Performance}\left(Y\right)
  \end{equation}
  Or:
  \begin{equation}\label{eq: execution time Y}
    \text{Execution Time}\left(Y\right) = \left(
      1 + \dfrac{n}{100}
    \right) \times \text{Execution Time}\left(X\right)
  \end{equation}
\end{itemize}

\begin{examplebox}
  If machine A executes a program in 10 sec and machine B executes same program in 15 sec: A is 50\% faster than B or A is 33\% faster than B?

  Using equation \ref{eq: execution time Y}, if A is 50\% faster than B, then it should be true that:
  \begin{equation*}
    \begin{array}{rcl}
      \text{Execution Time}\left(B\right) &=& \left(
        1 + \dfrac{n}{100}
      \right) \times \text{Execution Time}\left(A\right) \\ [1.5em]
      15 &=& \left(
        1 + \dfrac{50\%}{100}
      \right) \times 10 \\ [1.5em]
      15 &=& \dfrac{3}{2} \times 10 \\ [1.5em]
      15 &=& \dfrac{30}{2} \: \text{\textcolor{Green3}{\faIcon{check}}}
    \end{array}
  \end{equation*}
\end{examplebox}

\begin{flushleft}
  \textcolor{Green3}{\faIcon{clock} \textbf{Clock Cycles and Clock Frequency}}
\end{flushleft}
Modern processors operate by executing instructions synchronously according to a \textbf{clock signal}:
\begin{itemize}
  \item \definitionWithSpecificIndex{Clock Cycle Time $T_{\text{CLK}}$}{Formula: Clock Cycle Time $T_{\text{CLK}}$}{}: the \textbf{duration} of a single clock cycle, measured in seconds (s).

  \item \definitionWithSpecificIndex{Clock Frequency $f_{\text{CLK}}$}{Formula: Clock Cycle Time $f_{\text{CLK}}$}{}: the \textbf{number of clock cycles per second}, measured in Hertz (Hz). Defined as:
  \begin{equation}
    f_{\text{CLK}} = \dfrac{1}{T_{\text{CLK}}}
  \end{equation}
  Some examples:
  \begin{itemize}
    \item $f_{\text{CLK}} = 500\, \text{MHz}$ $\Leftrightarrow$ $T_{\text{CLK}} = \dfrac{1}{500 \times 10^{6}} = \dfrac{1}{500'000'000} = 2\, \text{ns}$
    \item $f_{\text{CLK}} = 1\, \text{GHz}$ $\Leftrightarrow$ $T_{\text{CLK}} = \dfrac{1}{1 \times 10^{9}} = \dfrac{1}{1'000'000'000} = 1\, \text{ns}$
  \end{itemize}
\end{itemize}

\highspace
\begin{flushleft}
  \textcolor{Green3}{\faIcon{stopwatch} \textbf{Execution Time (CPU Time)}}
\end{flushleft}
The \definitionWithSpecificIndex{execution time}{Formula: Execution Time}{Formula: CPU Time} (also called \textbf{CPU time}) is the \textbf{total time} needed by a processor to complete a given program. Two main ways to compute CPU time:
\begin{itemize}
  \item \textbf{Basic form}:
  \begin{equation}
    \text{CPU Time} = \text{Clock Cycles} \times T_{\text{CLK}} = \frac{\text{Clock Cycles}}{f_{\text{CLK}}}
  \end{equation}


  \item \textbf{Expanded form based on instructions}:
  \begin{equation}
    \begin{array}{rcl}
      \text{CPU Time} &=& \text{Instruction Count} \times \text{Cycles Per Instruction} \times T_{CLK} \\ [1em]
      &=& \dfrac{\text{Instruction Count} \times \text{Cycles Per Instruction}}{f_{\text{CLK}}}
    \end{array}
  \end{equation}


  \item \definitionWithSpecificIndex{Weighted CPU Time}{Formula: Weighted CPU Time}{}. If the program is composed of different instruction types, with different CPI values, the \textbf{total CPU time must be weighted} by how much each instruction type contributes:
  \begin{equation}
    \text{CPU Time} = \left( \displaystyle\sum_{i=1}^{n} (\text{CPI}_{i} \times \text{I}_{i}) \right) \times T_{CLK}
  \end{equation}
  Where:
  \begin{itemize}
    \item $\text{I}_{i}$ is the number of instructions of type $i$.
    \item $\text{T}_{CLK}$ is the clock cycle time.
  \end{itemize}
  The total CPU time is the sum of the CPU times \textbf{for each class of instruction}, weighted by how many times each instruction appears.

  Note that weighted CPU time can be calculated using the classic formula, but \textbf{using a weighted CPI} (shown on page \pageref{eq: Weighted Average CPI}):
  \begin{equation*}
    \text{CPU Time} = \dfrac{\text{IC} \times \text{CPI Weighted}}{f_{\text{CLK}}}
  \end{equation*}
\end{itemize}

\highspace
\begin{flushleft}
  \textcolor{Green3}{\faIcon{\speedIcon} \textbf{Cycles Per Instruction (CPI) and Instructions Per Cycle (IPC)}}
\end{flushleft}
\begin{itemize}
  \item \definitionWithSpecificIndex{Clocks Per Instruction (CPI)}{Formula: Clocks Per Instruction (CPI)}{} (clock cycles per instruction or clocks per instruction) is the \textbf{average number of clock cycles} each instruction requires.
  \begin{equation}
    \text{CPI} = \dfrac{
      \text{Total Clock Cycles}
    }{
      \text{Instruction Count (\texttt{IC})}
    }
  \end{equation}
  Where the Total CLock Cycles is:
  \begin{equation*}
    \text{Total Clock Cycles} = \texttt{IC} + 4 + \text{Stall Cycles}
  \end{equation*}
  And the Instruction Count is the total number of instructions executed.


  \item \definitionWithSpecificIndex{Instructions Per Cycle (IPC)}{Formula: Instructions Per Cycle (IPC)}{} is the reciprocal of CPI:
  \begin{equation}
    \text{IPC} = \frac{1}{\text{CPI}}
  \end{equation}


  \newpage


  \item \textbf{Mixed Programs}. In a real program, we have a mix of operations. Each instruction type can have a different CPI, thus the \textbf{total average performance depends on}:
  \begin{enumerate}
    \item \textbf{How often} each instruction type appears (its \textbf{frequency}).
    \item \textbf{How many cycles} each type needs (its \textbf{CPI}).
  \end{enumerate}
  Because \textbf{not all instructions appear equally}, we must give \textbf{more weight} to instructions that appear more frequently. The \definitionWithSpecificIndex{Weighted Average CPI}{Formula: Weighted Average CPI}{} formula is:
  \begin{equation}\label{eq: Weighted Average CPI}
    \text{CPI} = \displaystyle\sum_{i=1}^{n} \left(\text{CPI}_{i} \times \text{F}_{i}\right)
  \end{equation}
  Where:
  \begin{itemize}
    \item $\text{CPI}_{i}$ is the CPI of instruction type $i$.
    \item $\text{F}_{i}$ is the \textbf{frequency of instruction} type $i$ (percentage). It is calculated as:
    \begin{equation}\index{Formula: Instruction Frequency}
      \text{F}_{i} = \dfrac{\text{I}_{i}}{\text{IC}}
    \end{equation}
    Where:
    \begin{itemize}
      \item $\text{I}_{i}$ is the number of instructions of type $i$.
      \item $\text{IC}$ is the total \textbf{Instruction Count} (total number of instructions in the program).
    \end{itemize}
  \end{itemize}
\end{itemize}

\begin{examplebox}[: CPI and CPU Time Calculation]
  Given a program with \textbf{100 instructions} and the following instruction mix (500 MHz clock frequency):
  \begin{center}
    \begin{tabular}{@{} l c c @{}}
      \toprule
      Instruction Type & Frequency & CPI \\
      \midrule
      ALU       & 43\%  & 1 \\ [.3em]
      Load      & 21\%  & 4 \\ [.3em]
      Store     & 12\%  & 4 \\ [.3em]
      Branch    & 12\%  & 2 \\ [.3em]
      Jump      & 12\%  & 2 \\
      \bottomrule
    \end{tabular}
  \end{center}
  First, convert frequencies to decimals:
  \begin{itemize}
    \item 43\% $\rightarrow$ 0.43
    \item 21\% $\rightarrow$ 0.21
    \item 12\% $\rightarrow$ 0.12
    \item 12\% $\rightarrow$ 0.12
    \item 12\% $\rightarrow$ 0.12
  \end{itemize}
  Now apply the CPI weighted formula \ref{eq: Weighted Average CPI}:
  \begin{equation*}
    \begin{array}{rcl}
      \text{CPI} &=& \displaystyle\sum_{i=1}^{n} \left(\text{CPI}_{i} \times \text{F}_{i}\right) \\ [1.5em]
      &=& (0.43 \cdot 1) + (0.21 \cdot 4) + (0.12 \cdot 4) + (0.12 \cdot 2) + (0.12 \cdot 2) \\ [.6em]
      &=& 0.43 + 0.84 + 0.48 + 0.24 + 0.24 \\ [.3em]
      &=& 2.23
    \end{array}
  \end{equation*}
  Thus, the \textbf{average CPI} is $2.23$. For CPU time, we can use the weighted CPI, but we must calculate the Clock Cycle Time $T_{\text{CLK}}$:
  \begin{equation*}
    \begin{array}{rcl}
      \text{CPU Time} &=& \text{IC} \times \text{CPI Weighted} \times T_{\text{CLK}} \\ [1em]
      &=& 100 \times 2.23 \times \dfrac{1}{\left(500 \times 10^{6}\right)} \\ [1em]
      &=& 223 \times \dfrac{1}{\left(500 \times 10^{6}\right)} \\ [1.2em]
      &=& 223 \times 0.000'000'002\, \text{sec} \\ [.5em]
      &=& 223 \times 2\, \text{ns} \\ [.5em]
      &=& 446\, \text{ns}
    \end{array}
  \end{equation*}
\end{examplebox}

\begin{flushleft}
  \textcolor{Green3}{\faIcon{bolt} \textbf{MIPS (Millions of Instructions Per Second)}}
\end{flushleft}
\definitionWithSpecificIndex{MIPS}{Formula: MIPS}{} measures how many millions of instructions a processor can execute per second. Two equivalent formulas:
\begin{itemize}
  \item \textbf{Using frequency and CPI}:
  \begin{equation}
    \text{MIPS} = \dfrac{f_{\text{CLK}}}{\text{CPI} \times 10^6}
  \end{equation}
  \item \textbf{Using instruction count and execution time}:
  \begin{equation}
    \text{MIPS} = \frac{\text{Instruction Count}}{\text{Execution Time} \times 10^{6}}
  \end{equation}
  Where the Execution Time is the CPU Time.
\end{itemize}
\textbf{Higher MIPS does not necessarily mean better performance}, because MIPS does not account for instruction complexity or actual work done.