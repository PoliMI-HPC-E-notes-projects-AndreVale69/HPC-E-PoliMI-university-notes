\subsection{Pipelined Processors}

Pipelining affects CPU performance \textbf{differently} than simple frequency or instruction count.
\begin{itemize}
    \item[\textcolor{Green3}{\faIcon{check}}] \textcolor{Green3}{\textbf{Pipelining increases}} \textbf{instruction Throughput} (number of instructions completed per unit time).
    \item[\textcolor{Red2}{\faIcon{times}}] \textcolor{Red2}{\textbf{Pipelining does not reduce}} \textbf{Execution Time} (latency) of a single instruction.
\end{itemize}
In fact slight latency \textbf{increase} can happen due to:
\begin{itemize}
    \item \textbf{Imbalance} among \textbf{pipeline stages} (some stages are slower, limiting the clock frequency).
    \item \textbf{Overhead} from \textbf{pipeline control} (registers, clock skew, pipeline latch delays).
\end{itemize}
Also, we must remember that the \textbf{clock cycle time must be long enough for the slowest stage to finish}. Otherwise, the slowest stage will not finish its work before the clock ticks again, and the pipeline will become inconsistent or crash. So:
\begin{equation*}
    T_{\text{CLK}} > \text{time of the slowest stage}
\end{equation*}
So, if we \textbf{want to make the CPU faster}, we must \textbf{balance the stages} (make all stages take about the same time). Otherwise, one slow stage will limit the entire clock speed, even if other stages are very fast. This is why sometimes CPU designers split a slow stage into two smaller stages to balance the pipeline better!

\highspace
\begin{flushleft}
    \textcolor{Green3}{\faIcon{stream} \textbf{Performance Metrics for Pipelining}}
\end{flushleft}
Define:
\begin{itemize}
    \item IC as Instruction Count
    \item CPI as Cycles Per Instruction
    \item IPC as Instructions Per Clock $\frac{1}{\text{CPI}}$
\end{itemize}
We have:
\begin{itemize}
    \item \definitionWithSpecificIndex{Number of clock cycles in pipelined execution}{Formula Pipeline: Number of clock cycles}{}:
    \begin{equation}
        \text{Clock Cycles} = \text{IC} + \text{Stall Cycles} + \text{Pipeline fill penalty (4 cycles)}
    \end{equation}
    For a 5-stage pipeline like MIPS or RISC-V. And the CPI is:
    \begin{equation}
        \text{CPI} = \dfrac{\text{Clock Cycles}}{\text{IC}} = \dfrac{\left(\text{IC} + \text{Stall Cycles} + 4\right)}{\text{IC}}
    \end{equation}

    \item \definitionWithSpecificIndex{MIPS calculation}{Formula Pipeline: MIPS}{}:
    \begin{equation}
        \text{MIPS} = \dfrac{f_{\text{clock}}}{\text{CPI} \times 10^{6}}
    \end{equation}
\end{itemize}

\begin{examplebox}
    Given:
    \begin{itemize}
        \item IC = 5 instructions
        \item Stall cycles = 2
        \item Clock = 500 MHz
    \end{itemize}
    Step-by-step:
    \begin{itemize}
        \item Clock cycles:
        \begin{equation*}
            5 + 2 + 4 = 11
        \end{equation*}
        \item CPI:
        \begin{equation*}
            \frac{11}{5} = 2.2
        \end{equation*}
        \item MIPS:
        \begin{equation*}
            \frac{500 \times 10^6}{2.2 \times 10^6} \approx 227
        \end{equation*}
    \end{itemize}
    Thus, the effective throughput is about \textbf{227 MIPS}.
\end{examplebox}

\highspace
\begin{flushleft}
    \textcolor{Green3}{\faIcon{sync} \textbf{Performance in Loops}}
\end{flushleft}
Suppose we have:
\begin{itemize}
    \item $n$ \textbf{iterations}
    \item Loop of $m$ \textbf{instructions} per iteration
    \item $k$ \textbf{stalls} per iteration
\end{itemize}
Formulas per iteration:
\begin{itemize}
    \item \definitionWithSpecificIndex{IC per iteration}{Formula Pipeline Loops: IC per iter}{}:
    \begin{equation}
        \text{IC}_{\text{per iter}} = m
    \end{equation}
    \item \definitionWithSpecificIndex{Clock Cycles per iteration}{Formula Pipeline Loops: Clock Cycles per iter}{}:
    \begin{equation}
        \text{\# Clock Cycles}_{\text{per iter}} = m + k + 4
    \end{equation}
    \item \definitionWithSpecificIndex{Cycles Per Instruction per iteration}{Formula Pipeline Loops: Cycles Per Instruction per iter}{}:
    \begin{equation}
        \text{CPI}_{\text{per iter}} = \dfrac{\left(m + k + 4\right)}{m}
    \end{equation}
    \item \definitionWithSpecificIndex{MIPS per iteration}{Formula Pipeline Loops: MIPS per iter}{}:
    \begin{equation}
        \text{MIPS} = \dfrac{f_{\text{clock}}}{\text{CPI}_{\text{per iter}} \times 10^{6}}
    \end{equation}
\end{itemize}

\highspace
\begin{flushleft}
    \textcolor{Green3}{\faIcon{infinity} \textbf{Asymptotic Performance (Many Iterations)}}
\end{flushleft}
If the loop runs for $n \rightarrow \infty$ iterations:
\begin{itemize}
    \item \definitionWithSpecificIndex{Total Instruction Count per iteration}{Formula Pipeline Loops (AS): $\text{IC}_{\text{AS}}$ per iter}{}:
    \begin{equation}
        \text{IC}_{\text{AS}} = \text{Instruction Count}_{\text{AS}} = m \times n
    \end{equation}

    \item \definitionWithSpecificIndex{Total Clock Cycles per iteration}{Formula Pipeline Loops (AS): Total Clock Cycles per iter}{}:
    \begin{equation}
        \begin{array}{rcl}
            \text{\# Clock Cycles}_{\text{AS}} &=& \text{IC}_{\text{AS}} + \text{\# Stall Cycles}_{\text{AS}} + 4 \\ [1em]
            &=& \left(m \times n\right) + \left(k \times n\right) + 4
        \end{array}
    \end{equation}

    \item \definitionWithSpecificIndex{Cycles Per Instruction per iteration}{Formula Pipeline Loops AS: Cycles Per Instruction per iter}{}:
    \begin{equation}
        \text{CPI}_{\text{AS}} = \lim_{n \to \infty} \dfrac{\text{\# Clock Cycles}_{\text{AS}}}{\text{IC}_{\text{AS}}} = \lim_{n \to \infty} \dfrac{m \times n + k \times n + 4}{m \times n} = \dfrac{m+k}{m}
    \end{equation}
    The fixed 4 cycles become irrelevant when $n$ becomes very large.

    \item \definitionWithSpecificIndex{MIPS per iteration}{Formula Pipeline Loops (AS): MIPS per iteration}{}:
    \begin{equation}
        \text{MIPS}_{\text{AS}} = \dfrac{f_{\text{clock}}}{\text{CPI}_{\text{AS}} \times 10^{6}}
    \end{equation}
\end{itemize}

\highspace
\begin{flushleft}
    \textcolor{Green3}{\faIcon{balance-scale} \textbf{Ideal vs Realistic Pipelining}}
\end{flushleft}
The ideal CPI on a pipelined processor would be 1, but stalls cause the pipeline performance to degrade form the ideal performance, so we have:
\begin{itemize}
    \item \important{Ideal pipelining}:
    \begin{equation}
        \text{Ideal CPI} = 1
    \end{equation}
    1 instruction per cycle after filling the pipeline.
    \item \important{Realistic pipelining}:
    \begin{equation}
        \text{CPI} = 1 + (\text{stall cycles per instruction})
    \end{equation}
\end{itemize}
Stalls are caused by:
\begin{itemize}
    \item Structural Hazards (e.g., hardware resources conflict)
    \item Data Hazards (e.g., instruction dependencies)
    \item Control Hazards (e.g., branches)
    \item Memory Stalls (e.g., cache misses)
\end{itemize}

\newpage

\begin{flushleft}
    \textcolor{Green3}{\faIcon{forward} \textbf{Pipeline Speedup}}
\end{flushleft}
Speedup measures how much \textbf{pipelining} performance compared to \textbf{unpi-\break pelined execution}. General formula:
\begin{equation}\index{Formula Speedup: Pipeline Speedup}
    \text{Pipeline Speedup} =
    \dfrac{\text{Avg Exec Time Unpipelined}}{\text{Avg Exec Time Pipelined}} =
    \dfrac{\text{CPI}_{\text{unp}} \times T_{\text{clk, unp}}}{\text{CPI}_{\text{pipe}} \times T_{\text{clk, pipe}}}
\end{equation}
If we \textbf{ignore the cycle time overhead} of pipelining and we assume the \textbf{stages are perfectly balanced}, the clock cycle time of unpipelined/pipelined processors can be equal, so:
\begin{equation}
    \text{Pipeline Speedup} = \dfrac{\text{CPI}_{\text{unp}}}{1 + \text{CPI}_{\text{pipe}}} = \dfrac{\text{CPI}_{\text{unp}}}{1 + \text{Stall Cycles per Instruction}}
\end{equation}
If we assume that \textbf{each instruction takes the same number of cycles}, which must be \textbf{equal to the number of pipeline stages} (called pipeline depth):
\begin{equation}
    \text{Pipeline Speedup} = \frac{\text{Pipeline Depth}}{1 + \text{Stall Cycles per Instruction}}
\end{equation}
If there were \textbf{no pipeline stalls (ideal case)}, this leads to the intuitive result that pipelining improves performance by the depth of the pipeline:
\begin{equation}
    \text{Speedup} = \text{Pipeline Depth}
\end{equation}
Thus, a 5-stage pipeline \textbf{could at best achieve 5x speedup only if there are no stalls}.

\highspace
\begin{flushleft}
    \textcolor{Green3}{\faIcon{code-branch} \textbf{Impact of Branches}}
\end{flushleft}
Branches cause \textbf{pipeline stalls}, and their effect is quantified as:
\begin{equation}
    \text{Pipeline Speedup} = \frac{\text{Pipeline Depth}}{1 + \text{Branch Frequency} \times \text{Branch Penalty}}    
\end{equation}
\begin{itemize}
    \item \textbf{Branch Frequency}: fraction of \hl{instructions that are branches}.
    \item \textbf{Branch Penalty}: number of \hl{stall cycles} per branch.
\end{itemize}
Where the Pipeline Stall Cycles per Instruction due to Branches (PSCIB) is:
\begin{equation}
    \text{PSCIB} = \text{Branch Frequency} \times \text{Branch Penalty}
\end{equation}