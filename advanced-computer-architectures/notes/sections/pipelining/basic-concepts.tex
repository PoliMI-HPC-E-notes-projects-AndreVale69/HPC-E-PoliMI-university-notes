\section{Pipelining}

\subsection{Basic Concepts}

\definition{Pipelining} is a fundamental \textbf{technique} in computer architecture aimed at \textbf{improving instruction throughput by overlapping the execution of multiple instructions}. The main idea behind pipelining is to \hl{divide the execution of an instruction into distinct stages and process different instructions simultaneously in these stages}. This approach significantly increases the efficiency of instruction execution in modern processors.

\highspace
\begin{flushleft}
    \textcolor{Green3}{\faIcon{tools} \textbf{Understanding the RISC-V instruction set}}
\end{flushleft}
Before delving into pipelining, it is essential to understand the \textbf{basic instruction set} of the RISC-V architecture. The instruction set consists of three major categories:
\begin{itemize}
    \item \important{ALU Instructions (Arithmetic and Logic Operations)}
    \begin{itemize}
        \item Performs \textbf{addition between registers}:
        \begin{lstlisting}[language=riscv]
add rd, rs1, rs2\end{lstlisting}
        Performs the addition between the values in registers \texttt{rs1} and \texttt{rs2} and stores the result in register \texttt{rd}.
        \begin{equation*}
            \texttt{rd} \leftarrow \texttt{rs1} + \texttt{rs2}
        \end{equation*}

        \item Performs an \textbf{addition between a constant and a register}:
        \begin{lstlisting}[language=riscv]
addi rd, rs1, 4\end{lstlisting}
        Performs the addition between the value in register \texttt{rs1} and the value \texttt{4} and stores the result in register \texttt{rd}.
        \begin{equation*}
            \texttt{rd} \leftarrow \texttt{rs1} + \texttt{4}
        \end{equation*}
    \end{itemize}

    \item \important{Load/Store Instructions (Memory Operations)}
    \begin{itemize}
        \item \textbf{Loads} data from memory:
        \begin{lstlisting}[language=riscv]
ld rd, offset(rs1)\end{lstlisting}
        Load data into register \texttt{rd} from an address formed by adding \texttt{rs1} to a signed \texttt{offset}.
        \begin{equation*}
            \texttt{rd} \leftarrow M\left[\texttt{rs1} + \texttt{offset}\right]
        \end{equation*}

        \item \textbf{Stores} data in memory:
        \begin{lstlisting}[language=riscv]
sd rs2, offset(rs1)\end{lstlisting}
        Store data from register \texttt{rs2} to an address formed by adding \textsl{rs1} to a signed \textsl{offset}.
        \begin{equation*}
            M\left[\texttt{rs1} + \texttt{offset}\right] \leftarrow \texttt{rs2}
        \end{equation*}
    \end{itemize}

    \newpage

    \item \important{Branching Instructions (Control Flow Management)}
    \begin{itemize}
        \item \important{Conditional Branches}
        \begin{itemize}
            \item Branch on equal:
            \begin{lstlisting}[language=riscv]
beq rs1, rs2, L1\end{lstlisting}
            Branch to the label \texttt{L1} if the value in register \texttt{rs1} is equal to the value in register \texttt{rs2}.
            \begin{equation*}
                \texttt{rs1} = \texttt{rs1} \xRightarrow{\text{go to}} \texttt{L1}
            \end{equation*}
            \item Branch on not equal:
            \begin{lstlisting}[language=riscv]
bne rs1, rs2, L1\end{lstlisting}
            Branch to the label \texttt{L1} if the value in register \texttt{rs1} is not equal to the value in register \texttt{rs2}.
            \begin{equation*}
                \texttt{rs1} \ne \texttt{rs2} \xRightarrow{\text{go to}} \texttt{L1}
            \end{equation*}
        \end{itemize}
        \item \important{Unconditional Jumps}
        \begin{itemize}
            \item Jump to the label (jump):
            \begin{lstlisting}[language=riscv]
j L1\end{lstlisting}
            Jump directly to the \texttt{L1} label.
            \item Jump to the address stored in a register (jump register):
            \begin{lstlisting}[language=riscv]
jr ra\end{lstlisting}
            Take the value in register \texttt{ra} and use it as the address to jump to. So it is assumed that \texttt{ra} contains an address.
        \end{itemize}
    \end{itemize}
\end{itemize}
These basic instructions will be used throughout the course.

\highspace
\begin{flushleft}
    \textcolor{Green3}{\faIcon{stream} \textbf{Execution phases in RISC-V}}
\end{flushleft}
\begin{enumerate}
    \item \definition{IF (Instruction Fetch)}: The instruction is \textbf{fetched} from memory.
    \item \definition{ID (Instruction Decode)}: The instruction is \textbf{decoded}, and the \textbf{required registers are read}.
    \item \definition{EX (Execution)}: The instruction is \textbf{executed}, typically involving ALU operations.
    \item \definition{ME (Memory Access)}: For \emph{load}/\emph{store} instructions, this stage \textbf{reads from} or \textbf{writes to} memory.
    \item \definition{WB (Write Back)}: The \textbf{result is written back} to the destination register.
\end{enumerate}
These five stages form the basis of the RISC-V pipeline.

\newpage

\begin{flushleft}
    \textcolor{Green3}{\faIcon{tools} \textbf{Implementation of the RISC-V Data Path}}
\end{flushleft}
The \definition{RISC-V Data Path} is a fundamental component of the processor's architecture, responsible for \textbf{executing instructions efficiently by coordinating various hardware units}. It defines how instructions flow through different stages of execution, interacting with memory, registers, and the Arithmetic Logic Unit (ALU).

\begin{figure}[!htp]
    \centering
    \includegraphics[width=\textwidth]{img/generic-risc-v-dp.pdf}
    \caption{Generic implementation of the RISC-V Data Path.}
\end{figure}

\begin{figure}[!htp]
    \centering
    \includegraphics[width=\textwidth]{img/detailed-risc-v-dp.pdf}
    \caption{Specific implementation of the RISC-V Data Path.}
\end{figure}

\noindent
Its fundamental components include:
\begin{itemize}
    \item \important{Instruction Memory and Data Memory Separation}. RISC-V\break adopts a Harvard Architecture style, where the \textbf{Instruction Memory (IM)} and \textbf{Data Memory (DM) are separate}. This \textbf{prevents structural hazards} where instruction fetch and memory access could conflict in a single-memory design (this topic will be addressed later).

    \item \important{General-Purpose Register File (RF)}. It consists of \textbf{32 registers}, each \textbf{32-bit} wide. The register file has \textbf{two read ports} and \textbf{one write port} to \hl{support simultaneous read and write operations}. This setup allows faster register access, which is crucial for pipelined execution.

    \item \important{Program Counter (PC)}. It \textbf{holds the address of the next instruction to be fetched}. Automatically increments during execution, typically by 4 bytes (for 32-bit instructions).
    
    \item \important{Arithmetic Logic Unit (ALU)}. \textbf{Performs arithmetic and logical operations} required by instructions. Inputs to the ALU come from registers or immediate values decoded from the instruction.
\end{itemize}
Other components that we can see in the general implementation of the RISC-V data path are:
\begin{itemize}
    \item \textbf{Register File}. \hl{Stores temporary values used by instructions}. Contains read ports (two registers can be read simultaneously for ALU operations) and write port (one register can be updated per clock cycle). The register file ensures high-speed execution of operations by reducing memory accesses.
    \item \textbf{Instruction Fetch (IF)}. The PC (Program Counter) retrieves the next instruction from Instruction Memory. The PC is incremented using an adder (\texttt{PC + 4}), ensuring sequential instruction flow.
    \item \textbf{Instruction Decode (ID)}. Extracts opcode (determines the instruction type), source and destination registers, immediate values (if present). It \hl{reads values from the Register File based on instruction requirements}.
    \item \textbf{Execution (EX)}. The \hl{ALU performs arithmetic and logical operations}. A multiplexer (MUX) selects the second operand: a register value (for R-type instructions) or an immediate value (for I-type instructions like \texttt{addi}). The \hl{ALU result is forwarded} to the next stage.
    \item \textbf{Memory Access (ME)}. Load (\texttt{ld}) and Store (\texttt{sd}) instructions interact with data memory. Data is either \textbf{loaded from memory into a register} or \textbf{stored from a register into memory}.
    \item \textbf{Write Back (WB)}. The result from ALU or memory is \textbf{written back to the Register File}.
\end{itemize}

\newpage

\begin{examplebox}[: Data Path Execution Example]
    Let's consider a simple RISC-V \textbf{load instruction} (\texttt{ld x10, 40(x1)}) passing through the data path:
    \begin{enumerate}
        \item \textbf{IF Stage}: Instruction Fetch
        \begin{itemize}
            \item \texttt{PC} $\rightarrow$ Instruction Memory $\rightarrow$ \texttt{ld x10, 40(x1)} fetched
            \item \texttt{PC} updated to \texttt{PC + 4}
        \end{itemize}

        \item \textbf{ID Stage}: Instruction Decode
        \begin{itemize}
            \item Registers read: \texttt{x1} (base register for memory access)
            \item Immediate value extracted: \texttt{40}
        \end{itemize}

        \item \textbf{EX Stage}: Execution
        \begin{itemize}
            \item ALU calculates memory address: \texttt{x1 + 40}
        \end{itemize}

        \item \textbf{ME Stage}: Memory Access
        \begin{itemize}
            \item Data is loaded from \texttt{M[x1 + 40]}
        \end{itemize}

        \item \textbf{WB Stage}: Write Back
        \begin{itemize}
            \item Data stored in \texttt{x10}
        \end{itemize}
    \end{enumerate}
\end{examplebox}
