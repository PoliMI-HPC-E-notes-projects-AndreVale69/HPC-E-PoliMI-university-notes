\subsection{Performance evaluation}

Evaluating the performance of a pipelined processor is essential to understanding the impact of stalls, hazards, and instruction throughput. In an \textbf{ideal scenario}, a \textbf{pipeline achieves one instruction per cycle} ($\texttt{CPI} = 1$), but real-world execution includes pipeline stalls, which degrade performance.

\highspace
Several \textbf{key metrics} are used to evaluate the efficiency of pipelining:
\begin{itemize}
   \item \definition{Instruction Count (\texttt{IC})}. Represents the \textbf{total number of instructions executed}. Used as a basis for performance calculations.
   
   \item \definition{Clocks Per Instruction (\texttt{CPI})}. \texttt{CPI} measures the \textbf{average number of clock cycles required to execute one instruction}. \hl{Ideal \texttt{CPI} for a pipelined processor is 1}, but hazards and stalls increase \texttt{CPI}.
   \begin{equation}
      \texttt{CPI} = \dfrac{
         \text{Total Clock Cycles}
      }{
         \text{Instruction Count (\texttt{IC})}
      }
   \end{equation}
   Where the total clock cycles is:
   \begin{equation}
      \texttt{Total Clock Cycles} = \texttt{IC} + 4 + \text{Stall Cycles}
   \end{equation}
   Where $+4$ is the \textbf{fill time of the first instruction}. The $+4$ represents the initial pipeline fill time required before the pipeline reaches full execution throughput.

   \begin{examplebox}[: Why is the Pipeline Startup Overhead $+4$?]
      A 5-stage pipeline (IF, ID, EX, MEM, WB) requires 4 extra cycles before the first instruction completes. Consider the following scenario:
      \begin{center}
         \begin{tabular}{@{} c | c | c | c | c | c @{}}
            \toprule
            Clock Cycle & IF & ID & EX & MEM & WB \\
            \midrule
            1 & \texttt{I1} &    &    &    &    \\ [.3em]
            2 & \texttt{I2} & \texttt{I1} &    &    &    \\ [.3em]
            3 & \texttt{I3} & \texttt{I2} & \texttt{I1} &    &    \\ [.3em]
            4 & \texttt{I4} & \texttt{I3} & \texttt{I2} & \texttt{I1} &    \\ [.3em]
            5 & \texttt{I5} & \texttt{I4} & \texttt{I3} & \texttt{I2} & \texttt{I1} \\ [.3em]
            6 & \texttt{I6} & \texttt{I5} & \texttt{I4} & \texttt{I3} & \texttt{I2} \\ [.3em]
            7 & \texttt{I7} & \texttt{I6} & \texttt{I5} & \texttt{I4} & \texttt{I3} \\
            \bottomrule
         \end{tabular}
      \end{center}
      The first instruction (\texttt{I1}) requires 5 cycles to complete. The next instruction (\texttt{I2}) completes in cycle 6, and so on. After the first 5 cycles, the \textbf{pipeline reaches steady state}, completing 1 instruction per cycle (ideal scenario, no hazards).
   \end{examplebox}

   \item \definition{Instruction Per Clock (\texttt{IPC})}. \texttt{IPC} is the inverse of \texttt{CPI}:
   \begin{equation}
      \texttt{IPC} = \dfrac{1}{\texttt{CPI}}
   \end{equation}
   Measures \textbf{how many instructions complete per clock cycle}.

   \item \definition{Millions of Instructions Per Second (\texttt{MIPS})}. Evaluates processor speed in terms of \textbf{millions of instructions executed per second}:
   \begin{equation}
      \texttt{MIPS} = \dfrac{
         f_{\text{clock}}
      }{
         \texttt{CPI} \times 10^{6}
      }
   \end{equation}
   Higher \textbf{clock frequency} ($f_{\text{clock}}$) and lower \texttt{CPI} result in better MIPS.
\end{itemize}

\begin{examplebox}[: Performance Calculation]
   Given:
   \begin{itemize}
      \item $\text{Instruction Count (\texttt{IC})} = 5$
      \item $\text{Stall Cycles} = 2$
      \item $\text{Clock Frequency} = 500 \text{MHz}$
   \end{itemize}
   Metrics:
   \begin{itemize}
      \item Total Clock Cycles:
      \begin{equation*}
         \text{Clock Cycles} = \texttt{IC} + \text{Stall Cycles} + 4 = 5 + 2 + 4 = 11
      \end{equation*}
      \item \texttt{CPI} Calculation:
      \begin{equation*}
         \texttt{CPI} = \dfrac{11}{5} = 2.2
      \end{equation*}
      \item \texttt{MIPS} Calculation:
      \begin{equation*}
         \texttt{MIPS} = \dfrac{500 \text{ MHz}}{2.2 \times 10^{6}} = 227
      \end{equation*}
   \end{itemize}
   Without stalls, \texttt{CPI} would be 1 (ideal pipeline). But stalls increase \texttt{CPI}, reducing \texttt{MIPS} and overall efficiency.
\end{examplebox}

\newpage
\begin{flushleft}
   \textcolor{Green3}{\faIcon{\speedIcon} \textbf{Performance in Loops and Asymptotic Analysis}}
\end{flushleft}
When evaluating \textbf{loops} or \textbf{long-running programs}, we use asymptotic performance metrics.

\highspace
For a loop with:
\begin{itemize}
   \item $m$ \textbf{instructions} per iteration.
   \item $k$ \textbf{stall} cycles per iteration.
   \item $n$ \textbf{iterations}.
\end{itemize}
We have:
\begin{itemize}
   \item \textbf{Clock Cycles per Iteration}:
   \begin{equation}
      \text{Clock Cycles per Iteration} = m + k + 4
   \end{equation}

   \item \textbf{\texttt{CPI} per Iteration}:
   \begin{equation}
      \texttt{CPI}_{\text{iter}} = \dfrac{\left(m + k + 4\right)}{m}
   \end{equation}

   \item \textbf{\texttt{MIPS} per Iteration}:
   \begin{equation}
      \texttt{MIPS}_{\text{iter}} = \dfrac{f_{\text{clock}}}{\texttt{CPI}_{\text{iter}} \times 10^{6}}
   \end{equation}
\end{itemize}
For \textbf{large} $n$, the impact of pipeline startup delay ($+4$ cycles) is reduced:
\begin{itemize}
   \item \textbf{\texttt{CPI} per Iteration}:
   \begin{equation}
      \begin{array}{rcl}
         \texttt{CPI}_{\text{AS}} &=& \underset{n \rightarrow \infty}{\lim} \dfrac{
            \left(\texttt{IC}_{\text{AS}} + \text{Stall Cycles}_{\text{AS}} + 4\right)
         }{
            \texttt{IC}_{\text{AS}}
         } \\ [1em]
         &=& \underset{n \rightarrow \infty}{\lim} \dfrac{
            \left(m \times n + k \times n + 4\right)
         }{
            \left(m \times n\right)
         } \\ [1em]
         &=& \dfrac{\left(m + k\right)}{m}
      \end{array}
   \end{equation}
   
   \item \textbf{Millions of Instructions Per Second (\texttt{MIPS})}:
   \begin{equation}
      \texttt{MIPS}_{\text{AS}} = \dfrac{
         f_{\text{clock}}
      }{
         \texttt{CPI}_{\text{AS}} \times 10^{6}
      }
   \end{equation}
\end{itemize}
For \textbf{large programs}, startup stalls become negligible, and \textbf{performance depends mainly on stall cycles per iteration}. \hl{Minimizing} $k$ (\hl{stalls per iteration}) is crucial to achieving high efficiency.

\highspace
\begin{flushleft}
   \textcolor{Green3}{\faIcon{question-circle} \textbf{Why \texttt{CPI} is Greater than 1 in Real Pipelines}}
\end{flushleft}
Even with an \textbf{optimized pipeline}, \textbf{real-world execution is affected by hazards}. Thus, actual CPI is always greater than 1, even in well-optimized designs.
