\subsubsection{Pipeline Implementation}

The \textbf{RISC-V pipeline implementation} is designed to efficiently execute multiple instructions simultaneously, following the classical five-stage pipeline model: 
\begin{enumerate}
    \item IF (Instruction Fetch)
    \item ID (Instruction Decode)
    \item EX (Execution)
    \item MEM (Memory Access)
    \item WB (Write Back)
\end{enumerate}
Each clock cycle, a new instruction enters the pipeline while previous instructions move to the next stage, allowing \textbf{five different instructions to be in execution at the same time}.

\begin{figure}[!htp]
    \centering
    \includegraphics[width=\textwidth]{img/risc-v-pipeline-structure-1.pdf}
    \caption{Structure of RISC-V pipeline.}
    \label{fig: structure of risc-v pipeline}
\end{figure}

\begin{flushleft}
    \textcolor{Green3}{\faIcon{tools} \textbf{Execution Stages and Pipeline Modules}}
\end{flushleft}
\textbf{Each stage of the pipeline corresponds to a specific hardware module in the CPU}. The RISC-V pipeline is composed of five primary hardware modules:
\begin{itemize}
    \item \important{Instruction Fetch (IF) Module}: Fetches instructions from instruction memory and updates the PC.
    \item \important{Instruction Decode (ID) Module}: Decodes the fetched instruction and reads register values.
    \item \important{Execution (EX) Module}: Performs arithmetic/logical operations in the ALU or computes memory addresses.
    \item \important{Memory Access (MEM) Module}: Reads from or writes data to memory.
    \item \important{Write Back (WB) Module}: Writes the computed result back into the register file.
\end{itemize}
Each module is responsible for a specific \textbf{stage of execution}, and together they allow overlapping execution of multiple instructions.

\highspace
\begin{flushleft}
    \textcolor{Green3}{\faIcon{memory} \textbf{Pipeline Registers}}
\end{flushleft}
To \textbf{maintain separation between stages}, \definitionWithSpecificIndex{pipeline registers}{Pipeline Registers}{} are used (see Figure \ref{fig: structure of risc-v pipeline}, page \pageref{fig: structure of risc-v pipeline}). These registers \hl{store intermediate results and ensure proper communication between stages}:
\begin{itemize}
    \item \important{IF/ID Register}: Holds fetched instruction and updated PC.
    \item \important{ID/EX Register}: Stores decoded instruction, read register values, and control signals.
    \item \important{EX/MEM Register}: Holds ALU results, destination register, and memory access information.
    \item \important{MEM/WB Register}: Stores memory data or ALU result to be written back to registers.
\end{itemize}
These pipeline registers \textbf{eliminate the need for re-fetching or re-decoding instructions} at each cycle, thus maintaining pipeline efficiency.