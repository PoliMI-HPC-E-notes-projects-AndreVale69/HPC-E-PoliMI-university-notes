\section*{Preface}

Every theory section in these notes has been taken from two sources:
\begin{itemize}
    \item Computer Architecture: A Quantitative Approach.\cite{hennessy2017computer}
    \item Pipelining slides.\cite{pipelining-slides}
    \item Course slides.\cite{course-slides-polimi}
\end{itemize}
About:
\begin{itemize}
    \item[\faIcon{github}] \href{https://github.com/PoliMI-HPC-E-notes-projects-AndreVale69/HPC-E-PoliMI-university-notes}{GitHub repository}
    \begin{center}
        \qrcode{https://github.com/PoliMI-HPC-E-notes-projects-AndreVale69/HPC-E-PoliMI-university-notes}
    \end{center}
\end{itemize}
These notes are an unofficial resource and shouldn't replace the course material or any other book on advanced computer architectures. It is not made for commercial purposes. I've made the following notes to help me improve my knowledge and maybe it can be helpful for everyone.

As I have highlighted, a student should choose the teacher's material or a book on the topic. These notes can only be a helpful material.

\highspace
For the exam, I created a simple formulary that follows the professor's instructions. You can access it via the link below or by scanning the QR code.
\begin{center}
    \href{https://github.com/PoliMI-HPC-E-notes-projects-AndreVale69/HPC-E-PoliMI-university-notes/blob/main/advanced-computer-architectures/notes/formulary.pdf}{Formulary}
    \qrcode{https://github.com/PoliMI-HPC-E-notes-projects-AndreVale69/HPC-E-PoliMI-university-notes/blob/main/advanced-computer-architectures/notes/formulary.pdf}
\end{center}
