\subsubsection{Local vs Global Scheduling}

In modern compilers, especially for statically scheduled architectures like VLIW, \textbf{instruction scheduling can be applied at two levels}:
\begin{itemize}
    \item \important{Local Scheduling: Within Basic Blocks}. \textbf{Local scheduling} operates \textbf{inside a single basic block}, where:
    \begin{itemize}
        \item The control flow is \textbf{linear} (no internal branches).
        \item The compiler has \textbf{full knowledge} of all instructions and dependencies.
    \end{itemize}
    \textcolor{Green3}{\faIcon{bullseye}} The \textcolor{Green3}{\textbf{objective}} is to:
    \begin{itemize}
        \item Maximize instruction-level parallelism (ILP).
        \item Minimize execution time of each block.
        \item Efficiently fill VLIW slots with real instructions (reduce NOPs).
    \end{itemize}
    \textcolor{Green3}{\faIcon{tools} \textbf{Techniques}}
    \begin{itemize}
        \item ASAP scheduling (page \pageref{subsubsection: ASAP Scheduling Algorithm}).
        \item List-based scheduling (page \pageref{subsubsection: List-Based Scheduling Algorithm}).
        \item Loop unrolling (page \pageref{paragraph: Loop Unrolling}).
        \item Software pipelining.
    \end{itemize}
    Local scheduling is \textbf{simpler} and \textbf{effective}, but limited:
    \begin{itemize}[label=\textcolor{Red2}{\faIcon{times}}]
        \item It \textbf{cannot move instructions} across block boundaries.
        \item \textbf{ILP is bounded} by the size and dependencies of the block.
    \end{itemize}


    \item \important{Global Scheduling: Across Basic Blocks}. \textbf{Global Scheduling} considers a \textbf{wider scope}, across \textbf{multiple basic blocks}, \hl{allowing reordering of instructions}:
    \begin{itemize}
        \item Across \textbf{branches}.
        \item Across \textbf{loop boundaries}.
        \item Even across \textbf{entire control paths}.
    \end{itemize}
    \textcolor{Green3}{\faIcon{bullseye}} The \textcolor{Green3}{\textbf{goal}} is to:
    \begin{itemize}
        \item Exploit \textbf{more ILP than local scheduling} allows.
        \item Hide or overlap long-latency operations.
        \item Restructure code for better pipeline utilization.
    \end{itemize}
    \textcolor{Red2}{\faIcon{ban} \textbf{Challenges}}
    \begin{itemize}
        \item Must preserve \textbf{control and data dependences}.
        \item Must generate \textbf{compensation code} to fix incorrect speculative\break moves.
        \item More \textbf{complex} and \textbf{expensive} to implement in the compiler.
    \end{itemize}
\end{itemize}

\begin{table}[!htp]
    \centering
    \begin{tabular}{@{} l l p{17em} @{}}
        \toprule
        Technique & Type & Description \\
        \midrule
        \textbf{Loop Unrolling}         & Local     & Replicates loop body to increase basic block size and expose ILP.     \\ [.5em]
        \textbf{Software Pipelining}    & Local     & Reorganizes instructions from different iterations to overlap them.   \\ [.5em]
        \textbf{Trace Scheduling}       & Global    & Predicts a frequent path (trace) and aggressively schedules it.       \\ [.5em]
        \textbf{Superblock Scheduling}  & Global    & Extension of trace scheduling; allows multiple exits, single entry.   \\
        \bottomrule
    \end{tabular}
    \caption{Techniques Overview.}
\end{table}