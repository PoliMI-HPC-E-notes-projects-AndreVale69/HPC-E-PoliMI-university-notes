\subsubsection{Scheduling Basics}

A \definition{Basic Block} is the smallest unit of code for scheduling purposes. It's a straight-line code sequence that has:
\begin{itemize}
    \item \textbf{One single entry point} (i.e., no jumps into the middle).
    \item \textbf{One single exit point} (i.e., only the last instruction may jump elsewhere).
\end{itemize}
In practice:
\begin{itemize}
    \item No branches except at the \textbf{start (entry)} and \textbf{end (exit)}.
    \item All instructions inside a basic block \textbf{execute sequentially and always together}.
    \item Used by compilers to isolate code regions where \textbf{safe reordering} can be performed.
\end{itemize}
This \hl{isolation is essential for local scheduling techniques} like \textbf{loop unrolling} and \textbf{software pipelining}.

\highspace
\begin{flushleft}
    \textcolor{Green3}{\faIcon{book} \textbf{Dependence Graph (DAG, Directed Acyclic Graph)}}
\end{flushleft}
\hl{Once a basic block is identified}, the \textbf{compiler builds a dependence graph} to model how instructions depend on each other.
\begin{itemize}
    \item Each \textbf{node} in the graph represents an \textbf{instruction}.
    \item Each \textbf{edge} denotes a \textbf{data dependence} between two instructions.
    \item The \textbf{type of edge} can correspond to:
    \begin{itemize}
        \item \textbf{RAW} (true dependence).
        \item \textbf{WAR}, \textbf{WAW} (name dependencies, can often be eliminated via register renaming).
    \end{itemize}
\end{itemize}
This \textbf{graph helps determine}:
\begin{itemize}[label=\textcolor{Green3}{\faIcon{check}}]
    \item Which instructions are \textbf{ready to execute}.
    \item Which instructions must \textbf{wait for others to complete}.
    \item The \textbf{longest dependency chain}, which limits parallelism.
\end{itemize}

\highspace
\begin{flushleft}
    \textcolor{Green3}{\faIcon{question-circle} \textbf{Why is this graph useful?}}
\end{flushleft}
There are two main reasons:
\begin{enumerate}
    \item It \textbf{exposes parallelism}: independent instructions can be scheduled in the same cycle or in parallel execution units.
    \item It defines the \definition{Critical Path} of execution: the \textbf{longest path from a source to a sink node} (i.e., sequential dependencies that can't be parallelized).
\end{enumerate}