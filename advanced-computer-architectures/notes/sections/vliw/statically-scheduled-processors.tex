\subsection{Statically Scheduled Processors}

In \textbf{statically scheduled architectures}, the \textbf{compiler is responsible for arranging instructions to exploit Instruction-Level Parallelism} (ILP). By carefully analyzing data dependencies and resource constraints, the compiler reorders operations to maximize parallel execution and minimize stalls.

\highspace
A \definition{Basic Block} is the \textbf{main unit considered during static scheduling}. It consists of a \textbf{straight-line code sequence without branches except at the entry and exit points}. However, in practice, basic blocks tend to be \textbf{small}: for typical programs, only around 4 to 7 instructions are found between branches. Moreover, even within a basic block, \textbf{true data dependencies} (RAW, Read After Write) further \textbf{limit the amount of exploitable parallelism}, forcing some instructions to be issued sequentially.

\highspace
As a result, achieving substantial performance improvements requires techniques that go \textbf{beyond individual basic blocks}, aiming to exploit ILP across larger regions of code.

\highspace
\begin{flushleft}
    \textcolor{Green3}{\faIcon{check-circle} \textbf{VLIW Processors: Main Advantages}}
\end{flushleft}
\begin{itemize}
    \item VLIW architectures represent a \textbf{powerful solution for statically exploiting ILP}. The extensive use of compiler optimizations enables VLIW processors to achieve \textbf{high performance}, even \textbf{without the complex runtime scheduling hardware} used in superscalar designs.
    
    \item Since the compiler performs \emph{dependency checking} and \emph{scheduling at compile time}, it has the \textbf{advantage of analyzing the program with a much larger instruction window} than would be feasible in hardware. This allows \textbf{more opportunities to detect parallelism}, especially when aggressive code transformations, such as loop unrolling or software pipelining, are applied.
    
    \item By transferring complexity from hardware to software, VLIW processors achieve a \textbf{significant reduction in hardware complexity}. Smaller die areas lead to \textbf{cheaper production costs} and \textbf{lower power consumption}, making VLIWs attractive for \hl{embedded applications}.
    
    Moreover, the \textbf{fixed format of VLIW instructions simplifies the decode logic}, and \textbf{scaling} to a larger number of functional units \textbf{becomes easier}.
\end{itemize}

\highspace
\begin{flushleft}
    \textcolor{Red2}{\faIcon{exclamation-triangle} \textbf{Open Challenges of VLIW Architectures}}
\end{flushleft}
Despite these advantages, VLIW designs face several open challenges:
\begin{itemize}
    \item First, they rely heavily on \important{strong compiler technology}. Sophisticated compilation techniques are necessary not only to detect parallelism within basic blocks but also to \textbf{manage parallelism across larger code regions}, which increases the complexity of compiler design.

    \item Another issue is the \important{increase in code size}. Because the instruction bundles have a fixed structure, and NOPs must be inserted to handle scheduling gaps, \textbf{increase in code size}. While code compression techniques can mitigate this effect, they \textbf{add extra decoding complexity to the processor}.

    \item \important{Register management} is also a concern. The extensive use of \textbf{register renaming} to avoid WAR and WAW hazards \textbf{increases the number of required registers} and complicates the organization of the register file. Clustered VLIW designs have been proposed to partially address this problem.

    \item Finally, \important{binary incompatibility} is a \underline{\textbf{major limitation}}.

    \textbf{Even small changes} in the number of slots, the types of functional units, or their latencies can render code compiled for one \textbf{VLIW processor incompatible with another}, even if they share the same instruction set architecture.
    
    This \textbf{lack of portability contrasts with more dynamic architectures}, where binary compatibility is preserved across generations.
\end{itemize}
Although solutions such as \textbf{Just-In-Time compilation} have been explored, they \textbf{introduce additional complexity and are rarely adopted in practice}. For this reason, VLIW processors are primarily \textbf{employed in embedded systems}, where binaries can be compiled specifically for the target hardware and do not require high portability.
