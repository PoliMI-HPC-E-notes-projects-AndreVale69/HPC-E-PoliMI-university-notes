\documentclass[a4paper]{article}
\usepackage[T1]{fontenc}			% \chapter package
\usepackage[english]{babel}
\usepackage[english]{isodate}  		% date format
\usepackage{graphicx}				% manage images
\usepackage{amsfonts}
\usepackage{booktabs}				% high quality tables
\usepackage{amsmath}				% math package
\usepackage{amssymb}				% another math package (e.g. \nexists)
\usepackage{mathtools}				% emphasize equations
\usepackage{stmaryrd} 				% '\llbracket' and '\rrbracket'
\usepackage{amsthm}					% better theorems
\usepackage{enumitem}				% manage list
\usepackage{pifont}					% nice itemize
\usepackage{cancel}					% cancel math equations
\usepackage{caption}				% custom caption
\usepackage[]{mdframed}				% box text
\usepackage{multirow}				% more lines in a table
\usepackage{textcomp, gensymb}		% degree symbol
\usepackage[x11names]{xcolor}		% RGB color
\usepackage{tcolorbox}				% colorful box
\usepackage{multicol}				% more rows in a table (used for the lists)
\usepackage{url}
\usepackage{qrcode}


% draw a frame around given text
\newcommand{\framedtext}[1]{%
	\par%
	\noindent\fbox{%
		\parbox{\dimexpr\linewidth-2\fboxsep-2\fboxrule}{#1}%
	}%
}


% table of content links
\usepackage{xcolor}
\usepackage[linkcolor=black, citecolor=blue, urlcolor=cyan]{hyperref} % hypertexnames=false
\hypersetup{
	colorlinks=true
}


\newtheorem{theorem}{\textcolor{Red3}{\underline{Theorem}}}
\renewcommand{\qedsymbol}{QED}
\newcommand{\dquotes}[1]{``#1''}
\newcommand{\longline}{\noindent\rule{\textwidth}{0.4pt}}
\newcommand{\circledtext}[1]{\raisebox{.5pt}{\textcircled{\raisebox{-.9pt}{#1}}}}
\newcommand{\definition}[1]{\textcolor{Red3}{\textbf{#1}}}
\newcommand{\example}[1]{\textcolor{Green4}{\textbf{#1}}}


\begin{document}
    \newcounter{definition}[section]
    \newtcolorbox{boxdef}{colback=red!5!white,colframe=red!75!black,fonttitle=\bfseries,title=Definizione~\refstepcounter{definition}\thedefinition}

    \author{260236}
	\title{Applied Statistics - Notes}
	\date{\printdayoff\today}
	\maketitle

	\newpage
	
	\tableofcontents
	
	\newpage

    \section{Sample Geometry and Random Sampling}

    \subsection{The Geometry of the Sample}

    \underline{A single} \definition{multivariate observation} is the \textbf{collection of measurements on $p$ different variables taken on the same item or trial}. If \textbf{$n$ observations} have been obtained, the entire data set can be placed in an $n \times p$ array (or matrix), also called \definition{data frame}:
    \begin{equation}\label{eq: data frame matrix}
        \underset{\left(n \times p\right)}{\mathbf{X}} = \begin{bmatrix}
            x_{11} & x_{12} & \cdots & x_{1p} \\
            x_{21} & x_{22} & \cdots & x_{2p} \\
            \vdots & \vdots & \ddots & \vdots \\
            x_{n1} & x_{n2} & \cdots & x_{np}
        \end{bmatrix}
    \end{equation}
    Each \textbf{row} of $\mathbf{X}$ represents a \textbf{multivariate observation}. Since the entire data frame is often one particular realization of what might have been observed, we say that the data frame are a \textbf{sample of size $n$ from a $p$-variate \dquotes{population}}. The sample then consists of $n$ measurements, each of which has $p$ components.

    Look at the matrix, $n$ measurements (rows), each of which has $p$ components (columns). In mathematics, each $n$ row contains $p$ columns and vice versa.\newline

    \noindent
    The data frame can be plotted in two different ways:
    \begin{enumerate}
        \item $p$-dimensional scatter plot, where the rows represent $n$ points in $p$-dimensional space;
        \item Geometrical representation, $p$ vectors in $n$-dimensional space.
    \end{enumerate}

    \begin{flushleft}
        \large
        \textbf{Scatter plot}
    \end{flushleft}
    For the \definition{$p$-dimensional scatter plot}, the rows of $\mathbf{X}$ represent $n$ points in $p$-dimensional space:
    \begin{equation}\label{eq: p-dimensional scatter plot}
        \underset{\left(n \times p\right)}{\mathbf{X}} = \begin{bmatrix}
            x_{11} & x_{12} & \cdots & x_{1p} \\
            x_{21} & x_{22} & \cdots & x_{2p} \\
            \vdots & \vdots & \ddots & \vdots \\
            x_{n1} & x_{n2} & \cdots & x_{np}
        \end{bmatrix} = \left[\begin{array}{@{} c @{}}
            \mathbf{x}_{1}' \\
            \mathbf{x}_{2}' \\
            \vdots \\
            \mathbf{x}_{n}'
        \end{array}\right]
        \begin{array}{l}
            \leftarrow \text{1st (multivariate) observation} \\
            \phantom{\mathbf{x}_{2}'} \\
            \phantom{\vdots} \\
            \leftarrow n\text{th (multivariate) observation}
        \end{array}
    \end{equation}
    The row vector $\mathbf{x}_{j}'$, representing the $j$th observation, contains the coordinates of a point. The \textbf{scatter plot} of $n$ points in $p$-dimensional space \textbf{provides information} on the \textbf{locations and variability of the points}.\newline

    \noindent
    \underline{\textbf{Note}}: when $p$ (dimensional space) is greater than $3$, the \textbf{scatter plot} representation cannot actually be graphed. Yet the consideration of the data as $n$ points in $p$ dimensions provides \textbf{insights that are not readily available from algebraic expressions}.

    \newpage

    \begin{flushleft}
        \large
        \textbf{Geometrical representation}
    \end{flushleft}
    The alternative \definition{geometrical representation} is constructed by considering the data as \textbf{$p$ vectors in $n$-dimensional space}. Here we take the elements of the columns of the data frame to be the coordinates of the vectors:
    \begin{equation}\label{eq: geometrical representation}
        \underset{\left(n \times p\right)}{\mathbf{X}} = \begin{bmatrix}
            x_{11} & x_{12} & \cdots & x_{1p} \\
            x_{21} & x_{22} & \cdots & x_{2p} \\
            \vdots & \vdots & \ddots & \vdots \\
            x_{n1} & x_{n2} & \cdots & x_{np}
        \end{bmatrix} = \left[\mathbf{y}_{1} \: | \: \mathbf{y}_{2} \: | \: \cdots \: | \: \mathbf{y}_{p}\right]
    \end{equation}
    Then the \textbf{coordinates} of the first point $\mathbf{y}_{1} = \left[x_{11}, x_{21}, \dots, x_{n1}\right]$ \textbf{are the $n$ measurements} on the first variable. 
    
    In general, the $i$th point $\mathbf{y}_{i} = \left[x_{11}, x_{21}, \dots, x_{n1}\right]$ is determined by the $n$-tuple of all measurements on the $i$th variable.\newline

    \noindent
    \textbf{Geometrical representations} usually \textbf{facilitate understanding} and lead to further insights. The ability to \textbf{relate algebraic expressions to the geometric concepts} of length, angle and volume is therefore \textbf{very important}.
\end{document}