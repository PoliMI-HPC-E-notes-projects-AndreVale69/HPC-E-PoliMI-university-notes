\subsection{Covariance vs. Correlation in PCA}

In the previous sections, we discussed what PCA is and how we can use it. We know that PCA is based on the covariance matrix, that each principal component explains a portion of the total variance, and that PCA depends on the variance of the data. But ``\textbf{what happens when the variables are on completely different scales?}'' (Variable 1 is height, variable 2 is euros, variable 3 is just a count). If we apply PCA to the covariance matrix, the \textbf{variable with the largest variance will dominate the result}, even if it's not more important!

\highspace
More formally, \textbf{PCA is based on the spread} (variance) \textbf{and co-movement} (covariance) \textbf{between variables}. But this depends heavily on the units of measurement and scales of the variables. And the problem is, should \textbf{we use the covariance matrix or the correlation matrix} to perform PCA? The answer depends on the scales of our variables.

\highspace
\begin{flushleft}
    \textcolor{Red2}{\faIcon{exclamation-triangle} \textbf{Covariance Matrix: use with caution}}
\end{flushleft}
The \textbf{covariance matrix measures how variables change together}, in \hl{absolute units}.
\begin{itemize}
    \item[\textcolor{Green3}{\faIcon{check}}] This is fine only if \textbf{all variables are measured on comparable scales} (e.g., all in centimeters, or all in euros).

    \item[\textcolor{Red2}{\faIcon{times}}] If one variable has \textbf{much larger variance}, it will \textbf{dominate the PCA}, even if it's not the most important feature.
\end{itemize}
For example, suppose height (meters) has a variance of $0.01$ and income (euros) has a variance of $10'000$. Income will control the direction of PC1 simply because its variance is greater, not because it's more meaningful.

\highspace
\begin{flushleft}
    \textcolor{Green3}{\faIcon{check-circle} \textbf{Solution: Standardize the Data}}
\end{flushleft}
To prevent PCA from being biased toward large-scale variables, we \textbf{standardize} each variable:
\begin{equation}
    X'_{ij} = \frac{X_{ij} - \bar{X}_{j}}{\text{SD}_{j}}
\end{equation}
Where:
\begin{itemize}
    \item $\bar{x}_{j}$ is the mean of variable $j$.
    \item $\text{SD}_{j}$ is the standard deviation of variable $j$.
\end{itemize}
This standardization ensures that the \textbf{mean is zero} and the \textbf{standard deviation is one}. Now \textbf{all variables are on the same scale} and PCA treats them equally.

\newpage

\begin{flushleft}
    \textcolor{Green3}{\faIcon{check-circle} \textbf{Alternative: use the correlation matrix instead of the covariance matrix}}
\end{flushleft}
Another way to achieve this is to run PCA on the \textbf{correlation matrix}, which is simply the \textbf{covariance matrix of the standardized variables}. This work because:
\begin{equation}
    \text{Corr}(X_{j}, X_{k}) = \dfrac{\text{Cov}(X_{j}, X_{k})}{\text{SD}\left(X_{j}\right) \cdot \text{SD}\left(X_{k}\right)}
\end{equation}
So if we \textbf{standardize the variables first, the covariance becomes correlation}.

\highspace
\begin{table}[!htp]
    \centering
    \begin{adjustbox}{width={\textwidth},totalheight={\textheight},keepaspectratio}
        \begin{tabular}{@{} l | l | l @{}}
            \toprule
            \textbf{Case}                           & \textbf{Matrix to use}              & \textbf{Why} \\
            \midrule
            All variables are on the same scale     & Covariance matrix                   & Keeps units and variances as they are \\ [.3em]
            Variables have different units/scales   & Correlation matrix                  & Avoids bias toward large-variance variables \\ [.3em]
            We standardized the variables           & Covariance \texttt{=} correlation   & Because standardization equalizes them \\
            \bottomrule
        \end{tabular}
    \end{adjustbox}
    \caption{When to use Covariance or Correlation.}
\end{table}