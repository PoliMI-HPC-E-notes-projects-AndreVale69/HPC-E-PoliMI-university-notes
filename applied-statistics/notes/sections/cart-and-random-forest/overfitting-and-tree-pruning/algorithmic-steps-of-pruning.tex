\subsubsection{Algorithmic Steps of Pruning}

Find the subtree $T_\alpha \subseteq T_0$ that minimizes:
\begin{equation*}
    C_\alpha(T) = \text{RSS}(T) + \alpha \cdot \left|T\right|
\end{equation*}
Where:
\begin{itemize}
    \item $T_0$ is the large, overfitted tree.
    \item $|T|$ is the number of terminal nodes (leaves).
    \item $\alpha$ controls the trade-off between fit and complexity.
\end{itemize}

\highspace
\begin{flushleft}
    \textcolor{Green3}{\faIcon{tools} \textbf{Step-by-Step Pruning Algorithm}}
\end{flushleft}
\begin{enumerate}
    \item \important{Grow a large tree $T_0$}. Use \textbf{recursive binary splitting} (page \pageref{subsubsection: Recursive Binary Splitting}) without worrying about overfitting. This tree captures as many splits as possible (very low training error).

    \item \important{Generate a sequence of subtrees}. For increasing values of $\alpha$, prune the tree step by step. Each $\alpha$ corresponds to a subtree $T_{\alpha}$. This gives a \textbf{sequence of nested subtrees}:
    \begin{equation*}
        T_0 \supset T_1 \supset T_2 \supset \cdots \supset T_M
    \end{equation*}

    \item \important{Use $K$-fold Cross-Validation to select $\alpha$}. For each fold:
    \begin{enumerate}
        \item Remove 1 fold from the training data (validation fold).
        \item On the remaining $K-1$ folds:
        \begin{itemize}
            \item Grow a large tree $T_0$
            \item Compute the subtrees $T_{\alpha}$ for several values of $\alpha$
        \end{itemize}
        \item Evaluate the \textbf{prediction error} (e.g. MSE) of each subtree on the held-out fold.
    \end{enumerate}
    Average the results across all folds to estimate test error for each $\alpha$.

    \item \important{Choose the best $\alpha$}. Pick the value $\hat{\alpha}$ that minimizes the \textbf{average cross-validation error}. This gives us the \textbf{optimal subtree} $T_{\hat{\alpha}}$.

    \item \important{Retrain on full data}. Use the \textbf{entire training set} to grow a large tree. Then \textbf{prune it} using the selected $\hat{\alpha}$ to \hl{get the final tree}.
\end{enumerate}
It allows us to \textbf{compare subtrees of different sizes}.