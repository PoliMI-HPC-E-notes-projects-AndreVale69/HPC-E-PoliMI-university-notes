\subsubsection{Minimizing RSS: The Criterion Behind Tree Splits}

In regression, the \textbf{Residual Sum of Squares} (page \pageref{eq: RSS}) measures the \textbf{total squared error} between predicted and actual values:
\begin{equation*}
    \text{RSS} = \sum_{i=1}^n (y_i - \hat{y}_i)^2    
\end{equation*}
In a \textbf{regression tree}, each region $R_j$ (leaf) predicts a constant: the \textbf{average} $\hat{y}_j$ of $y$-values in that region. So the \definitionWithSpecificIndex{RSS}{Residual Sum of Squares (RSS) in Regression Trees}{} becomes:
\begin{equation}
    \text{Total RSS} = \sum_{j=1}^J \sum_{i \in R_j} \left(y_i - \bar{y}_{R_j}\right)^2    
\end{equation}

\highspace
\begin{flushleft}
    \textcolor{Green3}{\faIcon{question-circle} \textbf{How a Split Minimizes RSS}}
\end{flushleft}
To decide the \textbf{best split}, the algorithm:
\begin{enumerate}
    \item Tries \textbf{all possible features} $X_j$ and \textbf{all possible thresholds} $s$
    \item Splits the data into two regions:
    \begin{itemize}
        \item Left: $R_1 = \left\{ X_j < s \right\}$
        \item Right: $R_2 = \left\{ X_j \geq s \right\}$
    \end{itemize}
    \item Computes:
    \begin{equation*}
        \text{RSS}_{\text{split}} = \sum_{i \in R_1} \left(y_i - \bar{y}_{R_1}\right)^2 + \sum_{i \in R_2} \left(y_i - \bar{y}_{R_2}\right)^2
    \end{equation*}
    \item Chooses the split with the \textbf{lowest RSS}.
\end{enumerate}

\highspace
\begin{flushleft}
    \textcolor{Green3}{\faIcon{question-circle} \textbf{Why use RSS?}}
\end{flushleft}
It rewards \textbf{homogeneous groups}: if a region has responses $y_i$ all close together, its RSS is small. It penalizes \textbf{heterogeneous groups}, where responses vary widely.