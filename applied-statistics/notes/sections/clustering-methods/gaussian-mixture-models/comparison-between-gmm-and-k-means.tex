\subsubsection{Comparison between GMM and K-Means}

Even though both Gaussian Mixture Models (GMM) and K-Means are used for \textbf{clustering}, they differ significantly in their assumptions, outputs, and flexibility.
\begin{itemize}
    \item \important{Conceptual Difference}
    \begin{table}[!htp]
        \begin{adjustbox}{width={\textwidth},totalheight={\textheight},keepaspectratio}
            \centering
            \begin{tabular}{@{} l | l | p{18em} @{}}
                \toprule
                \textbf{Concept} & \textbf{K-Means} & \textbf{GMM} \\
                \midrule
                Type    & Geometric algorithm               & Probabilistic generative model                    \\ [.3em]
                Goal    & Minimize distances to centroids   & Maximize likelihood of data under Gaussian model  \\ [.3em]
                Output  & Hard assignments (1 cluster)      & Soft assignments (probabilities per cluster)      \\
                \bottomrule
            \end{tabular}
        \end{adjustbox}
    \end{table}

    \item \important{Shape and Structure}
    \begin{table}[!htp]
        \begin{adjustbox}{width={\textwidth},totalheight={\textheight},keepaspectratio}
            \centering
            \begin{tabular}{@{} l | p{11em} | p{20em} @{}}
                \toprule
                \textbf{Feature} & \textbf{K-Means} & \textbf{GMM} \\
                \midrule
                Cluster Shape   & Spherical (equal radius)             & Elliptical (any orientation/size via covariance) \\ [.3em]
                Covariance Info & Not modeled                          & Fully modeled via $\Sigma_k$ for each cluster    \\ [.3em]
                Equal size      & Assumes clusters are similar in size & Allows different size and shape per cluster      \\
                \bottomrule
            \end{tabular}
        \end{adjustbox}
    \end{table}
    

    \item \important{Assignment Method}
    \begin{itemize}
        \item \textbf{K-Means} assigns each point to the \textbf{nearest centroid} (based on Euclidean distance).
        \item \textbf{GMM} assigns each point a \textbf{probability} of belonging to each cluster (based on likelihood).
    \end{itemize}
    This makes GMM better for:
    \begin{itemize}
        \item Overlapping clusters.
        \item Uncertain or fuzzy boundaries.
        \item Data with correlated features.
    \end{itemize}


    \item \important{Optimization Strategy}
    \begin{itemize}
        \item \textbf{K-Means} minimizes \textbf{within-cluster sum of squared distances}.
        \item \textbf{GMM} maximizes the \textbf{log-likelihood} of the observed data via the \textbf{EM algorithm}.
    \end{itemize}


    \newpage


    \item \important{Computational Cost}
    \begin{table}[!htp]
        % \begin{adjustbox}{width={\textwidth},totalheight={\textheight},keepaspectratio}
            \centering
            \begin{tabular}{@{} l | l | l @{}}
                \toprule
                \textbf{Aspect} & \textbf{K-Means} & \textbf{GMM} \\
                \midrule
                Speed           & Faster        & Slower (EM is iterative)                      \\ [.3em]
                Convergence     & Guaranteed    & Also guaranteed, but may find local optima    \\ [.3em]
                Initialization  & Important     & Even more sensitive to initialization         \\
                \bottomrule
            \end{tabular}
        % \end{adjustbox}
    \end{table}

    
    \item \important{Summary Table}
    \begin{table}[!htp]
        % \begin{adjustbox}{width={\textwidth},totalheight={\textheight},keepaspectratio}
            \centering
            \begin{tabular}{@{} l | l | l @{}}
                \toprule
                \textbf{Feature} & \textbf{K-Means} & \textbf{GMM} \\
                \midrule
                Assignments                 & Hard                                  & Soft (probabilities)                      \\ [.3em]
                Cluster shape               & Spherical                             & Elliptical (via covariance)               \\ [.3em]
                Uses probability model?     & \textcolor{Red2}{\faIcon{times}} No   & \textcolor{Green3}{\faIcon{check}} Yes    \\ [.3em]
                Optimization objective      & Distance                              & Log-likelihood                            \\ [.3em]
                Suitable for overlapping?   & \textcolor{Red2}{\faIcon{times}} No   & \textcolor{Green3}{\faIcon{check}} Yes    \\
                \bottomrule
            \end{tabular}
        % \end{adjustbox}
    \end{table}


    \item \important{When to Use What?}
    \begin{table}[!htp]
        \begin{adjustbox}{width={\textwidth},totalheight={\textheight},keepaspectratio}
            \centering
            \begin{tabular}{@{} l | l @{}}
                \toprule
                \textbf{Use Case} & \textbf{Choose} \\
                \midrule
                We need fast, simple partitioning              & K-Means                      \\ [.3em]
                We want a statistical model or density info    & GMM                          \\ [.3em]
                Clusters overlap or vary in shape              & GMM                          \\ [.3em]
                We have very large datasets                    & K-Means (or approximate GMM) \\
                \bottomrule
            \end{tabular}
        \end{adjustbox}
    \end{table}
\end{itemize}
