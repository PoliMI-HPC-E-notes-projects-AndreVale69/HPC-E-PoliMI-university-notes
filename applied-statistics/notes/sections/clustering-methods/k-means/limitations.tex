\subsubsection{Limitations}

While K-Means is simple and efficient, it has several important limitations:
\begin{enumerate}
    \item \important{We must choose the number of clusters $K$ in advance}. Unlike \textbf{hierarchical clustering}, which builds a full tree of clusterings, K-Means requires we to \textbf{fix $K$ before the algorithm starts}.
    \begin{itemize}[label=\textcolor{Red2}{\faIcon{times}}]
        \item Choosing the wrong $K$ can lead to poor results.
        \item There's no universal rule for selecting $K$; methods like the Elbow or Silhouette must be used.
    \end{itemize}


    \item \important{Assumes clusters are globular (spherical) in shape}. K-Means tends to form \textbf{round-shaped clusters} due to the use of \textbf{Euclidean distance}.
    \begin{itemize}[label=\textcolor{Red2}{\faIcon{times}}]
        \item It \textbf{struggles with elongated, irregular, or non-convex shapes}.
        \item Works best when clusters are \textbf{well-separated, equally sized, and dense}.
    \end{itemize}


    \item \important{Fails with clusters of different sizes or densities}. If one cluster is very dense and another is sparse, K-Means might:
    \begin{itemize}[label=\textcolor{Red2}{\faIcon{times}}]
        \item Split one true cluster into several pieces
        \item Merge multiple small clusters into one
    \end{itemize}
    This happens because Euclidean distance doesn't account for cluster \textbf{spread or shape}.


    \item \important{Sensitive to outliers}. K-Means uses \textbf{centroids (means)} to define clusters. Since the \textbf{mean is sensitive to extreme values}, an outlier can significantly shift a centroid and distort the clustering.

    As an alternative, we could use \definition{K-Medoids}, which is \textbf{based on medians rather than means} and is more robust to outliers.

    
    \item \important{Sensitive to feature scaling and representation}. Because K-Means relies on distance, the algorithm is:
    \begin{itemize}[label=\textcolor{Red2}{\faIcon{times}}]
        \item Affected by \textbf{unscaled or poorly represented data}
        \item Sometimes improved by \textbf{dimensionality reduction} (e.g., PCA)
    \end{itemize}
\end{enumerate}