\subsection{Comparison LDA vs QDA}

\highspace
\begin{flushleft}
    \textcolor{Green3}{\faIcon{tools} \textbf{Core Similarities}}
\end{flushleft}

\begin{table}[!htp]
    \centering
    \begin{tabular}{@{} l p{18em} @{}}
        \toprule
        Aspect & Description \\
        \midrule
        \textbf{Both are generative}            & They model how each class generates data using a multivariate Gaussian.   \\ [.3em]
        \textbf{Both use Bayes' rule}           & Classification is based on computing posterior probabilities.             \\ [.3em]
        \textbf{Both are supervised}            & They require labeled training data.                                       \\ [.3em]
        \textbf{Both work for multiple classes} & Not limited to binary classification.                                     \\
        \bottomrule
    \end{tabular}
\end{table}

\highspace
\begin{flushleft}
    \textcolor{Green3}{\faIcon{balance-scale} \textbf{Key Differences}}
\end{flushleft}

\begin{table}[!htp]
    \centering
    \begin{adjustbox}{width={\textwidth},totalheight={\textheight},keepaspectratio}
        \begin{tabular}{@{} l p{13.5em} p{13.5em} @{}}
            \toprule
            Feature & \textbf{LDA} & \textbf{QDA} \\
            \midrule
            \textbf{Covariance assumption}      & Same for all classes: $\Sigma$    & Unique for each class: $\Sigma_k$                     \\ [.5em]
            \textbf{Decision boundary}          & Linear (straight lines/planes)    & Quadratic (curved boundaries)                         \\ [.5em]
            \textbf{Model complexity}           & Fewer parameters, then simpler    & More parameters, then more flexible                   \\ [.5em]
            \textbf{Training data needed}       & Works well with small datasets    & Needs more data (per class)                           \\ [.5em]
            \textbf{Risk of overfitting}        & Lower                             & Higher                                                \\ [.5em]
            \textbf{Bias-Variance tradeoff}     & Higher bias, lower variance       & Lower bias, higher variance                           \\ [.5em]
            \textbf{Shape of decision region}   & All classes have same shape       & Each class can have different shape/size/orientation  \\
            \bottomrule
        \end{tabular}
    \end{adjustbox}
\end{table}

\highspace
\begin{flushleft}
    \textcolor{Green3}{\faIcon{check-circle} \textbf{Rule of Thumb}}
\end{flushleft}

\begin{table}[!htp]
    \centering
    \begin{adjustbox}{width={\textwidth},totalheight={\textheight},keepaspectratio}
        \begin{tabular}{@{} l l @{}}
            \toprule
            Situation & Recommended \\
            \midrule
            We want a \textbf{stable model} with limited data                           & Use \textbf{LDA}                  \\ [.5em]
            We have \textbf{enough data} and suspect curved decision boundaries         & Try \textbf{QDA}                  \\ [.5em]
            Our classes look like \textbf{different clouds} (e.g. rotated, stretched)   & Use \textbf{QDA}                  \\ [.5em]
            Our classes are well-separated by a \textbf{straight line/plane}            & \textbf{LDA} is likely sufficient \\
            \bottomrule
        \end{tabular}
    \end{adjustbox}
\end{table}