\subsubsection{GLMs for Count Data - The Poisson Model}

Suppose we want to model \textbf{counts} (e.g., number of emails per day, number of accidents per month, number of clicks per ad). The response variable $Y_i$ is must be a \textbf{non-negative integer}: 0, 1, 2, etc.

\highspace
The standard model for counts is the \definition{Poisson Distribution}:
\begin{equation}
    Y_i \sim \text{Poisson}(\lambda_i)
\end{equation}
The \textbf{mean} of $Y_i$ is:
\begin{equation}
    E[Y_i] = \lambda_i \quad (\lambda_i > 0)
\end{equation}

\highspace
\begin{flushleft}
    \textcolor{Green3}{\faIcon{check-circle} \textbf{Link function and invert link function}}
\end{flushleft}
We need to force $\lambda_i > 0$. So we use the \textbf{log link}:
\begin{equation}
    g\left(\lambda_i\right) = \log\left(\lambda_i\right) = X_i^T \beta
\end{equation}
To get the mean back, invert the link:
\begin{equation}
    \lambda_i = e^{X_{i}^{T} \beta}
\end{equation}
So the GLM for counts is:
\begin{itemize}
    \item \important{Random part}:
    \begin{equation*}
        Y_i \sim \text{Poisson}(\lambda_i)
    \end{equation*}
    \item \important{Systematic part}:
    \begin{equation*}
        \eta_i = X_i^T \beta
    \end{equation*}
    \item \important{Link function}:
    \begin{equation*}
        \log(\lambda_i) = \eta_i
        \quad
        \Longrightarrow
        \quad
        \lambda_i = \exp(\eta_i)
    \end{equation*}
\end{itemize}

\highspace
\begin{flushleft}
    \textcolor{Green3}{\faIcon{question-circle} \textbf{What does this do?}}
\end{flushleft}
Ensures our predicted mean is always positive. Allows the \textbf{multiplicative effect}: a one-unit change in $X_j$ multiplies $\lambda$ by $e^{\beta_j}$. For example, if $\beta_j = 0.1$, then a one-unit increase in $X_j$ \textbf{multiplies the expected count} by $e^{0.1} \approx 1.105$. So we get about a \textbf{$10.5\%$ increase} in the expected count.