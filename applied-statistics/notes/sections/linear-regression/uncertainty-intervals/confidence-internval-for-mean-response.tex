\subsection{Uncertainty Intervals}

After fitting a regression model, we often want to \textbf{predict} the response $Y$ at a new value of $X$. But point predictions like:
\begin{equation*}
    \hat{Y} = \hat{\beta}_0 + \hat{\beta}_1 X
\end{equation*}
\emph{Only} give a \textbf{single number}. That's not enough. In real life, we care about:
\begin{itemize}
    \item \textbf{How uncertain is this prediction?}
    \item ``\emph{What range of outcomes should we expect?}''
\end{itemize}
\textcolor{Green3}{\faIcon{check-circle} \textbf{Goal.}} To build \textbf{intervals} around predictions that account for uncertainty. We study two types:
\begin{itemize}
    \item \important{Confidence Interval (CI) for the Mean Response} (page \pageref{subsubsection: Confidence Interval for the Mean Response}). It calculates the range of the \textbf{average} $Y$ value when the experiment is repeated many times at the same $X$ value. In other words, it \textbf{captures uncertainty in the mean prediction}.

    Used when:
    \begin{itemize}
        \item We're interested in the \textbf{mean trend}.
        \item We're estimating the regression line itself.
    \end{itemize}


    \item \important{Prediction Interval (PI) for a New Observation} (page \pageref{subsubsection: Prediction Interval for a New Observation}). It calculates the range of a new $Y$ value at a given $X$ value, taking randomness into account. In other words, it \textbf{captures the uncertainty in individual outcome}.
    
    Used when:
    \begin{itemize}
        \item We want to \textbf{predict an individual outcome}.
        \item We care about real-world variability.
    \end{itemize}
\end{itemize}

\newpage

\subsubsection{Confidence Interval (CI) for the Mean Response} \label{subsubsection: Confidence Interval for the Mean Response}

Suppose we have fit a regression model:
\begin{equation*}
    \hat{Y} = \hat{\beta}_0 + \hat{\beta}_1 X    
\end{equation*}
Now we pick a specific $X = x_0$, and ask: ``\emph{what's the range of values for the \textbf{mean of $Y$} at this $x_0$?}''. This is the \definition{Confidence Interval (CI) for the Mean Response} and the formula is:
\begin{equation}
    \hat{Y}_0 \pm t_{n-2} \cdot \text{SE}\left(\hat{Y}_0\right)
\end{equation}
Where:
\begin{itemize}
    \item $\hat{Y}_0 = \hat{\beta}_0 + \hat{\beta}_1 x_0$ is the predicted mean.
    \item $\text{SE}\left(\hat{Y}_0\right)$ is the standard error of that prediction.
    \item $t_{n-2}$ comes from a t-distribution with $n-2$ degrees of freedom.
\end{itemize}

\highspace
\begin{flushleft}
    \textcolor{Green3}{\faIcon{check-circle} \textbf{Why do we need it?}}
\end{flushleft}
Even if we know $x_0$, we don't know the true $\beta_0$, $\beta_1$, so our prediction $\hat{Y}_0$ is uncertain. This CI tells us: ``\emph{if we repeated this experiment many times and fit a new line each time, what range would the average $Y$ fall in at $x_0$?}''.

\highspace
\begin{flushleft}
    \textcolor{Green3}{\faIcon{question-circle} \textbf{How to compute the Standard Error}}
\end{flushleft}
\begin{equation}
    \text{SE}\left(\hat{Y}_0\right) = \sqrt{ \hat{\sigma}^2 \left( \dfrac{1}{n} + \dfrac{(x_0 - \bar{X})^2}{\displaystyle\sum \left(X_i - \bar{X}\right)^{2}} \right) }
\end{equation}
Where:
\begin{itemize}
    \item $\hat{\sigma}^2$ is the MSE (mean squared error).
    \item The first term $\frac{1}{n}$ accounts for uncertainty in the intercept.
    \item The second term captures how far $x_0$ is from the center of the data.
\end{itemize}
Note: the CI is \textbf{narrowest at} $\bar{X}$ and gets wider as $x_0$ moves away.

\begin{figure}[!htp]
    \centering
    \includegraphics[width=\textwidth]{img/linear-regression/ci-for-mean-response.pdf}
    \caption{This plot shows a visual of the Confidence Interval (CI) for the Mean Response at $x_0 = 5$. The red dashed vertical line indicates the 95\% confidence interval for the \textbf{average} $Y$ at $x_0 = 5$, even if it is small. So, if we repeated this regression many times, we're 95\% confident the average value of $Y$ when $X = 5$ would fall in that vertical interval.}
    \label{fig: Confidence Interval for the Mean Response}
\end{figure}
