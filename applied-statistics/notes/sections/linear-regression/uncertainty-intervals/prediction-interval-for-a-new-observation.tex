\subsubsection{Prediction Interval (PI) for a New Observation}\label{subsubsection: Prediction Interval for a New Observation}

We've fit a regression model:
\begin{equation*}
    \hat{Y} = \hat{\beta}_0 + \hat{\beta}_1 X    
\end{equation*}
Now we pick a specific value $X = x_0$ and ask: ``\emph{if we get a \textbf{new data point} with $X = x_0$, what range of values could the actual $Y$ fall into?}''. That's the \definition{Prediction Interval (PI) for a New Observation} and the formula is:
\begin{equation}
    \hat{Y}_0 \pm t_{n-2} \cdot \text{SE}_{\text{pred}}(x_0)
\end{equation}
Where:
\begin{equation}
    \text{SE}_{\text{pred}}(x_0) = \sqrt{ \hat{\sigma}^2 \left( 1 + \dfrac{1}{n} + \dfrac{\left(x_0 - \bar{X}\right)^2}{\displaystyle\sum \left(X_i - \bar{X}\right)^2} \right) }
\end{equation}
Compared to the Confidence Interval (page \pageref{subsubsection: Confidence Interval for the Mean Response}), this has an \textbf{extra} $+1$ inside the square root, which accounts for the \textbf{random error in a single new observation}.

\highspace
\begin{flushleft}
    \textcolor{Green3}{\faIcon{check-circle} \textbf{Why do we need it?}}
\end{flushleft}
Because real-world observations aren't always on the line, they \textbf{scatter around} due to noise $\varepsilon$. So we want a range that captures \textbf{both}:
\begin{itemize}
    \item The \textbf{uncertainty in the estimated mean} at $x_0$, and
    \item The \textbf{random variation in a new individual outcome}
\end{itemize}

\highspace
\begin{flushleft}
    \textcolor{Green3}{\faIcon{balance-scale} \textbf{Difference vs. Confidence Interval}}
\end{flushleft}
\begin{table}[!htp]
    \centering
    \begin{tabular}{@{} l l l @{}}
        \toprule
        Interval Type & Captures & Width \\
        \midrule
        \textbf{CI} & Uncertainty in \emph{average} $Y$ & Narrower \\
        \textbf{PI} & Uncertainty in \emph{single} $Y$  & \textbf{Wider}, includes noise! \\
        \bottomrule
    \end{tabular}
\end{table}

\begin{figure}[!htp]
    \centering
    \includegraphics[width=\textwidth]{img/linear-regression/pi-for-a-new-observation.pdf}
    \captionsetup{singlelinecheck=off}
    \caption[]{This plot illustrates the visual comparison of the two types of uncertainty interval at $x_{0} = 5$. The red point is the predicted value $\hat{Y}_0$ at $x_0 = 5$.
    
    \highspace
    Unlike Figure \ref{fig: Confidence Interval for the Mean Response} on page \pageref{fig: Confidence Interval for the Mean Response}, we also draw the prediction interval here.
    \begin{itemize}
        \item Green dashed line is the Confidence Interval (CI), for the mean of $Y$ at $X = 5$. It is \textbf{narrow} because it only accounts for uncertainty in estimating the mean. It seems invisible because it is smaller than PI.

        The CI tells us: ``\emph{where would the \textbf{average} $Y$ be at $X = 5$?}''.
        \item Red dotted line is the Prediction Interval (PI), for a \textbf{new individual} with $X = 5$. It is \textbf{wider} because it also includes randomness (noise) in future observations.

        The PI tells us: ``\emph{where would a \textbf{single new} $Y$ likely fall at $X = 5$?}''.
    \end{itemize}}
\end{figure}
