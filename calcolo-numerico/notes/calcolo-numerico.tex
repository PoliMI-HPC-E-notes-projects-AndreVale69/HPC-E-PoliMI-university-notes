\documentclass[a4paper]{article}
\usepackage[T1]{fontenc}			% \chapter package
\usepackage[italian]{babel}
\usepackage[italian]{isodate}  		% date format
\usepackage{graphicx}				% manage images
\usepackage{amsfonts}
\usepackage{booktabs}				% high quality tables
\usepackage{amsmath}				% math package
\usepackage{amssymb}				% another math package (e.g. \nexists)
\usepackage{bm}                     % bold math symbols
\usepackage{mathtools}				% emphasize equations
\usepackage{stmaryrd} 				% '\llbracket' and '\rrbracket'
\usepackage{amsthm}					% better theorems
\usepackage{enumitem}				% manage list
\usepackage{pifont}					% nice itemize
\usepackage{cancel}					% cancel math equations
\usepackage{caption}				% custom caption
\usepackage[]{mdframed}				% box text
\usepackage{multirow}				% more lines in a table
\usepackage{textcomp, gensymb}		% degree symbol
\usepackage[x11names]{xcolor}		% RGB color
\usepackage[many]{tcolorbox}		% colorful box
\usepackage{multicol}				% more rows in a table (used for the lists)
\usepackage{listings}
\usepackage{url}
\usepackage{qrcode}
\usepackage{fontawesome5}
\usepackage{ragged2e}
\usepackage{cite}                   % references
\usepackage{imakeidx}               % index
\makeindex[program=makeindex, columns=1,
           title=Index, 
           intoc,
           options={-s index-style.ist}]
\usepackage{fancyhdr}
\usepackage{soul}                   % highlight text (as a highlighter)
\usepackage{pdflscape}   % rotates the page in the PDF viewer
\usepackage{adjustbox}    % adjust boxes and tables to text width

%\pdfcompresslevel=0
%\pdfobjcompresslevel=0

\definecolor{codegreen}{rgb}{0,0.6,0}
\definecolor{codegray}{rgb}{0.5,0.5,0.5}
\definecolor{codepurple}{rgb}{0.58,0,0.82}
\definecolor{backcolour}{rgb}{0.95,0.95,0.92}
\lstdefinestyle{mystyle}{
    backgroundcolor=\color{backcolour},   
    commentstyle=\color{codegreen},
    keywordstyle=\color{magenta},
    numberstyle=\tiny\color{codegray},
    stringstyle=\color{codepurple},
    basicstyle=\ttfamily\footnotesize,
    breakatwhitespace=false,         
    breaklines=true,                 
    captionpos=b,                    
    keepspaces=true,                 
    numbers=left,                    
    numbersep=5pt,                  
    showspaces=false,                
    showstringspaces=false,
    showtabs=false,                  
    tabsize=2
}
\lstset{style=mystyle}

\makeatletter
\renewcommand\paragraph{\@startsection{paragraph}{4}{\z@}%
            {-2.5ex\@plus -1ex \@minus -.25ex}%
            {1.25ex \@plus .25ex}%
            {\normalfont\normalsize\bfseries}}
\makeatother
\setcounter{secnumdepth}{4} % how many sectioning levels to assign numbers to
\setcounter{tocdepth}{4}    % how many sectioning levels to show in ToC

% draw a frame around given text
\newcommand{\framedtext}[1]{%
	\par%
	\noindent\fbox{%
		\parbox{\dimexpr\linewidth-2\fboxsep-2\fboxrule}{#1}%
	}%
}


% table of content links
\usepackage{xcolor}
\usepackage[linkcolor=black, citecolor=blue, urlcolor=cyan]{hyperref} % hypertexnames=false
\hypersetup{
	colorlinks=true
}


\newtheorem{theorem}{\textcolor{Red3}{\underline{Teorema}}}
\renewcommand{\qedsymbol}{QED}
\newcommand{\dquotes}[1]{``#1''}
\newcommand{\important}[1]{\textcolor{red}{\textbf{#1}}}
\newcommand{\longline}{\noindent\rule{\textwidth}{0.4pt}}
\newcommand{\circledtext}[1]{\raisebox{.5pt}{\textcircled{\raisebox{-.9pt}{#1}}}}
\newcommand{\definition}[1]{\textcolor{Red3}{\textbf{#1}}\index{#1}}
\newcommand{\example}[1]{\textcolor{Green4}{\textbf{#1}}}
\newcommand{\highspace}{\vspace{1.2em}\noindent}
% I never remember every icon, so I use custom commands to do this
\newcommand{\speedIcon}{tachometer-alt}
\newcommand{\definitionWithSpecificIndex}[3]{\textcolor{Red3}{\textbf{#1}}\index{#2}\index{#3}}
\newcommand{\hqlabel}[2]{\label{#1}\hypertarget{#1}{#2}}
\newcommand{\hqpageref}[1]{\hyperlink{#1}{\hypergetpageref{#1}}}
\newcommand{\aaccent}{à }
\newcommand{\eaccent}{è }
\newcommand{\eiAccent}{é }
\newcommand{\oaccent}{ò }
\newcommand{\uaccent}{ù }
\newcommand{\iaccent}{ì }
\newcommand{\version}{v1.2.0}
\newenvironment{rowequmat}[1]{\left(\array{@{}#1@{}}}{\endarray\right)}
\newenvironment{rowequmatbra}[1]{\left[\array{@{}#1@{}}}{\endarray\right]}

% \includeonly{
%     sections/preface,
%     sections/domande-teoriche-frequenti,
%     sections/esami/24-07-2025,
%     sections/esami/14-06-2025
% }

\begin{document}
    \newcounter{definition}[section]
    \newcounter{example}[section]
    \newcounter{remark}[section]
    \newcounter{takeaways}[section]
    
    \newtcolorbox[use counter = definition]{definitionbox}[1][]{%
        breakable,
        enhanced,
        colback=red!5!white,
        colframe=red!75!black,
        fonttitle=\bfseries,
        title={Definizione \thetcbcounter #1} %
    }
    
    \newtcolorbox[use counter = example]{examplebox}[1][]{%
        breakable,
        enhanced,
        colback=Green4!5!white,
        colframe=Green4!75!black,
        fonttitle=\bfseries,
        title={Esempio \thetcbcounter #1} %
    }

    \newtcolorbox[use counter = takeaways]{takeawaysbox}[1][]{%
        breakable,
        enhanced,
        colback=Green3!5!white,
        colframe=Green3!75!black,
        fonttitle=\bfseries,
        title={Key Takeaways#1} %
    }

    \newtcolorbox[]{deepeningbox}[1][]{%
        breakable,
        enhanced,
        colback=DarkOrange3!5!white,
        colframe=DarkOrange3!75!black,
        fonttitle=\bfseries,
        title={Deepening#1} %
    }
    
    \newtcolorbox[use counter = remark]{remarkbox}[1][]{%
      breakable,
      enhanced,
      colback=blue!5!white,
      colframe=blue!75!black,
      fonttitle=\bfseries,
      title={Remark \thetcbcounter #1} %
    }

    %%%%%%%%%%%%%%%
    % Notes cover %
    %%%%%%%%%%%%%%%
    \author{260236}
\title{Quantum Computing - Notes - \version}
\date{\printdayoff\today}
\maketitle

    %%%%%%%%%%%
    % Preface %
    %%%%%%%%%%%
	\section*{Preface}

Every theory section in these notes has been taken from two sources:
\begin{itemize}
    \item Ian Goodfellow and Yoshua Bengio and Aaron Courville, \href{https://www.deeplearningbook.org/}{Deep Learning}, MIT Press. \cite{Goodfellow-et-al-2016}
    \item Course slides.\cite{course-slides-polimi}
\end{itemize}
About:
\begin{itemize}
    \item[\faIcon{github}] \href{https://github.com/PoliMI-HPC-E-notes-projects-AndreVale69/HPC-E-PoliMI-university-notes}{GitHub repository}
    \begin{center}
        \qrcode{https://github.com/PoliMI-HPC-E-notes-projects-AndreVale69/HPC-E-PoliMI-university-notes}
    \end{center}
\end{itemize}
These notes are an unofficial resource and shouldn't replace the course material or any other book on artificial neural networks and deep learning. It is not made for commercial purposes. I've made the following notes to help me improve my knowledge and maybe it can be helpful for everyone.

As I have highlighted, a student should choose the teacher's material or a book on the topic. These notes can only be a helpful material.

\highspace
During the Artificial Neural Networks and Deep Learning course, me and my other three colleagues Alberto Ondei, \href{https://it.linkedin.com/in/javedabdullah}{Abdullah Javed} and \href{https://www.hpc.cineca.it/staff/barbari-filippo/}{Filippo Barbari}, we created two projects:
\begin{itemize}
    \item[\faIcon{github}] \href{https://github.com/PoliMI-HPC-E-notes-projects-AndreVale69/AN2DL-Challenge-1}{Time Series Classification}. Deep Neural Netowrk (TCN $+$ BiLSTM with Attention) model on multivariate time-series classification.
    \begin{center}
        \qrcode{https://github.com/PoliMI-HPC-E-notes-projects-AndreVale69/AN2DL-Challenge-1}
    \end{center}
    \item[\faIcon{github}] \href{https://github.com/PoliMI-HPC-E-notes-projects-AndreVale69/AN2DL-Challenge-2}{Image Classification}. Histopathology image classification to predict molecular subtypes.
    \begin{center}
        \qrcode{https://github.com/PoliMI-HPC-E-notes-projects-AndreVale69/AN2DL-Challenge-2}
    \end{center}
\end{itemize}

    %%%%%%%%%%%%%%%%%%%%%
    % Table of contents %
    %%%%%%%%%%%%%%%%%%%%%
    \tableofcontents
    \newpage

    %%%%%%%%%%%%%%%%%%%
    % Fancy pagestyle %
    %%%%%%%%%%%%%%%%%%%
    \pagestyle{fancy}
    \fancyhead{} % clear all header fields
    \fancyhead[R]{\nouppercase{\leftmark\hfill\rightmark}}

    %%%%%%%%%%%%%%%%%%%%%%%%%
    % Equazioni non lineari %
    %%%%%%%%%%%%%%%%%%%%%%%%%
    \section{Equazioni non lineari}

\subsection{Introduzione}

Il \textbf{calcolo degli zeri di una funzione} $f$ reale di variabile reale o delle \textbf{radici dell'equazione} $f\left(x\right)=0$, è un problema assai ricorrente nel Calcolo Scientifico.

\highspace
In generale, \emph{non è possibile} approntare metodi numerici che calcolino gli zeri di una generica funzione in un numero finito di passi. I metodi numerici per la risoluzione di questo problema sono pertanto necessariamente \emph{iterativi}. A partire da uno o più dati iniziali, scelti convenientemente, essi generano una successione di valori $x^{\left(k\right)}$ che, sotto opportune ipotesi, convergerà ad uno zero $\alpha$ della funzione $f$ studiata.

\longline

\subsection{Il metodo di bisezione (o iterativo)}

Sia $f$ una funzione continua in $\left[a,b\right]$ tale che $f\left(a\right)f\left(b\right) < 0$. Per cui, vale il \href{https://www.youmath.it/lezioni/analisi-matematica/limiti-continuita-e-asintoti/723-teorema-degli-zeri.html}{teorema degli zeri di una funzione continua}, ossia $f$ ammette almeno uno zero in $\left(a,b\right)$.

\highspace
Si supponga che ci sia un solo zero, indicato con $\alpha$ e nel caso in cui ce ne sia più di uno, individuare un intervallo tale che ne contenga solo uno.

\highspace
Il \definition{metodo di bisezione} (o \textbf{iterativo}\index{metodo iterativo}) è una strategia che si suddivide nei seguenti passaggi:
\begin{enumerate}
    \item \textbf{Dimezzare l'intervallo di partenza};
    \item \textbf{Selezionare tra i due sotto-intervalli ottenuti quello nel quale} $f$ \textbf{cambia di segno agli estremi};
    \item \textbf{Applicare ricorsivamente questa procedura all'ultimo intervallo selezionato}.
\end{enumerate}

\highspace
Matematicamente parlando, dato $I^{(0)} = \left(a,b\right)$, e più in generale, $I^{(k)}$ il sotto-intervallo selezionato al passo $k$-esimo, si sceglie come $I^{\left(k+1\right)}$ il semi-intervallo di $I^{(k)}$ ai cui estremi $f$ cambia di segno.

\highspace
Questa procedura garantisce che ogni sotto-intervallo selezionato $I^{(k)}$ conterrà $\alpha$. Questo poiché la successione $\left\{x^{(k)}\right\}$ dei punti medi dei sotto-intervalli $I^{(k)}$ dovrà ineluttabilmente convergere a $\alpha$, in quanto la \textbf{lunghezza dei sotto-intervalli tende a} $0$ per $k$ che \textbf{tende all'infinito}.

\newpage

\noindent
Formalizziamo questa idea con un piccolo algoritmo. Ponendo:
\begin{equation*}
    a^{(0)} = a, \hspace{1em} b^{(0)} = b, \hspace{1em} I^{(0)} = \left(a^{(0)}, b^{(0)}\right), \hspace{1em} x^{(0)} = \dfrac{a^{(0)} + b^{(0)}}{2}
\end{equation*}
Al passo $k \ge 1$ il metodo di bisezione calcolerà il semi-intervallo $I^{(k)} = \left(a^{(k)}, b^{(k)}\right)$ dell'intervallo $I^{(k-1)} = \left(a^{\left(k-1\right)}, b^{\left(k-1\right)}\right)$, nel seguente modo (si ricorda che $\alpha$ è lo zero che si sta cercando):
\begin{enumerate}
    \item Calcolo $x^{\left(k-1\right)} = \dfrac{a^{\left(k-1\right)} + b^{\left(k-1\right)}}{2}$

    \item Se $f\left(x^{\left(k-1\right)}\right) = 0$:
    \begin{enumerate}
        \item Allora $\alpha = x^{\left(k-1\right)}$ e l'algoritmo \underline{termina}.
    \end{enumerate}
    
    \item Altrimenti, se $f\left(a^{\left(k-1\right)}\right) \cdot f\left(x^{\left(k-1\right)}\right) < 0$:
    \begin{enumerate}
        \item Si pone $a^{(k)} = a^{\left(k-1\right)}$
        \item Si pone $b^{(k)} = x^{\left(k-1\right)}$
        \item Si incrementa $k+1$ e si \underline{ripete ricorsivamente}.
    \end{enumerate}
    
    \item Altrimenti, se $f\left(x^{\left(k-1\right)}\right) \cdot f\left(b^{\left(k-1\right)}\right) < 0$:
    \begin{enumerate}
        \item Si pone $a^{(k)} = x^{\left(k-1\right)}$
        \item Si pone $b^{(k)} = b^{\left(k-1\right)}$
        \item Si incrementa $k+1$ e si \underline{ripete ricorsivamente}.
    \end{enumerate}
\end{enumerate}

\begin{examplebox}
    Data la funzione $f\left(x\right) = x^{2} - 1$, si parta da $a^{(0)} = -0.25$ e $b^{(0)} = 1.25$, e si applichi il metodo di bisezione:
    \begin{enumerate}
        \item Con $a^{(0)} = -0.25$ e $b^{(0)} = 1.25$:
        \begin{enumerate}
            \item Si calcola il punto medio:
            \begin{equation*}
                x^{(0)} = \dfrac{a^{(0)} + b^{(0)}}{2} = \dfrac{-0.25 + 1.25}{2} = 0.5
            \end{equation*}

            \item Si calcola la funzione con il punto medio come parametro:
            \begin{equation*}
                f\left(0.5\right) = 0.5^{2} - 1 = -0.75
            \end{equation*}

            \item Dato che la funzione nel punto medio non è uguale a zero, l'algoritmo deve continuare. Per farlo, bisogna sostituire il punto medio con uno dei due estremi. Per decidere quale dei due sostituire, è necessario capire in quale cambia valore la funzione. Si verifica inizialmente con $a^{(0)}$:
            \begin{equation*}
                \begin{array}{rcl}
                    f\left(a^{(0)}\right)f\left(x^{(0)}\right) < 0 &=& f\left(-0.25\right)f\left(0.5\right) < 0 \\ [.5em]
                    &=& \left(-0.9375 \cdot -0.75\right) < 0 \\ [.5em]
                    &=& 0.703125 \text{ \ding{55}}
                \end{array}
            \end{equation*}
            
            \item Si procede con l'algoritmo, provando adesso la $b^{(0)}$:
            \begin{equation*}
                \begin{array}{rcl}
                    f\left(x^{(0)}\right)f\left(b^{(0)}\right) < 0 &=& f\left(0.5\right)f\left(1.25\right) < 0 \\ [.5em]
                    &=& \left(-0.75 \cdot 0.5625\right) < 0 \\ [.5em]
                    &=& -0.421875 \text{ \ding{51}}
                \end{array}
            \end{equation*}

            \item Si pone $a^{(1)} = x^{(0)} = 0.5$

            \item Si pone $b^{(1)} = b^{(0)} = 1.25$

            \item Si incrementa $k$, $k = k + 1 = 0 + 1 = 1$
        \end{enumerate}

        \item Con $a^{(1)} = 0.5$ e $b^{(1)} = 1.25$:
        \begin{enumerate}
            \item Si calcola il punto medio:
            \begin{equation*}
                x^{(1)} = \dfrac{a^{(1)} + b^{(1)}}{2} = \dfrac{0.5 + 1.25}{2} = 0.875
            \end{equation*}

            \item Si calcola la funzione con il punto medio come parametro:
            \begin{equation*}
                f\left(0.875\right) = 0.875^{2} - 1 = -0.234375
            \end{equation*}

            \item Dato che la funzione nel punto medio non è uguale a zero, l'algoritmo deve continuare:
            \begin{equation*}
                \begin{array}{rcl}
                    f\left(a^{(1)}\right)f\left(x^{(1)}\right) < 0 &=& f\left(0.5\right)f\left(-0.234375\right) < 0 \\ [.5em]
                    &=& \left(-0.75 \cdot -0.945068359375\right) < 0 \\ [.5em]
                    &=& 0.70880126953125 \text{ \ding{55}}
                \end{array}
            \end{equation*}
            
            \item Si procede con l'algoritmo:
            \begin{equation*}
                \begin{array}{rcl}
                    f\left(x^{(1)}\right)f\left(b^{(1)}\right) < 0 &=& f\left(-0.234375\right)f\left(1.25\right) < 0 \\ [.5em]
                    &=& \left(-0.945068359375 \cdot 0.5625\right) < 0 \\ [.5em]
                    &=& -0.5316009521484375 \text{ \ding{51}}
                \end{array}
            \end{equation*}

            \item Si pone $a^{(2)} = x^{(1)} = 0.875$

            \item Si pone $b^{(2)} = b^{(1)} = 1.25$

            \item Si incrementa $k$, $k = k + 1 = 1 + 1 = 2$
        \end{enumerate}
    \end{enumerate}
    Si omettono i restanti calcoli per $k=2, k=3$, ma si lasciano qua di seguito i risultati:
    \begin{itemize}
        \item $I^{(2)} = \left(0.875, 1.25\right)$ e $x^{(2)} = 1.0625$
        \item $I^{(3)} = \left(0.875, 1.0625\right)$ e $x^{(2)} = 0.96875$
    \end{itemize}

    \newpage

    \noindent
    Nella seguente figura si possono vedere le iterazioni effettuate:
    \begin{center}
        \includegraphics[width=.7\textwidth]{img/metodo-di-bisezione-1.pdf}
        \captionof*{figure}{Iterazioni effettuate.\cite{quarteroni2017calcolo}}
    \end{center}
    
    \noindent
    Si noti che ogni intervallo $I^{(k)}$ contiene lo zero $\alpha$. Inoltre, la successione $\left\{x^{(k)}\right\}$ converge necessariamente allo zero $\alpha$ in quanto ad ogni passo l'ampiezza $\left| I^{(k)} \right| = b^{(k)} - a^{(k)}$ dell'intervallo $I^{(k)}$ si dimezza.
\end{examplebox}

\noindent
Il valore $I^{(k)}$ può essere riassunto come:
\begin{equation*}
    \left|I^{(k)}\right| = \left(\dfrac{1}{2}\right)^{k} \cdot \left| I^{(0)} \right|
\end{equation*}
E di conseguenza l'\textbf{errore al passo} $\bm{k}$ può essere calcolato come:
\begin{equation*}
    \left| e^{(k)} \right| = \left| x^{(k)} - \alpha \right| < \dfrac{1}{2} \cdot \left| I^{(k)} \right| = \left(\dfrac{1}{2}\right)^{k+1} \cdot \left(b-a\right)
\end{equation*}
Inoltre, data una certa \textbf{tolleranza} $\varepsilon$, per \textbf{garantire che l'errore al passo $k$ sia minore della tolleranza data} (ovvero, $\left| e^{(k)} \right| < \varepsilon$), basta applicare la seguente formula:
\begin{equation}
    k_{\min} > \log_{2} \left(\dfrac{b-a}{\varepsilon}\right) - 1
\end{equation}
Dove $k_{\min}$ rappresenta il \textbf{numero \underline{minimo} di iterazioni prima di trovare un intero che soddisfi la disuguaglianza}.

\begin{flushleft}
    \textcolor{Red2}{\faIcon{exclamation-triangle} \textbf{Possibile svantaggio}}
\end{flushleft}
Il metodo di bisezione \textbf{non garantisce una riduzione monotona dell'errore}, ma solo il dimezzamento dell'ampiezza dell'intervallo all'interno del quale si cerca lo zero. Infatti, \textbf{non viene tenuto conto del reale andamento di} $f$ e questo può provocare il mancato coinvolgimento di approssimazioni di $\alpha$ accurate.
    \subsection{Il metodo di Newton}

Il \definition{metodo di Newton} sfrutta la funzione $f$ maggiormente rispetto al metodo di bisezione, usando i suoi valori e la sua derivata.

\highspace
Si ricorda che la retta tangente alla curva $\left(x, f\left(x\right)\right)$ nel punto $x^{(k)}$ è:
\begin{equation*}
    y\left(x\right) = f\left(x^{(k)}\right) + f'\left(x^{(k)}\right)\left(x-x^{(k)}\right)
\end{equation*}
\textbf{Cercando} un $x^{\left(k+1\right)}$ tale che la \textbf{retta tangente in quel punto sia uguale a zero} $y\left(x^{\left(k+1\right)}\right) = 0$, allora si trova:
\begin{equation}\label{eq: metodo di Newton}
    x^{\left(k+1\right)} = x^{\left(k\right)} - \dfrac{f\left(x^{\left(k\right)}\right)}{f'\left(x^{\left(k\right)}\right)} \hspace{2em} k \ge 0
\end{equation}
Purché la derivata prima nel punto $x^{(k)}$ sia diversa da zero, cioè $f'\left(x^{k}\right) \ne 0$.

\highspace
Questa equazione consente di calcolare una successione di valori $x^{(k)}$ a partire da un dato iniziale $x^{(0)}$. In altre parole, il \textbf{metodo di Newton calcola lo zero di $f$ sostituendo localmente a $f$ la sua retta tangente}.

\highspace
A differenza del metodo di bisezione, tale \textbf{metodo converge allo zero in un solo passo quando la funzione $f$ è lineare}, ovvero nella forma $f\left(x\right) = a_{1}x + a_{0}$.

\begin{flushleft}
    \textcolor{Red2}{\faIcon{exclamation-triangle} \textbf{Limitazione}}
\end{flushleft}
La \textbf{convergenza} del metodo di Newton \underline{non} è garantita \textbf{per ogni scelta} di $x^{(0)}$, ma \textbf{soltanto} per valori di $x^{(0)}$ \textbf{sufficientemente vicini} ad $\alpha$, ovvero \textbf{appartenenti ad un intorno} $I\left(\alpha\right)$ sufficientemente piccolo di $\alpha$.

\highspace
Alcune osservazioni a seguito anche di questa limitazione:
\begin{itemize}
    \item A seguito di questa limitazione, risulta evidente che se $x^{(0)}$ è stato scelto opportunamente e se lo zero $\alpha$ è semplice ($f'\left(\alpha\right) \ne 0$), allora il metodo converge.
    
    \item Nel caso in cui $f$ è derivabile con continuità pari a due, allora si ottiene la seguente convergenza:
    \begin{equation}
        \displaystyle \lim_{k \rightarrow \infty} \dfrac{x^{\left(k+1\right) - \alpha}}{\left(x^{(k)} - \alpha\right)^{2}} = \dfrac{f''\left(\alpha\right)}{2f'\left(\alpha\right)}
    \end{equation}
    Il significato è: se $f'\left(\alpha\right) \ne 0$ il metodo di Newton converge almeno quadraticamente o con ordine 2.

    In parole povere, \textbf{per $k$ sufficientemente grande, l'errore al passo $\left(k+1\right)$-esimo si comporta come il quadrato dell'errore al passo $k$-esimo, moltiplicato per una costante indipendente da $k$}.

    \item Se lo zero $\alpha$ ha molteplicità $m$ maggiore di $1$, ovverosia:
    \begin{equation*}
        f'\left(\alpha\right) = 0, \dots, f^{\left(m-1\right)}\left(\alpha\right) = 0
    \end{equation*}
    Allora il metodo di Newton è ancora convergente, purché $x^{(0)}$ sia scelto opportunamente e $f'\left(x\right) \ne 0$ $\forall x \in I\left(\alpha\right) \setminus \left\{\alpha\right\}$. Tuttavia in questo caso l'ordine di convergenza è pari a 1. In tal caso, l'ordine 2 può essere ancora recuperato usando la seguente relazione al posto dell'equazione~\ref{eq: metodo di Newton} ufficiale:
    \begin{equation}
        x^{\left(k+1\right)} = x^{(k)} - m \cdot \dfrac{f\left(x^{(k)}\right)}{f'\left(x^{(k)}\right)} \hspace{2em} k \ge 0
    \end{equation}
    Purché $f'\left(x^{(k)}\right) \ne 0$. Naturalmente, questo \textbf{metodo di Newton modificato} richiede una conoscenza a priori di $m$.
\end{itemize}

\longline

\subsubsection{Come arrestare il metodo di Newton}

Data una tolleranza fissa $\varepsilon$, esistono due tecniche applicabili per capire quando è necessario fermarsi ed evitare di continuare ad iterare:
\begin{itemize}
    \item La \definition{differenza fra due iterate consecutive}, il quale si arresta in corrispondenza del più piccolo intero $k_{\min}$ per il quale:
    \begin{equation}
        \left| x^{\left(k_{\min}\right)} - x^{\left(k_{\min} - 1\right)} \right| < \varepsilon
    \end{equation}
    (test sull'incremento).

    \item Un'altra tecnica applicata anche per altri metodi iterativi è il \definition{residuo} al passo $k$, il quale è definito come:
    \begin{equation*}
        r^{(k)} = f\left(x^{(k)}\right)
    \end{equation*}
    Che è nullo quando $x^{(k)}$ è uno zero di $f$. In questo modo, il metodo viene arrestato alla prima iterata $k_{\min}$:
    \begin{equation}
        \left| r^{\left(k_{\min}\right)} \right| = \left| f\left(x^{\left(k_{\min}\right)}\right) \right| < \varepsilon
    \end{equation}
    Da notare che tale tecnica fornisce una \textbf{stima accurata dell'errore} \underline{solo quando} $\left| f'\left(x\right) \right|$ è circa pari a $1$ in un intorno di $I_{\alpha}$ dello zero $\alpha$ cercato.

    \underline{\textbf{Attenzione!}} Se la derivata non è circa pari a $1$ in un intorno dello zero cercato, la tecnica porterà:
    \begin{itemize}
        \item Ad una \textbf{sovrastima} dell'errore se $\left| f'\left(x\right) \right| \gg 1$ per $x \in I_{\alpha}$
        \item Ad una \textbf{sottostima} dell'errore se $\left| f'\left(x\right) \right| \ll 1$ per $x \in I_{\alpha}$
    \end{itemize}
\end{itemize}
    \subsection{Il metodo delle secanti}

Nel caso in cui la funzione $f$ non sia nota, il metodo di Newton non può essere applicato. Per fortuna, arriva in soccorso il \definition{metodo delle secanti}, il quale esegue una valutazione di $f'\left(x^{(k)}\right)$ andando a sostituire quest'ultima con un \textbf{rapporto incrementale calcolato su valori di $f$ già noti}.

\highspace
Più formalmente, assegnati due punti $x^{(0)}$ e $x^{(1)}$, per $k \ge 1$ si calcola:
\begin{equation}
    x^{(k+1)} = x^{(k)} - \left(\dfrac{f\left(x^{(k)}\right) - f\left(x^{(k-1)}\right)}{x^{(k)} - x^{(k-1)}}\right)^{-1} \cdot f\left(x^{(k)}\right)
\end{equation}

\begin{flushleft}
    \textcolor{Green3}{\faIcon{question-circle} \textbf{Quando converge?}}
\end{flushleft}
Il metodo delle secanti converge a seguito di certe condizioni:
\begin{itemize}
    \item \textbf{Converge ad} $\bm{\alpha}$, se:
    \begin{itemize}
        \item $\alpha$ radice semplice\footnote{$f'\left(\alpha\right) \ne 0$};
        \item $I\left(\alpha\right)$ è un opportuno intorno di $\alpha$;
        \item $x^{(0)}$ e $x^{(1)}$ sono sufficientemente vicini ad $\alpha$
        \item $f'\left(x\right) \ne 0$ $\forall x \in I\left(\alpha\right) \setminus \left\{\alpha\right\}$
    \end{itemize}

    \item \textbf{Converge con ordine} \definition{$p$ super-lineare}, se:
    \begin{itemize}
        \item $f \in \mathcal{C}^{2}\left(I\left(\alpha\right)\right)$
        \item $f'\left(\alpha\right) \ne 0$
    \end{itemize}
    Ovvero, esiste una costante $c > 0$ tale che:
    \begin{equation}
        \left| x^{\left(k+1\right)} - \alpha\right| \le c\left| x^{(k)} - \alpha \right|^{p} \hspace{2em} p = \dfrac{1 + \sqrt{5}}{2} \approx 1.618\dots
    \end{equation}

    \item \textbf{Convergenza lineare}, se:
    \begin{itemize}
        \item Radice $\alpha$ è multipla.
    \end{itemize}
    Come succederebbe usando il metodo di Newton.
\end{itemize}

% TODO: pag. 59
    \subsection{I sistemi di equazioni non lineari}

Di solito i metodi presentati nelle pagine precedenti vengono inseriti in dei sistemi. Nella realtà ci sono varie condizioni che influiscono sul sistema in analisi. È per questo motivo che si introducono i sistemi.

\highspace
Si consideri un generale sistema di equazioni non lineari:
\begin{equation*}
    \begin{cases}
        f_{1}\left(x_{1}, x_{2}, \dots, x_{n}\right) = 0 \\
        f_{2}\left(x_{1}, x_{2}, \dots, x_{n}\right) = 0 \\
        \vdots \\
        f_{n}\left(x_{1}, x_{2}, \dots, x_{n}\right) = 0 \\
    \end{cases}
\end{equation*}
Dove $f_{1}, \dots, f_{n}$ sono funzioni non lineari. Si pongono i seguenti vettori:
\begin{itemize}
    \item $\mathbf{f} \equiv \left(f_{1}, \dots, f_{n}\right)^{T}$
    \item $\mathbf{x} \equiv \left(x_{1}, \dots, x_{n}\right)^{T}$
\end{itemize}
Con l'obbiettivo di riscrivere il sistema in maniera più agevole:
\begin{equation*}
    \mathbf{f}\left(\mathbf{x}\right) = \mathbf{0}
\end{equation*}
\begin{examplebox}[: esempio di sistema non lineare]
    Un esempio banale di sistema non lineare:
    \begin{equation*}
        \begin{cases}
            f_{1}\left(x_{1}, x_{2}\right) = x_{1}^{2} + x_{2}^{2} - 1 = 0 \\
            \\
            f_{2}\left(x_{1}, x_{2}\right) = \sin\left(\pi \dfrac{x_{1}}{2}\right) + x_{2}^{3} = 0 \\
        \end{cases}
    \end{equation*}
\end{examplebox}

\noindent
Prima di estendere i metodi di Newton e delle secanti si introduce la matrice Jacobiana. 

\begin{definitionbox}[: matrice Jacobiana]
    Senza entrare troppo nel gergo matematico (non è l'obbiettivo del corso), la \definition{matrice Jacobiana} di una funzione è quella \textbf{matrice i cui elementi sono le derivate parziali prime della funzione}.
    \begin{equation}\label{eq: matrice Jacobiana}
        \mathbf{J}_{\mathbf{f}} = 
        \begin{rowequmat}{c c c}
            \dfrac{\partial f_{1}}{\partial x_{1}} & \cdots & \dfrac{\partial f_{1}}{\partial x_{n}} \\ [.3em]
            \vdots & \ddots & \vdots \\ [.3em]
            \dfrac{\partial f_{m}}{\partial x_{1}} & \cdots & \dfrac{\partial f_{m}}{\partial x_{n}}
        \end{rowequmat}
    \end{equation}
    Che può essere riscritto in modo più leggibile come:
    \begin{equation}\label{eq: matrice Jacobiana scritta come funzione vettoriale}
        \left(\mathbf{J}_{\mathbf{f}}\right)_{ij} \equiv \dfrac{\partial f_{i}}{\partial x_{j}} \hspace{2em} i,j = 1, \dots, n
    \end{equation}
    Dove rappresenta la derivata parziale della funzione $f_{i}$ rispetto a $x_{j}$.
\end{definitionbox}

\noindent
Il metodo di Newton e delle secanti può essere esteso sfruttando la matrice Jacobiana:
\begin{itemize}
    \item Il \textbf{metodo di Newton} usando un sistema di equazioni non lineari diventa: dato $\mathbf{x}^{\left(0\right)} \in \mathbb{R}^{n}$, per $k = 0, 1, \dots$, fino a convergenza
    \begin{equation}\label{eq: metodo di Newton in un sistema non lineare}
        \begin{array}{l c l}
            \text{risolvere}&& \mathbf{J}_{\mathbf{f}}\left(\mathbf{x}^{\left(k\right)}\right)\boldsymbol{\delta} \mathbf{x}^{\left(k\right)} = -\mathbf{f}\left(\mathbf{x}^{\left(k\right)}\right) \\ [1em]
            %
            \text{porre}    && \mathbf{x}^{\left(k+1\right)} = \mathbf{x}^{\left(k\right)} + \boldsymbol{\delta}\mathbf{x}^{\left(k\right)}
        \end{array}
    \end{equation}
    Se ne deduce che venga richiesto ad ogni passo la soluzione di un sistema lineare di matrice $\mathbf{J}_{\mathbf{f}}\left(\mathbf{x}^{\left(k\right)}\right)$.

    \item Il \textbf{metodo delle secanti} usando un sistema di equazioni non lineari si basa sulla matrice Jacobiana e sul metodo di Broyden.

    L'\textbf{idea di base} è sostituire le matrici Jacobiane $\mathbf{J}_{\mathbf{f}}\left(\mathbf{x}^{\left(k\right)}\right)$ (per $k \ge 0$) con delle matrici chiamate $\mathrm{B}_{k}$, definite ricorsivamente a partire da una matrice $\mathrm{B}_{0}$ che sia una approssimazione di $\mathbf{J}_{\mathbf{f}}\left(\mathbf{x}^{\left(0\right)}\right)$.

    Dato $\mathbf{x}^{\left(0\right)} \in \mathbb{R}^{n}$, data $\mathrm{B}_{0} \in \mathbb{R}^{n \times n}$ per $k = 0, 1, \dots$, fino a convergenza
    \begin{equation}\label{eq: metodo delle secanti in un sistema non lineare}
        \begin{array}{l c l}
            \text{risolvere}&& \mathrm{B}_{k}\boldsymbol{\delta}\mathbf{x}^{\left(k\right)} = -\mathbf{f}\left(\mathbf{x}^{\left(k\right)}\right) \\ [1em]
            %
            \text{porre}    && \mathbf{x}^{\left(k+1\right)} = \mathbf{x}^{\left(k\right)} + \boldsymbol{\delta}\mathbf{x}^{\left(k\right)} \\ [1em]
            %
            \text{porre}    && \boldsymbol{\delta}\mathbf{f}^{\left(k\right)} = \mathbf{f}\left(\mathbf{x}^{\left(k+1\right)}\right) - \mathbf{f}\left(\mathbf{x}^{\left(k\right)}\right) \\ [1em]
            %
            \text{calcolare}&& \mathrm{B}_{k+1} = \mathrm{B}_{k} + \dfrac{
                \left(\boldsymbol{\delta}\mathbf{f}^{\left(k\right)} - \mathrm{B}_{k}\boldsymbol{\delta}\mathbf{x}^{\left(k\right)}\right)\boldsymbol{\delta}\mathbf{x}^{\left(k\right)^{T}}
            }{
                \boldsymbol{\delta}\mathbf{x}^{\left(k\right)^{T}} \boldsymbol{\delta}\mathbf{x}^{\left(k\right)}
            }
        \end{array}
    \end{equation}
    Da notare che non si chiede alla successione $\left\{\mathrm{B}_{k}\right\}$ così costruita di convergere alla vera matrice Jacobiana $\mathbf{J}_{\mathbf{f}}\left(\boldsymbol{\alpha}\right)$ ($\boldsymbol{\alpha}$ è la radice del sistema); questo risultato non è garantito tuttavia.
\end{itemize}
    \subsection{Iterazioni di punto fisso}

\begin{examplebox}[: esempio di introduzione]
    Con una calcolatrice si può facilmente verificare che applicando ripetutamente la funzione $\cos$ partendo dal numero $1$ si genera la seguente successione di numeri reali:
    \begin{equation*}
        \begin{array}{lclcl}
            x^{\left(1\right)}  &=& \cos\left(1\right)                  &=& 0.54030230586814 \\ [1em]
            x^{\left(2\right)}  &=& \cos\left(x^{\left(1\right)}\right) &=& 0.85755321584639 \\
            \vdots              & &                                     & & \\
            x^{\left(10\right)} &=& \cos\left(x^{\left(9\right)}\right) &=& 0.74423735490056 \\
            \vdots              & &                                     & & \\
            x^{\left(20\right)} &=& \cos\left(x^{\left(19\right)}\right)&=& 0.73918439977149
        \end{array}
    \end{equation*}
    Che tende al valore $\alpha = 0.73908513$.
\end{examplebox}

\noindent
Con l'esempio di introduzione è possibile capire il punto fisso. Essendo per costruzione $x^{\left(k+1\right)} = \cos\left(x^{\left(k\right)}\right)$ per $k = 0, 1, \dots$ (con $x^{\left(0\right)} = 1$), $\alpha$ è tale che $\cos\left(\alpha\right) = \alpha$. Quindi, $\alpha$ viene detto punto fisso della funzione coseno.

\highspace
\begin{flushleft}
    \textcolor{Green3}{\faIcon{question-circle} \textbf{Perché è interessante?}}
\end{flushleft}
Se $\alpha$ è un punto fisso per il coseno, allora esso è uno zero della funzione $f\left(x\right) = x - \cos\left(x\right)$ ed il metodo appena proposto potrebbe essere usato per il calcolo degli zeri di $f$.

\highspace
\begin{flushleft}
    \textcolor{Red2}{\faIcon{exclamation-triangle} \textbf{Non tutte le funzioni hanno un punto fisso}}
\end{flushleft}
Non tutte le funzioni ammettono punti fissi. Ad esempio, ripetendo l'esperimento dell'esempio con una funzione esponenziale a partire da $x^{\left(0\right)} = 1$, dopo soli 4 passi si giunge ad una situazione di \emph{overflow} (figura \ref{fig: non tutte le funzioni hanno un punto fisso}, pagina \pageref{fig: non tutte le funzioni hanno un punto fisso}).

\begin{definitionbox}
    Data una funzione $\phi: \left[a, b\right] \rightarrow \mathbb{R}$, trovare $\alpha \in \left[a,b\right]$ tale che:
    \begin{equation*}
        \alpha = \phi\left(\alpha\right)
    \end{equation*}
    Se tale $\alpha$ esiste, viene detto un \definition{punto fisso} di $\phi$ e lo si può determinare come limite della seguente successione:
    \begin{equation}\label{eq: successione punto fisso}
        x^{\left(k+1\right)} = \phi\left(x^{\left(k\right)}\right) \hspace{2em} k \ge 0
    \end{equation}
    Dove $x^{\left(0\right)}$ è un dato iniziale. L'algoritmo viene chiamato \definition{iterazioni di punto fisso} e la funzione $\phi$ è detta \definition{funzione di iterazione}.
\end{definitionbox}

\noindent
Dalla definizione, si deduce che l'esempio introduttivo è un algoritmo di iterazioni di punto fisso per la funzione $\phi\left(x\right) = \cos\left(x\right)$.

\newpage

\begin{figure}[!htp]
    \centering
    \includegraphics[width=\textwidth]{img/iterazioni-di-punto-fisso-1.pdf}
    \caption{La funzione $\phi\left(x\right) = \cos\left(x\right)$ (sx) ammette un solo punto fisso, mentre la funzione $\phi\left(x\right) = e^{x}$ (dx) non ne ammette alcuno.}
    \label{fig: non tutte le funzioni hanno un punto fisso}
\end{figure}

\hfill

\begin{figure}[!htp]
    \centering
    \includegraphics[width=\textwidth]{img/iterazioni-di-punto-fisso-2.pdf}
    \caption{Rappresentazione delle prime iterazioni di punto fisso per due funzioni di iterazione. Le iterazioni convergono verso il punto fisso $\alpha$ (sx), mentre si allontanano da $\alpha$ (dx).}
    \label{fig: interpretazione geometrica di un punto fisso}
\end{figure}

\newpage

\begin{definitionbox}[: quando una funzione ha un punto fisso?]
    Si consideri la successione (formula) \ref{eq: successione punto fisso} a pagina \pageref{eq: successione punto fisso}.
    \begin{enumerate}
        \item Si supponga che $\phi\left(x\right)$ sia continua nell'intervallo $\left[a,b\right]$ e che $\phi\left(x\right) \in \left[a,b\right]$ per ogni $x \in \left[a,b\right]$; allora \textbf{esiste almeno un punto fisso} $\alpha \in \left[a,b\right]$.

        \item Si supponga inoltre che esista un valore $L$ minore di $1$ tale per cui:
        \begin{equation*}
            \left|\phi\left(x_{1}\right)-\phi\left(x_{2}\right)\right| \le L \left|x_{1}-x_{2}\right|
        \end{equation*}
        Per ogni $x_{1},x_{2}$ appartenente all'insieme $\left[a,b\right]$. Con tale supposizione, allora $\phi$ ha un \textbf{unico punto fisso} $\phi\in\left[a,b\right]$ e la successione definita nell'equazione \ref{eq: successione punto fisso} a pagina \pageref{eq: successione punto fisso} converge a $\alpha$, qualunque sia il dato iniziale $x^{\left(0\right)}$ in $\left[a,b\right]$.

        La supposizione scritta in precedenza può essere riassunta in un'equazione:
        \begin{equation}
            \exists L < 1 \:\: t.c. \:\: \left| \phi\left(x_{1}\right)-\phi\left(x_{2}\right)\right| \le L \left|x_{1}-x_{2}\right| \hspace{1em} \forall x_{1},x_{2} \in \left[a,b\right]
        \end{equation}
    \end{enumerate}
\end{definitionbox}

\noindent
Nella pratica è però spesso \textbf{difficile delimitare a priori l'ampiezza dell'intervallo} $\left[a,b\right]$; in tal caso è utile il seguente risultato di convergenza locale:
\begin{theorem}[di Ostrowski]\index{teorema di Ostrowski}\label{theorem: teorema di Ostrowski}
    Sia $\alpha$ un punto fisso di una funzione $\phi$ continua e derivabile con continuità in un opportuno intorno $\mathcal{J}$ di $\alpha$. Se risulta $\left|\phi'\left(\alpha\right)\right| < 1$, allora esiste un $\delta > 0$ in corrispondenza del quale la successione $\left\{x^{\left(k\right)}\right\}$ converge ad $\alpha$, per ogni $x^{\left(0\right)}$ tale che $\left|x^{\left(0\right)} - \alpha \right| < \delta$. Inoltre si ha:
    \begin{equation}
        \displaystyle\lim_{k \rightarrow \infty} \dfrac{
            x^{\left(k+1\right)} - \alpha
        }{
            x^{\left(k\right)}-\alpha
        } = \phi'\left(\alpha\right)
    \end{equation}
\end{theorem}

\highspace
Dal teorema si deduce che le iterazioni di punto fisso convergono almeno linearmente cioè che, per $k$ sufficientemente grande, l'errore del passo $k+1$ si comporta come l'errore al passo $k$ moltiplicato per una costante, $\phi'\left(\alpha\right)$ nel teorema, indipendente da $k$ ed il cui valore assoluto è minore di $1$. Per questo motivo la costante viene chiamata \definition{fattore di convergenza} e la convergenza sarà tanto più rapida quanto più piccola è tale costante.

\begin{definitionbox}[: quando il metodo di punto fisso è convergente]
    Si suppongano valide le ipotesi del teorema di Ostrowski \ref{theorem: teorema di Ostrowski}. Se, inoltre, $\phi$ è derivabile con continuità due volte e se:
    \begin{equation*}
        \phi'\left(\alpha\right) = 0 \hspace{3em} \phi''\left(\alpha\right) \ne 0
    \end{equation*}
    Allora il metodo di punto fisso (eq. \ref{eq: successione punto fisso}) è convergente di ordine 2 e si ha:
    \begin{equation}
        \displaystyle\lim_{k \rightarrow \infty} \dfrac{
            x^{\left(k+1\right)} - \alpha
        }{
            \left(x^{\left(k\right)} - \alpha\right)^{2}
        } = \dfrac{1}{2}\phi''\left(\alpha\right)
    \end{equation}
\end{definitionbox}

\newpage

\noindent
Un'ultima osservazione interessante:
\begin{itemize}
    \item Nel caso in cui $\left| \phi'\left(\alpha\right) \right| > 1$, se $x^{\left(k\right)}$ è sufficientemente vicino ad $\alpha$, in modo tale che $\left| \phi'\left(x^{\left(k\right)}\right)\right| > 1$, allora $\left| \alpha - x^{\left(k+1\right)} \right| > \left| \alpha - x^{\left(k\right)} \right|$, e \textbf{non è possibile che la successione converga al punto fisso}.

    \item Nel caso in cui $\left| \phi'\left(\alpha\right) \right| = 1$ \textbf{non si può trarre alcuna conclusione} perché potrebbero verificarsi sia la convergenza sia la divergenza, a seconda delle caratteristiche della funzione di punto fisso.
\end{itemize}

    %%%%%%%%%%%%%%%%%%%%%%%%%%%%%%%%%%%%%%%%%%%%%%%%%%%%%%%
    % Metodi risolutivi per sistemi lineari e non lineari %
    %%%%%%%%%%%%%%%%%%%%%%%%%%%%%%%%%%%%%%%%%%%%%%%%%%%%%%%
    \section{Metodi risolutivi per sistemi lineari e non lineari}

\subsection{Metodi diretti per sistemi lineari}

\subsubsection{Metodo delle sostituzioni in avanti e all'indietro}

\begin{flushleft}
    \textcolor{Green3}{\faIcon{question-circle} \textbf{Perché sono importanti i metodi numerici?}}
\end{flushleft}
Si consideri il seguente \textbf{sistema lineare}:
\begin{equation*}
    Ax = b
\end{equation*}
Dove:
\begin{itemize}
    \item $A \in \mathbb{R}^{n \times n}$ di componenti $a_{ij}$ e $b \in \mathbb{R}^{n}$ sono valori noti.

    \item $x \in \mathbb{R}^{n}$ è il vettore delle incognite.
    
    \item La costante $n$ rappresenta il numero di equazioni lineari delle incognite $x_{j}$.
\end{itemize}
Con queste caratteristiche, è possibile rappresentare la $i$-esima equazione nel seguente modo:
\begin{equation*}
    \displaystyle\sum_{j=1}^{n} a_{ij} x_{j} = b_{i} \:\: \rightarrow \:\: a_{1i}x_{1} + a_{i2} x_{2} + \cdots + a_{in}x_{n} = b_{i} \hspace{2em} \forall i = 1, \dots , n
\end{equation*}
La \textbf{soluzione esatta} del sistema, chiamata \definition{formula di Cramer}, è:
\begin{equation}\label{eq: formula di Cramer}
    x_{j} = \dfrac{\det\left(A_{j}\right)}{\det\left(A\right)}
\end{equation}
Con $A_{j} = \left|a_{1} \:\: \dots \:\: a_{j-1} \:\: b \:\: a_{j+1} \:\: \dots \:\: a_{n} \right|$ e $a_{i}$ le colonne di $A$. Ovviamente la soluzione \textbf{esiste ed è unica se il determinante} della matrice $A$ è \textbf{diverso da zero}:
\begin{equation*}
    \det\left(A\right) \ne 0
\end{equation*}
Purtroppo questo metodo è \textbf{inutilizzabile} poiché il calcolo di un determinante richiede all'incirca $n!$ (fattoriale di $n$) operazioni.

\highspace
Risulta evidente che sia necessario uno studio approfondito di \textbf{metodi numerici che si traducano in algoritmi efficienti} da farli eseguire su calcolatori. Nelle seguenti pagine si introducono i primi due algoritmi \dquotes{efficienti}.

\newpage

\begin{definitionbox}
    Il seguente algoritmo rappresenta il \definition{metodo delle sostituzioni in avanti}. Dati:
    \begin{itemize}
        \item $L \in \mathbb{R}^{n \times n}$ matrice triangolare inferiore non singolare (cioè con determinante diverso da zero $\det\left(L\right) \ne 0$)
        
        \item $\mathbf{b} \in \mathbb{R}^{n}$ vettore termine noto
    \end{itemize}
    La soluzione è data da $Lx = \mathbf{b}$ con $x \in \mathbb{R}^{n}$. Più in generale si ha:
    \begin{equation}
        x_{i} = \dfrac{
            b_{i} - \displaystyle\sum_{j=1}^{i=1} L_{ij} \: x_{j}
        }{
            L_{ii}
        }
    \end{equation}
\end{definitionbox}

\noindent
Per comprendere meglio la definizione del metodo di sostituzioni in avanti, è possibile visualizzare in modo generale la matrice triangolare inferiore $L$ (non singolare):
\begin{equation*}
    L_{n \times n} = \begin{bmatrix}
        L_{11} & 0 & 0 & 0 & 0 \\
         & \ddots & 0 & 0 & 0 \\
         & & \ddots & 0 & 0 \\
         & L_{ij} & & \ddots & 0 \\
         & & & & L_{nn}
    \end{bmatrix}
\end{equation*}
Dove ovviamente $n$ è la grandezza della matrice quadrata. Dalla matrice, è possibile rappresentare le prime tre iterazioni, ovvero $x_{1}$, $x_{2}$ e $x_{3}$:
\begin{itemize}
    \item La prima riga è:
    \begin{equation*}
        x_{1} = \dfrac{
            b_{1}
        }{
            L_{11}
        }
    \end{equation*}
    Si può scrivere anche in linea nel seguente modo: $L_{11}x_{1} = b_{1}$.

    \item La seconda riga:
    \begin{equation*}
        x_{2} = \dfrac{
            b_{2} - (L_{21} \cdot x_{1} + L_{22}  \cdot \cancelto{0}{x_{2}})
        }{
            L_{22}
        }
    \end{equation*}
    In cui $x_{1}$ è il risultato del punto precedente e $x_{2}$ è il risultato che attualmente si sta calcolando, quindi uguale a zero.

    \item La terza riga:
    \begin{equation*}
        x_{3} = \dfrac{
            b_{3} - (L_{31} \cdot x_{1} + L_{32} \cdot x_{2} + L_{33} \cdot \cancelto{0}{x_{3}})
        }{
            L_{33}
        }
    \end{equation*}
\end{itemize}

\noindent
Il \textbf{numero di operazioni} richieste dal metodo delle sostituzioni in avanti è dato da 1 sottrazione, $i-1$ moltiplicazioni, $i-2$ addizioni e 1 divisione:
\begin{equation}
    \# op. = \displaystyle\sum_{i=1}^{n} \left(i-1\right) + \left(i-2\right) + 1 + 1 = \sum_{i=1}^{n} \left(2i-1\right) = n^{2}
\end{equation}
Per completezza si presenta anche il metodo delle sostituzioni all'indietro.

\begin{definitionbox}
    Il seguente algoritmo rappresenta il \definition{metodo delle sostituzioni all'indietro}. Dati:
    \begin{itemize}
        \item $U \in \mathbb{R}^{n \times n}$ matrice triangolare superiore non singolare (cioè con determinante diverso da zero $\det\left(U\right) \ne 0$)
        
        \item $\mathbf{b} \in \mathbb{R}^{n}$ vettore termine noto
    \end{itemize}
    La soluzione è data da $Ux = \mathbf{b}$ con $x \in \mathbb{R}^{n}$. Più in generale si ha:
    \begin{equation}
        x_{i} = \dfrac{
            b_{i} - \displaystyle\sum_{j=i+1}^{n} U_{ij} \: x_{j}
        }{
            U_{ii}
        }
    \end{equation}
\end{definitionbox}

\noindent
Come per il metodo precedente, anche in questo caso è utile visualizzare la matrice generale:
\begin{equation*}
    U_{n \times n} = \begin{bmatrix}
        U_{11} &   &   &   &   \\
        0 & \ddots &   & U_{ij} & \\
        0 & 0 & \ddots &   &   \\
        0 & 0 & 0 & \ddots &   \\
        0 & 0 & 0 & 0 & U_{nn}
    \end{bmatrix}
\end{equation*}
Anche da questa matrice è possibile rappresentare le iterazioni:
\begin{itemize}
    \item La prima riga è:
    \begin{equation*}
        x_{1} = \dfrac{
            b_{1} - (U_{12} \cdot x_{2} + U_{13} \cdot x_{3} + U_{14} \cdot x_{4})
        }{
            U_{11}
        }
    \end{equation*}

    \item La seconda riga è:
    \begin{equation*}
        x_{2} = \dfrac{
            b_{2} - (U_{23} \cdot x_{3} + U_{24} \cdot x_{4})
        }{
            U_{22}
        }
    \end{equation*}

    \item La terza riga è:
    \begin{equation*}
        x_{3} = \dfrac{
            b_{3} - (U_{34} \cdot x_{4})
        }{
            U_{33}
        }
    \end{equation*}

    \item L'ultima riga è:
    \begin{equation*}
        x_{4} = \dfrac{
            b_{4}
        }{
            U_{44}
        }
    \end{equation*}
\end{itemize}
Si deduce ovviamente che l'ultima riga può essere generalizzata nel seguente modo:
\begin{equation*}
    x_{n} = \dfrac{b_{n}}{U_{nn}}
\end{equation*}

\noindent
Infine, il \textbf{numero di operazioni} è il medesimo del metodo delle sostituzioni in avanti, quindi $n^{2}$.
    \subsection{Metodi iterativi per sistemi lineari}

Un \definition{metodo iterativo} per la risoluzione del sistema lineare $Ax = b$ con:
\begin{itemize}
    \item $A \in \mathbb{R}^{n \times n}$
    \item $b \in \mathbb{R}^{n}$
    \item $x \in \mathbb{R}^{n}$
    \item $\det\left(A\right) \ne 0$
\end{itemize}
Consiste nel costruire una successione di vettori del tipo:
\begin{equation*}
    \left\{\mathbf{x}^{\left(k\right)}, \: k \ge 0\right\}
\end{equation*}
Di $\mathbb{R}^{n}$ che \textbf{converge} alla soluzione esatta $\mathbf{x}$, ossia tale che:
\begin{equation}
    \displaystyle\lim_{k \rightarrow \infty} \mathbf{x}^{\left(k\right)} = \mathbf{x}
\end{equation}
Per un qualunque vettore iniziale $\mathbf{x}^{\left(0\right)} \in \mathbb{R}$, ossia la \textbf{convergenza non deve dipendere dalla scelta di} $\mathbf{x}^{\left(0\right)}$.

\highspace
Inoltre, si definisce ricorsivamente:
\begin{equation}\label{eq: metodi iterativi x con k + 1}
    \mathbf{x}^{\left(k+1\right)} = \mathrm{B}\mathbf{x}^{\left(k\right)} + \mathbf{g} \hspace{2em} k \ge 0
\end{equation}
Essendo:
\begin{itemize}
    \item $\mathrm{B}$ una matrice opportuna (dipendente da $\mathrm{A}$)
    \item $\mathbf{g}$ un vettore opportuno (dipendente da $\mathrm{A}$ e da $\mathbf{b}$)
\end{itemize}

\begin{flushleft}
    \textcolor{Green3}{\faIcon{question-circle} \textbf{E come si possono scegliere questi valori?}}
\end{flushleft}
La scelta della matrice $\mathrm{B}$ e del vettore $\mathbf{g}$ è piuttosto semplice. Devono essere \textbf{rispettate due proprietà}:
\begin{itemize}
    \item \textbf{Condizione di consistenza}. I valori devono garantire che:
    \begin{equation}\label{eq: condizione di consistenza}
        \mathbf{x} = \mathrm{B}\mathbf{x} + \mathbf{g}
    \end{equation}
    Essendo $\mathbf{x} = \mathrm{A}^{-1}\mathbf{b}$, necessariamente dovrà aversi $\mathbf{g} = \left(\mathrm{I} - \mathrm{B}\right) \mathrm{A}^{-1} \mathbf{b}$. 

    \item \textbf{Condizione di convergenza}. Prima di dare la condizione, è necessario comprendere alcune cose. Prima di tutto, la seguente espressione:
    \begin{equation*}
        \mathbf{e}^{\left(k\right)} = \mathbf{x} - \mathbf{x}^{\left(k\right)}
    \end{equation*}
    Viene identificata come l'\textbf{errore al passo} $k$. Adesso sottraendo l'equazione \ref{eq: metodi iterativi x con k + 1} dalla condizione di consistenza \ref{eq: condizione di consistenza} si ottiene:
    \begin{equation*}
        \mathbf{e}^{\left(k+1\right)} = \mathrm{B}\mathbf{e}^{\left(k\right)}
    \end{equation*}
    Per tale ragione $\mathrm{B}$ viene detta \definition{matrice di iterazione} del metodo \ref{eq: metodi iterativi x con k + 1}.

    E adesso, si presenta la condizione di convergenza:
    \begin{itemize}
        \item La matrice $\mathrm{B}$ è simmetrica\footnote{Quindi corrisponde con la sua trasposta} e definita positiva\footnote{Definizione a pagina: \pageref{matrice definita positiva}}, allora:
        \begin{equation}
            \left|\left| \mathbf{e}^{\left(k+1\right)} \right|\right| 
            = 
            \left|\left| \mathrm{B}\mathbf{e}^{\left(k\right)} \right|\right|
            \le
            \rho\left(\mathrm{B}\right)
            \left|\left| \mathbf{e}^{\left(k\right)} \right|\right|
            \hspace{2em}
            \forall k \ge 0
        \end{equation}
        Dove $\rho\left(\mathrm{B}\right)$ è il \definition{raggio spettrale} di $\mathrm{B}$ ed è il massimo degli autovalori di $\mathrm{B}$. Si tenga conto che nel caso di matrici simmetriche e definite positive esso coincide con il massimo autovalore.

        \item Iterando a ritroso il punto precedente, si ottiene:
        \begin{equation}
            \left|\left| \mathbf{e}^{\left(k\right)} \right|\right| 
            \le
            \left[\rho\left(\mathrm{B}\right)\right]^{k}
            \left|\left| \mathbf{e}^{\left(0\right)} \right|\right| 
            \hspace{2em}
            k \ge 0
        \end{equation}
        Si noti che se $\rho\left(\mathrm{B}\right) < 1$, allora $\mathbf{e}^{\left(k\right)} \rightarrow \mathbf{0}$ per $k \rightarrow \infty$ per ogni possibile $\mathbf{e}^{0}$ (e, conseguentemente, per ogni $\mathbf{x}^{\left(0\right)}$).
    \end{itemize}
\end{itemize}
In generale, la condizione sufficiente e necessaria di convergenza è la seguente.

\begin{definitionbox}[: condizione sufficiente e necessaria di convergenza]
    Un metodo iterativo della forma dell'equazione \ref{eq: metodi iterativi x con k + 1}, la cui matrice di iterazione $B$ soddisfi la condizione di consistenza (eq. \ref{eq: condizione di consistenza}), è \textbf{convergente} per ogni $\mathbf{x}^{\left(0\right)}$ \emph{se e soltanto se} $\rho\left(\mathrm{B}\right) < 1$ (raggio spettrale minore di 1).

    \highspace
    Inoltre, minore è $\rho\left(\mathrm{B}\right)$, minore è il numero di iterazioni necessarie per ridurre l'errore iniziale di un dato fattore.
\end{definitionbox}

\highspace
\begin{flushleft}
    \textcolor{Green3}{\faIcon{question-circle} \textbf{Perché introdurre i metodi iterativi?}}
\end{flushleft}
L'esigenza di introdurre i metodi iterativi sorge nel momento in cui si ragiona sulla quantità di tempo spesa da un calcolatore per eseguire la fattorizzazione LU su matrici di grandi dimensioni. Difatti, con matrici con ordini di $10^{7}$, sono necessari circa 11 giorni.

\newpage

\subsubsection{Il metodo di Jacobi}

Se i coefficienti diagonali della matrice $A$ non sono nulli, allora:
\begin{equation*}
    D = \mathrm{diag}\left(a_{11}, a_{22}, \dots, a_{nn}\right)
\end{equation*}
Ovvero $D$ è la matrice diagonale costruita a partire dagli elementi diagonali di $A$. 

\highspace
Il \definition{metodo di Jacobi} corrisponde alla forma:
\begin{equation*}
    \mathrm{D}\mathbf{x}^{\left(k+1\right)} = \mathbf{b} - \left(\mathrm{A} - \mathrm{D}\right)\mathbf{x}^{\left(k\right)} \hspace{2em} k \ge 0
\end{equation*}
Che per componenti assume la forma:
\begin{equation}
    x_{i}^{\left(k+1\right)}
    =
    \dfrac{1}{a_{ii}}
    \left(
        b_{i} - \displaystyle\sum_{j=1, j \ne i}^{n} a_{ij} x_{j}^{\left(k\right)}
    \right)
    \hspace{2em}
    i = 1, \dots, n
\end{equation}
Dove $k \ge 0$ e $\mathbf{x}^{\left(0\right)} = \left(x_{1}^{\left(0\right)}, x_{2}^{\left(0\right)}, \dots, x_{n}^{\left(0\right)}\right)^{T}$ è il vettore iniziale.

\begin{flushleft}
    \textcolor{Green3}{\faIcon{question-circle} \textbf{Quando converge?}}
\end{flushleft}
In due casi:
\begin{enumerate}
    \item Se la matrice $\mathrm{A} \in \mathbb{R}^{n \times n}$ è a dominanza diagonale stressa per righe\footnote{\label{dominanza diagonale stressa per righe}
        In algebra lineare una \definition{matrice a dominanza diagonale stretta per righe} o \definition{matrice diagonale dominante stretta per righe} è una matrice quadrata $A \in \mathbb{C}^{n \times n}$ di ordine $n$ i cui elementi diagonali sono maggiori (non stretta sarebbe maggiori o uguali) in valore assoluto della somma di tutti i restanti elementi della stessa riga in valore assoluto:
        \begin{equation}
            \left| a_{ii} \right| > \displaystyle\sum_{j=1, j \ne i}^{n} \left| a_{ij} \right| 
        \end{equation}
        In senso \underline{non stretto}:
        \begin{equation}
            \left| a_{ii} \right| \ge \displaystyle\sum_{j=1, j \ne i}^{n} \left| a_{ij} \right|
        \end{equation}
    }, allora il metodo di Jacobi converge.

    \item Con la seguente definizione (si veda il paragrafo \ref{subsubsection: il metodo di Gauss-Seidel} per comprendere meglio le definizioni):
    \begin{definitionbox}[: convergenza di Jacobi e Gauss-Seidel]\label{definizione: convergenza di Jacobi e Gauss-Seidel}
        Se $A \in \mathbb{R}^{n \times n}$ è una \definition{matrice tridiagonale}, quindi una matrice quadrata che al di fuori della diagonale principale e delle linee immediatamente al di sopra e al di sotto di essa (la prima sovradiagonale e la prima sottodiagonale), ha solo valori nulli. Un esempio di amtrice tridiagonale:
        \begin{equation*}
            \begin{bmatrix}
                1 & 4 & 0 & 0 \\
                3 & 4 & 1 & 0 \\
                0 & 2 & 3 & 4 \\
                0 & 0 & 1 & 3
            \end{bmatrix}
        \end{equation*}
        Se tale matrice è non singolare, quindi $\det\left(A\right) \ne 0$, con valori della diagonale diversi da zero, allora i metodi di Jacobi e di Gauss-Seidel sono \textbf{entrambi convergenti o entrambi divergenti}.
        \begin{itemize}
            \item Se convergono, la velocità dei metodi:
            \begin{equation*}
                \text{Gauss-Seidel} \gg \text{Jacobi}
            \end{equation*}
            Precisamente il raggio spettrale della matrice di iterazione del metodo di Gauss-Seidel è il quadrato del raggio spettrale di quella del metodo di Jacobi.
        \end{itemize}
    \end{definitionbox}
\end{enumerate}

\highspace
HPC curiosity: è interessante notare che il metodo di Jacobi viene facilmente parallelizzato per aumentare le prestazioni di calcolo.

\newpage

\subsubsection{Il metodo di Gauss-Seidel}\label{subsubsection: il metodo di Gauss-Seidel}

Con il metodo di Jacobi, ogni componente $x_{i}^{\left(k+1\right)}$ del nuovo vettore $\mathbf{x}^{\left(k+1\right)}$ viene calcolata indipendentemente dalle altre. Questa può essere la base di partenza per costruire un nuovo metodo più ottimizzato, poiché se per il calcolo di $x_{i}^{\left(k+1\right)}$ venissero usate le nuove componenti già disponibili $x_{j}^{\left(k+1\right)}$, $j = 1, \dots, i-1$, assieme con le vecchie $x_{j}^{\left(k\right)}$, $j \ge i$.

\highspace
Supponendo che gli elementi della diagonale non siano nulli, per $k \ge 0$, il \definition{metodo di Gauss-Seidel}:
\begin{equation}\label{eq: metodo di Gauss-Seidel}
    x_{i}^{\left(k+1\right)} = 
    \dfrac{1}{a_{ii}} \left(
        b_{i} -
        \displaystyle\sum_{j=1}^{i-1} a_{ij}x_{j}^{\left(k+1\right)} -
        \displaystyle\sum_{j=i+1}^{n} a_{ij}x_{j}^{\left(k\right)}
    \right)
    \hspace{2em}
    i = 1, \dots, n
\end{equation}
HPC curiosity: a differenza di Jacobi, questo metodo non può essere parallelizzato, ma è necessario operare in modo sequenziale.

\begin{flushleft}
    \textcolor{Green3}{\faIcon{question-circle} \textbf{Quando converge?}}
\end{flushleft}
In tre casi:
\begin{enumerate}
    \item Se la matrice $\mathrm{A} \in \mathbb{R}^{n \times n}$ è a dominanza diagonale stressa per righe (vedi pag. \pageref{dominanza diagonale stressa per righe}), allora il metodo di Gauss-Seidel converge.

    \item Se la matrice $\mathrm{A}$ è simmetrica (uguale alla sua trasposta) e definita positiva (vedi pag. \pageref{matrice definita positiva}), allora Gauss-Seidel converge.

    \item Con la definizione di convergenza data per il metodo di Jacobi a pagina \pageref{definizione: convergenza di Jacobi e Gauss-Seidel}
\end{enumerate}
Da notare: \textbf{non esistono risultati generali che consentono di affermare che il metodo di Gauss-Seidel converga sempre più rapidamente di quello di Jacobi}, a parte il caso della definizione a pagina \pageref{matrice definita positiva}.

\newpage

\subsubsection{Il metodo di Richardson}

Una tecnica generale per costruire un metodo iterativo è basata sulla seguente \definition{decomposizione additiva} (o \definition{splitting}) della matrice $A$:
\begin{equation*}
    A = P - \left(P - A\right)
\end{equation*}
In cui $P$ è un'opportuna matrice non singolare ($\det\left(P\right) \ne 0$) chiamata \definition{precondizionatore} di $A$. Di conseguenza:
\begin{equation*}
    P\mathbf{x} = \left(P-A\right)\mathbf{x} + \mathbf{b}
\end{equation*}
È un sistema purché si ponga:
\begin{equation*}
    \begin{array}{rcl}
        B &=& P^{-1} \left(P-A\right) = I - P^{-1}A \\
        \mathbf{g} &=& P^{-1}\mathbf{b}
    \end{array}
\end{equation*}
Questa identità suggerisce la definizione del seguente metodo iterativo:
\begin{equation*}
    P\left(\mathbf{x}^{\left(k+1\right)} - \mathbf{x}^{\left(k\right)}\right) = \mathbf{r}^{\left(k\right)} \hspace{2em} k \ge 0
\end{equation*}
In cui:
\begin{equation}
    \mathbf{r}^{\left(k\right)} = \mathbf{b} - A\mathbf{x}^{\left(k\right)}
\end{equation}
Denota il vettore residuo alla $k$-esima iterazione. Una generalizzazione di questo metodo iterativo è
\begin{equation}\label{eq: metodo di Richardson}
    P\left(\mathbf{x}^{\left(k+1\right)} - \mathbf{x}^{\left(k\right)}\right) = \alpha_{k}\mathbf{r}^{\left(k\right)} \hspace{2em} k \ge 0
\end{equation}
Dove $\alpha_{k} \ne 0$ è un parametro che può cambiare ad ogni iterazione e che, a priori, servirà a migliorare le proprietà di convergenza della successione $\left\{\mathbf{x}^{\left(k\right)}\right\}$.

\highspace
L'equazione \ref{eq: metodo di Richardson} è nota come \definition{metodo di Richardson} o \definition{metodo di Richardson dinamico}; se $\alpha_{k} = \alpha$ per ogni $k \ge 0$ esso si dice \definition{metodo di Richardson stazionario}.

\highspace
Questo metodo richiede ad ogni passo di trovare il cosiddetto \definition{residuo precondizionato $\mathbf{z}^{\left(k\right)}$} dato dalla soluzione del sistema lineare:
\begin{equation}\label{eq: residuo precondizionato}
    P\mathbf{z}^{\left(k\right)} = \mathbf{r}^{\left(k\right)}
\end{equation}
Quindi, la nuova iterata è definita da $\mathbf{x}^{\left(k+1\right)} = \mathbf{x}^{\left(k\right)} + \alpha_{k}\mathbf{z}^{\left(k\right)}$.

Per questa ragione la matrice $P$ deve essere scelta in modo tale che il costo computazionale richiesto dalla risoluzione di \ref{eq: residuo precondizionato} sia modesto (ogni matrice $P$ diagonale, tridiagonale o triangolare andrebbe bene a questo scopo).

\newpage

\begin{flushleft}
    \textcolor{Green3}{\faIcon{question-circle} \textbf{Quando converge?}}
\end{flushleft}
Per studiare la convergenza, si definisce la \definition{norma dell'energia} associata alla matrice $A$:
\begin{equation}
    \left|\left| \mathbf{v} \right|\right|_{A} = \sqrt{\mathbf{v}^{T} A \mathbf{v}} \hspace{2em} \forall\mathbf{v} \in \mathbb{R}^{n}
\end{equation}
E si definisce la seguente definizione.

\begin{definitionbox}
    Sia $A \in \mathbb{R}^{n \times n}$.

    Per ogni matrice non singolare ($\det \ne 0$) $P \in \mathbb{R}^{n \times n}$ il \definition{metodo di Richardson stazionario} \underline{\textbf{converge}} se e solo se:
    \begin{equation*}
        \left| \lambda_{i} \right|^{2} < \dfrac{2}{\alpha} \mathrm{Re}\lambda_{i} \hspace{2em} \forall i = 1, \dots, n
    \end{equation*}
    In cui $\lambda_{i}$ sono gli autovalori della matrice risultato di $P^{-1} A$.

    \highspace
    In particolare, se gli autovalori della matrice risultato di $P^{-1} A$ sono reali, allora esso converge se e solo se:
    \begin{equation*}
        0 < \alpha \lambda_{i} < 2 \hspace{2em} \forall i = 1, \dots, n
    \end{equation*}

    \highspace
    Se $A$ e $P$ sono entrambe simmetriche (quindi uguali alle loro rispettive trasposte) e definite positive (definizione pagina \pageref{matrice definita positiva}) il metodo di Richardson stazionario \textbf{converge per ogni possibile scelta di} $\mathbf{x}^{\left(0\right)}$ se e solo se:
    \begin{equation*}
        0 < \alpha < \dfrac{2}{\lambda_{\max}}
    \end{equation*}
    Dove $\lambda_{\max}$ (che è maggiore di zero) è l'autovalore massimo della matrice risultato di $P^{-1} A$.

    \highspace
    E ancora, il raggio spettrale $\rho\left(B_{\alpha}\right)$ della matrice di iterazione $B_{\alpha} = I - \alpha P^{-1}A$ è il minimo quando $\alpha = \alpha_{\text{opt}}$, dove:
    \begin{equation}
        \alpha_{\text{opt}} = \dfrac{2}{\lambda_{\min} + \lambda_{\max}}
    \end{equation}
    Dove ovviamente $\lambda_{\min}$ è l'autovalore minimo della matrice risultato di $P^{-1} A$.

    \highspace
    Infine, se $\alpha = \alpha_{\text{opt}}$, vale la seguente \textbf{stima di convergenza}:
    \begin{equation}
        \left|\left| \mathbf{e}^{\left(k\right)} \right|\right|_{A} \le \left(
            \dfrac{K \left(P^{-1}A\right)-1}{K \left(P^{-1}A\right)+1}
        \right)^{k}
        \left|\left| \mathbf{e}^{\left(0\right)} \right|\right|_{A}
        \hspace{2em} k \ge 0
    \end{equation}
    Si noti che la massima velocità di convergenza è data anche dal raggio spettrale:
    \begin{equation}
        \rho_{\text{opt}} = \dfrac{K \left(P^{-1}A\right)-1}{K \left(P^{-1}A\right)+1}
    \end{equation}
\end{definitionbox}



    \subsection{Metodi numerici per sistemi non lineari}
    
    %%%%%%%%%%%%%%%%%%%%%%%%%%%%%%%%%%%%%%%%%
    % Approssimazione di funzioni e di dati %
    %%%%%%%%%%%%%%%%%%%%%%%%%%%%%%%%%%%%%%%%%
    \section{Approssimazione di funzioni e di dati}

\subsection{Interpolazione}

In molte applicazioni concrete si conosce una funzione solo attraverso i suoi valori in determinati punti. Si supponga di conoscere $n+1$ coppie di valori $\left\{x_{i}, y_{i}\right\}$ con $i = 0, \dots, n$, dove i punti $x_{i}$, tutti distinti, vengono chiamati \definition{nodi}.

\highspace
In tal caso, può apparire naturale richiedere che la funzione approssimante $\tilde{f}$ soddisfi le seguenti uguaglianze:
\begin{equation}\label{eq: interpolatore}
	\tilde{f}\left(x_{i}\right) = y_{i} \hspace{2em} i = 0, 1, \dots, n
\end{equation}
Una tale funzione $\tilde{f}$ viene chiamata \definition{interpolatore} dell'insieme di dati $\left\{y_{i}\right\}$ e le equazioni del tipo \ref{eq: interpolatore} sono le \textbf{condizioni di interpolazione}.

\highspace
Esistono vari tipi di interpolatori:
\begin{itemize}
	\item L'\definition{interpolatore polinomiale}:
	\begin{equation*}
		\tilde{f}\left(x\right) = a_{0} + a_{1}x + a_{2}x^{2} + \cdots + a_{n}x^{n}
	\end{equation*}
	
	\item L'\definition{interpolatore trigonometrico}:
	\begin{equation*}
		\tilde{f}\left(x\right) = a_{-M} e^{-i M x} + \cdots + a_{0} + \cdots + a_{M} e^{i M x}
	\end{equation*}
	Dove $M$ è un intero pari a $\frac{n}{2}$ se $n$ è pari, $\frac{\left(n+1\right)}{2}$ se $n$ è dispari, e $i$ è l'unità immaginaria.
	
	\item L'\definition{interpolatore razionale}:
	\begin{equation*}
		\tilde{f}\left(x\right) = \dfrac{
			a_{0} + a_{1}x + \cdots + a_{k}x^{k}
		}{
			a_{k+1} + a_{k+2}x + \cdots + a_{k+n+1} x^{n}
		}
	\end{equation*}
\end{itemize}
    \subsection{Interpolazione Lagrangiana}

L'interpolazione Lagrangiana è un'interpolazione di tipo polinomiale.

\begin{definitionbox}[: interpolazione Lagrangiana]
	Per ogni insieme di coppie $\left\{x_{i}, y_{i}\right\}$, $i = 0, \dots, n$, con i nodi $x_{i}$ distinti fra loro, esiste un unico polinomio di grado minore od uguale a $n$, che viene indicato con $\prod_{n}$ e viene chiamato \textbf{polinomio interpolatore} dei valori $y_{i}$ nei nodi $x_{i}$, tale che:
	\begin{equation}\label{eq: quantificazione dell'errore - interpolazione Lagrangiana}
		\displaystyle\prod_{n}\left(x_{i}\right) = y_{i} \hspace{2em} i = 0, \dots, n
	\end{equation}
	Quando i valori $\left\{y_{i}, i = 0, \dots, n\right\}$, rappresentano i valori assunti da una funzione continua $f$ (ovvero $y_{i} = f\left(x_{i}\right)$), $\prod_{n}$ è detto \textbf{polinomio interpolatore} di $f$ (in breve, interpolatore di $f$) e viene indicato con $\prod_{n}f$.
\end{definitionbox}

\noindent
Quindi un \emph{interpolatore} è una \textbf{funzione che assume il valore dei dati in corrispondenza dei nodi} $x_{i}$.

\begin{flushleft}
	\textcolor{Green3}{\faIcon{question-circle} \textbf{Come ottenere il polinomio interpolatore?}}
\end{flushleft}
Per ogni $k$ compreso tra $0$ e $n$ si costruisce un polinomio di grado $n$, denotato $\varphi_{k}\left(x\right)$, il quale interpola i valori $y_{i}$ tutti nulli fuorché quello per $i=k$ per il quale $y_{k} = 1$, ovvero:
\begin{equation*}
	\varphi_{k} \in \mathbb{P}_{n} \hspace{2em} \varphi_{k}\left(x_{j}\right) = \delta_{jk} = \begin{cases}
		1 & \text{se } j=k \\
		0 & \text{altrimenti}
	\end{cases}
\end{equation*}
Dove $\delta_{jk}$ è il simbolo di Kronecker. Si può dunque definire la formula dei \definition{polinomi caratteristici di Lagrange} $\varphi_{k}\left(x\right)$:
\begin{equation}
	\varphi_{k}\left(x\right) = \displaystyle\prod_{j=0, j \ne k}^{n} \dfrac{x-x_{j}}{x_{k}-x_{j}} \hspace{2em} k = 0, \dots, n
\end{equation}
Che non è altro che:
\begin{equation*}
	\varphi_{k}\left(x\right) = \dfrac{
		\left(x-x_{0}\right)\left(x-x_{1}\right) \cdots
		\left(x-x_{k-1}\right)\left(x-x_{k+1}\right) \cdots
		\left(x-x_{n}\right)
	}{
		\left(x_{k}-x_{0}\right)\left(x_{k}-x_{1}\right) \cdots \left(x_{k}-x_{k-1}\right)\left(x_{k}-x_{k+1}\right) \cdots
		\left(x-x_{n}\right)
	}
\end{equation*}
Essi sono dati dal prodotto di $n$ termini di primo grado, perciò sono dei \textbf{polinomi di grado $n$}.

\begin{examplebox}[: polinomi caratteristici di Lagrange]
	Si prenda:
	\begin{multicols}{2}
		\begin{itemize}
			\item $n = 2$
			\item $x_{0} = -1$
			\item $x_{1} = 0$
			\item $x_{2} = 1$
		\end{itemize}
	\end{multicols}
	I 3 polinomi caratteristici di Lagrange sono dati da:
	\begin{equation*}
		\begin{array}{rcl}
			\varphi_{0}\left(x\right) &=& \dfrac{\left(x-x_{1}\right)\left(x-x_{2}\right)}{\left(x_{0}-x_{1}\right)\left(x_{0}-x_{2}\right)} = \dfrac{1}{2} x\left(x-1\right) \\ [1.5em]
			%
			\varphi_{1}\left(x\right) &=& \dfrac{\left(x-x_{0}\right)\left(x-x_{2}\right)}{\left(x_{1}-x_{0}\right)\left(x_{1}-x_{2}\right)} = -\left(x+1\right)\left(x-1\right) \\ [1.5em]
			%
			\varphi_{2}\left(x\right) &=& \dfrac{\left(x-x_{0}\right)\left(x-x_{1}\right)}{\left(x_{2}-x_{0}\right)\left(x_{2}-x_{1}\right)} = \dfrac{1}{2}x\left(x+1\right)
		\end{array}
	\end{equation*}
	Come atteso, si nota che i 3 polinomi caratteristici di Lagrange sono dei polinomi di grado 2 e soddisfano la proprietà $\varphi_{k}\left(x_{i}\right) = \delta_{ik}$
\end{examplebox}

\highspace
Quanto detto finora, può essere generalizzato nel seguente modo:
\begin{equation}
	\displaystyle\prod_{n}\left(x\right) = \displaystyle\sum_{j=0}^{n} y_{j}\varphi_{j}\left(x\right)
\end{equation}

\highspace
\begin{flushleft}
	\textcolor{Green3}{\faIcon{book-reader} \textbf{Proprietà interpolatore Lagrangiano}}
\end{flushleft}
Le proprietà sono:
\begin{enumerate}
	\item $\prod_{n}\left(x\right)$ è un \textbf{interpolatore}. Difatti, valutandolo per un generico nodo $x_{i}$, si ottiene:
	\begin{equation*}
		\displaystyle\prod_{n}\left(x_{i}\right) = \displaystyle\sum_{j=0}^{n} y_{j}\varphi_{j}\left(x_{i}\right) = \sum_{j=0}^{n} y_{j}\delta_{ij} = y_{i}
	\end{equation*}
	
	\item $\prod_{n}\left(x\right)$ è un \textbf{polinomio di grado} $n$. Infatti, esso è dato dalla somma dei polinomi di grado $n$ $y_{j}\varphi_{j}\left(x\right)$.
	
	\item $\prod_{n}\left(x\right)$ è l'\textbf{unico polinomio di grado $m$ interpolante gli $n+1$ dati} $\left(x_{i}, y_{i}\right)$, $i = 0, \dots, n$. 
	
	Si supponga esista un altro interpolatore polinomiale di grado $n$ $\psi_{n}\left(x\right)$ che interpola i dati $\left(x_{i}, y_{i}\right)$, $i = 0, \dots, n$. Si introduce il seguente polinomio di grado $n$:
	\begin{equation*}
		D\left(x\right) = \prod_{n}\left(x\right) - \psi_{n}\left(x\right)
	\end{equation*}
	Allora vale il seguente risultato:
	\begin{equation*}
		D\left(x_{i}\right) = \prod_{n}\left(x_{i}\right) - \psi_{n}\left(x_{i}\right) = 0 \hspace{2em} \forall i = 0, \dots, n
	\end{equation*}
	Di conseguenza, $D\left(x\right)$ ha $n+1$ zeri. Essendo un polinomio di grado $n$, si deve avere $D\left(x\right) \equiv 0$, da cui segue l'unicità.
\end{enumerate}

\newpage

\subsubsection{Accuratezza (errore) nel caso di approssimazione di funzioni}

Si consideri il caso di approssimazione dei valori di una funzione $f\left(x\right)$. Si ha dunque il seguente risultato.

\begin{definitionbox}[: quantificazione dell'errore]
	Sia $I$ un intervallo limitato, e si considerino $n+1$ nodi di interpolazione distinti $\left\{x_{i}, i = 0, \dots, n\right\}$ in $I$. Sia $f$ derivabile con continuità fino all'ordine $n+1$ in $I$. Allora $\forall x \in I$, $\exists \xi_{x} \in I$ tale che:
	\begin{equation}
		E_{n}f\left(x\right) = f\left(x\right) - \displaystyle\prod_{n} f\left(x\right) = \dfrac{f^{\left(n+1\right)}\left(\xi_{x}\right)}{\left(n+1\right)!} \: \prod_{i=0}^{n} \left(x-x_{i}\right)
	\end{equation}
\end{definitionbox}

\noindent
In altre parole, la definizione rappresenta come si può \textbf{quantificare l'errore} che si commette sostituendo ad una funzione $f$ il suo polinomio interpolatore $\prod_{n}f$ (in questo caso Lagrangiano).

\highspace
Nel seguente grafico si può notare che la funzione $E_{n}f\left(x\right)$ si annulla in corrispondenza dei nodi $x_{i}$. Inoltre, è in generale diverso da zero lontano dai nodi, ovvero con $x \ne x_{i}$.

\begin{figure}[!htp]
	\centering
	\includegraphics[width=0.4\textwidth]{img/interpolazione-lagrangiana-1.png}
\end{figure}

\noindent
Nel caso di nodi equispaziati, vale la seguente \definition{stima dell'errore massimo dell'interpolatore Lagrangiano}:
\begin{equation}\label{eq: stima dell'errore massimo dell'interpolare Lagrangiano}
	\underset{x \in I}{\max} \left|E_{n} f\left(x\right)\right| \le \dfrac{\left|\underset{x \in I}{\max} f^{\left(n+1\right)\left(x\right)}\right|}{4\left(n+1\right)} \cdot h^{n+1}
\end{equation}

\begin{figure}[!htp]
	\centering
	\includegraphics[width=0.4\textwidth]{img/interpolazione-lagrangiana-2.png}
	\label{fig: stima dell'errore massimo dell'interpolatore Lagrangiano}
\end{figure}

\newpage

\subsubsection{Convergenza dell'interpolatore Lagrangiano}

All'aumento delle informazioni a disposizione ($n+1$), idealmente si vorrebbe un \textbf{miglioramento dell'accuratezza di una funzione approssimante}. In particolare, si vorrebbe il seguente risultato:
\begin{equation*}
	\lim\limits_{n \rightarrow \infty} \underset{x \in I}{\max} \left|E_{n}f\left(x\right)\right| = 0
\end{equation*}

\highspace
Dall'equazione \ref{eq: stima dell'errore massimo dell'interpolare Lagrangiano}, a pagina \pageref{eq: stima dell'errore massimo dell'interpolare Lagrangiano}, si può notare che il denominatore della frazione e il termine $h^{n+1}$ tendono a zero quando $n$ tende a infinito. Al contrario, il numeratore non è certo se sia limitato per $n$ che tende a infinito.

\highspace
Difatti, possono esistere casi in cui si ha:
\begin{equation*}
	\lim\limits_{n \rightarrow \infty} \left|\underset{x \in I}{\max} f^{\left(n+1\right)}\left(x\right)\right| = \infty
\end{equation*}
Che portano a:
\begin{equation*}
	\lim\limits_{n \rightarrow \infty} \underset{x \in I}{\max} \left|E_{n} f\left(x\right)\right| = \infty
\end{equation*}
Ovvero alla \textbf{\underline{non} convergenza dell'interpolatore Lagrangiano}.

\begin{examplebox}[: fenomeno di Runge]
	Il \definition{fenomeno di Runge} è un evento che si scatena quando \textbf{l'interpolatore Lagrangiano non raggiunge la convergenza}.
	
	In particolare, in presenza di questo fenomeno, la funzione errore $E_{n}$ presenta delle oscillazioni ai nodi estremi che crescono con il crescere di $n$.
	
	Un esempio di interpolatore $\prod_{12} f$ (in linea continua) calcolato su 13 nodi equispaziati nel caso della funzione di Runge (in linea tratteggiata):
	\begin{equation*}
		f\left(x\right) = \dfrac{1}{\left(1+x^{2}\right)}
	\end{equation*}
	\begin{center}
		\includegraphics[width=.7\textwidth]{img/fenomeno-di-runge-1.pdf}
	\end{center}
\end{examplebox}
    \subsection{Interpolazione Lagrangiana composita}

Una miglioria che è possibile fare all'interpolazione Lagrangiana, è quella di introdurre un \textbf{interpolatore continuo dato dall'unione di tanti interpolatori Lagrangiani di basso ordine}, ovvero $k \ll n$.

\highspace
Tali singoli interpolatori locali sono costruiti sugli intervalli disgiunti $I_{j}$ ognuno composto da $k+1$ nodi e di lunghezza $H = kh$:

\begin{figure}[!htp]
	\centering
	\includegraphics[width=0.7\linewidth]{img/interpolazione-lagrangiana-composita-1.png}
\end{figure}

\noindent
Tale interpolatore globale, chiamato \definition{interpolazione Lagrangiana composita}, viene indicato con $\prod_{k}^{H}\left(x\right)$. Se si vuole enfatizzare che l'interpolatore è stato costruito per approssimare una certa funzione $f$, allora si utilizza la notazione $\prod_{k}^{H} f\left(x\right)$.

\begin{examplebox}[: interpolatore Lagrangiano composito lineare]
	Si consideri il caso $k=1$. La funzione rappresentata in linea continua è:
	\begin{equation*}
		f\left(x\right) = x^{2} + \dfrac{10}{\left(\sin\left(x\right)+1.2\right)}
	\end{equation*}
	Ed il suo interpolatore lineare composito rappresentato con la linea tratteggiata:
	\begin{equation*}
		\displaystyle\prod_{1}^{H}f\left(x\right)
	\end{equation*}
	\begin{center}
		\includegraphics[width=.6\textwidth]{img/interpolazione-lagrangiana-composita-2.png}
	\end{center}
	
	\noindent
	Vista la sua semplicità, questo interpolatore è molto utilizzato nelle applicazioni.
\end{examplebox}

\newpage

\subsubsection{Accuratezza (errore) e convergenza dell'interpolatore Lagrangiano composito}

A differenza dell'interpolazione Lagrangiana, all'aumentare del numero di informazioni $n+1$ che si hanno a disposizione, l'accuratezza dell'interpolatore Lagrangiano composito subisce un \emph{miglioramento}.

\highspace
All'aumentare del valore di $n$, verranno aumentati anche il numero di intervalli $I_{j}$ su cui costruire gli interpolatori Lagrangiani locali, \textbf{senza variare il gradi locale $k$ che rimarrà costante}. Questa tecnica riesce ad evitare il fenomeno di Runge, ovverosia che l'interpolatore Lagrangiano non raggiunge la convergenza.

\highspace
Riguardo alla convergenza, si introduce l'\textbf{errore puntuale}:
\begin{equation}
	E_{k}^{H}f\left(x\right) = f\left(x\right) - \displaystyle\prod_{k}^{H}f\left(x\right)
\end{equation}
Il quale è dato dal massimo degli errori degli interpolatori Lagrangiani di grado $k$ su ogni $I_{j}$. In caso di nodi equispaziati, si ottiene che la \definition{stima dell'errore dell'interpolatore Lagrangiano composito} tende a zero nel caso in cui $H$ tende a zero:
\begin{equation}\label{eq: stima dell'errore interpolatore Lagrangiano composito}
	\begin{array}{rcl}
		\underset{x \in I}{\max} \left| E_{k}^{H} f\left(x\right) \right| & \le & \underset{j}{\max} \dfrac{\underset{x \in I_{j}}{\max} \left|f^{\left(k+1\right)}\left(x\right)\right|}{4\left(k+1\right)} \cdot h^{\left(k+1\right)} \\ [1.5em]
		%
		&\le& \dfrac{\underset{x \in I}{\max} \left|f^{\left(k+1\right)}\left(x\right)\right|}{4\left(k+1\right)} \cdot h^{\left(k+1\right)} \\ [1.5em]
		%
		&& \text{sostituendo } h \text{ con } \dfrac{H}{k} \\ [1.5em]
		%
		&\le& \underbrace{\dfrac{\underset{x \in I}{\max} \left|f^{\left(k+1\right)}\left(x\right)\right|}{4\left(k+1\right)k^{\left(k+1\right)}}}_{\text{Indipendente da }H} \cdot H^{\left(k+1\right)}
	\end{array}
\end{equation}
Oltre ad un errore, questo dimostra la \definition{convergenza dell'interpolatore Lagrangiano composito}.

\newpage

\begin{flushleft}
	\textcolor{Green3}{\faIcon{question-circle} \textbf{Se $H$ e $k$ sono importanti, come devono essere scelti?}}
\end{flushleft}
Dipende dall'origine dei dati:
\begin{itemize}
	\item Se i dati provengono da una funzione $f\left(x\right)$ nota sull'intervallo $\left[a,b\right]$ che genera $y_{i} = f\left(x_{i}\right)$ con $i = 0, \dots, n$, allora:
	\begin{itemize}
		\item Si \textbf{decide il valore di} $k$ da utilizzare cercando di non superare il valore $3$ per non incorrere nel fenomeno di Runge (non convergenza!);
		
		\item Si \textbf{sceglie il valore dell'ampiezza degli intervalli} $H$ in modo da avere l'errore desiderato in base alla stima dell'errore riportato nell'equazione \ref{eq: stima dell'errore interpolatore Lagrangiano composito};
		
		\item Si \textbf{partiziona l'intervallo} $\left[a,b\right]$ in intervallo di ampiezza $H$ e su \textbf{ognuno di essi si considerano $k+1$ nodi}.
	\end{itemize}
	
	\item Se i dati provengono da misure, il numero di questi $n+1$ è fissato. Per cui le operazioni da fare possono essere una delle seguenti:
	\begin{itemize}
		\item Un'\definition{interpolazione composita lineare}:
		\begin{equation*}
			k=1 \hspace{2em} H = \dfrac{\left(b-a\right)}{n}
		\end{equation*}
		
		\item Un'\definition{interpolazione composita quadratica}:
		\begin{equation*}
			k = 2 \hspace{2em} H = \dfrac{2\left(b-a\right)}{n}
		\end{equation*}
	\end{itemize}
\end{itemize}
    \subsection{Interpolazione sui nodi di Chebyshev}

L'interpolazione Lagrangiana composita consente di evitare il fenomeno di Runge. Tuttavia, l'interpolazione Lagrangiana può essere modificata \textbf{posizionando i nodi in precise posizioni che garantiscono la stabilità al crescere di} $n$. Negli interpolatori presentati, tutti si basavano su nodi equispaziati. Nel seguente capitolo si mostrano i nodi di Chebyshev, ovvero un'ubicazione specifica dei nodi nello spazio.

\highspace
\begin{flushleft}
	\textcolor{Green3}{\faIcon{question-circle} \textbf{Solo il fenomeno di Runge può essere evitato?}}
\end{flushleft}
La risposta chiaramente è no. L'applicazione di questa tecnica consente:
\begin{itemize}
	\item Di \textbf{utilizzare} \emph{sempre} con successo il \textbf{polinomio interpolatore Lagrangiano} di grado $n$.
	
	\item Di evitare la non convergenza, ovvero il fenomeno di Runge.
\end{itemize}

\highspace
\begin{flushleft}
	\textcolor{Red2}{\faIcon{bookmark} \textbf{Nodi di Chebyshev}}
\end{flushleft}
Si consideri il caso in cui il dominio sia $\left[-1, 1\right]$ e $n+1$ dati ubicati in corrispondenza delle seguenti ascisse:
\begin{equation}
	\widehat{x}_{i} = -\cos\left(\dfrac{\pi i}{n}\right) \hspace{2em} i = 0, \dots, n
\end{equation}
\begin{figure}[!htp]
	\centering
	\includegraphics[width=.5\textwidth]{img/nodi-di-chebyshev-1.pdf}
	\caption{Distribuzione dei nodi di Chebyshev nell'intervallo $\left[-1, 1\right]$.}
\end{figure}

\noindent
Questi \definition{nodi di Chebyshev} sono \textbf{ottenuti come proiezioni sull'asse $x$ di punti sulla circonferenza unitaria individuati da settori circolari ottenuti con lo stesso angolo} $\frac{\pi}{n}$.

\highspace
\begin{flushleft}
	\textcolor{Green3}{\faIcon{question-circle} \textbf{E se viene applicato l'interpolatore Lagrangiano sui nodi di Chebyshev?}}
\end{flushleft}
Come detto all'inizio, si può modificare direttamente l'interpolatore Lagrangiano. Quindi, si consideri l'interpolatore Lagrangiano di grado $n$ costruito sui nodi di Chebyshev. Tale interpolatore si ottiene applicando le equazioni presentate precedentemente, con l'unico vincolo di \textbf{prendere in considerazione $\widehat{x}_{i}$ come nodi su cui costruire gli $n$ polinomi caratteristici di Lagrange}:
\begin{equation}
	\begin{array}{rcl}
		\widehat{\varphi}_{k}\left(x\right) &=& \displaystyle\prod_{j = 0, j \ne k}^{n} \dfrac{x - \widehat{x}_{i}}{\widehat{x}_{k} - \widehat{x}_{i}} \\ [1.5em]
		\displaystyle\prod_{n}^{C}\left(x\right) &=& \displaystyle\sum_{j=0}^{n} y_{j} \widehat{\varphi}_{j}\left(x\right)
	\end{array}
\end{equation}
La variabile $\prod_{n}^{C} \left(x\right)$ rappresenta l'\definition{interpolatore Lagrangiano sui nodi di Chebyshev}. Si indica con la rappresentazione $\prod_{n}^{C} f\left(x\right)$ quando viene applicato a dati ottenuti dalla valutazione di una funzione $f$, ovvero $y_{i} = f\left(x_{i}\right)$.

\longline

\subsubsection{Convergenza dell'interpolazione sui nodi di Chebyshev}

\begin{theorem}
	Si supponga che la funzione $f\left(x\right)$ ammetta derivata continua fino all'ordine $s+1$ compreso, ovverosia $f \in C^{\left(s+1\right)}\left(\left[-1,1\right]\right)$.
	
	\highspace
	Allora, si ha il seguente risultato di \definition{convergenza per l'interpolazione sui nodi di Chebyshev}:
	\begin{equation}
		\underset{x \in \left[-1,1\right]}{\max} \left|f\left(x\right) - \displaystyle\prod_{n}^{C} f\left(x\right)\right| \le \tilde{C} \dfrac{1}{n^{s}}
	\end{equation}
	Per un'opportuna costante $C$.
\end{theorem}

\noindent
Dal precedente teorema, si possono fare 3 osservazioni:
\begin{enumerate}
	\item Se $s \ge 1$ allora si è sicuri che ci sarà \textbf{sempre convergenza}:
	\begin{equation}
		\lim\limits_{n \rightarrow \infty} \underset{x \in \left[-1,1\right]}{\max} \left|f\left(x\right) - \displaystyle\prod_{n}^{C} f\left(x\right)\right| = 0
	\end{equation}
	
	\item La \textbf{velocità di convergenza aumenta all'aumentare di $s$}.
	
	\item \underline{Se} il teorema è valido $\forall s > 0$, allora la \textbf{velocità di convergenza è \emph{esponenziale}}. Questa è la migliore stima che è possibile ottenere, dato che è più veloce di qualsiasi altro $\frac{1}{n^{s}}$:
	\begin{equation}
		\underset{x \in \left[-1, 1\right]}{\max} \left| f - \displaystyle\prod_{n}^{C} f\left(x\right) \right| \le \tilde{C} e^{-n}
	\end{equation}
\end{enumerate}

\newpage

\subsubsection{Generalizzazione dell'intervallo}

Nel caso in cui il dominio di interesse non sia più da $\left[-1, 1\right]$ (come viene utilizzato nelle precedenti equazioni), ma un intervallo generale $\left[a,b\right]$, la \dquotes{conversione} può avvenire mappando i nodi $\widehat{x}_{j}$ da $\left[-1,1\right]$ a $\left[a,b\right]$:
\begin{equation}
	x = \psi\left(\widehat{x}\right) = \dfrac{a+b}{2} + \dfrac{b-a}{2} \: \widehat{x}
\end{equation}
Applicando la \definition{generalizzazione dell'intervallo ai nodi di Chebyshev}, si ottiene:
\begin{equation}
	x_{j}^{C} = \psi\left(\widehat{x}_{j}\right) = \dfrac{a+b}{2} + \dfrac{b-a}{2} \: \widehat{x}_{j} \hspace{2em} j = 0, \dots, n
\end{equation}
Di conseguenza, il \definition{polinomio di Lagrange applicato sui nodi di Chebyshev in un intervallo generale} diventa:
\begin{equation}
	\varphi_{k}^{C}\left(x\right) = \prod_{j = 0, j \ne k}^{n} \dfrac{x - x_{j}^{C}}{x_{k}^{C} - x_{j}^{C}}
\end{equation}

    \subsection{Interpolazione trigonometrica}

Nel caso di funzioni periodiche, ad esempio con periodo $2\pi$ e con $f\left(0\right) = f\left(2\pi\right)$, si ha il valore delle ascisse pari a:
\begin{equation*}
	x_{j} = \dfrac{2\pi j}{\left(n+1\right)} \hspace{2em} j = 0, \dots, n
\end{equation*}
In questo caso, l'interpolatore Lagrangiano non approssima in modo ottimale la funzione $f$ lontana dai nodi, dato che è un polinomio di grado $n$ (per cui in generale non periodico!).

\highspace
Supponendo $n$ pari e $M = \frac{n}{2}$, si può costruire l'\definition{interpolatore trigonometrico $\tilde{f}\left(x\right)$} come la \textbf{combinazione lineare di seni e coseni}:
\begin{equation}\label{eq: interpolatore trigonometrico}
	f\left(x\right) = \dfrac{a_{0}}{2} + \displaystyle\sum_{k=1}^{M} \left[a_{k} \cos\left(kx\right) + b_{k} \sin\left(kx\right)\right]
\end{equation}
Per opportuni coefficienti complessi incogniti $a_{k}$, per $k = 0, \dots, M$ e $b_{k}$ per $k = 1, \dots, M$. Come si può vedere, la funzione periodica $\tilde{f}\left(x\right)$ è di periodo $2\pi$ caratterizzata da $M+1$ \textbf{frequenze} e \textbf{coefficienti} $a_{k}$ e $b_{k}$ con $k = 0, \dots, M$.

\highspace
\textbf{Sfruttando} la nota formula di \textbf{Eulero}:
\begin{equation*}
	e^{ikx} = \cos\left(kx\right) + i\sin\left(kx\right)
\end{equation*}
L'\textbf{interpolatore trigonometrico} può essere riscritto in modo più compatto:
\begin{equation}\label{eq: interpolatore trigonometrico con Eulero}
	\tilde{f}\left(x\right) = \displaystyle\sum_{k = -M}^{M} c_{k} e^{ikx}
\end{equation}
Inoltre, dall'equazione \ref{eq: interpolatore trigonometrico} i valori $a_{k}, b_{k}$ con $k = 0, \dots, M$ sono:
\begin{equation*}
	\begin{cases}
		a_{k} = c_{k} + c_{-k} \\
		b_{k} = i\left(c_{k} - c_{-k}\right)
	\end{cases}
\end{equation*}
E nell'interpolazione trigonometrica con Eulero \ref{eq: interpolatore trigonometrico con Eulero} si ha:
\begin{equation*}
	\begin{cases}
		c_{k} = \frac{1}{2}\left(a_{k} - ib_{k}\right) \\
		c_{-k} = \frac{1}{2}\left(a_{k} + ib_{k}\right)
	\end{cases}
\end{equation*}
Quindi, le funzioni interpolatrici trigonometriche $\tilde{f}\left(x\right)$ sono $2M+1 = n+1$ equazioni nelle $n+1$ incognite $c_{k}$.

\highspace
La \textbf{distanza} (costante) fra due nodi $h = \frac{2\pi}{\left(n+1\right)}$ come:
\begin{equation*}
	x_{j} = jh \hspace{2em} j = 0, \dots, n
\end{equation*}
Si impone inoltre il \textbf{vincolo di interpolazione}:
\begin{equation}\label{eq: vincolo di interpolazione}
	\tilde{f}\left(x_{j}\right) = \displaystyle\sum_{k = -M}^{M} c_{k}e^{ikjh} = f\left(x_{j}\right) \hspace{2em} j = 0, \dots, n
\end{equation}
A differenza dell'interpolazione Lagrangiana, i coefficienti della combinazione lineare $c_{k}$ con $k = -M, \dots, M$, sono incogniti. Si hanno di conseguenza $n+1$ equazioni nelle $2M + 1 = n+1$ incognite $c_{k}$. Tale equazione è una base di partenza per le formule nei paragrafi successivi.

\newpage

\subsubsection{Trasformata Discreta di Fourier}

Dato un intero $m \in \left[-M, M\right]$, si moltiplica all'equazione di base \ref{eq: vincolo di interpolazione}, a pagina \pageref{eq: vincolo di interpolazione}, a sinistra e a destra per $e^{-imx_{j}} \Rightarrow e^{imjh}$ e si sommano su $j$:
\begin{equation}
	\displaystyle\sum_{j=0}^{n}\sum_{k=-M}^{M} c_{k} e^{ikjh} e^{-imjh} = \sum_{j=0}^{n} f\left(x_{j}\right) e^{-imjh}
\end{equation}
In cui si può esporre una relazione esplicita fra i coefficienti e i valori noti della funzione $f\left(x_{j}\right)$:
\begin{equation}
	c_{m} = \dfrac{1}{n+1} \displaystyle\sum_{j=0}^{n} f\left(x_{j}\right) e^{-imjh} \hspace{2em} m = -M, \dots, M
\end{equation}
La funzione $f\left(x_{j}\right)$ viene chiamata \definition{Trasformata discreta di Fourier (DFT)}. \textbf{Queste sono $n+1$ equazioni nelle incognite $f\left(x_{j}\right)$}.

\highspace
Parlando di DTF, è necessario introdurre anche lo \textbf{spazio fisico} e delle \textbf{frequenze}. Dati i vettori $\mathbf{c} \in \mathbb{R}^{n+1}$ di componenti $c_{k}$ e $\mathbf{f} \in \mathbb{R}^{n+1}$ di componenti $f\left(x_{j}\right)$, è possibile riscrivere la trasformata di Fourier nello \textbf{spazio delle frequenze} $k$ come:
\begin{equation*}
	c_{k} = \dfrac{1}{n+1} \displaystyle\sum_{j=0}^{n} f\left(x_{j}\right) e^{-ikjh} \hspace{2em} k = -M, \dots, M
\end{equation*}
E in forma matriciale nel seguente modo:
\begin{equation}
	\mathbf{c} = T\mathbf{f}
\end{equation}
Con la matrice $T$ composta da elementi:
\begin{equation*}
	T_{kj} = \dfrac{1}{n+1} e^{-ikjh}
\end{equation*}

\highspace
Nel caso in cui si vuole passare dallo \textbf{spazio delle frequenze} $f\left(x_{j}\right)$ allo \textbf{spazio fisico} $c_{k}$, si può usare il vincolo di interpolazione (eq. \ref{eq: vincolo di interpolazione}, pag. \pageref{eq: vincolo di interpolazione}):
\begin{equation*}
	\tilde{f}\left(x_{j}\right) = \displaystyle\sum_{k = -M}^{M} c_{k}e^{ikjh} \hspace{2em} j = 0, \dots, n
\end{equation*}
E in forma matriciale nel seguente modo:
\begin{equation}
	\mathbf{f} = T^{-1} \mathbf{c}
\end{equation}
Con la matrice $T^{-1}$ composta da elementi:
\begin{equation*}
	\left(T^{-1}\right)_{kj} = e^{ikjh}
\end{equation*}
La \textbf{trasformazione dallo spazio delle frequenze allo spazio fisico} è chiamata \definition{Trasformata Discreta di Fourier inversa (Inverse Discrete Fourier Transform, IDFT)}.

\newpage

\subsubsection{Fast Fourier Transform (FFT)}

Il calcolo dei coefficienti $c_{k}$ può essere fatto ancora più rapidamente utilizzato la \definition{Trasformata rapida di Fourier (Fast Fourier Transform, FFT)}.

\highspace
Si consideri una generica matrice quadrata $A$ di dimensione $\left(n+1\right) \times \left(n+1\right)$ e un vettore $\mathbf{x}$ di dimensione $n+1$. È possibile calcolare la componente $w_{j}$ del vettore $\mathbf{w} = A\mathbf{x}$ come:
\begin{equation}
	w_{j} = \displaystyle\sum_{k=1}^{n} A_{jk} x_{k}
\end{equation}

\highspace
Nella FFT, la matrice $T$ ha una struttura particolare chiamata di Toeplitz. Questo comporta che l'elemento $w_{j}$ del vettore $\mathbf{w} = T\mathbf{x}$ è determinato solo dagli elementi della matrice $T$, ovverosia $T_{km}$ (con $j = km$). Una struttura tipica della matrice è:
\begin{equation*}
	T = \dfrac{1}{n+1} \begin{bmatrix}
		e^{-ih} & e^{-2ih} & e^{-3ih} & \cdots & \cdots & e^{-nih} \\
		e^{-2ih} & e^{-4ih} & e^{-6ih} & \cdots & \cdots & e^{-2nih} \\
		e^{-3ih} & e^{-6ih} & e^{-9ih} & \cdots & \cdots & e^{-3nih} \\
		\vdots & & \ddots & & & \vdots \\
		\vdots & & \ddots & & & \vdots \\
		e^{-nih} & e^{-2nih} & e^{-3nih} & \cdots & \cdots & e^{-n^{2}ih}
	\end{bmatrix}
\end{equation*}
Il \textbf{costo computazione} del vettore $\mathbf{w}$, a causa della forma particolare della matrice di Toeplitz, è uguale a \underline{$n \log_{2} n$ operazioni}.

\longline

\subsubsection{Espressione Lagrangiana dell'interpolatore trigonometrico}\index{interpolatore trigonomentrico in forma Lagrangiana}

L'interpolatore trigonometrico può essere scritto in forma Lagrangiana, ovvero con coefficienti nella combinazione lineare dati dal risultato dei vari $f\left(x_{j}\right)$, nel seguente modo:
\begin{equation}
	\tilde{f}\left(x\right) = \displaystyle\sum_{j=0}^{n} f\left(x_{j}\right) \varphi_{j}^{T}\left(x\right)
\end{equation}
Dove il polinomio $\varphi$ non è altro che:
\begin{equation}
	\varphi_{j}^{T}\left(x_{k}\right) = \delta_{jk}
\end{equation}

\begin{flushleft}
	\textcolor{Green3}{\faIcon{question-circle} \textbf{E in questo caso la convergenza?}}
\end{flushleft}
La convergenza è possibile esprimerla tramite il seguente teorema.
\begin{theorem}
	Si supponga che la funzione $f\left(x\right)$ ammetta derivata continua e periodica di periodo $2\pi$ fino all'ordine $s+1$ compreso.
	
	Allora, si ha il seguente risultato di \definition{convergenza per l'interpolatore trigonometrico in forma Lagrangiana}:
	\begin{equation}
		\underset{x \in \left[0, 2\pi\right]}{\max} \left|f\left(x\right) - \tilde{f}\left(x\right)\right| \le C\dfrac{1}{n^{s}}
	\end{equation}
	Per un'opportuna costante $C$.
\end{theorem}

\noindent
Dal precedente teorema, se ne deriva che la \textbf{convergenza è esponenziale qualora il precedente risultato valga per} $\forall s > 0$.

\newpage

\subsubsection{Fenomeno dell'aliasing e teorema di Shannon}

Se il numero di dati $n+1$ non è sufficientemente elevato, l'\textbf{interpolatore trigonometrico non è in grado di descrivere le frequenze $k$ più alte}. Questo problema viene chiamato \definition{fenomeno di aliasing}.

\highspace
Nel mondo reale, l'occhio umano campiona con una frequenza massima le informazioni luminose che provengono dal mondo esterno. Ma se tale campionamento non è sufficientemente frequente da catturare la frequenza massima del fenomeno esterno, allora l'immagine riprodotta dal cervello (che di fatto agisce in questo caso come un interpolatore trigonometrico) sarà caratterizzata da frequenze diverse (aliasing). E questo è anche il motivo per negli elicotteri si vedono le pale girare molto lentamente o addirittura in senso opposto.

\highspace
In particolare, sia $k_{\max}$ la \textbf{frequenza massima} della funzione $f\left(x\right)$. Se $n \le 2k_{\max}$, la \textbf{frequenza massima dell'interpolatore trigonometrico} $\tilde{f}\left(x\right)$ è minore di $k_{\max}$ (aliasing).

\begin{examplebox}[: aliasing]
	Gli effetti dell'aliasing si possono vedere confrontando, per esempio, le seguenti due funzioni:
	\begin{itemize}
		\item In linea continua:
		\begin{equation*}
			f\left(x\right) = \sin\left(x\right) + \sin\left(5x\right)
		\end{equation*}
		
		\item Linea tratteggiata, l'interpolatore trigonometrico $\tilde{f}\left(x\right)$ con $M=3$
	\end{itemize}
	
	\begin{center}
		\includegraphics[width=.5\textwidth]{img/aliasing-1.pdf}
	\end{center}
\end{examplebox}

\noindent
Quindi, il \definition{teorema di Shannon} afferma che $n$ deve essere:
\begin{equation}
	n > 2k_{\max}
\end{equation}
    \subsection{Il metodo dei minimi quadrati}

\begin{flushleft}
	\textcolor{Green3}{\faIcon{question-circle} \textbf{Perché è necessario un altro metodo?}}
\end{flushleft}
Al crescere del grado del polinomio dell'interpolazione di Lagrange, esso non garantisce una maggiore accuratezza nell'approssimazione di una funzione. Per aumentare l'accuratezza, è stato introdotto l'interpolazione Lagrangiana composita. Purtroppo, quest'ultima non è ottima per estrapolare informazioni da dati noti, ovvero \textbf{per generare nuove valutazioni in punti che giacciono al di fuori dell'intervallo di interpolazione}.

\highspace
Quindi, i precedenti metodi esposti non sono ottimi:
\begin{itemize}
	\item \textbf{Interpolazione Lagrangiana}: potrebbe soffrire del fenomeno di Runge, a causa del numero elevato di dati.
	
	\item \textbf{Interpolazione Lagrangiana composita}: terrebbe in conto solo gli ultimi $k+1$ dati, per cui non utilizzerebbe tutta la storia a disposizione e non darebbe un'espressione analiticamente semplice.
	
	\item \textbf{Interpolazione su nodi Chebyshev}: nella realtà è difficile da applicare poiché i dati a disposizione sono equispaziati.
\end{itemize}

\highspace
\begin{flushleft}
	\textcolor{Red2}{\faIcon{bookmark} \textbf{Ragionamento per arrivare alla definizione del metodo}}
\end{flushleft}
Si supponga di avere a disposizione un insieme di dati e ciascun elemento è formato da una coppia:
\begin{equation*}
	\left\{\left(x_{i}, y_{i}\right), \: i = 0, \dots, n\right\}
\end{equation*}
In cui gli $y_{i}$ potrebbero essere i valori che una funzione assume nei nodi $x_{i}$: $q\left(x_{i}\right) = y_{i}$.

\highspace
L'obbiettivo è cercare un \emph{polinomio globale} di grado $m$ (con $m \ge 1$) molto più basso rispetto a $n$ ($m \ll n$). Tale polinomio ha l'obbiettivo di \textbf{minimizzare lo \emph{scarto quadratico medio}}, ovverosia la somma dei quadrati degli errori nei nodi\footnote{Si ricorda che l'errore di un nodo è la distanza tra la funzione lineare e il suo corrispettivo interpolatore. A pagina \pageref{fig: stima dell'errore massimo dell'interpolatore Lagrangiano} è possibile vederlo visivamente.}.

\highspace
Dato lo \textbf{spazio dei polinomi di grado $m$}:
\begin{equation*}
	\mathbb{P}_{m} = \left\{p_{m} \: : \: \mathbb{R} \rightarrow \mathbb{R} \: : \: p_{m}\left(x\right) = b_{0} + b_{1}x + \cdots + b_{m}x^{m}\right\}
\end{equation*}
Allora, il (problema) \definition{metodo dei minimi quadrati} consiste nel cercare il polinomio $q \in \mathbb{P}_{m}$ tale che:
\begin{equation}\label{eq: metodo dei minimi quadrati}
	\displaystyle\sum_{i=0}^{n} \left[y_{i} - q\left(x_{i}\right)\right]^{2} \le \displaystyle\sum_{i=0}^{n} \left[y_{i} - p_{m}\left(x_{i}\right)\right]^{2} \hspace{2em} \forall p_{m} \in \mathbb{P}_{m}
\end{equation}

\highspace
Riscrivendo il polinomio $q\left(x\right)$ come:
\begin{equation*}
	q\left(x\right) = a_{0} + a_{1}x + \cdots + a_{m}x^{m}
\end{equation*}
Dove i coefficienti $a$ sono incogniti, è possibile riformulare il metodo dei minimi quadrati (eq. \ref{eq: metodo dei minimi quadrati}). Quindi determinare $a_{0}, a_{1}, \dots, a_{m}$ tali che:
\begin{equation*}
	\psi\left(a_{0}, a_{1}, \dots, a_{m}\right) = \underset{\left\{b_{i}, i = 0, \dots, m\right\}}{\min} \psi\left(b_{0}, b_{1}, \dots, b_{m}\right)
\end{equation*}
Dove:
\begin{equation*}
	 \psi\left(b_{0}, b_{1}, \dots, b_{m}\right) = \displaystyle\sum_{i=0}^{n} \left[y_{i} - \left(b_{0} + b_{1}x_{i} + \dots + b_{m}x_{i}^{m}\right)\right]^{2}
\end{equation*}
Nel caso in cui si abbia una \definition{retta di regressione} (cioè $m = 1$), il problema si riduce a:
\begin{equation*}
	\psi\left(b_{0}, b_{1}\right) = \displaystyle\sum_{i=0}^{n} \left[y_{i}^{2} + b_{0}^{2} + b_{1}^{2}x_{i}^{2} + 2b_{0}b_{1}x_{i} - 2b_{0}y_{i} - 2b_{1}x_{i}y_{i}\right]
\end{equation*}
Se ne ricava che il grafico della funzione $\psi$ è un \textbf{paraboloide convesso} il cui punto di \textbf{minimo} $\left(a_{0}, a_{1}\right)$ si trova imponendo le sue derivate parziali uguale a zero:
\begin{equation*}
	\dfrac{\partial \psi}{\partial b_{0}}\left(a_{0}, a_{1}\right) = 0 \hspace{2em} \dfrac{\partial \psi}{\partial b_{1}}\left(a_{0}, a_{1}\right) = 0
\end{equation*}
Che equivale a risolvere il seguente sistema di 2 equazioni e incognite:
\begin{equation*}
	\displaystyle \sum_{i=0}^{n} \left[a_{0} + a_{1}x_{i} - y_{i}\right] = 0 \hspace{2em} \displaystyle \sum_{i=0}^{n} \left[a_{0}x_{i} + a_{1}x_{i}^{2} - x_{i}y_{i}\right] = 0
\end{equation*}
Ponendo $D=\left(n+1\right)\displaystyle\sum_{i=0}^{n} x_{i}^{2} - \left(\displaystyle\sum_{i=0}^{n} x_{i}\right)^{2}$, la soluzione è:
\begin{equation*}
	\begin{array}{rcl}
		a_{0} &=& \dfrac{1}{D} \left[\displaystyle\sum_{i=0}^{n} y_{i} \: \displaystyle\sum_{j=0}^{n}x_{j}^{2} - \displaystyle\sum_{j=0}^{n} x_{j} \: \displaystyle\sum_{i=0}^{n}x_{i}y_{i} \right] \\ [2em]
		a_{1} &=& \dfrac{1}{D} \left[\left(n+1\right)\displaystyle\sum_{i=0}^{n}x_{i}y_{i} - \displaystyle\sum_{j=0}^{n}x_{j} \: \sum_{i=0}^{n}y_{i} \right]
	\end{array}
\end{equation*}
Il corrispondente polinomio:
\begin{equation}
	q\left(x\right) = a_{0} + a_{1}x
\end{equation}
È noto come \definition{retta dei minimi quadrati}, o \definition{retta di regressione}.

\highspace
\begin{flushleft}
	\textcolor{Red2}{\faIcon{exclamation-triangle} \textbf{Attenzione}}
\end{flushleft}
Da notare, anche se parzialmente scontato, che nel caso in cui $m$ sia uguale a $n$ ($m=n$), \textbf{il problema si riduce all'interpolatore Lagrangiano}.
    
    %%%%%%%%%%%%%%%%%%%%%%%%%
    % Integrazione numerica %
    %%%%%%%%%%%%%%%%%%%%%%%%%
    \section{Integrazione numerica}

\subsection{Introduzione}

Alcuni problemi ingegneristici richiedono il calcolo di integrali, talvolta pure molto complessi. Non sempre si riesce a trovare in forma esplicita la primitiva di una funzione, anche nel caso in cui la si conosca, potrebbe essere difficile valutarla (come ad esempio $f\left(x\right) = \cos\left(4x\right) \cos\left(3\sin\left(x\right)\right)$).

\highspace
In tutti questi casi, è necessario utilizzare metodi numerici in gradi di restituire un valore approssimato della quantità di interesse, indipendentemente da quanto complessa sia la funzione da integrare o da differenziare. In questo capitolo, vengono presentati alcuni metodi per l'\textbf{approssimazione numerica di integrali di funzioni}, i quali vengono chiamate \definition{formule di quadratura}.

\highspace
\begin{flushleft}
	\textcolor{Green3}{\faIcon{book} \textbf{Alcune notazioni e definizioni di introduzione}}
\end{flushleft}
Si consideri l'integrale:
\begin{equation}\label{eq: formula di quadratura base}
	I\left(f\right) = \int_{a}^{b} f\left(x\right) \:\mathrm{d}x
\end{equation}
Si suddivida l'intervallo $\left[a,b\right]$ in $M$ intervalli $I_{k}$ di ampiezza costante $H$:
\begin{equation*}
	\left[x_{k-1}, x_{k}\right] \hspace{1em} k = 1, \dots, M \hspace{3em}
	x_{k} = a + kH \hspace{1em} k = 0, \dots, M
\end{equation*}
Si introduce inoltre su ogni intervallo i punti medi:
\begin{equation*}
	\overline{x}_{k} = \dfrac{x_{k-1} + x_{k}}{2}
\end{equation*}
\begin{figure}[!htp]
	\centering
	\includegraphics[width=0.6\textwidth]{img/formule-di-quadratura-1.png}
\end{figure}

\highspace
Si consideri una \textbf{formula di quadratura} per il calcolo approssimato dell'integrale nell'equazione \ref{eq: formula di quadratura base} e sia $I_{H}\left(f\right)$ il valore approssimato ottenuto.

La \textbf{formula di quadratura} è di \textbf{ordine $p$} se l'\emph{errore} soddisfa la seguente stima:
\begin{equation}
	E_{H} = \left|I\left(f\right) - I_{H}\left(f\right)\right| \le CH^{p}
\end{equation}
Inoltre, la formula di quadratura ha \textbf{grado di esattezza pari ad $r$} se essa risulta esatta quando applicata \textbf{ai polinomi di grado minore o uguale ad $r$}, ovverosia se:
\begin{equation}
	E_{H} = \left|I\left(f\right) - I_{H}\left(f\right)\right| = 0 \hspace{2em} \forall f \in \mathbb{P}^{r} \left(a,b\right)
\end{equation}
    \subsection{Formula del punto medio composita}

Data la proprietà di additività dell'integrale:
\begin{equation*}
	I\left(f\right) = \displaystyle\sum_{k=1}^{M} \int_{I_{k}} f\left(x\right) \:\mathrm{d}x
\end{equation*}
Si possono approssimare il valore dell'integrale su ogni intervallo e dopodiché sommare i contributi.

\highspace
Per cui, l'idea del \definition{punto medio composita} è quella di approssimare il valore dell'integrale della funzione:
\begin{equation*}
	\int_{I_{k}} f\left(x\right) \:\mathrm{d}x
\end{equation*}
Con l'area del rettangolo con altezza $f\left(\overline{x}_{k}\right)$:
\begin{figure}[!htp]
	\centering
	\includegraphics[width=.4\textwidth]{img/formule-di-quadratura-2.png}
\end{figure}

\noindent
Dato che l'altezza dei rettangoli è sempre pari ad $H$, si avrà la seguente approssimazione per $I\left(f\right)$:
\begin{equation}
	I_{pm}\left(f\right) = H \displaystyle\sum_{k=1}^{M} f\left(\overline{x}_{k}\right)
\end{equation}
Dove $pm$ indica \emph{punto medio}.

\highspace
Inoltre, la \definition{stima dell'errore del punto medio composita} è:
\begin{equation}
	\left|I\left(f\right) - I_{pm}\left(f\right)\right| \le \underset{x}{\max}\left|f'' \left(x\right)\right| \: \dfrac{b-a}{24} H^{2}
\end{equation}
Si possono fare due \textbf{osservazioni} interessanti su questa stima dell'errore:\\
\begin{enumerate}
	\item La formula del \textbf{punto medio composita} è di \textbf{ordine 2}.
	
	\item La formula del \textbf{punto medio composita} ha \textbf{grado di esattezza pari a 1}. Dato che è stata assunta la continuità della derivata seconda di $f$, allora $f''=0$ in ogni $x$ se $f$ è una retta, ovverosia con $r=1$.
\end{enumerate}
Dall'ultima osservazione, risulta chiaro che la \textbf{formula del punto medio composita è esatta se applicata alle rette}.
    \subsection{Formula dei trapezi composita}

Data la proprietà di additività dell'integrale:
\begin{equation*}
	I\left(f\right) = \displaystyle\sum_{k=1}^{M} \int_{I_{k}} f\left(x\right) \:\mathrm{d}x
\end{equation*}
Si vuole approssimare l'area sottesa da $f$ nell'intervallo $I_{k}$ considerando il trapezio costruito sui punti $x_{k-1}$ e $x_{k}$ e sulle corrispondenti ordinate.

\highspace
La formula dei \definition{trapezi composita} è:
\begin{equation}
	I_{tr}\left(f\right) = \dfrac{H}{2} \displaystyle\sum_{k=1}^{M} \left(f\left(x_{k-1}\right) + f\left(x_{k}\right)\right)
\end{equation}
Per comodità di implementazione, è possibile riscriverla anche come:
\begin{equation*}
	I_{tr}\left(f\right) = \dfrac{H}{2} \left(f\left(a\right) + f\left(b\right)\right) + H \sum_{k=1}^{M-1} f\left(x_{k}\right)
\end{equation*}
\begin{figure}[!htp]
	\centering
	\includegraphics[width=.4\textwidth]{img/formule-di-quadratura-3.png}
\end{figure}

\noindent
Da notare che la formula dei trapezi composita equivale all'\textbf{integrale esatto dell'interpolatore Lagrangiano composito di grado 1}:
\begin{equation}
	I_{tr}\left(f\right) = \displaystyle \int_{a}^{b} \left(\displaystyle\prod_{1}^{H} f\left(x\right)\right) \:\mathrm{d}x
\end{equation}
Su ogni intervallo $I_{k}$ si può considerare l'errore di interpolazione Lagrangiana. 

\highspace
La \definition{stima dell'errore per la formula dei trapezi composita} è:
\begin{equation}
	\left|I\left(f\right) - I_{tr}\left(f\right)\right| \le \dfrac{1}{12} \underset{x}{\max} \left|f''\left(x\right)\right| \left(b-a\right) H^{2}
\end{equation}
Si possono fare due \textbf{osservazioni} riguardo alla stima dell'errore:
\begin{enumerate}
	\item La \textbf{formula dei trapezi composita} è di \textbf{ordine 2}.
	
	\item La \textbf{formula dei trapezi composita} ha \textbf{grado di esattezza pari a 1}.
\end{enumerate}
Da notare che sia grado che ordine sono uguali al punto medio composita. Per cui, \example{è meglio scegliere questo metodo o il punto medio?} La scelta risiede principalmente sull'\textbf{avere a disposizione le coordinate dei punti medi o dei punti estremi degli intervalli}.
    \include{sections/integrazione-numerica/formula-di-simpson-composita}
    
    %%%%%%%%%%%%%%%%%%%%%%%%%%%%%%%%%%%%%%%%%%%%%%%%%%%%%%%%%%%%%%%%%
    % Approssimazione numerica di equazioni differenziali ordinarie %
    %%%%%%%%%%%%%%%%%%%%%%%%%%%%%%%%%%%%%%%%%%%%%%%%%%%%%%%%%%%%%%%%%
    \section{Approssimazione numerica di ODE}

\subsection{Problema di Cauchy}

Un'equazione differenziale ordinaria ammette in generale infinite soluzioni. Per fissarne una è necessario imporre una condizione che prescriva il valore assunto dalla soluzione in un punto dell'intervallo di integrazione. In questa sezione ci si occuperà della risoluzione dei \definition{problemi di Cauchy}, ossia di problemi nella seguente forma.

\highspace
\begin{flushleft}
	\textcolor{Red2}{\faIcon{bookmark} \textbf{Problema di Cauchy}}
\end{flushleft}
Sia $I = \left[t_{0}, T\right]$ l'intervallo temporale di interesse. Trovare la funzione vettoriale $\mathbf{y}: I \rightarrow \mathbb{R}^{n}$ che soddisfa:
\begin{equation}
	\begin{cases}
		\mathbf{y}'\left(t\right) = \mathbf{f}\left(t, \mathbf{y}\left(t\right)\right) \\
		\mathbf{y}\left(t_{0}\right) = \mathbf{y}_{0}
	\end{cases}
\end{equation}
Con $\mathbf{f}: I \times \mathbb{R}^{n} \rightarrow \mathbb{R}^{n}$ funzione vettoriale assegnata.

\highspace
\begin{flushleft}
	\textcolor{Red2}{\faIcon{bookmark} \textbf{Problema di Cauchy \emph{scalare}}}
\end{flushleft}
Per semplicità, verrà considerato soltanto il \textbf{caso scalare}, ovvero in cui $n=1$, ma i metodi di approssimazione che verranno introdotti saranno facilmente estendibili e applicabili anche con $n>1$. Sia $I = \left[t_{0}, T\right]$ l'intervallo temporale di interesse. Trovare $y: I \subset \mathbb{R} \rightarrow \mathbb{R}$ tale che:
\begin{equation}
	\begin{cases}
		y'\left(t\right) = f\left(t, y\left(t\right)\right) \hspace{2em} \forall t \in I \\
		y\left(t_{0}\right) = y_{0}
	\end{cases}
\end{equation}
Con $f: I \times \mathbb{R} \rightarrow \mathbb{R}$ funzione assegnata.

\highspace
\begin{definitionbox}[: esistenza e unicità della soluzione continua]
	Si supponga che la funzione $f\left(t,y\right)$ sia:
	\begin{enumerate}
		\item Limitata e continua rispetto ad entrambi gli argomenti.
		
		\item Lipschitziana rispetto al secondo argomento, ossia esista una costante $L$ positiva (detta costante di Lipschitz) tale per cui:
		\begin{equation}
			\left|f\left(t, y_{1}\right) - f\left(t, y_{2}\right)\right| \le L\left|y_{1} - y_{2}\right| \hspace{2em} \forall t \in I, \:\: \forall y_{1}, y_{2} \in \mathbb{R}
		\end{equation}
	\end{enumerate}
	Allora la \textbf{soluzione} del problema di Cauchy \textbf{esiste}, è \textbf{unica} ed è di \textbf{classe} $C^{1}$ su $I$.
\end{definitionbox}
    \subsection{Approssimazione di derivate}

Al fine di approssimare il problema di Cauchy, è necessario prima introdurre un altro problema: \textbf{siano noti i valori $v\left(t_{n}\right)$ di una funzione $v$ in corrispondenza di noti $t_{n}$, $n=0, \dots, N$, e si vuole approssimare i valori della sua derivata prima $v'\left(t\right)$ negli stessi nodi}.

\highspace
\begin{flushleft}
	\textcolor{Red2}{\faIcon{bookmark} \textbf{Approssimazione in avanti della derivata prima}}
\end{flushleft}
Partendo dallo sviluppo in serie di Taylor:
\begin{equation*}
	v\left(t_{n+1}\right) = v\left(t_{n}\right) + hv'\left(t_{n}\right) + \dfrac{h^{2}}{2} v''\left(t_{n}\right) + \dfrac{h^{3}}{6} v'''\left(t_{n}\right) + O\left(h^{4}\right)
\end{equation*}
E ricordando che una quantità è un $O\left(h^{p}\right)$ se vale:
\begin{equation*}
	\lim\limits_{h \rightarrow 0} \dfrac{O\left(h^{p}\right)}{h^{p}} < +\infty
\end{equation*}
Ovverosia se va a $0$ con la stessa velocità o più velocemente di $h^{p}$. Scrivendo la derivata $v'\left(t_{n}\right)$ come:
\begin{equation*}
	v'\left(t_{n}\right) = \dfrac{v\left(t_{n+1}\right) - v\left(t_{n}\right)}{h} - \underbrace{
		\dfrac{h}{2} v''\left(t_{n}\right) - \dfrac{h^{2}}{6} v'''\left(t_{n}\right) + O\left(h^{3}\right)
	}_{O\left(h\right)}
\end{equation*}
Questo consente di introdurre un'\definition{approssimazione in avanti della derivata prima}:
\begin{equation}\label{eq: approssimazione in avanti della derivata prima}
	D^{+}v\left(t_{n}\right) = \dfrac{v\left(t_{n+1}\right) - v\left(t_{n}\right)}{h}
\end{equation}
Il cui \textbf{errore} è dato da:
\begin{equation}
	\left|v'\left(t_{n}\right) - D^{+}v\left(t_{n}\right)\right| = O\left(h\right)
\end{equation}

\highspace
\begin{flushleft}
	\textcolor{Red2}{\faIcon{bookmark} \textbf{Approssimazione all'indietro della derivata prima}}
\end{flushleft}
In modo analogo, considerando il seguente sviluppo di Taylor:
\begin{equation*}
	v\left(t_{n-1}\right) = v\left(t_{n}\right) + hv'\left(t_{n}\right) + \dfrac{h^{2}}{2} v''\left(t_{n}\right) + \dfrac{h^{3}}{6} v'''\left(t_{n}\right) + O\left(h^{4}\right)
\end{equation*}
Usando i passaggi analoghi a quelli di prima, si arriva all'\definition{approssimazione all'indietro della derivata prima}
\begin{equation}\label{eq: approssimazione all'indietro della derivata prima}
	D^{-}v\left(t_{n}\right) = \dfrac{v\left(t_{n}\right) - v\left(t_{n-1}\right)}{h}
\end{equation}
Il cui \textbf{errore} è dato da:
\begin{equation}
	\left|v'\left(t_{n}\right) - D^{-}v\left(t_{n}\right)\right| = O\left(h\right)
\end{equation}

\highspace
\begin{flushleft}
	\textcolor{Red2}{\faIcon{bookmark} \textbf{Approssimazione centrata della derivata prima}}
\end{flushleft}
Per completezza, si riporta anche l'\definition{approssimazione centrata della derivata prima}:
\begin{equation}\label{eq: approssimazione centrata della derivata prima}
	D^{c}v\left(t_{n}\right) = \dfrac{v\left(t_{n+1}\right) - v\left(t_{n-1}\right)}{2h}
\end{equation}
E in questo caso l'\textbf{errore} commesso è:
\begin{equation}
	\left|v'\left(t_{n}\right) - D^{c}v\left(t_{n}\right)\right| 0 O\left(h^{2}\right)
\end{equation}
Infine, si evince che è un metodo di ordine 2.

\highspace
\begin{flushleft}
	\textcolor{Green3}{\faIcon{question-circle} \textbf{E riguardo a Cauchy?}}
\end{flushleft}
L'approssimazione numerica del problema di Cauchy si suddivide in 4 passi:
\begin{enumerate}
	\item Si suddivide l'intervallo temporale in nodi:
	\begin{figure}[!htp]
		\centering
		\includegraphics[width=.3\textwidth]{img/approssimazione-cauchy.png}
	\end{figure}
	
	\item Si scrive il problema di Cauchy per il generico nodo $t_{n}$ valido per ogni $n = 1, \dots, N = \frac{T}{h}$:
	\begin{equation*}
		y'\left(t_{n}\right) = f\left(t_{n}, y\left(t_{n}\right)\right)
	\end{equation*}
	
	\item Si sostituisce per ogni $n$ a $y'\left(t_{n}\right)$ una delle approssimazioni presentate: in avanti (eq. \ref{eq: approssimazione in avanti della derivata prima}), all'indietro (eq. \ref{eq: approssimazione all'indietro della derivata prima}) o centrata (eq. \ref{eq: approssimazione centrata della derivata prima}).
	
	\item Si denota, per ogni $n$, con $u_{n}$ la soluzione del nuovo problema approssimato ottenuto, la quale è una candidata per essere una buona approssimazione di $y\left(t_{n}\right)$.
\end{enumerate}
La \textbf{soluzione approssimata} è composta in generale come:
\begin{equation*}
	\left\{u_{0} = y_{0}, u_{1}, u_{2}, \dots, u_{n}\right\}
\end{equation*}
    \subsection{I metodi di Eulero in avanti e all'indietro}

Dato il problema di Cauchy, esso vale per ogni $t$ nell'intervallo $I$, quindi in particolare anche in $t_{n}$:
\begin{equation*}
	y'\left(t_{n}\right) = f\left(t_{n}, y\left(t_{n}\right)\right)
\end{equation*}
Approssimando la derivata sinistra con l'approssimazione in avanti introdotta a pagina \pageref{eq: approssimazione in avanti della derivata prima}, si ottiene:
\begin{equation*}
	D^{+} y\left(t_{n}\right) = \dfrac{y\left(t_{n+1}\right) - y\left(t_{n}\right)}{h} \approxeq f\left(t_{n}, y\left(t_{n}\right)\right)
\end{equation*}
Da notare che l'espressione assomiglia ad un'uguaglianza poiché l'espressione a sinistra è un'approssimazione della derivata e di conseguenza della funzione $f$.

\highspace
L'idea è quella di introdurre come soluzione numerica $u_{n}$ con $n = 1, \dots, N$ la quale è una successione di valori che risolve in maniera esatta la precedente relazione.
\begin{equation*}
	\begin{array}{rcl}
		\dfrac{y\left(t_{n+1}\right) - y\left(t_{n}\right)}{h} &\approxeq& f\left(t_{n}, y\left(t_{n}\right)\right) \\ [1em]
		%
		&\downarrow& \\ [.5em]
		%
		\dfrac{
			u_{n+1} - u_{n}
		}{h} &=& f\left(t_{n}, u_{n}\right)
	\end{array}
\end{equation*}
Da qui si ricava comodamente il \definition{metodo di Eulero in avanti}:
\begin{equation}\label{eq: metodo di Eulero in avanti}
	u_{n+1} = u_{n} + hf\left(t_{n}, u_{n}\right) \hspace{2em} n = 0, \dots, N - 1
\end{equation}

\highspace
In maniera analoga, partendo dal problema di Cauchy:
\begin{equation*}
	y'\left(t_{n+1}\right) = f\left(t_{n+1}, y\left(t_{n+1}\right)\right)
\end{equation*}
Approssimando la derivata a sinistra con l'approssimazione all'indietro introdotto a pagina \pageref{eq: approssimazione all'indietro della derivata prima}, si ottiene:
\begin{equation*}
	D^{-} y\left(t_{n+1}\right) = \dfrac{y\left(t_{n+1}\right) - y\left(t_{n}\right)}{h} \approxeq f\left(t_{n+1}, y\left(t_{n+1}\right)\right)
\end{equation*}
Introducendo una soluzione numerica $u_{n}$ con $n = 1, \dots, N$ la quale è una successione di valori che risolve in maniera esatta la precedente relazione.
\begin{equation*}
	\begin{array}{rcl}
		\dfrac{y\left(t_{n+1}\right) - y\left(t_{n}\right)}{h} &\approxeq& f\left(t_{n+1}, y\left(t_{n+1}\right)\right) \\ [1em]
		%
		&\downarrow& \\ [.5em]
		%
		\dfrac{
			u_{n+1} - u_{n}
		}{h} &=& f\left(t_{n+1}, u_{n+1}\right)
	\end{array}
\end{equation*}
Da qui si ricava comodamente il \definition{metodo di Eulero all'indietro}:
\begin{equation}\label{eq: metodo di Eulero all'indietro}
	u_{n+1} = u_{n} + hf\left(t_{n+1}, u_{n+1}\right) \hspace{2em} n = 0, \dots, N - 1
\end{equation}

\newpage
\begin{flushleft}
	\textcolor{Green3}{\faIcon{question-circle} \textbf{Perché il metodo di Eulero in avanti viene chiamato \underline{esplicito}?}}
\end{flushleft}
Per il metodo di Eulero in avanti, la condizione iniziale $u_{0} = y_{0}$ è ben nota. Dopodiché, viene effettuato un \textbf{calcolo in sequenza}, ovverosia $u_{1}$, $u_{2}$, e così via. Tuttavia, \underline{non} è un metodo iterativo poiché la soluzione è significativa per ogni $n$, essendo l'approssimazione della $y$ a diversi istanti temporali.

\highspace
Ovviamente i due metodi calcolano la nuova $u_{n+1}$ in modo differente. Infatti, il metodo di Eulero in avanti calcola:
\begin{equation*}
	u_{n+1} = u_{n} + hf\left(t_{n}, u_{n}\right)
\end{equation*}
Il \textbf{nuovo valore è calcolato in maniera esplicita}. Difatti è sufficiente valutare la funzione $f$, anche se non lineare, per $t_{n}$, $u_{n}$ ottenendo quindi un'espressione esplicita per il termine a destra. Questo è il motivo per cui il metodo di Eulero in avanti viene chiamato anche \definition{metodo di Eulero esplicito}.

\highspace
\begin{flushleft}
	\textcolor{Green3}{\faIcon{question-circle} \textbf{Perché il metodo di Eulero all'indietro viene chiamato \underline{implicito}?}}
\end{flushleft}
Per il metodo di Eulero all'indietro:
\begin{equation*}
	u_{n+1} = u_{n} + hf\left(t_{n+1}, u_{n+1}\right)
\end{equation*}
\textbf{Non si ha un'espressione esplicita} per il nuovo valore $u_{n+1}$ \textbf{perché quest'ultimo compare anche a destra dell'uguale}!

\highspace
Si pone come incognita $x = u_{n+1}$. Se la funzione $f$ non è lineare rispetto a $y$, allora ad ogni passo temporale è necessario risolvere il \definition{problema di ricerca degli zeri} della funzione:
\begin{equation}
	g\left(x\right) = x - u_{n} - hf\left(t_{n+1}, x\right)
\end{equation}
Se la funzione $g\left(x\right)$ è derivabile, allora è necessario introdurre ad ogni istante temporale un metodo iterativo, come quello di Newton per esempio. In ogni caso, a causa del fatto che $u_{n+1}$ non può in generale essere determinato per via esplicita, il metodo di Eulero all'indietro viene chiamato anche \definition{metodo di Eulero implicito}.


\highspace
In generale, si possono \textbf{dividere i metodi numerici} per le equazioni ordinarie in \definition{metodi espliciti}, la cui \textbf{soluzione al nuovo passo temporale è calcolata mediante un'espressione esplicita}, e \definition{metodi impliciti} che invece \textbf{richiedono la soluzione di un'equazione non lineare}.

\newpage

\subsubsection{Assoluta stabilità}

\begin{flushleft}
	\textcolor{Green3}{\faIcon{question-circle} \textbf{È stabile il metodo di Eulero?}}
\end{flushleft}
Per discutere della stabilità, è necessario introdurre il concetto di \definition{assoluta stabilità}. Un metodo numerico viene detto \textbf{assolutamente stabile} per un determinato valore di $h$ maggiore di zero, se:
\begin{equation}
	\left|u_{n+1}\right| \le C_{AS}\left|u_{n}\right| \hspace{2em} C_{AS} < 1
\end{equation}
O, equivalentemente, se si è su un intervallo illimitato:
\begin{equation*}
	\lim\limits_{n \rightarrow +\infty} \left|u_{n}\right| = 0
\end{equation*}
Adesso si analizzano i metodi singolarmente.

\highspace
\begin{flushleft}\label{Assoluta stabilità: metodo di Eulero in avanti}
	\textcolor{Red2}{\faIcon{bookmark} \textbf{Assoluta stabilità: metodo di Eulero in avanti}}
\end{flushleft}
Dato il \definition{problema modello lineare}:
\begin{equation*}
	\begin{cases}
		y'\left(t\right) = \lambda y\left(t\right) \hspace{2em} \lambda < 0 \\
		y\left(0\right) = 1
	\end{cases}
\end{equation*}
La cui soluzione esatta è: $y\left(t\right) = e^{\lambda t}$. Applicando il metodo di Eulero in avanti, si ottiene:
\begin{equation*}
	u_{n+1} = u_{n} + h\lambda u_{n} = \left(1+h\lambda\right)u_{n} \longrightarrow C_{AS} = \left|1 + h\lambda\right|
\end{equation*}
Da cui l'assoluta stabilità è \textbf{garantita se}:
\begin{equation*}
	C_{AS} < 1 \longrightarrow -1 < 1 + h\lambda < 1
\end{equation*}
Ovvero $h$ deve rispettare la seguente condizione:
\begin{equation}
	h\lambda > -2 \Rightarrow h < -\dfrac{2}{\lambda}
\end{equation}
Da notare che $h\lambda < 0$ è sempre vera.

\begin{examplebox}[: stabilità metodo di Eulero in avanti]
	Data la soluzione di un problema con $\lambda = -1$, ottenuto con il metodo di Eulero in avanti, nella seguente figura si può vedere come:
	\begin{itemize}
		\item Linea tratteggiata: rappresentata $h = \frac{30}{14}$. Quindi, essendo:
		\begin{equation*}
			h < -\dfrac{2}{\lambda} \Rightarrow \dfrac{30}{14} < -\dfrac{2}{-1} \Rightarrow \dfrac{30}{14} < 2 \:\: \text{\ding{55}}
		\end{equation*}
		La soluzione numerica oscilla e infine esplode.
		
		\item Linea continua: rappresentata $h = \frac{30}{16}$. Quindi, essendo:
		\begin{equation*}
			\dfrac{30}{16} < 2 \:\: \text{\ding{51}}
		\end{equation*}
		La soluzione numerica oscilla ma non esplode. La sua oscillazione è dovuta al fatto che è molto vicina al limite $2$.
		
		\item Linea tratto-punto: rappresentata $h = \frac{1}{2}$. Quindi, essendo:
		\begin{equation*}
			\dfrac{1}{2} < 2 \:\: \text{\ding{51}}
		\end{equation*}
		La soluzione numerica non oscilla.
	\end{itemize}
	\begin{center}
		\includegraphics[width=.6\textwidth]{img/metodo-di-eulero-1.pdf}
	\end{center}
\end{examplebox}

\highspace
\begin{flushleft}
	\textcolor{Red2}{\faIcon{bookmark} \textbf{Assoluta stabilità: metodo di Eulero all'indietro}}
\end{flushleft}
Dato il solito problema modello lineare:
\begin{equation*}
	\begin{cases}
		y'\left(t\right) = \lambda y\left(t\right) \hspace{2em} \lambda < 0 \\
		y\left(0\right) = 1
	\end{cases}
\end{equation*}
La cui soluzione esatta è: $y\left(t\right) = e^{\lambda t}$. Applicando il metodo di Eulero all'indietro, si ottiene:
\begin{equation*}
	u_{n+1} = u_{n} + h\lambda u_{n+1} \rightarrow u_{n+1} = \dfrac{1}{1-h\lambda} u_{n} \rightarrow C_{AS} = \left|\dfrac{1}{1-h\lambda}\right|
\end{equation*}
In questo caso, l'\textbf{assoluta stabilità è \underline{sempre garantita}}, poiché:
\begin{equation*}
	C_{AS} < 1 \hspace{1em} \forall h
\end{equation*}
Quindi, per il metodo di Eulero all'indietro si può essere certi che è \textbf{incondizionatamente assolutamente stabile}.

\newpage

\subsubsection{Problemi di Cauchy generali}

Il concetto di stabilità introdotto nel paragrafo precedente riguarda il \emph{problema modello lineare}. Per cui, per i problemi di Cauchy, in generale, che stabilità si ha?

\highspace
\textbf{Un metodo numerico produce una soluzione stabile se a \emph{piccole} perturbazioni sul dato iniziale si producono perturbazioni \emph{piccole} e decrescenti per $n$ crescente.}

\highspace
Notando che per il \emph{problema modello lineare} si ha:
\begin{equation*}
	f\left(y\right) = \lambda y \: \rightarrow \: f'\left(y\right) = \lambda
\end{equation*}
Ha senso introdurre la \textbf{stima dell'intervallo dei valori} di $h$ che \underline{garantiscono} una \textbf{soluzione stabile} per il \textbf{\emph{metodo di Eulero in avanti}}:
\begin{equation}
	h < -\dfrac{2}{\overline{\lambda}} \hspace{3em} \overline{\lambda} = - \underset{t,y}{\max} \left|\dfrac{\partial f}{\partial y}\right| \hspace{3em} \dfrac{\partial f}{\partial y} < 0
\end{equation}
Essa rappresenta una situazione di cautela, la quale potrebbe essere molto restrittiva per il problema di Cauchy, ma sicuramente garantisce una soluzione numerica stabile.

\highspace
In generale, vale la proprietà che un \textbf{metodo numerico assolutamente stabile per il \emph{problema modello lineare}} sotto una condizione su $h$ (piccolo a sufficienza) \textbf{produce una soluzione numerica stabile anche per il problema di Cauchy generico sotto la stessa condizione opportunatamente adattata}.
    \subsection{Il metodo di Crank-Nicolson}

Dato il problema di Cauchy:
\begin{equation*}
	\begin{cases}
		y'\left(t\right) = f\left(t, y\left(t\right)\right) \hspace{2em} \forall t \in I
		y\left(t_{0}\right) = y_{0}
	\end{cases}
\end{equation*}
Dal teorema fondamentale del calcolo integrale limitato all'intervallo $\left[t_{n}, t_{n+1}\right]$ si ottiene:
\begin{equation*}
	y\left(t_{n+1}\right) = y\left(t_{n}\right) + \displaystyle\int_{t_{n}}^{t_{n+1}} f\left(t, y\left(t\right)\right) \: \mathrm{d}t
\end{equation*}
Adesso si applica una formula di quadratura a disposizione (punto medio composita, trapezi composita, Simpson composita) per approssimare l'integrale a destra dell'uguale e ottenere così un metodo di approssimazione per il problema di Cauchy.

\highspace
Ed ecco che nasce il \definition{metodo di Crank-Nicolson (CN)} applicando la formula dei trapezi composita (par. \ref{subsection: formula dei trapezi composita}, pag. \pageref{subsection: formula dei trapezi composita}) considerando solo un intervallo $\left[t_{n}, t_{n+1}\right]$:
\begin{equation}
	u_{n+1} = u_{n} + \dfrac{h}{2}\left(f\left(t_{n}, u_{n}\right) + f\left(t_{n+1}, u_{n+1}\right)\right)
\end{equation}
Esso è un \textbf{metodo implicito} e richiede ad ogni passo $n$ di determinare la radice della funzione:
\begin{equation*}
	g\left(x\right) = x - \dfrac{h}{2} f\left(t_{n+1}, x\right) - u_{n} - \dfrac{h}{2} f\left(t_{n}, u_{n}\right)
\end{equation*}

\highspace
Riguardo l'\textbf{assoluta stabilità}, si ha:
\begin{equation*}
	u_{n+1} = u_{n} + \dfrac{h \lambda}{2}\left(u_{n} + u_{n+1}\right) \: \rightarrow \: u_{n+1} = \dfrac{2+h\lambda}{2-h\lambda} u_{n}
\end{equation*}
Di conseguenza si ottiene:
\begin{equation*}
	C_{AS} = \left|\dfrac{2+h\lambda}{2-h\lambda}\right|
\end{equation*}
Che è sempre minore di 1. Per cui il metodo di Crank-Nicolson è \textbf{incondizionatamente assolutamente stabile}.
    \subsection{Il metodo di Heun}

Partendo dal metodo di Crank-Nicolson introdotto nel paragrafo \ref{subsection: il metodo di Crank-Nicolson} a pagina \pageref{subsection: il metodo di Crank-Nicolson}:
\begin{equation*}
	u_{n+1} = u_{n} + \dfrac{h}{2}\left(f\left(t_{n}, u_{n}\right) + f\left(t_{n+1}, u_{n+1}\right)\right)
\end{equation*}
E al posto della variabile $u_{n+1}$ a destra si introduce la sua approssimazione ottenuta con il metodo di Eulero in avanti, si ottiene il \definition{metodo di Heun}:
\begin{equation}
	u_{n+1} = u_{n} + \dfrac{h}{2}\left(f\left(t_{n}, u_{n}\right) + f\left(t_{n+1}, u_{n} + hf\left(t_{n}, u_{n}\right)\right)\right)
\end{equation}
Esso è un \textbf{metodo esplicito} con \textbf{condizione di assoluta stabilità uguale a quella del metodo di Eulero in avanti} (si veda a pagina \pageref{Assoluta stabilità: metodo di Eulero in avanti} l'assoluta stabilità di tale metodo).
    \subsection{Convergenza}

Per discutere l'\textbf{accuratezza della soluzione numerica} trovata da ciascun metodo introdotto per determinati valori di $h$, è necessario introdurre la definizione di \textbf{convergenza}:
\begin{equation}\label{eq: convergenza}
	\left|y\left(t_{n}\right) - u_{n}\right| = O\left(h\right) \hspace{2em} n = 0, \dots, N
\end{equation}
\begin{figure}[!htp]
	\centering
	\includegraphics[width=.4\textwidth]{img/convergenza-1.png}
\end{figure}

\noindent
Modificando la convergenza appena introdotta (eq. \ref{eq: convergenza}), essa rappresenta la \textbf{stima dell'errore di convergenza}:
\begin{equation}
	\left|y\left(t_{n}\right) - u_{n}\right| = O\left(h^{p}\right) \hspace{2em} n = 0, \dots, N
\end{equation}
Inoltre si dice che il \textbf{metodo numerico è di ordine $p$}. È interessante notare che all'\textbf{aumentare di $p$}, la \textbf{velocità di convergenza aumenta}.

\highspace
Utilizzando $\lambda$ del coefficiente del \emph{problema modello lineare}, oppure:
\begin{equation*}
	\overline{\lambda} = -\underset{t,y}{\max} \left|\dfrac{\partial f}{\partial y}\right| \hspace{2em} \text{con } \dfrac{\partial f}{\partial y} < 0
\end{equation*}
Nella tabella \ref{table: convergenza e stabilità} sono riassunti i metodi introdotti in questo capitolo.

\begin{table}[!htp]
	\centering
	\begin{tabular}{@{} c c c c @{}}
		\toprule
		\textbf{Metodo} & \textbf{Espl./Impl.} & \textbf{Convergenza} & \textbf{Assoluta stabilità} \\
		\midrule
		Eulero in avanti & Esplicito & $O\left(h\right)$ & $h < -\dfrac{2}{\lambda}$ \\
		\cmidrule{1-4}
		Eulero all'indietro & Implicito & $O\left(h\right)$ & Sempre garantita \\
		\cmidrule{1-4}
		Crank-Nicolson & Implicito & $O\left(h^{2}\right)$ & Sempre garantita \\
		\cmidrule{1-4}
		Heun & Esplicito & $O\left(h^{2}\right)$ & $h < -\dfrac{2}{\lambda}$ \\
		\bottomrule
	\end{tabular}
	\caption{Proprietà di convergenza e stabilità dei metodi per approssimare le ODE.}
	\label{table: convergenza e stabilità}
\end{table}

\newpage

\subsubsection{Consistenza dei metodi di Eulero}

Innanzitutto si introduce l'\definition{errore di troncamento locale $\tau_{n}$}, il quale si \textbf{commette introducendo la soluzione esatta} $y$ nello schema numerico. Per il \textbf{metodo di Eulero in avanti} si ha:
\begin{equation}
	\tau_{n} = \left|f\left(t_{n}, y\left(t_{n}\right)\right) - \dfrac{y\left(t_{n+1}\right)-y\left(t_{n}\right)}{h}\right|
\end{equation}
E per il \textbf{metodo di Eulero all'indietro} si ha:
\begin{equation}
	\tau_{n} = \left|f\left(t_{n+1}, y\left(t_{n+1}\right)\right) - \dfrac{y\left(t_{n+1}\right)-y\left(t_{n}\right)}{h}\right|
\end{equation}

\highspace
Un metodo numerico è \definition{consistente} se vale:
\begin{equation}
	\lim\limits_{h \rightarrow 0} \tau_{n} = 0 \hspace{2em} \forall n
\end{equation}
Inoltre, se vale:
\begin{equation}
	\tau_{n} = O\left(h^{p}\right) \hspace{2em} \forall n
\end{equation}
Si dice che l'\textbf{ordine di consistenza è $p$}. Riguardo i \textbf{\emph{metodi di Eulero}}, essi sono \textbf{consistenti del primo ordine}.

\highspace
Si noti che l'\textbf{errore di troncamento locale} $\tau_{n}$ non è l'errore necessario per verificare la convergenza: $\left|y\left(t_{n}\right) - u_{n}\right|$. L'errore:
\begin{equation*}
	e_{n} = \left|y\left(t_{n}\right) - u_{n}\right|
\end{equation*}
È dato da due contributi:
\begin{itemize}
	\item L'\textbf{errore} $e_{n}^{1}$ è \textbf{dovuto localmente al metodo numerico}, ottenuto partendo dalla vera soluzione $y\left(t_{n-1}\right)$.
	
	\item L'\textbf{errore} $e_{n}^{2}$ è \textbf{dovuto alla \emph{propagazione degli errori}} commessi agli istanti precedenti
\end{itemize}
\begin{figure}[!htp]
	\centering
	\includegraphics[width=.6\textwidth]{img/convergenza-2.png}
\end{figure}

\newpage

\noindent
Riguardo al \textbf{metodo di Eulero in avanti} l'\textbf{errore} è dato da:
\begin{itemize}
	\item Primo contributo dovuto localmente al metodo numerico:
	\begin{equation}
		e_{n}^{1} = \left|y\left(t_{n}\right) - y\left(t_{n-1}\right) - hf\left(t_{n-1}, y\left(t_{n-1}\right)\right)\right| = h\tau_{n-1}
	\end{equation}
	Da notare che il primo errore, a meno di un fattore $h$, è dato dall'errore di troncamento locale. Questo evidenzia come la \textbf{consistenza da sola non basti per avere la convergenza}.
	
	\item Secondo contributo dovuto alla propagazione degli errori:
	\begin{equation}
		e_{n}^{*} = \left|u_{n}^{*} - u_{n}\right|
	\end{equation}
\end{itemize}
Supponendo che valga la \textbf{condizione di assoluta stabilità} con $\overline{\lambda} = -\lambda_{\max}$:
\begin{equation*}
	h < \dfrac{2}{\lambda_{\max}}
\end{equation*}
Dunque si ha:
\begin{equation}
	\begin{array}{c}
		\left|e_{n}\right| \le \left|e_{n}^{1}\right| + \left|e_{n}^{2}\right| \le h\tau_{n-1} + \left|e_{n-1}\right| \\ [.5em]
		\downarrow \\ [.5em]
		\left|e_{n}\right| \le n h O\left(h\right) = \left(t_{n} - t_{0}\right)O\left(h\right)
	\end{array}
\end{equation}
Il \textbf{metodo di Eulero in avanti} è dunque \textbf{convergente} e del \textbf{primo ordine}. Lo stesso risultato vale anche per il \textbf{metodo di Eulero all'indietro}. In generale, è possibile affermare che, per un metodo convergente, l'\textbf{ordine di convergenza è uguale all'ordine di consistenza}.

\highspace
La \textbf{convergenza} è una \textbf{proprietà della soluzione numerica} (o meglio all'errore), per $n$ fissato, in ogni nodo e facendo tendere $h$ a $0$. Invece, l'assoluta stabilità analizza il comportamento della soluzione numerica per $h$ fissato e al crescere di $n$.

\highspace
In generale, i \textbf{metodi espliciti} (come Eulero in avanti e Heun) \textbf{garantiscono l'assoluta stabilità solo per un valore di $h$ piccolo}. I \textbf{metodi impliciti} godono invece di ottime proprietà di \textbf{assoluta stabilità}; spesso questa è \textbf{incondizionata} (come per Eulero all'indietro e Crank-Nicolson). 

    %%%%%%%%%%%%%%%
    % Laboratorio %
    %%%%%%%%%%%%%%%
    \section{Laboratorio}

\subsection{Introduzione al linguaggio MATLAB}

L'introduzione al linguaggio di programmazione MATLAB sarà molto rapido. Si assume dunque che l'interfaccia grafica sia familiare e che concetti base di programmazione (per esempio \dquotes{che cos'è una variabile?}) siano ben noti.

\highspace
In MATLAB, l'assegnazione di scalari a delle variabili è classica, quindi si utilizza il simbolo uguale: \texttt{a = 1} (assegnazione del valore \texttt{1} alla variabile \texttt{a}). Inoltre, il linguaggio è \emph{case sensitive}, di conseguenza la variabile \texttt{a} è diversa dalla variabile \texttt{A}. Alcuni comandi utili e generali:
\begin{itemize}
    \item \texttt{help \emph{nome-comando}}, per avere informazioni in più riguardo al comando \texttt{\emph{nome-comando}};

    \item \texttt{clear \emph{nome-variabile}}, per rimuovere la variabile \texttt{\emph{nome-variabile}} dalla memoria. Se non viene inserito il \texttt{\emph{nome-variabile}}, vengono rimosse tutte le variabili dalla memoria.

    \item \texttt{who}, per visualizzare le variabili attualmente in memoria.

    \item \texttt{clc}, per ripulire la \emph{Command Window}.
\end{itemize}

\begin{table}[!htp]
    \centering
    \begin{tabular}{@{} p{21em} l @{}}
        \toprule
        \textbf{Argomento} & \textbf{Pagina} \\
        \midrule
        Well-known variables    & Pag. \hyperlink{
            lab: Well-known variables
        }{
            \hypergetpageref{lab: Well-known variables}
        } \\
        Cambiare il formato delle variabili: \texttt{format} & Pag. 
        \hyperlink{
            lab: Cambiare il formato delle variabili: format
        }{
            \hypergetpageref{lab: Cambiare il formato delle variabili: format}
        } \\
        Assegnamento di vettori e matrici & Pag. 
        \hyperlink{
            lab: Assegnamento di vettori e matrici
        }{
            \hypergetpageref{lab: Assegnamento di vettori e matrici}
        } \\
        Operazioni su vettori e matrici & Pag. \hyperlink{
            lab: Operazioni su vettori e matrici
        }{
            \hypergetpageref{lab: Operazioni su vettori e matrici}
        } \\
        Funzioni intrinseche per vettori e matrici & Pag. \hyperlink{
            lab: Funzioni intrinseche per vettori e matrici
        }{
            \hypergetpageref{lab: Funzioni intrinseche per vettori e matrici}
        } \\
        Funzioni matematiche elementari & Pag. \hyperlink{
            lab: Funzioni matematiche elementari
        }{
            \hypergetpageref{lab: Funzioni matematiche elementari}
        } \\
        Funzioni per definire vettori o matrici particolari & Pag. \hyperlink{
            lab: Funzioni per definire vettori o matrici particolari
        }{
            \hypergetpageref{lab: Funzioni per definire vettori o matrici particolari}
        } \\
        \bottomrule
    \end{tabular}
    \caption{Argomenti trattati.}
\end{table}

\begin{flushleft}
    \large
    \hypertarget{
        lab: Well-known variables
    }{
        \textcolor{Red3}{\textbf{Well-known variables}}
    }
    \label{lab: Well-known variables}
\end{flushleft}
Esistono alcune variabili che sono ben note e hanno valori prestabiliti. Tra le più importanti:
\begin{itemize}
    \item \texttt{pi}, che rappresenta il $\pi$ e MATLAB gli assegna il valore \texttt{3.1416}
    
    \item \texttt{i}, che rappresenta l'unità immaginaria e MATLAB gli assegna il valore \texttt{0.0000 + 1.0000i}
    
    \item \texttt{eps}, che rappresenta il più piccolo valore rappresentabile nel calcolatore (PC) attualmente in uso. Solitamente, \texttt{eps} ritorna il valore \texttt{2.2204e-16}.
\end{itemize}
Questo tipo di variabili possono essere ridefinite, ma \underline{non} è una \emph{good practice}.

\newpage

\begin{flushleft}
    \large
    \hypertarget{
        lab: Cambiare il formato delle variabili: format
    }{
        \textcolor{Red3}{\textbf{Cambiare il formato delle variabili: \texttt{format}}}
    }
    \label{lab: Cambiare il formato delle variabili: format}
\end{flushleft}
Il comando \texttt{format} è utilizzato per cambiare il formato con cui sono rappresentate le variabili. MATLAB \underline{non} cambia la precisione della variabile (quindi non si ottiene una precisione maggiore dopo la virgola), ma modifica soltanto la rappresentazione. Di default MATLAB utilizza una rappresentazione di tipo \texttt{short}. Tra i più utilizzati (di default \texttt{pi} è uguale a \texttt{3.1416}):
\begin{itemize}
    \item \texttt{default} per reimpostare la rappresentazione di default.

    \item Decimale:
    \begin{itemize}
        \item \texttt{short}, rappresentazione a 5 cifre:
        \lstinputlisting[language=MATLAB]{code/introduzione-al-linguaggio-MATLAB/short.m}

        \item \texttt{long}, rappresentazione a 15 cifre:
        \lstinputlisting[language=MATLAB]{code/introduzione-al-linguaggio-MATLAB/long.m}
    \end{itemize}

    \item \emph{Floating point}:
    \begin{itemize}
        \item \texttt{short e}, rappresentazione a 5 cifre floating point:
        \lstinputlisting[language=MATLAB]{code/introduzione-al-linguaggio-MATLAB/short-floating-point.m}

        \item \texttt{long e}, rappresentazione a 15 cifre floating point:
        \lstinputlisting[language=MATLAB]{code/introduzione-al-linguaggio-MATLAB/long-floating-point.m}
    \end{itemize}
\end{itemize}
Altri formati si possono trovare nella \href{https://uk.mathworks.com/help/releases/R2024a/matlab/ref/format.html#btiwmh5-1-style}{documentazione ufficiale}.

\newpage

\begin{flushleft}
    \large
    \hypertarget{
        lab: Assegnamento di vettori e matrici
    }{
        \textcolor{Red3}{\textbf{Assegnamento di vettori e matrici}}
    }
    \label{lab: Assegnamento di vettori e matrici}
\end{flushleft}
\begin{itemize}
    \item \textbf{Vettore riga}, si può creare utilizzando uno spazio tra i valori o una virgola \texttt{,}:
    \lstinputlisting[language=MATLAB]{code/introduzione-al-linguaggio-MATLAB/vettori-1.m}

    \item \textbf{Vettore colonna}, si crea usando il punto e virgola \texttt{;}:
    \lstinputlisting[language=MATLAB]{code/introduzione-al-linguaggio-MATLAB/vettori-2.m}
\end{itemize}
Talvolta può essere utile la generazione automatica di un vettore riga (sono ammessi anche i valori negativi e con la virgola ovviamente):
\begin{itemize}
    \item \textbf{Vettore riga generato linearmente}, si crea usando i due punti e specificando il valore di inizio e il valore di fine:
    \lstinputlisting[language=MATLAB]{code/introduzione-al-linguaggio-MATLAB/vettori-3.m}

    \item \textbf{Vettore riga generato usando un passo}, si crea usando i due punti e specificando (in ordine) il valore di inizio, il \dquotes{salto}, e il valore di fine. Nel caso in cui il salto sia troppo grande e si superi il valore di fine, MATLAB prenderà il primo valore ammissibile:
    \lstinputlisting[language=MATLAB]{code/introduzione-al-linguaggio-MATLAB/vettori-4.m}

    \item \textbf{Vettore riga generato con valori uniformemente distanziati}, si crea usando la funzione \texttt{linspace}, la quale accetta tre parametri:
    \begin{itemize}
        \item \texttt{x1}, valore di partenza.
        \item \texttt{x2}, valore di fine.
        \item \texttt{n}, numero di valori da generare; se non specificato, di default è \texttt{100}; se il valore inserito è zero o minore, viene creato un vettore vuoto.
    \end{itemize}
    \lstinputlisting[language=MATLAB]{code/introduzione-al-linguaggio-MATLAB/vettori-5.m}
\end{itemize}
Le matrici possono essere create a mano o usando la combinazione delle tecniche viste in precedenza:
\begin{itemize}
    \item \textbf{Matrice}, le righe si creano usando gli spazi e le colonne si creano usando il punto e virgola:
    \lstinputlisting[language=MATLAB]{code/introduzione-al-linguaggio-MATLAB/matrici-1.m}

    \item \textbf{Matrice creata usando la generazione lineare dei vettori}, si possono utilizzare le tecniche precedenti e i punti e virgola:
    \lstinputlisting[language=MATLAB]{code/introduzione-al-linguaggio-MATLAB/matrici-2.m}
\end{itemize}

\newpage

\begin{flushleft}
    \large
    \hypertarget{
        lab: Operazioni su vettori e matrici
    }{
        \textcolor{Red3}{\textbf{Operazioni su vettori e matrici}}
    }
    \label{lab: Operazioni su vettori e matrici}
\end{flushleft}
\begin{itemize}
    \item \textbf{Trasposizione}, la classica operazione eseguita con le matrici o vettori, si esegue con la keyword \texttt{'} oppure usando la funzione \texttt{transpose}:
    \lstinputlisting[language=MATLAB]{code/introduzione-al-linguaggio-MATLAB/operazioni-vettori-matrici-1.m}

    \item \textbf{Somma e sottrazione}
    \begin{itemize}
        \item Tra \textbf{vettore e scalare}:
        \lstinputlisting[language=MATLAB]{code/introduzione-al-linguaggio-MATLAB/operazioni-vettori-matrici-2.m}

        \item Tra \textbf{vettore e matrice}:
        \lstinputlisting[language=MATLAB]{code/introduzione-al-linguaggio-MATLAB/operazioni-vettori-matrici-3.m}

        \item Tra \textbf{matrice e scalare}:
        \lstinputlisting[language=MATLAB]{code/introduzione-al-linguaggio-MATLAB/operazioni-vettori-matrici-4.m}

        \item Tra \textbf{matrice e matrice}:
        \lstinputlisting[language=MATLAB]{code/introduzione-al-linguaggio-MATLAB/operazioni-vettori-matrici-5.m}
    \end{itemize}

    \item \textbf{Prodotto}
    \begin{itemize}
        \item \textbf{Prodotto matriciale}:
        \lstinputlisting[language=MATLAB]{code/introduzione-al-linguaggio-MATLAB/operazioni-vettori-matrici-6.m}

        \item \textbf{Prodotto punto per punto}, in MATLAB è possibile moltiplicare ogni cella di una matrice (o vettore) per la corrispettiva cella della matrice (o vettore) moltiplicata. La keyword utilizzata è \texttt{.*}:
        \lstinputlisting[language=MATLAB]{code/introduzione-al-linguaggio-MATLAB/operazioni-vettori-matrici-7.m}
    \end{itemize}

    \item \textbf{Potenza}
    \begin{itemize}
        \item \textbf{Potenza matriciale}:
        \lstinputlisting[language=MATLAB]{code/introduzione-al-linguaggio-MATLAB/operazioni-vettori-matrici-8.m}

        \item \textbf{Potenza punto per punto}, come per il prodotto, è possibile elevare al quadrato ogni valore della matrice (o vettore):
        \lstinputlisting[language=MATLAB]{code/introduzione-al-linguaggio-MATLAB/operazioni-vettori-matrici-9.m}
    \end{itemize}
\end{itemize}

\newpage

\begin{flushleft}
    \large
    \hypertarget{
        lab: Funzioni intrinseche per vettori e matrici
    }{
        \textcolor{Red3}{\textbf{Funzioni intrinseche per vettori e matrici}}
    }
    \label{lab: Funzioni intrinseche per vettori e matrici}
\end{flushleft}
Qua di seguito si elencano le funzioni più importanti da utilizzare per i vettori e le matrici.
\begin{itemize}
    \item \texttt{\textbf{size}}, restituisce la dimensione del vettore o della matrice nel formato \texttt{\emph{righe} \emph{colonne}}. Specificando anche un valore (o vettore) come parametro, la funzione restituisce la dimensione (un vettore contenente le dimensioni richieste) nella \dquotes{dimensione} richiesta:
    \lstinputlisting[language=MATLAB]{code/introduzione-al-linguaggio-MATLAB/operazioni-vettori-matrici-10.m}

    \item \texttt{\textbf{length}}, restituisce la lunghezza del vettore e per le matrici restituisce il numero degli elementi per ogni riga:
    \lstinputlisting[language=MATLAB]{code/introduzione-al-linguaggio-MATLAB/operazioni-vettori-matrici-11.m}

    \item \texttt{\textbf{max}}, \texttt{\textbf{min}}, calcolano rispettivamente il massimo e il minimo valore delle componenti di un vettore; per le matrici viene presa in considerazione ogni colonna e calcolato il massimo o minimo:
    \lstinputlisting[language=MATLAB]{code/introduzione-al-linguaggio-MATLAB/operazioni-vettori-matrici-12.m}

    \item \texttt{\textbf{sum}}, \texttt{\textbf{prod}}, calcola rispettivamente la somma e il prodotto degli elementi che compongono il vettore; nel caso di una matrice, viene presa in considerazione ogni colonna e calcolata la somma o il prodotto. Inoltre, i due comandi possono prendere un argomento in più per eseguire il calcolo in una dimensione specifica (cosa sensata con le matrici):
    \lstinputlisting[language=MATLAB]{code/introduzione-al-linguaggio-MATLAB/operazioni-vettori-matrici-13.m}

    \item \texttt{\textbf{norm}}, la norma di un vettore o di una matrice. Passando un vettore o un matrice, viene calcolata di default la norma euclidea (norma 2):
    \begin{equation*}
        \left|\left| \mathrm{v} \right|\right|_{2} = \sqrt{\sum_{i = 2}^{\texttt{length(v)} \mathrm{v}_{i}^{2}}}
    \end{equation*}
    Passando un valore aggiuntivo, esso rappresenterà l'ordine della norma:
    \begin{equation*}
        \left|\left| \mathrm{v} \right|\right|_{n} = \left( \sum_{i = 2}^{\texttt{length(v)} \left| \mathrm{v}_{i} \right|^{n}} \right)^{\frac{1}{n}}
    \end{equation*}
    Infine, con \texttt{inf} viene calcolata la norma infinito:
    \begin{equation*}
        \left|\left| \mathrm{v} \right|\right|_{\infty} = \underset{1 \le i \le \texttt{length(v)}}{\max} \left| \mathrm{v}_{i} \right|
    \end{equation*}
    \lstinputlisting[language=MATLAB]{code/introduzione-al-linguaggio-MATLAB/operazioni-vettori-matrici-14.m}

    \item \texttt{\textbf{abs}}, rappresenta il valore assoluto e restituisce il vettore o matrice dopo aver applicato il valore assoluto a ciascun elemento:
    \lstinputlisting[language=MATLAB]{code/introduzione-al-linguaggio-MATLAB/operazioni-vettori-matrici-15.m}

    \newpage

    \item \texttt{\textbf{diag}}, estrae la diagonale di una matrice esistente, oppure ne crea una con i valori dati come input. Inoltre, può creare una matrice con la diagonale spostata a seconda del valore dato (si veda l'esempio):
    \lstinputlisting[language=MATLAB]{code/introduzione-al-linguaggio-MATLAB/operazioni-vettori-matrici-16.m}
\end{itemize}

\newpage

\begin{flushleft}
    \large
    \hypertarget{
        lab: Funzioni matematiche elementari
    }{
        \textcolor{Red3}{\textbf{Funzioni matematiche elementari}}
    }
    \label{lab: Funzioni matematiche elementari}
\end{flushleft}
Qua di seguito una lista di alcune funzioni matematiche elementari. Gli esempi e la sintassi non verranno mostrati poiché è sempre la medesima:
\begin{equation*}
    \texttt{\emph{funzione}(\emph{parametro})}
\end{equation*}
\begin{table}[!htp]
    \centering
    \begin{tabular}{@{} l l @{}}
        \toprule
        \textbf{Funzione} & \textbf{Comando} \\
        \midrule
        \textbf{Radice quadrata}                & \texttt{sqrt} \\
        \textbf{Esponenziale}                   & \texttt{exp} \\
        \textbf{Logaritmo Naturale}             & \texttt{log} \\
        \textbf{Logaritmo In Base \texttt{2}}   & \texttt{log2} \\
        \textbf{Logaritmo In Base \texttt{10}}  & \texttt{log10} \\
        \textbf{Seno}                           & \texttt{sin} \\
        \textbf{Arcoseno}                       & \texttt{asin} \\
        \textbf{Coseno}                         & \texttt{cos} \\
        \textbf{Arcocoseno}                     & \texttt{acos} \\
        \textbf{Tangente}                       & \texttt{tan} \\
        \bottomrule
    \end{tabular}
    \caption{Funzioni matematiche elementari.}
\end{table}

\longline

\begin{flushleft}
    \large
    \textcolor{Red3}{\textbf{Iterazione con il ciclo for}}
\end{flushleft}
In MATLAB il ciclo for viene eseguito con la seguente sintassi.
\lstinputlisting[language=MATLAB]{code/introduzione-al-linguaggio-MATLAB/for-1.m}
Di seguito si riporta un ciclo \texttt{for} che itera sulla diagonale secondaria di una matrice:
\lstinputlisting[language=MATLAB]{code/introduzione-al-linguaggio-MATLAB/for-2.m}

\newpage

\begin{flushleft}
    \large
    \hypertarget{
        lab: Funzioni per definire vettori o matrici particolari
    }{
        \textcolor{Red3}{\textbf{Funzioni per definire vettori o matrici particolari}}
    }
    \label{lab: Funzioni per definire vettori o matrici particolari}
\end{flushleft}
In queste pagine vengono presentate alcuni funzioni utili che consentono di creare matrici o vettori \dquotes{particolari}.
\begin{itemize}
    \item \textbf{Vettore/Matrice nulla}, con la funzione \texttt{zeros} è possibile creare una matrice o un vettore di tutti zeri. I parametri ammessi corrispondono alla dimensione del vettore o matrice:
    \lstinputlisting[language=MATLAB]{code/introduzione-al-linguaggio-MATLAB/vettori-matrici-particolari-1.m}

    \item \textbf{Vettore unario/Matrice unaria}, con la funzione \texttt{ones} è possibile creare una matrice o un vettore di tutti uni. I parametri ammessi corrispondono alla dimensione del vettore o matrice:
    \lstinputlisting[language=MATLAB]{code/introduzione-al-linguaggio-MATLAB/vettori-matrici-particolari-2.m}

    \newpage

    \item \textbf{Matrice identità}, con la funzione \texttt{eye} è possibile creare una matrice identità. I parametri ammessi corrispondono alla dimensione del vettore o matrice:
    \lstinputlisting[language=MATLAB]{code/introduzione-al-linguaggio-MATLAB/vettori-matrici-particolari-3.m}

    \item \textbf{Matrice/Vettore riga di numeri casuali interi e non}, con il comando \texttt{rand} si genera una matrice di numeri casuali nell'intervallo $\left[0,1\right]$ con la virgola, mentre con il comando \texttt{randi} si genera una matrice di numeri casuali interi (primo parametro deve essere specificato il range dei valori):
    \lstinputlisting[language=MATLAB]{code/introduzione-al-linguaggio-MATLAB/vettori-matrici-particolari-4.m}
\end{itemize}

\newpage

\subsubsection{Esercizio}

Creare una funzione (file) chiamato \texttt{mat\_hilbert.m} che fornisca la matrice di Hilbert avente una generica dimensione \texttt{n}. Ogni cella della matrice di Hilbert deve rispettare la seguente condizione:
\begin{equation*}
    a_{ij} = \dfrac{1}{i + j - 1}
\end{equation*}
Dopo aver creato la funzione, utilizzare la funzione nativa di MATLAB \texttt{hilb}, per verificare il risultato ottenuto.

\begin{flushleft}
    \textcolor{Green4}{\textbf{Soluzione}}
\end{flushleft}
Il codice non ha bisogno di grandi spiegazioni. Vi è un controllo iniziale per verificare l'argomento inserito dall'utente e successivamente due cicli \texttt{for} per popolare la matrice:
\lstinputlisting[language=MATLAB]{code/introduzione-al-linguaggio-MATLAB/mat_hilbert.m}
Il risultato:
\lstinputlisting[language=MATLAB]{code/introduzione-al-linguaggio-MATLAB/mat_hilbert-res.m}    
    \subsection{Zeri di funzione}

\subsubsection{Grafici di funzione}

In MATLAB una funzione $f\left(x\right)$ viene memorizzata come un vettore. In particolare, il vettore $y$ ottenuto valutando $f$ nel vettore delle ascisse $x$. Per cui la rappresentazione della funzione $f\left(x\right)$ è di fatto la rappresentazione del vettore $y$ contro il vettore $x$.

\highspace
Per introdurre i concetti di funzione e grafici di funzione, si presentano qua di seguito alcuni esempi di caso d'uso.

\highspace
\example{\emph{Definire le seguenti variabili:
\begin{itemize}
    \item $x$: vettore di estremi $0$ e $10$ con passo $0.1$
    \item $y = e^{x} + 1$
\end{itemize}}}

\noindent
Il vettore delle ascisse $x$ può essere costruito banalmente con il seguente costrutto:
\begin{lstlisting}[language=MATLAB]
x = [0 : 0.1 : 10];\end{lstlisting}
Per quanto riguarda la \textbf{funzione}, si utilizza la keyword \texttt{\@} per indicare che $f$ ha come input un valore (\texttt{x}) e rappresenta la funzione \texttt{exp(x)+1}. In questo caso, la funzione si dice anonima. Per dichiarare funzioni esplicite, si rimanda alla \href{https://it.mathworks.com/help/releases/R2024a/matlab/ref/function_handle.html}{documentazione ufficiale}.
\begin{lstlisting}[language=MATLAB]
f = @(x) exp(x) + 1\end{lstlisting}
Una volta definita una \textbf{funzione}, per \textbf{valutarla in uno o più punti}, si utilizzerà banalmente la sintassi matematica:
\begin{lstlisting}[language=MATLAB]
f(2)

ans =

    8.3891

f(0:3)

ans =

    2.0000    3.7183    8.3891   21.0855\end{lstlisting}
Da notare che se l'argomento è un vettore, allora il risultato sarà un vettore della medesima lunghezza del vettore dato in input.

\newpage

\noindent
\example{\emph{Utilizzando le variabili precedentemente definite, disegnare il grafico della funzione $y = e^{x} + 1$ nell'intervallo $\left[0, 10\right]$.}}

\highspace
Per disegnare il grafico si utilizza il comando \texttt{plot}. Di default questa funzione disegna i valori in un piano cartesiano usando segmenti rettilinei (retta spezzata):
\begin{lstlisting}[language=MATLAB]
y = f(x);
plot(x, y)\end{lstlisting}
\begin{figure}[!htp]
    \centering
    \includegraphics[width=.6\textwidth]{img/grafici-di-funzione-1.pdf}
\end{figure}

\noindent
Per evitare che MATLAB sovrascriva la figura nella finestra aperta, è possibile numerarle usando la funzione \texttt{figure} (e.g. \texttt{figure(1); plot(x,y); figure(2); plot(0:3, 0:3)}).

\highspace
La funzione \texttt{plot} accetta determinati valori per modificare il grafico finale. Nella \href{https://it.mathworks.com/help/releases/R2024a/matlab/ref/plot.html}{documentazione ufficiale} è possibile trovare l'intera lista e alcuni esempi. Scrivendo \texttt{plot(x, f(x), 'linewidth', 2)}, il parametro \texttt{'linewidth'} consente di definire lo spessore delle curve. Il valore che viene specificato in questo caso è \texttt{2} e il risultato:
\begin{figure}[!htp]
    \centering
    \includegraphics[width=.6\textwidth]{img/grafici-di-funzione-2.pdf}
\end{figure}

\newpage

\noindent
Usando il comando \texttt{hold on} per fare un confronto tra i vari grafici e invocando di nuovo la funzione \texttt{plot} ma con parametri differenti, si ottiene:
\begin{lstlisting}[language=MATLAB]
figure(1)
plot(x, f(x), 'linewidth', 2)
hold on
plot(x, -f(x), 'r', 'linewidth', 2)\end{lstlisting}

\begin{figure}[!htp]
    \centering
    \includegraphics[width=.6\textwidth]{img/grafici-di-funzione-3.pdf}
\end{figure}

\noindent
\example{\emph{Disegnare il grafico in scala semi-logaritmica (logaritmica solo per le ordinate) della funzione $y = e^{x}$ nell'intervallo $\left[0, 10\right]$. È possibile prevedere come sarà il grafico in scala semi-logaritmica della funzione $y=e^{2x}$? Verificare la risposta tracciando sulla medesima finestra le due funzioni utilizzando colori diversi per i due grafici.}}

\highspace
Per disegnare il grafico in scala semi-logaritmica (logaritmica sulle ordinate) si utilizza il comando \texttt{semilogy} e si aggiunge anche la griglia:
\begin{lstlisting}[language=MATLAB]
semilogy(x, exp(x), 'linewidth', 2)
grid on\end{lstlisting}
\begin{figure}[!htp]
    \centering
    \includegraphics[width=.6\textwidth]{img/grafici-di-funzione-4.pdf}
    \caption*{È una retta poiché $\log_{10}\left(y\right) = \log_{10}\left(e^{x}\right) = x\log_{10}\left(e\right)$.}
\end{figure}

\begin{itemize}
    \item Il comando \texttt{semilogy} è l'equivalente di \texttt{plot} ma traccia un \textbf{grafico con l'asse delle ordinate in scala logaritmica}.

    \item Il comando \texttt{semilogx} traccia un \textbf{grafico con l'asse delle ascisse logaritmico}.
    
    \item Il comando \texttt{loglog} traccia un grafico in cui entrambi gli assi sono in scala logaritmica.
\end{itemize}
Passando alla risoluzione dell'esercizio, dato che $\log_{10}\left(e^{2x}\right) = 2x\log_{10}\left(e\right)$, disegnando in scala semi-logaritmica la funzione $y = e^{2x}$, si otterrà una retta con pendenza doppia rispetto alla retta precedentemente disegnata.
\begin{lstlisting}[language=MATLAB]
hold on
semilogy(x, exp(2 * x), 'r', 'linewidth', 2)
% oppure in un solo comando senza usare hold on
% semilogy(x, exp(x), 'b', x, exp(2*x), 'r', 'linewidth', 2)
title('Grafico di exp(x) e di exp(2x)')
xlabel('Scala lineare')
ylabel('Scala logaritmica')
grid on
legend('exp(x)', 'exp(2*x)', 'Location', 'NorthWest')\end{lstlisting}
\begin{figure}[!htp]
    \centering
    \includegraphics[width=.7\textwidth]{img/grafici-di-funzione-5.pdf}
\end{figure}

\noindent
Il comando \texttt{legend} attribuisce alle curve disegnate da \texttt{plot} le stringhe di testo che gli vengono passate. Attenzione che alcune stringhe, come \texttt{'Location'} e \texttt{'NorthWest'}, vengono interpretate dalla funzione come comandi veri e propri. In questo caso si chiede di inserire una legenda in alto a sinistra.
    \subsection{Risoluzione di Sistemi di Equazioni Lineari}

\subsubsection{Metodi diretti}

Si consideri la matrice di dimensione $n \times n$:
\begin{equation*}
    A = \begin{bmatrix}
        1 & 1 & 1 & 1 & \cdots & 1 \\
        1 & -1 & 0 & 0 & \cdots & 0 \\
        0 & 1 & -1 & 0 & \cdots & 0 \\
        \vdots & 0 & \ddots & \ddots & & \vdots \\
        \vdots & \vdots & & \ddots & \ddots & \vdots \\
        0 & 0 & 0 & \cdots & 1 & -1
    \end{bmatrix}
\end{equation*}
E $\mathbf{b}$ il vettore di dimensione $n$:
\begin{equation*}
    \mathbf{b} = \left[ 2, 0, 0, \dots, 0 \right]^{T}
\end{equation*}
\begin{enumerate}
    \item Si ponga $n=20$ e si assegnino in MATLAB la matrice $A$ e il vettore dei termini noti $\mathbf{b}$.
    \lstinputlisting[language=MATLAB]{code/risoluzione-di-sistemi-di-equazioni-lineari/metodi-diretti/step1.m}
    La funzione \texttt{diag} ha un parametro particolare, \href{https://www.mathworks.com/help/releases/R2024a/matlab/ref/diag.html#bt79o5i-1-k}{vedi la documentazione}.

    
    \item Si calcoli la fattorizzazione LU della matrice $A$, mediante la funzione MATLAB \texttt{lu}. Verificare che la tecnica del pivoting non è stata usata in questo caso.
    \lstinputlisting[language=MATLAB]{code/risoluzione-di-sistemi-di-equazioni-lineari/metodi-diretti/step2.m}
    Si veda a pagina \pageref{eq: pivoting} la spiegazione della matrice di permutazione.


    \newpage


    \item Scrivere una funzione MATLAB \texttt{fwsub.m} che, dati in ingresso una matrice triangolare inferiore $L \in \mathbb{R}^{n \times n}$ e un vettore $\mathbf{f} \in \mathbb{R}^{n}$ restituisca in uscita il vettore $\mathbf{x}$, soluzione del sistema $L \mathbf{x} = \mathbf{f}$, calcolata mediante l'algoritmo della sostituzione in avanti (\emph{forward substitution}). L'intestazione della funzione sarà ad esempio: \texttt{[x] = fwsub(L, f)}.

    Analogamente, scrivere la funzione \texttt{bksub.m} che implementi l'algoritmo della sostituzione all'indietro (\emph{backward substitution}) per matrici triangolari superiori ($U$). Per controllare che le matrici $L$ e $U$ passate a \texttt{fwsub.m} e \texttt{bksub.m} siano effettivamente triangolari, è possibile utilizzare i comandi MATLAB \texttt{triu} e \texttt{tril} che, data una matrice, estraggono rispettivamente la matrice triangolare superiore e la matrice triangolare inferiore.

    Per creare una funzione, in MATLAB viene utilizzata la seguente sintassi:
\begin{lstlisting}[language=MATLAB]
function output_params = function_name(input_params)
    % Statements
end\end{lstlisting}
    Introdotta la sintassi, si introduce il codice della funzione \texttt{fwsub.m}:
    \lstinputlisting[language=MATLAB]{code/risoluzione-di-sistemi-di-equazioni-lineari/metodi-diretti/fwsub.m}
    Analogamente, si presenta il codice della funzione \texttt{bksub.m}:
    \lstinputlisting[language=MATLAB]{code/risoluzione-di-sistemi-di-equazioni-lineari/metodi-diretti/bksub.m}

    
    \item Risolvere numericamente, utilizzando le funzioni \texttt{fwsub.m} e \texttt{bksub.m} implementate al punto precedente, i due sistemi triangolari necessari per ottenere la soluzione del sistema di partenza $A\mathbf{x} =\mathbf{b}$ mediante la fattorizzazione LU.

    Si utilizza la tecnica del pivoting e l'equazione \ref{eq: pivoting - sistemi triangolari} a pagina \pageref{eq: pivoting - sistemi triangolari}:
    \lstinputlisting[language=MATLAB]{code/risoluzione-di-sistemi-di-equazioni-lineari/metodi-diretti/step3.m}


    \newpage


    \item Si calcoli la norma 2 dell'errore relativo
    \begin{equation*}
        \left|\left| \mathbf{err_{rel}} \right|\right| = \dfrac{
            \left|\left| \mathbf{x} - \widehat{\mathbf{x}} \right|\right|
        }{
            \left|\left| \mathbf{x} \right|\right|
        }
    \end{equation*}
    E la norma 2 del residuo normalizzata:
    \begin{equation*}
        \left|\left| \mathbf{r} \right|\right| = \dfrac{
            \left|\left| \mathbf{b} - A \widehat{\mathbf{x}} \right|\right|
        }{
            \left|\left| \mathbf{b} \right|\right|
        }
    \end{equation*}
    Sapendo che la soluzione esatta è il vettore di componenti:
    \begin{equation*}
        \mathbf{x}\left(i\right) = \dfrac{2}{n} \hspace{2em} i = 1, \dots, n
    \end{equation*}
    Si commenti il risultato ottenuto basandosi sul valore del numero di condizionamento della matrice $A$ (si utilizzino i comandi \texttt{norm} e \texttt{cond}).

    Il comando \texttt{norm} è stato spiegato a pagina \pageref{lab: norm}.
    \lstinputlisting[language=MATLAB]{code/risoluzione-di-sistemi-di-equazioni-lineari/metodi-diretti/step4.m}


    \item Si ripeta il punto precedente per $n = 10,20,40,80,160$. Si rappresentino su un grafico in scala semi-logaritmica gli andamenti dell'errore relativo, del residuo normalizzato (si usa dire residuo normalizzato per la norma normalizzata del residuo) e del numero di condizionamento in funzione di $n$. Commentare il grafico ottenuto.
    \lstinputlisting[language=MATLAB]{code/risoluzione-di-sistemi-di-equazioni-lineari/metodi-diretti/step5.m}

    La seguente figura mostra l'andamento dell'errore relativo, del residuo normalizzato e del numero di condizionamento in funzione di $n$, in scala semi-logaritmica. Si noti che sia il residuo normalizzato sia l'errore relativo sono molto piccoli, dall'ordine di $10^{-16}$, conseguenza del fatto che il numero di condizionamento $K\left(A\right)$ è in questo caso relativamente piccolo.

    \begin{figure}[!htp]
        \centering
        \includegraphics[width=.7\textwidth]{img/metodi-diretti.pdf}
        \caption{Andamento dell'errore relativo, del residuo normalizzato e del numero di condizionamento in funzione di $n$.}
    \end{figure}
\end{enumerate}

\newpage

\subsubsection{Metodi iterativi}

I metodi iterativi stazionari sono considerati in genere nella seguente forma:
\begin{equation*}
    \mathbf{x}^{\left(k+1\right)} = B\mathbf{x}^{\left(k\right)} + \mathbf{f} \hspace{2em} k \ge 0
\end{equation*}
Dove $B$ è detta matrice di iterazione. $B$ e $\mathbf{f}$ identificano il metodo.

\paragraph{Metodo di Jacobi}

Si consideri la matrice diagonale $D$ degli elementi diagonali di $A$. Tale matrice è facilmente invertibile, se gli $a_{ii} \ne 0, i = 1, \dots, n$, in quanto:
\begin{equation*}
    D = \begin{pmatrix}
        a_{11}  & 0     & 0     & \cdots    & 0     \\
        0       & a_{22}& 0     & \cdots    & 0     \\
        \vdots  & 0     & \ddots& \ddots    & \vdots\\
        \vdots  & \vdots& \ddots& \ddots    & 0     \\
        0       & 0     & \cdots& 0         & a_{nn}
    \end{pmatrix}
    \Longrightarrow
    D^{-1} = \begin{rowequmat}{ccccc}
        \dfrac{1}{a_{11}}   & 0                 & 0     & \cdots    & 0     \\ [.3em]
        0                   & \dfrac{1}{a_{22}} & 0     & \cdots    & 0     \\ [.3em]
        \vdots              & 0                 & \ddots& \ddots    & \vdots\\ [.3em]
        \vdots              & \vdots            & \ddots& \ddots    & 0     \\ [.3em]
        0                   & 0                 & \cdots& 0         & \dfrac{1}{a_{nn}}
    \end{rowequmat}
\end{equation*}
E il metodo può essere scritto direttamente in forma matriciale:
\begin{gather*}
    \mathbf{x}^{\left(0\right)} \text{ assegnato} \\
    \mathbf{x}^{\left(k+1\right)} = B_{J}\mathbf{x}^{\left(k\right)} + \mathbf{f}_{J} \\
\end{gather*}
Dove $B_{J} = I - D^{-1} A = D^{-1} \left(D-A\right)$ è la matrice di iterazione di Jacobi e $\mathbf{f}_{J} = D^{-1} \mathbf{b}$

\paragraph{Metodo di Gauss-Seidel}

Questo metodo si differenza dal metodo di Jacobi per il fatto che considera, oltre alla matrice $D$, anche le due matrici $-E$ e $-F$ triangolari superiore e inferiore della matrice $A$, ovvero:
\begin{equation*}
    -E = \begin{bmatrix}
        0 & 0 & 0 & \cdots &  0 \\
        a_{21} & 0 & 0 & \cdots & 0 \\
        \vdots & a_{32} & \ddots & \ddots & \vdots \\
        \vdots & \vdots & \ddots & \ddots & 0 \\
        a_{n1} & a_{n2} & \cdots & a_{nn-1} & 0
    \end{bmatrix}
    \hspace{2em}
    -F = \begin{bmatrix}
        0 & a_{12} & a_{13} & \cdots & a_{1n} \\
        0 & 0 & a_{23} & \cdots & a_{2n} \\
        \vdots & 0 & \ddots & \ddots & \vdots \\
        \vdots & \vdots & \ddots & \ddots & a_{n-1n} \\
        0 & 0 & \cdots & 0 & 0
    \end{bmatrix}
\end{equation*}
Dunque, il seguente algoritmo o le seguenti istruzioni:
\begin{gather*}
    \mathbf{x}^{\left(0\right)} \text{ assegnato} \\
    \mathbf{x}^{\left(k+1\right)} = B_{GS}\mathbf{x}^{\left(k\right)} + \mathbf{f}_{GS} \\
\end{gather*}
Dove $B_{GS} = \left(D-E\right)^{-1}F$ è la matrice d'iterazione di Gauss-Seidel e $\mathbf{f}_{GS} = \left(D-E\right)^{-1}\mathbf{b}$.

\paragraph{Esercizio}

Si considerino la matrice:
\begin{equation*}
    A = \begin{bmatrix}
        9 & -3 & 1 & & & & \\
        -3 & 9 & -3 & 1 & & & \\
        1 & -3 & 9 & -3 & 1 & & \\
        & 1 & -3 & 9 & -3 & 1 & \\
        && 1 & -3 & 9 & -3 & 1 \\
        &&& 1 & -3 & 9 & -3 \\
        &&&& 1 & -3 & 9 
    \end{bmatrix}
\end{equation*}
E il termine noto:
\begin{equation*}
    \mathbf{b} = \begin{bmatrix}
        7 & 4 & 5 & 5 & 5 & 4 & 7 
    \end{bmatrix}^{T}
\end{equation*}
\begin{enumerate}
    \item Costruire la matrice $A$ (utilizzando i comandi Matlab \texttt{diag} e \texttt{ones}) e determinare il numero di elementi non nulli tramite il comando \texttt{nnz}. La matrice $A$ è a dominanza diagonale per righe? È simmetrica e definita positiva?

    \item Si calcolino le matrici di iterazione $B_{J} = D^{-1}\left(D - A\right)$ e $B_{GS} = \left(D - E\right)^{-1}F$ associate rispettivamente ai metodi di Jacobi e Gauss-Seidel e i relativi raggi spettrali. La condizione necessaria e sufficiente per la convergenza del metodo iterativo è soddisfatta in entrambi i casi?

    \item Scrivere la funzione Matlab che implementi il metodo di Jacobi inversione \emph{matriciale} per il sistema lineare $A\mathbf{x} = \mathbf{b}$. L'intestazione della funzione sarà la seguente:
    \begin{equation*}
        \texttt{[x,k] = jacobi(A,b,x0,toll,nmax).}
    \end{equation*}
    Il processo iterativo si arresta quando:
    \begin{equation*}
        \dfrac{\left|\left|\mathbf{r}^{(k)}\right|\right|}{\left|\left|\mathbf{b}\right|\right|} \leq \texttt{toll}
    \end{equation*}
    (criterio d'arresto del residuo normalizzato).

    \item Scrivere una funzione Matlab che implementi il metodo di Gauss-Seidel inversione \emph{matriciale} per il sistema lineare $A\mathbf{x} = \mathbf{b}$. L'intestazione della funzione sarà la seguente:
    \begin{equation*}
        \texttt{[x,k] = gs(A,b,x0,toll,nmax).}
    \end{equation*}

    \item Costruire il termine noto $\mathbf{b}$. Utilizzando le funzioni costruite nei punti 3 e 4, risolvere il sistema $A\mathbf{x} = \mathbf{b}$ ponendo $x^{(0)} = \left[0,0,\dots,0\right]^{T}$, $\texttt{toll} = 10^{-6}$ e $\texttt{nmax} = 1000$. Confrontare il numero di iterazioni necessarie per arrivare a convergenza per i due metodi e commentare i risultati ottenuti.
\end{enumerate}

    \subsection{Sistema di equazioni non lineari}

\subsubsection{Metodo di Newton}

Si consideri il problema della ricerca degli zeri del sistema non lineare $\mathbf{f}\left(\mathbf{x}\right) = \mathbf{0}$, dove $\mathbf{f}: \mathbb{R}^{n} \rightarrow \mathbb{R}^{n}$. Definendo: 
\begin{itemize}
	\item $\mathbf{x} = \left(x_{1}, \dots, x_{n}\right)^{T}$
	\item $\mathbf{f} = \left(f_{1}, \dots, f_{n}\right)^{T}$ con $f_{1}, \dots, f_{n}$ funzioni $\mathbb{R}^{n} \rightarrow \mathbb{R}$
\end{itemize}
Il problema si può riscrivere nel seguente modo:
\begin{equation*}
	\begin{cases}
		f_{1}\left(x_{1}, \dots, x_{n}\right) = 0 \\
		f_{2}\left(x_{1}, \dots, x_{n}\right) = 0 \\
		\vdots \\
		f_{n}\left(x_{1}, \dots, x_{n}\right) = 0
	\end{cases}
\end{equation*}
Il metodo di Newton per sistemi non lineari è il seguente. Dato $\mathbf{x}^{\left(0\right)} \in \mathbb{R}^{n}$, per $k \ge 0$:
\begin{enumerate}
	\item Risolvere il sistema lineare:
	\begin{equation*}
		J\left(\mathbf{x}^{k}\right)\delta\mathbf{x}^{\left(k\right)} = -\mathbf{f}\left(\mathbf{x}^{\left(k\right)}\right)
	\end{equation*}
	
	\item Ponendo:
	\begin{equation*}
		\mathbf{x}^{\left(k+1\right)} = \mathbf{x}^{\left(k\right)} + \delta\mathbf{x}^{\left(k\right)}
	\end{equation*}
\end{enumerate}
Dove, per un generico punto $\mathbf{y}$, $J\left(\mathbf{y}\right)$ è la matrice Jacobiana della funzione $\mathbf{f}$ e consiste in una matrice $\mathbb{R}^{n} \times \mathbb{R}^{n}$ le cui componenti sono:
\begin{equation*}
	J_{il}\left(\mathbf{y}\right) = \dfrac{\partial f_{i}\left(\mathbf{y}\right)}{\partial y_{l}} \hspace{2em} i,l = 1, \dots, n
\end{equation*}
Si osservi che:
\begin{enumerate}
	\item L'applicazione del metodo di Newton richiede ad ogni iterazione la soluzione di un sistema lineare con matrice $A^{\left(k\right)} = J\left(\mathbf{x}^{\left(k\right)}\right)$.
	
	\item È possibile dimostrare che se:
	\begin{enumerate}
		\item $\mathbf{x}^{\left(0\right)}$ è \dquotes{sufficientemente} vicino alla soluzione $\mathbf{\alpha}$
		
		\item $J\left(\mathbf{x}^{\left(k\right)}\right)$ è una matrice non singolare con $k = 0, 1, \dots$
	\end{enumerate}
	 Allora il metodo di Newton converge con ordine 2.
\end{enumerate}

\newpage

Si consideri il sistema non lineare seguente la cui incognite è il vettore $\mathbf{x} = \left[x_{1}, x_{2}\right]$:
\begin{equation*}
	\begin{cases}
		-x_{1} + e^{3x_{2}} = 1 \\
		-x_{1} + x_{1}x_{2}^{2} = -2
	\end{cases}
\end{equation*}
\begin{enumerate}
	\item Scrivere Il sistema non lineare sotto la forma:
	\begin{equation*}
		\mathbf{f}\left(\mathbf{x}\right) = \left[f_{1}\left(\mathbf{x}\right), f_{2}\left(\mathbf{x}\right)\right]^{T} = \mathbf{0}
	\end{equation*}
	Rappresentare $\mathbf{f}$ mediante una funzione Matlab di tipo \emph{anonymous function} che, ricevuto in input un vettore colonna di 2 elementi, restituisce un vettore colonna di 2 elementi contenente la valutazione di $\mathbf{f}\left(\mathbf{x}\right)$. Analogamente, calcolare la matrice Jacobiana $J\left(\mathbf{x}\right)$ di $\mathbf{f}\left(\mathbf{x}\right)$ e scrivere la \emph{anonymous function} che restituisca tale matrice.
	
	Dunque, il sistema da risolvere è:
	\begin{equation*}
		\mathbf{f}\left(\mathbf{x}\right) = \begin{bmatrix}
			-x_{1} + e^{3x_{2}}-1 \\
			-x_{1} + x_{1}x_{2}^{2} + 2
		\end{bmatrix} = \mathbf{0}
	\end{equation*}
	Dove $\mathbf{x} = \left[x_{1}, x_{2}\right]^{T}$. La matrice Jacobiana è la seguente:
	\begin{equation*}
		J\left(\mathbf{x}\right) = \dfrac{\partial \mathbf{f}}{\partial \mathbf{x}} = \begin{bmatrix}
			-1 & 3e^{3x_{2}} \\
			-1+x_{2}^{2} & 2x_{1}x_{2}
		\end{bmatrix}
	\end{equation*}
	Si introduce il vettore $\mathbf{f}$ e la matrice $J$ utilizzando i seguenti comandi:
	\lstinputlisting[language=MATLAB]{code/risoluzione-di-sistemi-di-equazioni-non-lineari/newton_1.m}
	
	
	\item Implementare il metodo di Newton per sistemi. L'intestazione della funzione sarà ad esempio:
	\begin{center}
		\texttt{[xvect, it] = newtonsys(fun, Jf, x0, toll, nmax)}
	\end{center}
	Con ovvio significato dei parametri di ingresso e di uscita. Si utilizzi un criterio d'arresto basato sulla norma della differenza tra due iterate successive.
	
	\emph{Suggerimento}: si utilizzi il comando $\backslash$ per la risoluzione dei sistemi lineari.
	
	\lstinputlisting[language=MATLAB]{code/risoluzione-di-sistemi-di-equazioni-non-lineari/newtonsys.m}
	
	
	\item Utilizzare la funzione scritta al punto precedente per calcolare la soluzione del sitema non lineare proposto. Utilizzare una tolleranza di \texttt{toll = 1e-6}, un vettore iniziale \texttt{x0 = [1;0]}. Fornire il risultato e il numero di iterazioni effettuate.
	
	Si utilizza la funzione come richiesto.
	
	\lstinputlisting[language=MATLAB]{code/risoluzione-di-sistemi-di-equazioni-non-lineari/newton_2.m}
\end{enumerate}
    \subsection{Approssimazione di funzioni e di dati}

\subsubsection{Interpolazione Lagrangiana e Composita Lineare}

\begin{flushleft}
	\textbf{\underline{Esercizio 1}}
\end{flushleft}
Si valuti il problema dell'approssimazione della funzione di Runge:
\begin{equation*}
	f\left(x\right) = \dfrac{1}{1+x^{2}}
\end{equation*}
Mediante un'interpolazione polinomiale di Lagrange nell'intervallo $I = \left[-5, 5\right]$. Si costruiscano i polinomi interpolanti $\prod_{n}f$ e dell'errore $E_{n}f\left(x\right) = \left|f\left(x\right) - \prod_{n}f\left(x\right)\right|$ e si calcoli $\left|\left|E_{n}f\right|\right|_{\infty}$:
\begin{equation*}
	\left|\left|E_{n}f\right|\right|_{\infty} = \underset{x \in \left[-5, 5\right]}{\max} \left|f\left(x\right) - \prod_{n}f\left(x\right)\right|
\end{equation*}

\highspace
Si costruisce il polinomio interpolante $\prod_{n}f\left(x\right)$ di grado $n = 5$, $10$ della funzione $f$ considerando nodi equispaziati sull'intervallo $I$. Per ciascun valore di $n$ si vuole rappresentare graficamente $\prod_{n}f\left(x\right)$ e l'errore $E_{n}f\left(x\right) = \left|f\left(x\right) - \prod_{n}f\left(x\right)\right|$.
\begin{enumerate}
	\item Si definisce la funzione di Runge.
	\lstinputlisting[language=MATLAB]{code/approssimazione-di-funzioni-e-di-dati/interpolazione_1.m}
	
	\item Si memorizzano gli estremi dell'intervallo $I$ nelle variabili $a,b$ e il vettore contenente le ascisse degli $n+1$ nodi equispaziati.
	\lstinputlisting[language=MATLAB]{code/approssimazione-di-funzioni-e-di-dati/interpolazione_2.m}
	
	\item Si procede valutando la funzione in corrispondenza dei nodi.
	\lstinputlisting[language=MATLAB]{code/approssimazione-di-funzioni-e-di-dati/interpolazione_3.m}
	
	\item Si realizza il polinomio interpolante utilizzando le funzioni \texttt{polyfit} e \texttt{polyval}.
	\lstinputlisting[language=MATLAB]{code/approssimazione-di-funzioni-e-di-dati/interpolazione_4.m}
	
	\item Si visualizza il grafico della funzione e del polinomio interpolante $\prod_{n}f$ nell'intervallo $I$.
	\lstinputlisting[language=MATLAB]{code/approssimazione-di-funzioni-e-di-dati/interpolazione_5.m}
	\newpage
	\begin{figure}[!htp]
		\centering
		\includegraphics[width=.7\textwidth]{img/interpolazione-1.pdf}
	\end{figure}
	
	\item Infine si studia l'andamento dell'errore $E_{n}f\left(x\right)$ e lo si rappresenta graficamente.
	\lstinputlisting[language=MATLAB]{code/approssimazione-di-funzioni-e-di-dati/interpolazione_6.m}
	\begin{figure}[!htp]
		\centering
		\includegraphics[width=.7\textwidth]{img/interpolazione-2.pdf}
	\end{figure}
\end{enumerate}

\newpage

\begin{flushleft}
	\textbf{\underline{Esercizio 2}}
\end{flushleft}
Nella tabella sotto riportata vengono elencati i risultati di un esperimento eseguito per individuare il legame tra lo \emph{sforzo} $\sigma$ e la relativa \emph{deformazione} $\varepsilon$ di un campione di un tessuto biologico.
\begin{table}[!htp]
	\centering
	\begin{tabular}{@{} c c c @{}}
		\toprule
		test & $\sigma$ [MPa] & $\varepsilon$ [cm/cm] \\
		\midrule
		1 & 0.00 & 0.00 \\
		2 & 0.06 & 0.08 \\
		3 & 0.14 & 0.14 \\
		4 & 0.25 & 0.20 \\
		5 & 0.31 & 0.23 \\
		6 & 0.47 & 0.25 \\
		7 & 0.60 & 0.28 \\
		8 & 0.70 & 0.29 \\
		\bottomrule
	\end{tabular}
\end{table}

\noindent
A partire da questi dati si vuole stimare, utilizzando opportune tecniche di interpolazione, la deformazione $\varepsilon$ del tessuto in corrispondenza dei valori di sforzo per cui non si ha a disposizione un dato sperimentale. A tal fine, si considerino le seguenti funzioni interpolanti:
\begin{itemize}
	\item Interpolazione polinomiale di Lagrange (\texttt{polyfit} e \texttt{polyval});
	\item Interpolazione polinomiale composita lineare (\texttt{interp1});
\end{itemize}
In particolare, a partire dal codice assegnato si vuole:
\begin{enumerate}
	\item Rappresentare graficamente le singole funzioni interpolanti a confronto con i dati sperimentali;

	\item Confrontare in un unico grafico i dati sperimentali con tutte le funzioni interpolanti;
	
	\item Valutare, per ogni interpolante, la deformazione $\varepsilon$ in corrispondenza di $\sigma = 0.40$ MPa e $\sigma = 0.75$ MPa;
	
	\item Commentare i risultati ottenuti.
\end{enumerate}

\newpage

\noindent
La soluzione:
\begin{enumerate}
	\item Si definiscono in MATLAB i vettori componenti i dati sperimentali \texttt{sigma} (per lo sforzo $\sigma$) ed \texttt{epsilon} (per la deformazione $\varepsilon$).
	\lstinputlisting[language=MATLAB]{code/approssimazione-di-funzioni-e-di-dati/interpolazione_7.m}
	
	\item In figura viene riportato il grafico dei dati sperimentali, ottenuto con le seguenti istruzioni.
	\lstinputlisting[language=MATLAB]{code/approssimazione-di-funzioni-e-di-dati/interpolazione_8.m}
	\begin{figure}[!htp]
		\centering
		\includegraphics[width=.7\textwidth]{img/interpolazione-3.pdf}
	\end{figure}
	
	\item L'\emph{interpolazione polinomiale di Lagrange} viene realizzata mediante le funzioni MATLAB \texttt{polyfit} e \texttt{polyval}. Si ricorda che il numero di punti corrispondenti ai dati sperimentali determina (per definizione) il grado del polinomio di Lagrange, pari al numero di punti meno uno. Per disegnare il polinomio interpolatore di Lagrange si eseguono i seguenti comandi:
	\lstinputlisting[language=MATLAB]{code/approssimazione-di-funzioni-e-di-dati/interpolazione_9.m}
	\newpage
	\begin{figure}[!htp]
		\centering
		\includegraphics[width=.7\textwidth]{img/interpolazione-4.pdf}
	\end{figure}
\end{enumerate}

    %%%%%%%%%
    % Esami %
    %%%%%%%%%
    \section{Esami}

\subsection{14/06/2025}

\subsubsection*{Esercizio 1}

Si consideri la matrice $A \in \mathbb{R}^{n \times n}$ ed il vettore $\mathbf{x}^{*} \in \mathbb{R}$ tali che:
\begin{equation*}
    A = \begin{bmatrix}
        3 & -2 & 0 & 0 & \cdots & 0 \\
        -1 & 3 & -2 & 0 & \cdots & 0 \\
        0 & -1 & \ddots & \ddots & \ddots & \vdots \\
        0 & 0 & \ddots & \ddots & \ddots & 0 \\
        \vdots & \vdots & \vdots & -1 & 3 & -2 \\
        0 & 0 & \cdots & 0 & -1 & 3
    \end{bmatrix}
    \qquad
    \mathbf{x}^{*} = \begin{bmatrix}
        1 \\ 2 \\ 3 \\ \vdots \\ n - 1 \\ n
    \end{bmatrix}
\end{equation*}
\begin{enumerate}
    \item Utilizzando gli opportuni comandi MATLAB, si costruiscano la matrice $A$ ed il vettore $\mathbf{x}^{*}$ per $n = 10$. Si calcoli $\mathbf{b} \in \mathbb{R}^{10}$ tale che $\mathbf{b} = A \mathbf{x}^{*}$.

    \textcolor{Green3}{\textbf{\emph{Soluzione.}}} Il primo passaggio è quello di costruire la matrice $A$ e il vettore $\mathbf{x}^{*}$ in MATLAB. La matrice $A$ è una matrice tridiagonale con elementi specifici, mentre il vettore $\mathbf{x}^{*}$ è un vettore colonna con valori da 1 a 10. I comandi MATLAB per costruire questi oggetti sono i seguenti:
    \begin{lstlisting}[language=MATLAB]
n = 10;
A = diag(3*ones(n,1)) + diag(-1*ones(n-1,1),-1) + diag(-2*ones(n-1,1),1);
x_star = (1:n)';
b = A * x_star;
disp('Matrix A:');
disp(A);
disp('Vector x_star:');
disp(x_star);
disp('Vector b:');
disp(b);\end{lstlisting}
    \begin{lstlisting}
Matrix A:
     3    -2     0     0     0     0     0     0     0     0
    -1     3    -2     0     0     0     0     0     0     0
     0    -1     3    -2     0     0     0     0     0     0
     0     0    -1     3    -2     0     0     0     0     0
     0     0     0    -1     3    -2     0     0     0     0
     0     0     0     0    -1     3    -2     0     0     0
     0     0     0     0     0    -1     3    -2     0     0
     0     0     0     0     0     0    -1     3    -2     0
     0     0     0     0     0     0     0    -1     3    -2
     0     0     0     0     0     0     0     0    -1     3

Vector x_star:
     1
     2
     3
     4
     5
     6
     7
     8
     9
    10

Vector b:
    -1
    -1
    -1
    -1
    -1
    -1
    -1
    -1
    -1
    21\end{lstlisting}

    \item Utilizzando il comando \texttt{lu} di MATLAB, si calcoli la fattorizzazione LU della matrice $A$ precedentemente costruita. Utilizzando il comando \texttt{spy}, si determini se si è verificato il fenomeno del \emph{fill-in}. Si commenti il risultato e si riportino tutti i comandi utilizzati.

    \textcolor{Green3}{\textbf{\emph{Soluzione.}}} Per calcolare la fattorizzazione LU della matrice $A$ in MATLAB, si utilizza il comando \texttt{lu}. Successivamente, si utilizza il comando \texttt{spy} per visualizzare la struttura delle matrici $L$ e $U$ e verificare la presenza del fenomeno del \emph{fill-in}. I comandi MATLAB sono i seguenti:
    \begin{lstlisting}[language=MATLAB]
[L,U,P] = lu(A);

figure;
subplot(1, 3, 1);
spy(A);
title('Original Matrix A');

subplot(1, 3, 2);
spy(L + U - eye(n));
title('L + U - I (after LU factorization)');

subplot(1, 3, 3);
spy(L + U - eye(n) - A);
title('Fill-in (new non-zeros in L+U)');\end{lstlisting}
    Il fenomeno del \definition{fill-in} si verifica quando la fattorizzazione LU introduce nuovi elementi non nulli nelle matrici $L$ e $U$ che non erano presenti nella matrice originale $A$. Più formalmente, il \emph{fill-in} si verifica quando:
    \begin{equation*}
        \text{pattern}\left(L + U - I\right) \nsubseteq \text{pattern}\left(A\right)
    \end{equation*}
    Ovvero quando la fattorizzazione LU crea nuovi elementi non nulli in posizioni in cui $A$ aveva zeri. Quindi, la matrice corretta da analizzare per il \emph{fill-in} è $L + U - I$, dove $I$ è la matrice identità. Questo perché $L$ e $U$ contengono gli elementi della matrice originale $A$ più quelli introdotti dalla fattorizzazione. Sottraendo l'identità, si rimuovono gli elementi diagonali che sono sempre presenti in $L$ e $U$. Se la matrice risultante ha più elementi non nulli rispetto ad $A$, allora si è verificato il \emph{fill-in}.
    \begin{itemize}
        \item Con \texttt{spy(A)} si visualizza la struttura della matrice originale $A$.
        \item Con \texttt{spy(L + U - eye(n))} si visualizza la struttura della matrice risultante dalla somma di $L$ e $U$ meno l'identità. Ma attenzione, questa matrice include ancora gli elementi originali di $A$ e non mostra direttamente il \emph{fill-in}.
        \item Con \texttt{spy(L + U - eye(n) - A)} si visualizza la matrice che rappresenta il \emph{fill-in}, ossia i nuovi elementi non nulli introdotti dalla fattorizzazione LU che non erano presenti in $A$.
    \end{itemize}
    \begin{figure}[!htp]
        \centering
        \includegraphics[width=\textwidth]{img/14-06-2025/ex1_2.pdf}
    \end{figure}
    Dalla figura, si può osservare che nonostante la matrice \texttt{L + U - I} abbia una struttura identica a quella di $A$, la matrice che rappresenta il \emph{fill-in} mostra che sono stati introdotti nuovi elementi non nulli, indicando che il fenomeno del \emph{fill-in} si è verificato durante la fattorizzazione LU.


    \item Utilizzando le matrice $L$ e $U$ calcolate al punto precedente e le funzioni \texttt{fwsub.m} e \texttt{bksub.m} si risolva il sistema lineare $A\mathbf{x} = \mathbf{b}$. Si riporti il risultato ottenuto ed i comandi utilizzati.

    \textcolor{Green3}{\textbf{\emph{Soluzione.}}} Per risolvere il sistema lineare $A\mathbf{x} = \mathbf{b}$ utilizzando la fattorizzazione LU, si procede in due passi: prima si risolve il sistema $L\mathbf{y} = \mathbf{b}$ tramite sostituzione in avanti, e poi si risolve il sistema $U\mathbf{x} = \mathbf{y}$ tramite sostituzione all'indietro. I comandi MATLAB per eseguire questi passaggi sono i seguenti:
    \begin{lstlisting}[language=MATLAB]
% Risolvi Ly = b usando sostituzione in avanti
y = fwsub(L, b);
% Risolvi Ux = y usando sostituzione all'indietro
x = bksub(U, y);
disp('Soluzione x ottenuta risolvendo Ax = b:');
disp(x);\end{lstlisting}
    E il risultato ottenuto sarà:
    \begin{lstlisting}
Soluzione x ottenuta risolvendo Ax = b:
     1
     2
     3
     4
     5
     6
     7
     8
     9
    10\end{lstlisting}


    \item Si calcoli la fattorizzazione LU della matrice $B = AA^{T}$, dove $A^{T}$ è la matrice trasposta di $A$. Si determini se è stato eseguito \emph{pivoting} e si riportino i comandi utilizzati. Utilizzare le matrici calcolate al punto precedente ed il comando \texttt{\textbackslash} di MATLAB per risolvere il sistema lineare $B\mathbf{x} = \mathbf{b}$.
    
    \textcolor{Green3}{\textbf{\emph{Soluzione.}}} Per calcolare la fattorizzazione LU della matrice $B = AA^{T}$ in MATLAB, si utilizza il comando \texttt{lu}. Per verificare se è stato eseguito il \emph{pivoting}, si può controllare la matrice di permutazione $P$ restituita dalla funzione \texttt{lu}. I comandi MATLAB sono i seguenti:
    \begin{lstlisting}[language=MATLAB]
B = A * A';
[L_B, U_B, P_B] = lu(B);
disp('L factor of B:');
disp(L_B);
disp('U factor of B:');
disp(U_B);
disp('Permutation matrix P_B:');
disp(P_B);

if isequal(P_B, eye(n))
    disp('Il pivoting non e'' stato eseguito nella fattorizzazione LU di B.');
else
    disp('Il pivoting e'' stato eseguito nella fattorizzazione LU di B.');
end

x_B = B \ b;
disp('Soluzione x ottenuta risolvendo Bx = b usando \:');
disp(x_B);\end{lstlisting}
    Se il \emph{pivoting} è stato eseguito, la matrice di permutazione $P_B$ non sarà la matrice identità poiché il \emph{pivoting} comporta la riorganizzazione delle righe della matrice per migliorare la stabilità numerica della fattorizzazione LU. Quindi, se $P_B$ è diversa dalla matrice identità, significa che il \emph{pivoting} è stato eseguito. Si ricorda che la matrice di permutazione $P_B$ indica come le righe della matrice originale $B$ sono state riorganizzate durante il processo di fattorizzazione LU; gli elementi non nulli in $P_B$ indicano le posizioni delle righe originali nella matrice permutata. Per esempio, se la prima riga di $P_B$ ha un elemento non nullo nella terza colonna, significa che la terza riga di $B$ è stata spostata alla prima posizione nella matrice permutata.
    
    \begin{lstlisting}[basicstyle=\ttfamily\scriptsize,columns=fullflexible,breaklines=false]
L factor of B:
  1.0000        0        0        0        0        0        0        0        0    0
 -0.6923   1.0000        0        0        0        0        0        0        0    0
       0   0.2574   1.0000        0        0        0        0        0        0    0
       0        0  -0.2841   1.0000        0        0        0        0        0    0
  0.1538  -0.9802  -0.8847  -0.9461   1.0000        0        0        0        0    0
       0        0        0        0   0.4111   1.0000        0        0        0    0
       0        0        0        0        0  -0.3102   1.0000        0        0    0
       0        0        0        0        0        0  -0.4083   1.0000        0    0
       0        0        0        0        0        0        0  -0.4521   1.0000    0
       0        0        0  -0.3869  -0.9399  -0.7603  -0.7360  -0.7700  -0.8297  1.0

U factor of B:
 13.0  -9.0000   2.0000        0        0        0        0        0        0        0
    0   7.7692  -7.6154   2.0000        0        0        0        0        0        0
    0        0  -7.0396  13.4851  -9.0000   2.0000        0        0        0        0
    0        0        0  -5.1688  11.4430  -8.4318   2.0000        0        0        0
    0        0        0        0   4.8645  -6.2082   1.8922        0        0        0
    0        0        0        0        0  -6.4476  13.2220  -9.0000   2.0000        0
    0        0        0        0        0        0  -4.8986  11.2082  -8.3796   2.0000
    0        0        0        0        0        0        0  -4.4239  10.5788  -8.1834
    0        0        0        0        0        0        0        0  -4.2174   6.3003
    0        0        0        0        0        0        0        0        0   0.3978

Permutation matrix P_B:
    1     0     0     0     0     0     0     0     0     0
    0     1     0     0     0     0     0     0     0     0
    0     0     0     1     0     0     0     0     0     0
    0     0     0     0     1     0     0     0     0     0
    0     0     1     0     0     0     0     0     0     0
    0     0     0     0     0     0     1     0     0     0
    0     0     0     0     0     0     0     1     0     0
    0     0     0     0     0     0     0     0     1     0
    0     0     0     0     0     0     0     0     0     1
    0     0     0     0     0     1     0     0     0     0

Il pivoting e' stato eseguito nella fattorizzazione LU di B.
Soluzione x ottenuta risolvendo Bx = b usando l'operatore backslash:
     1.9624
     4.8872
     8.7367
    13.4358
    18.8339
    24.6302
    30.2228
    34.4079
    34.7782
    26.5188
\end{lstlisting}
    

    \item Definire il numero di condizionamento di una matrice $A$ e discutere la stabilità della soluzione numerica di un generico sistema lineare $A\mathbf{x} = \mathbf{b}$. Dimostrare come il residuo può essere usato come stimatore dell'errore.
    
    \textcolor{Green3}{\textbf{\emph{Soluzione.}}} 
    
    \textbf{Numero di condizionamento.} Sia $A \in \mathbb{R}^{n \times n}$ una matrice \textbf{invertibile} e sia $\left\| \cdot \right\|$ una norma matriciale indotta (ad esempio, la norma 2 o la norma infinito). Il \definition{numero di condizionamento} di $A$ rispetto alla norma $\left\| \cdot \right\|$ è definito come:
    \begin{equation*}
        \kappa(A) = \left\| A \right\| \, \left\| A^{-1} \right\|
    \end{equation*}
    Dove $\left\| A \right\|$ è la norma della matrice $A$ e $\left\| A^{-1} \right\|$ è la norma della matrice inversa di $A$. Il numero di condizionamento $\kappa(A)$ misura \textbf{quanto} la soluzione di un sistema lineare ($x$ in $A\mathbf{x} = \mathbf{b}$) può amplificare perturbazioni di dati (su $\mathbf{b}$ o su $A$). Più tale numero è grande, più il problema è \definition{mal condizionato}, il che significa che piccole variazioni nei dati di input possono causare grandi variazioni nella soluzione.

    \highspace
    \textbf{Stabilità della soluzione numerica di $A\mathbf{x} = \mathbf{b}$}
    \begin{enumerate}
        \item \textbf{Condizionamento del problema} (\emph{sensibilità ai dati}). Per studiare il condizionamento del problema, si introducono delle perturbazioni sui dati del problema.

        Se perturbiamo solo il termine noto:
        \begin{equation*}
            A\left(\mathbf{x} + \delta \mathbf{x}\right) = \mathbf{b} + \delta \mathbf{b}
        \end{equation*}
        Allora, $A \delta \mathbf{x} = \delta \mathbf{b}$, e quindi $\delta \mathbf{x} = A^{-1} \delta \mathbf{b}$. Utilizzando le norme e $\mathbf{b} = A\mathbf{x}$, si ottiene:
        \begin{equation*}
            \dfrac{\left\| \delta \mathbf{x} \right\|}{\left\| \mathbf{x} \right\|} \le \dfrac{\left\|A^{-1}\right\| \, \left\| \delta \mathbf{b} \right\|}{\left\| \mathbf{x} \right\|} \le \dfrac{\left\|A^{-1}\right\| \, \left\| \delta \mathbf{b} \right\|}{\left\| A^{-1} \right\|^{-1} \left\| \mathbf{b} \right\|} = \kappa(A) \dfrac{\left\| \delta \mathbf{b} \right\|}{\left\| \mathbf{b} \right\|}
        \end{equation*}
        Questo significa che \textbf{un piccolo errore relativo su $\mathbf{b}$} può produrre un errore relativo su $\mathbf{x}$ amplificato da un fattore pari a $\kappa(A)$.


        \item \index{stabile all'indietro} \textbf{Stabilità dell'algoritmo} (\emph{errori di arrotondamento}). Un algoritmo è \definition{backward stable} (``stabile all'indietro'') se la soluzione numerica $\hat{\mathbf{x}}$ calcolata dall'algoritmo è esattamente la soluzione di un problema ``vicino'':
        \begin{equation*}
            \left(A + \delta A\right) \hat{\mathbf{x}} = \mathbf{b} + \delta \mathbf{b}
        \end{equation*}
        Con $\dfrac{\left\| \delta A \right\|}{\left\| A \right\|}$ e $\dfrac{\left\| \delta \mathbf{b} \right\|}{\left\| \mathbf{b} \right\|}$ piccoli (dell'ordine della precisione di macchina, $O\left(\varepsilon_{\text{mach}}\right)$). Dunque, combinando il condizionamento del problema (punto precedente) con la stabilità dell'algoritmo, si ottiene che l'errore relativo sulla soluzione numerica è limitato da:
        \begin{equation*}
            \dfrac{\left\| \hat{\mathbf{x}} - \mathbf{x} \right\|}{\left\| \mathbf{x} \right\|} \approx O\left(\kappa(A)\right) \, O\left(\varepsilon_{\text{mach}}\right)
        \end{equation*}
        Che significa che l'errore relativo sulla soluzione numerica è proporzionale al numero di condizionamento della matrice $A$ e alla precisione di macchina. Quindi:
        \begin{itemize}
            \item Se $\kappa(A) \varepsilon_{\text{mach}} \ll 1$, la soluzione numerica è generalmente accurata.
            \item Se $\kappa(A) \varepsilon_{\text{mach}} \approx 1$, la soluzione numerica può essere imprecisa, con poche cifre significative. Si dice che il problema è \definition{mal condizionato}.
            \item Se $\kappa(A) \varepsilon_{\text{mach}} > 1$, anche un algoritmo stabile può dare una soluzione con pochi (o zero) cifre significative. Si dice che il problema è \definition{gravemente mal condizionato} (o \definition{fortemente mal condizionato}). In altre parole, non è un problema dell'algoritmo, ma è il problema matematico ad essere intrinsecamente instabile da risolvere numericamente.
        \end{itemize}
    \end{enumerate}
    \textbf{Il residuo come stimatore dell'errore.} Sia $\hat{\mathbf{x}}$ la soluzione numerica approssimata del sistema lineare $A\mathbf{x} = \mathbf{b}$. Il \definition{residuo} associato a questa soluzione è definito come:
    \begin{equation*}
        \mathbf{r} = \mathbf{b} - A\hat{\mathbf{x}}
    \end{equation*}
    Sia $\mathbf{x}$ la soluzione esatta del sistema lineare e sia $e = \mathbf{x} - \hat{\mathbf{x}}$ l'errore commesso nella soluzione numerica. Allora, si ha:
    \begin{equation*}
        \mathbf{r} = \mathbf{b} - A\hat{\mathbf{x}} = A\mathbf{x} - A\hat{\mathbf{x}} = A\left(\mathbf{x} - \hat{\mathbf{x}}\right) = Ae
    \end{equation*}
    Quindi, l'errore $e$ può essere espresso in termini del residuo $\mathbf{r}$ come:
    \begin{equation*}
        Ae = \mathbf{r} \implies e = A^{-1}\mathbf{r}
    \end{equation*}
    Inoltre, utilizzando le norme, si ottiene:
    \begin{equation*}
        \left\| e \right\| = \left\| A^{-1}\mathbf{r} \right\| \le \left\| A^{-1} \right\| \, \left\| \mathbf{r} \right\|
    \end{equation*}
    Dividendo entrambi i membri per $\left\| \mathbf{x} \right\|$ e utilizzando la relazione $\left\| \mathbf{b} \right\| = \left\| A\mathbf{x} \right\| \leq \left\| A \right\| \, \left\| \mathbf{x} \right\|$, si ottiene:
    \begin{equation*}
        \dfrac{\left\| e \right\|}{\left\| \mathbf{x} \right\|} \le \left\| A^{-1} \right\| \, \dfrac{\left\| \mathbf{r} \right\|}{\left\| \mathbf{x} \right\|} \le \left\| A^{-1} \right\| \, \dfrac{\left\| \mathbf{r} \right\|}{\left\| A \right\|^{-1} \, \left\| \mathbf{b} \right\|} = \kappa(A) \, \dfrac{\left\| \mathbf{r} \right\|}{\left\| \mathbf{b} \right\|}
    \end{equation*}
    Quindi \textbf{il residuo relativo} $\dfrac{\left\| \mathbf{r} \right\|}{\left\| \mathbf{b} \right\|}$, moltiplicato per il numero di condizionamento $\kappa(A)$, fornisce una stima superiore (\textbf{upper bound}) dell'errore relativo nella soluzione numerica $\hat{\mathbf{x}}$. In altre parole, \hl{se il residuo è piccolo, allora l'errore nella soluzione numerica è anche piccolo}, a meno che la matrice $A$ non sia \hl{mal condizionata} (cioè, abbia un grande numero di condizionamento).

    \begin{takeawaysbox}
        Alcuni punti chiave da ricordare:
        \begin{itemize}
            \item \textbf{Residuo piccolo} non implica necessariamente un \textbf{errore piccolo} se la matrice è \textbf{mal condizionata} ($\kappa(A)$ grande).
            \item Se $A$ è \textbf{ben condizionata} (cioè, $\kappa(A)$ piccolo), allora un \textbf{residuo piccolo} garantisce un \textbf{errore piccolo}, quindi è un buon indicatore di accuratezza.
            \item In pratica si usa spesso anche il \textbf{residuo normalizzato}:
            \begin{equation*}
                \dfrac{\left\| \mathbf{r} \right\|}{\left\| A \right\| \, \left\| \hat{\mathbf{x}} \right\| + \left\| \mathbf{b} \right\|}
            \end{equation*}
            Che è un indicatore di backward error (errore all'indietro) più robusto. Dopodiché, moltiplicando questo residuo normalizzato per $\kappa(A)$, si ottiene l'errore forward (errore in avanti) stimato.
        \end{itemize}
    \end{takeawaysbox}

    \item Calcolare il condizionamento della matrice $B$ ed il residuo per fornire una stima dell'errore relativo commesso nella risoluzione del sistema lineare $B\mathbf{x} = \mathbf{b}$. Riportare i comandi utilizzati.

    \textcolor{Green3}{\textbf{\emph{Soluzione.}}} Per calcolare il condizionamento della matrice $B$ e il residuo associato alla soluzione numerica ottenuta, si utilizzano i seguenti comandi MATLAB:
    \begin{lstlisting}[language=MATLAB]
cond_B = cond(B);
r = b - B * x_B;
disp(['Condizionamento della matrice B: ', num2str(cond_B)]);
disp('Residuo r = b - Bx:');
disp(r);
norm_r = norm(r);
norm_B = norm(B);
norm_x_B = norm(x_B);
norm_b = norm(b);
residuo_normalizzato = norm_r / (norm_B * norm_x_B + norm_b);
disp([ ...
    'Residuo normalizzato: ', ...
    num2str(residuo_normalizzato) ...
]);
errore_stimato = cond_B * residuo_normalizzato;
disp([ ...
    'Stima dell''errore relativo: ', ...
    num2str(errore_stimato) ...
]);\end{lstlisting}
    E il risultato ottenuto sarà:
    \begin{lstlisting}
Condizionamento della matrice B: 911.3637
Residuo r = b - Bx:
   1.0e-13 *

         0
    0.0355
   -0.0711
   -0.0711
   -0.2842
   -0.4263
   -0.2842
    0.1421
         0
         0

Residuo normalizzato: 2.4185e-17
Stima dell'errore relativo: 2.2042e-14\end{lstlisting}
    Il sistema lineare $B\mathbf{x} = \mathbf{b}$ presenta un numero di condizionamento di circa $911.36$, indicando che il problema è moderatamente mal condizionato ($\kappa(B) \approx 9.11 \times 10^{2}$). Tuttavia, calcolando il residuo normalizzato, si ottiene un valore molto piccolo di circa $2.42 \times 10^{-17}$. Moltiplicando questo residuo per il condizionamento della matrice $B$, si ottiene una stima dell'errore relativo commesso nella risoluzione del sistema lineare, che risulta essere circa $2.20 \times 10^{-14}$. Questo indica che, nonostante il problema sia moderatamente mal condizionato, l'errore relativo nella soluzione numerica è ancora molto piccolo, suggerendo che la soluzione ottenuta è affidabile. Infatti, calcolando il prodotto $\kappa(B) \cdot \varepsilon_{\text{mach}}$ (dove $\varepsilon_{\text{mach}} \approx 2.22 \times 10^{-16}$ per la precisione doppia), si ottiene un valore di circa $2.02 \times 10^{-13}$, che è ancora molto piccolo, confermando che la soluzione numerica è accurata nonostante il condizionamento della matrice.
\end{enumerate}

\newpage

\subsubsection*{Esercizio 2}

Si consideri la funzione nonlineare:
\begin{equation*}
    f(x) = \left(x - 2\right) e^{\left(x-1\right)}
\end{equation*}
Dotata dello zero $\alpha = 2$.
\begin{enumerate}
    \item Si scriva il metodo di Newton per la determinazione numerica delle radici di una funzione $f \, : \, \left[a,b\right] \to \mathbb{R}$ come metodo di iterazioni di punto di fisso e si scrivano con precisione le condizioni per la convergenza, specificandone l'ordine. Si definisca un criterio d'arresto affidabile dandone motivazione.
    
    \item Si verifichi graficamente che $\alpha = 2$ è una radice per la funzione $f(x)$. Si riportino tutti i comandi MATLAB usati.
    
    \item \phantomsection\label{itm:es2.3} Si consideri il metodo delle iterazioni di punto fisso per l'approssimazione dello zero $\alpha$ di $f(x)$ usando la funzione di iterazione
    \begin{equation*}
        \phi = x - \dfrac{\left(x-2\right) e^{\left(x-2\right)}}{2e-1}
    \end{equation*}
    Si verifichi graficamente che lo zero $\alpha$ di $f(x)$ coincide con un punto fisso di $\phi(x)$ e si riportino i comandi MATLAB usati.
    
    \item \phantomsection\label{itm:es2.4} Sempre considerando il metodo delle iterazioni di punto fisso e la funzione di iterazione (equazione al punto \ref{itm:es2.3}), si determini l'ordine di convergenza del metodo ad $\alpha$ per un'iterata iniziale $x^{(0)}$ ``sufficientemente'' vicino ad $\alpha$. Si motivi la risposta data alla luce della teoria.
    
    \item Considerando la funzione di iterazione $\phi(x)$ al punto \ref{itm:es2.3}, si utilizzi la funzione MATLAB \texttt{ptofis.m} con $x^{(0)} = 1.5$, tolleranza $tol = 10^{-4}$ e numero massimo di iterazioni $N_{max} = 1000$. Si riportino il numero di iterazioni $N$, i valori delle iterate $x^{(1)}$ e $x^{(2)}$, la differenza tra iterate successive $\left| x^{(N)} - x^{(N-1)} \right|$, insieme a tutti i comandi MATLAB utilizzati.
    
    \item Si utilizzi opportunamente la funzione MATLAB \texttt{stimap.m} per stimare l'ordine di convergenza. Si discutano criticamente i risultati ottenuti con quelli relativi al punto \ref{itm:es2.4} e si riportino i comandi MATLAB utilizzati.
\end{enumerate}

\newpage

\subsubsection*{Esercizio 3}

Data una funzione $f \in \mathcal{C}^{0} \left(\left[0, 1\right]\right)$, si consideri il problema di approssimazione numerica dell'integrale:
\begin{equation*}
    I(f) = \int_{0}^{1} f(x) \, dx
\end{equation*}
Mediante una formula di quadratura composita con $n$ sottointervalli equispaziati di $\left[0, 1\right]$.
\begin{enumerate}
    \item Si introduca, in generale, il problema dell'approssimazione numerica dell'integrale sopra definito, tramite formule di quadrature sia semplici che composite, definendo con precisione tutta la notazione utilizzata. Si introducano i concetti di ordine di convergenza e grado di esattezza.
    
    \item Sia ora $f(x) = \left(1 - 2x\right)e^{-2x}$. Utilizzando la funzione \texttt{quadcomp.m}, si calcoli l'integrale approssimato $I_{1}(f)$ corrispondente alla formula di quadratura di tipo semplice (ossia ottenuta con $n = 1$). Riportare il risultato ottenuto e i comandi MATLAB usati.

    \item \phantomsection\label{itm:es3.3} Verificare il grado di esattezza della formula di quadratura implementata in \texttt{quadcomp.m} calcolando gli errori $E_{i} = \left|I\left(p_i\right) - I_1 (p_i)\right|$ dove $p_i(x) = x^{i}$ per $i = 0, \dots, 4$. Riportare i comandi utilizzati e commentare i risultati ottenuti.

    \item Utilizzando la funzione \texttt{quadcomp.m}, si calcoli l'integrale approssimato $I_{n}(f)$ e, sapendo che $I(f) = e^{-2}$, si calcoli l'errore commesso $E_n (f) = \left| I(f) - I_n (f) \right|$ per $n = 5, 10, 20, 40$. Riportare i risultati ottenuti nella forma esponenziale e i comandi MATLAB usati.
    
    \item Riportare su un grafico in scala logaritmica l'andamento dell'errore in funzione di $h = 1 / n$ confrontandolo con l'andamento $h^{s}$ per un opportuno $s > 0$. Si riportino tutti i comandi utilizzati. Dedurre l'ordine di convergenza della formula di quadratura e commentare il risultato alla luce di quanto ottenuto al punto \ref{itm:es3.3}.
\end{enumerate}
    \section{Esami}

\subsection{24/07/2025}

\subsubsection*{Esercizio 1}

Si consideri la matrice $A \in \mathbb{R}^{n \times n}$ ed il vettore $\mathbf{b} \in \mathbb{R}^{n}$ tali che:
\begin{equation*}
    A = \begin{bmatrix}
        2 & -1 & 0 & 0 & \cdots & 0 \\
        -1 & 2 & -1 & 0 & \cdots & 0 \\
        0 & -1 & \ddots & \ddots & \ddots & \vdots \\
        0 & 0 & \ddots & \ddots & \ddots & 0 \\
        \vdots & \vdots & \vdots & -1 & 2 & -1 \\
        0 & 0 & \cdots & 0 & -1 & 2 \\
    \end{bmatrix}
    \hspace{3em}
    \mathbf{b} = \begin{bmatrix}
        -1 \\
        -1 \\
        -1 \\
        \vdots \\
        -1 \\
        -1 \\
    \end{bmatrix}
\end{equation*}
\begin{enumerate}
    \item Utilizzando gli opportuni comandi Matlab, si costruiscano la matrice $A$ ed il vettore $\mathbf{b}$ per $n = 7$. Utilizzando il comando \textbackslash si risolva il sistema lineare $A\mathbf{x} = \mathbf{b}$. Riportare $\mathbf{x}$ e tutti i comandi Matlab usati.
    
    \textcolor{Green3}{\textbf{\emph{Soluzione}}}
    \begin{lstlisting}[language=MATLAB]
% dimensione di n
n = 7;
% creiamo un vettore di 1 e aggiungiamo 1
A = diag(ones(n,1)+1) + ...      % diag. principale
    diag(ones(n-1,1)-2, 1) + ... % diag. sopra principale
    diag(ones(n-1,1)-2, -1);     % diag. sotto principale
% creiamo vettore b
b = -ones(n,1);
% calcola soluzione di x
x = A\b;\end{lstlisting}
Il vettore \texttt{x} produce il seguente risultato:
\begin{equation*}
    \mathbf{x} = \begin{bmatrix}
        -3.5 \\
        -6 \\
        -7.5 \\
        -8 \\
        -7.5 \\
        -6 \\
        -3.5
    \end{bmatrix}
\end{equation*}

    \item Elencare e discutere:
    \begin{enumerate}
        \item Le condizioni sufficienti per la convergenza del metodo di Gauss-Seidel.
        \item Le condizioni necessarie e sufficienti per la convergenza del metodo di Gauss-Seidel.
    \end{enumerate}
    Data $B_{GS}$ la matrice di iterazione di Gauss-Seidel, dimostrare che\break $\left\| B_{GS} \right\| < 1$ implica la convergenza.

    \textcolor{Green3}{\textbf{\emph{Soluzione.}}} Le condizioni sufficienti per la convergenza del metodo di Gauss-Seidel sono 3:
    \begin{itemize}
        \item La matrice $A$ deve essere a \textbf{dominanza diagonale stretta per righe}:
        \begin{equation*}
            \left|a_{ii}\right| > \displaystyle\sum_{j \ne i} \left|a_{ij}\right| \hspace{1em} \forall i
        \end{equation*}
        \item La matrice $A$ deve essere \textbf{simmetrica definita positiva (SPD)}:
        \begin{equation*}
            \mathbf{x}^{T} A \mathbf{x} > 0 \hspace{1em} \forall \mathbf{x} \ne 0
        \end{equation*}
        \item La matrice $A$ deve essere una \textbf{Matrice di classe M (M-Matrice) non singolare}. Ovvero una matrice con diagonale positiva e elementi extradiagonali (non sulla diagonale) non positivi.
    \end{itemize}
    L'unica condizione necessarie e sufficiente per la convergenza è quando il raggio spettrale della matrice di iterazione $B_{GS}$ è minore di uno:
    \begin{equation*}
        \text{convergenza} \iff \rho(B_{GS}) < 1
    \end{equation*}
    \begin{proof}[Dimostrazione ($\left\| B_{GS} \right\| < 1$ implica la convergenza)]
        Il metodo di Gauss-Seidel si scrive
        \begin{equation*}
            x^{(k+1)} = B_{GS}x^{(k)} + c
        \end{equation*}
        Definendo l'errore $e^{(k)} = x^{(k)} - x^{*}$, si ha
        \begin{equation*}
            e^{(k+1)} = B_{GS} e^{(k)} \quad \Rightarrow \quad e^{(k)} = B_{GS}^k e^{(0)}
        \end{equation*}
        Se $\|B_{GS}\| < 1$, allora:
        \begin{equation*}
            \|e^{(k)}\| \le \|B_{GS}\|^k \|e^{(0)}\| \to 0,
        \end{equation*}
        Per $k \to \infty$. Quindi $x^{(k)} \to x^\ast$: il metodo converge.
    \end{proof}


    \item Verificare se la matrice $A$ soddisfa le condizioni precedenti riportate. Motivare le risposte e riportare i comandi utilizzati.

    \textcolor{Green3}{\textbf{\emph{Soluzione.}}}
    \begin{lstlisting}[language=MATLAB]
first_cond = false;
for row = 1:n
    first_cond = abs(A(row, row)) > (abs(sum(A(row, 1:n))) - abs(A(row, row)));
    if first_cond == 0
        disp("La matrice NON e' a dominanza diagonale stretta per righe")
        break
    end
end

if first_cond
    disp("La matrice e' a dominanza diagonale stretta per righe")
end

% dimostrare che e' simmetrica e definita positiva
second_cond = issymmetric(A) && all(eig(A) > 0);
if second_cond
    disp("La matrice e' simmetrica definita positiva")
else
    disp("La matrice NON e' simmetrica definita positiva")
end

% dimostrare che e' una M-Matrice non singolare
% 1. Extradiagonali non positivi
n = size(A, 1);
% creiamo la matrice con solo la diagonale principale
% diag(diag(A)) e sottraiamo alla matrice A la diagonale
% principale per ottenere la matrice con solo le
% extradiagonali
offdiag = A - diag(diag(A));
% verifichiamo che tutte le extradiagonali siano <= 0
first_m_cond = all(offdiag(:) <= 0);
if first_m_cond == false
    disp("La matrice NON e' una M-matrice: extradiagonali positive")
end
% 2. Diagonale principale positiva
second_m_cond = all(diag(A) > 0);
if second_m_cond == false
    disp("La matrice NON e' una M-matrice: diagonale principale non positiva")
end
% 3. A e' non singolare, cioe' rango(A) = n
% si ricorda che una matrice e' singolare se det(A) = 0
% e che rango(A) = n se e solo se det(A) != 0
% perche' il determinante e' il prodotto degli autovalori
% e una matrice e' singolare se ha almeno un autovalore nullo
third_m_cond = (rank(A) == n);
if third_m_cond == false
    disp("La matrice NON e' una M-matrice: matrice singolare")
end

if first_m_cond && second_m_cond && third_m_cond
    disp("La matrice e' una M-matrice non singolare")
end\end{lstlisting}
    Come risultato finale, si ottiene che la matrice \texttt{A} rispetta le 3 condizioni sufficienti per convergere usando il metodo di Gauss-Seidel.


    \item \label{item: punto d - 24-07-2025} Usando la funzione \texttt{gs.m} si risolva il sistema lineare $Ax = b$ con il metodo iterativo di Gauss-Seidel per $x^{(0)} = \left(0, 0, 0, 0, 0, 0, 0\right)^{T}$, tolleranza $tol = 10^{-7}$ e massimo numero di iterazioni $N_{\max} = 700$. Indicata con $x_{GS}$ la soluzione ottenuta approssimata ottenuta, si riportino: il numero di iterazioni effettuate, l'errore $\left\| x_{GS} - x \right\|$ e tutti i comandi Matlab utilizzati.

    \textcolor{Green3}{\textbf{\emph{Soluzione.}}}
    \begin{lstlisting}[language=MATLAB]
x0 = zeros(n, 1);
tol = 1e-7;
n_max = 700;
[x_gs, k] = gs(A, b, x0, tol, n_max);
disp("Soluzione con Gauss-Seidel:")
disp(x_gs)
disp("Numero di iterazioni:")
disp(k)
% errore x_gs - x
err_gs = norm(x_gs - x);
disp("Errore tra soluzione esatta e soluzione con Gauss-Seidel:")
disp(err_gs)\end{lstlisting}
    Output:
    \begin{lstlisting}
Soluzione con Gauss-Seidel:
   -3.5000
   -6.0000
   -7.5000
   -8.0000
   -7.5000
   -6.0000
   -3.5000

Numero di iterazioni:
   103

Errore tra soluzione esatta e soluzione con Gauss-Seidel:
   1.4486e-06\end{lstlisting}


    \item Si riportino il valore del parametro $\alpha$ che massimizza la velocità di convergenza del metodo di Richardson stazionario ed il corrispondente raggio spettrale della matrice di iterazione.
   
    \textcolor{Green3}{\textbf{\emph{Soluzione.}}} Da definizione, il parametro $\alpha_{\mathrm{opt}}$ che massimizza la velocità di convergenza del metodo di Richardson è:
    \begin{equation*}
        \alpha_{\mathrm{opt}} = \dfrac{2}{\lambda_{\min}(A) + \lambda_{\max}(A)}
    \end{equation*}
    Dove $\lambda$ rappresenta il vettore degli autovalori della matrice $A$. Invece, il raggio spettrale della matrice di iterazione corrispondente al parametro $\alpha_{\mathrm{opt}}$ é:
    \begin{equation*}
        \rho\left(I - \alpha_{\mathrm{opt}}A\right) = \dfrac{
            \lambda_{\max}(A) - \lambda_{\min}(A)
        }{
            \lambda_{\max}(A) + \lambda_{\min}(A)
        }
    \end{equation*}
    Implementato quanto detto su MATLAB, il codice è:
    \begin{lstlisting}[language=MATLAB]
eigen_values = eig(A);
lambda_min = min(eigen_values);
lambda_max = max(eigen_values);
alpha_opt = 2 / (lambda_min + lambda_max);
disp("Parametro alpha che massimizza la velocita' di convergenza: " + alpha_opt)

rho = (lambda_max - lambda_min) / (lambda_max + lambda_min);
disp("Raggio spettrale del metodo di Richardson: " + rho);\end{lstlisting}
    Output:
    \begin{lstlisting}
Alpha che massimizza la velocita' di convergenza: 0.5
Raggio spettrale del metodo di Richardson: 0.92388\end{lstlisting}


    \item Si ripeta il punto \ref{item: punto d - 24-07-2025} usando la funzione \texttt{richardson.m} per applicare il metodo iterativo di Richardson stazionario alla soluzione del sistema lineare con parametro $\alpha_{\mathrm{opt}}$ calcolato al punto precedente. Si riportino la soluzione approssimata ottenuta ed il numero di iterazioni effettuate. Sulla base dei risultati ottenuti quale dei due metodi converge più velocemente?
    
    \textcolor{Green3}{\textbf{\emph{Soluzione.}}} Al punto \ref{item: punto d - 24-07-2025}, il metodo di iterazione aveva le seguenti caratteristiche:
    \begin{itemize}
        \item $x^{(0)} = \left(0, 0, 0, 0, 0, 0, 0\right)^{T}$
        \item $tol = 10^{-7}$
        \item $N_{\max} = 700$
    \end{itemize}
    Quindi, il codice MATLAB:
    \begin{lstlisting}[language=MATLAB]
x0 = zeros(n, 1);
tol = 1e-7;
n_max = 700;
[x_rich, k_rich, err_rich] = richardson(A, b, x0, alpha_opt, tol, n_max);
disp("Soluzione con Richardson:")
disp(x_rich)
disp("Numero di iterazioni: " + k_rich)
% errore x_rich - x
err_rich_final = norm(x_rich - x);
disp("Errore tra soluzione esatta e soluzione con Richardson: " + err_rich_final)\end{lstlisting}
    Output:
    \begin{lstlisting}
Soluzione con Richardson:
   -3.5000
   -6.0000
   -7.5000
   -8.0000
   -7.5000
   -6.0000
   -3.5000

Numero di iterazioni: 203
Errore tra soluzione esatta e soluzione con Richardson: 1.7286e-06\end{lstlisting}
    Il metodo di Gauss-Seidel ha impiegato la metà del numero di iterazioni rispetto al metodo di Richardson (103 contro 203), dimostrando che il metodo di iterazione che converge più rapidamente è Gauss-Seidel.
\end{enumerate}

\newpage

\subsubsection*{Esercizio 2}

Sia $f(x) = \log(5x^{2} + 1)$ definita nell'intervallo $I = \left[-2, 2\right]$. Si vuole approssimare $f$ con un polinomio interpolante $\Pi_{n} f$ di grado $n$ su $n+1$ nodi $x_{i}$ definiti su $I$. Siano inoltre $\left\{\overline{x}_{j}, \, j = 1, \dots, 701 \right\}$ 701 punti equispaziati su $I$ (estremi inclusi), per il calcolo dell'errore di interpolazione.

\begin{remarkbox}
    Prima di continuare con l'esercizio, è importante ricordare e capire il testo.
    \begin{itemize}
        \item Cosa si intende con ``approssimare $f$ con un polinomio interpolante $\prod_{n} f$ di grado $n$ su $n+1$ nodi $x_{i}$ definiti su $I$''? Vuol dire che:
        \begin{itemize}
            \item Si scelgono $n+1$ \textbf{nodi} (cioè punti distinti nell'intervallo $I = [-2, 2]$), chiamiamoli $x_0, x_1, \dots, x_n$.
            \item Si calcolano i valori della funzione in quei nodi: $f(x_0), f(x_1),\break \dots, f(x_n)$.
            \item Esiste ed è unico un \textbf{polinomio di grado al più} $n$ che ``passa'' per tutti questi punti, ovvero:
            \begin{equation*}
                \Pi_{n} f(x_i) = f(x_i), \quad i = 0, 1, \dots, n
            \end{equation*}
            Questo polinomio si chiama \definition{Polinomio Interpolante} di $f$ nei nodi dati.
        \end{itemize}
        Quindi, in altre parole, $\Pi_{n} f$ non è altro che una funzione polinomiale che coincide con $f$ in un insieme finito di punti, ed è usata per approssimare $f$ anche altrove nell'intervallo.

        \highspace
        Nota: si dice ``polinomio interpolante'' perché il polinomio \emph{interpola} la funzione, ovvero prende esattamente gli stessi valori della funzione $f$ in un insieme finito di punti (i nodi di interpolazione) e ``si inserisce'' fra questi valori, passando per essi. Inoltre, interpolare non significa approssimare. Interpolare vuol dire costruire una funzione (nel nostro caso un polinomio) che coincide \textbf{esattamente} con $f$ in punti scelti (i nodi), e la approssima negli altri.

        
        \item Perché vengono dati 701 punti equispaziati? Questi punti $\left\{\bar{x}_{j}\right\}$ non sono nodi di interpolazione. Servono, invece, come \textbf{griglia di controllo}:
        \begin{itemize}
            \item In ciascun punto $\bar{x}_{j}$ si calcola il valore esatto $f\left(\bar{x}_{j}\right)$.
            \item Poi si calcola il valore approssimato usando il polinomio interpolante $\Pi_{n} f\left(\bar{x}_{j}\right)$.
            \item La differenza:
            \begin{equation*}
                E\left(\bar{x}_{j}\right) = f\left(\bar{x}_{j}\right) - \Pi_{n} f\left(\bar{x}_{j}\right)
            \end{equation*}
            è l'\definition{Errore di Interpolazione} in quel punto.
        \end{itemize}
        Avendo tanti punti equispaziati (701 è un numero grande, quindi la griglia è fitta), si può stimare bene \textbf{quanto e come il polinomio interpolante si discosta da} $f$ su tutto l'intervallo $\left[-2, 2\right]$. Chiaramente, più punti vengono usati per testare, più accurata sarà la stima della \textbf{norma del massimo dell'errore}:
        \begin{equation*}
            \left\| f - \Pi_{n} f \right\|_{\infty} \approx \max_{1\le j \le 701} \left| f\left(\bar{x}_j\right) - \Pi_{n} f\left(\bar{x}_j\right) \right|
        \end{equation*}


        \item Cos'è l'errore di interpolazione? Teoricamente, per una funzione $f \in C^{n+1}$, vale la formula:
        \begin{equation*}
            f(x) - \Pi_{n} f(x) = \dfrac{f^{(n+1)}(\xi(x))}{(n+1)!} \displaystyle\prod_{i=0}^{n}(x - x_i)
        \end{equation*}
        Dove $\xi(x)$ è un punto (non noto) nell'intervallo. Questo indica che l'errore dipende:
        \begin{enumerate}
            \item Dalla derivata $(n+1)$-esima di $f$ (quanto $f$ è ``curva'');
            \item E dal termine $\displaystyle\prod (x - x_{i})$ che cresce con $n$ e con la scelta dei nodi.
        \end{enumerate}
        Nella pratica, non potendo conoscere $\xi(x)$, si calcola l'errore numericamente sui 701 punti equispaziati.
    \end{itemize}
    Quindi, per riassumere: interpolazione significa costruire un polinomio che ``passa'' per i valori noti di $f$ in $n+1$ nodi. L'errore di interpolazione è la differenza $f(x) - \Pi_{n} f(x)$, stimata sui 701 punti equispaziati. Quei punti servono solo come ``griglia di test'' per misurare quanto il polinomio approssima bene la funzione sull'intero intervallo.
\end{remarkbox}

\begin{enumerate}
    \item Fissando $n = 10$, costruire il polinomio interpolante $\Pi_{10} f$ su nodi equispaziati (estremi inclusi) e calcolare il massimo dell'errore di approssimazione sui punti $\left\{\bar{x}_{j}\right\}_{j = 1, \dots, 701}$. Riportare il risultato ottenuto e i comandi MATLAB usati.
    
    \textcolor{Green3}{\textbf{\emph{Soluzione}}}
    \begin{lstlisting}[language=MATLAB]
% funzione
f = @(x) log(5*x.^2 + 1); % funzione da approssimare

% griglia di controllo x_j
xj = linspace(-2, 2, 701);

% grado del polinomio
n = 10;

% nodi
xn = linspace(-2, 2, n + 1);

% polinomio interpolante di grado n sui nodi xn
Pf = polyfit(xn, f(xn), n);
% massimo errore di approssimazione
err = max(abs(f(xj) - polyval(Pf, xj)));
disp('Massimo errore di approssimazione');
disp(err);\end{lstlisting}
    Output:
    \begin{lstlisting}
Massimo errore di approssimazione
    0.9989\end{lstlisting}
    Escludendo la parte di setup, la quale non ha bisogno di spiegazioni approfondite, nella parte di calcolo effettivo, si lasciano alcune osservazioni:
    \begin{itemize}
        \item Comando \texttt{polyfit}:
        \begin{lstlisting}[language=MATLAB]
Pf = polyfit(xn, f(xn), n);\end{lstlisting}
        In una sola invocazione, esegue due compiti:
        \begin{enumerate}
            \item Prende i \textbf{nodi} \texttt{xn} e i valori della funzione \texttt{f(xn)};
            \item Calcola i coefficienti del \textbf{polinomio interpolante} di grado $n$ che passa esattamente per quei punti.
        \end{enumerate}
        Ovviamente, si potrebbe costruire a mano con la formula di Lagrange o Newton, ma MATLAB fornisce già un API. Il risultato di questa chiamata, è un vettore riga \texttt{Pf} con i coefficienti del polinomio in forma classica:
        \begin{equation*}
            p(x) = Pf(1) \, x^n + Pf(2) \, x^{n-1} + \dots + Pf(n) \, x + Pf(n+1)
        \end{equation*}


        \item Comando \texttt{polyval}:
        \begin{lstlisting}[language=MATLAB]
polyval(Pf, xj)\end{lstlisting}
        Valuta il polinomio definito da \texttt{Pf} nei punti \texttt{xj}. Quindi sta calcolando $\Pi_{10} f(\bar{x}_{j})$ per i 701 punti equispaziati nell'intervallo.


        \item A questo punto, si hanno due vettori:
        \begin{itemize}
            \item \texttt{f(xj)} per i valori veri della funzione in 701 punti.
            \item \texttt{polyval(Pf, xj)} per i valori del polinomio interpolante negli stessi 701 punti.
        \end{itemize}
        Per calcolare l'errore, dobbiamo fare la differenza tra i valori veri e quelli interpolati:
        \begin{lstlisting}[language=MATLAB]
f(xj) - polyval(Pf, xj)\end{lstlisting}
        Il quale è il \textbf{vettore degli error puntuali} $E\left(\bar{x}_{j}\right)$. Infine, prendiamo il \textbf{massimo in valore assoluto} di questi errori, ovvero la norma infinito dell'errore di interpolazione.
    \end{itemize}


    \item Ripetere il punto precedente usando i nodi di Chebyshev (estremi inclusi) nell'intervallo $I$. Indicando con $\Pi_{10}^{C}f$ l'interpolante corrispondente, riportare il massimo dell'errore di approssimazione sui punti $\left\{\overline{x}_{j}\right\}_{j = 1, \dots, 701}$ e i comandi Matlab usati.
    
    \begin{remarkbox}
        Alcune osservazioni:
        \begin{itemize}
            \item Perché introdurre i nodi di Chebyshev? Se si scelgono i \textbf{nodi equispaziati}, come è stato fatto nell'esercizio precedente, l'errore di interpolazione può diventare molto grande per funzioni un po' oscillanti, soprattutto vicino agli estremi (\definition{Fenomeno di Runge}). Per ridurre questo problema, si usano i \textbf{nodi di Chebyshev}, che non sono equispaziati: sono più \textbf{densi agli estremi} e più radi al centro.
            
            \item I nodi di Chebyshev di ordine $n+1$ (quindi per un polinomio di grado $n$) in un intervallo generico $\left[a,b\right]$ sono:
            \begin{equation*}
                x_{k} = \dfrac{a+b}{2} + \dfrac{b-a}{2}\cdot\cos\left(\dfrac{2k+1}{2(n+1)}\cdot\pi\right), \quad k = 0,1,\dots,n
            \end{equation*}
            Noti anche come \textbf{nodi di Chebyshev di prima specie}. Per esempio, con $\left[a,b\right] = \left[-1,1\right]$ diventano semplicemente:
            \begin{equation*}
                x_k = \cdot\cos\left(\dfrac{2k+1}{2(n+1)}\pi\right)
            \end{equation*}
            Attenzione, questi nodi sono usati nel caso in cui gli estremi \textbf{non} siano inclusi.

            \item Differenza tra i nodi equispaziati e i nodi di Chebyshev:
            \begin{itemize}
                \item \textbf{Equispaziati}: uniformi su $\left[a,b\right]$
                \item \textbf{Chebyshev}: più concentrati agli estremi, quindi si riducono al minimo l'effetto di oscillazioni indesiderate, e quindi garantiscono che l'errore massimo teorico sia quasi ottimale.
            \end{itemize}

            \item Con i \textbf{nodi di Chebyshev di seconda specie}, gli estremi vengono presi in considerazione:
            \begin{equation*}
                x_{k} = \dfrac{a+b}{2} + \frac{b-a}{2}\cdot\cos\left(\dfrac{k}{n}\cdot\pi\right), \quad k=0,\dots,n
            \end{equation*}
            Quindi $x_{0} = a$ e $x_{n} = b$.
        \end{itemize}
    \end{remarkbox}

    \textcolor{Green3}{\textbf{\emph{Soluzione.}}} Il codice è identico al precedente ma differisce soltanto come si costruisce il vettore $x$:
    \begin{lstlisting}[language=MATLAB]
% estremi intervallo
a = -2; b = 2;
k = 0:n;

% nodi di Chebyshev (seconda specie, estremi inclusi)
xnC = (a+b)/2 + (b-a)/2 * cos(k/n * pi);
% se fosse stato di prima specie:
% xnC = (a+b)/2 + (b-a)/2 * cos((2*k+1)/(2*(n+1)) * pi);

% polinomio interpolante di grado n sui nodi di Chebyshev
PfC = polyfit(xnC, f(xnC), n);
% massimo errore di approssimazione
erC = max(abs(f(xj) - polyval(PfC, xj)));
disp('Massimo errore di approssimazione con nodi di Chebyshev');
disp(erC);\end{lstlisting}
    Risultato:
    \begin{lstlisting}
Massimo errore di approssimazione con nodi di Chebyshev
     0.060307\end{lstlisting}


    \item Rappresentare in un unico grafico: la funzione $f(x)$, i polinomi $\Pi_{10}f$ e $\Pi_{10}^{C}f$. Caricare l'immagine generata in formato \texttt{png} con opportuna legenda.
    
    \textcolor{Green3}{\textbf{\emph{Soluzione}}}

    \begin{lstlisting}[language=MATLAB]
figure;
hold on;
plot(xj, f(xj), 'r', 'DisplayName', 'funzione originale');
plot(xj, polyval(Pf, xj), '-b', 'DisplayName', 'interp con nodi equispaziati');
plot(xj, polyval(PfC, xj), '-g', 'DisplayName', 'interp con nodi di Chebyshev');
legend show;
title('Confronto tra funzione e polinomi interpolanti');
xlabel('x');
ylabel('y');
grid on;
hold off;\end{lstlisting}
    \begin{figure}[!htp]
        \centering
        \includegraphics[width=.8\textwidth]{img/24-07-2025/24-07-2025-compare.pdf}
    \end{figure}

    \newpage

    \item Costruire il polinomio interpolante su nodi di Chebyshev al variare di $n = 10, 20, 30, 40$. Calcolare l'errore di interpolazione sui $701$ punti $\overline{x}_{j}$ e riportare in notazione esponenziale il vettore degli errori al variare di $n$. Quale andamento si osserva?
    
    \textcolor{Green3}{\textbf{\emph{Soluzione}}}
    \begin{lstlisting}[language=MATLAB]
minVal = 10;
maxVal = 40;
step = 10;
errors = zeros(1, 5);

for n = minVal:step:maxVal
    k = 0:n;
    xnC = (a+b)/2 + (b-a)/2 * cos(k/n * pi);
    PfC = polyfit(xnC, f(xnC), n);
    erC = max(abs(f(xj) - polyval(PfC, xj)));
    errors(n/step) = erC;
end

disp('Errori al variare di n:');
disp(errors);\end{lstlisting}
    Risultato:
    \begin{lstlisting}
Errori al variare di n:
     0.060307     0.004226    0.0003243   2.7112e-05\end{lstlisting}
    Si può osservare una decrescita esponenziale dell'errore al variare di \texttt{n}.


    \item Si introduca l'interpolante di Lagrange composito $\Pi_{k}^{H}$ con $k \ge 1$ definendo con precisione la notazione utilizzata. A partire dalla stima di convergenza dell'interpolatore Lagrangiano, si deduca la stima dell'errore di interpolazione composita in funzione di $H$ ed $k$.
    
    \begin{remarkbox}
        L'interpolazione di Lagrange è un approccio diverso al problema dell'interpolazione.
        \begin{itemize}
            \item \important{Interpolante di Lagrange classico}. Si costruisce \textbf{un unico polinomio} $\Pi_{n} f$ di grado $n$ che interpola $f$ in $n+1$ nodi su tutto l'intervallo $\left[a,b\right]$. Ma in questo caso, se $n$ diventa grande, si hanno oscillazioni (\definition{Fenomeno di Runge}), instabilità numerica, costo alto.
            
            \item \important{Interpolante di Lagrange composito}. L'idea \eaccent di \textbf{non usare un unico polinomio di alto grado}, ma tanti polinomi di basso grado costruiti \textbf{pezzo per pezzo} su sottointervalli.
            \begin{itemize}
                \item Si divide l'intervallo $\left[a,b\right]$ in $M$ sottointervalli di ampiezza massima $H = \max \left|x_{i + 1} - x_{i}\right|$.
                \item Su ogni sottointervallo si costruisce un \textbf{polinomio interpolante di grado} $k$ (con $k+1$ nodi per sottointervallo).
                \item L'interpolante composito $\Pi_{k}^{H}f$ \eaccent la funzione ``a tratti'' che, su ogni sottointervallo, coincide con il polinomio interpolante locale.
            \end{itemize}

            \item \important{Convergenza e stima dell'errore}. Per l'interpolazione di Lagrange su un singolo intervallo, \eaccent noto che (se $f \in C^{k+1}$):
            \begin{equation*}
                \left|f(x) - \Pi_{k}f(x)\right| \le C \, h^{k+1}, \quad h = \text{ampiezza dell'intervallo}
            \end{equation*}
            Per la versione composita:
            \begin{itemize}
                \item Ogni sottointervallo ha ampiezza $\le H$;
                \item L'errore su ciascun pezzo \eaccent $\mathcal{O}\left(H^{k+1}\right)$;
                \item Quindi l'errore globale dell'interpolante composito $\Pi_{k}^{H}$ \eaccent anch'esso:
                \begin{equation*}
                    \left\| f - \Pi_{k}^{H} f \right\|_{\infty} \le C \, H^{\,k+1}
                \end{equation*}
            \end{itemize}
        \end{itemize}
    \end{remarkbox}

    \textcolor{Green3}{\textbf{\emph{Soluzione.}}} Sia $I = \left[a,b\right]$ un intervallo e sia data una \textbf{partizione} (o griglia):
    \begin{equation*}
        \mathcal T_H={a=x_0<x_1<\dots<x_M=b}, \qquad I_j=[x_j,x_{j+1}],\; j=0,\dots,M-1
    \end{equation*}
    Con \textbf{passo massimo}:
    \begin{equation*}
        H:=\max_{0\le j\le M-1} h_j, \qquad h_j:=x_{j+1}-x_j
    \end{equation*}
    Fissato un \textbf{grado} $k \ge 1$, su ciascun sottointervallo $I_{j}$, si scelgono $k+1$ \textbf{nodi locali} $\left\{x^{(i)}_j\right\}_{i=0}^k\subset I_j$ (ad es. equispaziati in $I_{j}$) e si definisce $\Pi_{k}^{H} f$ \textbf{a tratti} come il \important{polinomio di Lagrange} di grado $k$ che interpola $f$ nei nodi di $I_{j}$:
    \begin{equation*}
        \left(\Pi_k^H f\right)|_{I_j} \;\; := \;\; \Pi_k\left(f|_{I_j}\right)
        \quad\text{con}\quad
        \left(\Pi_k f\right)\left(x^{(i)}_j\right)=f\left(x^{(i)}_j\right),\; i=0,\dots,k
    \end{equation*}

    \textbf{Stima dell'errore}. Si assuma $f \in C^{k+1}\left(\left[a,b\right]\right)$. Sulla \textbf{singola} cella $I_{j}$ di ampiezza $h_{j}$ la classica stima dell'errore dell'interpolazione di Lagrange di grado $k$:
    \begin{equation*}
        \left\| f - \Pi_{k} f \right\|_{L^\infty \left(I_j\right)} \; \le \; C \: h_{j}^{k+1} \left\| f^{\left(k+1\right)} \right\|_{L^{\infty} \left(I_{j}\right)}
    \end{equation*}
    Con $C$ indipendente da $h_{j}$ (deriva dalla formula dell'errore con $\prod \left(x-x_{i}\right)$). Prendendo il massimo su tutte le celle e usando $h_{j} \le H$:
    \begin{equation*}
        \left\| f - \Pi_{k}^{H} f \right\|_{L^{\infty}\left(\left[a,b\right]\right)} \; \le \;
        C \, H^{k+1} \, \left\| f^{\left(k+1\right)} \right\|_{L^{\infty}\left(\left[a,b\right]\right)}
    \end{equation*}
    Ossia \textbf{ordine} $\left(k+1\right)$ in $H$.

    \textbf{Caso $k=1$, lineare composito}. In particolare, se $f \in C^{2}$ e si usando $k=1$ e nodi agli estremi di ogni $I_{j}$, vale:
    \begin{equation*}
        \left\| f - \Pi_{1}^{H} f \right\|_{\infty} \; \le \; \dfrac{H^{2}}{8} \, \left\| f'' \right\|_{\infty}
    \end{equation*}
    Che mostra esplicitamente l'ordine 2 in $H$. Spiegazione dettagliata e suggerimenti per ricordare questa formula si trovano a pagina \pageref{takeaways:interpolazione_composita}.
\end{enumerate}

\newpage

\subsubsection*{Esercizio 3}

Si consideri il seguente problema di Cauchy:
\begin{equation*}
    \begin{cases}
        y'(t) = \left[\dfrac{\pi \cos\left(\pi t\right)}{2 + \sin\left(\pi t\right)} - \dfrac{1}{2}\right] y(t) & t \in \left(0, 10\right) \\[1em]
        y(0) = 2
    \end{cases}
\end{equation*}
La cui soluzione esatta è $y(t) = \left(2 + \sin\left(\pi t\right)\right) e^{- t/2}$.
\begin{enumerate}
    \item Si riporti l'algoritmo del metodo di Eulero in avanti application al problema di Cauchy definendo con precisione tutta la notazione utilizzata. Dopo aver posto $f(t, y) = -\lambda y$, con $\lambda > 0$, si ricavi la condizione di assoluta stabilità per il metodo di Eulero in avanti.

    \textcolor{Green3}{\textbf{\emph{Soluzione}}}
    \begin{enumerate}
        \item Si ha $t \in \left[0, 10\right]$, la condizione iniziale $y(0) = 2$, e la funzione:
        \begin{equation*}
            f(t, y) = \left[\dfrac{\pi \cos\left(\pi t\right)}{2 + \sin\left(\pi t\right)} - \dfrac{1}{2}\right] y
        \end{equation*}
        Quindi:
        \begin{itemize}
            \item Intervallo: $t \in \left[0, 10\right]$
            \item Numero di passi $N$
            \item Passo, bisogna sceglierlo molto piccolo per evitare errori, quindi:
            \begin{equation*}
                h = \dfrac{10 - 0}{N} = \dfrac{10}{N}
            \end{equation*}
            \item Punti temporali su cui si approssima la soluzione:
            \begin{equation*}
                t_n = t_0 + n h \quad n = 0, 1, \dots, N
            \end{equation*}
            \item Passo iniziale identico alla condizione iniziale:
            \begin{equation*}
                u_0 = y(0) = 2
            \end{equation*}
        \end{itemize}

        \item In generale, il metodo di Eulero in avanti si scrive come:
        \begin{equation*}
            u_{n+1} = u_{n} + h \cdot f(t_{n}, u_{n})
        \end{equation*}
        Dove $u_{n}$ rappresenta l'approssimazione di $y(t_{n})$ e $f(t_{n}, u_{n})$ \eaccent la pendenza della curva in quel punto, ovvero la derivata.

        Nel nostro caso, si ha:
        \begin{equation*}
            u_{n+1} = u_{n} + h \cdot \left(
                \dfrac{\pi \cos\left(\pi t_{n}\right)}{2 + \sin\left(\pi t_{n}\right)} - \dfrac{1}{2}
            \right) \cdot u_{n}
        \end{equation*}

        \item A questo punto si studia la stabilità del metodo ponendo:
        \begin{equation*}
            f(t, y) = -\lambda y, \quad \lambda > 0
        \end{equation*}
        Quindi, il metodo di Eulero muta in:
        \begin{equation*}
            u_{n+1} = u_{n} + h \cdot \left(-\lambda u_{n}\right) = u_{n} - h\lambda u_{n} = u_{n} \cdot \left(1 - h\lambda\right)
        \end{equation*}

        \item La stabilità del metodo dipende dal fattore moltiplicativo:
        \begin{equation*}
            g = 1 - h\lambda
        \end{equation*}
        Dato che viene moltiplicato ad ogni passo. Per garantire che la soluzione numerica sia stabile (ovvero non cresca indefinitamente), si richiede che il suo valore assoluto sia minore di 1:
        \begin{equation*}
            |g| < 1 \quad \Rightarrow \quad |1 - h\lambda| < 1
        \end{equation*}
        Per definizione di valore assoluto, questo implica:
        \begin{equation*}
            -1 < 1 - h\lambda < 1
        \end{equation*}
        L'obbiettivo è isolare $h \lambda$. Per cui si deve come primo passo rimuovere il $1 - \dots$. Per fare questo, si sottrae $1$ in tutte e due le disuguaglianze:
        \begin{equation*}
            \left(-1\right) + \left(-1\right) < \left(1 - h\lambda\right) - 1 < 1 + \left(- 1\right) \quad \Rightarrow \quad -2 < - h\lambda < 0
        \end{equation*}
        A questo punto, si moltiplica per $-1$ (cambiando il verso delle disuguaglianze):
        \begin{equation*}
            2 > h\lambda > 0 \quad \Rightarrow \quad 0 < h\lambda < 2
        \end{equation*}
        Infine, dividendo per $\lambda$ (positivo, quindi non cambia il verso):
        \begin{equation*}
            \dfrac{1}{\lambda} \cdot 0 < \dfrac{1}{\lambda} \cdot h\lambda < \dfrac{1}{\lambda} \cdot 2 \quad \Rightarrow \quad 0 < h < \dfrac{2}{\lambda}
        \end{equation*}
        Quindi, la condizione di stabilità per il metodo di Eulero in avanti è:
        \begin{equation*}
            0 < h < \dfrac{2}{\lambda}
        \end{equation*}
        Dove $h$ è il passo scelto per l'iterazione e $\lambda$ è il parametro della funzione $f(t, y) = -\lambda y$.
    \end{enumerate}
    A pagina \hqpageref{problema-cauchy-con-eulero-avanti} si trova un riepilogo del metodo di Eulero in avanti e della sua stabilità (con una spiegazione più dettagliata).
\end{enumerate}

    %%%%%%%%%%%%%%%%%%%%%%%%%%%%%%
    % Domande teoriche frequenti %
    %%%%%%%%%%%%%%%%%%%%%%%%%%%%%%
    \section{Domande Teoriche Frequenti}

In questa sezione sono raccolte le domande di teoria più ricorrenti, emerse durante le lezioni, le esercitazioni e le prove d'esame. L'obiettivo è fornire un ``compendio'' sintetico e mirato, che permetta di:
\begin{itemize}
    \item Avere un quadro immediato dei concetti fondamentali,
    \item Ripassare rapidamente i punti teorici più importanti,
    \item Orientarsi sulle domande che più spesso vengono utilizzate per verificare la comprensione.
\end{itemize}
Questa raccolta non sostituisce lo studio completo dei materiali, ma rappresenta una guida veloce per fissare e richiamare alla memoria i temi principali.

\begin{itemize}
    \item \textcolor{Red2}{\faIcon{question-circle} \textbf{\emph{%
        Elencare e discutere:
        \begin{enumerate}
            \item Le condizioni sufficienti per la convergenza del metodo di Gauss-Seidel.
            \item Le condizioni necessarie e sufficienti per la convergenza del metodo di Gauss-Seidel.
        \end{enumerate}
        Data $B_{GS}$ la matrice di iterazione di Gauss-Seidel, dimostrare che $\left\| B_{GS} \right\| < 1$ implica la convergenza.}}}

    \textcolor{Green3}{\faIcon{check-circle} \textbf{\emph{Soluzione.}}} Le condizioni sufficienti per la convergenza del metodo di Gauss-Seidel sono 3:
    \begin{itemize}
        \item La matrice $A$ deve essere a \textbf{dominanza diagonale stretta per righe}:
        \begin{equation*}
            \left|a_{ii}\right| > \displaystyle\sum_{j \ne i} \left|a_{ij}\right| \hspace{1em} \forall i
        \end{equation*}
        \item La matrice $A$ deve essere \textbf{simmetrica definita positiva (SPD)}:
        \begin{equation*}
            \mathbf{x}^{T} A \mathbf{x} > 0 \hspace{1em} \forall \mathbf{x} \ne 0
        \end{equation*}
        \item La matrice $A$ deve essere una \textbf{Matrice di classe M (M-Matrice) non singolare}. Ovvero una matrice con diagonale positiva e elementi extradiagonali (non sulla diagonale) non positivi.
    \end{itemize}
    L'unica condizione necessarie e sufficiente per la convergenza è quando il raggio spettrale della matrice di iterazione $B_{GS}$ è minore di uno:
    \begin{equation*}
        \text{convergenza} \iff \rho(B_{GS}) < 1
    \end{equation*}
    \begin{proof}[Dimostrazione ($\left\| B_{GS} \right\| < 1$ implica la convergenza)]
        Il metodo di Gauss-Seidel si scrive
        \begin{equation*}
            x^{(k+1)} = B_{GS}x^{(k)} + c
        \end{equation*}
        Definendo l'errore $e^{(k)} = x^{(k)} - x^{*}$, dove $x^{*}$ rappresenta la soluzione esatta del sistema lineare che Gauss-Seidel cerca di raggiungere, e $x^{(k)}$ l'approssimazione della soluzione al passo $k$, si ha
        \begin{equation*}
            e^{(k+1)} = B_{GS} e^{(k)} \quad \Rightarrow \quad e^{(k)} = B_{GS}^k e^{(0)}
        \end{equation*}
        Se $\|B_{GS}\| < 1$, allora:
        \begin{equation*}
            \|e^{(k)}\| \le \|B_{GS}\|^k \|e^{(0)}\| \to 0,
        \end{equation*}
        Per $k \to \infty$. Quindi $x^{(k)} \to x^\ast$: il metodo converge.
    \end{proof}

    \begin{deepeningbox}[: Spiegazione della Dimostrazione]
        L'idea principale della dimostrazione è: \textbf{l'errore ad ogni passo viene moltiplicato da una matrice}; se questa matrice ``schiaccia'' tutti i vettori di almeno un fattore $q < 1$, allora l'errore cala geometricamente $\to 0$.

        \begin{itemize}
            \item \important{Passo 0: Dal sistema $Ax=b$ all'iterazione}. Per Gauss-Seidel si scrive (con $A = D + L + U$):
            \begin{equation*}
                \begin{array}{rcl}
                    x^{(k+1)} &=& B_{GS}\,x^{(k)} + c \\ [.3em]
                    B_{GS} &=& -(D+L)^{-1}U \\ [.3em]
                    c &=& (D+L)^{-1}b
                \end{array}
            \end{equation*}
            Questa è una \textbf{iterazione a punto fisso} $x^{(k+1)}=T(x^{(k)})$ con $T(x)=B_{GS}x+c$. Si ricorda che l'iterazione a punto fisso è un metodo iterativo per risolvere $A\mathbf{x} = \mathbf{b}$ che si può sempre scrivere nella forma $x^{(x+1)} = T(x^{(k)})$, dove $T$ è una funzione; se esiste una $x^{*}$ tale che $T(x^{*}) = x^{*}$, allora $x^{*}$ si chiama punto fisso di $T$.

            \highspace
            Nel caso Gauss-Seidel, si ha:
            \begin{equation*}
                x^{(k+1)} = B_{GS} x^{(k)} + c
            \end{equation*}
            Quindi $T(x) = B_{GS} x^{(k)} + c$. Il punto fisso $x^{*}$ soddisfa:
            \begin{equation*}
                x^{*} = B_{GS} x^{*} + C
            \end{equation*}
            Cioè proprio il sistema originale $Ax = b$.


            \item \important{Passo 1: Errore che si propaga}. Sia $x^{*}$ la soluzione (il punto fisso): soddisfa $x^{*} = B_{GS}x^{*} + c$. Definiamo l'errore $e^{(k)} = x(k) - x^{*}$. Sottraendo le due relazioni:
            \begin{equation*}
                e^{(k+1)} = B_{GS} e^{(k)}
            \end{equation*}
            Quindi l'errore al passo successivo è semplicemente $B_{GS}$ per l'errore attuale.


            \item \important{Passo 2: Usiamo una norma ``coerente''}. Prendiamo una norma matriciale indotta (coerente con una norma vettoriale), cioè una per cui vale: $\left\| A v \right\| \le \left\| A \right\| \left\| v \right\|$. Applicandola:
            \begin{equation*}
                \left\| e^{(k+1)} \right\| = \left\| B_{GS}e^{(k)} \right\| \le \left\| B_{GS} \right\| \left\| e^{(k)} \right\|
            \end{equation*}
            Iterando:
            \begin{equation*}
                \left\| e^{(k)} \right\| \le \left\| B_{GS} \right\|^{k} \left\| e^{(0)} \right\|
            \end{equation*}

            
            \item \important{Passo 3: Conclusione (contrazione geometrica)}. Se $\left\| B_{GS} \right\| = q < 1$, allora $q^{k} \to 0$. Dunque $\left\| e^{(k)} \right\| \to 0$ e quindi $x^{(k)} \to x^{*}$. \textbf{Questo è tutto}. L'ipotesi $\left\|B_{GS}\right\| < 1$ garantisce convergenza \textbf{da qualunque} $x^{(0)}$.
        \end{itemize}
    \end{deepeningbox}


    \item \textcolor{Red2}{\textbf{\emph{Si introduca l'interpolante di Lagrange composito $\Pi_{k}^{H}$ con $k \ge 1$ definendo con precisione la notazione utilizzata. A partire dalla stima di convergenza dell'interpolatore Lagrangiano, si deduca la stima dell'errore di interpolazione composita in funzione di $H$ ed $k$.}}}
    
    \textcolor{Green3}{\faIcon{check-circle} \textbf{\emph{Soluzione.}}}
\end{itemize}

    %%%%%%%%%%%%%%%%%%%%%%%%%%
    % Bibliography and index %
    %%%%%%%%%%%%%%%%%%%%%%%%%%
    \pagestyle{fancy}
\fancyhead{} % clear all header fields
\fancyhead[R]{\nouppercase{\leftmark\hfill\rightmark}}

\pagestyle{fancy}
\fancyhead{} % clear all header fields
\fancyhead[R]{\nouppercase{\leftmark}}

\bibliography{bibtex}{}
\bibliographystyle{plain}

\newpage

\printindex
\end{document}