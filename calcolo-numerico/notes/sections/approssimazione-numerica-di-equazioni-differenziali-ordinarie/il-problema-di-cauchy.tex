\section{Approssimazione numerica di ODE}

\subsection{Problema di Cauchy}

Un'equazione differenziale ordinaria ammette in generale infinite soluzioni. Per fissarne una è necessario imporre una condizione che prescriva il valore assunto dalla soluzione in un punto dell'intervallo di integrazione. In questa sezione ci si occuperà della risoluzione dei \definition{problemi di Cauchy}, ossia di problemi nella seguente forma.

\highspace
\begin{flushleft}
	\textcolor{Red2}{\faIcon{bookmark} \textbf{Problema di Cauchy}}
\end{flushleft}
Sia $I = \left[t_{0}, T\right]$ l'intervallo temporale di interesse. Trovare la funzione vettoriale $\mathbf{y}: I \rightarrow \mathbb{R}^{n}$ che soddisfa:
\begin{equation}
	\begin{cases}
		\mathbf{y}'\left(t\right) = \mathbf{f}\left(t, \mathbf{y}\left(t\right)\right) \\
		\mathbf{y}\left(t_{0}\right) = \mathbf{y}_{0}
	\end{cases}
\end{equation}
Con $\mathbf{f}: I \times \mathbb{R}^{n} \rightarrow \mathbb{R}^{n}$ funzione vettoriale assegnata.

\highspace
\begin{flushleft}
	\textcolor{Red2}{\faIcon{bookmark} \textbf{Problema di Cauchy \emph{scalare}}}
\end{flushleft}
Per semplicità, verrà considerato soltanto il \textbf{caso scalare}, ovvero in cui $n=1$, ma i metodi di approssimazione che verranno introdotti saranno facilmente estendibili e applicabili anche con $n>1$. Sia $I = \left[t_{0}, T\right]$ l'intervallo temporale di interesse. Trovare $y: I \subset \mathbb{R} \rightarrow \mathbb{R}$ tale che:
\begin{equation}
	\begin{cases}
		y'\left(t\right) = f\left(t, y\left(t\right)\right) \hspace{2em} \forall t \in I \\
		y\left(t_{0}\right) = y_{0}
	\end{cases}
\end{equation}
Con $f: I \times \mathbb{R} \rightarrow \mathbb{R}$ funzione assegnata.

\highspace
\begin{definitionbox}[: esistenza e unicità della soluzione continua]
	Si supponga che la funzione $f\left(t,y\right)$ sia:
	\begin{enumerate}
		\item Limitata e continua rispetto ad entrambi gli argomenti.
		
		\item Lipschitziana rispetto al secondo argomento, ossia esista una costante $L$ positiva (detta costante di Lipschitz) tale per cui:
		\begin{equation}
			\left|f\left(t, y_{1}\right) - f\left(t, y_{2}\right)\right| \le L\left|y_{1} - y_{2}\right| \hspace{2em} \forall t \in I, \:\: \forall y_{1}, y_{2} \in \mathbb{R}
		\end{equation}
	\end{enumerate}
	Allora la \textbf{soluzione} del problema di Cauchy \textbf{esiste}, è \textbf{unica} ed è di \textbf{classe} $C^{1}$ su $I$.
\end{definitionbox}