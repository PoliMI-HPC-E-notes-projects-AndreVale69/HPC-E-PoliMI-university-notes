\section{Domande Teoriche Frequenti}

In questa sezione sono raccolte le domande di teoria più ricorrenti, emerse durante le lezioni, le esercitazioni e le prove d'esame. L'obiettivo è fornire un ``compendio'' sintetico e mirato, che permetta di:
\begin{itemize}
    \item Avere un quadro immediato dei concetti fondamentali,
    \item Ripassare rapidamente i punti teorici più importanti,
    \item Orientarsi sulle domande che più spesso vengono utilizzate per verificare la comprensione.
\end{itemize}
Questa raccolta non sostituisce lo studio completo dei materiali, ma rappresenta una guida veloce per fissare e richiamare alla memoria i temi principali.

\begin{itemize}
    \item \textcolor{Red2}{\faIcon{question-circle} \textbf{\emph{%
        Elencare e discutere:
        \begin{enumerate}
            \item Le condizioni sufficienti per la convergenza del metodo di Gauss-Seidel.
            \item Le condizioni necessarie e sufficienti per la convergenza del metodo di Gauss-Seidel.
        \end{enumerate}
        Data $B_{GS}$ la matrice di iterazione di Gauss-Seidel, dimostrare che $\left\| B_{GS} \right\| < 1$ implica la convergenza.}}}

    \textcolor{Green3}{\faIcon{check-circle} \textbf{\emph{Soluzione.}}} Le condizioni sufficienti per la convergenza del metodo di Gauss-Seidel sono 3:
    \begin{itemize}
        \item La matrice $A$ deve essere a \textbf{dominanza diagonale stretta per righe}:
        \begin{equation*}
            \left|a_{ii}\right| > \displaystyle\sum_{j \ne i} \left|a_{ij}\right| \hspace{1em} \forall i
        \end{equation*}
        \item La matrice $A$ deve essere \textbf{simmetrica definita positiva (SPD)}:
        \begin{equation*}
            \mathbf{x}^{T} A \mathbf{x} > 0 \hspace{1em} \forall \mathbf{x} \ne 0
        \end{equation*}
        \item La matrice $A$ deve essere una \textbf{Matrice di classe M (M-Matrice) non singolare}. Ovvero una matrice con diagonale positiva e elementi extradiagonali (non sulla diagonale) non positivi.
    \end{itemize}
    L'unica condizione necessarie e sufficiente per la convergenza è quando il raggio spettrale della matrice di iterazione $B_{GS}$ è minore di uno:
    \begin{equation*}
        \text{convergenza} \iff \rho(B_{GS}) < 1
    \end{equation*}
    \begin{proof}[Dimostrazione ($\left\| B_{GS} \right\| < 1$ implica la convergenza)]
        Il metodo di Gauss-Seidel si scrive
        \begin{equation*}
            x^{(k+1)} = B_{GS}x^{(k)} + c
        \end{equation*}
        Definendo l'errore $e^{(k)} = x^{(k)} - x^{*}$, dove $x^{*}$ rappresenta la soluzione esatta del sistema lineare che Gauss-Seidel cerca di raggiungere, e $x^{(k)}$ l'approssimazione della soluzione al passo $k$, si ha
        \begin{equation*}
            e^{(k+1)} = B_{GS} e^{(k)} \quad \Rightarrow \quad e^{(k)} = B_{GS}^k e^{(0)}
        \end{equation*}
        Se $\|B_{GS}\| < 1$, allora:
        \begin{equation*}
            \|e^{(k)}\| \le \|B_{GS}\|^k \|e^{(0)}\| \to 0,
        \end{equation*}
        Per $k \to \infty$. Quindi $x^{(k)} \to x^\ast$: il metodo converge.
    \end{proof}

    \begin{deepeningbox}[: Spiegazione della Dimostrazione]
        L'idea principale della dimostrazione è: \textbf{l'errore ad ogni passo viene moltiplicato da una matrice}; se questa matrice ``schiaccia'' tutti i vettori di almeno un fattore $q < 1$, allora l'errore cala geometricamente $\to 0$.

        \begin{itemize}
            \item \important{Passo 0: Dal sistema $Ax=b$ all'iterazione}. Per Gauss-Seidel si scrive (con $A = D + L + U$):
            \begin{equation*}
                \begin{array}{rcl}
                    x^{(k+1)} &=& B_{GS}\,x^{(k)} + c \\ [.3em]
                    B_{GS} &=& -(D+L)^{-1}U \\ [.3em]
                    c &=& (D+L)^{-1}b
                \end{array}
            \end{equation*}
            Questa è una \textbf{iterazione a punto fisso} $x^{(k+1)}=T(x^{(k)})$ con $T(x)=B_{GS}x+c$. Si ricorda che l'iterazione a punto fisso è un metodo iterativo per risolvere $A\mathbf{x} = \mathbf{b}$ che si può sempre scrivere nella forma $x^{(x+1)} = T(x^{(k)})$, dove $T$ è una funzione; se esiste una $x^{*}$ tale che $T(x^{*}) = x^{*}$, allora $x^{*}$ si chiama punto fisso di $T$.

            \highspace
            Nel caso Gauss-Seidel, si ha:
            \begin{equation*}
                x^{(k+1)} = B_{GS} x^{(k)} + c
            \end{equation*}
            Quindi $T(x) = B_{GS} x^{(k)} + c$. Il punto fisso $x^{*}$ soddisfa:
            \begin{equation*}
                x^{*} = B_{GS} x^{*} + C
            \end{equation*}
            Cioè proprio il sistema originale $Ax = b$.


            \item \important{Passo 1: Errore che si propaga}. Sia $x^{*}$ la soluzione (il punto fisso): soddisfa $x^{*} = B_{GS}x^{*} + c$. Definiamo l'errore $e^{(k)} = x(k) - x^{*}$. Sottraendo le due relazioni:
            \begin{equation*}
                e^{(k+1)} = B_{GS} e^{(k)}
            \end{equation*}
            Quindi l'errore al passo successivo è semplicemente $B_{GS}$ per l'errore attuale.


            \item \important{Passo 2: Usiamo una norma ``coerente''}. Prendiamo una norma matriciale indotta (coerente con una norma vettoriale), cioè una per cui vale: $\left\| A v \right\| \le \left\| A \right\| \left\| v \right\|$. Applicandola:
            \begin{equation*}
                \left\| e^{(k+1)} \right\| = \left\| B_{GS}e^{(k)} \right\| \le \left\| B_{GS} \right\| \left\| e^{(k)} \right\|
            \end{equation*}
            Iterando:
            \begin{equation*}
                \left\| e^{(k)} \right\| \le \left\| B_{GS} \right\|^{k} \left\| e^{(0)} \right\|
            \end{equation*}

            
            \item \important{Passo 3: Conclusione (contrazione geometrica)}. Se $\left\| B_{GS} \right\| = q < 1$, allora $q^{k} \to 0$. Dunque $\left\| e^{(k)} \right\| \to 0$ e quindi $x^{(k)} \to x^{*}$. \textbf{Questo è tutto}. L'ipotesi $\left\|B_{GS}\right\| < 1$ garantisce convergenza \textbf{da qualunque} $x^{(0)}$.
        \end{itemize}
    \end{deepeningbox}


    \item \textcolor{Red2}{\textbf{\emph{Si introduca l'interpolante di Lagrange composito $\Pi_{k}^{H}$ con $k \ge 1$ definendo con precisione la notazione utilizzata. A partire dalla stima di convergenza dell'interpolatore Lagrangiano, si deduca la stima dell'errore di interpolazione composita in funzione di $H$ ed $k$.}}}
    
    \textcolor{Green3}{\faIcon{check-circle} \textbf{\emph{Soluzione.}}}
\end{itemize}