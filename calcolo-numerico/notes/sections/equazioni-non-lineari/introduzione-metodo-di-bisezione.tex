\section{Equazioni non lineari}

\subsection{Introduzione}

Il \textbf{calcolo degli zeri di una funzione} $f$ reale di variabile reale o delle \textbf{radici dell'equazione} $f\left(x\right)=0$, è un problema assai ricorrente nel Calcolo Scientifico.

\highspace
In generale, \emph{non è possibile} approntare metodi numerici che calcolino gli zeri di una generica funzione in un numero finito di passi. I metodi numerici per la risoluzione di questo problema sono pertanto necessariamente \emph{iterativi}. A partire da uno o più dati iniziali, scelti convenientemente, essi generano una successione di valori $x^{\left(k\right)}$ che, sotto opportune ipotesi, convergerà ad uno zero $\alpha$ della funzione $f$ studiata.

\longline

\subsection{Il metodo di bisezione (o iterativo)}

Sia $f$ una funzione continua in $\left[a,b\right]$ tale che $f\left(a\right)f\left(b\right) < 0$. Per cui, vale il \href{https://www.youmath.it/lezioni/analisi-matematica/limiti-continuita-e-asintoti/723-teorema-degli-zeri.html}{teorema degli zeri di una funzione continua}, ossia $f$ ammette almeno uno zero in $\left(a,b\right)$.

\highspace
Si supponga che ci sia un solo zero, indicato con $\alpha$ e nel caso in cui ce ne sia più di uno, individuare un intervallo tale che ne contenga solo uno.

\highspace
Il \definition{metodo di bisezione} (o \textbf{iterativo}\index{metodo iterativo}) è una strategia che si suddivide nei seguenti passaggi:
\begin{enumerate}
    \item \textbf{Dimezzare l'intervallo di partenza};
    \item \textbf{Selezionare tra i due sotto-intervalli ottenuti quello nel quale} $f$ \textbf{cambia di segno agli estremi};
    \item \textbf{Applicare ricorsivamente questa procedura all'ultimo intervallo selezionato}.
\end{enumerate}

\highspace
Matematicamente parlando, dato $I^{(0)} = \left(a,b\right)$, e più in generale, $I^{(k)}$ il sotto-intervallo selezionato al passo $k$-esimo, si sceglie come $I^{\left(k+1\right)}$ il semi-intervallo di $I^{(k)}$ ai cui estremi $f$ cambia di segno.

\highspace
Questa procedura garantisce che ogni sotto-intervallo selezionato $I^{(k)}$ conterrà $\alpha$. Questo poiché la successione $\left\{x^{(k)}\right\}$ dei punti medi dei sotto-intervalli $I^{(k)}$ dovrà ineluttabilmente convergere a $\alpha$, in quanto la \textbf{lunghezza dei sotto-intervalli tende a} $0$ per $k$ che \textbf{tende all'infinito}.

\newpage

\noindent
Formalizziamo questa idea con un piccolo algoritmo. Ponendo:
\begin{equation*}
    a^{(0)} = a, \hspace{1em} b^{(0)} = b, \hspace{1em} I^{(0)} = \left(a^{(0)}, b^{(0)}\right), \hspace{1em} x^{(0)} = \dfrac{a^{(0)} + b^{(0)}}{2}
\end{equation*}
Al passo $k \ge 1$ il metodo di bisezione calcolerà il semi-intervallo $I^{(k)} = \left(a^{(k)}, b^{(k)}\right)$ dell'intervallo $I^{(k-1)} = \left(a^{\left(k-1\right)}, b^{\left(k-1\right)}\right)$, nel seguente modo (si ricorda che $\alpha$ è lo zero che si sta cercando):
\begin{enumerate}
    \item Calcolo $x^{\left(k-1\right)} = \dfrac{a^{\left(k-1\right)} + b^{\left(k-1\right)}}{2}$

    \item Se $f\left(x^{\left(k-1\right)}\right) = 0$:
    \begin{enumerate}
        \item Allora $\alpha = x^{\left(k-1\right)}$ e l'algoritmo \underline{termina}.
    \end{enumerate}
    
    \item Altrimenti, se $f\left(a^{\left(k-1\right)}\right) \cdot f\left(x^{\left(k-1\right)}\right) < 0$:
    \begin{enumerate}
        \item Si pone $a^{(k)} = a^{\left(k-1\right)}$
        \item Si pone $b^{(k)} = x^{\left(k-1\right)}$
        \item Si incrementa $k+1$ e si \underline{ripete ricorsivamente}.
    \end{enumerate}
    
    \item Altrimenti, se $f\left(x^{\left(k-1\right)}\right) \cdot f\left(b^{\left(k-1\right)}\right) < 0$:
    \begin{enumerate}
        \item Si pone $a^{(k)} = x^{\left(k-1\right)}$
        \item Si pone $b^{(k)} = b^{\left(k-1\right)}$
        \item Si incrementa $k+1$ e si \underline{ripete ricorsivamente}.
    \end{enumerate}
\end{enumerate}

\begin{examplebox}
    Data la funzione $f\left(x\right) = x^{2} - 1$, si parta da $a^{(0)} = -0.25$ e $b^{(0)} = 1.25$, e si applichi il metodo di bisezione:
    \begin{enumerate}
        \item Con $a^{(0)} = -0.25$ e $b^{(0)} = 1.25$:
        \begin{enumerate}
            \item Si calcola il punto medio:
            \begin{equation*}
                x^{(0)} = \dfrac{a^{(0)} + b^{(0)}}{2} = \dfrac{-0.25 + 1.25}{2} = 0.5
            \end{equation*}

            \item Si calcola la funzione con il punto medio come parametro:
            \begin{equation*}
                f\left(0.5\right) = 0.5^{2} - 1 = -0.75
            \end{equation*}

            \item Dato che la funzione nel punto medio non è uguale a zero, l'algoritmo deve continuare. Per farlo, bisogna sostituire il punto medio con uno dei due estremi. Per decidere quale dei due sostituire, è necessario capire in quale cambia valore la funzione. Si verifica inizialmente con $a^{(0)}$:
            \begin{equation*}
                \begin{array}{rcl}
                    f\left(a^{(0)}\right)f\left(x^{(0)}\right) < 0 &=& f\left(-0.25\right)f\left(0.5\right) < 0 \\ [.5em]
                    &=& \left(-0.9375 \cdot -0.75\right) < 0 \\ [.5em]
                    &=& 0.703125 \text{ \ding{55}}
                \end{array}
            \end{equation*}
            
            \item Si procede con l'algoritmo, provando adesso la $b^{(0)}$:
            \begin{equation*}
                \begin{array}{rcl}
                    f\left(x^{(0)}\right)f\left(b^{(0)}\right) < 0 &=& f\left(0.5\right)f\left(1.25\right) < 0 \\ [.5em]
                    &=& \left(-0.75 \cdot 0.5625\right) < 0 \\ [.5em]
                    &=& -0.421875 \text{ \ding{51}}
                \end{array}
            \end{equation*}

            \item Si pone $a^{(1)} = x^{(0)} = 0.5$

            \item Si pone $b^{(1)} = b^{(0)} = 1.25$

            \item Si incrementa $k$, $k = k + 1 = 0 + 1 = 1$
        \end{enumerate}

        \item Con $a^{(1)} = 0.5$ e $b^{(1)} = 1.25$:
        \begin{enumerate}
            \item Si calcola il punto medio:
            \begin{equation*}
                x^{(1)} = \dfrac{a^{(1)} + b^{(1)}}{2} = \dfrac{0.5 + 1.25}{2} = 0.875
            \end{equation*}

            \item Si calcola la funzione con il punto medio come parametro:
            \begin{equation*}
                f\left(0.875\right) = 0.875^{2} - 1 = -0.234375
            \end{equation*}

            \item Dato che la funzione nel punto medio non è uguale a zero, l'algoritmo deve continuare:
            \begin{equation*}
                \begin{array}{rcl}
                    f\left(a^{(1)}\right)f\left(x^{(1)}\right) < 0 &=& f\left(0.5\right)f\left(-0.234375\right) < 0 \\ [.5em]
                    &=& \left(-0.75 \cdot -0.945068359375\right) < 0 \\ [.5em]
                    &=& 0.70880126953125 \text{ \ding{55}}
                \end{array}
            \end{equation*}
            
            \item Si procede con l'algoritmo:
            \begin{equation*}
                \begin{array}{rcl}
                    f\left(x^{(1)}\right)f\left(b^{(1)}\right) < 0 &=& f\left(-0.234375\right)f\left(1.25\right) < 0 \\ [.5em]
                    &=& \left(-0.945068359375 \cdot 0.5625\right) < 0 \\ [.5em]
                    &=& -0.5316009521484375 \text{ \ding{51}}
                \end{array}
            \end{equation*}

            \item Si pone $a^{(2)} = x^{(1)} = 0.875$

            \item Si pone $b^{(2)} = b^{(1)} = 1.25$

            \item Si incrementa $k$, $k = k + 1 = 1 + 1 = 2$
        \end{enumerate}
    \end{enumerate}
    Si omettono i restanti calcoli per $k=2, k=3$, ma si lasciano qua di seguito i risultati:
    \begin{itemize}
        \item $I^{(2)} = \left(0.875, 1.25\right)$ e $x^{(2)} = 1.0625$
        \item $I^{(3)} = \left(0.875, 1.0625\right)$ e $x^{(2)} = 0.96875$
    \end{itemize}

    \newpage

    \noindent
    Nella seguente figura si possono vedere le iterazioni effettuate:
    \begin{center}
        \includegraphics[width=.7\textwidth]{img/metodo-di-bisezione-1.pdf}
        \captionof*{figure}{Iterazioni effettuate.\cite{quarteroni2017calcolo}}
    \end{center}
    
    \noindent
    Si noti che ogni intervallo $I^{(k)}$ contiene lo zero $\alpha$. Inoltre, la successione $\left\{x^{(k)}\right\}$ converge necessariamente allo zero $\alpha$ in quanto ad ogni passo l'ampiezza $\left| I^{(k)} \right| = b^{(k)} - a^{(k)}$ dell'intervallo $I^{(k)}$ si dimezza.
\end{examplebox}

\noindent
Il valore $I^{(k)}$ può essere riassunto come:
\begin{equation*}
    \left|I^{(k)}\right| = \left(\dfrac{1}{2}\right)^{k} \cdot \left| I^{(0)} \right|
\end{equation*}
E di conseguenza l'\textbf{errore al passo} $\bm{k}$ può essere calcolato come:
\begin{equation*}
    \left| e^{(k)} \right| = \left| x^{(k)} - \alpha \right| < \dfrac{1}{2} \cdot \left| I^{(k)} \right| = \left(\dfrac{1}{2}\right)^{k+1} \cdot \left(b-a\right)
\end{equation*}
Inoltre, data una certa \textbf{tolleranza} $\varepsilon$, per \textbf{garantire che l'errore al passo $k$ sia minore della tolleranza data} (ovvero, $\left| e^{(k)} \right| < \varepsilon$), basta applicare la seguente formula:
\begin{equation}
    k_{\min} > \log_{2} \left(\dfrac{b-a}{\varepsilon}\right) - 1
\end{equation}
Dove $k_{\min}$ rappresenta il \textbf{numero \underline{minimo} di iterazioni prima di trovare un intero che soddisfi la disuguaglianza}.

\begin{flushleft}
    \textcolor{Red2}{\faIcon{exclamation-triangle} \textbf{Possibile svantaggio}}
\end{flushleft}
Il metodo di bisezione \textbf{non garantisce una riduzione monotona dell'errore}, ma solo il dimezzamento dell'ampiezza dell'intervallo all'interno del quale si cerca lo zero. Infatti, \textbf{non viene tenuto conto del reale andamento di} $f$ e questo può provocare il mancato coinvolgimento di approssimazioni di $\alpha$ accurate.