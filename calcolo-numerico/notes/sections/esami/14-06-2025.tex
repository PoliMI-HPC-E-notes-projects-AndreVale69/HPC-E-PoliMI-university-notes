\section{Esami}

\subsection{14/06/2025}

\subsubsection*{Esercizio 1}

Si consideri la matrice $A \in \mathbb{R}^{n \times n}$ ed il vettore $\mathbf{x}^{*} \in \mathbb{R}$ tali che:
\begin{equation*}
    A = \begin{bmatrix}
        3 & -2 & 0 & 0 & \cdots & 0 \\
        -1 & 3 & -2 & 0 & \cdots & 0 \\
        0 & -1 & \ddots & \ddots & \ddots & \vdots \\
        0 & 0 & \ddots & \ddots & \ddots & 0 \\
        \vdots & \vdots & \vdots & -1 & 3 & -2 \\
        0 & 0 & \cdots & 0 & -1 & 3
    \end{bmatrix}
    \qquad
    \mathbf{x}^{*} = \begin{bmatrix}
        1 \\ 2 \\ 3 \\ \vdots \\ n - 1 \\ n
    \end{bmatrix}
\end{equation*}
\begin{enumerate}
    \item Utilizzando gli opportuni comandi MATLAB, si costruiscano la matrice $A$ ed il vettore $\mathbf{x}^{*}$ per $n = 10$. Si calcoli $\mathbf{b} \in \mathbb{R}^{10}$ tale che $\mathbf{b} = A \mathbf{x}^{*}$.

    \item Utilizzando il comando \texttt{lu} di MATLAB, si calcoli la fattorizzazione LU della matrice $A$ precedentemente costruita. Utilizzando il comando \texttt{spy}, si determini se si è verificato il fenomeno del \emph{fill-in}. Si commenti il risultato e si riportino tutti i comandi utilizzati.


    \item Utilizzando le matrice $L$ e $U$ calcolate al punto precedente e le funzioni \texttt{fwsub.m} e \texttt{bksub.m} si risolva il sistema lineare $A\mathbf{x} = \mathbf{b}$. Si riporti il risultato ottenuto ed i comandi utilizzati.
    

    \item Si calcoli la fattorizzazione LU della matrice $B = AA^{T}$, dove $A^{T}$ è la matrice trasposta di $A$. Si determini se è stato eseguito \emph{pivoting} e si riportino i comandi utilizzati. Utilizzare le matrici calcolate al punto precedente ed il comando \texttt{\textbackslash} di MATLAB per risolvere il sistema lineare $B\mathbf{x} = \mathbf{b}$.
    

    \item Definire il numero di condizionamento di una matrice $A$ e discutere la stabilità della soluzione numerica di un generico sistema lineare $A\mathbf{x} = \mathbf{b}$. Dimostrare come il residuo può essere usato come stimatore dell'errore.
    

    \item Calcolare il condizionamento della matrice $B$ ed il residuo per fornire una stima dell'errore relativo commesso nella risoluzione del sistema lineare $B\mathbf{x} = \mathbf{b}$. Riportare i comandi utilizzati.
\end{enumerate}

\newpage

\subsubsection*{Esercizio 2}

Si consideri la funzione nonlineare:
\begin{equation*}
    f(x) = \left(x - 2\right) e^{\left(x-1\right)}
\end{equation*}
Dotata dello zero $\alpha = 2$.
\begin{enumerate}
    \item Si scriva il metodo di Newton per la determinazione numerica delle radici di una funzione $f \, : \, \left[a,b\right] \to \mathbb{R}$ come metodo di iterazioni di punto di fisso e si scrivano con precisione le condizioni per la convergenza, specificandone l'ordine. Si definisca un criterio d'arresto affidabile dandone motivazione.
    
    \item Si verifichi graficamente che $\alpha = 2$ è una radice per la funzione $f(x)$. Si riportino tutti i comandi MATLAB usati.
    
    \item \phantomsection\label{itm:es2.3} Si consideri il metodo delle iterazioni di punto fisso per l'approssimazione dello zero $\alpha$ di $f(x)$ usando la funzione di iterazione
    \begin{equation*}
        \phi = x - \dfrac{\left(x-2\right) e^{\left(x-2\right)}}{2e-1}
    \end{equation*}
    Si verifichi graficamente che lo zero $\alpha$ di $f(x)$ coincide con un punto fisso di $\phi(x)$ e si riportino i comandi MATLAB usati.
    
    \item \phantomsection\label{itm:es2.4} Sempre considerando il metodo delle iterazioni di punto fisso e la funzione di iterazione (equazione al punto \ref{itm:es2.3}), si determini l'ordine di convergenza del metodo ad $\alpha$ per un'iterata iniziale $x^{(0)}$ ``sufficientemente'' vicino ad $\alpha$. Si motivi la risposta data alla luce della teoria.
    
    \item Considerando la funzione di iterazione $\phi(x)$ al punto \ref{itm:es2.3}, si utilizzi la funzione MATLAB \texttt{ptofis.m} con $x^{(0)} = 1.5$, tolleranza $tol = 10^{-4}$ e numero massimo di iterazioni $N_{max} = 1000$. Si riportino il numero di iterazioni $N$, i valori delle iterate $x^{(1)}$ e $x^{(2)}$, la differenza tra iterate successive $\left| x^{(N)} - x^{(N-1)} \right|$, insieme a tutti i comandi MATLAB utilizzati.
    
    \item Si utilizzi opportunamente la funzione MATLAB \texttt{stimap.m} per stimare l'ordine di convergenza. Si discutano criticamente i risultati ottenuti con quelli relativi al punto \ref{itm:es2.4} e si riportino i comandi MATLAB utilizzati.
\end{enumerate}

\newpage

\subsubsection*{Esercizio 3}

Data una funzione $f \in \mathcal{C}^{0} \left(\left[0, 1\right]\right)$, si consideri il problema di approssimazione numerica dell'integrale:
\begin{equation*}
    I(f) = \int_{0}^{1} f(x) \, dx
\end{equation*}
Mediante una formula di quadratura composita con $n$ sottointervalli equispaziati di $\left[0, 1\right]$.
\begin{enumerate}
    \item Si introduca, in generale, il problema dell'approssimazione numerica dell'integrale sopra definito, tramite formule di quadrature sia semplici che composite, definendo con precisione tutta la notazione utilizzata. Si introducano i concetti di ordine di convergenza e grado di esattezza.
    
    \item Sia ora $f(x) = \left(1 - 2x\right)e^{-2x}$. Utilizzando la funzione \texttt{quadcomp.m}, si calcoli l'integrale approssimato $I_{1}(f)$ corrispondente alla formula di quadratura di tipo semplice (ossia ottenuta con $n = 1$). Riportare il risultato ottenuto e i comandi MATLAB usati.

    \item \phantomsection\label{itm:es3.3} Verificare il grado di esattezza della formula di quadratura implementata in \texttt{quadcomp.m} calcolando gli errori $E_{i} = \left|I\left(p_i\right) - I_1 (p_i)\right|$ dove $p_i(x) = x^{i}$ per $i = 0, \dots, 4$. Riportare i comandi utilizzati e commentare i risultati ottenuti.

    \item Utilizzando la funzione \texttt{quadcomp.m}, si calcoli l'integrale approssimato $I_{n}(f)$ e, sapendo che $I(f) = e^{-2}$, si calcoli l'errore commesso $E_n (f) = \left| I(f) - I_n (f) \right|$ per $n = 5, 10, 20, 40$. Riportare i risultati ottenuti nella forma esponenziale e i comandi MATLAB usati.
    
    \item Riportare su un grafico in scala logaritmica l'andamento dell'errore in funzione di $h = 1 / n$ confrontandolo con l'andamento $h^{s}$ per un opportuno $s > 0$. Si riportino tutti i comandi utilizzati. Dedurre l'ordine di convergenza della formula di quadratura e commentare il risultato alla luce di quanto ottenuto al punto \ref{itm:es3.3}.
\end{enumerate}