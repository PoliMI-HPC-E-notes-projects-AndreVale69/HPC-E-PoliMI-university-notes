\section{Esami}

\subsection{24/07/2025}

\subsubsection*{Esercizio 1}

Si consideri la matrice $A \in \mathbb{R}^{n \times n}$ ed il vettore $\mathbf{b} \in \mathbb{R}^{n}$ tali che:
\begin{equation*}
    A = \begin{bmatrix}
        2 & -1 & 0 & 0 & \cdots & 0 \\
        -1 & 2 & -1 & 0 & \cdots & 0 \\
        0 & -1 & \ddots & \ddots & \ddots & \vdots \\
        0 & 0 & \ddots & \ddots & \ddots & 0 \\
        \vdots & \vdots & \vdots & -1 & 2 & -1 \\
        0 & 0 & \cdots & 0 & -1 & 2 \\
    \end{bmatrix}
    \hspace{3em}
    \mathbf{b} = \begin{bmatrix}
        -1 \\
        -1 \\
        -1 \\
        \vdots \\
        -1 \\
        -1 \\
    \end{bmatrix}
\end{equation*}
\begin{enumerate}
    \item Utilizzando gli opportuni comandi Matlab, si costruiscano la matrice $A$ ed il vettore $\mathbf{b}$ per $n = 7$. Utilizzando il comando \textbackslash si risolva il sistema lineare $A\mathbf{x} = \mathbf{b}$. Riportare $\mathbf{x}$ e tutti i comandi Matlab usati.
    
    \textcolor{Green3}{\textbf{\emph{Soluzione}}}
    \begin{lstlisting}[language=MATLAB]
% dimensione di n
n = 7;
% creiamo un vettore di 1 e aggiungiamo 1
A = diag(ones(n,1)+1) +         % diag. principale
    diag(ones(n-1,1)-2, 1) +    % diag. sopra principale
    diag(ones(n-1,1)-2, -1);    % diag. sotto principale
% creiamo vettore b
b = -ones(n,1);
% calcola soluzione di x
x = A\b;\end{lstlisting}
Il vettore \texttt{x} produce il seguente risultato:
\begin{equation*}
    \mathbf{x} = \begin{bmatrix}
        -3.5 \\
        -6 \\
        -7.5 \\
        -8 \\
        -7.5 \\
        -6 \\
        -3.5
    \end{bmatrix}
\end{equation*}

    \item Elencare e discutere:
    \begin{enumerate}
        \item Le condizioni sufficienti per la convergenza del metodo di Gauss-Seidel.
        \item Le condizioni necessarie e sufficienti per la convergenza del metodo di Gauss-Seidel.
    \end{enumerate}
    Data $B_{GS}$ la matrice di iterazione di Gauss-Seidel, dimostrare che\break $\left\| B_{GS} \right\| < 1$ implica la convergenza.

    \textcolor{Green3}{\textbf{\emph{Soluzione.}}} Le condizioni sufficienti per la convergenza del metodo di Gauss-Seidel sono 3:
    \begin{itemize}
        \item La matrice $A$ deve essere a \textbf{dominanza diagonale stretta per righe}:
        \begin{equation*}
            \left|a_{ii}\right| > \displaystyle\sum_{j \ne i} \left|a_{ij}\right| \hspace{1em} \forall i
        \end{equation*}
        \item La matrice $A$ deve essere \textbf{simmetrica definita positiva (SPD)}:
        \begin{equation*}
            \mathbf{x}^{T} A \mathbf{x} > 0 \hspace{1em} \forall \mathbf{x} \ne 0
        \end{equation*}
        \item La matrice $A$ deve essere una \textbf{Matrice di classe M (M-Matrice) non singolare}. Ovvero una matrice con diagonale positiva e elementi extradiagonali (non sulla diagonale) non positivi.
    \end{itemize}
    L'unica condizione necessarie e sufficiente per la convergenza è quando il raggio spettrale della matrice di iterazione $B_{GS}$ è minore di uno:
    \begin{equation*}
        \text{convergenza} \iff \rho(B_{GS}) < 1
    \end{equation*}
    \begin{proof}[Dimostrazione ($\left\| B_{GS} \right\| < 1$ implica la convergenza)]
        Il metodo di Gauss-Seidel si scrive
        \begin{equation*}
            x^{(k+1)} = B_{GS}x^{(k)} + c
        \end{equation*}
        Definendo l'errore $e^{(k)} = x^{(k)} - x^{*}$, si ha
        \begin{equation*}
            e^{(k+1)} = B_{GS} e^{(k)} \quad \Rightarrow \quad e^{(k)} = B_{GS}^k e^{(0)}
        \end{equation*}
        Se $\|B_{GS}\| < 1$, allora:
        \begin{equation*}
            \|e^{(k)}\| \le \|B_{GS}\|^k \|e^{(0)}\| \to 0,
        \end{equation*}
        Per $k \to \infty$. Quindi $x^{(k)} \to x^\ast$: il metodo converge.
    \end{proof}
\end{enumerate}