\section{Laboratorio}

\subsection{Introduzione al linguaggio MATLAB}

L'introduzione al linguaggio di programmazione MATLAB sarà molto rapido. Si assume dunque che l'interfaccia grafica sia familiare e che concetti base di programmazione (per esempio \dquotes{che cos'è una variabile?}) siano ben noti.

\highspace
In MATLAB, l'assegnazione di scalari a delle variabili è classica, quindi si utilizza il simbolo uguale: \texttt{a = 1} (assegnazione del valore \texttt{1} alla variabile \texttt{a}). Inoltre, il linguaggio è \emph{case sensitive}, di conseguenza la variabile \texttt{a} è diversa dalla variabile \texttt{A}. Alcuni comandi utili e generali:
\begin{itemize}
    \item \texttt{help \emph{nome-comando}}, per avere informazioni in più riguardo al comando \texttt{\emph{nome-comando}};

    \item \texttt{clear \emph{nome-variabile}}, per rimuovere la variabile \texttt{\emph{nome-variabile}} dalla memoria. Se non viene inserito il \texttt{\emph{nome-variabile}}, vengono rimosse tutte le variabili dalla memoria.

    \item \texttt{who}, per visualizzare le variabili attualmente in memoria.

    \item \texttt{clc}, per ripulire la \emph{Command Window}.
\end{itemize}

\begin{table}[!htp]
    \centering
    \begin{tabular}{@{} p{21em} l @{}}
        \toprule
        \textbf{Argomento} & \textbf{Pagina} \\
        \midrule
        Well-known variables    & Pag. \hyperlink{
            lab: Well-known variables
        }{
            \hypergetpageref{lab: Well-known variables}
        } \\
        Cambiare il formato delle variabili: \texttt{format} & Pag. 
        \hyperlink{
            lab: Cambiare il formato delle variabili: format
        }{
            \hypergetpageref{lab: Cambiare il formato delle variabili: format}
        } \\
        Assegnamento di vettori e matrici & Pag. 
        \hyperlink{
            lab: Assegnamento di vettori e matrici
        }{
            \hypergetpageref{lab: Assegnamento di vettori e matrici}
        } \\
        Operazioni su vettori e matrici & Pag. \hyperlink{
            lab: Operazioni su vettori e matrici
        }{
            \hypergetpageref{lab: Operazioni su vettori e matrici}
        } \\
        Funzioni intrinseche per vettori e matrici & Pag. \hyperlink{
            lab: Funzioni intrinseche per vettori e matrici
        }{
            \hypergetpageref{lab: Funzioni intrinseche per vettori e matrici}
        } \\
        Funzioni matematiche elementari & Pag. \hyperlink{
            lab: Funzioni matematiche elementari
        }{
            \hypergetpageref{lab: Funzioni matematiche elementari}
        } \\
        Funzioni per definire vettori o matrici particolari & Pag. \hyperlink{
            lab: Funzioni per definire vettori o matrici particolari
        }{
            \hypergetpageref{lab: Funzioni per definire vettori o matrici particolari}
        } \\
        \bottomrule
    \end{tabular}
    \caption{Argomenti trattati.}
\end{table}

\begin{flushleft}
    \large
    \hypertarget{
        lab: Well-known variables
    }{
        \textcolor{Red3}{\textbf{Well-known variables}}
    }
    \label{lab: Well-known variables}
\end{flushleft}
Esistono alcune variabili che sono ben note e hanno valori prestabiliti. Tra le più importanti:
\begin{itemize}
    \item \texttt{pi}, che rappresenta il $\pi$ e MATLAB gli assegna il valore \texttt{3.1416}
    
    \item \texttt{i}, che rappresenta l'unità immaginaria e MATLAB gli assegna il valore \texttt{0.0000 + 1.0000i}
    
    \item \texttt{eps}, che rappresenta il più piccolo valore rappresentabile nel calcolatore (PC) attualmente in uso. Solitamente, \texttt{eps} ritorna il valore \texttt{2.2204e-16}.
\end{itemize}
Questo tipo di variabili possono essere ridefinite, ma \underline{non} è una \emph{good practice}.

\newpage

\begin{flushleft}
    \large
    \hypertarget{
        lab: Cambiare il formato delle variabili: format
    }{
        \textcolor{Red3}{\textbf{Cambiare il formato delle variabili: \texttt{format}}}
    }
    \label{lab: Cambiare il formato delle variabili: format}
\end{flushleft}
Il comando \texttt{format} è utilizzato per cambiare il formato con cui sono rappresentate le variabili. MATLAB \underline{non} cambia la precisione della variabile (quindi non si ottiene una precisione maggiore dopo la virgola), ma modifica soltanto la rappresentazione. Di default MATLAB utilizza una rappresentazione di tipo \texttt{short}. Tra i più utilizzati (di default \texttt{pi} è uguale a \texttt{3.1416}):
\begin{itemize}
    \item \texttt{default} per reimpostare la rappresentazione di default.

    \item Decimale:
    \begin{itemize}
        \item \texttt{short}, rappresentazione a 5 cifre:
        \lstinputlisting[language=MATLAB]{code/introduzione-al-linguaggio-MATLAB/short.m}

        \item \texttt{long}, rappresentazione a 15 cifre:
        \lstinputlisting[language=MATLAB]{code/introduzione-al-linguaggio-MATLAB/long.m}
    \end{itemize}

    \item \emph{Floating point}:
    \begin{itemize}
        \item \texttt{short e}, rappresentazione a 5 cifre floating point:
        \lstinputlisting[language=MATLAB]{code/introduzione-al-linguaggio-MATLAB/short-floating-point.m}

        \item \texttt{long e}, rappresentazione a 15 cifre floating point:
        \lstinputlisting[language=MATLAB]{code/introduzione-al-linguaggio-MATLAB/long-floating-point.m}
    \end{itemize}
\end{itemize}
Altri formati si possono trovare nella \href{https://uk.mathworks.com/help/releases/R2024a/matlab/ref/format.html#btiwmh5-1-style}{documentazione ufficiale}.

\newpage

\begin{flushleft}
    \large
    \hypertarget{
        lab: Assegnamento di vettori e matrici
    }{
        \textcolor{Red3}{\textbf{Assegnamento di vettori e matrici}}
    }
    \label{lab: Assegnamento di vettori e matrici}
\end{flushleft}
\begin{itemize}
    \item \textbf{Vettore riga}, si può creare utilizzando uno spazio tra i valori o una virgola \texttt{,}:
    \lstinputlisting[language=MATLAB]{code/introduzione-al-linguaggio-MATLAB/vettori-1.m}

    \item \textbf{Vettore colonna}, si crea usando il punto e virgola \texttt{;}:
    \lstinputlisting[language=MATLAB]{code/introduzione-al-linguaggio-MATLAB/vettori-2.m}
\end{itemize}
Talvolta può essere utile la generazione automatica di un vettore riga (sono ammessi anche i valori negativi e con la virgola ovviamente):
\begin{itemize}
    \item \textbf{Vettore riga generato linearmente}, si crea usando i due punti e specificando il valore di inizio e il valore di fine:
    \lstinputlisting[language=MATLAB]{code/introduzione-al-linguaggio-MATLAB/vettori-3.m}

    \item \textbf{Vettore riga generato usando un passo}, si crea usando i due punti e specificando (in ordine) il valore di inizio, il \dquotes{salto}, e il valore di fine. Nel caso in cui il salto sia troppo grande e si superi il valore di fine, MATLAB prenderà il primo valore ammissibile:
    \lstinputlisting[language=MATLAB]{code/introduzione-al-linguaggio-MATLAB/vettori-4.m}

    \item \textbf{Vettore riga generato con valori uniformemente distanziati}, si crea usando la funzione \texttt{linspace}, la quale accetta tre parametri:
    \begin{itemize}
        \item \texttt{x1}, valore di partenza.
        \item \texttt{x2}, valore di fine.
        \item \texttt{n}, numero di valori da generare; se non specificato, di default è \texttt{100}; se il valore inserito è zero o minore, viene creato un vettore vuoto.
    \end{itemize}
    \lstinputlisting[language=MATLAB]{code/introduzione-al-linguaggio-MATLAB/vettori-5.m}
\end{itemize}
Le matrici possono essere create a mano o usando la combinazione delle tecniche viste in precedenza:
\begin{itemize}
    \item \textbf{Matrice}, le righe si creano usando gli spazi e le colonne si creano usando il punto e virgola:
    \lstinputlisting[language=MATLAB]{code/introduzione-al-linguaggio-MATLAB/matrici-1.m}

    \item \textbf{Matrice creata usando la generazione lineare dei vettori}, si possono utilizzare le tecniche precedenti e i punti e virgola:
    \lstinputlisting[language=MATLAB]{code/introduzione-al-linguaggio-MATLAB/matrici-2.m}
\end{itemize}

\newpage

\begin{flushleft}
    \large
    \hypertarget{
        lab: Operazioni su vettori e matrici
    }{
        \textcolor{Red3}{\textbf{Operazioni su vettori e matrici}}
    }
    \label{lab: Operazioni su vettori e matrici}
\end{flushleft}
\begin{itemize}
    \item \textbf{Trasposizione}, la classica operazione eseguita con le matrici o vettori, si esegue con la keyword \texttt{'} oppure usando la funzione \texttt{transpose}:
    \lstinputlisting[language=MATLAB]{code/introduzione-al-linguaggio-MATLAB/operazioni-vettori-matrici-1.m}

    \item \textbf{Somma e sottrazione}
    \begin{itemize}
        \item Tra \textbf{vettore e scalare}:
        \lstinputlisting[language=MATLAB]{code/introduzione-al-linguaggio-MATLAB/operazioni-vettori-matrici-2.m}

        \item Tra \textbf{vettore e matrice}:
        \lstinputlisting[language=MATLAB]{code/introduzione-al-linguaggio-MATLAB/operazioni-vettori-matrici-3.m}

        \item Tra \textbf{matrice e scalare}:
        \lstinputlisting[language=MATLAB]{code/introduzione-al-linguaggio-MATLAB/operazioni-vettori-matrici-4.m}

        \item Tra \textbf{matrice e matrice}:
        \lstinputlisting[language=MATLAB]{code/introduzione-al-linguaggio-MATLAB/operazioni-vettori-matrici-5.m}
    \end{itemize}

    \item \textbf{Prodotto}
    \begin{itemize}
        \item \textbf{Prodotto matriciale}:
        \lstinputlisting[language=MATLAB]{code/introduzione-al-linguaggio-MATLAB/operazioni-vettori-matrici-6.m}

        \item \textbf{Prodotto punto per punto}, in MATLAB è possibile moltiplicare ogni cella di una matrice (o vettore) per la corrispettiva cella della matrice (o vettore) moltiplicata. La keyword utilizzata è \texttt{.*}:
        \lstinputlisting[language=MATLAB]{code/introduzione-al-linguaggio-MATLAB/operazioni-vettori-matrici-7.m}
    \end{itemize}

    \item \textbf{Potenza}
    \begin{itemize}
        \item \textbf{Potenza matriciale}:
        \lstinputlisting[language=MATLAB]{code/introduzione-al-linguaggio-MATLAB/operazioni-vettori-matrici-8.m}

        \item \textbf{Potenza punto per punto}, come per il prodotto, è possibile elevare al quadrato ogni valore della matrice (o vettore):
        \lstinputlisting[language=MATLAB]{code/introduzione-al-linguaggio-MATLAB/operazioni-vettori-matrici-9.m}
    \end{itemize}
\end{itemize}

\newpage

\begin{flushleft}
    \large
    \hypertarget{
        lab: Funzioni intrinseche per vettori e matrici
    }{
        \textcolor{Red3}{\textbf{Funzioni intrinseche per vettori e matrici}}
    }
    \label{lab: Funzioni intrinseche per vettori e matrici}
\end{flushleft}
Qua di seguito si elencano le funzioni più importanti da utilizzare per i vettori e le matrici.
\begin{itemize}
    \item \texttt{\textbf{size}}, restituisce la dimensione del vettore o della matrice nel formato \texttt{\emph{righe} \emph{colonne}}. Specificando anche un valore (o vettore) come parametro, la funzione restituisce la dimensione (un vettore contenente le dimensioni richieste) nella \dquotes{dimensione} richiesta:
    \lstinputlisting[language=MATLAB]{code/introduzione-al-linguaggio-MATLAB/operazioni-vettori-matrici-10.m}

    \item \texttt{\textbf{length}}, restituisce la lunghezza del vettore e per le matrici restituisce il numero degli elementi per ogni riga:
    \lstinputlisting[language=MATLAB]{code/introduzione-al-linguaggio-MATLAB/operazioni-vettori-matrici-11.m}

    \item \texttt{\textbf{max}}, \texttt{\textbf{min}}, calcolano rispettivamente il massimo e il minimo valore delle componenti di un vettore; per le matrici viene presa in considerazione ogni colonna e calcolato il massimo o minimo:
    \lstinputlisting[language=MATLAB]{code/introduzione-al-linguaggio-MATLAB/operazioni-vettori-matrici-12.m}

    \item \texttt{\textbf{sum}}, \texttt{\textbf{prod}}, calcola rispettivamente la somma e il prodotto degli elementi che compongono il vettore; nel caso di una matrice, viene presa in considerazione ogni colonna e calcolata la somma o il prodotto. Inoltre, i due comandi possono prendere un argomento in più per eseguire il calcolo in una dimensione specifica (cosa sensata con le matrici):
    \lstinputlisting[language=MATLAB]{code/introduzione-al-linguaggio-MATLAB/operazioni-vettori-matrici-13.m}

    \item \texttt{\textbf{norm}}, la norma di un vettore o di una matrice. Passando un vettore o un matrice, viene calcolata di default la norma euclidea (norma 2):
    \begin{equation*}
        \left|\left| \mathrm{v} \right|\right|_{2} = \sqrt{\sum_{i = 2}^{\texttt{length(v)} \mathrm{v}_{i}^{2}}}
    \end{equation*}
    Passando un valore aggiuntivo, esso rappresenterà l'ordine della norma:
    \begin{equation*}
        \left|\left| \mathrm{v} \right|\right|_{n} = \left( \sum_{i = 2}^{\texttt{length(v)} \left| \mathrm{v}_{i} \right|^{n}} \right)^{\frac{1}{n}}
    \end{equation*}
    Infine, con \texttt{inf} viene calcolata la norma infinito:
    \begin{equation*}
        \left|\left| \mathrm{v} \right|\right|_{\infty} = \underset{1 \le i \le \texttt{length(v)}}{\max} \left| \mathrm{v}_{i} \right|
    \end{equation*}
    \lstinputlisting[language=MATLAB]{code/introduzione-al-linguaggio-MATLAB/operazioni-vettori-matrici-14.m}

    \item \texttt{\textbf{abs}}, rappresenta il valore assoluto e restituisce il vettore o matrice dopo aver applicato il valore assoluto a ciascun elemento:
    \lstinputlisting[language=MATLAB]{code/introduzione-al-linguaggio-MATLAB/operazioni-vettori-matrici-15.m}

    \newpage

    \item \texttt{\textbf{diag}}, estrae la diagonale di una matrice esistente, oppure ne crea una con i valori dati come input. Inoltre, può creare una matrice con la diagonale spostata a seconda del valore dato (si veda l'esempio):
    \lstinputlisting[language=MATLAB]{code/introduzione-al-linguaggio-MATLAB/operazioni-vettori-matrici-16.m}
\end{itemize}

\newpage

\begin{flushleft}
    \large
    \hypertarget{
        lab: Funzioni matematiche elementari
    }{
        \textcolor{Red3}{\textbf{Funzioni matematiche elementari}}
    }
    \label{lab: Funzioni matematiche elementari}
\end{flushleft}
Qua di seguito una lista di alcune funzioni matematiche elementari. Gli esempi e la sintassi non verranno mostrati poiché è sempre la medesima:
\begin{equation*}
    \texttt{\emph{funzione}(\emph{parametro})}
\end{equation*}
\begin{table}[!htp]
    \centering
    \begin{tabular}{@{} l l @{}}
        \toprule
        \textbf{Funzione} & \textbf{Comando} \\
        \midrule
        \textbf{Radice quadrata}                & \texttt{sqrt} \\
        \textbf{Esponenziale}                   & \texttt{exp} \\
        \textbf{Logaritmo Naturale}             & \texttt{log} \\
        \textbf{Logaritmo In Base \texttt{2}}   & \texttt{log2} \\
        \textbf{Logaritmo In Base \texttt{10}}  & \texttt{log10} \\
        \textbf{Seno}                           & \texttt{sin} \\
        \textbf{Arcoseno}                       & \texttt{asin} \\
        \textbf{Coseno}                         & \texttt{cos} \\
        \textbf{Arcocoseno}                     & \texttt{acos} \\
        \textbf{Tangente}                       & \texttt{tan} \\
        \bottomrule
    \end{tabular}
    \caption{Funzioni matematiche elementari.}
\end{table}

\longline

\begin{flushleft}
    \large
    \textcolor{Red3}{\textbf{Iterazione con il ciclo for}}
\end{flushleft}
In MATLAB il ciclo for viene eseguito con la seguente sintassi.
\lstinputlisting[language=MATLAB]{code/introduzione-al-linguaggio-MATLAB/for-1.m}
Di seguito si riporta un ciclo \texttt{for} che itera sulla diagonale secondaria di una matrice:
\lstinputlisting[language=MATLAB]{code/introduzione-al-linguaggio-MATLAB/for-2.m}

\newpage

\begin{flushleft}
    \large
    \hypertarget{
        lab: Funzioni per definire vettori o matrici particolari
    }{
        \textcolor{Red3}{\textbf{Funzioni per definire vettori o matrici particolari}}
    }
    \label{lab: Funzioni per definire vettori o matrici particolari}
\end{flushleft}
In queste pagine vengono presentate alcuni funzioni utili che consentono di creare matrici o vettori \dquotes{particolari}.
\begin{itemize}
    \item \textbf{Vettore/Matrice nulla}, con la funzione \texttt{zeros} è possibile creare una matrice o un vettore di tutti zeri. I parametri ammessi corrispondono alla dimensione del vettore o matrice:
    \lstinputlisting[language=MATLAB]{code/introduzione-al-linguaggio-MATLAB/vettori-matrici-particolari-1.m}

    \item \textbf{Vettore unario/Matrice unaria}, con la funzione \texttt{ones} è possibile creare una matrice o un vettore di tutti uni. I parametri ammessi corrispondono alla dimensione del vettore o matrice:
    \lstinputlisting[language=MATLAB]{code/introduzione-al-linguaggio-MATLAB/vettori-matrici-particolari-2.m}

    \newpage

    \item \textbf{Matrice identità}, con la funzione \texttt{eye} è possibile creare una matrice identità. I parametri ammessi corrispondono alla dimensione del vettore o matrice:
    \lstinputlisting[language=MATLAB]{code/introduzione-al-linguaggio-MATLAB/vettori-matrici-particolari-3.m}

    \item \textbf{Matrice/Vettore riga di numeri casuali interi e non}, con il comando \texttt{rand} si genera una matrice di numeri casuali nell'intervallo $\left[0,1\right]$ con la virgola, mentre con il comando \texttt{randi} si genera una matrice di numeri casuali interi (primo parametro deve essere specificato il range dei valori):
    \lstinputlisting[language=MATLAB]{code/introduzione-al-linguaggio-MATLAB/vettori-matrici-particolari-4.m}
\end{itemize}

\newpage

\subsubsection{Esercizio}

Creare una funzione (file) chiamato \texttt{mat\_hilbert.m} che fornisca la matrice di Hilbert avente una generica dimensione \texttt{n}. Ogni cella della matrice di Hilbert deve rispettare la seguente condizione:
\begin{equation*}
    a_{ij} = \dfrac{1}{i + j - 1}
\end{equation*}
Dopo aver creato la funzione, utilizzare la funzione nativa di MATLAB \texttt{hilb}, per verificare il risultato ottenuto.

\begin{flushleft}
    \textcolor{Green4}{\textbf{Soluzione}}
\end{flushleft}
Il codice non ha bisogno di grandi spiegazioni. Vi è un controllo iniziale per verificare l'argomento inserito dall'utente e successivamente due cicli \texttt{for} per popolare la matrice:
\lstinputlisting[language=MATLAB]{code/introduzione-al-linguaggio-MATLAB/mat_hilbert.m}
Il risultato:
\lstinputlisting[language=MATLAB]{code/introduzione-al-linguaggio-MATLAB/mat_hilbert-res.m}    