\documentclass[a4paper]{article}
\usepackage[T1]{fontenc}			% \chapter package
\usepackage[english]{babel}
\usepackage[english]{isodate}  		% date format
\usepackage{graphicx}				% manage images
\usepackage{amsfonts}
\usepackage{booktabs}				% high quality tables
\usepackage{amsmath}				% math package
\usepackage{amssymb}				% another math package (e.g. \nexists)
\usepackage{bm}                     % bold math symbols
\usepackage{mathtools}				% emphasize equations
\usepackage{stmaryrd} 				% '\llbracket' and '\rrbracket'
\usepackage{amsthm}					% better theorems
\usepackage{enumitem}				% manage list
\usepackage{pifont}					% nice itemize
\usepackage{cancel}					% cancel math equations
\usepackage{caption}				% custom caption
\usepackage[]{mdframed}				% box text
\usepackage{multirow}				% more lines in a table
\usepackage{textcomp, gensymb}		% degree symbol
\usepackage[x11names]{xcolor}		% RGB color
\usepackage{tcolorbox}				% colorful box
\usepackage{multicol}				% more rows in a table (used for the lists)
\usepackage{listings}
\usepackage{url}
\usepackage{qrcode}
\usepackage{fontawesome5}
\usepackage{ragged2e}
\usepackage{cite}                   % references
\usepackage{imakeidx}               % index
\makeindex[program=makeindex, columns=1,
           title=Index, 
           intoc,
           options={-s index-style.ist}]
\usepackage{fancyhdr}


\definecolor{codegreen}{rgb}{0,0.6,0}
\definecolor{codegray}{rgb}{0.5,0.5,0.5}
\definecolor{codepurple}{rgb}{0.58,0,0.82}
\definecolor{backcolour}{rgb}{0.95,0.95,0.92}
\lstdefinestyle{mystyle}{
    backgroundcolor=\color{backcolour},   
    commentstyle=\color{codegreen},
    keywordstyle=\color{magenta},
    numberstyle=\tiny\color{codegray},
    stringstyle=\color{codepurple},
    basicstyle=\ttfamily\footnotesize,
    breakatwhitespace=false,         
    breaklines=true,                 
    captionpos=b,                    
    keepspaces=true,                 
    numbers=left,                    
    numbersep=5pt,                  
    showspaces=false,                
    showstringspaces=false,
    showtabs=false,                  
    tabsize=2
}
\lstset{style=mystyle}


% draw a frame around given text
\newcommand{\framedtext}[1]{%
	\par%
	\noindent\fbox{%
		\parbox{\dimexpr\linewidth-2\fboxsep-2\fboxrule}{#1}%
	}%
}


% table of content links
\usepackage{xcolor}
\usepackage[linkcolor=black, citecolor=blue, urlcolor=cyan]{hyperref} % hypertexnames=false
\hypersetup{
	colorlinks=true
}


\newtheorem{theorem}{\textcolor{Red3}{\underline{Theorem}}}
\renewcommand{\qedsymbol}{QED}
\newcommand{\dquotes}[1]{``#1''}
\newcommand{\longline}{\noindent\rule{\textwidth}{0.4pt}}
\newcommand{\circledtext}[1]{\raisebox{.5pt}{\textcircled{\raisebox{-.9pt}{#1}}}}
\newcommand{\definition}[1]{\textcolor{Red3}{\textbf{#1}}\index{#1}}
\newcommand{\example}[1]{\textcolor{Green4}{\textbf{#1}}}
\newcommand{\highspace}{\vspace{1.2em}\noindent}


\begin{document}
    \newcounter{definition}[section]
    \newcounter{example}[section]
    
    \newtcolorbox[use counter = definition]{definitionbox}{%
        colback=red!5!white,
        colframe=red!75!black,
        fonttitle=\bfseries,
        title=Definition \thetcbcounter %
    }
    
    \newtcolorbox[use counter = example]{examplebox}{%
        colback=Green4!5!white,
        colframe=Green4!75!black,
        fonttitle=\bfseries,
        title=Example \thetcbcounter %
    }

    \newtcolorbox[]{deepeningbox}[1][]{%
        colback=DarkOrange3!5!white,
        colframe=DarkOrange3!75!black,
        fonttitle=\bfseries,
        title={Deepening#1} %
    }

    \author{260236}
	\title{Computing Infrastructures - Notes}
	\date{\printdayoff\today}
	\maketitle

	\newpage

    \section*{Preface}

    Every theory section in these notes has been taken from two sources:
    \begin{itemize}
        \item The Datacenter as a Computer: Designing Warehouse-Scale Machines, Third Edition.\cite{barroso2022datacenter}

        \item Quantitative System Performance: Computer System Analysis Using Queueing Network Models.\cite{lazowska1984quantitative}
    \end{itemize}
    About:
    \begin{itemize}
        \item[\faIcon{github}] \href{https://github.com/PoliMI-HPC-E-notes-projects-AndreVale69/HPC-E-PoliMI-university-notes}{GitHub repository}
    \end{itemize}
    
    \newpage
	
	\tableofcontents

    \newpage

    \pagestyle{fancy}
    \fancyhead{} % clear all header fields
    \fancyhead[R]{\nouppercase{\leftmark\hfill\rightmark}}

    \section{Introduction: definition of Data Center and Computing Infrastructure}

    There's no single definition of a Data Center, but it can be summarized as follows.
    
    \begin{definitionbox}\label{Data Center definition}
        \definition{Data Center}s are buildings where multiple servers and communication gear are co-located because of their common environmental requirements and physical security needs, and for ease of maintenance.\cite{barroso2022datacenter}
    \end{definitionbox}

    \begin{definitionbox}
        A \definition{Computing Infrastructure} (or IT Infrastructure) is a technological infrastructure that provides hardware and software for computation to other systems and services.
    \end{definitionbox}

    \noindent
    Traditional data centres have the following characteristics:
    \begin{itemize}
        \item \textbf{Host a large number} of relatively small or medium sized \textbf{applications};
        
        \item Each \textbf{application is running on a dedicated HW infrastructure} that is de-coupled and protected from other systems in the same facility;
        
        \item \textbf{Applications tend not to communicate each other}.
    \end{itemize}
    Those \textbf{data centers host} hardware and software \textbf{for multiple organizational units} or even \textbf{different companies}.

    \newpage

    \section{Hardware Infrastructures}

    \subsection{System-level}

    \subsubsection{Computing Infrastructures and Data Center Architectures}

    \begin{center}
        \textcolor{Red2}{\textbf{Overview of Computing Infrastructures}}
    \end{center}

    \noindent
    A number of computing infrastructures exist:
    \begin{itemize}
        \item \textbf{Cloud} offers virtualized computing, storage and network resources with highly-elastic capacity.
        
        
        \item \textbf{Edge Servers} are on-premises hardware resources that perform more compute-intensive data processing.
        
        In other words, an edge server is a piece of hardware that performs data computation at the end (or \dquotes{edge}) of a network. Like a regular server, an edge server can provide compute, networking, and storage functions.\footnote{More info \href{https://phoenixnap.com/blog/edge-server}{here}.}

        
        \item \textbf{IoT} and \textbf{AI-enabled Edge Sensors} are hardware devices where the data acquisition and partial processing can be performed at the edge of the network.
    \end{itemize}

    \begin{figure}[!htp]
        \centering
        \includegraphics[width=\textwidth]{img/example-computing-infrastructure-1.pdf}
        \caption{An \example{example} of Computing Infrastructures.\cite{computing-infrastructures-slides}}
    \end{figure}

    \noindent
    The \definition{Computing Continuum}, a novel paradigm that extends beyond the current silos of cloud and edge computing, can enable the seamless and dynamic deployment of applications across diverse infrastructures.\cite{marino2023computing}

    \newpage

    \begin{figure}[!htp]
        \centering
        \includegraphics[width=\textwidth]{img/computing-continuum-1.png}
        \caption{The Computing Continuum.\cite{computing-infrastructures-slides}}
    \end{figure}

    \noindent
    In the following pages, we analyze the computing infrastructures mentioned in the previous example.

    \longline

    \begin{center}
        \large
        \textcolor{Red3}{\textbf{Data Centers}}
    \end{center}
    
    \noindent
    The definition of a Data Centers can be found on page \pageref{Data Center definition}.

    \begin{flushleft}
        \textcolor{Green3}{\faIcon{check} \textbf{Data Centers Advantages}}
    \end{flushleft}
    \begin{itemize}
        \item \textbf{Lower IT costs}.
        \item \textbf{High Performance}.
        \item \textbf{Instant software updates}.
        \item \textbf{\dquotes{Unlimited} storage capacity}.
        \item \textbf{Increased data reliability}.
        \item \textbf{Universal data access}.
        \item \textbf{Device Independence}.
    \end{itemize}

    \begin{flushleft}
        \textcolor{Red2}{\faIcon{thumbs-down} \textbf{Data Centers Disadvantages}}
    \end{flushleft}
    \begin{itemize}
        \item \textbf{Require a constant internet connection}.
        \item \textbf{Do not work well with low-speed connections}.
        \item \textbf{Hardware Features might be limited}.
        \item \textbf{Privacy and security issues}.
        \item \textbf{High power Consumption}.
        \item \textbf{\underline{Latency in taking decision}}.
    \end{itemize}

    \newpage

    \begin{center}
        \large
        \textcolor{Red3}{\textbf{Internet-of-Things (IoT)}}
    \end{center}

    \noindent
    An \definition{Internet of Things (IoT)} \textbf{device is any everyday object embedded with sensors, software, and internet connectivity}.

    \highspace
    This allows to collect and exchange data with other devices and systems, typically over the internet, with limited need of process and store data.

    \highspace
    Some \example{examples} are \href{https://www.arduino.cc/}{Arduino}, \href{https://www.st.com/en/microcontrollers-microprocessors/stm32-32-bit-arm-cortex-mcus.html}{STM32}, \href{https://en.wikipedia.org/wiki/ESP32}{ESP32}, \href{https://docs.particle.io/argon/}{Particle Argon}.

    \begin{flushleft}
        \textcolor{Green3}{\faIcon{check} \textbf{Internet-of-Things Advantages}}
    \end{flushleft}
    \begin{itemize}
        \item \textbf{Highly Pervasive}.
        \item \textbf{Wireless connection}.
        \item \textbf{Battery Powered}.
        \item \textbf{Low costs}.
        \item \textbf{Sensing and actuating}.
    \end{itemize}

    \begin{flushleft}
        \textcolor{Red2}{\faIcon{thumbs-down} \textbf{Internet-of-Things Disadvantages}}
    \end{flushleft}
    \begin{itemize}
        \item \textbf{Low computing ability}.
        \item \textbf{Constraints on energy}.
        \item \textbf{Constraints on memory (RAM/FLASH)}.
        \item \textbf{Difficulties in programming}.
    \end{itemize}

    \longline

    \begin{center}
        \large
        \textcolor{Red3}{\textbf{Embedded (System) PCs}}
    \end{center}

    \noindent
    An \definition{Embedded System} is a computer system, a combination of a computer processor, computer memory, and input/output peripheral devices, that has a dedicated function within a larger mechanical or electronic system.

    \highspace
    A few \example{examples}: \href{https://www.hardkernel.com/}{Odroid}, \href{https://www.raspberrypi.com/}{Raspberry}, \href{https://developer.nvidia.com/embedded/jetson-nano}{jetson nano}, \href{https://www.coral.ai/}{Google Coral}.

    \begin{flushleft}
        \textcolor{Green3}{\faIcon{check} \textbf{Embedded System Advantages}}
    \end{flushleft}
    \begin{itemize}
        \item \textbf{Persuasive computing}.
        \item \textbf{High performance unit}.
        \item \textbf{Availability of development boards}.
        \item \textbf{Programmed as PC}.
        \item \textbf{Large community}.
    \end{itemize}

    \begin{flushleft}
        \textcolor{Red2}{\faIcon{thumbs-down} \textbf{Embedded System Disadvantages}}
    \end{flushleft}
    \begin{itemize}
        \item \textbf{Pretty high power consumption}.
        \item \textbf{(Some) Hardware design has to be done}.
    \end{itemize}

    \newpage

    \begin{center}
        \large
        \textcolor{Red3}{\textbf{Edge/Fog Computing Systems}}
    \end{center}

    \noindent
    The key \textbf{difference} between \definition{Fog Computing} and \definition{Edge Computing} is associated with the location \textbf{where the data is processed}:
    \begin{itemize}
        \item In \textbf{edge computing}, the data is processed closest to the sensors.

        \item In \textbf{fog computing}, the computing is moved to processors linked to a local area network (IoT gateway).
    \end{itemize}
    Edge computing places the intelligence in the connected devices themselves, whereas, fog computing puts in the local area network.

    \begin{figure}[!htp]
        \centering
        \includegraphics[width=.8\textwidth]{img/edge-fog-computing-systems-1.pdf}
    \end{figure}

    \begin{flushleft}
        \textcolor{Green3}{\faIcon{check} \textbf{Fog/Edge Advantages}}
    \end{flushleft}
    \begin{itemize}
        \item \textbf{High computational capacity}.
        \item \textbf{Distributed computing}.
        \item \textbf{Privacy and security}.
        \item \textbf{Reduced Latency in making a decision}.
    \end{itemize}

    \begin{flushleft}
        \textcolor{Red2}{\faIcon{thumbs-down} \textbf{Fog/Edge Disadvantages}}
    \end{flushleft}
    \begin{itemize}
        \item \textbf{Require a power connection}.
        \item \textbf{Require connection with the Cloud}.
    \end{itemize}

    \newpage

    \begin{table}[!htp]
        \centering
        \begin{tabular}{@{} l p{11.5em} p{11.5em} @{}}
            \toprule
            \textbf{Feature} & \textbf{Edge Computing} & \textbf{Fog Computing} \\
            \midrule
            \textbf{Location} & Directly on device or nearby device. & Intermediary devices between edge and cloud. \\
            \\
            \textbf{Processing Power} & Limited due to device constraints, sending data to central server for analysis. & More powerful than edge devices. However, sending data to a central server for analysis. \\
            \\
            \textbf{Primary Function} & Real-time decision-making, low latency. However, central server analyzing combined data and sending only relevant information further. & Pre-process and aggregate data, reduce bandwidth usage. However, central server analyzing combined data and sending only relevant information further. \\
            \\
            \textbf{Advantages} & Low latency, reduced reliance on cloud, security for sensitive data. & Bandwidth efficiency, lower cloud costs, complex analysis capabilities. \\
            \\
            \textbf{Disadvantages} & Limited processing power, single device focus. & Increased complexity, additional infrastructure cost. \\
            \bottomrule
        \end{tabular}
        \caption{Differences between Edge and Fog Computing Systems.}
    \end{table}

    \newpage

    \begin{center}
        \textcolor{Red2}{\textbf{The Datacenter as a Computer}}
    \end{center}

    \noindent
    In the last few decades, computing and storage have moved from PC-like clients to smaller, often mobile, devices combined with extensive internet services. Furthermore, traditional enterprises are also shifting to Cloud computing.

    \highspace
    The \example{advantages} of this migration are:
    \begin{itemize}
        \item \textbf{User-side}:
        \begin{itemize}
            \item \textbf{Ease of management} (no configuration or backups needed);
            
            \item The \textbf{availability of the service is everywhere}, but we need connectivity.
        \end{itemize}
        
        \item \textbf{Vendors-side}:
        \begin{itemize}
            \item \href{https://en.wikipedia.org/wiki/Software_as_a_service}{SaaS (Software-as-a-Service)} allows \textbf{faster application development} (more accessible to make changes and improvements);

            \item \textbf{Improvements and fixes} in the software are \textbf{more straightforward inside their data centers} (instead of updating many millions of clients with peculiar hardware and software configurations);
            
            \item The \textbf{hardware deployment} is restricted to a \textbf{few well-tested configurations}.
        \end{itemize} 

        \item \textbf{Server-side}:
        \begin{itemize}
            \item \textbf{Faster introduction of new hardware devices} (e.g., HW accelerators or new hardware platforms);

            \item Many application \textbf{services can run at a low cost per user}.
        \end{itemize}
    \end{itemize}
    Finally, another advantage is that \textbf{some workloads require so much computing capability that they are a more natural fit in the datacenter} (and not in client-side computing). For example, the search services (web, images, and so on) or the Machine and Deep Learning.

    \newpage

    \begin{center}
        \textcolor{Red2}{\textbf{Warehouse-Scale Computers}}
    \end{center}

    \noindent
    The trends toward server-side computing and widespread internet services created a new class of computing systems: \textbf{Warehouse-Scale Computers}.

    \begin{definitionbox}
        \definition{Warehouse-Scale Computers (WSCs)} is intended to draw attention to the most distinctive feature of these machines: \textbf{the massive scale of their software infrastructure, data repositories and hardware platform}.
    \end{definitionbox}

    \begin{flushleft}
        \textcolor{Green3}{\faIcon{question-circle} \textbf{What is a \emph{program} at a WSC?}}
    \end{flushleft}

    \noindent
    In Warehouse-Scale Computing \textbf{the \underline{program} is an internet service}, which may \textbf{consist of tens or more individual programs that interact to implement complex end-user services} such as \emph{email}, \emph{search}, or \emph{maps}. These programs might be implemented and maintained by different teams of engineers, perhaps even across organizational, geographic, and company boundaries.

    \begin{flushleft}
        \textcolor{Green3}{\faIcon{balance-scale} \textbf{Difference between WSCs and Data Centers}}
    \end{flushleft}

    \noindent
    WSCs currently power the services offered by companies such as Google, Amazon, Microsoft, and others. The main difference from traditional data centers (see more on page \pageref{Data Center definition}) is that \textbf{WSCs belong to a single organization, use a relatively homogeneous hardware and system software platform, and share a common systems management layer}. In contrast with the typical data center that belongs to multiple organizational units or even different companies, use dedicated HW infrastructure in order to run a large number of applications (more details on page \pageref{Data Center definition}).
    
    \begin{flushleft}
        \textcolor{Green3}{\faIcon{question-circle} \textbf{How is the WSC organized?}}
    \end{flushleft}

    \noindent
    The \textbf{software on WSCs}, such as Gmail, runs on a scale far beyond a single machine or rack: it \textbf{runs on clusters of hundreds to thousands of individual servers}. Therefore, the machine, the computer, is itself this \textbf{large cluster or aggregation of servers} and must be \textbf{considered a single computing unit}.

    \highspace
    Most importantly, WSCs run fewer vast applications (internet services). An \example{advantage} is that the \textbf{shared resource management infrastructure allows significant deployment flexibility}. Finally, the requirements of:
    \begin{itemize}
        \item \textbf{Homogeneity}
        \item \textbf{Single-Organization Control}
        \item \textbf{Cost Efficiency}
    \end{itemize}
    Motivate designers to take new approaches to constructing and operating these systems.

    \newpage

    \begin{center}
        \textcolor{Red2}{\textbf{Multiple Data Centers}}
    \end{center}

    \noindent
    Sometimes the data centers are \textbf{located far apart}. \textbf{Multiple data centers} are (often) \textbf{replicas of the same service}:
    \begin{itemize}
        \item To \emph{reduce user latency}
        \item To \emph{improve service throughput}
    \end{itemize}
    Typically, a request is fully processed within one data center.

    \highspace
    The world is divided into \definition{Geographic Areas (GAs)}. Each Area is defined by Geo-political boundaries (or country borders). Also, there are at least two computing regions in each geographical Area.

    \highspace
    The \definition{Computing Regions (CRs)} are the smallest geographic unit of the infrastructure from the customer's perspective. Multiple Data centers within the same region are not exposed to customers.
    
    However, they are defined by a latency-defined perimeter, typically less than 2ms for round-trip latency.
    Finally, they're located hundreds of miles apart, with considerations for different flood zones, etc. It is too far from synchronous replication but suitable for disaster recovery.

    \highspace
    The \definition{Availability Zones (AZs)} are finer-grain \textbf{locations within a single computing region}. They allow customers to run mission-critical applications with high availability and fault tolerance to Data Center failures. Because there are fault-isolated locations with redundant power, cooling, and networking (they are different from the concept of the Availability Set).

    \highspace
    This hierarchical structure ensures efficient data management and compliance with local data laws while optimizing network performance through strategically placing data centers.

    \begin{figure}[!htp]
        \centering
        \includegraphics[width=.8\textwidth]{img/availability-zones-region-geography.png}
        \caption{Example of \href{https://learn.microsoft.com/en-us/azure/reliability/availability-zones-overview?tabs=azure-cli}{Azure Availability Zones}.}
    \end{figure}

    \newpage

    \begin{center}
        \textcolor{Red2}{\textbf{Warehouse-Scale Computing / Data Centers Availability}}
    \end{center}

    The services provided through WSCs (or DCs) \textbf{must guarantee high availability}, typically aiming for at least 99.99\% uptime (e.g. one hour of downtime per year).

    \highspace
    Some \example{examples}:
    \begin{itemize}
        \item 99,90\% on single instance VMs with premium storage for a more accessible lift and shift;
        
        \item 99,95\% VM uptime SLA for Availability Sets (AS) to protect for failures within a data center;

        \item 99,99\% VM uptime SLA through Availability Zones.
    \end{itemize}
    Such fault-free operation is more accessible when an extensive collection of hardware and system software is involved.
    
    \highspace
    \textbf{WSC workloads must be designed to gracefully tolerate large numbers of component faults with little or no impact on service level performance and availability!}

    \hfill

    \longline

    \hfill

    \begin{center}
        \textcolor{Red2}{\textbf{Architectural overview of Warehouse-Scale Computing}}
    \end{center}

    \begin{figure}[!htp]
        \centering
        \includegraphics[width=\textwidth]{img/WSC-architecture-1.pdf}
        \caption{Architectural overview of Warehouse-Scale Computing.}
    \end{figure}

    \newpage
    
    \begin{itemize}
        \item \textbf{Server} (section~\ref{subsubsection: Server (computation, HW accelerators)}, page~\pageref{subsubsection: Server (computation, HW accelerators)}). Servers are the \textbf{leading processing equipment}: different types according to CPUs, RAM, local storage, accelerators, and form factor. The servers are \textbf{hosted on individual shelves} and are the \textbf{basic building blocks of Data Centers and Warehouse-Scale Computers}. They are interconnected by hierarchies of networks and supported by the shared power and cooling infrastructure.

        \item \textbf{Storage} (section~\ref{subsubsection: Storage (type, technology)}, page~\pageref{subsubsection: Storage (type, technology)}). Disks, flash SSDs, and Tapes are the \textbf{building blocks} of today's \textbf{WSC storage systems}. These devices are \textbf{connected to the Data Center network and managed by sophisticated distributed systems}.
        
        Some \example{examples}:
        \begin{itemize}
            \item Direct Attached Storage (DAS)

            \item Network Attached Storage (NAS)

            \item Storage Area Networks (SAN)

            \item RAID controllers
        \end{itemize}

        \item \textbf{Networking} (section~\ref{subsubsection: Networking (architecture and technology)}, page~\pageref{subsubsection: Networking (architecture and technology)}). The \textbf{Data Center Network (DCN) enables efficient data transfer and interaction between various components}. The data processing ecosystem within the DCs needs to reach the DC services from outside.
        Communication equipment includes switches, Routers, cables, DNS or DHCP servers, Load balancers, Firewalls, etc.

        \item \textbf{Building and Infrastructure}. WSC has other essential components related to power delivery, cooling, and building infrastructure that must be considered. Some interesting numbers:
        \begin{itemize}
            \item Data Centers with up to 110 football-pitch size.
            
            \item 2-100s MW power consumption (100k houses), and the largest in the world is 650 MW.
        \end{itemize}
    \end{itemize}

    \newpage

    \subsection{Node-level}

    \subsubsection{Server (computation, HW accelerators)}\label{subsubsection: Server (computation, HW accelerators)}

    \definition{Server}s are like ordinary PCs, usually more powerful, but with a \textbf{form factor that allows them to fit into the shelves} (such as rack, blade enclosure format, or tower; the differences are explained later). They are usually built in a tray or blade enclosure format, \textbf{housing the motherboard}, the \textbf{chipset}, and \textbf{additional plug-in components}.

    \highspace
    The \textbf{motherboard} acts as the central hub, \textbf{connecting all the crucial components of the server and enabling them to communicate and work together}.
    
    It provides sockets and plug-in slots to install CPUs, memory modules (DIMMs), local storage (such as Flash SSDs or HDDs), and network interface cards (NICs) to satisfy the range of resource requirements.

    \highspace
    The \textbf{chipsets} and \textbf{additional components} are grouped in the following way:
    \begin{itemize}
        \item Number and type of CPUs:
        \begin{itemize}
            \item From 1 to 8 CPU socket.
            \item Intel Xeon Family, AMD EPYC, etc.
        \end{itemize}

        \item Available RAM:
        \begin{itemize}
            \item From 2 to 192 DIMM Slots.
        \end{itemize}

        \item Locally attached disks:
        \begin{itemize}
            \item From 1 to 24 Drive Bays. 
            \item HDD or SSD.
            \item SAS (higher performance but more expensive) or SATA (for entry-level servers).
        \end{itemize}

        \item Other special purpose devices:
        \begin{itemize}
            \item From 1 to 20 GPUs per node, or TPUs.
            \item NVIDIA Pascal, Volta, etc.
        \end{itemize}
        
        \item Form factor:
        \begin{itemize}
            \item From 1 unit to 10 units.
            \item Tower.
        \end{itemize}
    \end{itemize}

    \newpage

    \begin{center}
        \textcolor{Red2}{\textbf{Differences between Rack, Blade and Tower}}
    \end{center}

    \subsubsection*{Tower Server}

    A \definition{Tower Server} looks and feels much like a \textbf{traditional} tower \textbf{PC}.

    \begin{flushleft}
        \textcolor{Green3}{\faIcon{check} \textbf{Advantages}}
    \end{flushleft}
    \begin{itemize}[label=\ding{51}]
        \item \textbf{Scalability} and \textbf{ease of upgrade}. Customized and upgraded based on necessity.

        \item \textbf{Cost-effective}. Tower servers are probably the \emph{cheapest of all kinds of servers}.

        \item \textbf{Cools easily}. Since a tower server has a low overall component density, it cools down easily.
    \end{itemize}

    \begin{flushleft}
        \textcolor{Red2}{\faIcon{thumbs-down} \textbf{Disadvantages}}
    \end{flushleft}
    \begin{itemize}[label=\ding{55}]
        \item \textbf{Consumes a lot of space}. These servers are difficult to manage physically.

        \item \textbf{Provides a basic level of performance}. A tower server is \emph{ideal for small business that have a limited number of clients}.

        \item \textbf{Complicated cable management}. Devices aren't easily routed together.
    \end{itemize}

    \longline

    \subsubsection*{Rack Servers}

    \definition{Rack Servers} are unique \textbf{shelves that accommodate all the IT equipment} and allow their interconnection. The racks are used to store these rack servers.

    \highspace
    Server racks are measured in \definition{Rack Units}, or "U". One U is approximately 44.45 millimeters. The \underline{main advantage} of these racks is that they \textbf{allow designers to stack up other electronic devices and servers}.

    \highspace
    A rack server is designed to be positioned in a bay by vertically stacking servers one over the other along with other devices (storage units, cooling systems, network peripherals, and batteries).

    \begin{flushleft}
        \textcolor{Green3}{\faIcon{check} \textbf{Advantages}}
    \end{flushleft}
    \begin{itemize}[label=\ding{51}]
        \item \textbf{Failure containment}. Very little effort to identify, remove, and replace a malfunctioning server with another.

        \item \textbf{Simplified cable management}. Easy and efficient to organize cables.

        \item \textbf{Cost-effective}. Computing power and efficiency at relatively lower costs.
    \end{itemize}

    \newpage

    \begin{flushleft}
        \textcolor{Red2}{\faIcon{thumbs-down} \textbf{Disadvantages}}
    \end{flushleft}
    \begin{itemize}[label=\ding{55}]
        \item \textbf{Power usage}. Needs of additional cooling systems due to their high overall component density, thus consuming more power.

        \item \textbf{Maintenance}. Since multiple devices are placed in racks together, maintaining them gets considerably though with the increasing number of racks.
    \end{itemize}

    \newpage

    \subsubsection*{Blade Servers}

    \definition{Blade Servers} are the latest and the most advanced type of servers in the market. They can be termed hybrid rack servers, where servers are placed inside blade enclosures, forming a blade system.

    \highspace
    The \textbf{most significant advantage} of blade servers is that these servers are the \textbf{most minor types of servers available now} and are \textbf{great for conserving space}.

    \highspace
    Finally, a blade system also meets the IEEE standard for rack units, and each rack is measured in the units of "U".

    \begin{flushleft}
        \textcolor{Green3}{\faIcon{check} \textbf{Advantages}}
    \end{flushleft}
    \begin{itemize}[label=\ding{51}]
        \item \textbf{Size and form factor}. They are smallest and the most compact servers, requiring minimal physical space. Blade servers offer \emph{higher space efficiency} compared to traditional rack-mounted servers.

        \item \textbf{Cabling}. Blade server don't involve the cumbersome tasks of setting up cabling. Although you still might have to deal with the cabling, it is near to negligible when compared to tower and rack servers.

        \item \textbf{Centralized management}. Blade enclosures typically come with centralized management tools that allow administrators to easily monitor, configure and update all blades from a single interface.

        \item \textbf{Load balancing, failover, scalability}. Uniform system, shared components (including network), simple addition/removal of servers.
    \end{itemize}

    \begin{flushleft}
        \textcolor{Red2}{\faIcon{thumbs-down} \textbf{Disadvantages}}
    \end{flushleft}
    \begin{itemize}[label=\ding{55}]
        \item \textbf{Expensive configuration and Higher initial cost}. Although upgrading the blade server is easy to handle and manage, the initial configuration or the setup requires more effort and higher initial investment.

        \item \textbf{Vendor Lock-In}. Blade servers typically require the use of the manufacturer's specific blades and enclosures, leading to vendor lock-in. This can limit flexibility and potentially increase costs in the long run.

        \item \textbf{Cooling}. Blade servers come with high component density. Therefore, special accommodations have to be arranged for these servers to ensure they don't get overheated. Heating, ventilation, and air conditioning systems (HVAC) must be carefully managed and designed.
    \end{itemize}

    \newpage

    \begin{deepeningbox}[: Machine Learning (supervised learning)]
        \textbf{Machine learning} (ML) is a branch of artificial intelligence (AI) and computer science that focuses on the using data and algorithms to enable AI to imitate the way that humans learn, gradually improving its accuracy (\href{https://www.ibm.com/topics/machine-learning}{source}).

        \highspace
        \href{https://ischoolonline.berkeley.edu/blog/what-is-machine-learning/}{UC Berkeley} breaks out the learning system of a machine learning algorithm into three main parts:
        \begin{enumerate}
            \item A Decision Process: In general, machine learning algorithms are used to make a prediction or classification. Based on some input data, which can be labeled or unlabeled, your algorithm will produce an estimate about a pattern in the data.

            \item An Error Function: An error function evaluates the prediction of the model. If there are known examples, an error function can make a comparison to assess the accuracy of the model.

            \item A Model Optimization Process: If the model can fit better to the data points in the training set, then weights are adjusted to reduce the discrepancy between the known example and the model estimate. The algorithm will repeat this iterative \dquotes{evaluate and optimize} process, updating weights autonomously until a threshold of accuracy has been met.
        \end{enumerate}

        \highspace
        The main goal is to learn a target function that can be used for prediction. Given a training set of labeled examples $\left\{\left(x_{1}, y_{1}\right), \dots, \left(x_{n}, y_{n}\right)\right\}$, estimate the prediction function $f$ by minimizing the prediction error on the training set:
        \begin{equation*}
            y = f\left(x\right)
        \end{equation*}
        Where $y$ is the output, $f$ is the prediction function and the $x$ is an image feature.
    \end{deepeningbox}

    \begin{deepeningbox}[: Artificial Neural Network]
        The \textbf{Artificial Neural Network} is a computational model inspired by the human brain (perceptron). It consists of interconnected nodes (neurons) organized in layers to process and analyze data and used to learn data representation from data (learn features and the classifier/regressor).

        \highspace
        The learning process of a Neural Network is as follows: Neurons make decisions (activation functions). There are wights, so the connections between neurons are strengthened or weakened through training- randomly initialized.

        The (training data) Neural Networks learn from historical data and examples. Then, labeled data are provided.
    \end{deepeningbox}

    \begin{deepeningbox}[: effects of ML and ANN]
        Deep learning models began to appear and be widely adopted, enabling specialized hardware to power a broad spectrum of machine learning solutions.
        
        Since 2013, AI learning compute requirements have doubled every 3.5 months (vs. 18-24 months expected from \href{https://en.wikipedia.org/wiki/Moore's_law}{Moore's Law}).

        To satisfy the growing compute needs for deep learning, \textbf{WSCs deploy specialized accelerator hardware}:
        \begin{itemize}
            \item GPUs
            \item TPU (or ad hoc accelerators)
            \item FPGAs
        \end{itemize}
    \end{deepeningbox}

    \newpage

    \subsubsection{Storage (type, technology)}\label{subsubsection: Storage (type, technology)}

    \subsubsection{Networking (architecture and technology)}\label{subsubsection: Networking (architecture and technology)}


    %%%%%%%%

    \newpage

    \pagestyle{fancy}
    \fancyhead{} % clear all header fields
    \fancyhead[R]{\nouppercase{\leftmark}}

    \bibliography{bibtex}{}
    \bibliographystyle{plain}

    \newpage

    \printindex
\end{document}