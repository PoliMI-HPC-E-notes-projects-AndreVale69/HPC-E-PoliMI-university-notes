\subsubsection{February 7}

\subsubsection*{True (\trueIcon) false (\falseIcon) questions}

\begin{enumerate}
    \item \trueIcon \: RAID 5 can handle sequential reads and writes better than RAID 10 with the same amount of disks.
    \item \trueIcon \: Monolithic hypervisors are better suited for high-performance computing than microkernel hypervisors.
    \begin{deepeningbox}[: \emph{Why is this True?}]
        \begin{itemize}
            \item In the \textbf{monolithic hypervisor} (like VMware ESXi, Xen in monolithic mode, KVM's kernel module), the device drivers run within the hypervisor.
            \item In the \textbf{microkernel hypervisor} (like early Xen's microkernel style or L4-based designs), the device drivers run within a service virtual machine.
        \end{itemize}
        Monolithic hypervisors deliver \textbf{lower overhead and faster context switching} because they handle virtual machine (VM) operations in the same kernel address space as device drivers and the scheduler.
        
        Microkernel hypervisors \textbf{add extra isolation} (good for security or fault tolerance) but this comes with \textbf{higher overhead}, so we lose raw performance.
    \end{deepeningbox}
    \item \trueIcon \: Virtualization can help reduce the need for physical space and infrastructure by consolidating multiple servers onto a single physical machine.
    \item \trueIcon \: A PUE of 1.0 means that all of the power consumed by a data center is being used by the IT equipment.
    \item \trueIcon \: GPUs require specialized software and programming frameworks to fully leverage their parallel processing capabilities.
    \item \trueIcon \: Warehouse-scale computers require specialized cooling and power management systems.
    \item \falseIcon \: Hot/cold aisle containment reduces the efficiency of the cooling method for data- centers.
    \item \falseIcon \: SSDs use spinning disks to store data.
    \item \trueIcon \: In a leaf-spine topology, adding or removing a leaf switch does not impact the connectivity of other switches in the network.
    \item \falseIcon \: POD, Virtual-Chassis, DCell and BCube models are all evolutions of the leaf-spine network architectures.
\end{enumerate}

\newpage

\subsubsection*{Exercises}

\begin{enumerate}
    \setcounter{enumi}{10}

    \item Suppose we have a computer system composed of 6 different components, and designed to have an RBD as shown in the image below. The four types of components (A, B, C, and D) have different reliability characteristics. We know that after 2 years the reliability of components B, C, and D are respectively $R_{B}(2y) = 0.8$, $R_{C}(2y) = 0.75$, and $R_{D}(2y) = 0.9$. What should be the MTTF for component A, if we want to have a Reliability of the whole system after 2 years equal to $R_{sys}(2y) = 0.85$?
    
    \begin{center}
        \begin{tikzpicture}[node distance=1cm]
            \node[
                draw,
                minimum width=1cm,
                minimum height=0.5cm,
                name=A
            ] at (1.5,0) {A};
            \node[
                draw,
                minimum width=1cm,
                minimum height=0.5cm,
                name=B1
            ] at (4.5,1) {B};
            \node[
                draw,
                minimum width=1cm,
                minimum height=0.5cm,
                name=B2
            ] at (4.5,-1) {B};
            \node[
                draw,
                minimum width=1cm,
                minimum height=0.5cm,
                name=C1
            ] at (7.5,1.5) {C};
            \node[
                draw,
                minimum width=1cm,
                minimum height=0.5cm,
                name=C2
            ] at (7.5,0.5) {C};
            \node[
                draw,
                minimum width=1cm,
                minimum height=0.5cm,
                name=D
            ] at (7.5,-1) {D};

            \draw (0,0) -- (A);
            \draw (A) -- (3,0);

            \draw (3,0) -- (3,1);
            \draw (3,0) -- (3,-1);

            \draw (3,1) -- (B1);
            \draw (3,-1) -- (B2);
            \draw (B1) -- (6,1);
            \draw (6,1) -- (6,1.5);
            \draw (6,1) -- (6,0.5);
            \draw (6,1.5) -- (C1);
            \draw (6,0.5) -- (C2);
            \draw (B2) -- (D);
            \draw (C1) -- (9,1.5);
            \draw (C2) -- (9,0.5);
            \draw (9,1.5) -- (9,0);
            \draw (D) -- (9,-1);
            \draw (9,-1) -- (9,0);
            \draw (9,0) -- (10,0);
        \end{tikzpicture}
    \end{center}

    \textcolor{Green3}{\textbf{\emph{Answer:}}} We have 5 groups here:
    \begin{enumerate}[label=Group \arabic*., labelwidth=4em, labelsep=1em, leftmargin=!]
        \item The two C components are \textbf{parallel}. The reliability of C is $0.75$, so:
        \begin{equation*}
            \begin{array}{rcl}
                R_{\text{Group 1}} &=& 1 - \displaystyle\prod_{i=1}^{n=2} \left(1 - R_{C_{i}}\right) \\ [1.3em]
                &=& 1 - \left[
                    \left(1 - 0.75\right)
                    \cdot
                    \left(1 - 0.75\right)
                \right] \\ [.5em]
                &=& 1 - \left[ 0.25 \cdot 0.25 \right] \\ [.5em]
                &=& 1 - 0.0625 \\ [.5em]
                &=& 0.9375
            \end{array}
        \end{equation*}

        \item Group 1 and Component B are in \textbf{series}. Group 1 has a reliability of $0.9375$ and component B has a reliability of $0.8$.
        \begin{equation*}
            R_{\text{Group 2}} = \displaystyle\prod_{i=1}^{n=2} R_{i} = 0.9375 \cdot 0.8 = 0.75
        \end{equation*}

        \item Components B and D are in \textbf{series}. If Component B has a reliability of $0.8$ and Component D has a reliability of $0.9$, then the reliability of this group is:
        \begin{equation*}
            R_{\text{Group 3}} = \displaystyle\prod_{i=1}^{n=2} R_{i} = 0.8 \cdot 0.9 = 0.72
        \end{equation*}

        \newpage

        \item Groups 2 and 3 are calculated in \textbf{parallel}. If Group 2 has a reliability of $0.75$ and Group 3 has a reliability of $0.72$, then:
        \begin{equation*}
            \begin{array}{rcl}
                R_{\text{Group 4}} &=& 1 - \displaystyle\prod_{i=1}^{n=2} \left(1 - R_{i}\right) \\ [1.3em]
                &=& 1 - \left[
                    \left(1 - 0.75\right)
                    \cdot
                    \left(1 - 0.72\right)
                \right] \\ [.5em]
                &=& 1 - \left[ 0.25 \cdot 0.28 \right] \\ [.5em]
                &=& 1 - 0.07 \\ [.5em]
                &=& 0.93
            \end{array}
        \end{equation*}

        \item Groups 4 and A are calculated in \textbf{series}. Here, we don't have the reliability of Component A; however, we can easily find it since we have the reliability of the system:
        \begin{equation*}
            \begin{array}{rcl}
                R_{\text{Group 5}} &=& \displaystyle\prod_{i=1}^{n=2} R_{i} \\ [1.3em]
                0.85 &=& 0.93 \cdot R_{A}\left(2y\right) \\ [.5em]
                R_{A}\left(2y\right) &=& 0.85 \cdot \dfrac{1}{0.93} \\ [1.1em]
                &=& 0.913978495
            \end{array}
        \end{equation*}
    \end{enumerate}
    Now that we have the reliability value of component A, we can use the reliability formula with the Euler number exponent to find the mean time to failure (MTTF):
    \begin{equation*}
        \begin{array}{rcl}
            R_{A}\left(t\right) &=& e^{-\lambda t} \\ [1em]
            R_{A}\left(2y\right) &=& e^{-\lambda 2} \\ [1em]
            0.913978495 &=& e^{-\lambda 2} \\ [1em]
            \ln\left(0.913978495\right) &=& -\lambda 2 \\ [1em]
            -0.089948236 &=& -\lambda 2 \\ [1em]
            0.089948236 &=& \dfrac{1}{\text{MTTF}} \cdot 2 \\ [1em]
            \text{MTTF} \cdot 0.089948236 &=& \dfrac{2}{\cancel{\text{MTTF}}} \cdot \cancel{\text{MTTF}} \\ [1em]
            \dfrac{1}{\cancel{0.089948236}} \cdot \text{MTTF} \cdot \cancel{0.089948236} &=& 2 \cdot \dfrac{1}{0.089948236} \\ [1em]
            \text{MTTF} &=& \mathbf{22.23}5010812 \, \textbf{years}
        \end{array}
    \end{equation*}


    \newpage


    \item We have to design a RAID 5 storage architecture composed of an array of 7 disks. Knowing that each disk has a MTTF equal to 415 days and that we would like to have a MTTF for the storage infrastructure ($MTTF_{RAID5}$) higher than 15 years, what is the maximum MTTR that we have to consider to satisfy the requirement? Consider all the disks with the same characteristics.

    \textcolor{Green3}{\textbf{\emph{Answer:}}} We use the Mean Time To Failure formula for RAID 5 (see page \pageref{eq: RAID 5 - MTTF}):
    \begin{equation*}
        \begin{array}{rcl}
            \text{MTTF}_{\text{RAID 5}} &=& \dfrac{\text{MTTF}_{\text{disk}}^{2}}{G\left(G-1\right) \cdot \text{MTTR}} \\ [1.3em]
            \left(15 \cdot 365\right) \, \text{days} &=& \dfrac{\left(415\right)^{2}}{7\left(7-1\right) \cdot \text{MTTR}} \\ [1.3em]
            5475 \, \text{days} &=& \dfrac{172'225}{42 \cdot \text{MTTR}} \\ [1.3em]
            \text{MTTR} \cdot 5475 &=& \dfrac{172'225}{42 \cdot \cancel{\text{MTTR}}} \cdot \cancel{\text{MTTR}} \\ [1.3em]
            \dfrac{1}{\cancel{5475}} \cdot \text{MTTR} \cdot \cancel{5475} &=& \dfrac{172'225}{42} \cdot \dfrac{1}{5475} \\ [1.3em]
            \text{MTTR} &=& \dfrac{172'225}{42} \cdot \dfrac{1}{5475} \\ [1.3em]
            &=& \dfrac{172'225}{229'950} \\ [1.3em]
            &=& 0.748967167 \, \text{days} \\ [.5em]
            &\approx& \mathbf{0.75} \, \textbf{days}
        \end{array}
    \end{equation*}


    \item Consider an HDD with a data transfer rate of 10 MB/s, a rotation speed of 12000 RPM, a mean seek time of 4 ms, and a negligible overhead controller. What is the minimum locality required to achieve a mean I/O service time of 1.25 ms to transfer a sector of 4 KB?

    \textcolor{Green3}{\textbf{\emph{Answer:}}} We can use the average I/O time formula (see page \pageref{eq: Input Output average time}) to find the data locality:
    \begin{equation*}
        T_{\text{I/O Avg}} = \left(1 - DL\right)\left(T_{\text{seek}} + T_{\text{rotation}}\right) + T_{\text{transfer}} + T_{\text{controller overhead}}
    \end{equation*}
    In our case, the only well-known parts of this formula are the I/O average time and seek time. The controller overhead is negligible, so we can consider it zero. However, the rotation and transfer times are easy to obtain since we have all the necessary data!

    \newpage

    \begin{itemize}
        \item \textbf{Rotation Time}. We determine the rotation using two formulas: the full rotation delay and the total rotation average, which is what we are looking for.
        \begin{equation*}
            \begin{array}{rcl}
                R &=& \dfrac{1}{\text{RPM}} = \dfrac{1}{12'000} = 0.000083333 \cdot 60 = 0.005 \, \text{sec} \\ [1.3em]
                T_{\text{rotation}} &=& \dfrac{R}{2} = \dfrac{0.005}{2} = 0.0025 \, \text{sec} = 2.5 \, \text{ms}
            \end{array}
        \end{equation*}

        \item \textbf{Transfer Time}. The transfer time can be easily calculated by dividing the read/write speed of a sector by the data transfer rate:
        \begin{equation*}
            \begin{array}{rcl}
                T_{\text{transfer}} &=& \dfrac{\text{R/W sector}}{\text{Data Transfer Rate}} \\ [1.3em]
                &=& \dfrac{4}{\left(10 \cdot 1024\right) \, \text{KB/s}} \\ [1.3em]
                &=& \dfrac{4}{10'240} \\ [1.3em]
                &=& 0.000390625 \, \text{sec} \\ [.5em]
                &=& 0.390625 \, \text{ms}
            \end{array}
        \end{equation*}
    \end{itemize}
    We now have all the necessary ingredients to calculate the desired data locality:
    \begin{equation*}
        \begin{array}{rcl}
            T_{\text{I/O Avg}} &=& \left(1 - DL\right)\left(T_{\text{seek}} + T_{\text{rotation}}\right) + T_{\text{transfer}} + T_{\text{controller}} \\ [.5em]
            1.25 &=& \left(1-DL\right)\left(4 + 2.5\right) + 0.390625 + 0 \\ [.5em]
            1.25 &=& 4 + 2.5 - 4DL - 2.5DL + 0.390625 \\ [.5em]
            0 &=& - 6.5DL + 6.890625 - 1.25 \\ [.5em]
            6.5DL &=& 5.640625 \\ [.5em]
            DL &=& \dfrac{5.640625}{6.5} \\ [1.3em]
            &=& 0.867788462 \\ [.3em]
            &\approx& \mathbf{86.78 \, \%}
        \end{array}
    \end{equation*}

    
    \newpage


    \item A company wants to evaluate the performance of the services provided to its users. The computer system includes two servers $S_1$ and $S_2$. The system is initially considered as an open queue network model and the following measurements were obtained during 20-minute monitoring:
    \begin{itemize}
        \item Number of requests served at the system level: $C = 300$
        \item Number of requests served by $S_1$: $C_{S1} = 600$
        \item Number of requests served by $S_2$: $C_{S2} = 100$
        \item Busy time $S_1$: $B_{S1} = 350$ sec
        \item Busy time $S_2$: $B_{S2} = 200$ sec
    \end{itemize}
    What are the service demand for $S_{1}$ and $S_{2}$ ($D_{S1}$ and $D_{S2}$) and their utilization ($U_{S1}$ and $U_{S2}$)?

    \textcolor{Green3}{\textbf{\emph{Answer:}}} This exercise is not complicated, but it does require knowledge of basic formulas before delving deeper.
    \begin{itemize}
        \item \textbf{Service Demand}. The formula for service demand $D$ is:
        \begin{equation*}
            D_{k} = V_{k} \cdot S_{k}
        \end{equation*}
        What is missing here are the number of visitors and the mean service time.
        \begin{itemize}
            \item \textbf{Number of visitors} $V_{k}$. The number of visitors is easy to determine:
            \begin{equation*}
                \begin{array}{rcl}
                    V_{k} &=& \dfrac{C_{k}}{C} \\ [1.1em]
                    V_{S1} &=& \dfrac{600}{300} = 2 \\ [1.1em]
                    V_{S2} &=& \dfrac{100}{300} = 0.333333333
                \end{array}
            \end{equation*}
            \item \textbf{Mean service time} $S_{k}$. The mean service time is calculated using the busy time for each component and the number of requests:
            \begin{equation*}
                \begin{array}{rcl}
                    S_{k} &=& \dfrac{B_{k}}{C_{k}} \\ [1.1em]
                    S_{S1} &=& \dfrac{350}{600} = 0.583333333 \\ [1.1em]
                    S_{S2} &=& \dfrac{200}{100} = 2
                \end{array}
            \end{equation*}
        \end{itemize}
        We are now ready to calculate the service demand for each component:
        \begin{equation*}
            \begin{array}{rcl}
                D_{k} &=& V_{k} \cdot S_{k} \\ [.5em]
                D_{S1} &=& 2 \cdot 0.583333333 = \mathbf{1.16}6666666 \, \textbf{sec} \\ [.5em]
                D_{S2} &=& 0.333333333 \cdot 2 = \mathbf{0.66}6666666 \, \textbf{sec}
            \end{array}
        \end{equation*}

        \newpage

        \item \textbf{Utilization}. The formula for utilization $U$ is:
        \begin{equation*}
            U_{k} = X_{k} \cdot S_{k}
        \end{equation*}
        The only thing missing here is the throughput of each component. We could use the basic formula with the number of requests and the observation time:
        \begin{equation*}
            \begin{array}{rcl}
                X_{k} &=& \dfrac{C_{k}}{T} \\ [1.1em]
                X_{S1} &=& \dfrac{600}{\left(20 \cdot 60\right) \, \text{sec}} = 0.5 \\ [1.1em]
                X_{S2} &=& \dfrac{100}{\left(20 \cdot 60\right) \, \text{sec}} = 0.083333333
            \end{array}
        \end{equation*}
        So, the utilization is as follows:
        \begin{equation*}
            \begin{array}{rcl}
                U_{k} &=& X_{k} \cdot S_{k} \\ [.5em]
                U_{S1} &=& 0.5 \cdot 0.583333333 = 0.291666667 \approx \mathbf{29.17 \%} \\ [.5em]
                U_{S2} &=& 0.083333333 \cdot 2 = 0.166666666 \approx \mathbf{16.67 \%}
            \end{array}
        \end{equation*}
    \end{itemize}


    \item Consider now the same system presented in \emph{Questions 14} as a closed model with a think time $Z = 18$ sec, and the same values for the demand $D_{S1}$ and $D_{S2}$ calculated in the previous question (open model version). In the context of the asymptotic bounds, what is the system throughput upper bound for $N = 30$ users?
    
    \textcolor{Green3}{\textbf{\emph{Answer:}}} We analyze the situation (heavy or light) before calculating the throughput:
    \begin{equation*}
        N^{*} = \dfrac{D + Z}{D_{\max}} = \dfrac{\left(1.16 + 0.66\right) + 18}{1.16} = 17.086206897
    \end{equation*}
    Since the number of users exceeds $N^{*}$ ($N > N^{*}$), the situation is \textbf{heavy}. Therefore, the throughput bounds are:
    \begin{equation*}
        \begin{array}{rcl}
            \dfrac{N}{N \cdot D + Z} \le & X & \le \min\left(
                \dfrac{1}{D_{\max}},\, \dfrac{N}{D + Z}
            \right) \\ [1.3em]
            \max\left(D,\, N \cdot D - Z\right) \le & R & \le N \cdot D
        \end{array}
    \end{equation*}
    Therefore, the maximum throughput is:
    \begin{equation*}
        \begin{array}{rcl}
            X_{\max} &=& \min\left(\dfrac{1}{D_{\max}},\, \dfrac{N}{D+Z}\right) \\ [1.3em]
            &=& \min\left(\dfrac{1}{1.16},\, \dfrac{30}{1.82 + 18}\right) \\ [1.3em]
            &=& \min\left(0.862068966,\, 1.513622603\right) \\ [.5em]
            &=& 0.862068966 \, \text{jobs/sec}
        \end{array}
    \end{equation*}
    It may not be the same as the professor's solution, but it depends on how many values we take as precise.

    
    \newpage


    \item Considering the same system as in \emph{Questions 14 and 15}, if you consider to add another instance of $S_{1}$ in the system that equally splits the number of visits with the other $S_{1}$ instance, how do the bounds change?

    \textcolor{Green3}{\textbf{\emph{Answer:}}} Adding another instance for $S_1$ changes the system's service demand:
    \begin{equation*}
        D = \dfrac{1.16}{2} + 0.66 = 0.58 + 0.66 = 1.24
    \end{equation*}
    However, the situation remains \textbf{heavy} even if we add another instance. Clearly, the bottleneck is now $S_2$.
    \begin{equation*}
        N^{*} = \dfrac{D + Z}{D_{\max}} = \dfrac{1.24 + 18}{0.66} = 29.15151515
    \end{equation*}
    The maximum throughput is:
    \begin{equation*}
        \begin{array}{rcl}
            X_{\max} &=& \min\left(\dfrac{1}{D_{\max}},\, \dfrac{N}{D+Z}\right) \\ [1.3em]
            &=& \min\left(\dfrac{1}{0.66},\, \dfrac{30}{1.24 + 18}\right) \\ [1.3em]
            &=& \min\left(1.515151515,\, 1.559251559\right) \\ [.5em]
            &=& \mathbf{1.51}5151515
        \end{array}
    \end{equation*}
\end{enumerate}

%\newpage

\subsubsection*{Open Questions}

\begin{enumerate}
    \setcounter{enumi}{16}

    \item In the context of Virtualization, describe a Type 2 hypervisor providing also advantages and drawbacks?

    \textcolor{Green3}{\textbf{\emph{Answer:}}} Hosted Hypervisors run within the operating system of the host machine. Although hosted hypervisors run within the OS, additional operating systems can be installed on top of it. Hosted hypervisors have higher latency than bare metal hypervisors (type 1) because requests between the hardware and the hypervisor must pass through the extra layer of the OS.

    \highspace
    Its advantages are its simplicity of use and flexibility in terms of underlying hardware. However, its main disadvantage is its high resource consumption.


    \item Provide the definition of \emph{Geographic Areas}, \emph{Compute Regions}, and \emph{Availability Zones} in the context of data centers. What are the advantages and drawbacks of placing all compute instances for my service within a single availability zone?

    \textcolor{Green3}{\textbf{\emph{Answer:}}}
    \begin{itemize}
        \item Geographic Areas are the highest level of the organization, defined by geopolitical boundaries or country-specific regulations. For example North America, Europe, Asia.
        
        \item Compute Regions are a finer granularity within geographic areas, representing large-scale infrastructure within a geographic region. Each Geographic Area contains at least two Computing Regions for redundancy. For example, AWS has US-East-1 and US-West-1, which refer to different regions.
        
        \item Availability Zones are within a Compute Region, and physically separate datacenters within a region, each with redundant power, cooling and network.
    \end{itemize}
    The advantages of placing all compute instances for my service within a single availability zone are:
    \begin{itemize}
        \item \textbf{Low Latency, High Bandwidth}. All instances are physically close, so we get minimal network latency and maximum intra-AZ bandwidth.
        \item \textbf{Simpler Design \& Management}.
        \item \textbf{Cost Efficiency}. By staying within one AZ, we save on inter-AZ data transfer costs (eventually fees to pay).
    \end{itemize}
    However, the drawbacks are very risky:
    \begin{itemize}
        \item \textbf{Single Point of Failure}. If the AZ has an outage (power, network, disaster), our entire service is down.
        \item \textbf{Limited Fault Tolerance}. We're exposed to hardware, network, or maintenance failures that can't be mitigated by simply moving load elsewhere.
        \item \textbf{Scalability Limits}. One AZ has finite capacity. Spreading across AZs allows us to tap larger total capacity.
        \item \textbf{No Geo-Redundancy}. In public cloud, AZs are designed to be isolated failure domains, not using them means we're ignoring built-in resilience.
    \end{itemize}
\end{enumerate}