\subsection{2024}

\subsubsection{September 5}

\subsubsection*{True (\trueIcon) false (\falseIcon) questions}

\begin{enumerate}
    \item \falseIcon \: Warehouse-scale computers are primarily used in small businesses with limited computing needs.
    \item \falseIcon \: Given their efficiency, accelerators like GPUs and TPUs have no impact on the cooling requirements of datacenters.
    \item \falseIcon \: Hypervisors can only be used to run virtual machines with the same amount of memory as the physical machine on which they are hosted.
    \item \falseIcon \: Leaf-spine topologies are not suitable for highly virtualized environments with a large number of virtual machines.
    \item \falseIcon \: GPUs in datacenters are primarily used for accelerating graphics rendering and video processing tasks.
    \item \trueIcon \: RAID 10 provides better random write performance than RAID 6.
    \item \trueIcon \: Virtualization can reduce costs by enabling the sharing of hardware resources between virtual machines.
    \item \falseIcon \: RAID architectures are used to substitute regular backups.
    \item \trueIcon \: A UPS provides backup power to equipment during a power outage.
    \item \falseIcon \: Cooling towers are not suitable for high-density datacenters.
\end{enumerate}

\newpage

\subsubsection*{Exercises}

\begin{enumerate}
    \setcounter{enumi}{10}

    \item A monitoring system for a rack inside a data center is designed to be composed of 2 hardware sensors ($M_{\text{HW}}$), all necessary to extract information on the use of each blade. Each of the sensors is characterized by a MTTF of 27 days and a MTTR of 1 day. If the monitoring system were to be updated to include a purely software system ($M_{\text{SW}}$), from which it would be possible to extract the same usage information obtained with the HW sensors, what would be the availability of the entire blade-usage monitoring system ($M_{\text{HW}} + M_{\text{SW}}$) if the availability of the software system (M-SW) was 96\%? Use always at least 5 decimal digits for each calculation.

    \textcolor{Green3}{\textbf{\emph{Answer:}}} We can draw the RBD where the hardware sensors are sequential and the software is inserted in parallel with the hardware components:
    \begin{center}
        \begin{tikzpicture}[node distance=2cm]
            \draw (0,0) -- (1,0);

            \draw (1,0) -- (1,1);
            \draw (1,0) -- (1,-1);

            \node[
                draw,
                minimum width=2cm,
                minimum height=1cm,
                name=H1
            ] at (3,1) {$\text{HW}_{1}$};

            \node[
                draw,
                minimum width=2cm,
                minimum height=1cm,
                name=H2
            ] at (6,1) {$\text{HW}_{2}$};

            \draw (1,1) -- (H1);
            \draw (H1) -- (H2);
            
            \node[
                draw,
                minimum width=2cm,
                minimum height=1cm,
                name=SW
            ] at (4.5,-1) {SW};

            \draw (1,-1) -- (SW);
            \draw (H2) -- (8,1);
            \draw (SW) -- (8,-1);

            \draw (8,1) -- (8,0);
            \draw (8,-1) -- (8,0);

            \draw (8,0) -- (9,0);
        \end{tikzpicture}
    \end{center}
    \begin{itemize}
        \item The \textbf{availability of the series group} can be calculated using the product of the availability of each component (see page \pageref{eq: Availability of Series System}):
        \begin{equation*}
            A_{\text{HW}} = \displaystyle\prod_{i=1}^{n=2} A_{i}
        \end{equation*}
        The availability of each component is as follows:
        \begin{equation*}
            A_{i} = \dfrac{\text{MTTF}_{i}}{\text{MTTF}_{i} + \text{MTTR}_{i}} = \dfrac{27}{27 + 1} = \dfrac{27}{28} = 0.964285714
        \end{equation*}
        The series group is available as follows:
        \begin{equation*}
            A_{\text{HW}} = 0.964285714 \cdot 0.964285714 = 0.929846938
        \end{equation*}

        \item We calculate the \textbf{system's overall availability} with the parallel formula (see page \pageref{eq: Availability of Parallel System - Simplified}) since the hardware and parallel groups are in parallel. We have everything necessary to calculate it:
        \begin{equation*}
            \begin{array}{rcl}
                A_{\text{system}} &=& 1 - \displaystyle\prod_{i=1}^{n=2} \left(1-A_{i}\right) \\ [1.5em]
                &=& 1 - \left[
                    \left(1 - A_{\text{HW}}\right)
                    \cdot
                    \left(1 - A_{\text{SW}}\right)
                \right] \\ [.5em]
                &=& 1 - \left[
                    \left(1 - 0.929846938\right)
                    \cdot
                    \left(1 - 0.96\right)
                \right] \\ [.5em]
                &=& 1 - \left[
                    0.070153062 \cdot 0.04
                \right] \\ [.5em]
                &=& 1 - 0.002806122 \\ [.5em]
                &=& \mathbf{0.99719}3878
            \end{array}
        \end{equation*}
    \end{itemize}

    \newpage

    \item A company uses a temperature sensor to monitor a critical environment within an industrial plant. The sensor has a MTTF (Mean Time To Failure) of 4000 hours. If the company decides to replace the original sensor with a redundant version consisting of two sensors like the previous one, what is the overall reliability of the monitoring system after 2000 hours? Use always at least 4 decimal digits for each calculation.

    \textcolor{Green3}{\textbf{\emph{Answer:}}} A system composed of only one sensor becomes a system composed of two sensors in parallel because they substitute for one another. The Reliability Block Diagram is:
    \begin{center}
        \begin{tikzpicture}[node distance=2cm]
            \node[
                draw,
                minimum width=2cm,
                minimum height=1cm,
                name=Sensor
            ] at (2,0) {Sensor};

            \draw (0,0) -- (Sensor);
            \draw (Sensor) -- (4,0);

            %%%%%%%%%%%%%%%%%%%%%%%%%%

            \draw (6-1,0) -- (7-1,0);

            \draw (7-1,0) -- (7-1,1);
            \draw (7-1,0) -- (7-1,-1);

            \node[
                draw,
                minimum width=2cm,
                minimum height=1cm,
                name=Sensor1
            ] at (9-1,1) {$\text{Sensor}_{1}$};
            \node[
                draw,
                minimum width=2cm,
                minimum height=1cm,
                name=Sensor2
            ] at (9-1,-1) {$\text{Sensor}_{2}$};

            \draw (7-1,1) -- (Sensor1);
            \draw (7-1,-1) -- (Sensor2);
            \draw (Sensor1) -- (11-1,1);
            \draw (Sensor2) -- (11-1,-1);
            \draw (11-1,1) -- (11-1,0);
            \draw (11-1,-1) -- (11-1,0);
            \draw (11-1,0) -- (12-1,0);
        \end{tikzpicture}
    \end{center}
    \begin{itemize}
        \item The \textbf{system's reliability} when \textbf{only one sensor is used} is as follows:
        \begin{equation*}
            \begin{array}{rcl}
                R\left(t\right) &=& e^{-\lambda t} \quad \text{where }\lambda = \dfrac{1}{\text{MTTF}} \\ [1.3em]
                R\left(2000\, \text{hours}\right) &=& e^{-\frac{1}{4000} \cdot 2000} \\ [1em]
                &=& e^{-0.5} \\ [1em]
                &=& 0.60653066
            \end{array}
        \end{equation*}

        \item The \textbf{reliability} of the system when \textbf{two parallel sensors} are added is given by the formula (see page \pageref{eq: system fails when the last component fails - parallel}):
        \begin{equation*}
            \begin{array}{rcl}
                R\left(t\right) &=& 1 - \displaystyle\prod_{i=1}^{n=2} \left(1 - R_{i}\left(t\right)\right) \\ [1.3em]
                R\left(2000 \, \text{hours}\right) &=& 1 - \displaystyle\prod_{i=1}^{n=2} \left(1 - R_{i}\left(2000\right)\right) \\ [1.3em]
                &=& 1 - \left[
                    \left(1 - 0.60653066\right)
                    \cdot
                    \left(1 - 0.60653066\right)
                \right] \\ [.5em]
                &=& 1 - \left[
                    0.39346934 \cdot 0.39346934
                \right] \\ [.5em]
                &=& 1 - 0.154818122 \\ [.5em]
                &=& \mathbf{0.8451}81878
            \end{array}
        \end{equation*}
    \end{itemize}

    \newpage

    \item Consider the following RAID $1+0$ setup composed of 6 disks, each one with an MTTF equal to 410 days and an MTTR equal to 2 days, and a single mirror case for the RAID 1 part. What is the MTTF of the storage infrastructure?

    \textcolor{Green3}{\textbf{\emph{Answer:}}} We use the Mean Time To Failure of RAID 10 (see page \pageref{eq: RAID 10 - MTTF}):
    \begin{equation*}
        \begin{array}{rcl}
            \text{MTTF}_{\text{RAID 10}} &=& \dfrac{\text{MTTF}_{\text{disk}}^{2}}{N \cdot \text{MTTR}} \\ [1.3em]
            &=& \dfrac{\left(410\right)^{2}}{6 \cdot 2} \\ [1.3em]
            &=& \dfrac{168'100}{12} \\ [1.3em]
            &=& 14008.333333333 \\ [.5em]
            &\approx& \mathbf{14008} \, \textbf{days}
        \end{array}
    \end{equation*}

    \item You are tasked with performing capacity planning for an application cluster. Your system can be modeled using a closed model. On average, there are $N = 30$ users logged into the system, with a think time of $Z = 10$ seconds. Your objective is to determine whether it is more efficient to introduce:
    \begin{itemize}
        \item 10 servers, \textbf{each} with a service demand of $D_{10} = 10$ seconds, or
        \item A single server that is 10 times faster than each server in the previous case, with a service demand of $D_{1} = 0.1$ seconds.
    \end{itemize}
    Calculate the lower bound on response time for both scenarios.

    \textcolor{Green3}{\textbf{\emph{Answer:}}}
    \begin{itemize}
        \item \textbf{Option A - 10 servers}. In order to determine which formula to apply, we must first establish whether the situation is light or heavy. To determine this, we use the $N^{*}$:
        \begin{equation*}
            N^{*} = \dfrac{D + Z}{D_{\max}} = \dfrac{D + Z}{\dfrac{D_{\text{server}}}{m}} = \dfrac{10 + 10}{\dfrac{10}{10}} = \dfrac{20}{1} = 20
        \end{equation*}
        Since the number of logged users is greater than $N^{*}$, we are in a \textbf{heavy situation}. It means that the response time is:
        \begin{equation*}
            R_{\min} = \max\left(D, \, N \cdot \dfrac{D_{\max}}{m} - Z\right) = \max\left(10, \, 30 \cdot \dfrac{10}{10} - 10\right) = 20 \, \text{s}
        \end{equation*}
        Where $m$ is the number of servers: $\dfrac{D_{\text{server}}}{m}$. The heavy-load limit is about how \emph{fast} the system can drain all users, so we divide by $m$ to get the effective bottleneck demand for the whole system.

        \newpage

        \item \textbf{Option B - 1 server}. In this case, the $N^{*}$ is:
        \begin{equation*}
            N^{*} = \dfrac{D + Z}{D_{\max}} = \dfrac{0.1 + 10}{0.1} = 101
        \end{equation*}
        Since the number of logged users is less than $N^{*}$, we are in a \textbf{light situation}. This means that the response time equals the demand service $D$: $R_{\min} = D = 0.1 \, \text{s}$.
    \end{itemize}
    For 30 users, \textbf{option B is the most efficient method}, with a response time of $0.1$ seconds.


    \item Based on the system described in question 14, calculate the upper bound on throughput for both scenarios.

    \textcolor{Green3}{\textbf{\emph{Answer:}}}
    \begin{itemize}
        \item \textbf{Option A - 10 servers}. In this context, we are in a \textbf{heavy situation}, and the throughput is as follows:
        \begin{equation*}
            \begin{array}{rcl}
                X &=& \min\left(\dfrac{1}{D_{\max}}, \, \dfrac{N}{D+Z}\right) \\ [1.5em]
                &=& \min\left(\dfrac{1}{\frac{10}{10}}, \, \dfrac{30}{10+10}\right) \\ [1.5em]
                &=& \min\left(1, \, 1.5\right) \\ [.5em]
                &=& 1 \, \text{req/sec}
            \end{array}
        \end{equation*}
        Where $D_{\max} = \dfrac{D_{\text{server}}}{m}$.

        \item \textbf{Option B - 1 server}. In this context, we are in a \textbf{light situation}, and the throughput is as follows:
        \begin{equation*}
            \begin{array}{rcl}
                X &=& \dfrac{N}{D + Z} \\ [1.5em]
                &=& \dfrac{30}{0.1 + 10} \\ [1.5em]
                &=& 2.97029703 \, \text{req/sec}
            \end{array}
        \end{equation*}
    \end{itemize}

    \item Considering the system in question 14, compute the number of users $N^{*}$ that determines whether the system should be analyzed using the light load or heavy load optimistic bounds.
    
    \textcolor{Green3}{\textbf{\emph{Answer:}}}
    \begin{itemize}
        \item \textbf{Option A - 10 servers}, where $D_{\max} = \dfrac{D_{\text{server}}}{m}$.
        \begin{equation*}
            N^{*} = \dfrac{D + Z}{D_{\max}} = \dfrac{D + Z}{\dfrac{D_{\text{server}}}{m}} = \dfrac{10 + 10}{\dfrac{10}{10}} = \dfrac{20}{1} = 20
        \end{equation*}
        \item \textbf{Option B - 1 server}.
        \begin{equation*}
            N^{*} = \dfrac{D + Z}{D_{\max}} = \dfrac{0.1 + 10}{0.1} = 101
        \end{equation*}
    \end{itemize}
\end{enumerate}

\newpage

\subsubsection*{Open Questions}

\begin{enumerate}
    \setcounter{enumi}{16}

    \item Describe characteristics, advantages, and disadvantages of Direct Attached Storage (DAS), Network Attached Storage (NAS) and Storage Area Networks (SAN).

    \textcolor{Green3}{\textbf{\emph{Answer:}}}
    \begin{itemize}
        \item \textbf{Direct Attached Storage (DAS)}. \hl{Storage directly connected to a single host} (server or PC). The main advantages are high performance for local workloads (low latency), low cost, simple to deploy, and good for dedicated single-server applications. The main disadvantages are not shareable over a network by default, is hard to scale beyond the local bus, poor resource utilization when multiple servers need the same data, and limited redundancy features compared to shared storage.
        \item \textbf{Network Attached Storage (NAS)}. A \hl{dedicated file server},\break a \hl{storage accessible over a TCP/IP network} using file-level protocols. The main advantages are centralized file sharing across multiple hosts, simple network setup, flexible because any client with network access can mount shares, and good for collaborative environments. The main disadvantages are file protocol overhead, ethernet can be a performance bottleneck for I/O-intensive applications, file-level permissions add complexity, and is not ideal for very high throughput or database block workloads.
        \item \textbf{Storage Area Network (SAN)}. A \hl{dedicated high-speed network} providing block-level \hl{storage to multiple servers}. Hosts see remote storage as local disks, the OS mounts volumes just like DAS. The main advantages are high-performance, low latency, block-level access (flexible for filesystems), highly scalable (allocates volumes dynamically), and high availability (e.g., RAID). The main disadvantages are higher cost, more complex to deploy and administer than DAS/NAS, requires specialized skills for setup and maintenance, and misconfigurations can affect multiple servers.
    \end{itemize}


    \item Describe, comment, and contextualize the four main types of Clouds.

    \textcolor{Green3}{\textbf{\emph{Answer:}}}
    \begin{itemize}
        \item \textbf{Public Cloud}. A public cloud is a \hl{shared, multi-tenant environment} operated by a third-party provider (e.g., AWS, Azure, Google Cloud). The physical infrastructure (data centers, networks, storage) is owned and managed by the provider. Users rent virtualized resources on demand via the internet.

        \highspace
        This is the most common cloud model for startups, developers, and businesses who want to avoid large upfront infrastructure investments. Public clouds provide elasticity, pay-as-you-go pricing, and massive scale.

        \highspace
        So it is quick to deploy, scalable, cost-effective, but we have less control over underlying hardware.


        \item \textbf{Private Cloud}. A private cloud is a \hl{single-tenant environment} dedicated to one organization. It can be hosted on-premises (in the company's own data center) or by a third-party provider, but the key point is that the resources are \hl{not shared} with other tenants.
        
        \highspace
        It is often chosen by large enterprises and regulated sectors, such as finance, healthcare, and the public sector, that require strong governance, data sovereignty, and full control over security policies.

        \highspace
        The main advantages are full control, customizable security, and can leverage existing investments. But the main disadvantages are higher cost, less elastic than public clouds, requires in-house expertise for management and updates.

        
        \item \textbf{Hybrid Cloud}. A hybrid cloud combines \hl{two or more clouds} (private and public) connected via standardized or proprietary technology. This allows workloads to \hl{move or burst} between environments as needed.
        
        \highspace
        Typical for enterprises that want to keep sensitive data in a private environment while using the public cloud for flexible capacity (e.g., handling seasonal peaks).

        \highspace
        The main advantages are flexibility to optimize cost, performance, and compliance; supports gradual migration to the cloud. But the disadvantages are complex integration; needs careful governance to avoid shadow IT or misconfigurations.
        
        
        \item \textbf{Community Cloud}. A community cloud is \hl{shared by several organizations} with a common concern (e.g., same compliance requirements, mission, or industry standards). The infrastructure is jointly owned or managed by the community or a third party.
        
        \highspace
        Used in sectors like government agencies, research institutions, or collaborative industries that benefit from a shared infrastructure but want more control than the public cloud.

        \highspace
        The main advantages are shared costs, tailored policies, and collaboration within a trusted group. However, the disadvantages are less common than with other models. Governance and funding models can be complex, and scalability is more limited than with large public clouds.
    \end{itemize}
\end{enumerate}