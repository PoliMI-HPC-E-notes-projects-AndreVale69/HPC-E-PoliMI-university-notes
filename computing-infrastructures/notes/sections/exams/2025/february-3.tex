\section{Exams}

\subsection{2025}

\subsubsection{February 3}

\subsubsection*{True (\trueIcon) false (\falseIcon) questions}

\begin{enumerate}
    \item \trueIcon \: Data centers requires specialized fire suppression systems.
    \item \falseIcon \: In current datacenters, east-west traffic is typically less bandwidth-\break intensive than north-south traffic.
    \item \trueIcon \: Cloud computing does not require the user to manage the underlying hardware infrastructure, instead relying on the cloud provider to manage it.
    \item \falseIcon \: RAID 6 offers better random read performance than RAID 5.
    \item \trueIcon \: Cloud architectures can help reduce the cost of IT infrastructure by enabling pay-as-you-go pricing.
    \item \trueIcon \: In-rack cooling involves placing cooling units directly in the server racks.
    \item \falseIcon \: RAID 5 has a higher storage overhead than RAID 4.
    \item \falseIcon \: Oversubscription is not used in TOR switches
    \item \falseIcon \: A three-tier network architecture is never used in leaf-spine architectures
    \item \trueIcon \: Raised floor systems in data centers help with cable management and airflow distribution.
\end{enumerate}

\subsubsection*{Exercises}

\begin{enumerate}
    \setcounter{enumi}{10}
    \item Let us consider a set of requests in the disk queue referring to the following cylinders of the disk: 65, 23, 50, 8, 39. Consider the initial position of the disk head at cylinder 52 and it is moving from inside (lower cylinder number) to outside (higher cylinder number). If no further requests arrive, write the order of the served requests (from the first to the last) if the disk head scheduling algorithm adopted is SCAN? Use the cylinder number to refer to the request.

    \emph{\textcolor{Green3}{\textbf{Answer:}}} SCAN (Elevator Algorithm) is a disk scheduling algorithm (see page \pageref{def: SCAN - Elevator Algorithm}). In this algorithm, the head moves in one direction (like an elevator), serving all requests in that direction. When it reaches the end, it reverses direction.

    The exercise text gives us a hint about the original movement: ``\texttt{[...]} from inside (lower cylinder number) to outside (higher cylinder number) \texttt{[...]}''. Therefore, we know that the next value will be 65, and then it will return since it is the highest value (end).
    \begin{equation*}
        \left(52 \to\right) 65 \xrightarrow{\text{end reached, moving inside...}} 50 \to 39 \to 23 \to 8
    \end{equation*}


    \item A company uses a computing system composed of three servers: one front-end server and two back-end servers. The system remains operational even if one of the two back-end servers fails. In all other cases, the system is considered offline. All servers have a Mean Time To Failure (MTTF) of 8 months. (i) Compute the probability that at least one of the three servers fails within the first 40 days, and (ii) compute the overall reliability of the computing system after 40 days. \emph{Notes: Use at least 4 decimal places for each intermediate calculation}.
    
    \emph{\textcolor{Green3}{\textbf{Answer:}}} The system works if the front-end works \textbf{AND} at least 1 back-end works. So only 1 back-end can fail. The system is down if both fails or if front-end fails.
    \begin{enumerate}
        \item First of all, we compute \textbf{single server reliability}. We use the reliability exponential function (see page \pageref{eq: failure rate}):
        \begin{equation*}
            R(t) = e^{-\lambda \: t} \quad \text{where } \lambda = \dfrac{1}{\text{MTTF}}
        \end{equation*}
        We convert MTTF to days: $8 \times 30 = 240$. Then:
        \begin{equation*}
            \lambda = \dfrac{1}{240} = 0.004166667
        \end{equation*}
        The reliability is:
        \begin{equation*}
            \begin{array}{rcl}
                R(t) &=& e^{- 0.004166667 \times 40} \\ [.3em]
                &=& e^{-0.166666667} \\ [.3em]
                &=& \dfrac{1}{e^{0.166666667}} \\ [1em]
                &=& \dfrac{1}{1.181360413} \\ [1em]
                &=& 0.846481725
            \end{array}
        \end{equation*}
        So probability of failure in 40 days (using unreliability function, see page \pageref{eq: unreliability}):
        \begin{equation*}
            Q(t) = 1 - R(t) = 1 - 0.846481725 = 0.153518275
        \end{equation*}

        \item Part (i) - \textbf{Probability that at least one of the 3 servers fails}. All 3 must work to have no failures. So:
        \begin{equation*}
            \begin{array}{rcl}
                P(\text{at least 1 fails}) &=& 1 - R^3 \\ [.3em]
                &=& 1 - \left(0.846481725\right)^3 \\ [.3em]
                &=& 1 - 0.60653066 \\ [.3em]
                &=& \mathbf{0.3934}6934
            \end{array}
        \end{equation*}

        \item Part (ii) - \textbf{Compute system reliability}.
        \begin{itemize}
            \item Front-end must work, so the probability is $R$.
            \item About back-end, at least 1 must work. So the formula is: $ 1 -$ both fail. The probability both back-ends fail (unreliability) is:
            \begin{equation*}
                Q^2 = 0.153518275^2 = 0.023567861
            \end{equation*}
            So, the probability at least 1 back-end works:
            \begin{equation*}
                \begin{array}{rcl}
                    P(\text{at least 1 back-end works}) &=& 1 - Q^2 \\ [.3em]
                    &=& 1 - 0.023567861 \\ [.3em]
                    &=& 0.976432139
                \end{array}
            \end{equation*}
            So final system reliability:
            \begin{equation*}
                \begin{array}{rcl}
                    R_{\text{system}} &=& R_{\text{FE}} \times R_{\text{BE}} \\ [.3em]
                    &=& 0.846481725 \times 0.976432139 \\ [.3em]
                    &=& \mathbf{0.8265}31961
                \end{array}
            \end{equation*}
        \end{itemize}
    \end{enumerate}

    \newpage

    \item A company uses a distributed data storage system composed of five storage units, organized in a \emph{primary} and \emph{secondary} storage groups, according to the following scheme: (i) Two redundant \emph{primary} storage units ($S_1$ and $S_2$), meaning that the system continues to work as long as at least one of the two remains optional; (ii) Three \emph{secondary} storage units ($S_3$, $S_4$, and $S_5$) in a configuration that requires that all three secondary units must be operational; (iii) Both storage groups should be operational to have the data storage system working. Each storage unit has the following parameters: Mean Time To Failure (MTTF) $=$ 8 months; Mean Time To Repair (MTTR) $=$ 5 day. Compute the overall availability of the system. \emph{Notes: Use at least 5 decimal places for each intermediate calculation}.

    \emph{\textcolor{Green3}{\textbf{Answer:}}} The problem is one of availability. The text here describes the schema.
    \begin{center}
        \begin{tikzpicture}[node distance=2cm]
            \draw (0,0) -- (0.5,0);

            \draw (0.5,0) -- (0.5,1);
            \draw (0.5,0) -- (0.5,-1);

            \draw (0.5,1) -- (1,1);
            \draw (0.5,-1) -- (1,-1);

            \node[
                draw,
                minimum width=1cm,
                minimum height=1cm,
                name=s1
            ] at (1.5,1) {$S_1$};
            \node[
                draw,
                minimum width=1cm,
                minimum height=1cm,
                name=s2
            ] at (1.5,-1) {$S_2$};

            \draw (s1) -- (2.5,1);
            \draw (s2) -- (2.5,-1);

            \draw (2.5,1) -- (2.5,0);
            \draw (2.5,-1) -- (2.5,0);

            \draw (2.5,0) -- (3,0);

            \node[
                draw,
                minimum width=1cm,
                minimum height=1cm,
                name=s3
            ] at (3.5,0) {$S_3$};

            \draw (4,0) -- (4.5,0);

            \node[
                draw,
                minimum width=1cm,
                minimum height=1cm,
                name=s4
            ] at (5,0) {$S_4$};

            \draw (5.5,0) -- (6,0);

            \node[
                draw,
                minimum width=1cm,
                minimum height=1cm,
                name=s5
            ] at (6.5,0) {$S_5$};

            \draw (7,0) -- (7.5,0);
        \end{tikzpicture}
    \end{center}

    The overall availability of the system is given by the availability of the parallel group times the availability of the series group.
    \begin{itemize}
        \item \textbf{Availability of each component}. Since the availability of each component is the same throughout the system, this value is pre-computed (see page \pageref{eq: Availability}).
        \begin{equation*}
            A = \dfrac{\text{MTTF}}{\text{MTTF} + \text{MTTR}} = \dfrac{240}{240 + 5} = 0.979591837
        \end{equation*}
        Note: we have converted MTTF from months to days ($8 \times 30 = 240$).

        \item \textbf{Availability Parallel Group}. Since all components have the same availability value, we are using the simplified version (see page \pageref{eq: Availability of Parallel System - Simplified}):
        \begin{equation*}
            \begin{array}{rcl}
                A_p &=& 1 - \left(1 - A\right)^{n} \\ [.3em]
                &=& 1 - \left(1 - 0.979591837\right)^{2} \\ [.3em]
                &=& 1 - \left(0.020408163\right)^{2} \\ [.3em]
                &=& 1 - 0.000416493 \\ [.3em]
                &=& 0.999583507
            \end{array}
        \end{equation*}

        \item \textbf{Availability Series Group}. Since all components have the same availability value, we are using the simplified version (see page \pageref{eq: Availability of Series System - Simplified}):
        \begin{equation*}
            A_s = A^{n} = \left(0.979591837\right)^3 = 0.94001649
        \end{equation*}

        \item \textbf{Overall System Availability}:
        \begin{equation*}
            A_{\text{system}} = A_p \times A_s = 0.999583507 \times 0.94001649 = \mathbf{0.93962}498
        \end{equation*}
    \end{itemize}

    \newpage

    \item You have been assigned the responsibility of optimizing the web server of your company. Through the analysis of your system logs, you have collected the following data (assume that this data is gathered during the peak hour, i.e., $T = \text{1h}$, on a representative business day).
    \begin{itemize}
        \item Number of completed requests $C = 10000$
        \item Average system utilization $U = 0.5$
    \end{itemize}
    What is the system throughput X and the requests service demand D?

    \textcolor{Green3}{\emph{\textbf{Answer:}}} Throughput is calculated by dividing the number of completed requests by the length of time the system is observed. The exercise gives us a time of 1 hour and 10,000 requests. Therefore, the throughput is (see page \pageref{eq: Throughput}):
    \begin{equation*}
        X = \dfrac{10'000}{1 \, \text{hour}} = \dfrac{10'000}{60 \, \text{minutes}} = \dfrac{10'000}{\left(60 \cdot 60\right)\, \text{seconds}} = 2.777777778 \, \text{req/sec}
    \end{equation*}
    The Service Demand can be obtained using the Utilization Law because we already have the average system utilization and system throughput (see page \pageref{eq: Utilization Law with Service Demand}):
    \begin{equation*}
        \begin{array}{rcl}
            U   &=& D \cdot X \\ [.3em]
            0.5 &=& D \cdot 2.777777778 \\ [.4em]
            D   &=& \dfrac{0.5}{2.777777778} \\ [1em]
            D   &=& 0.18 \, \text{sec} \\ [.3em]
            D   &=& 180 \, \text{ms} \\
        \end{array}
    \end{equation*}

    \item Considering the system described in Question 14, what is the maximum throughput that the system can achieve?
    
    \textcolor{Green3}{\emph{\textbf{Answer:}}} The maximum throughput is achieved at 100\% utilization. Then, apply the Utilization Law:
    \begin{equation*}
        \begin{array}{rcl}
            U &=& D \cdot X \\ [.3em]
            1 &=& 180 \, \text{ms} \, \cdot X_{\max} \\ [.4em]
            X_{\max} &=& \dfrac{1 \, \text{req}}{180 \, \text{ms}} \\ [1em]
            X_{\max} &=& 0.005555556 \\ [.3em]
            X_{\max} &=& 5.555556 \, \text{req/sec}
        \end{array}
    \end{equation*}

    \newpage

    \item Given the system described in Question 14, assume an expected peak workload growth of 4x (i.e. four times more requests within the same time interval). Determine the minimum number of additional servers required to ensure that the system utilization does not exceed 60\%. Assume that all new servers have the same configuration as the existing ones and that the incoming requests are evenly distributed across all servers. The answer should include the total number of servers required, including the original one.
    
    \textcolor{Green3}{\emph{\textbf{Answer:}}} System utilization must be 60\%, so $U = 0.6$.

    Since the peak growth is 4x, we can increase the number of completed requests from $10,000$ to $40,000$.

    Since each server cannot exceed 60\% utilization, we can use the utilization law to determine how many requests each server can handle.
    \begin{equation*}
        \begin{array}{rcl}
            U &=& D \cdot X \\ [.3em]
            0.6 &=& 180 \, \text{ms} \cdot X_{\max} \\ [.4em]
            X_{\max} &=& \dfrac{0.6}{180 \, \text{ms}} \\ [1em]
            X_{\max} &=& 0.003333333 \, \text{req/ms} \\ [.3em]
            X_{\max} &=& 3.333333 \, \text{req/s} \\ [.3em]
            X_{\max} &=& \left(3.333333 \cdot 60 \cdot 60\right) \, \text{req/hour} \\ [.3em]
            X_{\max} &\approx& 12'000 \, \text{req/hour}
        \end{array}
    \end{equation*}
    If a server can handle a maximum throughput of $12,000$ requests per hour, then we need $4$ servers:
    \begin{equation*}
        \text{Required Servers} = \dfrac{40,000}{12,000} = 3.333 \approx 4
    \end{equation*}
\end{enumerate}

\subsubsection*{Open Questions}

\begin{enumerate}
    \setcounter{enumi}{16}
    \item Explain the concept of Wear Leveling in the context of SSD.

    \textcolor{Green3}{\textbf{\emph{Answer:}}} Flash memory has a limited number of erase/write cycles for each block. To avoid uneven usage of specific blocks, wear leveling techniques spread erase/write operations across all blocks. On the hardware side, this is implemented using a flash translation layer (FTL) that monitors erase/write cycles per block, identifies cold blocks, and periodically reads valid data from cold blocks. The FTL then moves the data to fresher blocks, erases the original blocks, and reuses them.


    \item Discuss the concept of Datacenter Availability, and define what are the differences among the Tier Levels established by the Uptime Institute.
    
    \textcolor{Green3}{\textbf{\emph{Answer:}}} Data Center availability is defined by in four different tier level:
    \begin{enumerate}
        \item Tier 1: Single path, no redundancy, 99.671\% uptime.
        \item Tier 2: Single path, redundancy, 99.741\% uptime.
        \item Tier 3: Multiple paths, concurrently maintainable, 99.982\% uptime.
        \item Tier 4: Fault-tolerant, fully redundant, multiple independent systems, 99.995\% uptime.
    \end{enumerate}
\end{enumerate}