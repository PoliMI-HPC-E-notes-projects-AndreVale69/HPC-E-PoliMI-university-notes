\section{Hardware Infrastructures}

\subsection{System-level}

\subsubsection{Computing Infrastructures and Data Center Architectures}

\paragraph{Overview of Computing Infrastructures}

The \definition{Computing Continuum} is a \textbf{distributed computing environment} that seamlessly \textbf{integrates endpoints} (IoT devices), \textbf{edge computing}, and \textbf{cloud computing} to \textbf{optimize data processing across different layers of infrastructure}. Therefore, it consists of three main layer:
\begin{enumerate}
    \item \important{Endpoints} (IoT Devices \& Sensors) - The \textbf{Data Collectors}. At the very edge of the continuum, we find Endpoints. They are \textbf{small}, \textbf{low-power devices} embedded with sensors that \textbf{collect and transmit data}.
    
    \textcolor{Green3}{\faIcon{tools} \textbf{Examples}}
    \begin{itemize}
        \item A \textbf{temperature sensor} in a smart home, detecting room conditions.
        \item A \textbf{fitness tracker} measuring heart rate and step count.
        \item A \textbf{manufacturing robot} equipped with vibration sensors to detect faults.
    \end{itemize}

    \textcolor{Red2}{\faIcon{exclamation-triangle} \textbf{Challenges}}. While IoT devices are \textbf{great at collecting data}, they have \textbf{limited processing power} and cannot perform complex calculations. As a result, they often send raw data to the next level for further analysis.


    \item \important{Edge \& Fog Computing} - \textbf{Processing Data Close to the Source}. To \textbf{reduce the dependence on cloud processing}, the Edge Computing layer introduces \textbf{local computation} at the network's edge. This means that \hl{instead of sending all raw data to the cloud, some processing occurs closer to where the data is generated}.
    
    \textcolor{Green3}{\faIcon{tools} \textbf{Examples}}
    \begin{itemize}
        \item \textbf{Smart security cameras} analyzing video footage locally to detect intruders before sending alerts to a cloud-based security system.  
        \item \textbf{Autonomous vehicles} processing sensor data in real-time to make split-second driving decisions without relying on a remote server.  
    \end{itemize}

    \textcolor{Red2}{\faIcon{balance-scale} \textbf{Fog Computing vs. Edge Computing}}. \textbf{Fog Computing} refers to \textbf{processing data at an intermediate level}, such as an IoT gateway or a local network node.
    \begin{itemize}
        \item \textbf{Edge Computing} happens \textbf{directly on the device} producing the data.
        \item \textbf{Fog Computing} moves \textbf{processing to a nearby network gateway} or \textbf{edge server}.
    \end{itemize}

\newpage

    \textcolor{Green3}{\faIcon{check-circle} \textbf{Advantages}}
    \begin{itemize}[label=\textcolor{Green3}{\faIcon{check}}]
        \item \textcolor{Green3}{\textbf{Lower latency}}: Decisions are made \textbf{faster} since they don't rely on a cloud response.
        \item \textcolor{Green3}{\textbf{Bandwidth savings}}: Only important data is sent to the cloud, \textbf{reducing network congestion}.
        \item \textcolor{Green3}{\textbf{Privacy and security}}: Some data can remain \textbf{local}, avoiding unnecessary cloud exposure.
    \end{itemize}

    \textcolor{Red2}{\faIcon{exclamation-triangle} \textbf{Challenges}}. While Edge Computing is beneficial, it \textbf{requires additional hardware} and \textbf{power} at the network's edge, which can \textbf{increase infrastructure complexity and costs}.


    \item \important{Cloud Computing} - The \textbf{Powerhouse of Large-Scale Processing}. At the top of the continuum, cloud computing provides \textbf{massive storage and computation capabilities}. It is well-suited for applications that require significant computational resources, such as big data analytics, machine learning training, scalable web services (e.g., Netflix, YouTube, Amazon AWS).
    
    \textcolor{Green3}{\faIcon{tools} \textbf{Examples}}
    \begin{itemize}
        \item \textbf{AI model training} is typically done in cloud data centers because it requires significant GPU/TPU power.
        \item \textbf{E-commerce platforms} like Amazon use cloud computing to manage global inventory and transactions.
        \item \textbf{Streaming services} store and distribute high-quality video content from cloud servers.
    \end{itemize}

    \textcolor{Green3}{\faIcon{check-circle} \textbf{Advantages}}
    \begin{itemize}[label=\textcolor{Green3}{\faIcon{check}}]
        \item \textcolor{Green3}{\textbf{Scalability}}: Computing power can be adjusted based on demand.
        \item \textcolor{Green3}{\textbf{Resource efficiency}}: Data centers consolidate computing, making operations more cost-effective.
        \item \textcolor{Green3}{\textbf{Advanced analytics}}: Supports complex workloads like deep learning and data mining.
    \end{itemize}

    \textcolor{Red2}{\faIcon{exclamation-triangle} \textbf{Challenges}}
    \begin{itemize}
        \item \textbf{Latency issues} make real-time decision-making difficult.
        \item \textbf{Network dependency}: A constant internet connection is required.
        \item \textbf{High operational costs} for companies relying solely on cloud resources.
    \end{itemize}
\end{enumerate}
To summarize Computing Continuum creates a balanced computing environment by ensuring that workloads are distributed intelligently across IoT, edge, and cloud layers.

\newpage

\begin{figure}[!htp]
    \centering
    \includegraphics[width=\textwidth]{img/example-computing-infrastructure-1.pdf}
    \caption{\example{Examples} of Computing Infrastructures.\cite{computing-infrastructures-slides}}
\end{figure}

\highspace

\begin{figure}[!htp]
    \centering
    \includegraphics[width=\textwidth]{img/computing-continuum-1.png}
    \caption{This figure represents the Computing Continuum, how different computing layers interact to process data efficiently. It is a hierarchical structure moving from IoT devices at the left (closest to the data source) to cloud computing at the right (handling large-scale processing).\cite{computing-infrastructures-slides}}
\end{figure}

\highspace
In the following pages, we analyze the computing infrastructures mentioned in the previous example.

\newpage

\begin{center}
    \large
    \textcolor{Red3}{\textbf{Data Centers}}
\end{center}

\noindent
The definition of a Data Centers can be found on page \pageref{Data Center definition}.

\begin{flushleft}
    \textcolor{Green3}{\faIcon{check} \textbf{Data Centers Advantages}}
\end{flushleft}
\begin{itemize}
    \item \textbf{Lower IT costs}.
    \item \textbf{High Performance}.
    \item \textbf{Instant software updates}.
    \item \textbf{\dquotes{Unlimited} storage capacity}.
    \item \textbf{Increased data reliability}.
    \item \textbf{Universal data access}.
    \item \textbf{Device Independence}.
\end{itemize}

\begin{flushleft}
    \textcolor{Red2}{\faIcon{thumbs-down} \textbf{Data Centers Disadvantages}}
\end{flushleft}
\begin{itemize}
    \item \textbf{Require a constant internet connection}.
    \item \textbf{Do not work well with low-speed connections}.
    \item \textbf{Hardware Features might be limited}.
    \item \textbf{Privacy and security issues}.
    \item \textbf{High power Consumption}.
    \item \textbf{\underline{Latency in taking decision}}.
\end{itemize}

\newpage

\begin{center}
    \large
    \textcolor{Red3}{\textbf{Internet-of-Things (IoT)}}
\end{center}

\noindent
An \definition{Internet of Things (IoT)} \textbf{device is any everyday object embedded with sensors, software, and internet connectivity}.

\highspace
This allows to collect and exchange data with other devices and systems, typically over the internet, with limited need of process and store data.

\highspace
Some \example{examples} are \href{https://www.arduino.cc/}{Arduino}, \href{https://www.st.com/en/microcontrollers-microprocessors/stm32-32-bit-arm-cortex-mcus.html}{STM32}, \href{https://en.wikipedia.org/wiki/ESP32}{ESP32}, \href{https://docs.particle.io/argon/}{Particle Argon}.

\begin{flushleft}
    \textcolor{Green3}{\faIcon{check} \textbf{Internet-of-Things Advantages}}
\end{flushleft}
\begin{itemize}
    \item \textbf{Highly Pervasive}.
    \item \textbf{Wireless connection}.
    \item \textbf{Battery Powered}.
    \item \textbf{Low costs}.
    \item \textbf{Sensing and actuating}.
\end{itemize}

\begin{flushleft}
    \textcolor{Red2}{\faIcon{thumbs-down} \textbf{Internet-of-Things Disadvantages}}
\end{flushleft}
\begin{itemize}
    \item \textbf{Low computing ability}.
    \item \textbf{Constraints on energy}.
    \item \textbf{Constraints on memory (RAM/FLASH)}.
    \item \textbf{Difficulties in programming}.
\end{itemize}

\longline

\begin{center}
    \large
    \textcolor{Red3}{\textbf{Embedded (System) PCs}}
\end{center}

\noindent
An \definition{Embedded System} is a computer system, a combination of a computer processor, computer memory, and input/output peripheral devices, that has a dedicated function within a larger mechanical or electronic system.

\highspace
A few \example{examples}: \href{https://www.hardkernel.com/}{Odroid}, \href{https://www.raspberrypi.com/}{Raspberry}, \href{https://developer.nvidia.com/embedded/jetson-nano}{jetson nano}, \href{https://www.coral.ai/}{Google Coral}.

\begin{flushleft}
    \textcolor{Green3}{\faIcon{check} \textbf{Embedded System Advantages}}
\end{flushleft}
\begin{itemize}
    \item \textbf{Persuasive computing}.
    \item \textbf{High performance unit}.
    \item \textbf{Availability of development boards}.
    \item \textbf{Programmed as PC}.
    \item \textbf{Large community}.
\end{itemize}

\begin{flushleft}
    \textcolor{Red2}{\faIcon{thumbs-down} \textbf{Embedded System Disadvantages}}
\end{flushleft}
\begin{itemize}
    \item \textbf{Pretty high power consumption}.
    \item \textbf{(Some) Hardware design has to be done}.
\end{itemize}

\newpage

\begin{center}
    \large
    \textcolor{Red3}{\textbf{Edge/Fog Computing Systems}}
\end{center}

\noindent
The key \textbf{difference} between \definition{Fog Computing} and \definition{Edge Computing} is associated with the location \textbf{where the data is processed}:
\begin{itemize}
    \item In \textbf{edge computing}, the data is processed closest to the sensors.

    \item In \textbf{fog computing}, the computing is moved to processors linked to a local area network (IoT gateway).
\end{itemize}
Edge computing places the intelligence in the connected devices themselves, whereas, fog computing puts in the local area network.

\begin{figure}[!htp]
    \centering
    \includegraphics[width=.8\textwidth]{img/edge-fog-computing-systems-1.pdf}
\end{figure}

\begin{flushleft}
    \textcolor{Green3}{\faIcon{check} \textbf{Fog/Edge Advantages}}
\end{flushleft}
\begin{itemize}
    \item \textbf{High computational capacity}.
    \item \textbf{Distributed computing}.
    \item \textbf{Privacy and security}.
    \item \textbf{Reduced Latency in making a decision}.
\end{itemize}

\begin{flushleft}
    \textcolor{Red2}{\faIcon{thumbs-down} \textbf{Fog/Edge Disadvantages}}
\end{flushleft}
\begin{itemize}
    \item \textbf{Require a power connection}.
    \item \textbf{Require connection with the Cloud}.
\end{itemize}

\newpage

\begin{table}[!htp]
    \centering
    \begin{tabular}{@{} l p{11.5em} p{11.5em} @{}}
        \toprule
        \textbf{Feature} & \textbf{Edge Computing} & \textbf{Fog Computing} \\
        \midrule
        \textbf{Location} & Directly on device or nearby device. & Intermediary devices between edge and cloud. \\
        \\
        \textbf{Processing Power} & Limited due to device constraints, sending data to central server for analysis. & More powerful than edge devices. However, sending data to a central server for analysis. \\
        \\
        \textbf{Primary Function} & Real-time decision-making, low latency. However, central server analyzing combined data and sending only relevant information further. & Pre-process and aggregate data, reduce bandwidth usage. However, central server analyzing combined data and sending only relevant information further. \\
        \\
        \textbf{Advantages} & Low latency, reduced reliance on cloud, security for sensitive data. & Bandwidth efficiency, lower cloud costs, complex analysis capabilities. \\
        \\
        \textbf{Disadvantages} & Limited processing power, single device focus. & Increased complexity, additional infrastructure cost. \\
        \bottomrule
    \end{tabular}
    \caption{Differences between Edge and Fog Computing Systems.}
\end{table}

\newpage

\paragraph{The Datacenter as a Computer}

The concept of treating the datacenter itself as a singular, unified computing system represents a significant evolution in computing infrastructures. \hl{Traditionally}, computing resources were individually managed and isolated (\hl{client-server computing}); now, \hl{the paradigm shifts towards} collective resource management and holistic infrastructure design (\hl{datacenter-based computing}, particularly Warehouse-Scale Computers, WSCs).

\highspace
\begin{flushleft}
    \textcolor{Green3}{\faIcon{check-circle} \textbf{User Experience Improvements}}
\end{flushleft}
For end-users, the shift to centralized computing in datacenters provides significant advantages:
\begin{itemize}
    \item \textcolor{Green3}{\textbf{Ease of Management}}. \textbf{No need for local configuration or backups}.
    \begin{itemize}[label=\textcolor{Green3}{\faIcon{check}}]
        \item Users no longer need to manually configure and maintain software or hardware on their personal devices.
        \item Centralized cloud-based systems handle updates, security patches, and maintenance automatically.
    \end{itemize}
    \item \textcolor{Green3}{\textbf{Ubiquity of Access}}
    \begin{itemize}[label=\textcolor{Green3}{\faIcon{check}}]
        \item \textbf{Users can access} services and data \textbf{from any device}, \textbf{anywhere}, as long as they have an internet connection.
        \item This allows for seamless synchronization\footnote{Seamless synchronization: The automatic and continuous updating of data across multiple devices or platforms without user intervention, ensuring consistency and accessibility in real-time.} across multiple devices (e.g., accessing Google Drive files from a laptop, tablet, or phone).
    \end{itemize}
    \item \textcolor{Green3}{\textbf{Reduced Hardware Dependencies}}
    \begin{itemize}[label=\textcolor{Green3}{\faIcon{check}}]
        \item \textbf{Computing power} is \textbf{shifted} from end-user devices \textbf{to cloud services}.
        \item Even low-powered devices (e.g., smartphones, Chromebooks) can perform complex tasks by offloading computation to datacenters.
    \end{itemize}
    For example, cloud-based applications like Google Docs allow real-time document editing across multiple devices without requiring powerful local hardware.
\end{itemize}

\newpage

\begin{flushleft}
    \textcolor{Green3}{\faIcon{check-circle} \textbf{Advantages for Service Providers (vendors)}}
\end{flushleft}
From the perspective of software and hardware vendors, the transition to datacenter computing brings several operational and economic benefits.
\begin{itemize}
    \item \textcolor{Green3}{\textbf{Software-as-a-Service (SaaS) Enables Faster Development}}
    \begin{itemize}
        \item[\textcolor{Red2}{\faIcon{times}}] Traditional software development involved deploying updates individually to millions of heterogeneous client devices.
        \item[\textcolor{Green3}{\faIcon{check}}] Now, \textbf{cloud-based applications} allow:
        \begin{itemize}
            \item Centralized software deployment.
            \item Rapid updates and feature rollouts.
            \item Easier bug fixes and security patches.
        \end{itemize}
    \end{itemize}
    For example, instead of releasing a new version of Microsoft Office every few years, Microsoft now provides Microsoft 365, where updates are continuously applied.

    \item \textcolor{Green3}{\textbf{Simplified Hardware Deployment}}
    \begin{itemize}
        \item[\textcolor{Red2}{\faIcon{times}}] Traditional software development required compatibility with millions of different client hardware configurations.
        \item[\textcolor{Green3}{\faIcon{check}}] Now, with server-side computing, companies \textbf{only need to support a few well-tested hardware platforms} in their datacenters.
    \end{itemize}
    For example, instead of optimizing software for thousands of different PC models, Google can optimize Gmail and YouTube to run efficiently in their controlled cloud environment.
    
    \item \textcolor{Green3}{\textbf{Better Resource Utilization}}
    \begin{itemize}
        \item[\textcolor{Green3}{\faIcon{check}}] Datacenters allow for \textbf{efficient resource allocation}, ensuring that computing power is shared dynamically across multiple applications.
        \item[\textcolor{Green3}{\faIcon{check}}] Cloud providers can achieve \textbf{high resource utilization}, reducing costs and improving performance.
    \end{itemize}
    For example, cloud providers like AWS and Google Cloud use containerization (e.g., Kubernetes) to dynamically allocate computing resources to workloads as needed.
\end{itemize}

\newpage

\begin{flushleft}
    \textcolor{Green3}{\faIcon{check-circle} \textbf{Benefits of Server-Side Computing}}
\end{flushleft}
Beyond user experience and vendor advantages, server-side computing enables several key technological improvements.
\begin{itemize}
    \item \textcolor{Green3}{\textbf{Faster Introduction of New Hardware}}
    \begin{itemize}
        \item[\textcolor{Red2}{\faIcon{times}}] In a traditional computing model, upgrading hardware required end-users to buy new devices.
        \item[\textcolor{Green3}{\faIcon{check}}] In a datacenter-centric model, \textbf{hardware upgrades happen in the cloud}, where service providers can deploy: new processors, AI accelerators (e.g., TPUs, GPUs), more efficient storage solutions.
    \end{itemize}
    For example, Google introduced TPUs (Tensor Processing Units) for machine learning, allowing AI workloads to run more efficiently in their datacenters without requiring users to upgrade their devices.

    \item \textcolor{Green3}{\textbf{Cost Efficiency for Large-Scale Applications}}
    \begin{itemize}
        \item[\textcolor{Red2}{\faIcon{times}}] Many applications require enormous computational resources that are impractical for client devices.
        \item[\textcolor{Green3}{\faIcon{check}}] Running these applications in datacenters allows for \textbf{better scaling} and \textbf{lower cost per user}.
    \end{itemize}
    For example, Google Search processes billions of queries daily, requiring petabytes of storage and high-speed computation. Another example is training deep learning models (e.g., ChatGPT, image recognition, autonomous vehicles) is computationally expensive and only feasible in large datacenters.
\end{itemize}

\newpage

\paragraph{Warehouse-Scale Computers}

With the advent of server-side computing and large-scale Internet services, a new class of computing systems emerged: \definition{Warehouse-Scale Computers (WSCs)}. These systems were designed to meet the unique requirements of large-scale applications, which often involve multiple interdependent programs interacting within a unified infrastructure. Characteristics of WSCs are:
\begin{itemize}
    \item Unlike traditional data centers, \textbf{WSCs are} not just collections of servers, but \textbf{complete computing environments}.
    \item A single WSC might consist of \textbf{thousands of servers}, all working together as a \textbf{single, large computing unit}.
    \item \textbf{WSCs support complex, large-scale services} such as:
    \begin{itemize}
        \item Search engines (Google Search, Bing)
        \item Cloud-based email (Gmail, Outlook)
        \item Online maps and navigation services (Google Maps, Waze)
        \item Machine learning and artificial intelligence platforms (Google's TensorFlow, OpenAI's models)
    \end{itemize}
    The scale at which these systems operate requires a fundamentally different architectural approach-one optimized for efficiency, scalability, and unified resource management.
\end{itemize}

\begin{definitionbox}[: Warehouse-Scale Computers (WSCs)]
    \definition{Warehouse-Scale Computers (WSCs)} is a computing system designed to operate at an extreme scale, integrating massive software infrastructure, vast data repositories, and a unified hardware platform to function as a single, cohesive computing entity, typically owned and managed by a single organization.
\end{definitionbox}

\highspace
\begin{flushleft}
    \textcolor{Green3}{\faIcon{balance-scale} \textbf{Warehouse-Scale Computers vs. Traditional Data Centers}}
\end{flushleft}
\begin{itemize}
    \item \important{Traditional Data Centers: A More Fragmented Approach}. A traditional data center is a facility that houses a collection of servers and networking equipment, providing computing resources for various applications and organizations. These environments are characterized by:
    \begin{itemize}
        \item A \textbf{diverse range of applications}, each running on \textbf{its own dedicated hardware infrastructure}.
        \item \textbf{Applications that do not communicate} with one another or share resources.
        \item A \textbf{multi-tenant model}, where hardware and software are used by \hl{multiple organizations or business units}.  
    \end{itemize}
    This model works well for enterprises hosting multiple independent applications, such as corporate databases, enterprise software, and web hosting services.

    \item \important{Warehouse-Scale Computers: A Unified Approach}. In contrast, WSCs are designed to function as a \textbf{single, highly integrated computing unit}. Unlike traditional data centers, which support numerous small applications in isolation, WSCs are optimized for \textbf{massive, unified workloads} controlled by a \textbf{single organization}. Key characteristics of WSCs include:
    \begin{itemize}
        \item \textbf{Homogeneous hardware and software}: The infrastructure is standardized, making it easier to manage and scale.
        \item \textbf{Centralized resource management}: Instead of treating each server as an independent entity, WSCs manage computing resources dynamically across a large-scale cluster.
        \item \textbf{Support for large-scale applications}: WSCs are built to run a \textbf{small number of massive applications}, such as search engines, AI models, or cloud services.
    \end{itemize}
    By shifting towards WSCs, companies like Google, Amazon, and Microsoft have been able to achieve better efficiency, scalability, and cost-effectiveness, leading to the next evolution of computing infrastructure.
    
    For example, when we perform a Google Search, the request is processed across multiple servers simultaneously in a WSC, with different servers handling indexing, ranking, and query matching. The user only sees the final result, but behind the scenes, thousands of machines collaborate in milliseconds.
\end{itemize}

\begin{itemize}
    \item \important{Ownership}
    \begin{itemize}
        \item \textbf{Traditional DC}: Multiple organizations/tenants
        \item \textbf{WSCs}: Single organization
    \end{itemize}
    \item \important{Application Scale}
    \begin{itemize}
        \item \textbf{Traditional DC}: Many small-to-medium apps
        \item \textbf{WSCs}: Few massive applications
    \end{itemize}
    \item \important{Hardware}
    \begin{itemize}
        \item \textbf{Traditional DC}: Heterogeneous, diverse setups
        \item \textbf{WSCs}: Homogeneous architecture
    \end{itemize}
    \item \important{Resource Management}
    \begin{itemize}
        \item \textbf{Traditional DC}: Isolated per application
        \item \textbf{WSCs}: Centralized, dynamic allocation
    \end{itemize}
    \item \important{Software Execution}
    \begin{itemize}
        \item \textbf{Traditional DC}: Independent workloads
        \item \textbf{WSCs}: Interdependent, large-scale services
    \end{itemize}
\end{itemize}

\highspace
\begin{flushleft}
    \textcolor{Green3}{\faIcon{undo} \textbf{From Data Centers to WSCs and Back}}
\end{flushleft}
Initially, WSCs were designed exclusively for handling large-scale, internet-facing applications. However, as cloud computing evolved, the \textbf{lines} \textbf{bet-\break ween} \textbf{traditional data centers and warehouse-scale computing began to blur}. Modern cloud computing is a convergence of these two models: traditional Datacenter and WSC.
\begin{itemize}
    \item Today's public cloud platforms (AWS, Google Cloud, Microsoft Azure) run many small applications, much like traditional data centers.
    \item These \textbf{applications rely on Virtual Machines (VMs) and Containers}, which efficiently distribute workloads across a warehouse-scale infrastructure.
\end{itemize}

\begin{table}[!htp]
    \centering
    \begin{tabular}{@{} l | l | l @{}}
        \toprule
        \textbf{Era} & \textbf{Computing Model} & \textbf{Example} \\
        \midrule
        \textbf{Past} & Traditional Data Centers & Enterprise IT infrastructure \\ [.3em]
        \textbf{Present} & Warehouse-Scale Computers & Google Search, AI training \\ [.3em]
        \textbf{Future} & Hybrid WSC-DC Model & Cloud computing (AWS) \\
        \bottomrule
    \end{tabular}
\end{table}

\newpage

\paragraph{Multiple Data Centers}

\noindent
Sometimes the data centers are \textbf{located far apart}. \textbf{Multiple data centers} are (often) \textbf{replicas of the same service}:
\begin{itemize}
    \item To \emph{reduce user latency}
    \item To \emph{improve service throughput}
\end{itemize}
Typically, a request is fully processed within one data center.

\highspace
The world is divided into \definition{Geographic Areas (GAs)}. Each Area is defined by Geo-political boundaries (or country borders). Also, there are at least two computing regions in each geographical Area.

\highspace
The \definition{Computing Regions (CRs)} are the smallest geographic unit of the infrastructure from the customer's perspective. Multiple Data centers within the same region are not exposed to customers.

However, they are defined by a latency-defined perimeter, typically less than 2ms for round-trip latency.
Finally, they're located hundreds of miles apart, with considerations for different flood zones, etc. It is too far from synchronous replication but suitable for disaster recovery.

\highspace
The \definition{Availability Zones (AZs)} are finer-grain \textbf{locations within a single computing region}. They allow customers to run mission-critical applications with high availability and fault tolerance to Data Center failures. Because there are fault-isolated locations with redundant power, cooling, and networking (they are different from the concept of the Availability Set).

\highspace
This hierarchical structure ensures efficient data management and compliance with local data laws while optimizing network performance through strategically placing data centers.

\begin{figure}[!htp]
    \centering
    \includegraphics[width=.8\textwidth]{img/availability-zones-region-geography.png}
    \caption{Example of \href{https://learn.microsoft.com/en-us/azure/reliability/availability-zones-overview?tabs=azure-cli}{Azure Availability Zones}.}
\end{figure}

\newpage

\paragraph{Warehouse-Scale Computing / Data Centers Availability}

The services provided through WSCs (or DCs) \textbf{must guarantee high availability}, typically aiming for at least 99.99\% uptime (e.g. one hour of downtime per year).

\highspace
Some \example{examples}:
\begin{itemize}
    \item 99,90\% on single instance VMs with premium storage for a more accessible lift and shift;
    
    \item 99,95\% VM uptime SLA for Availability Sets (AS) to protect for failures within a data center;

    \item 99,99\% VM uptime SLA through Availability Zones.
\end{itemize}
Such fault-free operation is more accessible when an extensive collection of hardware and system software is involved.

\highspace
\textbf{WSC workloads must be designed to gracefully tolerate large numbers of component faults with little or no impact on service level performance and availability!}

\hfill

\longline

\hfill

\paragraph{Architectural overview of Warehouse-Scale Computing}

\begin{figure}[!htp]
    \centering
    \includegraphics[width=\textwidth]{img/WSC-architecture-1.pdf}
    \caption{Architectural overview of Warehouse-Scale Computing.}
\end{figure}

\newpage

\begin{itemize}
    \item \textbf{Server} (section~\ref{subsubsection: Server (computation, HW accelerators)}, page~\pageref{subsubsection: Server (computation, HW accelerators)}). Servers are the \textbf{leading processing equipment}: different types according to CPUs, RAM, local storage, accelerators, and form factor. The servers are \textbf{hosted on individual shelves} and are the \textbf{basic building blocks of Data Centers and Warehouse-Scale Computers}. They are interconnected by hierarchies of networks and supported by the shared power and cooling infrastructure.

    \item \textbf{Storage} (section~\ref{subsubsection: Storage (type, technology)}, page~\pageref{subsubsection: Storage (type, technology)}). Disks, flash SSDs, and Tapes are the \textbf{building blocks} of today's \textbf{WSC storage systems}. These devices are \textbf{connected to the Data Center network and managed by sophisticated distributed systems}.
    
    Some \example{examples}:
    \begin{itemize}
        \item Direct Attached Storage (DAS)

        \item Network Attached Storage (NAS)

        \item Storage Area Networks (SAN)

        \item RAID controllers
    \end{itemize}

    \item \textbf{Networking} (section~\ref{subsubsection: Networking (architecture and technology)}, page~\pageref{subsubsection: Networking (architecture and technology)}). The \definition{Data Center Network (DCN)} \textbf{enables efficient data transfer and interaction between various components}. The data processing ecosystem within the DCs needs to reach the DC services from outside.
    Communication equipment includes switches, Routers, cables, DNS or DHCP servers, Load balancers, Firewalls, etc.

    \item \textbf{Building and Infrastructure}. WSC has other essential components related to power delivery, cooling, and building infrastructure that must be considered. Some interesting numbers:
    \begin{itemize}
        \item Data Centers with up to 110 football-pitch size.
        
        \item 2-100s MW power consumption (100k houses), and the largest in the world is 650 MW.
    \end{itemize}
\end{itemize}