\subsection{Disk performance}\label{subsection: Disk performance}

\subsubsection{HDD}\label{subsubsection: HDD performance}

We can calculate some performance metrics related to the types of delay of HDD (page \pageref{four types of hdd delay}).
\begin{itemize}
    \item \definition{Full Rotation Delay} $R$ is:
    \begin{equation}
        R = \dfrac{1}{\texttt{DiskRPM}}
    \end{equation}
    And in seconds:
    \begin{equation}
        R_{\text{sec}} = 60 \times R
    \end{equation}
    From the $R_{\text{sec}}$ we can also calculate the \definition{total rotation average}:
    \begin{equation}
        T_{\text{rotation AVG}} = \dfrac{R_{\text{sec}}}{2}
    \end{equation}
    It is half of a full rotation, because on average, the sector will be halfway around the platter from the current head position.

    \item \definition{Seek Time}. The time to seek from one track to another depends on the distance moved. In real systems, this relation isn't perfectly linear, but it's often approximated as (\definition{seek average}):
    \begin{equation}
        T_{\text{seek AVG}} = \dfrac{T_{\text{seek MAX}}}{3}
    \end{equation}
    Where $T_{\text{seek MAX}}$ is the \textbf{time for the longest possible seek} (from \emph{outermost to innermost} track) and the division by 3 assumes a \textbf{uniform random distribution of seeks} across the disk.

    \item \definition{Transfer time}. It is the \textbf{time that data is either read from or written to the surface}. It \textbf{includes} the time the \textbf{head needs to pass on the sectors} and the \textbf{I/O transfer}:
    \begin{equation}
        T_{\text{transfer}} = \dfrac{\text{R/W of a sector}}{\text{Data transfer rate}}
    \end{equation}
    The \textbf{total time to complete a disk I/O operation} is called the $T_{\text{I/O}}$ \definition{Service Time}:
    \begin{equation}
        T_{\text{I/O}} = T_{\text{seek}} + T_{\text{rotation}} + T_{\text{transfer}} + T_{\text{overhead}}
    \end{equation}
    If the disk is shared among processes, we must also consider \textbf{queueing time}. And the \definition{Response Time} is:
    \begin{equation}
        T_{\text{response}} = T_{\text{queue}} + T_{\text{I/O}}
    \end{equation}
    Where $T_{\text{queue}}$ depends on queue length, disk utilization rate, variance in request time, arrival rate of I/O requests.
    \newpage
    \begin{figure}[!htp]
        \centering
        \includegraphics[width=.7\textwidth]{img/performance-hdd-1.pdf}
        \caption{Service and response time.}
    \end{figure}
\end{itemize}

\begin{exercisebox}[: mean service time of an I/O operation]\label{exercise: mean service time of an I/O operation}
    The data of the exercise are:
    \begin{itemize}
        \item Read/Write of a sector of $512$ bytes ($0.5$ KB)
        \item Data \underline{transfer} rate: $50$ MB/sec
        \item \underline{Rotation} speed: $10000$ RPM (Round Per Minute)
        \item Mean \underline{seek} time: $6$ ms
        \item Overhead \underline{Controller}: $0.2$ ms
    \end{itemize}
    To calculate the \emph{service time} $T_{\text{I/O}}$, we need the following information:
    \begin{itemize}
        \item $T_{\text{seek}}$, which we already have, and it is $6$ ms.
        \item $T_{\text{rotation}}$
        \item $T_{\text{transfer}}$
        \item $T_{\text{overhead}}$, which we already have, and it is $0.2$ ms.
    \end{itemize}
    We also know the rotation and transfer information, but we want to know the \emph{mean} service time. Then we calculate the total rotation average $T_{\text{rotation AVG}}$:
    \begin{equation*}
        \begin{array}{rcl}
            R &=& \dfrac{1}{\texttt{DiskRPM}} = \dfrac{1}{10000} = 0.0001 \\ [1em]
            R_{\text{sec}} &=& 60 \cdot R = 60 \cdot 0.0001 = 0.006 \text{ seconds} \\ [1em]
            T_{\text{rotation AVG}} &=& \dfrac{R_{\text{sec}}}{2} = \dfrac{0.006}{2} = 0.003 \text{ seconds} = 3 \text{ ms}
        \end{array}
    \end{equation*}
    Finally, the transfer time is easy to calculate because we have the R/W of a sector and the data transfer rate. First we do a conversions from megabytes to kilobytes:
    \begin{equation*}
        \begin{array}{rcl}
            \text{Data transfer rate: }&& 50 \text{ MB/sec} \\ [1em] 
            &=& 50 \cdot 1024 \text{ KB/sec} \\ [1em] 
            &=& 51200 \text{ KB/sec} \\ [1em]
            T_{\text{transfer}} &=& \dfrac{0.5 \text{ KB/sec}}{51200 \text{ KB/sec}} \\ [1em]
            &=& 0.000009765625 \text{ sec } \cdot 1000 \\ [1em]
            &=& 0.009765625 \text{ ms } \approx 0.01 \text{ ms}
        \end{array}
    \end{equation*}
    The exercise can be completed by calculating the mean I/O service time required:
    \begin{equation*}
        \begin{array}{rcl}
            T_{\text{I/O}} &=& T_{\text{seek}} + T_{\text{rotation}} + T_{\text{transfer}} + T_{\text{overhead}} \\ [.8em]
            T_{\text{I/O}} &=& 6 + 3 + 0.01 + 0.2 = 9.21 \text{ ms}
        \end{array}
    \end{equation*}
\end{exercisebox}

\noindent
The previous service time is supposed to be only for very \textbf{pessimistic cases} where \textbf{sectors are fragmented on the disk in the worst possible way}. This can happen because the files are tiny (each file is contained in one block) or the disk is externally fragmented.

\highspace
Thus, each access to a sector requires to pay rotational latency and seek time.
This is not the case in many circumstances because the files are larger than one block and stored contiguously.

\highspace
We can measure the \definition{Data Locality $DL$} of a disk as the \textbf{percentage of blocks that do not need seek or rotational latency to be found}.

\highspace
Thanks to the Data Locality, it is possible to calculate the \definition{Average Service Time}:
\begin{equation}
    T_{\text{I/O AVG}} = \left(1 - DL\right) \cdot \left(T_{\text{seek}} + T_{\text{rotation}}\right) + T_{\text{transfer}} + T_{\text{controller}}
\end{equation}


\begin{exercisebox}[: data locality]
    The data of the exercise are:
    \begin{itemize}
        \item Read/Write of a sector of $512$ bytes ($0.5$ KB)
        \item \underline{Data Locality}: $DL = 75\%$
        \item Data \underline{transfer} rate: $50$ MB/sec
        \item \underline{Rotation} speed: $10000$ RPM (Round Per Minute)
        \item Mean \underline{seek} time: $6$ ms
        \item Overhead \underline{Controller}: $0.2$ ms
    \end{itemize}
    Since the Data Locality is $75\%$, only $25\%$ of the operations are affected by the DL:
    \begin{equation*}
        \left(1-DL\right) = \left(1 - 0.75\right) = 0.25
    \end{equation*}
    See the exercise on page \pageref{exercise: mean service time of an I/O operation} to understand the values of $T_{\text{seek}}$, $T_{\text{rotation}}$, $T_{\text{transfer}}$ and $T_{\text{overhead}}$:
    \begin{itemize}
        \item $T_{\text{seek}} = 6$
        \item $T_{\text{rotation}} = 3$
        \item $T_{\text{transfer}} = 0.01$
        \item $T_{\text{overhead}} = 0.2$
    \end{itemize}
    Finally the average time for read/write a sector of $0.5$ KB with a DL of $75\%$ is:
    \begin{equation*}
        \begin{array}{rcl}
            T_{\text{I/O AVG}} &=& 0.25 \cdot \left(6 + 3\right) + 0.01 + 0.2 \\ [1em]
            &=& 0.25 \cdot 9 + 0.21 \\ [1em]
            &=& 2.46 \text{ ms}
        \end{array}
    \end{equation*}
\end{exercisebox}

\begin{exercisebox}[: influence of \dquotes{not optimal} data allocation]
    The data of the exercise are:
    \begin{itemize}
        \item $10$ blocks of $1/10$ MB for each block ($10$ blocks of $1/10$ MB \dquotes{not well} distributed on disk)
        \item $T_{\text{seek}} = 6$ ms
        \item $T_{\text{rotation AVG}} = 3$ ms
        \item Data transfer rate: $50$ MB/sec
    \end{itemize}
    In the exercise you were asked to calculate the time taken to transfer a 1 MB file with 100\% and 0\% data locality:
    \begin{itemize}
        \item Data Locality equals to 100\%:
        \begin{itemize}
            \item An initial seek ($6$ ms)
            \item A total rotation average ($3$ ms)
            \item Now it's possible to do the 1MB global transfer directly because there are no blocks to seek or rotation latency:
            \begin{equation*}
                \text{1 MB of 50 MB} = \dfrac{1}{50} = 0.02 \text{ seconds } \cdot 1000 = 20 \text{ ms}
            \end{equation*}
            \item The total time is:
            \begin{equation*}
                T = 6 + 3 + 20 = 29 \text{ ms}
            \end{equation*}
        \end{itemize}

        \newpage
        \item Data Locality equals to 0\%:
        \begin{itemize}
            \item An initial seek ($6$ ms)
            \item A total rotation average ($3$ ms)
            \item In this case, it's not possible to do a global transfer directly, because each block is affected by the seek or rotation latency. Then we have to transfer block by block and calculate the delay:
            \begin{equation*}
                \text{1 MB of 10 MB} = \dfrac{1}{10} = 0.1 \text{ seconds } \cdot 1000 = 100 \text{ ms}
            \end{equation*}
            \item The total time is:
            \begin{equation*}
                T = \left(6 + 3 + 2\right) \cdot 10 = 110 \text{ ms}
            \end{equation*}
            Where $10$ is the number of blocks.
        \end{itemize}
    \end{itemize}
    Note: the controller times is not considered.
\end{exercisebox}

\newpage

\subsubsection{RAID}

We can calculate some performance metrics related to the RAID technology (page \pageref{paragraph: RAID}).
\begin{itemize}
    \item Let's assume:
    \begin{itemize}
        \item A constant Failure Rate;
        \item An exponentially distributed time to failure;
        \item The case of independent failures.
    \end{itemize}
    (conditions usually used to determine the disk MTTF). 
    
    The \definition{Mean Time To Failure of a disk array $\texttt{MTTF}_{\text{diskArray}}$} is equal to the relationship between the \texttt{MTTF} of a single disk and the number of disks:
    \begin{equation}\label{eq: Mean Time To Failure of a disk array}
        \texttt{MTTF}_{\text{diskArray}} = \dfrac{\texttt{MTTF}_{\text{singleDisk}}}{\# \: \texttt{Disks}}
    \end{equation}
    Large disk arrays are \textbf{too unstable to be used without any fault tolerance approach}. Disks do not have huge \texttt{MTTF} since it is highly probable they will be replaced in a "short time". Note that the \textbf{RAID 0 has no redundancy}!
    \begin{equation}
        \texttt{MTTF}_{\texttt{RAID 0}} = \texttt{MTTF}_{\text{diskArray}} = \dfrac{\texttt{MTTF}_{\text{singleDisk}}}{\# \: \texttt{Disks}}
    \end{equation}


    \item RAID levels greater than level zero use redundancy to improve reliability. Then, when a disk fails, it should be replaced, and the information should be reconstructed on the new disk using the redundant information. The \texttt{MTTR} is the \textbf{time needed for this action}! As always, the $N$ value is the number of disks in the array. The \definition{Mean Time To Failure of a RAID $\texttt{MTTF}_{\texttt{RAID}}$} (except the level zero!) is:
    \begin{equation}\label{eq: Mean Time To Failure of a RAID}
        \texttt{MTTF}_{\texttt{RAID}} = \left(\dfrac{\texttt{MTTF}_{\text{singleDisk}}}{N}\right) \times \left(\dfrac{1}{\text{Probability}_{\text{additionalCriticalFailuresInMTTR}}}\right)
    \end{equation}
    Where:
    \begin{itemize}
        \item $\dfrac{\texttt{MTTF}_{\text{singleDisk}}}{N}$ is the \textbf{MTTF for the array of $N$ disks}.
        
        \item $\dfrac{1}{\text{Probability}_{\text{additionalCriticalFailuresInMTTR}}}$ is the \textbf{probability of other critical failures in the array before repairing the failed disk}. The RAID level and type of redundancy determine it.
    \end{itemize}
    In detail, the \textbf{Mean Time To Failure of each RAID level} (except the zero) is:
    \begin{itemize}
        \item \texttt{\underline{RAID 1}} - With a single copy of each disk, one drive can fail, and if we are lucky, $N \div 2$ drives can fail without data loss. Then the \definition{MTTF of \texttt{RAID 1} $\texttt{MTTF}_{\text{RAID 1}}$} is:
        \begin{equation}
            \texttt{MTTF}_{\text{RAID 1}} = \left(
                \dfrac{\texttt{MTTF}_{\text{singleDisk}}}{N}
            \right) \times \left(
                \dfrac{1}{\text{Probability}_{\text{2ndCriticalFailureInMTTR}}}
            \right)
        \end{equation}
        \begin{equation}
            \text{Probability}_{\text{2ndCriticalFailureInMTTR}} = \left(\dfrac{1}{\texttt{MTTF}_{\text{singleDisk}}}\right) \times \texttt{MTTR}
        \end{equation}
        Where:
        \begin{itemize}
            \item $\dfrac{1}{\texttt{MTTF}_{\text{singleDisk}}}$ is the \textbf{failure rate for the copy of the failed disk}.
            
            \item $\texttt{MTTR}$ is the \textbf{period of interest before replacement}.
        \end{itemize}


        \item \texttt{\underline{RAID 0 + 1}} - When one disk in a stripe group fails, the entire group goes off. Then the \definition{MTTF of \texttt{RAID 01} $\texttt{MTTF}_{\text{RAID 0 + 1}}$} is:
        \begin{equation}
            \texttt{MTTF}_{\text{RAID 01}} = \left(
                \dfrac{\texttt{MTTF}_{\text{singleDisk}}}{N}
            \right) \times \left(
                \dfrac{1}{\text{Probability}_{\text{2ndCriticalFailureInMTTR}}}
            \right)
        \end{equation}
        It is not the same as RAID 1 because the probability is:
        \begin{equation}
            \text{Probability}_{\text{2ndCriticalFailureInMTTR}} = \left(\dfrac{G}{\texttt{MTTF}_{\text{singleDisk}}}\right) \times \texttt{MTTR}
        \end{equation}
        Where:
        \begin{itemize}
            \item $G$ is the \textbf{number of disks in a stripe group}.

            \item $\dfrac{G}{\texttt{MTTF}_{\text{singleDisk}}}$ is the \textbf{failure rate for one of the disks in the other group}.
            
            \item $\texttt{MTTR}$ is the \textbf{period of interest before replacement}.
        \end{itemize}


        \item \texttt{\underline{RAID 1 + 0}} - To fail, the same copy in both groups has to fail, but multiple failure can be tolerated. Then the \definition{MTTF of \texttt{RAID 10} $\texttt{MTTF}_{\text{RAID 1 + 0}}$} is:
        \begin{equation}
            \texttt{MTTF}_{\text{RAID 10}} = \left(
                \dfrac{\texttt{MTTF}_{\text{singleDisk}}}{N}
            \right) \times \left(
                \dfrac{1}{\text{Probability}_{\text{2ndCriticalFailureInMTTR}}}
            \right)
        \end{equation}
        It is not the same as RAID 1 because the probability is:
        \begin{equation}
            \text{Probability}_{\text{2ndCriticalFailureInMTTR}} = \left(\dfrac{1}{\texttt{MTTF}_{\text{singleDisk}}}\right) \times \texttt{MTTR}
        \end{equation}
        Where:
        \begin{itemize}
            \item $\dfrac{1}{\texttt{MTTF}_{\text{singleDisk}}}$ is the \textbf{failure rate for the copy of the failed disk}.

            \item $\texttt{MTTR}$ is the \textbf{period of interest before replacement}.
        \end{itemize}


        \item \texttt{\underline{RAID 4}} and \texttt{\underline{RAID 5}} - To fail, two disks have to fail before replacement. Then the \definition{MTTF of \texttt{RAID 4} $\texttt{MTTF}_{\text{RAID 4}}$} and the \definition{MTTF of \texttt{RAID 5} $\texttt{MTTF}_{\text{RAID 5}}$} is:
        \begin{equation}
            \texttt{MTTF}_{\text{RAID 4}} = \texttt{MTTF}_{\text{RAID 5}} = \left(
                \dfrac{\texttt{MTTF}_{\text{singleDisk}}}{N}
            \right) \times \left(
                \dfrac{1}{\text{Probability}_{\text{2ndFailureInMTTR}}}
            \right)
        \end{equation}
        And the probability is:
        \begin{equation}
            \text{Probability}_{\text{2ndFailureInMTTR}} = \left(\dfrac{\left(N-1\right)}{\texttt{MTTF}_{\text{singleDisk}}}\right) \times \texttt{MTTR}
        \end{equation}
        Where:
        \begin{itemize}
            \item $\dfrac{\left(N-1\right)}{\texttt{MTTF}_{\text{singleDisk}}}$ is the \textbf{failure rate for one of the other disks}.

            \item $\texttt{MTTR}$ is the \textbf{period of interest before replacement}.
        \end{itemize}


        \item \texttt{\underline{RAID 6}} - Two disks failures at the same time are tolerated. Then the \definition{MTTF of \texttt{RAID 6} $\texttt{MTTF}_{\text{RAID 6}}$} is:
        \begin{equation}
            \texttt{MTTF}_{\text{RAID 6}} = \left(
                \dfrac{\texttt{MTTF}_{\text{singleDisk}}}{N}
            \right) \times \left(
                \dfrac{1}{\text{Probability}_{\text{2ndAnd3rdFailureInMTTR}}}
            \right)
        \end{equation}
        And the probability is:
        \begin{equation}
            \text{Probability}_{\text{2ndAnd3rdFailureInMTTR}} = \text{Probability}_{\text{2ndFailure}} \times \text{Probability}_{\text{3ndFailure}}
        \end{equation}
        Where:
        \begin{itemize}
            \item $\text{Probability}_{\text{2ndFailure}}$:
            \begin{equation}
                \text{Probability}_{\text{2ndFailure}} = \left(\dfrac{\left(N-1\right)}{\texttt{MTTF}_{\text{singleDisk}}}\right) \times \texttt{MTTR}
            \end{equation}
            \begin{itemize}
                \item $\dfrac{\left(N-1\right)}{\texttt{MTTF}_{\text{singleDisk}}}$ is the \textbf{failure rate for one of the other disks}.

                \item \texttt{MTTR} is the \textbf{period of interest before the replacement}.
            \end{itemize}

            \item $\text{Probability}_{\text{3ndFailure}}$:
            \begin{equation}
                \text{Probability}_{\text{3ndFailure}} = \left(\dfrac{\left(N-2\right)}{\texttt{MTTF}_{\text{singleDisk}}}\right) \times \dfrac{\texttt{MTTR}}{2}
            \end{equation}
            \begin{itemize}
                \item $\dfrac{\left(N-2\right)}{\texttt{MTTF}_{\text{singleDisk}}}$ is the \textbf{failure rate for one of the remaining disks}.\vspace{.5em}

                \item $\dfrac{\texttt{MTTR}}{2}$ is the \textbf{average overlapping period between first and second disk replacement} (both disk not yet replaced).
            \end{itemize}
        \end{itemize}
    \end{itemize}
\end{itemize}

\newpage

\begin{table}[!htp]
    \centering
    \begin{tabular}{@{} l l @{}}
        \toprule
        \textbf{RAID level} & \textbf{Metric} \\
        \midrule
        \texttt{RAID 0}     & $\texttt{MTTF}_{\texttt{RAID 0}} = \dfrac{\texttt{MTTF}_{\text{singleDisk}}}{N}$ \\ \\
        \texttt{RAID 1 + 0} & $\texttt{MTTF}_{\texttt{RAID 10}} = \dfrac{\left(\texttt{MTTF}_{\text{singleDisk}}\right)^{2}}{\left(N \times \texttt{MTTR}\right)}$ \\ \\
        \texttt{RAID 0 + 1} & $\texttt{MTTF}_{\texttt{RAID 01}} = \dfrac{\left(\texttt{MTTF}_{\text{singleDisk}}\right)^{2}}{\left(N \times G \times \texttt{MTTR}\right)}$ \\ \\
        \texttt{RAID 4}     & $\texttt{MTTF}_{\texttt{RAID 4}} = \dfrac{\left(\texttt{MTTF}_{\text{singleDisk}}\right)^{2}}{\left(N \times N-1 \times \texttt{MTTR}\right)}$ \\ \\
        \texttt{RAID 5}     & $\texttt{MTTF}_{\texttt{RAID 5}} = \dfrac{\left(\texttt{MTTF}_{\text{singleDisk}}\right)^{2}}{\left(N \times N-1 \times \texttt{MTTR}\right)}$ \\ \\
        \texttt{RAID 6}     & $\texttt{MTTF}_{\texttt{RAID 6}} = \dfrac{2 \times \left(\texttt{MTTF}_{\text{singleDisk}}\right)^{3}}{\left(N \times N-1 \times N-2 \times \texttt{MTTR}^{2}\right)}$ \\
        \bottomrule
    \end{tabular}
    \caption{MTTF summary RAID levels.}
\end{table}