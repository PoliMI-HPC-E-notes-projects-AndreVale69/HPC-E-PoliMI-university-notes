\section{Methods}

\subsection{Reliability and availability of datacenters}

\subsubsection{Definition}

\definition{Dependability} \textbf{measures how much we trust a system}. More technically, it is the \textbf{ability of a system to perform its functionality while exposing}:
\begin{itemize}
    \item \textbf{Reliability}. Continuity of correct service.
    \item \textbf{Availability}. Readiness for correct service.
    \item \textbf{Maintainability}. Ability for easy maintenance.
    \item \textbf{Safety}. Absence of catastrophic consequences.
    \item \textbf{Security}. Confidentiality and integrity of data.
\end{itemize}

\begin{flushleft}
    \textcolor{Green3}{\faIcon{question-circle} \textbf{Ok, but why should we be interested in dependability?}}
\end{flushleft}
During the implementation of a product, there is much effort to make sure that the implementation:
\begin{itemize}
    \item matches specifications,
    \item fulfils requirements,
    \item meets constraints,
    \item optimizes selected parameters (such as performance, energy, etc.).
\end{itemize}
Nevertheless, even if all the above aspects are satisfied, the systems fail because something broke! The causes can be multiple: defects, process variation, degraded transistors, radiation, noise, design errors, software bugs, OS bugs, malicious attacks, and human errors.

\highspace
Then, \textbf{dependability is essential} to check how much we can trust a system despite the effects of failure. If we are not convinced, consider that \textbf{a failure may have high costs if it impacts economic losses or physical damage}. Not only that, a single system failure may affect a large number of people and may cause information loss with a high consequent recovery cost. For the previous reasons, the systems that are not dependable are likely \emph{not to be used or adopted}.

\begin{flushleft}
    \textcolor{Green3}{\faIcon{question-circle} \textbf{It seems very important, so when should we think about dependability?}}
\end{flushleft}
Consistently, both at \emph{design-time} to:
\begin{itemize}
    \item Analyze the system under design;
    \item Measure dependability properties;
    \item Modify the design if required;
\end{itemize}
And \emph{runtime} to:
\begin{itemize}
    \item Detect malfunctions;
    \item Understand causes;
    \item React.
\end{itemize}
Furthermore, the failures in development should be avoided, and the design should take failures into account and guarantee that control and safety are achieved when failures occur. The effects of such failures should be predictable and deterministic, not catastrophic!

\begin{flushleft}
    \textcolor{Green3}{\faIcon{question-circle} \textbf{Always think about dependability, but where should it be applied?}}
\end{flushleft}
In the past, dependability was relevant only for \emph{safety-critical} and \emph{mission-critical} application environments: space, nuclear, and avionics. Note that:
\begin{itemize}
    \item \definition{Mission-critical systems} are architectures where a \textbf{failure} during operation \textbf{can have severe or irreversible effects on the mission the system is carrying out} (for example, satellites, surveillance drones, unmanned vehicles, etc.).
    
    \item \definition{Safety-critical systems} are architectures where a \textbf{failure} during operation can \textbf{directly threaten human life} (for example, aircraft control systems, medical instrumentation, railway signaling, nuclear reactor control systems).
\end{itemize}
However, in the computing infrastructures, the \textbf{downtime is the enemy of every data center}! So it is important to consider the dependability in each scenario in order to guarantee that everything works properly.
