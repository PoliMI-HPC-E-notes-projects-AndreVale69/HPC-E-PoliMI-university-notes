\paragraph{Complexity of algorithm}

\begin{flushleft}
    \textcolor{Green3}{\faIcon{clock} \textbf{BFS Algorithm - Time Complexity}}
\end{flushleft}
The BFS time complexity%
\footnote{In theoretical computer science, the time complexity is the computational complexity that describes the amount of computer time it takes to run an algorithm. Time complexity is commonly estimated by counting the number of elementary operations performed by the algorithm, supposing that each elementary operation takes a fixed amount of time to perform. Thus, the amount of time taken and the number of elementary operations performed by the algorithm are taken to be related by a constant factor. (\href{https://en.wikipedia.org/wiki/Time_complexity}{source})}
can be expressed as $O\left(\left|N\right|+\left|A\right|\right)$, since \textbf{every node and every edge will be explored in the \underline{worst case}}.
\begin{itemize}
    \item $\left|N\right|$ is the number of \textbf{nodes};
    \item $\left|A\right|$ is the number of \textbf{edges} in the graph.
\end{itemize}
Note that $O\left(\left|A\right|\right)$ may vary between $O\left(1\right)$ and $O\left(N^{2}\right)$, depending on how sparse the input graph is. For example, for \textbf{dense graphs}, exactly $O\left(N^{2}\right)$.

\highspace
\begin{flushleft}
    \textcolor{Green3}{\faIcon{memory} \textbf{BFS Algorithm - Space Complexity}}
\end{flushleft}
When the number of nodes (or vertices) in the graph is known ahead of time, and additional data structures are used to determine which vertices have already been added to the queue, the space complexity%
\footnote{The space complexity of an algorithm or a data structure is the amount of memory space required to solve an instance of the computational problem as a function of characteristics of the input. It is the memory required by an algorithm until it executes completely. This includes the memory space used by its inputs, called input space, and any other (auxiliary) memory it uses during execution, which is called auxiliary space. (\href{https://en.wikipedia.org/wiki/Space_complexity}{source})}
can be expressed as $O\left(\left|N\right|\right)$, where $\left|N\right|$ is the number of vertices. This is in addition to the space required for the graph itself, which may vary depending on the graph representation used by an implementation of the algorithm.

\highspace
In other words, the algorithm needs:
\begin{itemize}
    \item The \textbf{space to store} the set $N$, i.e. the \textbf{set of all nodes} in the graph.
    \item The \textbf{space to store the graph itself} depends on the implementation used.
\end{itemize}