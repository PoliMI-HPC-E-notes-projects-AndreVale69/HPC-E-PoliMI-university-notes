\subsubsection{Graph reachability problem}

In general the \definitionWithSpecificIndex{graph reachability problem}{Graph reachability problem} can be formulated as follows.

\begin{definitionbox}[: graph reachability problem]
    Given a directed graph $G = \left(N,A\right)$ and a node $s$, determine all the node that are reachable from $s$.
\end{definitionbox}

\noindent
Where $N$ is the set of nodes and $A$ is the set of edges.

\highspace
The problem takes:
\begin{itemize}
    \item As \textbf{input} a \emph{\textbf{graph}} $G = \left(N,A\right)$ described by the successor lists and node $s \in N$.
    
    \item As \textbf{output} produces a \emph{\textbf{subset}} $M \subseteq N$ \emph{\textbf{of nodes}} of the graph $G$ reachable from $s$.
\end{itemize}
Our goal is to devise an efficient algorithm that allows us to find all nodes reachable from $s$.

\newpage

\paragraph{Description and algorithm}

\begin{definitionbox}[: Breadth-First Search]
    \definition{Breadth-First Search (BFS)} is an \textbf{algorithm for searching a tree data structure for a node that satisfies a given property}. It starts at the tree root and explores all nodes at the present depth prior to moving on to the nodes at the next depth level. Extra memory, usually a queue, is needed to keep track of the child nodes that were encountered but not yet explored.
\end{definitionbox}

\begin{lstlisting}[language=pseudo-code, caption={Graph reachability problem: Breadth-First Search $O\left(\left|N\right|+\left|A\right|\right)$}]
Q $\leftarrow$ $\{$s$\}$; $\label{bfs: q-definition}$
M $\leftarrow$ $\emptyset$; $\label{bfs: m-definition}$
while Q $\ne$ $\emptyset$ do: $\label{bfs: while cycle}$
    u $\leftarrow$ node in Q; $\label{bfs: take an element from the queue}$
    Q $\leftarrow$ Q $\setminus$ $\{$u$\}$; $\label{bfs: remove the popped item from the queue}$
    // label u
    M $\leftarrow$ M $\cup$ $\{$u$\}$ $\label{bfs: labeled as explored}$;
    for (u, v) $\in$ $\delta^{+}$ (u) do: $\label{bfs: for each tuple in outgoing cut}$
        if v $\notin$ M and v $\notin$ Q: $\label{bfs: if adjacent node is not in reachable set and not in the queue}$
            Q $\leftarrow$ Q $\cup$ $\{$v$\}$ $\label{bfs: add the node v to the queue}$
\end{lstlisting}

\begin{itemize}
    \item[Rows \ref{bfs: q-definition}-\ref{bfs: m-definition}.] Declare a queue \texttt{Q} containing the nodes reachable from the source \texttt{s} and \textbf{not yet processed}. It is managed as a FIFO (First-In First-Out) queue. By definition, we add the \texttt{s} node at the beginning because it is our entry point.
    
    Then we declare the set \texttt{M}. It represents the subset of nodes of the graph that are reachable from the source \texttt{s}. Obviously, it is empty at the beginning of the algorithm.

    \item[Row \ref{bfs: while cycle}.] The BFS algorithm continues to process the nodes until the queue is empty. As long as there is an element, it continues.
    
    \item[Rows \ref{bfs: take an element from the queue}-\ref{bfs: remove the popped item from the queue}.] Take a node from the queue \texttt{Q} and assign it to the variable \texttt{u}. Also remove the element \texttt{u} from the set \texttt{Q}. In other words, perform a difference operation between the sets \texttt{Q} and the set composed only of the element \texttt{u} ($\texttt{Q} \setminus \left\{\texttt{u}\right\}$).
    
    For example, in Python we can get the same result using the \href{https://docs.python.org/3/library/collections.html#collections.deque.popleft}{\texttt{popleft}} method of the \href{https://docs.python.org/3/tutorial/datastructures.html#using-lists-as-queues}{\texttt{deque} data structure}.

    \item[Row \ref{bfs: labeled as explored}.] Using the union between sets, add the visited node \texttt{u} to the subset \texttt{M} of reachable nodes. This operation is also called \dquotes{labeling} because you are \emph{labeling} a node as \emph{visited}.
    
    \item[Row \ref{bfs: for each tuple in outgoing cut}.] Iterate each tuple (node \texttt{u} just popped from the queue, node \texttt{v} adjacent to node \texttt{u}) in the outgoing cut set of node \texttt{u}.
    
    \item[Rows \ref{bfs: if adjacent node is not in reachable set and not in the queue}-\ref{bfs: add the node v to the queue}.] If the adjacent node \texttt{v} is not in the reachable set \texttt{M} and it is not in the queue (so it is not waiting to be evaluated), add the adjacent node \texttt{v} to \texttt{Q} using the union set operation.
\end{itemize}
As we said, the algorithm continues until the queue is not empty. Note that the queue is updated each time a neighboring node is found that is not already in the solution set (\texttt{M}).