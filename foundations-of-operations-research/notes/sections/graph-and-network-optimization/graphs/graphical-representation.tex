\subsubsection{Graphical representation}

Such a matrix can easily be represented as a graph. This guarantees that it can be stored efficiently in a computer. But to understand how to do this in general, it's important to understand some other important properties:
\begin{itemize}
    \item For any graph $G$ with $n$ nodes, the \textbf{number of edges} satisfies:
    \begin{itemize}
        \item $m \le \dfrac{n\left(n-1\right)}{2}$ if $G$ is undirected.
        \item $m \le n\left(n-1\right)$ if $G$ is directed.
    \end{itemize}

    \item A graph is \definitionWithSpecificIndex{dense}{Dense graph} if $m \approx n^{2}$ and \definitionWithSpecificIndex{sparse}{Sparse graph} if $m \ll n^{2}$. Where $m$ is the number of arcs and $n$ the number of nodes.

    \item For dense directed graphs, exist an \textbf{adjacency matrix} $A_{n \times n}$:
    \begin{equation*}
        \begin{cases}
            a_{ij} = 1 & \text{if } \left(i,j\right) \in A \\
            a_{ij} = 0 & \text{otherwise}
        \end{cases}
    \end{equation*}
\end{itemize}
To build the adjacency matrix it is necessary to create a \textbf{list of successors} for each node. In other words, for \textbf{each node} we need to write the \textbf{outgoing edges} and write the matrix.
\begin{equation*}
    A = \begin{bmatrix}
        0 & 1 & 0 & 1 & 0 \\
        0 & 0 & 1 & 0 & 0 \\
        0 & 0 & 0 & 1 & 1 \\
        0 & 1 & 0 & 0 & 1 \\
        0 & 0 & 0 & 1 & 0
    \end{bmatrix}
    \hspace{2em}
    \begin{array}{rcl}
        S\left(1\right) &=& \left\{2,4\right\} \\
        S\left(2\right) &=& \left\{3\right\} \\
        S\left(3\right) &=& \left\{4,5\right\} \\
        S\left(4\right) &=& \left\{2,5\right\} \\
        S\left(5\right) &=& \left\{4\right\}
    \end{array}
\end{equation*}
Each row represents a node, and we set the value 1 if the column index is a node that has the arc of the row node as its incoming edge. So row one (node one) has the value one in column two (node two) and column four (node four).