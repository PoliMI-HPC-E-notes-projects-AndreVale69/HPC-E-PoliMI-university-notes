\subsection{Multi-objective programming}\label{subsection: Multi-objective programming}

\begin{flushleft}
    \textcolor{Green3}{\faIcon{question-circle} \textbf{How is it born?}}
\end{flushleft}
Even though some real-word problems can be reduced to a matter of a single objective very often it is hard to define all the aspects in terms of a single objective. Defining multiple objectives often gives a better idea of the task.

\highspace
\definition{Multi-objective programming} (also known as \definition{Multi-objective optimization} or \definition{Pareto optimization}) is an \textbf{area of multiple-criteria decision making} that is concerned with mathematical optimization problems involving \textbf{more than one objective function to be optimized simultaneously}. Multi-objective is a type of vector optimization that has been applied in many fields of science, including engineering, economics and logistics where optimal decisions need to be taken in the presence of trade-offs between two or more conflicting objectives. 

Minimizing cost while maximizing comfort while buying a car, and maximizing performance whilst minimizing fuel consumption and emission of pollutants of a vehicle are \example{examples} of multi-objective optimization problems involving two and three objectives, respectively.\cite{abraham2005evolutionary}

\highspace
Suppose to minimize $f_{1}\left(\mathbf{x}\right)$ and maximize $f_{2}\left(\mathbf{x}\right)$ (e.g. laptop: $f_{1}$ is cost and $f_{2}\left(\mathbf{x}\right)$ is performance):
\begin{enumerate}
    \item Turn it into a \definition{single objective problem} by expressing the two objectives in terms of the same unit (e.g. monetary unit):
    \begin{equation*}
        \min \lambda_{1}f_{1}\left(\mathbf{x}\right) - \lambda_{2}f_{2}\left(\mathbf{x}\right)
    \end{equation*}
    for appropriate scalars $\lambda_{1}$ and $\lambda_{2}$.

    \item Optimize the \definition{primary objective} function and turn the other objective into a constraint:
    \begin{equation*}
        \underset{x \in \tilde{X}}{\max} f_{2}\left(\mathbf{x}\right) \hspace{1em} \text{where} \hspace{1em} \tilde{X} = \left\{\mathbf{x} \in X \: : \: f_{1}\left(\mathbf{x}\right) \le \varepsilon \right\}
    \end{equation*}
    for appropriate constant $\varepsilon$.
\end{enumerate}
This is a simple introduction, the more detailed explanation will be explained in the following pages.