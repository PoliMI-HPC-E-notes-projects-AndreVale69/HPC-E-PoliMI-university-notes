\subsubsection{Diet problem}

A canteen has to plan the composition of the meals that it provides. A meal can be composed of the types of food indicated in the following table. Costs, in Euro per hg, and availabilities, in hg, are also indicated.

\begin{table}[!htp]
    \centering
    \begin{tabular}{@{} c c c @{}}
        \toprule
        \textbf{Food} & \textbf{Cost} & \textbf{Availability} \\
        \midrule
        Bread & 0.1 & 4 \\
        Milk & 0.5 & 3  \\
        Eggs & 0.12 & 1 \\
        Meat & 0.9 & 2  \\
        Cake & 1.3 & 2  \\
        \bottomrule
    \end{tabular}
\end{table}

\noindent
A meal must contain at least the following amount of each nutrient:

\begin{table}[!htp]
    \centering
    \begin{tabular}{@{} c c @{}}
        \toprule
        \textbf{Nutrient} & \textbf{Minimum quantity} \\
        \midrule
        Calories    & 600 cal   \\
        Proteins    & 50 g      \\
        Calcium     & 0.7 g     \\
        \bottomrule
    \end{tabular}
\end{table}

\noindent
Each hg of each type of food contains to following amount of nutrients:

\begin{table}[!htp]
    \centering
    \begin{tabular}{@{} c c c c @{}}
        \toprule
        \textbf{Food} & \textbf{Calories} & \textbf{Proteins} & \textbf{Calcium} \\
        \midrule
        Bread   & 30 cal    & 5 g   & 0.02 g \\
        Milk    & 50 cal    & 15 g  & 0.15 g \\
        Eggs    & 150 cal   & 30 g  & 0.05 g \\
        Meat    & 180 cal   & 90 g  & 0.08 g \\
        Cake    & 400 cal   & 70 g  & 0.01 g \\
        \bottomrule
    \end{tabular}
\end{table}

\noindent
Give a linear programming formulation for the problem of finding a diet (a meal) of minimum total cost which satisfies the minimum nutrient requirements.

\highspace
\begin{flushleft}
    \textcolor{Green3}{\faIcon{square-root-alt} \textbf{Diet problem formulation}}
\end{flushleft}
\begin{itemize}
    \item Sets
    \begin{itemize}
        \item $I$: food types
        \item $J$: nutrients
    \end{itemize}

    \item Parameters
    \begin{itemize}
        \item $c_{i}$: unit cost of food type $i \in I$ 
        \item $q_{i}$: available quantity of food type $i \in I$
        \item $b_{j}$: minimum quantity of nutrient $j \in J$ required
        \item $a_{ij}$: quantity of nutrient $j \in J$ per unit of food of type $i \in I$
    \end{itemize}

    \newpage

    \item Variables
    \begin{itemize}
        \item $x_i$: quantity of food of type $i \in I$ included in the diet
    \end{itemize}

    \item Model
    \begin{equation*}
        \begin{array}{llll}
            \min & \displaystyle\sum_{i\in I} c_{i} x_{i} \qquad & & \text{(cost)} \\ [1.4em]
            \text{s.t.} & \displaystyle\sum_{i\in I}a_{ij}x_{ij} \ge b_j & j \in J & \text{(min nutrients)} \\ [1.4em]
            & x_i \leq q_i & i \in I & \text{(availability)} \\ [.5em]
            & x_i \geq 0 &  i \in I & \text{(nonnegativity)}
        \end{array}
    \end{equation*}
\end{itemize}

\highspace
\begin{flushleft}
    \textcolor{Green3}{\faIcon{laptop-code} \textbf{Diet problem implementation}}
\end{flushleft}

\begin{enumerate}
    \item Write the dataset specified by Input:
    \begin{lstlisting}[language=Python]
# We need to import the mip package (useful later)
import mip

# Food
I = {'Bread', 'Milk', 'Eggs', 'Meat', 'Cake'}

# Nutrients
J = {'Calories', 'Proteins', 'Calcium'}

# Cost in Euro per hg of food
c = {
    'Bread': 0.1, 
    'Milk': 0.5, 
    'Eggs': 0.12,
    'Meat': 0.9,
    'Cake': 1.3
}

# Availability per hg of food
q = {
    'Bread': 4,
    'Milk': 3,
    'Eggs': 1,
    'Meat': 2,
    'Cake': 2
}

# Minimum nutrients 
b = {
    'Calories': 600,
    'Proteins': 50,
    'Calcium': 0.7
}

# Nutrients per hf of food
a = {
    ('Bread', 'Calories'): 30,
    ('Milk', 'Calories'): 50,
    ('Eggs', 'Calories'): 150,
    ('Meat', 'Calories'): 180,
    ('Cake', 'Calories'): 400,
    ('Bread', 'Proteins'): 5,
    ('Milk', 'Proteins'): 15,
    ('Eggs', 'Proteins'): 30,
    ('Meat', 'Proteins'): 90,
    ('Cake', 'Proteins'): 70,
    ('Bread', 'Calcium'): 0.02,
    ('Milk', 'Calcium'): 0.15,
    ('Eggs', 'Calcium'): 0.05,
    ('Meat', 'Calcium'): 0.08,
    ('Cake', 'Calcium'): 0.01
}\end{lstlisting}

    \item Now we create an empty model and add the variables:
    \begin{lstlisting}[language=Python]
# Define a model
model = mip.Model()

# Define variables
x = [model.add_var(name = i, lb = 0) for i in I]

# Define the objective function
model.objective = mip.minimize(
    mip.xsum(x[i] * c[food] for i, food in enumerate(I))
)

# CONSTRAINTS
# Availability constraint
for i, food in enumerate(I):
    model.add_constr(x[i] <= q[food])

# Minimum nutrients
for j in J:
    model.add_constr(
        mip.xsum(
            x[i] * a[food, j] for i, food in enumerate(I)
        ) >= b[j]
    )\end{lstlisting}

    \item The model is complete. Let us solve it and print the optimal solution:
    \begin{lstlisting}[language=Python]
# Optimizing command
print('Optimizing: ' + model.optimize())
# Optimizing: OptimizationStatus.OPTIMAL

# Optimal objective function value
print('Optimal objective function value: {}'.format(
    model.objective.x
))
# Optimal objective function value: 3.37

# Printing the variables values
for i in model.vars:
    print(i.name, i.x)
# Meat 1.5000000000000002
# Bread 3.9999999999999996
# Cake 0.0
# Milk 3.0
# Eggs 1.0
\end{lstlisting}
\end{enumerate}