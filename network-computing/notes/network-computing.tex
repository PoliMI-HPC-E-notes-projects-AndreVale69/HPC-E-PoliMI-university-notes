\documentclass[a4paper]{article}
\usepackage[T1]{fontenc}		    % \chapter package
\usepackage[english]{babel}
\usepackage[english]{isodate}  		% date format
\usepackage{graphicx}			    % manage images
\usepackage{subcaption}             % manage subfigures
\usepackage{amsfonts}
\usepackage{booktabs}			    % high quality tables
\usepackage{amsmath}			    % math package
\usepackage{amssymb}			    % another math package (e.g. \nexists)
\usepackage{bm}                     % bold math symbols
\usepackage{mathtools}			    % emphasize equations
\usepackage{amsthm}			        % better theorems
\usepackage{enumitem}			    % manage list
\usepackage{pifont}			        % nice itemize
\usepackage{cancel}			        % cancel math equations
\usepackage{caption}			    % custom caption
\usepackage[]{mdframed}			    % box text
\usepackage{multirow}			    % more lines in a table
\usepackage[x11names]{xcolor}		% RGB color
\usepackage[many]{tcolorbox}		% colorful box
\usepackage{listings}               % use to writing code
\usepackage{qrcode}                 % use for qr-codes
\usepackage{fontawesome5}           % use for icons
\usepackage{ragged2e}               % use for \flushleft command
\usepackage{cite}                   % references
\usepackage{imakeidx}               % index
\usepackage{fancyhdr}               % extensive facilities, both for constructing headers and footers 
\usepackage{wrapfig}                % allows figures or tables to have text wrapped around them
\makeindex[program=makeindex, columns=1,
           title=Index, 
           intoc,
           options={-s index-style.ist}]
\usepackage{multicol}               % multi-column itemize
\usepackage{soul}                   % highlight text (as a highlighter)
\usepackage{adjustbox}              % avoid table overfull

%\pdfcompresslevel=0
%\pdfobjcompresslevel=0

\definecolor{codegreen}{rgb}{0,0.6,0}
\definecolor{codegray}{rgb}{0.5,0.5,0.5}
\definecolor{codepurple}{rgb}{0.58,0,0.82}
\definecolor{backcolour}{rgb}{255,255,255}
\lstdefinestyle{mystyle}{
    backgroundcolor=\color{backcolour},   
	commentstyle=\color{codegreen},
	keywordstyle=\color{magenta},
	numberstyle=\tiny\color{codegray},
	stringstyle=\color{codepurple},
	basicstyle=\ttfamily\footnotesize,
	breakatwhitespace=false,         
	breaklines=true,                 
	captionpos=b,                    
	keepspaces=true,                 
	numbers=left,                    
	numbersep=5pt,                  
	showspaces=false,                
	showstringspaces=false,
	showtabs=false,                  
	tabsize=2,
    mathescape,
    frame=lines
}

\lstdefinelanguage{JavaScript}{
	keywords={typeof, new, true, false, catch, function, return, null, catch, switch, var, if, in, while, do, else, case, break, reduce, map},
	keywordstyle=\color{blue}\bfseries,
	ndkeywords={class, export, boolean, throw, implements, import, this, const},
	ndkeywordstyle=\color{darkgray}\bfseries,
	identifierstyle=\color{black},
	sensitive=false,
	comment=[l]{//},
	morecomment=[s]{/*}{*/},
	commentstyle=\color{codegreen}\ttfamily,
	stringstyle=\color{red}\ttfamily,
	morestring=[b]',
	morestring=[b]"
}

\lstset{
	language=JavaScript,
	backgroundcolor=\color{lightgray},
	extendedchars=true,
	basicstyle=\footnotesize\ttfamily,
	showstringspaces=false,
	showspaces=false,
	numbers=left,
	numberstyle=\footnotesize,
	numbersep=9pt,
	tabsize=2,
	breaklines=true,
	showtabs=false,
	captionpos=b
}

\lstset{style=mystyle}


% thanks Mico: https://tex.stackexchange.com/a/60218/312896
\makeatletter
\renewcommand\paragraph{\@startsection{paragraph}{4}{\z@}%
            {-2.5ex\@plus -1ex \@minus -.25ex}%
            {1.25ex \@plus .25ex}%
            {\normalfont\normalsize\bfseries}}
\makeatother
\setcounter{secnumdepth}{4} % how many sectioning levels to assign numbers to
\setcounter{tocdepth}{4}    % how many sectioning levels to show in ToC


% draw a frame around given text
\newcommand{\framedtext}[1]{%
	\par%
	\noindent\fbox{%
		\parbox{\dimexpr\linewidth-2\fboxsep-2\fboxrule}{#1}%
	}%
}


% table of content links
\usepackage{xcolor}
\usepackage[linkcolor=black, citecolor=blue, urlcolor=cyan]{hyperref} % hypertexnames=false
\hypersetup{
	colorlinks=true
}


\newtheorem{theorem}{\textcolor{Red3}{\underline{Theorem}}}
\newtheorem{lemma}[theorem]{\textcolor{Red3}{\underline{Lemma}}}
\renewcommand{\qedsymbol}{QED}
\newcommand{\dquotes}[1]{``#1''}
\newcommand{\longline}{\noindent\rule{\textwidth}{0.4pt}}
\newcommand{\circledtext}[1]{\raisebox{.5pt}{\textcircled{\raisebox{-.9pt}{#1}}}}
\newcommand{\important}[1]{\textcolor{red}{\textbf{#1}}}
\newcommand{\definition}[1]{\textcolor{Red3}{\textbf{#1}}\index{#1}}
\newcommand{\blackdefinition}[1]{\textbf{#1}\index{#1}}
\newcommand{\definitionWithSpecificIndex}[3]{\textcolor{Red3}{\textbf{#1}}\index{#2}\index{#3}}
\newcommand{\example}[1]{\textcolor{Green4}{\textbf{#1}}}
\newcommand{\highspace}{\vspace{1.2em}\noindent}
\newcommand{\hqlabel}[2]{\label{#1}\hypertarget{#1}{#2}}
\newcommand{\hqpageref}[1]{\hyperlink{#1}{\hypergetpageref{#1}}}
\newcommand{\version}{v0.5.0}
% fontawesome5 macro:
% I never remember every icon, so I use custom commands to do this
\newcommand{\speedIcon}{tachometer-alt}

% \includeonly{%
%     sections/datacenter-layer-3-load-balancing/recap-datacenters,
%     sections/datacenter-layer-3-load-balancing/introduction,
%     sections/datacenter-layer-3-load-balancing/packet-spraying,
%     sections/datacenter-layer-3-load-balancing/equal-cost-multi-path-ecmp
% }

\begin{document}
    \newcounter{definition}[section]
    \newcounter{example}[section]
    \newcounter{takeaways}[section]
    
    \newtcolorbox[use counter = definition]{definitionbox}[1][]{%
        breakable,
        enhanced,
        colback=red!5!white,
        colframe=red!75!black,
        fonttitle=\bfseries,
        title={Definition \thetcbcounter#1} %
    }
    
    \newtcolorbox[use counter = example]{examplebox}[1][]{%
        breakable,
        enhanced,
        colback=Green4!5!white,
        colframe=Green4!75!black,
        fonttitle=\bfseries,
        title={Example \thetcbcounter#1} %
    }

    \newtcolorbox[]{remarkbox}[1][]{%
        breakable,
        enhanced,
        colback=DarkOrange3!5!white,
        colframe=DarkOrange3!75!black,
        fonttitle=\bfseries,
        title={Remark#1} %
    }

    \newtcolorbox[use counter = takeaways]{takeawaysbox}[1][]{%
        breakable,
        enhanced,
        colback=Green3!5!white,
        colframe=Green3!75!black,
        fonttitle=\bfseries,
        title={Key Takeaways#1} %
    }

    \newtcolorbox[]{deepeningbox}[1][]{%
        breakable,
        enhanced,
        colback=DarkOrange3!5!white,
        colframe=DarkOrange3!75!black,
        fonttitle=\bfseries,
        title={Deepening#1} %
    }

    %%%%%%%%%%%%%%%
    % Notes cover %
    %%%%%%%%%%%%%%%
    \author{260236}
\title{Quantum Computing - Notes - \version}
\date{\printdayoff\today}
\maketitle

    %%%%%%%%%%%
    % Preface %
    %%%%%%%%%%%
	\section*{Preface}

Every theory section in these notes has been taken from two sources:
\begin{itemize}
    \item Ian Goodfellow and Yoshua Bengio and Aaron Courville, \href{https://www.deeplearningbook.org/}{Deep Learning}, MIT Press. \cite{Goodfellow-et-al-2016}
    \item Course slides.\cite{course-slides-polimi}
\end{itemize}
About:
\begin{itemize}
    \item[\faIcon{github}] \href{https://github.com/PoliMI-HPC-E-notes-projects-AndreVale69/HPC-E-PoliMI-university-notes}{GitHub repository}
    \begin{center}
        \qrcode{https://github.com/PoliMI-HPC-E-notes-projects-AndreVale69/HPC-E-PoliMI-university-notes}
    \end{center}
\end{itemize}
These notes are an unofficial resource and shouldn't replace the course material or any other book on artificial neural networks and deep learning. It is not made for commercial purposes. I've made the following notes to help me improve my knowledge and maybe it can be helpful for everyone.

As I have highlighted, a student should choose the teacher's material or a book on the topic. These notes can only be a helpful material.

\highspace
During the Artificial Neural Networks and Deep Learning course, me and my other three colleagues Alberto Ondei, \href{https://it.linkedin.com/in/javedabdullah}{Abdullah Javed} and \href{https://www.hpc.cineca.it/staff/barbari-filippo/}{Filippo Barbari}, we created two projects:
\begin{itemize}
    \item[\faIcon{github}] \href{https://github.com/PoliMI-HPC-E-notes-projects-AndreVale69/AN2DL-Challenge-1}{Time Series Classification}. Deep Neural Netowrk (TCN $+$ BiLSTM with Attention) model on multivariate time-series classification.
    \begin{center}
        \qrcode{https://github.com/PoliMI-HPC-E-notes-projects-AndreVale69/AN2DL-Challenge-1}
    \end{center}
    \item[\faIcon{github}] \href{https://github.com/PoliMI-HPC-E-notes-projects-AndreVale69/AN2DL-Challenge-2}{Image Classification}. Histopathology image classification to predict molecular subtypes.
    \begin{center}
        \qrcode{https://github.com/PoliMI-HPC-E-notes-projects-AndreVale69/AN2DL-Challenge-2}
    \end{center}
\end{itemize}

    %%%%%%%%%%%%%%%%%%%%%
    % Table of contents %
    %%%%%%%%%%%%%%%%%%%%%
    \tableofcontents
    \newpage

    %%%%%%%%%%%%%%%%%%%
    % Fancy pagestyle %
    %%%%%%%%%%%%%%%%%%%
    \pagestyle{fancy}
    \fancyhead{} % clear all header fields
    \fancyhead[R]{\nouppercase{\leftmark\hfill\rightmark}}

    %%%%%%%%%%%%%%%
    % Datacenters %
    %%%%%%%%%%%%%%%
    \section{Datacenters}

\subsection{What is a Datacenter?}
A \definition{Datacenter} is a specialized \textbf{facility that houses multiple computing resources}, including servers, networking equipment, and storage systems. These \textbf{resources are co-located} (placed together in the same physical location) to ensure \hl{efficient operations}, \hl{leverage shared environmental controls} (such as cooling and power), and \hl{maintain physical security}.

\highspace
So the main characteristics are:
\begin{itemize}
    \item \textbf{Centralized Infrastructure}: Unlike traditional computing models where resources are scattered, datacenters consolidate thousands to millions of machines in a single administrative domain.
    \item \textbf{Full Control over Network and Endpoints}: Datacenters operate under a single administrative entity, \hl{allowing customized configurations} beyond conventional network standards.
    \item \textbf{Traffic Management}: Unlike the open Internet, datacenter traffic is highly structured, and the \hl{organization can define routing, congestion control, and network security policies}.
\end{itemize}

\begin{table}[!htp]
    \centering
    \begin{tabular}{@{} l p{12.8em} p{12.8em} @{}}
        \toprule
        \textbf{Feature} & \textbf{Datacenter Networks} & \textbf{Traditional Networks} \\
        \midrule
        \textbf{Ownership}  & Fully controlled by a single organization & Usually spans multiple independent ISPs \\ [.5em]
        \textbf{Traffic}    & High-speed internal communication (east-west traffic) & Lower-speed, external client-based traffic (north-south) \\ [.5em]
        \textbf{Routing}    & Customizable (non standard protocols) & Uses standard internet protocols (BGP, OSPF, etc.) \\ [.5em]
        \textbf{Latency}    & Optimized for ultra-low latency & Variable latency, dependent on ISPs \\ [.5em]
        \textbf{Redundancy} & High redundancy to ensure failover and fault tolerance & Often limited by ISP policies \\
        \bottomrule
    \end{tabular}
    \caption{Difference between Datacenters and other networks (e.g., LANs).}
\end{table}

\highspace
\begin{flushleft}
    \textcolor{Green3}{\faIcon{question-circle} \textbf{Why are datacenters important?}}
\end{flushleft}
Datacenters are the backbone of modern cloud computing, large-scale data processing, and AI/ML workloads. They provide \textbf{high computational power and storage} for various applications, such as:
\begin{enumerate}
    \item \important{Web Search \& Content Delivery}. For example, when a user searches for ``Albert Einstein'' on Google, the request is processed in a datacenter where:
    \begin{enumerate}
        \item The query is parsed and sent to multiple servers.
        \item Indexed data is retrieved.
        \item A ranked list of results is generated and sent back to the user.
    \end{enumerate}

    \item \important{Cloud Computing}. Services like Amazon Web Services (AWS), Microsoft Azure, and Google Cloud offer computation, storage, and networking resources on-demand.
    \begin{itemize}
        \item Infrastructure as a Service (IaaS): Virtual machines, storage, and networking.
        \item Platform as a Service (PaaS): Databases, development tools, AI models.
        \item Software as a Service (SaaS): Google Drive, Microsoft Office 365.
    \end{itemize}

    \item \important{AI and Big Data Processing}. Large-scale computations like MapReduce and deep learning training rely on distributed datacenter resources.

    \item \important{Enterprise Applications}. Datacenters host internal IT infrastructure for businesses, including databases, ERP systems, and virtual desktops.
\end{enumerate}

\highspace
\begin{flushleft}
    \textcolor{Green3}{\faIcon{history} \textbf{Evolution of Datacenters}}
\end{flushleft}
While the concept of centralized computing dates back to the 1960s, the modern datacenter model emerged with cloud computing in the 2000s. Notable developments include:
\begin{itemize}
    \item 1970s: IBM mainframes operated in controlled environments similar to early datacenters.
    \item 1990s: Rise of client-server computing required dedicated server rooms.
    \item 2000s-Present: Hyperscale datacenters by Google, Microsoft, and Amazon revolutionized networking, storage, and scalability.
\end{itemize}

\highspace
\begin{flushleft}
    \textcolor{Green3}{\faIcon{\speedIcon} \textbf{What's new in Datacenters?}}
\end{flushleft}
Datacenters have been around for decades, but modern datacenters have undergone significant changes in scale, architecture, and service models. The primary \textbf{factors driving these changes} include:
\begin{enumerate}[label=\textcolor{Green3}{\faIcon{check}}]
    \item The exponential \textbf{growth of internet services} (Google, Facebook, Amazon, etc.).
    \item The \textbf{shift to cloud computing} and on-demand services.
    \item The need for \textbf{better network scalability, fault tolerance, and efficiency}.
\end{enumerate}

\newpage

\noindent
One of the most striking changes in modern data centers is their massive scale:
\begin{itemize}
    \item Companies like Google, Microsoft, Amazon, and Facebook operate \textbf{datacenters with over a million servers at a single site}.
    \item \textbf{Microsoft alone has more than 100,000 switches and routers} in some of its datacenters.
    \item \textbf{Google processes billions of queries per day}, requiring vast computational resources.
    \item \textbf{Facebook and Instagram serve billions of active users}, with every interaction generating requests to datacenters.
\end{itemize}
Another major change is the \textbf{shift from owning dedicated computing infrastructure} to \textbf{renting scalable cloud resources}. Datacenters no longer just host enterprise applications, \textbf{they now offer computing, storage, and network infrastructure as a service}. The most common cloud computing models are:
\begin{itemize}
    \item \textbf{Infrastructure as a Service (IaaS)}. User rent virtual machines (VMs), storage, and networking instead of maintaining their own physical servers (e.g., Amazon EC2).
    \item \textbf{Platform as a Service (PaaS)}. Provides a platform with pre-configured environments for software development (databases, frameworks, etc.).
    \item \textbf{Software as a Service (SaaS)}. Full software applications hosted in datacenters and delivered via the internet (e.g., Google Drive).
\end{itemize}
The move to cloud computing has fundamentally changed datacenters, shifting the focus to resource allocation, security, and performance guarantees. They are also moving from multi-tenancy to single-tenancy:
\begin{itemize}
    \item \definition{Single-Tenancy}. A client gets \textbf{dedicated infrastructure} for their services.
    \item \definition{Multi-Tenancy}. \textbf{Resources are shared among multiple clients while ensuring isolation}.
\end{itemize}
\textcolor{Red2}{\faIcon{times-circle} \textbf{Implications}}. But this massive scale brings new challenges:
\begin{itemize}
    \item \important{Scalability}: The need for \textbf{efficient network designs} to handle rapid growth.
    
    Traditional datacenter topologies, such as three-based architectures, are inefficient at scale. New designs, like \textbf{Clos-based networks (Fat Tree)} and \textbf{Jellyfish (random graphs)}, are being developed to:
    \begin{itemize}[label=\textcolor{Green3}{\faIcon{check}}]
        \item Ensure \textbf{high bisection bandwidth} (allow any-to-any communication efficiently).
        \item Provide \textbf{scalable and fault-tolerant networking}.
    \end{itemize}
    
    \newpage

    \item \important{Cost management}: More machines mean \textbf{higher power, cooling, and hardware costs}.
    
    Datacenters are \textbf{expensive to build and maintain}, requiring:
    \begin{itemize}
        \item \textbf{Efficient resource utilization} (prevent idle servers from wasting power).
        \item \textbf{Energy-efficient cooling solutions} (cooling accounts for a \emph{huge} portion of operational costs).
        \item \textbf{Automation to reduce human intervention} (e.g., AI-based network optimization).
    \end{itemize}
    
    
    \item \important{Reliability}: Hardware failures become \textbf{common at scale}, requiring \textbf{automated fault-tolerant solutions}.
    
    At the scale of modern datacenters, \textbf{hardware and software failures are common}. A key principle is: ``\emph{In large-scale systems, failures are the norm rather than the exception.}'' (Microsoft, ACM SIGCOMM 2015).

    Thus, new \textbf{automated failover mechanisms} are required to:
    \begin{itemize}
        \item Detect failures \textbf{quickly}.
        \item Redirect traffic \textbf{seamlessly}.
        \item Ensure \textbf{minimal service disruption}.
    \end{itemize}

    \item \important{Performance \& Isolation Guarantees}: In modern datacenters, \textbf{customers expect strict performance guarantees} for applications like: low-latency financial transactions, high-bandwidth video streaming, machine learning model training.

    To meet these demands, datacenters implement:
    \begin{itemize}[label=\textcolor{Green3}{\faIcon{check}}]
        \item \textcolor{Green3}{\textbf{Performance Guarantees}}: Allocating bandwidth and compute power dynamically.
        \item \textcolor{Green3}{\textbf{Isolation Guarantees}}: Ensuring one user's workload does not interfere with another's.
    \end{itemize}

    But this requires \textbf{advanced networking techniques}, such as:
    \begin{itemize}
        \item \textbf{Traffic engineering} to avoid congestion.
        \item \textbf{Load balancing} to distribute workloads efficiently.
        \item \textbf{Software-defined networking (SDN)} for centralized control over traffic flows.
    \end{itemize}
\end{itemize}

\newpage

\begin{takeawaysbox}[: What is a Datacenter?]
    \begin{itemize}
        \item \textbf{Datacenters centralize} computing resources for performance, security, and scalability.
        \item \textbf{They differ from traditional networks} by offering more control, lower latency, and higher redundancy.
        \item \textbf{Applications include cloud services, AI, and enterprise computing}.
        \item \textbf{Scalability is a key challenge}, with hyperscale datacenters hosting millions of machines.
        \item \textbf{Efficiency and cost containment are major concerns}, requiring innovative architectures.
    \end{itemize}
\end{takeawaysbox}

    \subsection{Datacenter Applications}

Modern datacenters host a variety of applications that range from web services to large-scale data processing. These \textbf{applications can be classified based on their traffic patterns and computational needs}.  

\highspace
\begin{flushleft}
    \textcolor{Green3}{\faIcon{question-circle} \textbf{Customer-Facing Applications (North-South Traffic)}}
\end{flushleft}
Customer-facing applications involve direct interaction with users. This type of traffic follows a \definitionWithSpecificIndex{North-South communication model}{North-South traffic}{}, meaning that \textbf{data flows between external users and the datacenter}.

\begin{examplebox}[: North-South Traffic]
    Examples include:
    \begin{itemize}
        \item \textcolor{Green3}{\textbf{Web Search}} (e.g., Google, Bing)
        \begin{itemize}
            \item A user submits a query (e.g., ``Albert Einstein'').
            \item The request is routed through the datacenter's frontend servers.
            \item Backend database and indexing servers fetch relevant results.
            \item The response is assembled and sent back to the user.
        \end{itemize}
    
        \item \textcolor{Green3}{\textbf{Social Media Platforms}} (e.g., Facebook, Instagram, X (ex Twitter))
        \begin{itemize}
            \item Users interact with content hosted in the datacenter (e.g., loading a feed, liking posts).
            \item Each interaction requires queries to databases and caching systems.
            \item Content delivery is optimized using load balancers.
        \end{itemize}
    
        \item \textcolor{Green3}{\textbf{Cloud Services}} (e.g., Google Drive, Dropbox, OneDrive)
        \begin{itemize}
            \item Users upload, store, and retrieve files.
            \item Requests must be efficiently distributed across storage nodes.
        \end{itemize}
    \end{itemize}
\end{examplebox}

\highspace
\begin{flushleft}
    \textcolor{Green3}{\faIcon{question-circle} \textbf{Large-Scale Computation (East-West Traffic)}}
\end{flushleft}
Unlike customer-facing applications, backend computations do not involve direct interaction with external users. Instead, they focus on \textbf{processing massive datasets within the datacenter}. This type of traffic is known as \definition{East-West traffic} because it occurs \textbf{between servers inside the datacenter} \emph{rather than between the datacenter and the external world}.

\begin{examplebox}[: East-West Traffic]
    Examples include:
    \begin{itemize}
        \item \textcolor{Green3}{\textbf{Big Data Processing}} (e.g., MapReduce, Hadoop, Spark)
        \begin{itemize}
            \item Large datasets are distributed across multiple servers.
            \item Each server processes a portion of the data in parallel.
            \item Results are combined to generate insights (e.g., web indexing, analytics).
        \end{itemize}
        
        \item \textcolor{Green3}{\textbf{Machine Learning \& AI Training}} (e.g., Deep Learning Models)
        \begin{itemize}
            \item AI models are trained on massive datasets using clusters of GPUs/TPUs.
            \item The process requires high-bandwidth, low-latency communication.
            \item Synchronization between nodes is critical (e.g., gradient updates in distributed training).
        \end{itemize}

        \item \textcolor{Green3}{\textbf{Distributed Storage \& Backup Systems}} (e.g., Google File System, Amazon S3)
        \begin{itemize}
            \item Data is replicated across multiple locations for reliability.
            \item Servers frequently exchange data to ensure consistency and fault tolerance.
        \end{itemize}
    \end{itemize}
\end{examplebox}

\highspace
\begin{flushleft}
    \textcolor{Green3}{\faIcon{balance-scale} \textbf{Key differences between North-South and East-West traffic}}
\end{flushleft}
\begin{table}[!htp]
    \centering
    \begin{tabular}{@{} l l l @{}}
        \toprule
        \textbf{Feature} & \textbf{N-S traffic} & \textbf{E-W traffic} \\
        \midrule
        \textbf{Direction} & External users $\leftrightarrow$ Datacenter & Within datacenter \\ [.5em]
        \textbf{Examples} & File downloads & AI training \\ [.5em]
        \textbf{Bandwidth Needs} & Moderate & Very High \\ [.5em]
        \textbf{Latency Sensitivity} & High & Critical \\ [.5em]
        \textbf{Traffic Type} & Query-response & Bulk data transfer \\
        \bottomrule
    \end{tabular}
    \caption{Differences between North-South and East-West traffic.}
\end{table}
In terms of latency sensitivity, North-South traffic is high because user interactions must be fast. On the other hand, East-West traffic is critical because synchronization delays affect computation.

\newpage

\begin{flushleft}
    \textcolor{Red2}{\faIcon{book} \textbf{Traffic Patterns and Their Impact on Networking}}
\end{flushleft}
The way data moves within a datacenter \textbf{heavily influences network design}. The main \textbf{goal} is to \textbf{ensure high bandwidth, low latency, and efficient resource utilization}.
\begin{itemize}
    \item \important{Any-to-Any Communication Model}
    \begin{itemize}
        \item In \textbf{large-scale distributed applications}, \textbf{any server should be able to communicate with any other server at full bandwidth}.
        \item \textbf{Network congestion can severely degrade performance}, especially for AI/ML workloads and big data processing.
    \end{itemize}
    \item \important{High-Bandwidth Requirements}
    \begin{itemize}
        \item Applications like \textbf{MapReduce} and \textbf{deep learning} require \textbf{high data transfer rates}.
        \item If bandwidth is insufficient, \textbf{bottlenecks occur}, leading to delays.
    \end{itemize}
    \item \important{Latency is a Critical Factor}
    \begin{itemize}
        \item \textbf{Low-latency networking is essential} for interactive applications and distributed computing.
        \item AI training, for example, requires nodes to synchronize frequently; a delay in one node slows down the entire process.
    \end{itemize}
    \item \important{Worst-Case (Tail) Latency Matters}
    \begin{itemize}
        \item It's not enough for \textbf{most requests} to be fast; \textbf{the slowest request} can delay the entire computation.
        \item \textbf{Minimizing tail latency} is crucial for efficient AI model training and database queries.
    \end{itemize}
\end{itemize}

\highspace
\begin{flushleft}
    \textcolor{Green3}{\faIcon{exclamation-triangle} \textbf{Challenges in Datacenter Traffic Management}}
\end{flushleft}
The massive scale and complexity of modern datacenters introduce \textbf{several networking challenges}, including:  
\begin{itemize}
    \item \important{Network Congestion and Bottlenecks}. When multiple servers communicate simultaneously, \textbf{some network links become overloaded}, leading to congestion.

    For example, if many AI training jobs share the same network path, it can become a bottleneck, slowing down training.

    This can be a \textbf{critical issue for applications requiring real-time performance} (e.g., financial transactions, cloud gaming).


    \item \important{Load Balancing and Traffic Engineering}. How do we distribute traffic \emph{efficiently} across network links? The solutions are: \textbf{Equal-Cost Multipath Routing} (ECMP, \hl{spreads traffic across multiple paths}); \textbf{Dynamic Traffic Engineering} (\hl{adjusts paths in real time based on congestion levels}).


    \item \important{Avoiding Link Over-Subscription}. If too many servers send data over a single link, the available \textbf{bandwidth is divided}, leading to \textbf{slow performance}. Modern datacenters aim for \textbf{full-bisection bandwidth}, meaning \textbf{any server can talk to any other server at full capacity}.


    \item \important{Scaling Challenges}. Traditional datacenter network architectures do not scale well beyond a certain point. \textbf{New network topologies} (e.g., Fat Tree, Jellyfish) are being adopted to address these limitations.
\end{itemize}

\highspace
\begin{takeawaysbox}[: Datacenter Applications]
    \begin{itemize}
        \item Datacenters handle \textbf{two major types of applications}:
        \begin{enumerate}
            \item \textbf{Customer-facing applications (North-South traffic)} involve external users.
            \item \textbf{Large-scale computations (East-West traffic)} occur within the datacenter.
        \end{enumerate}  
        \item \textbf{Traffic patterns} affect \textbf{bandwidth, latency, and congestion control}.  
        \item \textbf{Managing congestion and ensuring high bandwidth} is critical for performance.  
        \item \textbf{New network topologies and routing techniques} help address scaling challenges.  
    \end{itemize}
\end{takeawaysbox}
    \subsection{Network Architecture}
The \textbf{primary goal} of a datacenter network is to \textbf{interconnect thousands to millions of servers} efficiently. Unlike traditional networks, which focus on wide-area communication, datacenter networks emphasize:
\begin{itemize}
    \item \important{High throughput}: Supporting massive data transfers.
    \item \important{Low latency}: Ensuring real-time performance for applications.
    \item \important{Scalability}: Accommodating rapid growth without performance degradation.
    \item \important{Fault tolerance}: Handling hardware failures with minimal disruption.
\end{itemize}
Datacenter \textbf{networks physically and logically connect servers through a multi-tiered architecture}. This hierarchical structure ensures that servers in different racks, pods, or clusters can communicate efficiently.

\highspace
\begin{flushleft}
    \textcolor{Green3}{\faIcon{book} \textbf{Traditional Three-Tier Datacenter Network}}
\end{flushleft}
Most datacenter networks follow a \definition{Three-Tier design}, which is optimized for scalability and efficiency. The three tiers are:
\begin{itemize}
    \item \definition{Edge Layer (Access Layer)}
    \begin{itemize}
        \item Located at the \textbf{bottom of the hierarchy}, closest to the servers.
        \item Consists of \textbf{Top-of-Rack (ToR) switches} that connect servers within a rack.
        \item \textcolor{Green3}{\textbf{Purpose}}: Aggregates traffic from multiple servers and forwards it to the higher layers.
        \item Typically uses \textbf{high-speed links (10-100 Gbps per port)} to connect servers.
    \end{itemize}


    \item \definition{Aggregation Layer (Distribution Layer)}
    \begin{itemize}
        \item Intermediate layer \textbf{between the edge and core layers}.
        \item \textbf{Connects multiple ToR switches} within a datacenter pod.
        \item \textcolor{Green3}{\textbf{Purpose}}: Helps distribute traffic efficiently \textbf{without overwhelming core routers}.
        \item Implements \textbf{load balancing, redundancy, and failover mechanisms}.
    \end{itemize}


    \item \definition{Core Layer (Backbone Layer)}
    \begin{itemize}
        \item The \textbf{top layer} of the hierarchy.
        \item Composed of \textbf{high-capacity, high-speed switches and routers}.
        \item \textcolor{Green3}{\textbf{Purpose}}: Responsible for:
        \begin{itemize}
            \item \textbf{Routing large volumes of traffic} between different aggregation switches.
            \item \textbf{Connecting the datacenter to external networks} (e.g., the Internet or private backbones).
        \end{itemize}
        \item Core switches often run \textbf{at 100 Gbps or higher per port} to support high aggregate bandwidth.
    \end{itemize}
\end{itemize}
Key \textbf{characteristics of the Three-Tier model}:
\begin{itemize}
    \item \important{Position}:
    \begin{itemize}
        \item \textbf{Edge Layer}: Closest to servers.
        \item \textbf{Aggregation Layer}: Intermediate between edge and core.
        \item \textbf{Core Layer}: Backbone layer.
    \end{itemize}
    \item \important{Primary Function}:
    \begin{itemize}
        \item \textbf{Edge Layer}: Connects servers within racks.
        \item \textbf{Aggregation Layer}: Aggregates ToR traffic.
        \item \textbf{Core Layer}: Routes traffic between datacenters or externally.
    \end{itemize}
    \item \important{Switch Type}:
    \begin{itemize}
        \item \textbf{Edge Layer}: Top-of-Rack (ToR).
        \item \textbf{Aggregation Layer}: Aggregation switches.
        \item \textbf{Core Layer}: Core routers.
    \end{itemize}
    \item \important{Speed (per port)}:
    \begin{itemize}
        \item \textbf{Edge Layer}: 10-100 Gbps.
        \item \textbf{Aggregation Layer}: 40-100 Gbps.
        \item \textbf{Core Layer}: 100 and more Gbps.
    \end{itemize}
    \item \important{Fault Tolerance}:
    \begin{itemize}
        \item \textbf{Edge Layer}: Redundant paths to aggregation layer.
        \item \textbf{Aggregation Layer}: Load balancing across core switches.
        \item \textbf{Core Layer}: High redundancy \& backup links.
    \end{itemize}
\end{itemize}

\highspace
\begin{flushleft}
    \textcolor{Red2}{\faIcon{exclamation-triangle} \textbf{Limitations of the Traditional Tree-Based Model}}
\end{flushleft}
Although widely used, the traditional three-tier model faces \textbf{scalability and performance challenges} as datacenters grow.
\begin{itemize}
    \item \textcolor{Red2}{\textbf{Scalability Issues}}. \textbf{Traditional networks are hierarchical}, meaning most communication \textbf{must pass through the core layer}. As datacenters scale, \textbf{core switches become bottlenecks} due to increased traffic.

    \item \textcolor{Red2}{\textbf{Bandwidth Bottlenecks}}. The model assumes that the \textbf{most traffic is North-South} (client to server). However, modern workloads involve \textbf{high East-West traffic} (server-to-server communication).
    
    \textbf{Over-subscription occurs} when the network cannot handle full-bisection bandwidth.

    \item \textcolor{Red2}{\textbf{Over-Subscription Problem}}. \definition{Over-Subscription} refers to \textbf{the ratio of worst-case achievable bandwidth to total bisection bandwidth}. For example:
    \begin{itemize}
        \item If 40 servers per rack each have a 10 Gbps link, total demand is 400 Gbps.
        \item If the uplink capacity to the aggregation layer is only 80 Gbps, we have a 5:1 over-subscription.
        \item This means only 20\% of the potential bandwidth is available, causing congestion.
    \end{itemize}
    Over-subscription ratios in large-scale networks can reach 50:1 or even 500:1, \textbf{severely limiting performance}.

    \item \textcolor{Red2}{\textbf{Performance Issues in High-Density Environments}}. High latency when traffic must \textbf{traverse multiple hops} to reach other racks. \textbf{Failures in core routers} can impact a large number of servers. \textbf{Inconsistent network performance} due to congestion in aggregation switches.
\end{itemize}

\highspace
\begin{flushleft}
    \textcolor{Green3}{\faIcon{check-circle} \textbf{Modern Datacenter Network Designs}}
\end{flushleft}
To overcome the \textbf{scalability and congestion challenges} of traditional tree-based networks, modern datacenters use alternative architectures.
\begin{itemize}[label=\textcolor{Green3}{\faIcon{check}}]
    \item \textcolor{Green3}{\textbf{Fat Tree (Clos Network)}}. Fat Tree is a \textbf{multi-stage switching architecture} designed to:
    \begin{itemize}
        \item \textbf{Ensure full-bisection bandwidth}: Every server can communicate at full capacity.
        \item \textbf{Provide multiple paths} between any two servers (high redundancy).
        \item \textbf{Balance traffic dynamically} to avoid congestion.
    \end{itemize}
    It uses K-ary fat tree topology where each pod consists of aggregation and edge switches, and core switches connect multiple pods. The advantages are:
    \begin{itemize}
        \item \textbf{Scalability}: Expands easily by adding more pods.
        \item \textbf{Fault Tolerance}: Multiple paths prevent failures from disrupting traffic.
        \item \textbf{Better Load Balancing}: Traffic is evenly distributed.
    \end{itemize}

    \item \textcolor{Green3}{\textbf{Jellyfish: Random Graph-Based Topology}}. Instead of a strict hierarchical structure, Jellyfish uses a \textbf{randomized topology}. The advantages are:
    \begin{itemize}
        \item \textbf{Higher network capacity} with \textbf{lower cost}.
        \item \textbf{More flexible scaling} than Fat Tree.
        \item \textbf{Better fault tolerance} since the network adapts dynamically.
    \end{itemize}
    
    \item \textcolor{Green3}{\textbf{BCube: Datacenter Network for Cloud Computing}}. Designed for high-performance cloud computing environments. It is optimized for: multi-path communication, resilience against failures and lowe latency compared to hierarchical models.
\end{itemize}

\highspace
\begin{takeawaysbox}[: Network Architecture]
    \begin{itemize}
        \item Traditional \textbf{three-tier datacenter networks} include \textbf{Edge, Aggregation, and Core layers}.
        \item \textbf{Core switches bottlenecks} as datacenters scale.
        \item \textbf{Over-subscription limits bandwidth}, causing congestion.
        \item \textbf{Modern topologies like Fat Tree and Jellyfish improve scalability, fault tolerance, and load balancing}.
    \end{itemize}
\end{takeawaysbox}
    \subsection{High and Full-Bisection Bandwidth}

\begin{flushleft}
    \textcolor{Green3}{\faIcon{question-circle} \textbf{Why is High-Bandwidth important in Datacenters?}}
\end{flushleft}
Modern datacenters \textbf{handle massive amounts of data} due to applications like AI training, cloud services, and big data processing. These workloads \emph{require}:
\begin{itemize}
    \item \textbf{High-bandwidth connections} to support fast data transfers.
    \item \textbf{Low latency} to ensure real-time performance.
    \item \textbf{Scalability} to accommodate increasing workloads.
\end{itemize}
Unlike traditional networks, where traffic primarily flows between users and servers (North-South), \textbf{datacenters experience heavy East-West traffic} (server-to-server communication). This shift \textbf{demands high-bandwidth and scalable network designs}.

\highspace
\begin{flushleft}
    \textcolor{Green3}{\faIcon{question-circle} \textbf{\emph{One step at a time}: What a Bisection Bandwidth is and why Full-Bisection Bandwidth is important}}
\end{flushleft}
\definition{Bisection Bandwidth} is a key metric that measures the \textbf{total bandwidth available between two halves of a network}.
\begin{definitionbox}[: Bisection Bandwidth]
    If a network is split into two equal halves, the \definition{Bisection Bandwidth} is the \textbf{total data transfer rate available between them}.
\end{definitionbox}
\begin{definitionbox}[: Full-Bisection Bandwidth]
    The \definition{Full-Bisection Bandwidth} is when every server can communicate with every other server at \textbf{full network speed}.
\end{definitionbox}

\noindent
In other words, bisection bandwidth can be thought of as cutting a data center network in half and measuring the total capacity of the links connecting the two halves. This tells us how much data can flow between the two sections simultaneously.

\begin{examplebox}[: Understand what bisection bandwidth is]
    Imagine a 1000-server datacenter, where 500 servers are processing data while 500 servers store the results. If the bisection bandwidth is \textbf{low}, the \textbf{data transfer between processing and storage nodes will be delayed}. This results in slow machine learning model training or delayed database queries.
\end{examplebox}

\noindent
As we can imagine, the full-bisection bandwidth is a real and critical aspect:
\begin{itemize}
    \item \important{Prevents bottlenecks}: Ensures high-throughput communication across racks and clusters.
    \item \important{Essential for AI/ML training}: AI models require massive parallel computations with continuous data exchanges.
    \item \important{Optimized for cloud computing}: Services like AWS, Google Cloud, and Azure depend on fast, reliable inter-server communication.
\end{itemize}

\highspace
\begin{flushleft}
    \textcolor{Red2}{\faIcon{exclamation-triangle} \textbf{Then try to get high-bandwidth all the time! Yes, but there are some challenges...}}
\end{flushleft}
Ideally, high-bandwidth should be the ultimate goal, but unfortunately, there are some problems with traditional three-based networks:
\begin{itemize}[label=\textcolor{Red2}{\faIcon{times}}]
    \item \textcolor{Red2}{\textbf{The Problem with Traditional Three-Based Networks}}. The standard \textbf{three-tier (core-aggregation-edge) topology} struggles to scale due to:
    \begin{enumerate}
        \item \textbf{Over-subscription} (definition on page \hqpageref{def: over-subscription problem}): The ratio of available bandwidth to required bandwidth is too high.
        \item \textbf{Core congestion}: Core routers become bottlenecks as traffic grows.
        \item \textbf{Single points of failure}: A failure in a core switch can affect a large portion of the datacenter.
    \end{enumerate}

    \item \textcolor{Red2}{\textbf{Over-Subscription and Its Impact on Network Performance}}. A naive solution would be to use over-subscription to solve these problems, but this limits performance. \definition{Over-Subscription} happens when the \textbf{network is provisioned with less bandwidth than needed} to cut costs.
    \begin{equation*}
        \text{Over-subscription} = \dfrac{\text{Total server bandwidth demand}}{\text{Available bandwidth at aggregation/core layer}}
    \end{equation*}
    Common over-subscription ratios are:
    \begin{itemize}
        \item 5:1, only 20\% of host bandwidth is available.
        \item 50:1, only 2\% of host bandwidth is available.
        \item 500:1, only 0.5\% of host bandwidth is available.
    \end{itemize}
    At 500:1 over subscription, congestion becomes severe, \textbf{limiting network efficiency}.
    
    \item \textcolor{Red2}{\textbf{The cost problem: scaling is expensive!}}
    \begin{itemize}
        \item Increasing bisection bandwidth requires \textbf{more high-performance network hardware}.
        \item \textbf{Scaling traditional networks} (adding more core switches) is extremely costly.
        \item \textbf{Energy consumption rises} with additional hardware.
    \end{itemize}
\end{itemize}
Thus, \textbf{alternative solutions} are needed to achieve high-bandwidth networking \textbf{without excessive costs}.

\newpage

\begin{flushleft}
    \textcolor{Green3}{\faIcon{check-circle} \textbf{Solutions to Achieve High and Full-Bisection Bandwidth}}
\end{flushleft}
To overcome these challenges, researchers and engineers have designed \textbf{new network architectures}.
\begin{itemize}[label=\textcolor{Green3}{\faIcon{check}}]
    \item \textcolor{Green3}{\textbf{Fat Tree (Clos Network) - The Scalable Solution}}. Unlike traditional three-based designs, Fat Tree provides \textbf{multiple paths} for traffic.
    
    \textcolor{Green3}{\faIcon{check-circle} \textbf{Advantages}}
    \begin{itemize}[label=\textcolor{Green3}{\faIcon{check}}]
        \item \textbf{Ensure full-bisection bandwidth} by allowing traffic to take alternative routes.
        \item \textbf{Eliminates single points of failure} using redundant paths.
        \item \textbf{Load balancing} optimizes network utilization.
    \end{itemize}


    \item \textcolor{Green3}{\textbf{Jellyfish - A More Flexible Approach}}. Uses a \textbf{randomized, non-hierarchical} topology instead of a fixed three structure.

    \textcolor{Green3}{\faIcon{check-circle} \textbf{Advantages}}
    \begin{itemize}[label=\textcolor{Green3}{\faIcon{check}}]
        \item \textbf{Better bandwidth scaling} as new servers are added.
        \item \textbf{More resilient to failures} (no single critical point of failure).
    \end{itemize}


    \item \textcolor{Green3}{\textbf{BCube - Optimized for Cloud Services}}. Designed for high-performance cloud environments with \textbf{massive inter-server communication}.
    
    \textcolor{Green3}{\faIcon{check-circle} \textbf{Advantages}}
    \begin{itemize}[label=\textcolor{Green3}{\faIcon{check}}]
        \item \textbf{Fast re-routing} in case of failures.
        \item \textbf{Low-latency communication for cloud applications}.
    \end{itemize}
\end{itemize}

\highspace
\begin{takeawaysbox}[: High and Full-Bisection Bandwidth]
    \begin{itemize}
        \item \textbf{High-bandwidth networking} is essential for modern datacenters.
        \item \textbf{Full-bisection bandwidth} ensures servers communicate at \textbf{full speed}.
        \item \textbf{Over-subscription} creates \textbf{bottlenecks}, limiting performance.
        \item \textbf{New network architectures} (Fat Tree, Jellyfish, BCube) solve scalability issues.
    \end{itemize}
\end{takeawaysbox}

    \subsection{Fat-Tree Network Architecture}

A \definition{Fat-Tree} is a \textbf{multi-layer, hierarchical network topology} that provides \emph{high scalability}, \emph{full-bisection bandwidth}, and \emph{fault tolerance}. It is a \textbf{special type of Clos Network}\footnote{%
    A \definition{Clos Network} is a type of multistage \textbf{switching topology that enables high-bandwidth and fault-tolerant communication by interconnecting multiple small switches instead of relying on a few large ones}. It is commonly used in datacenter networks (e.g., Google Jupiter Fabric) to maximize scalability and minimize congestion.
}, designed to \textbf{overcome bandwidth bottlenecks} in traditional three-based networks.

\highspace
The key idea is: Instead of a traditional tree where higher levels become bottlenecks, Fat-Tree ensures equal bandwidth at every layer by \textbf{increasing the number of links as we move higher in the hierarchy}.

\highspace
\begin{flushleft}
    \textcolor{Green3}{\faIcon{tools} \textbf{Structure of a K-Ary Fat-Tree}}
\end{flushleft}
A \textbf{K-ary Fat-Tree} consists of \textbf{three layers}:
\begin{enumerate}
    \item \important{Edge Layer (Top-of-Rack, ToR switches)}:
    \begin{itemize}
        \item Connects directly to the servers.
        \item Each edge switch connects $\frac{k}{2}$ servers and $\frac{k}{2}$ aggregation switches.
    \end{itemize}

    \item \important{Aggregation Layer}
    \begin{itemize}
        \item Connects multiple edge switches.
        \item Ensures \textbf{local traffic routing} between racks before sending to the core.
        \item Each aggregation switch connects $\frac{k}{2}$ edge switches and $\frac{k}{2}$ core\break switches.
    \end{itemize}

    \item \important{Core Layer}
    \begin{itemize}
        \item The backbone of the Fat-Tree, interconnecting multiple aggregation layers.
        \item Consists of $\left(\frac{k}{2}\right)^{2}$ core switches, where each connects to $k$ pods.
    \end{itemize}
\end{enumerate}  

\begin{examplebox}[: Fat-Tree with $k = 4$]
    \begin{itemize}
        \item Each pod contains:
        \begin{itemize}
            \item $\left(\frac{4}{2}\right)^{2} = 4$ servers.
            \item 2 layers of 2 2-port switches (Edge and Aggregation).
        \end{itemize}

        \item Each Edge Switch connects 2 servers and 2 aggregation switches.
        
        \item Each Aggregation Switch connects 2 Edge switches and 2 Core switches.

        \item The Core Layer consists of $\left(\frac{k}{2}\right)^{2} = 4$ core switches.
    \end{itemize}
    As a result, multiple paths between servers ensure no single point of failure and full-bisection bandwidth.
\end{examplebox}

\highspace
\begin{flushleft}
    \textcolor{Green3}{\faIcon{check-circle} \textbf{Why Use Fat-Tree in Datacenters?}}
\end{flushleft}
\begin{enumerate}[label=\textcolor{Green3}{\faIcon{check}}]
    \item \textcolor{Green3}{\textbf{Cost-Effective Scaling}}
    \begin{itemize}
        \item Can be built using \textbf{cheap, commodity switches} instead of expensive core routers.
        \item All switches operate at \textbf{uniform capacity}, simplifying hardware requirements.
    \end{itemize}
    \item \textcolor{Green3}{\textbf{Full-Bisection Bandwidth}}
    \begin{itemize}
        \item Each switch and server has \textbf{equal access to bandwidth}, preventing bottlenecks.
        \item Every packet has \textbf{multiple available paths}, ensuring \textbf{load balancing}.
    \end{itemize}
    \item \textcolor{Green3}{\textbf{High Fault Tolerance}}
    \begin{itemize}
        \item If one \textbf{switch or link fails}, traffic is rerouted through \textbf{alternative paths}.
        \item \textbf{No single point of failure}, unlike traditional three-based architectures.
    \end{itemize}
    \item \textcolor{Green3}{\textbf{Efficient Load Balancing}}
    \begin{itemize}
        \item \textbf{Multipath Routing} ensures traffic is evenly distributed.
        \item \textbf{No congestion at higher layers}, as each pod has equal bandwidth allocation.
    \end{itemize}
\end{enumerate}

\highspace
\begin{flushleft}
    \textcolor{Red2}{\faIcon{times-circle} \textbf{Problems in Fat-Tree Networks}}
\end{flushleft}
Fat-Tree is a highly scalable and efficient network topology, but \textbf{practical challenges exist} when handling real-world workloads.
\begin{itemize}
    \item \textcolor{Red2}{\textbf{Many flows running simultaneously}}. In large datacenters, multiple applications generate concurrent flows. Some flows are \textbf{small but latency-sensitive} (mice flows), while others are \textbf{large data transfers} (elephant flows). The Fat-Tree must \textbf{efficiently balance all these flows} across available paths.

    \item \textcolor{Red2}{\textbf{Traffic locality is unpredictable}}. Some services (e.g., Facebook/Meta workloads) have localized communication within a rack, while others require data exchange across the entire network. Fat-Tree must \textbf{dynamically adapt to different workload patterns}.

    \item \textcolor{Red2}{\textbf{Traffic is bursty}}. Some applications \textbf{generate sudden traffic spikes}, leading to temporary congestion. This is problematic for routing since \textbf{congestion-aware path selection is difficult}.

    \item \textcolor{Red2}{\textbf{Too Many Paths Between a Source and Destination}}. Unlike traditional network that have a single best route, Fat-Tree networks offer multiple equal-cost paths. \emph{Which path should be used?} Random selection might lead to congestion.

    \item \textcolor{Red2}{\textbf{Random Path Selection Leads to Collisions}}. If routing randomly assigns traffic flows, two large elephant flows may end up on the same link. This creates a congestion hotspot, even though other links remain underutilized.
    \begin{itemize}
        \item \textcolor{Green3}{\textbf{Ideal case}}: Traffic should be spread evenly across all available links.
        \item \textcolor{Red2}{\textbf{Reality}}: Without congestion awareness, routing \textbf{cannot react to traffic conditions dynamically}.
    \end{itemize}

    \item \textcolor{Red2}{\textbf{Short-Lived vs. Long-Lived Flows Create Conflicts}}. An \textbf{ideal routing scenario} would be to evenly distribute all flows. However, if a short, latency-sensitive flow suddenly appears on a congested link, its performance suffers. The key problem is that \textbf{Fat-Tree does not inherently prioritize latency-sensitive flows}.
\end{itemize}

\highspace
\begin{flushleft}
    \textcolor{Red2}{\faIcon{exclamation-triangle} \textbf{TCP Incast: A Major Issue in Fat-Tree Datacenters}}
\end{flushleft}
Large-scale parallel requests cause network congestion. In fact, some workloads (e.g., distributed storage systems, AI training) involve a \textbf{single client requesting data from multiple servers simultaneously}. This means that all servers respond at once, \textbf{overwhelming the switch's buffer capacity}. This results in \textbf{packet loss and retransmissions}, significantly increasing latency.

\begin{definitionbox}[: TCP Incast]
    \definition{TCP Incast} is a \textbf{network congestion issue} that occurs in datacenters when multiple servers send data to a single receiver simultaneously, overwhelming the switch's buffer capacity and causing severe packet loss and performance degradation.

    In other words, TCP Incast happens when many-to-one communication causes network congestion, leading to packet loss, TCP retransmissions, and increased latency.
\end{definitionbox}

\highspace
But in this scenario, how does \definition{TCP Incast} happen?
\begin{enumerate}
    \item A client application requests data from multiple storage servers.
    \item All storage servers respond \textbf{simultaneously}.
    \item The switch \textbf{cannot handle all packets at once}, causing \textbf{buffer overflow}.
    \item \textbf{Packet loss triggers TCP retransmissions}, further slowing down performance.
\end{enumerate}
This involves several issues:
\begin{itemize}
    \item Causes \textbf{severe latency spikes}, affecting (AI training and large-scale cloud) workloads.
    \item Traditional TCP \textbf{was not designed for this kind of bursty traffic}.
    \item \textbf{Fat-Tree cannot solve this issue alone}, it requires transport-layer optimizations.
\end{itemize}

\highspace
\begin{flushleft}
    \textcolor{Green3}{\faIcon{check-circle} \textbf{Google's Approach to Solving Fat-Tree Challenges}}
\end{flushleft}
Google faced severe scalability, congestion, and failure recovery challenges in its datacenters. Instead of using a traditional Fat-Tree model, they \textbf{developed a Clos-based architecture} known as \definition{Google Jupiter Fabric}. The key challenges that Google is addressing are:
\begin{itemize}
    \item \textbf{Scalability}. Traditional networks could not handle Google's exponential growth. Needed a network that scales gracefully by adding more capacity in stages.
    \item \textbf{Failure Tolerance}. A single failure should not impact traffic significantly. Needed path redundancy to ensure seamless operations.
    \item \textbf{Performance and Cost}. High-performance custom-built switching to support full-bisection bandwidth. Used commodity merchant silicon (off-to-shelf networking chips) instead of proprietary network devices, reducing costs.
\end{itemize}
The solutions adopted by Google are:
\begin{itemize}[label=\textcolor{Green3}{\faIcon{check}}]
    \item \textcolor{Green3}{\textbf{Clos Topology for Scalability \& Fault Tolerance}}. Google moved from traditional Fat-Tree to Clos networks to improve scalability.
    \begin{itemize}
        \item \textbf{Multiple layers of switches}, with multiple paths between every two endpoints.
        \item \textbf{Graceful fault recovery}: if one switch fails, traffic is rerouted dynamically.
        \item \textbf{Incremental scalability}: new switching stages can be added without network downtime.
    \end{itemize}
    A Clos network was chosen because, unlike Fat-Tree, which suffers from static oversubscription, \textbf{Clos networks offer more flexible bandwidth allocation}.

    Note that Fat-Tree inherits the scalability and fault tolerance of Clos, but its hierarchical and structured nature leads to congestion, routing complexity, and TCP Incast problems. Google recognized that Fat-Tree had structural limitations, so they modified Clos into the Jupiter Fabric.

    \item \textcolor{Green3}{\textbf{Custom Hardware: Merchant Silicon Instead of Proprietary\break Switches}}. Google avoided vendor lock-in by using commodity hardware (merchant silicon). The reasons are:
    \begin{itemize}
        \item \textbf{Lower cost} than custom ASIC-based routers.
        \item \textbf{Faster} hardware \textbf{upgrade cycles}.
        \item \textbf{More control} over network design and software stack.
    \end{itemize}


    \item \textcolor{Green3}{\textbf{Centralized Control for Routing and Network Management}}. In traditional datacenters, routing is distributed, meaning each switch makes independent routing decisions. This approach does not scale well in Clos networks with thousands of switches.

    The solution is \textbf{precomputed routing decisions}. Instead of switches making their own decisions, Google precomputes traffic flows centrally and pushes them to switches.

    \textcolor{Green3}{\faIcon{check-circle} \textbf{Advantages}}
    \begin{itemize}[label=\textcolor{Green3}{\faIcon{check}}]
        \item \textcolor{Green3}{\textbf{Improves traffic engineering}}: Load balancing decisions are optimized globally rather than per switch.
        \item \textcolor{Green3}{\textbf{More predictable performance}}.
        \item \textcolor{Green3}{\textbf{Less congestion}}: Can react dynamically to network failures.
    \end{itemize}
\end{itemize}

\begin{takeawaysbox}[: Fat-Tree Network Architecture]
    \begin{itemize}
        \item \textbf{Fat-Tree is a special type of Clos Network} that overcomes bottlenecks in traditional tree networks.  
        \item \textbf{K-ary} Fat-Tree has three layers (Edge, Aggregation, Core), \textbf{ensuring equal bandwidth for all nodes}.  
        \item \textbf{Fat-Tree} provides multiple paths, but \textbf{routing is difficult due to unpredictable traffic patterns}.  
        \item \textbf{Collisions between large flows create network hotspots}.  
        \item \textbf{TCP Incast is a major issue}, where too many responses at once cause packet loss.
        \item Google's Datacenter Network Strategy:
        \begin{itemize}
            \item \textbf{Moved from Fat-Tree to Clos topology} for better scalability and failure recovery.
            \item \textbf{Used merchant silicon instead of proprietary hardware} to cut costs and improve flexibility.
            \item \textbf{Implemented centralized control for routing} to optimize traffic flows.
            \item \textbf{Designed the Jupiter Fabric} to handle \textbf{Google-scale workloads} with incremental scalability.
        \end{itemize}
    \end{itemize}
\end{takeawaysbox}


    %%%%%%%%%%%%%%%%%%%%%%%%%%%%%%%
    % Software Defined Networking %
    %%%%%%%%%%%%%%%%%%%%%%%%%%%%%%%
    \section{VLIW (Very Long Instruction Word)}

\subsection{Introduction}

Traditional \textbf{compilers} use \textbf{static code scheduling} to exploit Instruction-Level Parallelism (ILP). Key tasks of the compiler:
\begin{itemize}
    \item  Detect \textbf{parallelizable} instructions considering: Hardware resource constraints and Data dependencies.
    \item \textbf{Schedule} instructions to execute \textbf{in parallel} when possible.
    \item Otherwise, \textbf{insert \texttt{NOP}s} (No Operations) if no safe parallel execution is possible.
\end{itemize}
Statically scheduled processors trust the compiler to ``fill the pipeline'' and avoid hazards at compile time.

\highspace
\begin{flushleft}
    \textcolor{Green3}{\faIcon{book} \textbf{VLIW Processors: Alternative Way to Extract ILP}}
\end{flushleft}
\definition{VLIW (Very Long Instruction Word)} \textbf{processors} are a class of architectures designed to \textbf{execute multiple operations in parallel during a single clock cycle}, but unlike superscalar processors, they \textbf{rely on the compiler} rather than on dynamic hardware mechanisms \textbf{to detect and schedule parallelism}.

\highspace
The fundamental idea behind VLIW is to \textbf{group several independent operations together into a single long instruction word}, called a \hl{bundle}. This bundle is \textbf{composed of several fixed slots}, each one \textbf{corresponding to a different functional unit in the processor}, such as an integer ALU, a floating-point unit, a load/store unit, or a branch unit.

\highspace
For example, in a 5-issue VLIW processor, a single bundle would typically carry up to five operations: integer, floating-point, load/store, branch, etc.

\highspace
The \textbf{key difference} compared \textbf{to traditional superscalar} processors lies in who decides what can run in parallel. In superscalar designs, the hardware dynamically analyzes dependencies between instructions at runtime, which requires complex circuitry for hazard detection and scheduling. In VLIW architectures, this \hl{analysis is performed entirely at compile time}. The \textbf{compiler is responsible for}:
\begin{itemize}
    \item \textbf{Statically identify} independent operations.
    \item \textbf{Solve structural hazards} (e.g., two operations trying to use the same hardware unit).
    \item \textbf{Solve data hazards} (dependencies between instructions).
    \item Insert \textbf{NOPs} when necessary.
\end{itemize}
When no useful instruction is available for a particular slot, the compiler inserts a \texttt{NOP} (no-operation). In cases where conflicts cannot be avoided, NOPs are inserted into the bundle to fill empty slots.

\highspace
\begin{flushleft}
    \textcolor{Green3}{\faIcon{question-circle} \textbf{Why move to Compiler?}}
\end{flushleft}
The problem with Superscalar is that the hardware requires complex dynamic scheduling and dependency checking, which \textbf{costs area and power}.

\highspace
\textbf{VLIW Idea} is:
\begin{itemize}
    \item Push \textbf{complexity} to the \textbf{compiler}.
    \item \textbf{Compiler groups parallel operations} into a \textbf{single bundle}.
    \item No need for runtime dependency checking anymore.
\end{itemize}
The VLIW paradigm is characterized by the use of \textbf{very wide instruction words}, \textbf{each containing multiple independent operations} (``syllables''). A multiple-issue VLIW processor typically features specialized units for integer, floating-point, memory access, and branch instructions. For example, a 4-issue VLIW processor will have four operation slots per bundle, each connected to a different unit.

\highspace
\begin{flushleft}
    \textcolor{Green3}{\faIcon{list-ul} \textbf{Multiple-issue VLIW: Operation Latencies}}
\end{flushleft}
An important aspect in VLIW scheduling is the \textbf{management of operation latencies}. Each operation type may require a \textbf{different number of clock cycles to complete}. Integer operations, memory accesses, and floating-point computations may all have varying latencies, and these differences must be carefully considered during scheduling. The \textbf{compiler must plan the execution} so that operations complete in the correct order and without causing unnecessary stalls in the pipeline.

\highspace
In summary, \hl{VLIW architectures represent a shift of complexity from the hardware to the compiler}, aiming for more efficient and simpler processor designs while demanding sophisticated compiler techniques to fully exploit available parallelism.

\highspace
\begin{flushleft}
    \textcolor{Green3}{\faIcon{book} \textbf{Single Program Counter and Branch Management}}
\end{flushleft}
In a VLIW processor, even though multiple operations are issued simultaneously, the architecture still uses a \textbf{single program counter} to fetch instructions. Each instruction fetched corresponds to a \textbf{bundle}, which can contain multiple parallel operations.

\highspace
Importantly, within each bundle, there can be \textbf{at most one branch instruction} that affects control flow. This constraint simplifies the handling of branches, making it easier for the processor to predict and manage the program's execution path.

\newpage

\begin{flushleft}
    \textcolor{Green3}{\faIcon{book} \textbf{Shared Multi-Ported Register File}}
\end{flushleft}
Since a VLIW processor issues multiple operations in parallel, the \textbf{register file must support multiple simultaneous reads and writes}. In a typical 4-issue VLIW machine, the register file needs to have enough ports to read \textbf{eight source operands} and write \textbf{four destination results} every clock cycle.

\highspace
This requirement implies a \textbf{shared, multi-ported register file}, which is one of the non-trivial aspects of the hardware design. It ensures that all functional units can access their operands without creating bottlenecks.

\highspace
\begin{flushleft}
    \textcolor{Green3}{\faIcon{book} \textbf{Importance of Parallelism and the Use of NOPs}}
\end{flushleft}
To achieve high performance, the \textbf{compiler must ensure that the source code has enough parallelism} to fill all the available operation slots in each bundle. When \textbf{sufficient independent instructions are not available}, \textbf{NOPs are inserted} into the empty slots. This insertion preserves the fixed structure of the bundle but can lead to inefficiencies if the application does not expose enough ILP.

\highspace
\begin{flushleft}
    \textcolor{Green3}{\faIcon{book} \textbf{Pipeline Organization}}
\end{flushleft}
VLIW processors often use a \textbf{pipelined execution model} similar to traditional RISC architectures. A standard pipeline might include stages like \textbf{Instruction Fetch (IF)}, \textbf{Instruction Decode (ID)}, \textbf{Register Read (RR)}, \textbf{Execution (EX)}, and \textbf{Write Back (WB)}.

\begin{figure}[!htp]
    \centering
    \includegraphics[width=\textwidth]{img/vliw-arch.pdf}
\end{figure}

\highspace
In the previous figure, a \textbf{5-stage pipeline} is used, with each bundle passing through these stages. Each operation within a bundle proceeds independently through the functional units connected to the pipeline.

\newpage

\begin{flushleft}
    \textcolor{Green3}{\faIcon{book} \textbf{Operation Dispatch and Decode}}
\end{flushleft}
If the architecture dedicates \textbf{one functional unit per slot}, the decode stage becomes relatively simple. Each operation is directed straight to its corresponding functional unit for execution without the need for complex arbitration.

\highspace
However, if there are \textbf{more functional units than slots}, meaning there is a surplus of parallel hardware resources, the architecture must include a \textbf{dispatch network}. This network is responsible for forwarding each operation, along with its operands, to the appropriate available functional unit. In this way, the design of the dispatch system depends heavily on the organization of functional units relative to the number of issue slots defined by the instruction bundle format.

    \subsection{Legacy Router \& Switch Architecture}

Legacy network devices, such as routers and switches, are \textbf{built with integrated control and data planes}, meaning \hl{each device independently makes forwarding decisions}. These devices consist of:
\begin{enumerate}
    \item \important{Hardware Components}
    \begin{itemize}
        \item \textbf{Application-Specific Integrated Circuits (ASICs)}, specialized chips for packet forwarding.
        \item \textbf{Memory (buffers \& TCAMs)}, stores forwarding tables and processing queues.
        \item \textbf{Network Interfaces (NICs, Ports)}, physical ports for connecting network cables.
    \end{itemize}
    \item \important{Software Components}
    \begin{itemize}
        \item \textbf{Router OS (Operating System)}, runs network protocols and management interfaces.
        \item \textbf{Routing Protocols (OSPF, BGP, RIP)}, determines paths for packet forwarding.
        \item \textbf{Forwarding Table}, maps destination addresses to outgoing ports.
    \end{itemize}
    \item \important{Management and Control Interfaces}
    \begin{itemize}
        \item \textbf{Command-Line Interface (CLI)}, used for configuring routers\break manually.
        \item \textbf{SNMP (Simple Network Management Protocol)}, enables\break monitoring and automation.
    \end{itemize}
\end{enumerate}

\begin{figure}[!htp]
    \centering
    \includegraphics[width=.7\textwidth]{img/router-switch-arc.pdf}
    \caption{Legacy router and switch architecture.}
\end{figure}

\noindent
Traditional network devices have two primary operational planes:
\begin{itemize}
    \item \textbf{Control Plane}: makes forwarding decisions based on routing protocols.
    \item \textbf{Data Plane}: physically forwards packets based on control plane decisions. For example MAC lookup and IP forwarding.
\end{itemize}
\hl{Each router operates autonomously}, \hl{using routing tables} built through protocols like OSPF and BGP. These protocols dynamically learn network paths and update the forwarding tables, ensuring efficient packet delivery.

\highspace
\begin{flushleft}
    \textcolor{Green3}{\faIcon{list-ol} \textbf{Packet Processing in a Legacy Router}}
\end{flushleft}
\begin{enumerate}
    \item \important{Lookup Destination IP} $\rightarrow$ Find matching entry in the forwarding table.
    \item \important{Update Header} $\rightarrow$ Modify packet headers if needed (e.g., TTL decrement).
    \item \important{Queue Packet} $\rightarrow$ Send packet to the appropriate output interface.
\end{enumerate}
\textcolor{Red2}{\faIcon{times-circle}} Since \textcolor{Red2}{\textbf{every device handles its own control and forwarding}}, \textbf{large-scale changes require individual device updates}, making traditional networking \hl{complex and inflexible}.

\highspace
\begin{flushleft}
    \textcolor{Red2}{\faIcon{times-circle} \textbf{Challenges in Traditional Network Management}}
\end{flushleft}
\begin{enumerate}[label=\textcolor{Red2}{\faIcon{times}}]
    \item \textcolor{Red2}{\textbf{Complex Configuration \& Management}}. \textbf{Each network device} has to be \textbf{configured individually}. Protocols like BGP and OSPF require manual tuning for optimal performance. Network engineers must interact with vendor-specific CLIs, which vary by manufacturer.
    
    \item \textcolor{Red2}{\textbf{Limited Innovation \& Vendor Lock-In}}. New network \textbf{features} require \textbf{firmware or software updates from vendors}. Custom networking solutions are \textbf{difficult to implement due to proprietary hardware and software}.
    
    \item \textcolor{Red2}{\textbf{Slow Response to Failures \& Traffic Changes}}. Routing adjustments depend on distributed algorithms that can take seconds to minutes to converge. Manual troubleshooting is often needed when failures occur.
    
    \item \textcolor{Red2}{\textbf{Scalability Issues}}. Growing networks require more hardware and manual configurations. \textbf{Updating} policies across multiple routers is \textbf{time-consuming and error-prone}.
\end{enumerate}

\highspace
\begin{takeawaysbox}[: Legacy Router and switch architecture]
    \begin{itemize}
        \item Legacy networking relies on \textbf{autonomous devices} with tightly integrated \textbf{control and data planes}.
        \item Routing is handled by protocols like OSPF and BGP, which operate \textbf{independently on each device}.
        \item Challenges include manual configuration, vendor lock-in, slow failure response, and scalability issues.
        \item These limitations paved the way for SDN, which offers centralized, programmable networking.
    \end{itemize}
\end{takeawaysbox}

    \subsection{SDN Architecture}

The \textbf{core} concept of \definition{Software-Defined Networking (SDN)} is the \textbf{separation} of the \textbf{control plane} from the \textbf{data plane}:
\begin{itemize}
    \item In SDN, \textbf{network devices} (switches/routers) become \textbf{simple forwarding elements}, executing decisions made by a centralized \textbf{controller}.
    \item The \definition{SDN Controller} is a \textbf{software-based system} that \hl{manages, programs, and monitors the entire network}.
\end{itemize}
Key architecture:
\begin{itemize}
    \item \textbf{Data Plane (Forwarding Engine)} $\rightarrow$ Located on switches; handles packet forwarding.
    \item \textbf{Control Plane (SDN Controller)} $\rightarrow$ Runs on external servers; computes forwarding rules.
    \item \textbf{Communication Channel} $\rightarrow$ Allows the controller to instruct the data plane; typically uses OpenFlow.
\end{itemize}
But \emph{why decouple?} Enables centralized decision-making, consistent policy enforcement, and simplified management. Facilitates dynamic updates to the network without hardware changes.

\highspace
\begin{flushleft}
    \textcolor{Green3}{\faIcon{book} \textbf{The Role o the SDN Controller}}
\end{flushleft}
The SDN Controller is the \textbf{central brain} of the network. It performs:
\begin{enumerate}[label=\textcolor{Green3}{\faIcon{check}}]
    \item \textcolor{Green3}{\textbf{Network State Monitoring}}: Gathers real-time information from all forwarding devices.
    \item \textcolor{Green3}{\textbf{Decision-Making}}: Calculates the best routes, applies policies, and enforces security.
    \item \textcolor{Green3}{\textbf{Rule Installation}}: Pushes flow rules to switches, determining how packets should be handled.
\end{enumerate}
Controllers provide a \textbf{global, up-to-date view of the entire network}, enabling smarter control than traditional distributed routing.

\highspace
\begin{flushleft}
    \textcolor{Green3}{\faIcon{code} \textbf{Communication Interfaces}}
\end{flushleft}
SDN uses two types of APIs to manage communication between layers:
\begin{itemize}
    \item \important{Southbound Interface}: Connects the controller to the data plane devices (e.g., \textbf{OpenFlow protocol}). It instructs switches via OpenFlow or similar protocols, installing/removing flow rules and collecting stats.
    \item \important{Northbound Interface}: Allows \textbf{applications to interact with the controller} via APIs (e.g., REST APIs). Applications (e.g., security monitoring, load balancing) query and command the controller to implement network policies.
\end{itemize}

\highspace
\begin{flushleft}
    \textcolor{Green3}{\faIcon{book} \textbf{Network Operating System (Network OS)}}
\end{flushleft}
The controller runs a Network OS, providing:
\begin{itemize}
    \item \textbf{Abstractions} over the physical network (e.g., topology view, link status).
    \item \textbf{Programmatic Interfaces} for developing control programs.
    \item \textbf{Consistency \& Global View}: All decisions are made based on coherent, synchronized data.
\end{itemize}
The Network OS simplifies the task of writing network control logic by exposing standardized APIs.

\begin{figure}[!htp]
    \centering
    \includegraphics[width=\textwidth]{img/sdn-arch.pdf}
\end{figure}

\highspace
\begin{takeawaysbox}[: SDN Architecture]
    \begin{itemize}
        \item \textbf{Traditional networking} embeds the control plane within each device; SDN \textbf{centralizes control} in software.
        \item The \textbf{SDN Controller} dynamically manages the \textbf{data plane devices} using a \textbf{communication protocol}.
        \item \textbf{OpenFlow} is the primary protocol used to communicate between the controller and switches.
        \item \textbf{Network OS} provides an \textbf{abstraction layer} and \textbf{programming environment} for writing control logic.
    \end{itemize}
\end{takeawaysbox}


    \subsection{OpenFlow}

\definition{OpenFlow} is the \textbf{first and most widely adopted protocol} used in Software-Defined Networking (SDN) to \textbf{enable communication} between the \textbf{SDN Controller} (control plane) and the \textbf{data plane devices} (e.g., switches, routers, they are the forwarding engine). It allows the controller to \textbf{program flow tables} in the switches and \textbf{control how packets are forwarded}, enabling centralized management of traffic.

\highspace
In other words, OpenFlow is the practical implementation of SDN, standardizing how controllers manage packet forwarding.

\highspace
\begin{flushleft}
    \textcolor{Green3}{\faIcon{tools} \textbf{How OpenFlow works}}
\end{flushleft}
Each \textbf{OpenFlow switch} contains:
\begin{enumerate}
    \item \textbf{Flow Table}: Contains rules in the form of $Match \rightarrow Action$ pairs.
    \item \textbf{Communication Interface}: Connects to the SDN controller via the OpenFlow protocol.
    \item \textbf{Stats Module}: Collects statistics about packet flows.
\end{enumerate}

\highspace
\begin{examplebox}[: Flow Rules]
    \begin{enumerate}
        \item If $Header = p \Rightarrow$ send to port 5.
        \item If $Header = q \Rightarrow$ modify header to $r$, then send to ports 6 and 7.
        \item If $Header = p \Rightarrow$ send packet to the controller.
    \end{enumerate}
\end{examplebox}

\highspace
\textbf{Flow table operation}:
\begin{enumerate}
    \item Packet arrives at the switch.
    \item Switch checks for a \textbf{matching rule} in its flow table.
    \item If matched $\rightarrow$ \textbf{apply action} (e.g., forward, modify, drop).
    \item If no match $\rightarrow$ \textbf{send packet header to controller} for instructions.
\end{enumerate}

\highspace
\begin{examplebox}[: OpenFlow]
    \begin{enumerate}
        \item New packet arrives at switch.
        \item Match?
        \begin{itemize}
            \item Yes $\rightarrow$ forward according to rule.
            \item No $\rightarrow$ forward header to controller.
        \end{itemize}
        \item Controller analyzes packet and installs a new rule in switch.
        \item Next packets of same type $\rightarrow$ directly processed by switch using newly installed rule.
    \end{enumerate}
\end{examplebox}

\begin{figure}[!htp]
    \centering
    \includegraphics[width=.6\textwidth]{img/openflow.pdf}
\end{figure}

\highspace
\begin{flushleft}
    \textcolor{Green3}{\faIcon{list-ul} \textbf{Actions in OpenFlow}}
\end{flushleft}
OpenFlow supports many types of actions, such as:
\begin{itemize}
    \item \textbf{Forwarding} to one or multiple ports.
    \item \textbf{Dropping packets}.
    \item \textbf{Modifying headers} (e.g., VLAN tags, IP addresses).
    \item \textbf{Sending packets to controller}.
    \item \textbf{Statistics collection} for flow monitoring.
\end{itemize}
This \textbf{flexibility} allows SDN to implement advanced functions like load balancing, traffic shaping, and security filtering without specialized hardware.

\highspace
\begin{flushleft}
    \textcolor{Green3}{\faIcon{balance-scale} \textbf{Reactive vs. Proactive Flow Rules}}
\end{flushleft}
In OpenFlow, the \textbf{controller installs flow rules} into switches to determine how packets are processed. There are \textbf{two main modes} of operation for rule installation: Reactive and Proactive. These \textbf{modes define when and how flow entries are populated in the flow tables} of switches.
\begin{itemize}
    \item \definition{Reactive Mode}
    \begin{itemize}
        \item[\textcolor{Green3}{\faIcon{question-circle}}] \textcolor{Green3}{\textbf{How it works?}}
        \begin{enumerate}
            \item When a \textbf{new flow} (a new type of packet) arrives at a switch, and \textbf{no rule matches} it, the \textbf{packet header is sent to the controller}.
            \item The \textbf{controller analyzes the packet} and \textbf{decides what rule} should be installed in the switch to handle it.
            \item After the controller sends back a rule, the switch installs it and \textbf{forwards the packet} accordingly.
        \end{enumerate}
        \item[\textcolor{Green3}{\faIcon{list-ul}}] \textcolor{Green3}{\textbf{Key Characteristics}}
        \begin{itemize}
            \item \textbf{Efficient use of flow tables} - Only rules for \hl{active flows} are installed.
            \item \textbf{Every new flow} incurs \textbf{small setup time} (controller interaction delay).
            \item \textbf{Switch depends on the controller} for flow rule installation.
            \item If the \textbf{controller connection is lost}, the switch has \textbf{limited utility} for new flows.
        \end{itemize}
        \item[\textcolor{Green3}{\faIcon{check-circle}}] \textcolor{Green3}{\textbf{Advantages}}
        \begin{itemize}[label=\textcolor{Green3}{\faIcon{check}}]
            \item \textbf{Dynamic} and \textbf{adaptive} to real-time network traffic.
            \item \textbf{Minimizes unused flow entries}.
        \end{itemize}
        \item[\textcolor{Red2}{\faIcon{times-circle}}] \textcolor{Red2}{\textbf{Disadvantages}}
        \begin{itemize}[label=\textcolor{Red2}{\faIcon{times}}]
            \item Adds \textbf{latency} for the \textbf{first packet} of each flow.
            \item \textbf{High control plane load} in environments with many short flows.
        \end{itemize}
    \end{itemize}

    \item \definition{Proactive Mode}
    \begin{itemize}
        \item[\textcolor{Green3}{\faIcon{question-circle}}] \textcolor{Green3}{\textbf{How it works?}}
        \begin{enumerate}
            \item The \textbf{controller pre-installs flow rules} in the switches before any packet arrives.
            \item The switch \textbf{immediately processes packets} using pre-defined rules without contacting the controller.
        \end{enumerate}
        \item[\textcolor{Green3}{\faIcon{list-ul}}] \textcolor{Green3}{\textbf{Key Characteristics}}
        \begin{itemize}
            \item \textbf{Zero setup delay} for packet processing - packets are forwarded \textbf{immediately}.
            \item Requires \textbf{aggregated or wildcard rules} to efficiently use flow table space.
            \item \textbf{Independent of controller connectivity} - continues to operate even if controller is unreachable.
        \end{itemize}
        \item[\textcolor{Green3}{\faIcon{check-circle}}] \textcolor{Green3}{\textbf{Advantages}}
        \begin{itemize}[label=\textcolor{Green3}{\faIcon{check}}]
            \item \textbf{Fast packet forwarding} with \textbf{no initial delay}.
            \item \textbf{No dependency} on controller for flow rule installation during packet arrival.
            \item Ideal for \textbf{predictable traffic patterns} or \textbf{mission-critical environments}.
        \end{itemize}
        \item[\textcolor{Red2}{\faIcon{times-circle}}] \textcolor{Red2}{\textbf{Disadvantages}}
        \begin{itemize}[label=\textcolor{Red2}{\faIcon{times}}]
            \item Can \textbf{waste flow table space} if many pre-installed rules are unused.
            \item Requires \textbf{good planning} of rules; less flexible to dynamic traffic changes.
        \end{itemize}
    \end{itemize}
\end{itemize}
In summary, reactive mode is adaptive, but introduces latency and higher controller load; proactive mode is fast and resilient, but requires advance planning of rules.

\highspace
\begin{takeawaysbox}
    \begin{itemize}
        \item \textbf{OpenFlow} enables the SDN controller to manage \textbf{flow tables} in switches.
        \item \textbf{Flow rules} define how packets are handled, allowing \textbf{centralized, programmable networking}.
        \item OpenFlow supports \textbf{fine-grained traffic control} via a wide range of \textbf{match/action rules}.
        \item Two operation modes:
        \begin{itemize}
            \item Reactive (dynamic but with latency).
            \item Proactive (fast but needs good planning).
        \end{itemize}
    \end{itemize}
\end{takeawaysbox}
    \subsection{OpenFlow limitations}

OpenFlow, conceived around 2007, introduced centralized control by standardizing how switches expose forwarding behavior to an SDN controller (as we discussed in the previous section, page \pageref{subsection: OpenFlow}). The \textbf{insight} at that time was that \textbf{most switches perform similar tasks} (Ethernet switching, IPv4 routing, VLAN tagging, ACL enforcement) \textbf{all via fixed}, \textbf{predictable behaviors}.

\highspace
OpenFlow capitalized on this fixed-function approach. Controllers could install flow rules into switches, dictating how they process known packet headers. However, a \textcolor{Red2}{\textbf{critical limitation}} emerged: \textbf{we couldn't add new protocols or processing capabilities easily}. Because \textbf{OpenFlow assumes a static data plane}, \textcolor{Red2}{\textbf{hardcoded to process only a predefined set of protocols and headers}}.

\highspace
\begin{flushleft}
    \textcolor{Green3}{\faIcon{check-circle} \textbf{Expanding OpenFlow: pushing its limits}}
\end{flushleft}
As networking needs evolved, particularly in \textbf{virtualized environments} and \textbf{cloud datacenters}, operators needed more \textbf{specialized packet processing}. For example, \texttt{VXLAN}, used to identify tenants in multi-tenant environments, wasn't supported in early OpenFlow.

\highspace
\textcolor{Green3}{\faIcon{check}} To address this, vendors and the OpenFlow community developed \textbf{new versions} (\texttt{1.1}, \texttt{1.2}, \texttt{1.3}, $\dots$). Each iteration \textbf{added support for more header types}, up to 50 different header types, but \textcolor{Red2}{\textbf{the process was slow and cumbersome}}. Each new feature needed:
\begin{itemize}
    \item New OpenFlow specification extensions.
    \item New ASICs in hardware to support the processing logic.
\end{itemize}

\highspace
\begin{flushleft}
    \textcolor{Red2}{\faIcon{exclamation-triangle} \textbf{Hardware Bottlenecks: The ASIC Development Bottleneck}}
\end{flushleft}
Here lies the \textbf{\underline{core problem}}: even with updated protocols, \textbf{switches couldn't adapt until vendors redesigned and shipped new ASICs} (Application-Specific Integrated Circuits).

\highspace
This hardware dependency meant:
\begin{itemize}[label=\textcolor{Red2}{\faIcon{times}}]
    \item New features took \textbf{years to reach production}.
    \item Network owners \textbf{couldn't simply get a software upgrade}.
    \item The result: \textbf{slow innovation} in data plane capabilities.
\end{itemize}

\newpage
\begin{examplebox}[: \texttt{VXLAN}]
    Virtual Extensible LAN (\texttt{VXLAN}) was urgently needed by cloud providers and datacenters to enable multi-tenant network virtualization.
    
    Despite this high demand, hardware vendors took $\approx 4$ years to support \texttt{VXLAN} in switches due to ASIC development cycles and the fixed-function nature of OpenFlow switches.
    
    Even though vendors delayed its release, once \texttt{VXLAN} was available, it became a standard requirement in data centers.

    But attention! In the meantime, network operators used complex software overlays or kludges to simulate \texttt{VXLAN} functionality, increasing network complexity and cost.
\end{examplebox}

\highspace
\begin{flushleft}
    \textcolor{Red2}{\faIcon{thumbs-down} \textbf{The Cost of Delay: Workarounds and Complexity}}
\end{flushleft}
When vendors take years to deliver a new feature, network engineers often \textbf{develop complex workarounds}, \textbf{increasing network complexity} and \textbf{technical debt}. Even when the vendor releases the official feature:
\begin{itemize}
    \item The workaround may already be deeply integrated.
    \item The official solution may no longer solve the problem.
    \item Worse, it may require a forklift upgrade, replacing hardware at high cost.
\end{itemize}
This inertia locks networks into \textbf{suboptimal solutions} and \textbf{impedes the agility promised by SDN}.

\highspace
\begin{flushleft}
    \textcolor{Green3}{\faIcon{\speedIcon} \textbf{The Missing Ingredient: Programmability at the Data Plane}}
\end{flushleft}
The shift from fixed-function to programmable data planes mirrors other computing domains:

\begin{table}[!htp]
    \centering
    \begin{tabular}{@{} l | l | l @{}}
        \toprule
        \textbf{Domain} & \textbf{Hardware} & \textbf{Compiler/SW Stack} \\
        \midrule
        General Computing & CPU             & Java, C, OS Kernels          \\ [.3em]
        Graphics          & GPU             & OpenCL, CUDA                 \\ [.3em]
        Signal Processing & DSP             & Matlab Compiler              \\ [.3em]
        Machine Learning  & TPU             & TensorFlow Compiler          \\ [.3em]
        \textbf{Networking} & \textbf{PISA Switch} & \textbf{P4 Language, P4 Compiler} \\
        \bottomrule
    \end{tabular}
\end{table}

\noindent
Just as CPUs became programmable via compilers, \textbf{networking needs} flexible \textbf{data planes programmable via languages} like P4, running on PISA (Protocol Independent Switch Architecture).

\highspace
\begin{takeawaysbox}[: OpenFlow limitations]
    \begin{itemize}
        \item OpenFlow was a \textbf{revolution in control plane innovation}, but its \textbf{rigid data plane} became a bottleneck. The industry's response, iterative protocol updates and ASIC redesigns, proved \textbf{slow and reactive}.

        \item A true solution lies in programmable data planes, where \textbf{software defines packet processing}, and the network evolves \textbf{as fast as the application demands}.

        \item This transition is \textbf{not trivial}, it requires new hardware, new abstractions, and operator retraining, but it's essential to \textbf{fulfill SDN's promise} of \textbf{rapid, flexible, and scalable networking}.
    \end{itemize}
\end{takeawaysbox}


    %%%%%%%%%%%%%%%%%%%%%%%%%
    % Programmable Switches %
    %%%%%%%%%%%%%%%%%%%%%%%%%
    \section{VLIW (Very Long Instruction Word)}

\subsection{Introduction}

Traditional \textbf{compilers} use \textbf{static code scheduling} to exploit Instruction-Level Parallelism (ILP). Key tasks of the compiler:
\begin{itemize}
    \item  Detect \textbf{parallelizable} instructions considering: Hardware resource constraints and Data dependencies.
    \item \textbf{Schedule} instructions to execute \textbf{in parallel} when possible.
    \item Otherwise, \textbf{insert \texttt{NOP}s} (No Operations) if no safe parallel execution is possible.
\end{itemize}
Statically scheduled processors trust the compiler to ``fill the pipeline'' and avoid hazards at compile time.

\highspace
\begin{flushleft}
    \textcolor{Green3}{\faIcon{book} \textbf{VLIW Processors: Alternative Way to Extract ILP}}
\end{flushleft}
\definition{VLIW (Very Long Instruction Word)} \textbf{processors} are a class of architectures designed to \textbf{execute multiple operations in parallel during a single clock cycle}, but unlike superscalar processors, they \textbf{rely on the compiler} rather than on dynamic hardware mechanisms \textbf{to detect and schedule parallelism}.

\highspace
The fundamental idea behind VLIW is to \textbf{group several independent operations together into a single long instruction word}, called a \hl{bundle}. This bundle is \textbf{composed of several fixed slots}, each one \textbf{corresponding to a different functional unit in the processor}, such as an integer ALU, a floating-point unit, a load/store unit, or a branch unit.

\highspace
For example, in a 5-issue VLIW processor, a single bundle would typically carry up to five operations: integer, floating-point, load/store, branch, etc.

\highspace
The \textbf{key difference} compared \textbf{to traditional superscalar} processors lies in who decides what can run in parallel. In superscalar designs, the hardware dynamically analyzes dependencies between instructions at runtime, which requires complex circuitry for hazard detection and scheduling. In VLIW architectures, this \hl{analysis is performed entirely at compile time}. The \textbf{compiler is responsible for}:
\begin{itemize}
    \item \textbf{Statically identify} independent operations.
    \item \textbf{Solve structural hazards} (e.g., two operations trying to use the same hardware unit).
    \item \textbf{Solve data hazards} (dependencies between instructions).
    \item Insert \textbf{NOPs} when necessary.
\end{itemize}
When no useful instruction is available for a particular slot, the compiler inserts a \texttt{NOP} (no-operation). In cases where conflicts cannot be avoided, NOPs are inserted into the bundle to fill empty slots.

\highspace
\begin{flushleft}
    \textcolor{Green3}{\faIcon{question-circle} \textbf{Why move to Compiler?}}
\end{flushleft}
The problem with Superscalar is that the hardware requires complex dynamic scheduling and dependency checking, which \textbf{costs area and power}.

\highspace
\textbf{VLIW Idea} is:
\begin{itemize}
    \item Push \textbf{complexity} to the \textbf{compiler}.
    \item \textbf{Compiler groups parallel operations} into a \textbf{single bundle}.
    \item No need for runtime dependency checking anymore.
\end{itemize}
The VLIW paradigm is characterized by the use of \textbf{very wide instruction words}, \textbf{each containing multiple independent operations} (``syllables''). A multiple-issue VLIW processor typically features specialized units for integer, floating-point, memory access, and branch instructions. For example, a 4-issue VLIW processor will have four operation slots per bundle, each connected to a different unit.

\highspace
\begin{flushleft}
    \textcolor{Green3}{\faIcon{list-ul} \textbf{Multiple-issue VLIW: Operation Latencies}}
\end{flushleft}
An important aspect in VLIW scheduling is the \textbf{management of operation latencies}. Each operation type may require a \textbf{different number of clock cycles to complete}. Integer operations, memory accesses, and floating-point computations may all have varying latencies, and these differences must be carefully considered during scheduling. The \textbf{compiler must plan the execution} so that operations complete in the correct order and without causing unnecessary stalls in the pipeline.

\highspace
In summary, \hl{VLIW architectures represent a shift of complexity from the hardware to the compiler}, aiming for more efficient and simpler processor designs while demanding sophisticated compiler techniques to fully exploit available parallelism.

\highspace
\begin{flushleft}
    \textcolor{Green3}{\faIcon{book} \textbf{Single Program Counter and Branch Management}}
\end{flushleft}
In a VLIW processor, even though multiple operations are issued simultaneously, the architecture still uses a \textbf{single program counter} to fetch instructions. Each instruction fetched corresponds to a \textbf{bundle}, which can contain multiple parallel operations.

\highspace
Importantly, within each bundle, there can be \textbf{at most one branch instruction} that affects control flow. This constraint simplifies the handling of branches, making it easier for the processor to predict and manage the program's execution path.

\newpage

\begin{flushleft}
    \textcolor{Green3}{\faIcon{book} \textbf{Shared Multi-Ported Register File}}
\end{flushleft}
Since a VLIW processor issues multiple operations in parallel, the \textbf{register file must support multiple simultaneous reads and writes}. In a typical 4-issue VLIW machine, the register file needs to have enough ports to read \textbf{eight source operands} and write \textbf{four destination results} every clock cycle.

\highspace
This requirement implies a \textbf{shared, multi-ported register file}, which is one of the non-trivial aspects of the hardware design. It ensures that all functional units can access their operands without creating bottlenecks.

\highspace
\begin{flushleft}
    \textcolor{Green3}{\faIcon{book} \textbf{Importance of Parallelism and the Use of NOPs}}
\end{flushleft}
To achieve high performance, the \textbf{compiler must ensure that the source code has enough parallelism} to fill all the available operation slots in each bundle. When \textbf{sufficient independent instructions are not available}, \textbf{NOPs are inserted} into the empty slots. This insertion preserves the fixed structure of the bundle but can lead to inefficiencies if the application does not expose enough ILP.

\highspace
\begin{flushleft}
    \textcolor{Green3}{\faIcon{book} \textbf{Pipeline Organization}}
\end{flushleft}
VLIW processors often use a \textbf{pipelined execution model} similar to traditional RISC architectures. A standard pipeline might include stages like \textbf{Instruction Fetch (IF)}, \textbf{Instruction Decode (ID)}, \textbf{Register Read (RR)}, \textbf{Execution (EX)}, and \textbf{Write Back (WB)}.

\begin{figure}[!htp]
    \centering
    \includegraphics[width=\textwidth]{img/vliw-arch.pdf}
\end{figure}

\highspace
In the previous figure, a \textbf{5-stage pipeline} is used, with each bundle passing through these stages. Each operation within a bundle proceeds independently through the functional units connected to the pipeline.

\newpage

\begin{flushleft}
    \textcolor{Green3}{\faIcon{book} \textbf{Operation Dispatch and Decode}}
\end{flushleft}
If the architecture dedicates \textbf{one functional unit per slot}, the decode stage becomes relatively simple. Each operation is directed straight to its corresponding functional unit for execution without the need for complex arbitration.

\highspace
However, if there are \textbf{more functional units than slots}, meaning there is a surplus of parallel hardware resources, the architecture must include a \textbf{dispatch network}. This network is responsible for forwarding each operation, along with its operands, to the appropriate available functional unit. In this way, the design of the dispatch system depends heavily on the organization of functional units relative to the number of issue slots defined by the instruction bundle format.

    \subsection{Why didn't programmable switches exist before?}

In short, \textbf{programmable switches didn't make sense before} because we lacked the technical feasibility and practical justification. But now, due to advances in chip design and network complexity, it's finally possible, and necessary, to build them.

\highspace
\begin{flushleft}
    \textcolor{Red2}{\faIcon{dollar-sign} \textbf{In the past: Programmability was too expensive}}
\end{flushleft}
In the past, the trade-off between programmability and cost was too high:
\begin{enumerate}
    \item \textcolor{Red2}{\textbf{Performance was too low}}. Programmable hardware, like FPGAs or general-purpose CPUs, was much slower than fixed-function ASICs.
    \begin{itemize}
        \item[\textcolor{Green3}{\faIcon{check}}] A \textbf{fixed switch chip} could forward billions of packets per second.
        \item[\textcolor{Red2}{\faIcon{times}}] A \textbf{programmable one}? Too slow for line-rate performance.
    \end{itemize}
    So \textbf{if we wanted programmability, we had to sacrifice speed}. That was a deal-breaker for core network equipment.


    \item \textcolor{Red2}{\textbf{Chip area and power cost were too high}}. Fixed-function logic is compact and power-efficient. Programmable logic, by contrast, used to \textbf{take up more silicon and required more power}. Result: vendors and data center operators couldn't justify using programmable switches, they were too big, too hot, and too slow.
\end{enumerate}

\highspace
\begin{flushleft}
    \textcolor{Green3}{\faIcon{check-circle} \textbf{What changed?}}
\end{flushleft}
\textbf{Three technological trends} made programmable switches finally viable:
\begin{enumerate}
    \item[\textcolor{Green3}{\faIcon{\speedIcon}}] \textcolor{Green3}{\textbf{Chip speed caught up}}. We now have programmable switch chips (like Barefoot Tofino) that can \textbf{run at line rate}, just like fixed-function ones. In other words, programmability no longer costs us speed.
    \item[\textcolor{Red2}{\faIcon{exclamation-triangle}}] \textcolor{Red2}{\textbf{Network complexity exploded}}. There are now \textbf{too many protocols and features} to hard-code everything into silicon:
    \begin{itemize}
        \item New protocols, encapsulations (VXLAN, GTP, QUIC, etc.)
        \item Monitoring, load balancing, AI, security; all need custom, real-time logic.
    \end{itemize}
    Hard-coding all of this would take years, and would never be flexible enough.
    \item[\textcolor{Green3}{\faIcon{check}}] \textcolor{Green3}{\textbf{Moore's Law made logic ``free''}}. Thanks to Moore's Law:
    \begin{itemize}
        \item We can double the amount of logic in the same area every 2 years.
        \item The \textbf{cost} of programmability in terms \textbf{of chip area} and \textbf{power} has become \textbf{negligible}.
    \end{itemize}
    Now, the logic that makes a switch programmable barely takes up more space than a fixed-function design.
\end{enumerate}

\newpage

\begin{table}[!htp]
    \centering
    \begin{tabular}{@{} l | l | l @{}}
        \toprule
        \textbf{Factor} & \textbf{Before}               & \textbf{Now}              \\
        \midrule
        Chip speed      & Too slow for line-rate        & Equal to fixed-function   \\ [.3em]
        Logic cost      & Too expensive (area + power)  & Basically free            \\ [.3em]
        Protocols       & Few, stable                   & Too many to hard-code     \\ [.3em]
        Urgency         & Low                           & High (cloud, IoT, 5G, ML) \\
        \bottomrule
    \end{tabular}
    \caption{Why didn't programmable switches exist before?}
\end{table}
    \subsection{Data Plane Programming and P4}

\textbf{Traditionally}, configuring a switch meant \textbf{writing static forwarding rules}, usually via vendor-specific commands or protocols like OpenFlow. But this was \textbf{not true programmability}. We could \textbf{configure behavior}, but we \textbf{couldn't change how the switch processes packets internally}. With P4 (Programming Protocol-independent Packet Processors), that changes.

\highspace
\begin{flushleft}
    \textcolor{Green3}{\faIcon{book} \textbf{What is P4?}}
\end{flushleft}
\definition{P4 (Programming Protocol-independent Packet Processors)} is a \textbf{high-level}, \textbf{domain-specific programming language} designed to \hl{describe how packets should be processed by the data plane of a network device}.

\highspace
Unlike general-purpose languages like C or Python, P4 is not Turing-complete. Instead, it is built to:
\begin{itemize}
    \item Define \textbf{how to parse packet headers}
    \item Specify \textbf{how to match on those headers}
    \item Decide \textbf{what actions to take}
\end{itemize}
The \textbf{goal of P4 is to describe the behavior of the switch pipeline}, not to implement general algorithms. Specifically, P4 was designed with four main goals in mind:
\begin{enumerate}
    \item \important{Reconfigurability}: We should be able to \hl{change switch behavior \emph{after} deployment}.
    \item \important{Protocol Independence}: The switch should not be tied to Ethernet/IP/TCP. \hl{We define the packet format}.
    \item \important{Target Independence}: The same \hl{P4 program should run on different hardware} (ASICs, FPGAs, software switches).
    \item \important{Flexibility} and \important{Abstraction}: Developers write in P4, and the \hl{compiler maps it to the switch's low-level pipeline architecture}.
\end{enumerate}

\highspace
\begin{flushleft}
    \textcolor{Green3}{\faIcon{balance-scale} \textbf{P4 is so cool, but OpenFlow is not the same?}}
\end{flushleft}
We already discussed what OpenFlow is in Section \ref{subsection: OpenFlow}, page \pageref{subsection: OpenFlow}. The short answer is no, P4 is different.
\begin{itemize}
    \item \textbf{OpenFlow} is a \textbf{control protocol} for \hl{configuring predefined forwarding behavior}.
    \item \textbf{P4} is a \textbf{programming language} for \hl{defining the forwarding behavior} itself.
\end{itemize}

\highspace
Let's make an analogy to understand the difference.
\begin{itemize}
    \item \textbf{OpenFlow is like the driver of a regular car}. The driver can:
    \begin{itemize}[label=\textcolor{Green3}{\faIcon{check}}]
        \item Steer left or right
        \item Press the gas or brake
        \item Use turn signals, radio, windshield wipers
    \end{itemize}
    But the driver can't:
    \begin{itemize}[label=\textcolor{Red2}{\faIcon{times}}]
        \item Change how the engine works
        \item Reprogram how turning the wheel affects the tires
        \item Add a new driving mode (e.g., ``turbo boost'')
    \end{itemize}
    That's OpenFlow. We're in control of what happens (where to drive, how fast), but how the car works internally is fixed. We're controlling pre-built behavior, we're not changing the system.

    \item \textbf{P4 is like the car engineer or mechanic}. The car engineer can:
    \begin{itemize}[label=\textcolor{Green3}{\faIcon{check}}]
        \item Redefine how the steering works (e.g., make left turn rotate only one wheel)
        \item Change how the engine responds to the pedal
        \item Add entirely new modules (e.g., self-driving mode, rocket engine, etc.)
    \end{itemize}
    That's P4. We're not just driving the car, we're deciding what the car is capable of doing in the first place. We write the ``rules'' for how the system should behave.
\end{itemize}

\highspace
\begin{flushleft}
    \textcolor{Green3}{\faIcon{tools} \textbf{Workflow}}
\end{flushleft}
\begin{enumerate}
    \item Before starting to write a P4 program, is \textbf{necessary to know the P4 Architecture Model}. The \definition{P4 Architecture Model} is a \textbf{logical interface} between:
    \begin{itemize}
        \item The \textbf{P4 program} written by the developer.
        \item The underlying \textbf{hardware target} (e.g., ASIC, FPGA, software switch)
    \end{itemize}
    This model tells the compiler: ``here's what the hardware looks like, these are the building blocks our P4 program can use.''. This abstracts away hardware details and makes P4 programs portable across multiple targets.

    It's pretty obvious that the P4 architecture model is defined by the hardware switch we have. Because if our switch doesn't support some feature (e.g. packet cloning, a second pipeline), we can't use it.


    \item \important{Write the P4 Program}. The network operator or developer writes a P4 program to describe:
    \begin{itemize}
        \item Which \textbf{packet headers} to parse (e.g., Ethernet, IP, or custom)
        \item What \textbf{tables} to build (match fields, actions)
        \item How the \textbf{control flow} works (pipeline logic)
        \item What actions to perform (forward, drop, modify, etc.)
    \end{itemize}
    This is written in a \texttt{.p4} file.

    
    \item \important{Compile the P4 Program}. The P4 program is passed to a \definition{P4 Compiler}, which does two main things:
    \begin{enumerate}
        \item \textbf{Generates a device-specific binary}. This is tailored to the target hardware (e.g., Tofino, FPGA, software switch like MBv2).
        \item \textbf{Produces a runtime API}. This allows a controller (or CLI) to: install rules (e.g., match on \texttt{dstIP=10.0.0.1} forward to port 3), modify tables dynamically.
    \end{enumerate}
    The result is something the switch can understand and execute.


    \item \important{Deploy to the Switch (Target)}. The compiled \textbf{output is loaded onto a P4-capable target}, such as: an ASIC (e.g., Barefoot Tofino), an FPGA-based switch, a software simulator (e.g., BMv2). At this point, the switch now knows how to: parse packets, match them in tables, take programmed actions.
    

    \item \important{Runtime Table Configuration}. Once the program is installed, we still need to:
    \begin{itemize}
        \item \textbf{Populate the tables} with actual forwarding rules.
        \item This is usually done via a \textbf{controller}, using a runtime API (e.g., gRPC, Thrift, P4Runtime)
    \end{itemize}
    It's like programming the switch with policy, after the logic has been defined.
\end{enumerate}
Finally, the user is only concerned with the P4 program and the controller (to populate the tables). Instead, the P4 compiler, the P4 architecture model, and the switch (e.g., ASIC) are provided by the vendor.


    \subsection{PISA and Compiler Pipeline Mapping}

\definition{Protocol-Independent Switch Architecture (PISA)} is the \textbf{hardware abstraction} used by modern programmable switches (e.g., Barefoot Tofino). The idea behind PISA is simple but powerful: instead of building fixed-function blocks into hardware (e.g., IP routers, firewalls), \textbf{expose a generic pipeline of programmable stages}, and \textbf{let software define what each stage does}.

\highspace
\begin{flushleft}
    \textcolor{Green3}{\faIcon{microchip} \textbf{PISA Architecture}}
\end{flushleft}
A PISA switch consists of the following main components:
\begin{itemize}
    \item \important{Parser}. \textbf{Extracts} packet \textbf{headers} and \textbf{creates} a structured \textbf{representation} (called a \definition{Packet Header Vector}, or PHV). The PHV contains the keys for the match-Action units.
    
    \item \important{Multiple Match-Action Stages}. A pipeline of identical stages. Each stage:
    \begin{itemize}
        \item Matches on some fields (using SRAM or TCAM)
        \item Executes simple actions (via Arithmetic Logic Units - ALUs)
        \item Modifies the PHV (e.g., changing a header field, setting a drop flag)
    \end{itemize}
    
    \item \important{Deparser}. \textbf{Reassembles the packet} by combining the (possibly modified) headers and payload. Every packet flows through this pipeline, so the logic must be fully deterministic and parallelizable.
\end{itemize}

\begin{figure}[!htp]
    \centering
    \includegraphics[width=\textwidth]{img/pisa-arch.pdf}
    \caption{PISA architecture.}
\end{figure}

\highspace
\begin{flushleft}
    \textcolor{Green3}{\faIcon{question-circle} \textbf{Why use a pipelined architecture instead of a single processor?}}
\end{flushleft}
A naive design would use \textbf{one CPU} to handle every packet: perform all lookups (routing, ACLs, NAT, etc.), apply all rules. But this would require an \textbf{unrealistically high frequency} to process billions of packets per second.

\highspace
Just like in CPUs, we divide the processing into \textbf{stages}, each with: local memory (tables), local ALU, fixed resources. Each packet moves one stage forward per clock cycle, so we can \textbf{process many packets in parallel}.

\highspace
\begin{flushleft}
    \textcolor{Green3}{\faIcon{check-circle} \textbf{Protocol Independence}}
\end{flushleft}
One of \textbf{PISA}'s most powerful features is that the chip \textbf{knows nothing in advance}.
\begin{itemize}
    \item It doesn't recognize IP, Ethernet, TCP, or any protocol at all.
    \item The \textbf{programmer defines everything}: what headers to parse, what fields to match, what actions to perform.
\end{itemize}
This is what makes it \textbf{protocol-independent}, and feature-proof.

\highspace
\begin{flushleft}
    \textcolor{Green3}{\faIcon{tools} \textbf{What does the compile do?}}
\end{flushleft}
Here's the key part of PISA and P4: we don't directly \textbf{program the pipeline}, \textbf{the compiler does}. We write a logical program in P4, and the P4 compiler:
\begin{itemize}
    \item Analyzes dependencies between operations:
    \begin{itemize}
        \item \textbf{Match dependency}: A table needs data generated by a previous match.
        \item \textbf{Action dependency}: An action needs a value produced by a previous action.
    \end{itemize}
    \item Packs logic into stages without violating resource limits
    \item Ensure parallelism and no data hazards
\end{itemize}

    %%%%%%%%%%%%%%%%%%%
    % Data Structures %
    %%%%%%%%%%%%%%%%%%%
    \section{VLIW (Very Long Instruction Word)}

\subsection{Introduction}

Traditional \textbf{compilers} use \textbf{static code scheduling} to exploit Instruction-Level Parallelism (ILP). Key tasks of the compiler:
\begin{itemize}
    \item  Detect \textbf{parallelizable} instructions considering: Hardware resource constraints and Data dependencies.
    \item \textbf{Schedule} instructions to execute \textbf{in parallel} when possible.
    \item Otherwise, \textbf{insert \texttt{NOP}s} (No Operations) if no safe parallel execution is possible.
\end{itemize}
Statically scheduled processors trust the compiler to ``fill the pipeline'' and avoid hazards at compile time.

\highspace
\begin{flushleft}
    \textcolor{Green3}{\faIcon{book} \textbf{VLIW Processors: Alternative Way to Extract ILP}}
\end{flushleft}
\definition{VLIW (Very Long Instruction Word)} \textbf{processors} are a class of architectures designed to \textbf{execute multiple operations in parallel during a single clock cycle}, but unlike superscalar processors, they \textbf{rely on the compiler} rather than on dynamic hardware mechanisms \textbf{to detect and schedule parallelism}.

\highspace
The fundamental idea behind VLIW is to \textbf{group several independent operations together into a single long instruction word}, called a \hl{bundle}. This bundle is \textbf{composed of several fixed slots}, each one \textbf{corresponding to a different functional unit in the processor}, such as an integer ALU, a floating-point unit, a load/store unit, or a branch unit.

\highspace
For example, in a 5-issue VLIW processor, a single bundle would typically carry up to five operations: integer, floating-point, load/store, branch, etc.

\highspace
The \textbf{key difference} compared \textbf{to traditional superscalar} processors lies in who decides what can run in parallel. In superscalar designs, the hardware dynamically analyzes dependencies between instructions at runtime, which requires complex circuitry for hazard detection and scheduling. In VLIW architectures, this \hl{analysis is performed entirely at compile time}. The \textbf{compiler is responsible for}:
\begin{itemize}
    \item \textbf{Statically identify} independent operations.
    \item \textbf{Solve structural hazards} (e.g., two operations trying to use the same hardware unit).
    \item \textbf{Solve data hazards} (dependencies between instructions).
    \item Insert \textbf{NOPs} when necessary.
\end{itemize}
When no useful instruction is available for a particular slot, the compiler inserts a \texttt{NOP} (no-operation). In cases where conflicts cannot be avoided, NOPs are inserted into the bundle to fill empty slots.

\highspace
\begin{flushleft}
    \textcolor{Green3}{\faIcon{question-circle} \textbf{Why move to Compiler?}}
\end{flushleft}
The problem with Superscalar is that the hardware requires complex dynamic scheduling and dependency checking, which \textbf{costs area and power}.

\highspace
\textbf{VLIW Idea} is:
\begin{itemize}
    \item Push \textbf{complexity} to the \textbf{compiler}.
    \item \textbf{Compiler groups parallel operations} into a \textbf{single bundle}.
    \item No need for runtime dependency checking anymore.
\end{itemize}
The VLIW paradigm is characterized by the use of \textbf{very wide instruction words}, \textbf{each containing multiple independent operations} (``syllables''). A multiple-issue VLIW processor typically features specialized units for integer, floating-point, memory access, and branch instructions. For example, a 4-issue VLIW processor will have four operation slots per bundle, each connected to a different unit.

\highspace
\begin{flushleft}
    \textcolor{Green3}{\faIcon{list-ul} \textbf{Multiple-issue VLIW: Operation Latencies}}
\end{flushleft}
An important aspect in VLIW scheduling is the \textbf{management of operation latencies}. Each operation type may require a \textbf{different number of clock cycles to complete}. Integer operations, memory accesses, and floating-point computations may all have varying latencies, and these differences must be carefully considered during scheduling. The \textbf{compiler must plan the execution} so that operations complete in the correct order and without causing unnecessary stalls in the pipeline.

\highspace
In summary, \hl{VLIW architectures represent a shift of complexity from the hardware to the compiler}, aiming for more efficient and simpler processor designs while demanding sophisticated compiler techniques to fully exploit available parallelism.

\highspace
\begin{flushleft}
    \textcolor{Green3}{\faIcon{book} \textbf{Single Program Counter and Branch Management}}
\end{flushleft}
In a VLIW processor, even though multiple operations are issued simultaneously, the architecture still uses a \textbf{single program counter} to fetch instructions. Each instruction fetched corresponds to a \textbf{bundle}, which can contain multiple parallel operations.

\highspace
Importantly, within each bundle, there can be \textbf{at most one branch instruction} that affects control flow. This constraint simplifies the handling of branches, making it easier for the processor to predict and manage the program's execution path.

\newpage

\begin{flushleft}
    \textcolor{Green3}{\faIcon{book} \textbf{Shared Multi-Ported Register File}}
\end{flushleft}
Since a VLIW processor issues multiple operations in parallel, the \textbf{register file must support multiple simultaneous reads and writes}. In a typical 4-issue VLIW machine, the register file needs to have enough ports to read \textbf{eight source operands} and write \textbf{four destination results} every clock cycle.

\highspace
This requirement implies a \textbf{shared, multi-ported register file}, which is one of the non-trivial aspects of the hardware design. It ensures that all functional units can access their operands without creating bottlenecks.

\highspace
\begin{flushleft}
    \textcolor{Green3}{\faIcon{book} \textbf{Importance of Parallelism and the Use of NOPs}}
\end{flushleft}
To achieve high performance, the \textbf{compiler must ensure that the source code has enough parallelism} to fill all the available operation slots in each bundle. When \textbf{sufficient independent instructions are not available}, \textbf{NOPs are inserted} into the empty slots. This insertion preserves the fixed structure of the bundle but can lead to inefficiencies if the application does not expose enough ILP.

\highspace
\begin{flushleft}
    \textcolor{Green3}{\faIcon{book} \textbf{Pipeline Organization}}
\end{flushleft}
VLIW processors often use a \textbf{pipelined execution model} similar to traditional RISC architectures. A standard pipeline might include stages like \textbf{Instruction Fetch (IF)}, \textbf{Instruction Decode (ID)}, \textbf{Register Read (RR)}, \textbf{Execution (EX)}, and \textbf{Write Back (WB)}.

\begin{figure}[!htp]
    \centering
    \includegraphics[width=\textwidth]{img/vliw-arch.pdf}
\end{figure}

\highspace
In the previous figure, a \textbf{5-stage pipeline} is used, with each bundle passing through these stages. Each operation within a bundle proceeds independently through the functional units connected to the pipeline.

\newpage

\begin{flushleft}
    \textcolor{Green3}{\faIcon{book} \textbf{Operation Dispatch and Decode}}
\end{flushleft}
If the architecture dedicates \textbf{one functional unit per slot}, the decode stage becomes relatively simple. Each operation is directed straight to its corresponding functional unit for execution without the need for complex arbitration.

\highspace
However, if there are \textbf{more functional units than slots}, meaning there is a surplus of parallel hardware resources, the architecture must include a \textbf{dispatch network}. This network is responsible for forwarding each operation, along with its operands, to the appropriate available functional unit. In this way, the design of the dispatch system depends heavily on the organization of functional units relative to the number of issue slots defined by the instruction bundle format.

    \subsection{Ternary Content Addressable Memory (TCAM)}

\definition{Ternary Content Addressable Memory (TCAM)} is a specialized kind of (hardware) memory that works differently from standard RAM. Instead of accessing data by address, TCAM lets we input data and instantly tells we \textbf{if and where it's stored}. This is called \textbf{associative memory} or \textbf{content-based lookup}. It is \hl{built specifically for fast parallel search}.

\highspace
Unlike binary memories (which store 0s and 1s), \textbf{TCAMs can store 0, 1, or a third state} (ternary) called ``\emph{don't care}''. This third value allows flexible and partial matching, making TCAMs very effective for operations like Longest Prefix Match (LPM) in IP routing.

\highspace
\begin{flushleft}
    \textcolor{Green3}{\faIcon{balance-scale} \textbf{RAM vs TCAM}}
\end{flushleft}
\begin{itemize}
    \item RAM (Random Access Memory):
    \begin{itemize}
        \item Ask: ``\textbf{What} is stored at address \texttt{X}?''.
        \item Classic address-value access.
    \end{itemize}

    \item TCAM:
    \begin{itemize}
        \item Ask: ``\textbf{Where} is the value \texttt{X} stored?''.
        \item The memory searches all entries in parallel and returns the matching address in constant time. In other words, it \textbf{returns the address where the value is stored}.
    \end{itemize}
\end{itemize}
This associative search is \textbf{very fast}, which is why TCAM is often used in \textbf{packet classification} and \textbf{routing tables} in high-speed switches. Usually these two hardware are \textbf{put together} because the TCAM gives the index, we use it to index the RAM, and we get the information.

\highspace
\begin{flushleft}
    \textcolor{Green3}{\faIcon{check-circle} \textbf{Pros}}
\end{flushleft}
\begin{itemize}[label=\textcolor{Green3}{\faIcon{check}}]
    \item \textcolor{Green3}{\textbf{Speed}}: Lookup happens in constant time, regardless of the number of entries.
    \item \textcolor{Green3}{\textbf{Wildcard Matching}}: TCAMs handle ``don't care'' bits, allowing prefix and pattern-based lookups.
    \item \textcolor{Green3}{\textbf{Ideal for Match-Action Pipelines}}: TCAM is a good fit for hardware pipelines like those found in P4-programmable switches.
\end{itemize}

\highspace
\begin{flushleft}
    \textcolor{Red2}{\faIcon{times-circle} \textbf{Cons}}
\end{flushleft}
\begin{itemize}[label=\textcolor{Red2}{\faIcon{times}}]
    \item \textcolor{Red2}{\textbf{High power consumption}}: Every lookup checks all entries in parallel.
    \item \textcolor{Red2}{\textbf{Expensive}}: Due to the hardware complexity and power demands.
\end{itemize}

\begin{examplebox}[: TCAM in packet routing]
    Imagine a TCAM storing IP prefixes:
    \begin{itemize}
        \item \texttt{0: 192.168.3.0/24}
        \item \texttt{1: 192.168.1.0/24}
        \item \texttt{2: 192.168.2.0/24}
    \end{itemize}
    If an incoming packet has destination IP \texttt{192.168.2.1}, the TCAM instantly finds that it matches entry 2.

    However, this match index alone isn't enough to decide what to do. So, we usually pair TCAM with a RAM block that stores the actual action:
    \begin{enumerate}
        \item TCAM gives the index
        \item Use it to index RAM
        \item Get forwarding info, output port, etc.
    \end{enumerate}
\end{examplebox}

\highspace
\begin{flushleft}
    \textcolor{Red2}{\faIcon{exclamation-triangle} \textbf{Dealing with Multiple Matches}}
\end{flushleft}
Sometimes a destination IP can match multiple entries. For example:
\begin{itemize}
    \item \texttt{Entry 2: 192.168.2.0/24}
    \item \texttt{Entry 3: 192.168.2.0/28}
\end{itemize}
Both may match the same address, but \textbf{only one result is returned}. Depending on the hardware, this could be:
\begin{itemize}
    \item The \textbf{lowest matching index} (first match)
    \item The \textbf{highest matching index} (last match)
\end{itemize}
To ensure correct behavior (e.g., always choosing the most specific prefix), \textbf{entries need to be} carefully \textbf{ordered}. This introduces \underline{extra logic} during configuration or compile time.

\highspace
\begin{flushleft}
    \textcolor{Red2}{\faIcon{microchip} \textbf{Extra Hardware: SRAM}}
\end{flushleft}
Alongside TCAM, \textbf{each pipeline stage} may also have \textbf{SRAM}. It's used for:
\begin{itemize}
    \item Storing values linked to TCAM matches.
    \item Keeping state (e.g., counters, flags).
    \item Performing fast value retrieval during match-action processing.
\end{itemize}
SRAM is faster and cheaper than TCAM, but does not supper associative lookup, so it \textbf{complements TCAM} rather than replacing it.

\highspace
\begin{takeawaysbox}
    \begin{itemize}
        \item TCAM is \textbf{fast}, parallel, and supports wildcards, great for networking.
        \item It's \textbf{costly and power-hungry}, so it's used sparingly and carefully.
        \item Works in tandem with \textbf{SRAM} for decision and action pipelines.
        \item Entry \textbf{ordering matters} to get correct behavior (e.g., longest prefix match).
    \end{itemize}
\end{takeawaysbox}
    \subsection{Deterministic Lookup with Probabilistic Performance}

\begin{flushleft}
    \textcolor{Red2}{\faIcon{question-circle} \textbf{Problem Setup}}
\end{flushleft}
We want to store a collection of elements (a set) in memory, and be able to:
\begin{itemize}
    \item \textbf{Insert} new elements.
    \item \textbf{Check} if an element exists.
    \item \textbf{Do it fast}, ideally in \underline{constant time}.
\end{itemize}
There are two broad strategies:
\begin{itemize}
    \item \important{Deterministic} (this section):
    \begin{itemize}
        \item[\textcolor{Green3}{\faIcon{check}}] \textcolor{Green3}{\textbf{Always}} gives the \textcolor{Green3}{\textbf{correct answer}} (\emph{deterministic lookup}, answer is always correct).
        \item[\textcolor{Red2}{\faIcon{times}}] \textcolor{Red2}{\textbf{Slower}} or \textcolor{Red2}{\textbf{require more memory}} (\emph{probabilistic performance}, time depends on insertion history).
    \end{itemize}
    With this type of data structures, we \textbf{always get the correct answer} (\texttt{true} if present, \texttt{false} if not), but the \textbf{number of steps} (e.g. in Separate Chaining, explained below, the steps are determined by traversing a chain) is \textbf{not fixed} because it depends on: collisions, load factory, quality of the hash function.

    \item \important{Probabilistic} (section \ref{subsection: Probabilistic Data Structures}, page \pageref{subsection: Probabilistic Data Structures}):
    \begin{itemize}
        \item[\textcolor{Green3}{\faIcon{check}}] \textcolor{Green3}{\textbf{Uses less memory}} or \textcolor{Green3}{\textbf{is faster}} (\emph{deterministic performance}, always the same number of operations).
        \item[\textcolor{Red2}{\faIcon{times}}] Might give \textcolor{Red2}{\textbf{false positives/negatives}} (\emph{probabilistic lookup}, result might be wrong with some small probability).
    \end{itemize}
    With this type of data structures, the \textbf{time is constant}, we have a fixed number of bit checks (usually one or a few), but the \textbf{answer can be wrong}. We can get a false positive (return \texttt{true} if the element isn't in the set), or we never get a false negative (if it says \texttt{false}, the element definitely wasn't inserted).
\end{itemize}
In this section we analyze the deterministic approach, so an output that we know what is, but the number of operations required is unknown (probabilistic). This is not suitable for network computing because it is detached from the PISA idea, but we present it for academic purposes.

\highspace
\begin{flushleft}
    \textcolor{Green3}{\faIcon{book} \textbf{Hash Table}}
\end{flushleft}
A \definition{Hash Function} \textbf{maps data} of arbitrary size (e.g., strings like ``hello'') \textbf{into a fixed-size integer space}. This \textbf{integer} is then \textbf{used as an index in an array} called a \definition{Hash Table}.

\highspace
Pseudo-code:
\begin{lstlisting}[language=python]
index = hash("hello")
hash_table[index] = "hello"
\end{lstlisting}
But \textbf{collisions can occur}, multiple inputs may hash to the same index. To handle this, we \textbf{use separate chaining}.

\highspace
\begin{flushleft}
    \textcolor{Green3}{\faIcon{check-circle} \textbf{Separate Chaining: The Basic Idea}}
\end{flushleft}
The basic idea of \definition{Separate Chaining} is as follows. If two values hash to the same index, we \textbf{chain} them \textbf{together in a list}:
\begin{equation*}
    \texttt{Index }10 \rightarrow \texttt{ ``hello'' } \rightarrow \texttt{ ``port'' } \rightarrow \texttt{ ``fire''}
\end{equation*}
So instead of storing just one value per index, we allow \textbf{each index to store a linked list} (or vector, or queue).

\highspace
\begin{flushleft}
    \textcolor{Green3}{\faIcon{\speedIcon} \textbf{Performance Analysis}}
\end{flushleft}
Let's say we're inserting $N$ elements into a table with $M$ buckets.
\begin{itemize}
    \item \textbf{Average list size}: $\dfrac{N}{M}$
    \item \textcolor{Green3}{\textbf{Best case}} (uniform distribution): All chains are of similar length $\rightarrow$ fast lookups.
    \item \textcolor{Red2}{\textbf{Worst case}}: All $N$ elements hash to the same bucket $\rightarrow$ one long chain $\rightarrow$ $O\left(N\right)$ lookup time.
\end{itemize}
So the \textbf{load factor} $\dfrac{N}{M}$ is key to understanding performance.

\highspace
\begin{flushleft}
    \textcolor{Red2}{\faIcon{exclamation-triangle} \textbf{Collision Probability}}
\end{flushleft}
How likely is it to avoid collisions at all?
\begin{itemize}
    \item 1st insertion: no collision.
    \item 2nd: no collision with probability $1 - \dfrac{1}{M}$
    \item 3rd: no collision with probability $1 - \dfrac{2}{M}$
    \item ...
    \item $N$-th: no collision with probability $1 - \dfrac{N-1}{M}$
\end{itemize}
Multiply all together to get the probability of \textbf{zero collisions}:
\begin{equation}
    P\left(N, M\right) = \displaystyle\prod_{i=0}^{N-1} \left(1 - \dfrac{i}{M}\right)
\end{equation}
This means that even if $M = 10000$ and $N = 100$, the chance of having at least one collision is about 40\%! In other words, \textbf{collisions are almost inevitable unless} $M \gg N$.

\highspace
\begin{flushleft}
    \textcolor{Green3}{\faIcon{check-circle} \textbf{Pros}} and \textcolor{Red2}{\faIcon{times-circle} \textbf{Cons}} of Separate Chaining
\end{flushleft}
\begin{itemize}
    \item[\textcolor{Green3}{\faIcon{check}}] \textbf{Output deterministic and accurate}, no false positives/negatives.
    \item[\textcolor{Green3}{\faIcon{check}}] Simple and well-understood.
    \item[\textcolor{Green3}{\faIcon{check}}] \textbf{Performs well if load factor is low}.
    \item[\textcolor{Red2}{\faIcon{times}}] \textbf{Memory usage can grow} if many chains form.
    \item[\textcolor{Red2}{\faIcon{times}}] \textbf{Slower when many elements are inserted and collisions increase}.
    \item[\textcolor{Red2}{\faIcon{times}}] Not ideal for extremely large-scale or memory-constrained environments.
\end{itemize}
    \include{sections/data-structures/probabilistic-data-structures/one-hash-bloom-filters}
    \subsubsection{Bloom Filters}

A \definition{Bloom Filter} is a space-efficient probabilistic data structure used for \textbf{membership queries}:
\begin{itemize}
    \item \textbf{Fast} insertions lookups.
    \item \textbf{No false negatives}, but may return \textbf{false positives}.
    \item This trade-off is ideal for \textbf{fixed-latency, high-speed systems} (like programmable switches).
\end{itemize}

\highspace
\begin{flushleft}
    \textcolor{Green3}{\faIcon{\speedIcon} \textbf{Generalization of the 1-hash Bloom filter to $k$-hash}}
\end{flushleft}
To \textcolor{Green3}{\textbf{reduce false positives}}, we \textbf{extend the 1-hash Bloom Filter}:
\begin{itemize}
    \item Instead of just 1 hash function, we use $K$ different hash functions.
    \item Each function maps the input element to a different positions in the bit array.
\end{itemize}

\highspace
\begin{flushleft}
    \textcolor{Green3}{\faIcon{tools} \textbf{How Insertion Works}}
\end{flushleft}
Let's say we want to insert ``\texttt{Rust}'':
\begin{enumerate}
    \item Compute $K$ hash functions in parallel:
    \begin{equation*}
        h_{1}\left(x\right) \hspace{2em} h_{2}\left(x\right) \hspace{2em} \dots \hspace{2em} h_{K}\left(x\right)
    \end{equation*}

    \item For each $h_{i}\left(x\right)$, set the bit at position $h_{1}\left(x\right)$ to \texttt{1}.
    \begin{itemize}
        \item $h_{1}\left(x\right) = 1$
        \item $h_{2}\left(x\right) = 1$
        \item $\dots$
        \item $h_{K}\left(x\right) = 1$
    \end{itemize}
\end{enumerate}

\highspace
\begin{flushleft}
    \textcolor{Green3}{\faIcon{search} \textbf{How Lookup Works}}
\end{flushleft}
To check whether an element is in the set:
\begin{itemize}
    \item Compute all $K$ hashes
    \item If \textbf{all corresponding bits are set to 1}, we return \texttt{true} (element may be present)
    \item If \textbf{at least one bit is 0}, we return \texttt{false} (element is definitely not present)
\end{itemize}

\newpage

\begin{flushleft}
    \textcolor{Red2}{\faIcon{exclamation-triangle} \textbf{False Positives}}
\end{flushleft}
Let's say ``\texttt{Fire}'' is not inserted but happens to have all its hash bits already set by ``\texttt{Rust}'', ``\texttt{Hello}'', or ``\texttt{Fine}''. The filter will wrongly return \texttt{true}, a false positive. Still, \textbf{no false negatives} can occur: if an element was inserted, all bits are set, and it will always return \texttt{true}.

\highspace
\begin{flushleft}
    \textcolor{Green3}{\faIcon{percent} \textbf{Probability Analysis}}
\end{flushleft}
Let:
\begin{itemize}
    \item $N$: number of \textbf{inserted elements}
    \item $M$: number of \textbf{bits in the filter}
    \item $K$: number of \textbf{hash functions}
\end{itemize}
Then:
\begin{itemize}
    \item \textbf{Probability} a particular \textbf{cell is still 0 after inserting $N$ elements}:
    \begin{equation}
        \left(1 - \dfrac{1}{M}\right)^{\left(K \cdot N\right)}
    \end{equation}

    \item \textbf{Probability} of a false positive (all $K$ bits set for a non-inserted element), the \definition{False Positive Rate (FPR)}:
    \begin{equation}
        \text{FPR } = \left(1 - \left(1 - \dfrac{1}{M}\right)^{\left(K \cdot N\right)}\right)^{K}
    \end{equation}
\end{itemize}
Just as an idea, with $1'000$ elements inserted, $10'000$ bits in the filter (cells), and 7 hash computations, we get a probability of FPR of only 0.82\%. And if we increase the bits in the filter ($M$) to $100'000$, the FPR is about 0\%! So, with a \textbf{moderate increase in memory and hash computations}, we can get extremely low FPRs.

\highspace
\begin{flushleft}
    \textcolor{Green3}{\faIcon{check-circle} \textbf{Pros}}
\end{flushleft}
\begin{itemize}[label=\textcolor{Green3}{\faIcon{check}}]
    \item Very \textbf{memory-efficient}, uses up to $10\times$ less memory \textbf{than separate chaining}.
    \item Lookup and insertions are \textbf{predictable and fast}, constant time with $K$ steps.
    \item Still \textbf{no false negatives}.
\end{itemize}

\highspace
\begin{flushleft}
    \textcolor{Red2}{\faIcon{times-circle} \textbf{Cons}}
\end{flushleft}
\begin{itemize}[label=\textcolor{Red2}{\faIcon{times}}]
    \item Requires \textbf{more computation} than the single-hash version (e.g., 7 hash functions).
    \item Slightly \textbf{more complex to implement in hardware}.
\end{itemize}
    \subsubsection{Dimensioning a Bloom Filter}

We want to design a Bloom Filter that:
\begin{itemize}
    \item Stores $N$ elements
    \item Uses $M$ bits (memory size)
    \item Applies $K$ hash functions
\end{itemize}
But we also to \textbf{control the False Positive Rate (FPR)} and avoid unnecessary computation.

\highspace
There are three parameters in play:
\begin{enumerate}
    \item \textbf{Memory} $M$: more bits $\Rightarrow$ lower FPR
    \item \textbf{Number of Hashes} $K$: more hashes $\Rightarrow$ lower FPR, but higher computational cost
    \item \textbf{False Positive Rate} (FPR): we want this to be as low as possible.
\end{enumerate}
Improving one usually worsens another. This is the classic \textbf{space/time/error trade-off}.

\highspace
\begin{flushleft}
    \textcolor{Green3}{\faIcon{check-circle} \textbf{Asymptotic Approximation for FPR}}
\end{flushleft}
In our case, the Asymptotic Approximation is a simplified mathematical expression that \textbf{estimates the False Positive Rate (FPR)} of a Bloom Filter when the number of \textbf{cells $M$ is large}. It's derived from the exact expression but uses limits and approximations that hold when $M \gg N$. It's much easier to work with and very accurate in practice.

\highspace
If we insert $N$ elements into a Bloom filter with $M$ bits and use $K$ hash functions, the \textbf{exact False Positive Rate (FPR)}:
\begin{equation}
    \text{Exact FPR } = \left(1 - \left(1 - \dfrac{1}{M}\right)^{\left(K \cdot N\right)}\right)^{K}
\end{equation}
This expression can be tedious to compute, especially for large values of $M$, $N$, and $K$. By using the approximation:
\begin{equation*}
    \left(1 - \dfrac{1}{M}\right)^{K \cdot N} \approx e^{\left(-K \cdot \frac{N}{M}\right)} \hspace{2em} \text{when } M \gg 1
\end{equation*}
The \definition{Asymptotic Approximation of False Positive Rate (FPR)} is:
\begin{equation}
    \text{FPR } \approx \left(1 - e^{\left(-K \cdot \frac{N}{M}\right)}\right)^{K}
\end{equation}
This approximation is easier to analyze and is widely used in practice.

\highspace
\begin{flushleft}
    \textcolor{Green3}{\faIcon{question-circle} \textbf{Finding the Optimal Number of Hash Functions}}
\end{flushleft}
The optimal number of hash functions $K$ \textbf{minimizes the FPR} for given $M$ and $N$. We can find it by minimizing the FPR formula:
\begin{equation}
    K_{\text{opt}} = \dfrac{M}{N} \cdot \ln \left(2\right)
\end{equation}
    \subsubsection{Counting Bloom Filters}\label{sec:counting-bloom-filters}

In the standard Bloom Filter:
\begin{itemize}
    \item Inserting an element means setting multiple bits to \texttt{1}.
    \item But we \textbf{never know which element caused a bit to be \texttt{1}}, because multiple elements may share the same hash outputs.
\end{itemize}

\highspace
\begin{flushleft}
    \textcolor{Red2}{\faIcon{exclamation-triangle} \textbf{What happens if we try to delete?}}
\end{flushleft}
Let's say we inserted: ``\texttt{Rust}'' and ``\texttt{Hello}''. And now we want to delete ``\texttt{Rust}''. If ``\texttt{Rust}'' and ``\texttt{Hello}'' both caused a bit (say, index 9) to be set to \texttt{1}, and we reset it to \texttt{0} to delete ``\texttt{Rust}'', now:
\begin{itemize}
    \item When we query ``\texttt{Hello}'', it might show a \texttt{0} in one of its position.
    \item This creates a \textbf{false negative}, which violates one of the core guarantees of Bloom filters!
\end{itemize}
So, \textbf{manually unsetting bits can remove evidence of other elements}.

\highspace
\begin{flushleft}
    \textcolor{Green3}{\faIcon{check-circle} \textbf{Solution: Counting Bloom Filters}}
\end{flushleft}
To enable deletion, we \textbf{upgrade each bit into a counter}, this structure is called a \definition{Counting Bloom Filter}. It works like this:
\begin{itemize}
    \item Instead of a bit array, we use an \textbf{array of small integers}.
    \item When \textbf{inserting}, for each hash $h_{i}\left(x\right)$, increment \texttt{counter[$h_{i}\left(x\right)$]}.
    \item When \textbf{deleting}, for each hash $h_{i}\left(x\right)$, decrement \texttt{counter[$h_{i}\left(x\right)$]}.
\end{itemize}
We can safely decrement counters, knowing that \textbf{only when the last element that hashed to that index is deleted will the counter reach zero}. All previous analyses about false positives, FPR formula and $K$ optimal are still valid, but now we \textbf{use more memory} and \textbf{add increment/decrement logic}.

\highspace
\begin{flushleft}
    \textcolor{Red2}{\faIcon{exclamation-triangle} \textbf{Risk: Counter Overflow}}
\end{flushleft}
Counters must be large enough:
\begin{itemize}
    \item If they \textbf{overflow} (e.g., go above 255 for 8-bit counters), the \textbf{filter can become corrupted}.
    \item Worse, if a counter \textbf{underflows} (e.g., we delete too many times), we might \textbf{accidentally remove bits} for elements still in the set $\Rightarrow$ \textbf{false negatives}.
\end{itemize}
    \subsubsection{Invertible Bloom Lookup Tables (IBLTs)}

With Count Bloom Filters, we can:
\begin{itemize}
    \item Insert elements
    \item Delete them
    \item But we \textbf{can't list what's inside}, or \textbf{retrieve keys/values}, the information is ``smeared'' across the structure.
\end{itemize}
Now we want something more powerful that can also list all entries or recover a specific key-value pair.

\highspace
\begin{flushleft}
    \textcolor{Green3}{\faIcon{book} \textbf{What is an IBLT?}}
\end{flushleft}
An \definition{Invertible Bloom Lookup Table} is a data structure that:
\begin{itemize}
    \item Stores \textbf{key-value pairs}
    \item Supports \textbf{deletion} and \textbf{enumeration (listing)}
    \item Is inspired by Bloom Filters, but has a richer cell structure.
\end{itemize}
Each cell contains three values:
\begin{enumerate}
    \item \important{Count}: how many key-value pairs map to this cell.
    \item \important{KeySum}: XOR (or sum) of all keys that mapped here.
    \item \important{ValueSum}: XOR (or sum) of all values that mapped here.
\end{enumerate}
We hash the key using multiple hash functions, just like a Bloom filter, and update each corresponding cell.

\highspace
\begin{flushleft}
    \textcolor{Green3}{\faIcon{plus} \textbf{Insertion}}
\end{flushleft}
To \textbf{insert a key-value} pair:
\begin{enumerate}
    \item Use $K$ hash functions to map the key to $K$ cells.
    \item For each cell:
    \begin{enumerate}
        \item Increment the \textbf{count}
        \item Add the key to \textbf{KeySum}
        \item Add the value to \textbf{ValueSum}
    \end{enumerate}
\end{enumerate}

\newpage

\begin{flushleft}
    \textcolor{Green3}{\faIcon{minus} \textbf{Deletion}}
\end{flushleft}
To \textbf{delete a key-value} pair:
\begin{enumerate}
    \item Use the same $K$ hash functions.
    \item For each cell:
    \begin{itemize}
        \item \textbf{Decrement} the \textbf{count}
        \item \textbf{Subtract} the key from \textbf{KeySum}
        \item \textbf{Subtract} the value from \textbf{ValueSum}
    \end{itemize}
\end{enumerate}
If the key was inserted, this will perfectly remove it.

\highspace
\begin{flushleft}
    \textcolor{Green3}{\faIcon{search} \textbf{Lookup and Recovery}}
\end{flushleft}
To \textbf{find a value for a key}:
\begin{enumerate}
    \item Try to find a cell where \texttt{count == 1} and the \texttt{KeySum == input key}
    \item If found, then \texttt{ValueSum} gives the value associated with that key
\end{enumerate}
But:
\begin{itemize}
    \item If the key is mixed with other keys in all $K$ cells, recovery is hard.
    \item That's why \textbf{some keys may not be recoverable immediately}.
\end{itemize}

\highspace
\begin{flushleft}
    \textcolor{Green3}{\faIcon{question-circle} \textbf{Enumerate everything stored in it}}
\end{flushleft}
Once the structure is filled with multiple key-value pairs, we may want to enumerate everything stored in it, not just individual lookups. This process is known as \textbf{decoding} or \textbf{peeling} the IBLT. This restore operation is often used in real-world scenarios, for example, when we want to compare two sets of two different devices.

\highspace
The \textbf{decoding algorithm} is:
\begin{enumerate}
    \item Scan the table for a cell where:
    \begin{itemize}
        \item \texttt{count == 1}
        \item \texttt{KeySum} and \texttt{ValueSum} correspond to an actual key-value pair
    \end{itemize}
    \item When found:
    \begin{itemize}
        \item Add the pair to output
        \item Simulate deletion: subtract this key and value from all corresponding cells
        \item Update the IBLT
    \end{itemize}
\end{enumerate}

\newpage

\noindent
For example:
\begin{enumerate}
    \item Initial IBLT contains:
    \begin{table}[!htp]
        \centering
        \begin{tabular}{@{} c c c @{}}
            \toprule
            Count   & KeySum    & ValueSum  \\
            \midrule
            1       & 7         & 98        \\
            2       & 202       & 48        \\
            3       & 209       & 146       \\
            2       & 159       & 101       \\
            1       & 50        & 45        \\
            \bottomrule
        \end{tabular}
    \end{table}

    \item First, a cell with \texttt{count = 1} reveals:
    \begin{itemize}
        \item \texttt{(7, 98)} $\Rightarrow$ added to output
        \item Remove it from the IBLT (as if deleting it)
    \end{itemize}
    \begin{table}[!htp]
        \centering
        \begin{tabular}{@{} c c c @{}}
            \toprule
            Count   & KeySum    & ValueSum  \\
            \midrule
            \textbf{0}       & \textbf{0}         & \textbf{0}         \\
            2       & 202       & 48        \\
            \textbf{2}       & \textbf{202}       & \textbf{48}        \\
            \textbf{1}       & \textbf{152}       & \textbf{3}         \\
            1       & 50        & 45        \\
            \bottomrule
        \end{tabular}
    \end{table}

    \item After update:
    \begin{enumerate}
        \item Next, find \texttt{(152, 3)} $\Rightarrow$ decode and remove
        \begin{table}[!htp]
            \centering
            \begin{tabular}{@{} c c c @{}}
                \toprule
                Count   & KeySum    & ValueSum  \\
                \midrule
                0       & 0         & 0         \\
                \textbf{1}       & \textbf{50}        & \textbf{45}        \\
                \textbf{1}       & \textbf{50}        & \textbf{45}        \\
                \textbf{0}       & \textbf{0}         & \textbf{0}         \\
                1       & 50        & 45        \\
                \bottomrule
            \end{tabular}
        \end{table}
        \item Finally, we can easily retrieve the last one which is \texttt{50, 45}.
    \end{enumerate}

    \item The final result is:
    \begin{equation*}
        \left\{
            \left(\texttt{7, 98}\right), \: \left(\texttt{152, 3}\right), \: \left(\texttt{50, 45}\right)
        \right\}
    \end{equation*}
\end{enumerate}
This process works \textbf{only if at least one of the key-value pairs is initially recoverable}, and then the remaining pairs become recoverable as the IBLT gets simplified.

\newpage

\begin{flushleft}
    \textcolor{Red2}{\faIcon{exclamation-triangle} \textbf{Decoding problems}}
\end{flushleft}
Sometimes, the decoding process \textbf{gets stuck}:
\begin{itemize}
    \item All cells have \texttt{count > 1}, or are tangled with other keys
    \item We cannot isolate any key-value pair
\end{itemize}
When this happens:
\begin{itemize}
    \item Listing \textbf{fails}
    \item The \textbf{IBLT} is said to be in a \textbf{non-decodable state}
    \item This usually happens when the \textbf{load factor is too high} (i.e., too many elements for the number of cells)
\end{itemize}
So IBLTs are powerful because allowing insertion, deletion, lookup and enumeration; but we need to allocate enough space, because if we overloaded, we risk failure to decode.

\highspace
\begin{table}[!htp]
    \centering
    \begin{adjustbox}{width={\textwidth},totalheight={\textheight},keepaspectratio}
        \begin{tabular}{@{} l | c | c | c @{}}
            \toprule
            \textbf{Feature} & \textbf{Standard Bloom} & \textbf{Counting Bloom} & \textbf{IBLT} \\
            \midrule
            Insert               & \textcolor{Green3}{\faIcon{check}}               & \textcolor{Green3}{\faIcon{check}}    & \textcolor{Green3}{\faIcon{check}} (key-value)            \\
            Delete               & \textcolor{Red2}{\faIcon{times}}                 & \textcolor{Green3}{\faIcon{check}}    & \textcolor{Green3}{\faIcon{check}}                        \\
            Membership Test      & \textcolor{Green3}{\faIcon{check}} (yes/no)      & \textcolor{Green3}{\faIcon{check}}    & \textcolor{Green3}{\faIcon{check}} (via decoding)         \\
            False Negatives      & \textcolor{Red2}{\faIcon{times}}                 & \textcolor{Red2}{\faIcon{times}}      & \textcolor{Red2}{\faIcon{times}} (unless corrupted)       \\
            False Positives      & \textcolor{Green3}{\faIcon{check}}               & \textcolor{Green3}{\faIcon{check}}    & \textcolor{Red2}{\faIcon{times}} (when decoding works)    \\
            Listing Elements     & \textcolor{Red2}{\faIcon{times}}                 & \textcolor{Red2}{\faIcon{times}}      & \textcolor{Green3}{\faIcon{check}} (if decodable)         \\
            Memory Efficiency    & Very high                                        & Moderate                              & Lower (more fields)                                       \\
            \bottomrule
        \end{tabular}
    \end{adjustbox}
    \caption{IBLT vs Bloom Filters.}
\end{table}
    \subsubsection{Count-Min Sketch}

The \definition{Count-Min Sketch} is a \textbf{probabilistic data structure} used to estimate the \textbf{frequency of elements} in a stream.
\begin{itemize}
    \item We don't store each element individually.
    \item Instead, we use a \textbf{compat structure to maintain approximate\break counts}.
    \item It's designed for efficiency, especially when tracking millions of elements would be too memory-intensive.
\end{itemize}

\highspace
\begin{flushleft}
    \textcolor{Green3}{\faIcon{tools} \textbf{How does it work?}}
\end{flushleft}
We create a 2D array of counters with:
\begin{itemize}
    \item $d$ rows (one per hash function)
    \item $w$ columns (size of each hash domain)
\end{itemize}
This gives a table of size $w \times d$, much smaller than a full hash table for all possible items. \textbf{Each row} has a \textbf{different hash function}.

\highspace
\begin{flushleft}
    \textcolor{Green3}{\faIcon{plus} \textbf{Insertion}}
\end{flushleft}
To insert an element (e.g. \texttt{ip.dest1}):
\begin{enumerate}
    \item \textbf{Hash the element} with each of the $d$ hash functions.
    \item \textbf{Each hash} gives we a \textbf{column index in its row}.
    \item \textbf{Increment} the corresponding \textbf{counters}.
\end{enumerate}
So we increment \textbf{1 counter per row}, total $d$ counters updated.

\highspace
\begin{flushleft}
    \textcolor{Green3}{\faIcon{search} \textbf{Querying the Frequency}}
\end{flushleft}
To estimate the count of an element:
\begin{enumerate}
    \item Hash it again with the same $d$ hash functions.
    \item Get the counter values from the same positions.
    \item \textbf{Return the minimum} of those $d$ counters.
\end{enumerate}
The \textbf{minimum} because:
\begin{itemize}
    \item Collisions with other elements can cause overestimation (counters get inflated).
    \item But the \textbf{minimum is never less than the true count}, so it's a \textbf{safe lower bound}.
\end{itemize}
That's where the name comes from: count, because it estimates the frequency, and min, because it takes the minimum over multiple counters.

\highspace
\begin{flushleft}
    \textcolor{Green3}{\faIcon{check-circle} \textbf{Advantages}}
\end{flushleft}
\begin{itemize}
    \item Sublinear space: uses \textbf{much less memory} than a full table.
    \item \textbf{Fast}: insertions and queries are both $O\left(d\right)$ time (constant if $d$ is fixed).
    \item Suitable for \textbf{high-speed data streams} (e.g., network flows, telemetry, monitoring).
\end{itemize}

\highspace
\begin{table}[!htp]
    \centering
    \begin{tabular}{@{} l | l @{}}
        \toprule
        \textbf{Feature} & \textbf{Value} \\
        \midrule
        Use case            & Approximate frequency counts \\ [.3em]
        Memory              & Sublinear ($ w \times d $)   \\ [.3em]
        Insertion time      & $ O(d) $                     \\ [.3em]
        Query time          & $ O(d) $, returns minimum    \\ [.3em]
        Overestimates       & Possible                     \\ [.3em]
        Underestimates      & Never                        \\ [.3em]
        Similar to          & Counting Bloom Filter        \\
        \bottomrule
    \end{tabular}
    \caption{Count-Min Sketch summary.}
\end{table}

    %%%%%%%%%%%%%%%%%%%%%%%%%
    % Datacenter Monitoring %
    %%%%%%%%%%%%%%%%%%%%%%%%%
    \section{Datacenter Monitoring}

\subsection{Why Datacenter Monitoring Matters}

Image we're running a distributed application in a datacenter, and performance suddenly degrades. The possible root causes can be multiple:
\begin{itemize}
    \item A software bug in the application logic.
    \item Network congestion between the servers.
    \item A broken fiber cable disrupting communication.
    \item A hardware failure, e.g. broken switch.
    \item A network misconfiguration that reroutes traffic inefficiently.
    \item A bug in the routing protocol.
    \item And many more...
\end{itemize}
There is a huge space of possible issues, and \textbf{pinpointing the root cause without visibility is extremely difficult}.

\highspace
Many papers describe the importance of monitoring in datacenters.
\begin{itemize}
    \item In the \emph{Pingmesh}\cite{10.1145/2829988.2787496} article, they point to research that has begun to investigate \textbf{how to distinguish network problems from application-level bugs}. They highlights the diagnostic ambiguity in complex systems. Without monitoring, it's extremely hard to tell whether a slowdown is due to:
    \begin{itemize}
        \item Software bugs;
        \item Application overload;
        \item Or actual network failures.
    \end{itemize}
    \textbf{Monitoring systems must disambiguate the root cause across layers}, application vs network.


    \item In the ``\emph{Understanding and Mitigating Packet Corruption in Data Center Networks}''\cite{10.1145/3098822.3098849} article, they showing how \textbf{minor misconfiguration or failures} (e.g., wrong routing entry) can ripple through a system, \textbf{creating major outages}. It stresses that even low-level, seemingly unimportant events mu be visible to prevent or debug large-scale issues. For example, a single corrupted forwarding rule in a switch might cause traffic loss affecting thousands of users.

    \textbf{Monitoring must include fine-grained data} (like per-packet or per-flow telemetry) \textbf{to detect these small but critical problems}.


    \item In the ``\emph{Flow Event Telemetry on Programmable Data Plane}''\cite{10.1145/3387514.3406214} article, they show that \textbf{performance degradation often happens silently}, with no clear immediate failures. These ``gray failures'' don't crash systems but hurt performance. They're invisible without high-resolution monitoring (latency histograms, queue lengths, retransmits, etc.).
    
    \textbf{Monitoring should detect subtle deviations}, not just crashes or timeouts.


    \item In the \emph{CloudCluster}\cite{276948} article, they push toward deep programmability and visibility withing the network. This points to the \textbf{evolution of monitoring tools}:
    \begin{itemize}
        \item From passive logs and SNMP stats;
        \item To programmable packet tracing and real-time telemetry;
        \item That help pinpoint network issues quickly and accurately.
    \end{itemize}
    \textbf{Visibility must be deep, dynamic, and distributed across the system}.
\end{itemize}
    \subsection{Network Monitoring}

\definition{Network Monitoring} is the \textbf{continuous observation of a computer network to detect slowdowns or failures in components}. Its purpose is to detect, localize and respond to faults before they impact users.

\highspace
\begin{flushleft}
    \textcolor{Green3}{\faIcon{question-circle} \textbf{Monitoring Scope}}
\end{flushleft}
Monitoring spans the \textbf{entire network path}. Each router (or switch/server) is a point where failures or slowdowns can occur:
\begin{itemize}
    \item A switch could drop the packet silently.
    \item A routing issue could cause the packet to loop.
    \item A delay could occur due to congestion in queues.
\end{itemize}
\textbf{To effectively detect and diagnose problems}, the monitoring system must observe not just endpoints, but the entire path, or at least enough of it to: detect \emph{where} things go wrong, or understand \emph{why} a packet failed to reach its destination. This is why datacenter monitoring often tries to trace or mirror packets at different points in the network, to reconstruct the packet's journey and find anomalies.

\highspace
\begin{flushleft}
    \textcolor{Green3}{\faIcon{tools} \textbf{Monitoring Techniques}}
\end{flushleft}
There are many ways to monitor a network. It can be done:
\begin{enumerate}
    \item \textbf{From Switches}:
    \begin{enumerate}
        \item \textbf{Built-in Features}
        \begin{itemize}
            \item \emph{NetFlow}: collects IP traffic statistics.
            \item \emph{Mirroring}: duplicates selected packets for analysis.
            \item \emph{SNMP (Simple Network Management Protocol)}: polls device stats.
        \end{itemize}
        \item \textbf{Programmable Switches}
        \begin{itemize}
            \item Use \emph{data plane programmability} (e.g., P4 language) to define custom monitoring behaviors.
            \item Enables \emph{custom counters}, tagging, filtering, or tracing at wire speed.
        \end{itemize}
    \end{enumerate}
    
    \item \textbf{From Servers}:
    \begin{enumerate}
        \item \textbf{Standard Tools}
        \begin{itemize}
            \item \texttt{netstat}: network connections and stats.
            \item \texttt{tcpdump}: packet capture and inspection.
            \item \texttt{traceroute}: path tracing and latency.
        \end{itemize}
        \item \textbf{Ad-hoc Monitoring Services}
        \begin{itemize}
            \item Lightweight daemons or agents tailored for the datacenter.
            \item Export performance metrics or send alerts.
        \end{itemize}
    \end{enumerate}
\end{enumerate}
There is no single way to monitor, a \textbf{mix of passive and active, centralized and distributed methods is used}. Monitoring systems must collect data from multiple vantage points to build a full picture of the network's health.

\highspace
\begin{flushleft}
    \textcolor{Red2}{\faIcon{exclamation-triangle} \textbf{Why traditional monitoring isn't enough}}
\end{flushleft}
In large-scale datacenters many failures are subtle and effect only specific flows of packets:
\begin{itemize}
    \item \important{Silent packet drops}. Packets are dropped but not reported by switches. The causes are software bugs or faulty hardware.
    \item \important{Silent blackholes}. Traffic is blackholed without showing in forwarding tables. The causes are corrupted TCAM entries.
    \item \important{Inflated end-to-end latency}. Packet flow experiences unexpected delays. The causes are congestion or queuing.
    \item \important{Loops}. Packets circulate endlessly. The causes are middleboxes modify headers or breaking routing logic.
\end{itemize}
These failures are:
\begin{itemize}
    \item \textbf{Not visible} in flow-level stats.
    \item \textbf{Not logged} by switches.
    \item \textbf{Hard to localize} with only endpoint observations.
\end{itemize}
\textcolor{Green3}{\faIcon{check-circle}} We need \textcolor{Green3}{\textbf{per-packet visibility}} to detect and understand them.

\highspace
\begin{flushleft}
    \textcolor{Green3}{\faIcon{question-circle} \textbf{So can we monitor every packet on the network?}}
\end{flushleft}
Tracing all packets in large datacenters is not scalable:
\begin{itemize}
    \item Aggregate traffic can exceed 100 terabit per seconds.
    \item Microsoft estimated 3200 servers needed just to collect and analyze th data (in 2015).
\end{itemize}
To make packet-level telemetry practical, some strategies are required:
\begin{enumerate}
    \item Monitoring must be \textbf{selective and smart} (e.g., sample important flows).
    \item Diagnosing problems  often requires \textbf{correlating behaviors across multiple hops}.
    \item \textbf{Passive tracing alone is insufficient}:
    \begin{itemize}
        \item It may miss transient problems.
        \item It lacks the context to localize root causes effectively.
    \end{itemize}
\end{enumerate}
    \subsection{Everflow}\label{subsection: Everflow}

\subsubsection{What is Everflow?}

\definition{Everflow}\cite{greenberg2015packet-level} is Microsoft's system for \textbf{packet-level telemetry in production datacenters}, and it is built around three key concepts:
\begin{enumerate}
    \item \important{Match and Mirror on the Switch}. Everflow leverages the match-action capability of commodity switches. It \textbf{defines rules to match specific packets and then mirror (copy) them to a monitoring collector}. Three matching rules:
    \begin{itemize}
        \item \texttt{TCP SYN / FIN / RST}: to trace connection setup/teardown.
        \item \textbf{Special debug bit}: used to flag packets for tracing.
        \item \textbf{Protocol traffic}: such as BGP or other control plane packets.
    \end{itemize}
    This allows the system to \textbf{monitor important or suspicious traffic patterns without touching every packet}.
    
    
    \item \important{Switch-Based Reshuffler}. Mirroring packets from all switches generates huge data volumes. A single analysis server can't handle this load. The solution is to \textbf{use one or more intermediate switches} (reshuffler) that:
    \begin{enumerate}
        \item Receive mirrored packets.
        \item Distribute them intelligently across multiple collectors.
    \end{enumerate}
    This \textbf{balances load and scales the telemetry infrastructure}.
    

    \item \important{Guided Probing}. The system can \textbf{inject specific test packets into the network}. These packets are crafted to \textbf{explore or verify behaviors} (e.g., path correctness, loss, latency). They are useful because:
    \begin{itemize}
        \item Helps when match and mirror alone misses packets (e.g., for complete TCP flow analysis).
        \item Can reproduce or test suspected failures.
        \item Distinguishes between persistent and transient issues.
    \end{itemize}
    Uses DSCP bits (in IP headers) and parts of the IPID field to mark and sample packets.
\end{enumerate}

\begin{table}[!htp]
    \centering
    \begin{adjustbox}{width={\textwidth},totalheight={\textheight},keepaspectratio}
        \begin{tabular}{@{} l | l | p{15em} @{}}
            \toprule
            Idea & Purpose & Key Technique \\
            \midrule
            \textbf{Match and Mirror}   & Capture relevant packets          & Matching on SYN / FIN / debug bits; mirroring to collectors. \\ [.5em]
            \textbf{Reshuffler}         & Scale analysis                    & Distribute mirrored packets across servers. \\ [.5em]
            \textbf{Guided Probing}     & Actively test network behavior    & Inject custom packets using special bit fields. \\
            \bottomrule
        \end{tabular}
    \end{adjustbox}
    \caption{Summary of Everflow concepts.}
\end{table}

    \subsection{How it works}

The multigrid method is divided into \textbf{seven parts} that make the MG method work.
\begin{enumerate}
    \item \textbf{Coarse Grids} (page \pageref{subsubsection: Coarse Grids})
    \item \textbf{Correction} (page \pageref{subsubsection: Correction})
    \item \textbf{Interpolation Operator} (page \pageref{subsubsection: Interpolation Operator})
    \item \textbf{Restriction Operator} (page \pageref{subsubsection: Restriction Operator})
    \item \textbf{Two-Grid Scheme} (page \pageref{subsubsection: Two-Grid Scheme})
    \item \textbf{V-Cycle Scheme} (page \pageref{subsubsection: V-Cycle Scheme})
\end{enumerate}

\noindent
These elements \textbf{work together to handle errors at different scales}, making the method \textbf{highly effective for solving large and complex systems} of linear equations.

\highspace
Note that this \textbf{is \underline{not} an algorithm!} We can think of the MG method as a toolbox filled with powerful tools, each designed to address different aspects of solving complex problems efficiently.

\begin{flushleft}
    \textcolor{Green3}{\faIcon{square-root-alt} \textbf{Notation used in MG methods}}
\end{flushleft}
We will use the subscript $h$ to indicate the Grid Spacing. The variable $h$ represents the \textbf{distance between two successive grid points on the fine grid}. For example, if the domain is divided into $N$ intervals, the grid spacing $h$ is typically $\frac{1}{N}$.
\begin{itemize}
    \item \textbf{Residual} $\mathbf{r}_{h}$ represents the residual calculated on the fine grid with spacing $h$. It's the difference between the current solution and the exact solution on this grid.

    \item \textbf{Solution} $\mathbf{x}_{h}$ indicates the approximate solution on the fine grid. This solution is updated iteratively using the Multigrid method.
    
    \item \textbf{Operator} $A_{h}$ is the matrix or operator that represents the system of equations on the fine grid. This operator acts on the solution $\mathbf{x}_{h}$.
    
    \item \textbf{Right-Hand Side} $\mathbf{b}_{h}$ is the right-hand side vector of the system of equations on the fine grid. It's what the solution $\mathbf{x}_{h}$ should ideally satisfy when acted upon by $A_{h}$.
    
    \item \textbf{Error on the Fine Grid} $\mathbf{e}_{h}$ is the error estimate or correction term calculated on the fine grid. It represents the difference between the true solution and the current approximate solution on the fine grid with spacing $h$.
\end{itemize}
What's more, \textbf{if we move to a coarser grid}, the grid spacing will be greater, indicated by $2h$ or even $4h$ if we make a significant jump to a coarser level.
    \subsection{Architecture}

At a high level, the architecture of Fastpass is quite simple.

\begin{figure}[!htp]
    \centering
    \includegraphics[width=.7\textwidth]{img/fastpass-arch-2.pdf}
    \caption{Architecture of Fastpass.\cite{10.1145/2740070.2626309}}
\end{figure}

\noindent
It has five main components:
\begin{itemize}
    \item \important{Endpoints} are normal servers that send and receive traffic, placed in the datacenter. However, unlike TCP:
    \begin{itemize}
        \item[\textcolor{Red2}{\faIcon{times}}] They \textbf{do not freely send packets} to the network.
        \item[\textcolor{Green3}{\faIcon{check}}] They must first request permission from the arbiter.
    \end{itemize}
    So the workflow is:
    \begin{enumerate}
        \item Endpoint has a packet to send.
        \item It sends a request to the arbiter.
        \item Arbiter assigns: a transmission timeslot and a path to the destination.
        \item Endpoint sends only in that assigned timeslot, and only on that assigned path.
    \end{enumerate}
    In this way, the endpoints are \textbf{obedient transmitter}, not congestion controllers. However, \hl{the endpoints must be modified (kernel modifications) to implement this behavior. Instead, the switches and the network are completely unmodified.} We will discuss this in more detail in the FCP section.
    
    
    \item \important{Central Arbiter} is the \textbf{brain} of Fastpass. Its responsibilities are:
    \begin{itemize}
        \item Maintain a global view of who wants to send what, and the network topology (i.e., the state of the network).
        \item Compute matchings for each timeslot, i.e., which endpoints can transmit at the same time without causing congestion.
        \item Assign paths to the endpoints, i.e., which path they should use to send their packets.
        \item Ensure no link conflict occurs, i.e., that no two endpoints are assigned to transmit on the same link at the same time.
    \end{itemize}
    It runs as a \textbf{software service} on a \textbf{commodity server} (i.e., it is not a specialized hardware component). The software is \textbf{highly optimized} to make fast decisions, and it is \textbf{multi-core scalable} to handle a large number of requests.
    
    
    \item \important{Timeslot Allocator}. The \hl{time is divided into fixed-size slots.} Each slot corresponds to one MTU (Maximum Transmission Unit)\footnote{
        MTU is the maximum size of a packet that can be transmitted on the network. For example, if the MTU is 1500 bytes, then each timeslot is long enough to transmit one 1500-byte packet.
    } transmission time on the slowest link in the network.
    
    For each timeslot, the arbiter chooses a \textbf{set of endpoints} (sender $\to$ receiver pairs) such that:
    \begin{itemize}
        \item Each sender sends at most 1 packet.
        \item Each receiver receives at most 1 packet.
    \end{itemize}
    For example, suppose we have 3 senders ($A$, $B$, $C$) and 3 receivers ($X$, $Y$, $Z$). In one timeslot, the arbiter might assign:
    \begin{itemize}
        \item $A$ wants to send to $X$.
        \item $B$ wants to send to $Z$.
        \item $C$ wants to send to $Y$.
    \end{itemize}
    Although this \textbf{constraint ensures there are no conflicts at the endpoints}, it does not guarantee there are no conflicts in the network (i.e., two paths may share a link). This is where the path selector comes in.
    
    
    \item \important{Path Selector} is \textbf{responsible for assigning a path to each sender-receiver pair selected by the timeslot allocator}. The path selector must ensure that \hl{no two packets use the same internal link in same timeslot}, otherwise there would be a conflict and congestion would occur. For example, if $A$ sends to $X$ and $B$ sends to $Z$, and both paths share a link, then there would be a conflict on that link.
    
    
    \item \important{Fastpass Control Protocol (FCP)} is the \textbf{protocol used by the endpoints to communicate with the arbiter} and vice versa. It is a \textbf{lightweight protocol} designed for low latency and high throughput. It is \hl{implemented in the kernel of the endpoints}, and it allows them to:
    \begin{itemize}
        \item Send requests to the arbiter when they have packets to send.
        \item Receive assignments from the arbiter (timeslot and path).
        \item Transmit packets according to the assignments.
    \end{itemize}
    The FCP is designed to be \textbf{simple and efficient}, with minimal overhead, to ensure that the communication between endpoints and arbiter does not become a bottleneck.
\end{itemize}
    \subsubsection{Data Structure used in FlowRadar}

\textbf{Switches have very limited memory}, but we want to count \textbf{how many packets are part of each individual flow}. Instead of using a separate counter per flow, which would consume too much space, FlowRadar stores \textbf{compressed aggregate information} in a fixed-size structure.

\highspace
Each switch maintains three tables of $m$ cells:
\begin{itemize}
    \item \definition{FlowXOR}. XOR\footnote{%
        XOR (Exclusive OR) is a binary operation denoted by $\otimes$, where the result is 1 if the two input bits are different, and 0 if they are the same.
    } of all flow IDs hashed into each cell.
    \item \definition{FlowCount}. Number of flows that map to each cell.
    \item \definition{PacketCount}. Total number of packets from those flows.
\end{itemize}
For example, assume flows A, B, and C are all seen by the switch. Each flow is hashed into multiple cells. The tables get updated like this:
\begin{itemize}
    \item FlowXOR: $a \otimes b$, $b \otimes c$, etc.
    \item FlowCount: how many flows are hashed into each cell (1, 2, etc.)
    \item PacketCount: total packets seen in each cell (e.g., $S(a) + S(b)$)
\end{itemize}
So each cell contains a \textbf{mix of data from different flows}.
\begin{figure}[!htp]
    \centering
    \includegraphics[width=\textwidth]{img/flowradar-1.pdf}
\end{figure}

\highspace
\begin{flushleft}
    \textcolor{Green3}{\faIcon{question-circle} \textbf{Why use XOR?}}
\end{flushleft}
Because the XOR operation is \textbf{reversible}. If we know the XOR of two values and one of them, we can recover the other. This \textbf{allows the collector to decode the original flows} by:
\begin{enumerate}
    \item Getting reports from multiple switches.
    \item Iteratively solving the system of XOR equations.
\end{enumerate}

\highspace
\begin{flushleft}
    \textcolor{Red2}{\faIcon{exclamation-triangle} \textbf{What is Flow Filter and why do we need it?}}
\end{flushleft}
The \definition{Flow Filter} is a small data structure (like a Bloom Filter) used inside the switch to \textbf{remember which flows the switch has already seen}.

\highspace
Let's say flow $a$ sends 10 packets. All those packets will pass through the switch. But we don't want to treat each packet like a new flow, \textbf{we only want to register flow $a$ once in the compressed counters}. If we update the XOR and FlowCount on every packet:
\begin{itemize}
    \item The \textbf{FlowXOR} would get corrupted.
    \item The \textbf{FlowCount} would become too high.
    \item We'd \textbf{lose the ability to decode} the flows correctly later.
\end{itemize}
So the \textbf{Flow Filter helps us avoid this}.

\highspace
\begin{flushleft}
    \textcolor{Green3}{\faIcon{question-circle} \textbf{What does Flow Filter actually do?}}
\end{flushleft}
For each packet:
\begin{enumerate}
    \item The switch looks at the Flow ID (e.g., \texttt{source IP + dest IP + ports}).
    \item It checks the Flow Filter:
    \begin{itemize}
        \item If the flow is \textbf{new} (not in the filter yet):
        \begin{enumerate}
            \item It updates:
            \begin{itemize}
                \item FlowXOR: add this flow's ID via XOR.
                \item FlowCount: increment the count.
                \item PacketCount: add 1.
            \end{itemize}
            \item It marks this flow as \emph{seen} in the filter.
        \end{enumerate}
        \item If the flow is \textbf{already known}:
        \begin{enumerate}
            \item It updates \textbf{only PacketCount}.
        \end{enumerate}
    \end{itemize}
\end{enumerate}

\begin{figure}[!htp]
    \centering
    \includegraphics[width=\textwidth]{img/flowradar-2.pdf}
    \caption{Correct representation of the data structure used in FlowRadar.}
\end{figure}
    \subsubsection{Collector Decode}

Once the switch sends its tables (FlowXOR, FlowCount, PacketCount) to the collector, there are \textbf{two stages of decoding}:
\begin{enumerate}
    \item \textbf{Single Decode} (local), performed at the collector level, recovers flows where the data structure is clean enough.
    \item \textbf{Network-Wide Decode} (distributed), performed across multiple collectors/switches, combines views from many switches to fully decode the remaining flows.
\end{enumerate}

\highspace
\begin{flushleft}
    \textcolor{Green3}{\faIcon{book} \textbf{Step 1 - Single Decode}}
\end{flushleft}
\definition{Single Decode} is the \textbf{local decoding stage}, it happens at one switch (or collector handling one switch's data). The goal is to \textbf{recover as many flows as possible} using \textbf{only the local FlowXOR, FlowCount, and PacketCount tables} collected from that switch. The key idea is to find ``pure'' cells (cells containing information from only one stream) and start the decoding process:
\begin{enumerate}
    \item \important{Find a Pure Cell}. A cell is \textbf{pure} if \texttt{FlowCount = 1}. It means only one flow was hashed to that cell. For example, let the following FlowRadar table, this step identifies the first row:
    \begin{table}[!htp]
        \centering
        \begin{tabular}{@{} c | c | c @{}}
            \toprule
            \texttt{FlowXOR} & \texttt{FlowCount} & \texttt{PacketCount} \\
            \midrule
            $a$ & 1 & 5 \\
            $a \otimes b$ & 2 & 12 \\
            $b \otimes c \otimes d$ & 3 & 13 \\
            0 & 0 & 0 \\
            0 & 0 & 0 \\
            $b \otimes c \otimes d$ & 3 & 13 \\
            0 & 0 & 0 \\
            $a$ & 1 & 5 \\
            0 & 0 & 0 \\
            $c \otimes d$ & 2 & 6 \\
            \bottomrule
        \end{tabular}
    \end{table}

    
    \newpage


    \item \important{Remove the Flow's Contribution from Other Cells}. Flow $a$ was hashed to multiple cells. So now we remove $a$'s effect from all its associated cells:
    \begin{table}[!htp]
        \centering
        \begin{tabular}{@{} c | c | c @{}}
            \toprule
            \texttt{FlowXOR} & \texttt{FlowCount} & \texttt{PacketCount} \\
            \midrule
            0 & 0 & 0 \\
            $b$ & 1 & 7 \\
            $b \otimes c \otimes d$ & 3 & 13 \\
            0 & 0 & 0 \\
            0 & 0 & 0 \\
            $b \otimes c \otimes d$ & 3 & 13 \\
            0 & 0 & 0 \\
            0 & 0 & 0 \\
            0 & 0 & 0 \\
            $c \otimes d$ & 2 & 6 \\
            \bottomrule
        \end{tabular}
    \end{table}

    
    \item \important{Removal may produce purer cells}. By removing flow $a$, other cells might now have \texttt{FlowCount = 1} (pure cell). Repeat the previous steps until everything is decoded.

    \textcolor{Red2}{\faIcon{exclamation-triangle} \textbf{Possible Stall}}. Some flows are mixed together in such a way that no cell has \texttt{FlowCount = 1}. The solution here is to \textbf{apply} the second decoding stage, called \textbf{network-wide decode}.
\end{enumerate}

\highspace
\begin{flushleft}
    \textcolor{Green3}{\faIcon{check-circle} \textbf{Step 2 - Network-Wide Decode}}
\end{flushleft}
In the previous stage, each switch tries to decode as many flows as it can \textbf{locally}, by identifying \emph{pure} cells. But sometimes decoding gets stuck because:
\begin{itemize}
    \item Flow cells contain multiple flow mixed.
    \item No pure cells remain.
\end{itemize}
To solve this, we use \textbf{network-wide redundancy}: packets of the \textbf{same flow traverse multiple switches}, and those switches may store \textbf{different parts} of the encoded data. By combining these views, we can \textbf{solve flows that are undecodable at any single switch}.

\highspace
For example, image the following situation:
\begin{table}[!htp]
    \centering
    \begin{adjustbox}{width={\textwidth},totalheight={\textheight},keepaspectratio}
        \begin{tabular}{@{} c | c | c | c | c | c | c @{}}
            \toprule
            \multicolumn{3}{c|}{Switch 1} & & \multicolumn{3}{c}{Switch 2} \\
            \midrule
            \texttt{FlowXOR} & \texttt{FlowCount} & \texttt{PacketCount} & & \texttt{FlowXOR} & \texttt{FlowCount} & \texttt{PacketCount} \\
            \midrule
            $a$ & 1 & $-$ & & $a \otimes d$ & 2 & $-$ \\
            $a \otimes c \otimes d$ & 3 & $-$ & & $a \otimes c$ & 2 & $-$ \\
            $b \otimes c \otimes d$ & 3 & $-$ & & $b \otimes c \otimes d$ & 3 & $-$ \\
            $a \otimes b \otimes c$ & 3 & $-$ & & $a \otimes b \otimes c$ & 3 & $-$ \\
            $b \otimes d$ & 2 & $-$ & & $b \otimes d$ & 2 & $-$ \\
            \bottomrule
        \end{tabular}
    \end{adjustbox}
\end{table}

\noindent
In both switches, single decode fails. But when \textbf{merged}, we have enough constraints to decode all flows. This is similar to solving a system of equations with more equations that unknowns. In other words, we don't need massive memory in each switch, we can use \textbf{small, compressed flow encodings per switch} and decode everything later by \textbf{combining views across the network}.

\highspace
\begin{flushleft}
    \textcolor{Green3}{\faIcon{question-circle} \textbf{But how do we know which switch saw which flow?}}
\end{flushleft}
To decode accurately, we must know which switch processed which flow, because otherwise we might combine FlowXORs from switches that didn't see a particular flow (wrong result). The \textbf{solution is the Flow Filter}. Each switch has a Flow Filter, so we can:
\begin{enumerate}
    \item Query the flow filter: ``Did you see the flow $a$?''
    \item If yes, we use that switch's data for decoding flow $a$.
\end{enumerate}
This \textbf{guarantees correctness in multi-switch decoding}.

\highspace
\begin{table}[!htp]
    \centering
    \begin{adjustbox}{width={\textwidth},totalheight={\textheight},keepaspectratio}
        \begin{tabular}{@{} l | l @{}}
            \toprule
            \textbf{Step} & \textbf{Description} \\
            \midrule
            1. & Each switch reports its compressed counters (FlowXOR, FlowCount, PacketCount) \\ [.3em]
            2. & Some flows decoded via Single Decode \\ [.3em]
            3. & Remaining flows are solved by combining equations from multiple switches \\ [.3em]
            4. & Use Flow Filters to determine which switch saw which flows \\ [.3em]
            5. & Network-wide correlation fully decodes the remaining flows \\
            \bottomrule
        \end{tabular}
    \end{adjustbox}
    \caption{Network-Wide Decode summary.}
\end{table}

\highspace
\begin{flushleft}
    \textcolor{Red2}{\faIcon{exclamation-triangle} \textbf{What if Switches Disagree Due to Packet Loss?}}
\end{flushleft}
Image that:
\begin{itemize}
    \item A packet of flow $f$ passes through Switch A and Switch B.
    \item Due to \textbf{transient issue}, one of the switches \textbf{misses that packet} (e.g., due to mirroring loss or memory overwrite).
\end{itemize}
Now, when both switches report their Flow Radar data structure:
\begin{itemize}
    \item The values for flow $f$ might not match across switches.
    \item This creates \textbf{inconsistencies} when trying to decode.
\end{itemize}
\textcolor{Green3}{\faIcon{check-circle} \textbf{Solution: Redundancy}}. Even if packet counts differ, the set of flows seen by each switch can still be decoded. And more importantly, once we know which flows each switch saw, we can treat each switch's data as a system of linear equations and solve for the actual packet counts.
\begin{enumerate}
    \item We \textbf{use the Flow Filter} to determine which flows each switch saw.
    \item We \textbf{decode flow IDs} using the FlowXOR and FlowCount tables.
    \item We set up a \textbf{linear system of equations} per switch:
    \begin{itemize}
        \item Each \textbf{cell} gives us an equation:
        \begin{equation*}
            \texttt{XOR}\left(f_{1}, f_{2}, \dots\right) \Rightarrow \text{total packet count} = P
        \end{equation*}
        \item We solve for the \textbf{unknown packet counts} of individual flows.
    \end{itemize}
    \item If the same flow has \textbf{different counts} across switches:
    \begin{itemize}
        \item It signals a possible \textbf{packet loss};
        \item Or a measurement \textbf{inconsistency}
    \end{itemize}
\end{enumerate}
    \subsection{In-Band Network Telemetry (INT)}

\subsubsection{What is INT?}

\definition{In-Band Network Telemetry} is a framework where the \textbf{data plane itself collects telemetry information} as packets traverse the network. Instead of relying on mirrored copies or external probes (like Everflow or FlowRadar), INT \textbf{embeds telemetry instructions directly into packets}. This means that the \textbf{packet \emph{asks} switches along its path to record certain metadata} (e.g., delay, queue size, switch ID).

\highspace
INT demonstrated the power and usefulness of programmable switches. It was one of the first real-world use cases where P4 offered something no traditional switch could do.

\highspace
\begin{flushleft}
    \textcolor{Green3}{\faIcon{tools} \textbf{How it works?}}
\end{flushleft}
\begin{enumerate}
    \item Packets carry \textbf{INT headers}.
    \item INT-capable devices \textbf{read the instructions} in those headers.
    \item They \textbf{\emph{collect}} specific \textbf{network state} and \textbf{\emph{append} it to the packet} as it moves.
    \item The telemetry data is delivered \textbf{in-band}, alongside the normal traffic.
\end{enumerate}
The data collected cloud include: switch ID, input and output ports, queue occupancy, timestamp (arrival and departure), packet latency per hop.

\highspace
\begin{flushleft}
    \textcolor{Green3}{\faIcon{question-circle} \textbf{Why INT?}}
\end{flushleft}
INT \textbf{solves key limitations of older monitoring tools}:
\begin{itemize}[label=\textcolor{Green3}{\faIcon{check}}]
    \item \textbf{No} need for \textbf{mirrored} traffic (like Everflow).
    \item \textbf{Real-time per-packet} information.
    \item \textbf{High visibility} into what happens at every hop.
\end{itemize}
This is especially useful for: fine-grained performance monitoring; diagnostic congestion, jitter, and path problems; dynamic traffic engineering.
    \subsubsection{Modes}

Telemetry \textbf{data can be collected and exported in different ways}, depending on:
\begin{itemize}
    \item Whether packets are modified or untouched
    \item Whether data is inserted in packets or sent to collectors separately
    \item How much pressure is put on switches vs collectors.
\end{itemize}
Each mode offers \hl{trade-offs} between \textbf{performance}, \textbf{visibility}, and \textbf{system overhead}. There are three different modes:
\begin{enumerate}
    \item \definition{INT-XD (eXport Data)}. Switches do \textbf{not modify the packet}. Instead, they \textbf{send telemetry} data \textbf{directly to the collector} based on local configuration.
    \begin{itemize}
        \item[\textcolor{Green3}{\faIcon{check-circle}}] \textcolor{Green3}{\textbf{Pros}}
        \begin{itemize}[label=\textcolor{Green3}{\faIcon{check}}]
            \item \textbf{No changes} to the packet $\Rightarrow$ avoids MTU issues.
            \item Easier for legacy packet flows.
        \end{itemize}

        \item[\textcolor{Red2}{\faIcon{times-circle}}] \textcolor{Red2}{\textbf{Cons}}
        \begin{itemize}[label=\textcolor{Red2}{\faIcon{times}}]
            \item Heavy \textbf{pressure on collectors} (they must gather data from all switches).
            \item Telemetry query is based on \textbf{switch config}, not what the packet asks.
        \end{itemize}
    \end{itemize}

    \item \definition{INT-MX (eMbed instruct(X)ions)}. The \textbf{packet is marked} to indicate it wants telemetry. Switches \textbf{send telemetry data out-of-band} (to a collector), but \textbf{based on the packet's mark}.
    \begin{itemize}
        \item[\textcolor{Green3}{\faIcon{check-circle}}] \textcolor{Green3}{\textbf{Pros}}
        \begin{itemize}[label=\textcolor{Green3}{\faIcon{check}}]
            \item Telemetry is \textbf{packet-driven}, more dynamic than INT-XD.
        \end{itemize}

        \item[\textcolor{Red2}{\faIcon{times-circle}}] \textcolor{Red2}{\textbf{Cons}}
        \begin{itemize}[label=\textcolor{Red2}{\faIcon{times}}]
            \item Still puts \textbf{pressure on collectors}.
            \item Requires \textbf{modifying packets}, could affect headers, MTU.
        \end{itemize}
    \end{itemize}

    \item \definition{INT-MD (eMbed Data)}. The \textbf{packet itself is modified} to carry Telemetry metadata in-band. \textbf{Each switch inserts data into the packet} as it passes through.
    \begin{itemize}
        \item[\textcolor{Green3}{\faIcon{check-circle}}] \textcolor{Green3}{\textbf{Pros}}
        \begin{itemize}[label=\textcolor{Green3}{\faIcon{check}}]
            \item \textbf{No extra pressure} on collectors.
            \item Telemetry query is \textbf{packet-dependent}, enabling full visibility along the path.
        \end{itemize}

        \item[\textcolor{Red2}{\faIcon{times-circle}}] \textcolor{Red2}{\textbf{Cons}}
        \begin{itemize}[label=\textcolor{Red2}{\faIcon{times}}]
            \item \textbf{Packets are modified}, which may:
            \begin{enumerate}
                \item Break some applications;
                \item Exceed MTU (Maximum Transmission Unit);
                \item Require special handling at end hosts.
            \end{enumerate}
        \end{itemize}
    \end{itemize}
\end{enumerate}

\newpage

\begin{itemize}
    \item Use \texttt{INT-XD} if we want no packet modifications and can tolerate heavy collector load.
    \item Use \texttt{INT-MK} for moderate flexibility but still out-of-band.
    \item Use \texttt{INT-MD} if we want \textbf{maximum in-path visibility} and can handle packet growth.
\end{itemize}

\begin{figure}[!htp]
    \centering
    \includegraphics[width=\textwidth]{img/int-1.pdf}
    \caption{INT modes.}
\end{figure}

\begin{table}[!htp]
    \centering
    \begin{adjustbox}{width={\textwidth},totalheight={\textheight},keepaspectratio}=
        \begin{tabular}{@{} l | c | c | c | c | l @{}}
            \toprule
            Mode & Packet Modified & Collector Load & Query Driven By & Pros & Cons \\
            \midrule
            \texttt{INT-XD} & \textcolor{Red2}{\faIcon{times}} No    & \textcolor{Red2}{\faIcon{times}} High  & Switch Config   & MTU-safe              & Inflexible, collector-heavy \\ [.3em]
            \texttt{INT-MK} & \textcolor{Green3}{\faIcon{check}} Yes & \textcolor{Red2}{\faIcon{times}} High  & Packet Mark     & Flexible              & MTU risk, collector-heavy \\ [.3em]
            \texttt{INT-MD} & \textcolor{Green3}{\faIcon{check}} Yes & \textcolor{Green3}{\faIcon{check}} Low & Packet          & No collector pressure & MTU impact, complex parsing \\
            \bottomrule
        \end{tabular}
    \end{adjustbox}
    \caption{Summary of INT modes.}
\end{table}

    %%%%%%%%%%%%%%%%%%%%%%%%%%%%%%%%%%%%%
    % Datacenter Layer 3 Load Balancing %
    %%%%%%%%%%%%%%%%%%%%%%%%%%%%%%%%%%%%%
    \section{Datacenter Layer 3 Load Balancing}

\subsection{Recap: Datacenters}

\begin{flushleft}
    \textcolor{Green3}{\faIcon{book} \textbf{Characteristics of Workloads}}
\end{flushleft}
Datacenters support very \textbf{different types of applications}, which shape the way networks are designed.
\begin{itemize}
    \item \textbf{HPC (High-Performance Computing)}. Focus on scientific simulations, weather modeling, genome analysis. Workloads: large parallel computations. Require \textbf{low latency} and \textbf{high throughput} for inter-node communication.
    \item \textbf{Web services}. Think Google Search, Facebook, Instagram. Many short-lived requests/responses. Traffic is bursty, dominated by \textbf{small ``mice flows''}. Latency-sensitive: users notice if it's slow.
    \item \textbf{Machine Learning (ML)}. Training large models across GPUs/TPUs. Requires frequent \textbf{synchronization} (gradient updates). Workloads dominated by \textbf{large ``elephant flows''} (bulk transfers). Need predictable, low tail latency (slowest worker delays the whole training).
    \item \textbf{Big Data (MapReduce, Spark, etc.)}. Shuffle phase $=$ massive all-to-all data exchange. Demands \textbf{high aggregate bandwidth}. Workloads are throughput-oriented, but also sensitive to stragglers.
\end{itemize}

\highspace
\begin{flushleft}
    \textcolor{Green3}{\faIcon{book} \textbf{Traffic Variability}}
\end{flushleft}
Datacenter traffic is not uniform. It's a \textbf{mixture of short and long flows}, and this mix complicates network design.
\begin{itemize}
    \item \textbf{Short flows (mice)}. Few KBs to MBs, bursty, latency-sensitive. For example: a web request, RPC, or cache lookup. Critical: the user's experience depends on their fast completion.
    \item \textbf{Long flows (elephants)}. Hundreds of MBs to GBs, throughput-\break oriented. For example: dataset shuffling, ML gradient exchange, backups. Can easily congest network links if not managed carefully.
    \item \textbf{Traffic patterns}
    \begin{itemize}
        \item \textbf{Shuffle traffic}: many-to-many, typical in MapReduce or Spark.
        \item \textbf{Gradient aggregation}: many-to-one or all-to-all, in ML training.
        \item \textbf{Incast}: one client requests data from many servers at once $\rightarrow$ bursts that overwhelm buffers.
    \end{itemize}
\end{itemize}
Networks must serve both mice and elephants efficiently. Prioritizing one can hurt the other.

\newpage

\begin{flushleft}
    \textcolor{Green3}{\faIcon{bullseye} \textbf{Goal: High Bisection Bandwidth}}
\end{flushleft}
Split the network into two equal halves; the \textbf{bisection bandwidth} is the total capacity of the links connecting them. Instead, \textbf{full-bisection bandwidth} means that every server can communicate with every other at full NIC speed, without bottlenecks.

\highspace
\textcolor{Green3}{\faIcon{question-circle} \textbf{Why it matters?}}
\begin{itemize}
    \item HPC: ensures all nodes in a parallel job can exchange data efficiently.
    \item Web services: prevents hotspots\footnote{%
        A \definition{Hotspot} in a datacenter network is a \textbf{point of congestion}, usually a specific link or switch that gets overloaded with too much traffic, while other parts of the network remain underutilized.
    } when thousands of requests are routed.
    \item ML/Big Data: allows large-scale shuffles without stragglers.
\end{itemize}
The main \textbf{challenge} is achieving full bisection bandwidth at low cost, but it requires special topologies (Fat-Tree, Clos, Jellyfish).

\highspace
In conclusion, datacenters run a \textbf{mix of workloads} (HPC, web, ML big data), which produce \textbf{variable traffic patterns} (mice vs. elephant flows, shuffle, gradient aggregation). The \textbf{main networking goal} is to deliver \textbf{high bisection bandwidth} so any-to-any communication can happen efficiently.
    \section{VLIW (Very Long Instruction Word)}

\subsection{Introduction}

Traditional \textbf{compilers} use \textbf{static code scheduling} to exploit Instruction-Level Parallelism (ILP). Key tasks of the compiler:
\begin{itemize}
    \item  Detect \textbf{parallelizable} instructions considering: Hardware resource constraints and Data dependencies.
    \item \textbf{Schedule} instructions to execute \textbf{in parallel} when possible.
    \item Otherwise, \textbf{insert \texttt{NOP}s} (No Operations) if no safe parallel execution is possible.
\end{itemize}
Statically scheduled processors trust the compiler to ``fill the pipeline'' and avoid hazards at compile time.

\highspace
\begin{flushleft}
    \textcolor{Green3}{\faIcon{book} \textbf{VLIW Processors: Alternative Way to Extract ILP}}
\end{flushleft}
\definition{VLIW (Very Long Instruction Word)} \textbf{processors} are a class of architectures designed to \textbf{execute multiple operations in parallel during a single clock cycle}, but unlike superscalar processors, they \textbf{rely on the compiler} rather than on dynamic hardware mechanisms \textbf{to detect and schedule parallelism}.

\highspace
The fundamental idea behind VLIW is to \textbf{group several independent operations together into a single long instruction word}, called a \hl{bundle}. This bundle is \textbf{composed of several fixed slots}, each one \textbf{corresponding to a different functional unit in the processor}, such as an integer ALU, a floating-point unit, a load/store unit, or a branch unit.

\highspace
For example, in a 5-issue VLIW processor, a single bundle would typically carry up to five operations: integer, floating-point, load/store, branch, etc.

\highspace
The \textbf{key difference} compared \textbf{to traditional superscalar} processors lies in who decides what can run in parallel. In superscalar designs, the hardware dynamically analyzes dependencies between instructions at runtime, which requires complex circuitry for hazard detection and scheduling. In VLIW architectures, this \hl{analysis is performed entirely at compile time}. The \textbf{compiler is responsible for}:
\begin{itemize}
    \item \textbf{Statically identify} independent operations.
    \item \textbf{Solve structural hazards} (e.g., two operations trying to use the same hardware unit).
    \item \textbf{Solve data hazards} (dependencies between instructions).
    \item Insert \textbf{NOPs} when necessary.
\end{itemize}
When no useful instruction is available for a particular slot, the compiler inserts a \texttt{NOP} (no-operation). In cases where conflicts cannot be avoided, NOPs are inserted into the bundle to fill empty slots.

\highspace
\begin{flushleft}
    \textcolor{Green3}{\faIcon{question-circle} \textbf{Why move to Compiler?}}
\end{flushleft}
The problem with Superscalar is that the hardware requires complex dynamic scheduling and dependency checking, which \textbf{costs area and power}.

\highspace
\textbf{VLIW Idea} is:
\begin{itemize}
    \item Push \textbf{complexity} to the \textbf{compiler}.
    \item \textbf{Compiler groups parallel operations} into a \textbf{single bundle}.
    \item No need for runtime dependency checking anymore.
\end{itemize}
The VLIW paradigm is characterized by the use of \textbf{very wide instruction words}, \textbf{each containing multiple independent operations} (``syllables''). A multiple-issue VLIW processor typically features specialized units for integer, floating-point, memory access, and branch instructions. For example, a 4-issue VLIW processor will have four operation slots per bundle, each connected to a different unit.

\highspace
\begin{flushleft}
    \textcolor{Green3}{\faIcon{list-ul} \textbf{Multiple-issue VLIW: Operation Latencies}}
\end{flushleft}
An important aspect in VLIW scheduling is the \textbf{management of operation latencies}. Each operation type may require a \textbf{different number of clock cycles to complete}. Integer operations, memory accesses, and floating-point computations may all have varying latencies, and these differences must be carefully considered during scheduling. The \textbf{compiler must plan the execution} so that operations complete in the correct order and without causing unnecessary stalls in the pipeline.

\highspace
In summary, \hl{VLIW architectures represent a shift of complexity from the hardware to the compiler}, aiming for more efficient and simpler processor designs while demanding sophisticated compiler techniques to fully exploit available parallelism.

\highspace
\begin{flushleft}
    \textcolor{Green3}{\faIcon{book} \textbf{Single Program Counter and Branch Management}}
\end{flushleft}
In a VLIW processor, even though multiple operations are issued simultaneously, the architecture still uses a \textbf{single program counter} to fetch instructions. Each instruction fetched corresponds to a \textbf{bundle}, which can contain multiple parallel operations.

\highspace
Importantly, within each bundle, there can be \textbf{at most one branch instruction} that affects control flow. This constraint simplifies the handling of branches, making it easier for the processor to predict and manage the program's execution path.

\newpage

\begin{flushleft}
    \textcolor{Green3}{\faIcon{book} \textbf{Shared Multi-Ported Register File}}
\end{flushleft}
Since a VLIW processor issues multiple operations in parallel, the \textbf{register file must support multiple simultaneous reads and writes}. In a typical 4-issue VLIW machine, the register file needs to have enough ports to read \textbf{eight source operands} and write \textbf{four destination results} every clock cycle.

\highspace
This requirement implies a \textbf{shared, multi-ported register file}, which is one of the non-trivial aspects of the hardware design. It ensures that all functional units can access their operands without creating bottlenecks.

\highspace
\begin{flushleft}
    \textcolor{Green3}{\faIcon{book} \textbf{Importance of Parallelism and the Use of NOPs}}
\end{flushleft}
To achieve high performance, the \textbf{compiler must ensure that the source code has enough parallelism} to fill all the available operation slots in each bundle. When \textbf{sufficient independent instructions are not available}, \textbf{NOPs are inserted} into the empty slots. This insertion preserves the fixed structure of the bundle but can lead to inefficiencies if the application does not expose enough ILP.

\highspace
\begin{flushleft}
    \textcolor{Green3}{\faIcon{book} \textbf{Pipeline Organization}}
\end{flushleft}
VLIW processors often use a \textbf{pipelined execution model} similar to traditional RISC architectures. A standard pipeline might include stages like \textbf{Instruction Fetch (IF)}, \textbf{Instruction Decode (ID)}, \textbf{Register Read (RR)}, \textbf{Execution (EX)}, and \textbf{Write Back (WB)}.

\begin{figure}[!htp]
    \centering
    \includegraphics[width=\textwidth]{img/vliw-arch.pdf}
\end{figure}

\highspace
In the previous figure, a \textbf{5-stage pipeline} is used, with each bundle passing through these stages. Each operation within a bundle proceeds independently through the functional units connected to the pipeline.

\newpage

\begin{flushleft}
    \textcolor{Green3}{\faIcon{book} \textbf{Operation Dispatch and Decode}}
\end{flushleft}
If the architecture dedicates \textbf{one functional unit per slot}, the decode stage becomes relatively simple. Each operation is directed straight to its corresponding functional unit for execution without the need for complex arbitration.

\highspace
However, if there are \textbf{more functional units than slots}, meaning there is a surplus of parallel hardware resources, the architecture must include a \textbf{dispatch network}. This network is responsible for forwarding each operation, along with its operands, to the appropriate available functional unit. In this way, the design of the dispatch system depends heavily on the organization of functional units relative to the number of issue slots defined by the instruction bundle format.

    \subsection{Packet Spraying}

\definition{Packet Spraying} is a load balancing technique where \textbf{each packet} of a flow is sent over a \textbf{different path} in the network, instead of keeping the whole flow on a single path. This technique \textbf{balances the load immediately} across all available links and ensures that no path is left idle, thus \hl{optimizing network capacity utilization}. However, there is one \textbf{significant issue}: \hl{packets of the same flow may arrive out of order} due to the varying delays of the different paths taken. TCP is confused by out-of-order delivery and thinks packets are lost, \hl{resulting in unnecessary retransmissions and reduced throughput}.

\highspace
\begin{flushleft}
    \textcolor{Green3}{\faIcon[regular]{lightbulb} \textbf{Idea of Packet Spraying}}
\end{flushleft}
Instead of assigning an entire flow to \textbf{one path}, packet spraying sends \textbf{each packet} of the flow independently across different available paths. For example, we have 4 equal-cost paths.
\begin{itemize}
    \item Packet 1 $\rightarrow$ Path A
    \item Packet 2 $\rightarrow$ Path B
    \item Packet 3 $\rightarrow$ Path C
    \item Packet 4 $\rightarrow$ Path D
\end{itemize}
The intuition of Packet Spraying: By spreading packets at the \emph{finest granularity}, all network links are used more evenly, and congestion is less likely to form.

\highspace
\begin{flushleft}
    \textcolor{Green3}{\faIcon{check-circle} \textbf{Pros}} \textbf{and} \textcolor{Red2}{\faIcon{times-circle} \textbf{Cons}}
\end{flushleft}
\begin{itemize}
    \item[\textcolor{Green3}{\faIcon{check-circle}}] \textcolor{Green3}{\textbf{Pros}}
    \begin{itemize}
        \item[\textcolor{Green3}{\faIcon{check}}] \textcolor{Green3}{\textbf{Great load distribution}}. No path stays idle while another is overloaded. Utilizes all network capacity.
        \item[\textcolor{Green3}{\faIcon{check}}] \textcolor{Green3}{\textbf{Simple logic}}. Doesn't need complex flow classification or scheduling. Just a round-robin or randomized assignment of packets.
        \item[\textcolor{Green3}{\faIcon{check}}] \textcolor{Green3}{\textbf{Fast reaction}}. Even short flows (mice) benefit, because their few packets can be split across paths immediately.
    \end{itemize}
    \item[\textcolor{Red2}{\faIcon{times-circle}}] \textcolor{Red2}{\textbf{Cons}}
    \begin{itemize}
        \item[\textcolor{Red2}{\faIcon{times}}] \textcolor{Red2}{\textbf{TCP reordering problem}}. TCP expects packets of a flow to arrive \textbf{in order}. With packet spraying, packets take different paths $\rightarrow$ different latencies $\rightarrow$ \hl{arrive out of order}. TCP interprets out-of-order packets as \textbf{loss} $\rightarrow$ triggers retransmissions, reduces congestion window, lowers throughput.
        \item[\textcolor{Red2}{\faIcon{times}}] \textcolor{Red2}{\textbf{Hardware complexity}}. Switches need per-packet decisions at line rate, which can be expensive.
        \item[\textcolor{Red2}{\faIcon{times}}] \textcolor{Red2}{\textbf{Not suitable for elephants}}. Large flows generate so many packets that reordering overhead becomes very high.
    \end{itemize}
\end{itemize}

\highspace
\begin{flushleft}
    \textcolor{Red2}{\faIcon{exclamation-triangle} \textbf{The Reordering Issue (and why Packet Spraying is absolutely avoided in production)}}
\end{flushleft}
Let's say \emph{flow F} has 3 packets:
\begin{itemize}
    \item P1 goes on Path 1 (latency 5 ms).
    \item P2 goes on Path 2 (latency 8 ms).
    \item P3 goes on Path 3 (latency 6 ms).
\end{itemize}
They arrive at the receiver as: P1 $\rightarrow$ P3 $\rightarrow$ P2. This out-of-order process produces the following errors:
\begin{itemize}
    \item TCP sees missing sequence numbers (it expected P2 after P1).
    \item Receiver sends \textbf{duplicate ACKs} (saying ``I didn't get P2'').
    \item Sender wrongly assumes \textbf{congestion/loss} $\rightarrow$ retransmits and slows down.
\end{itemize}
As a result, throughput drops and latency increases. CPU is wasted on useless retransmissions. That's why it is \textbf{not used in production} as a general solution.
    \subsection{Equal Cost Multi Path (ECMP)}

In datacenter networks (e.g., Fat-Tree/Clos topologies), there are \textbf{multiple equal-cost paths} between a source and a destination. Traditional IP routing normally picks \textbf{one path}, which wastes capacity. \definition{Equal Cost Multi-Path (ECMP)} is the standard mechanism that allows a router/switch to use \textbf{all equal-cost paths}.

\highspace
Main characteristics:
\begin{itemize}
    \item \textbf{Per-flow load balancing}: ECMP does not spray packets individually (like packet spraying). Instead, it \hl{ensures} that \textbf{all packets of the same flow follow the same path} and this avoids TCP reordering.
    \item \textbf{Hashing}: Switches compute a hash of packet header fields (usually 5-tuple: source IP, destination IP, source port, destination port, protocol). The hash value is mapped to one of the available next-hop paths. All packets of the same flow produce the same hash $\rightarrow$ go on the same path.
\end{itemize}
In summary, ECMP is a Layer 3 load balancing technique where each flow is assigned to one of the available equal-cost paths using a hash function on packet headers, ensuring packets stay in order and TCP remains happy.

\begin{examplebox}
    Say there are 4 equal-cost paths. A hash function outputs values 0-3.
    \begin{itemize}
        \item Flow A (src, dst, IP and port) hashes to 0 $\rightarrow$ path 1.
        \item Flow B hashes to 2 $\rightarrow$ path 3.
        \item Flow C hashes to 2 $\rightarrow$ also path 3.
        \item Flow D hashes to 1 $\rightarrow$ path 2.
    \end{itemize}
    Each flow is consistently mapped to one path. Packets stay in order.
\end{examplebox}

\highspace
\begin{flushleft}
    \textcolor{Red2}{\faIcon[regular]{thumbs-down} \textbf{Hash Collisions and Inefficiency}}
\end{flushleft}
In ECMP, the hash function maps each flow to one of the available paths. If two or more \textbf{large elephant flows} hash to the same path, that path becomes congested and other paths may stay underutilized. This is called a \definition{Hash Collision}. Collisions are harmless for tiny mice flows, but disastrous when multiple elephants collide.

\newpage

\noindent
\textcolor{Red2}{\faIcon{question-circle} \textbf{Why Collisions Matter.}} \textbf{Elephant flows dominate traffic volume}. Even if 90\% of flows are small, the few elephants carry most of the bytes. If elephants collide on the same link, throughput is reduced and latency spikes for other flows sharing that links. It creates \textbf{hotspots} while parallel links sit idle.

\highspace
\textcolor{Red2}{\faIcon{exclamation-triangle} \textbf{Inefficiency.}} Hashing spreads flows \emph{randomly}, not \emph{evenly}. With $k$ paths, the load per path can vary widely, especially when:
\begin{itemize}
    \item The number of elephant flows is small.
    \item A few unlucky hashes cluster them together.
\end{itemize}
So the network's \textbf{theoretical capacity} is high, but \textbf{effective throughput} is lower due to imbalance.

\begin{examplebox}[: Hash Collisions and Inefficiency]
    Imagine 4 equal-cost paths and 3 elephant flows.
    \begin{itemize}
        \item Flow A $\rightarrow$ hashes to path 1.
        \item Flow B $\rightarrow$ hashes to path 1.
        \item Flow C $\rightarrow$ hashes to path 3.
    \end{itemize}
    Path usage:
    \begin{itemize}
        \item Path 1: 2 elephants (overloaded).
        \item Path 2: empty.
        \item Path 3: 1 elephant.
        \item Path 4: empty.
    \end{itemize}
    Outcome:
    \begin{itemize}
        \item Path 1 congests $\rightarrow$ throughput limited.
        \item 50\% of available network capacity wasted (paths 2 and 4 unused).
    \end{itemize}
\end{examplebox}

\noindent
ECMP's reliance on static hashing leads to \textbf{hash collisions}, where multiple large flows land on the same path. This causes \textbf{inefficient bandwidth utilization} and \textbf{network hotspots}, even though other paths are free.

\newpage

\begin{flushleft}
    \textcolor{Red2}{\faIcon{radiation-alt} \textbf{There are two problems that ECMP still cannot resolve}}
\end{flushleft}
Incast and rack skew are two problems that ECMP alone cannot solve. This is one of the reasons that pushes researchers to find a better solution.
\begin{itemize}
    \item \definition{Bursty Traffic (Incast)}. Incast happens when \textbf{one receiver asks for data from many servers at the same time}. For example, a storage node requests blocks from 50 servers; all 50 servers respond \textbf{at once} and their packets all converge on the \textbf{same final link} to the receiver.

    \highspace
    \textcolor{Red2}{\faIcon{question-circle} \textbf{Why ECMP doesn't help.}} ECMP can spread traffic across multiple \emph{upstream} paths. But the \textbf{last hop into the receiver} is always the same physical link. That link suddenly gets a burst of packets from 50 sources.

    \highspace
    \textcolor{Red2}{\faIcon{exclamation-triangle} \textbf{The consequence.}} The buffer at that last-hop switch \textbf{overflows}. Packets are dropped, so TCP retransmits. Latency increases dramatically.

    \highspace
    So even if ECMP spreads flows earlier in the path, it \hl{\textbf{cannot prevent congestion at the final bottleneck}} in incast scenarios.


    \item \definition{Flow Skew Across Racks}. Skew means imbalance. Some racks \textbf{generate much more traffic} than others, depending on what services run there.
    
    For example:
    \begin{itemize}
        \item Rack A: runs a database cluster $\rightarrow$ produces many \textbf{elephant flows}.
        \item Rack B: runs lightweight web servers $\rightarrow$ produces mostly \textbf{mice flows}.
    \end{itemize}
    ECMP hashes flows randomly, but Rack A already has more elephants, so its outgoing paths are \textbf{more likely to get congested}; Rack B's paths stay underutilized.

    \highspace
    \textcolor{Red2}{\faIcon{question-circle} \textbf{Why this is a problem.}} ECMP doesn't adapt to traffic intensity differences between racks. It treats all flows equally, ignoring that some racks are ``heavy hitters''. So \hl{persistent \textbf{hotspots near busy racks}} and wasted capacity elsewhere.
\end{itemize}

\highspace
\begin{flushleft}
    \textcolor{Green3}{\faIcon{industry} \textbf{ECMP in Production and Its Limitations}}
\end{flushleft}
\begin{itemize}
    \item[\textcolor{Green3}{\faIcon{question-circle}}] \textcolor{Green3}{\textbf{Why ECMP Was Adopted.}} There are three main reasons:
    \begin{itemize}
        \item \textbf{Industry standard}: ECMP is built into traditional routing protocols (OSPF, IS-IS, BGP).
        \item \textbf{Easy to deploy}: No special hardware or centralized controller\break needed.
        \item \textbf{Good enough for mice flows}: in web workloads (many small flows), ECMP spreads traffic fairly evenly. Avoids TCP reordering (a major plus over packet spraying).
    \end{itemize}


    \item[\textcolor{Red2}{\faIcon{exclamation-triangle}}] \textcolor{Red2}{\textbf{Observed Problems in Production.}} But at hyperscale (tens or hundreds of thousands of servers), \textbf{ECMP inefficiencies become visible}:
    \begin{itemize}
        \item \textbf{Static \& oblivious to congestion}. ECMP only looks at header hashes, not at link utilization. A congested link may still attract new elephant flows while other links remain idle.
        \item \textbf{Flow collisions}. In large topologies, even with thousands of equal-cost paths, collisions between elephants are common. A few unlucky hashes waste a lot of bisection bandwidth.
        \item \textbf{Wasted capacity}. Studies (e.g., on fat-tree topologies with $\approx 27$k hosts) showed ECMP could waste \textbf{over 60\% of available bisection bandwidth} on average due to imbalance.
        \item \textbf{Long-lived collisions}. Once a flow is hashed to a path, it stays there. If that assignment is bad, the flow suffers for its entire lifetime.
    \end{itemize}
\end{itemize}
ECMP became the \textbf{default production solution} because it's simple, distributed, and TCP-friendly. But at datacenter scale, its \textbf{static, hash-based nature} makes it inefficient, prompting research into \textbf{smarter, traffic-aware load balancers} like Hedera (SDN-based, page \pageref{subsection: Hedera - Dynamic Flow Scheduling}) and HULA (P4-based, page \pageref{subsection: HULA - Load Balancing in P4}).

\highspace
\begin{flushleft}
    \textcolor{Green3}{\faIcon{check-circle} \textbf{Pros}} \textbf{and} \textcolor{Red2}{\faIcon{times-circle} \textbf{Cons}}
\end{flushleft}
\begin{itemize}
    \item[\textcolor{Green3}{\faIcon{check-circle}}] \textcolor{Green3}{\textbf{Advantages}}
    \begin{itemize}
        \item[\textcolor{Green3}{\faIcon{check}}] \textcolor{Green3}{\textbf{Simplicity}}. Uses a straightforward hashing mechanism. No central controller or complex scheduling needed. Easy to implement in commodity switches.
        \item[\textcolor{Green3}{\faIcon{check}}] \textcolor{Green3}{\textbf{Avoids packet reordering}}. All packets of the same flow follow the same path. TCP sees packets in order, so it doesn't mistakenly trigger retransmissions.
        \item[\textcolor{Green3}{\faIcon{check}}] \textcolor{Green3}{\textbf{Good for many short flows (mice)}}. With millions of short, random flows, the hashing tends to spread them fairly well. This makes ECMP very effective in web-service workloads where flows are small and numerous.
        \item[\textcolor{Green3}{\faIcon{check}}] \textcolor{Green3}{\textbf{Scalability}}. ECMP is distributed: each switch does hashing locally. No centralized bottleneck, works across large-scale datacenters.
        \item[\textcolor{Green3}{\faIcon{check}}] \textcolor{Green3}{\textbf{Widely supported}}. ECMP is built into IP routing standards\break (OSPF, IS-IS, BGP). Already deployed in real datacenters today.
    \end{itemize}
    \item[\textcolor{Red2}{\faIcon{times-circle}}] \textcolor{Red2}{\textbf{Disadvantages}}
    \begin{itemize}
        \item[\textcolor{Red2}{\faIcon{times}}] \textcolor{Red2}{\textbf{Hash collisions}}. Two or more elephant flows (large flows) may hash to the same path. Result: some links get congested while others are idle $\rightarrow$ creates hotspots.
        \item[\textcolor{Red2}{\faIcon{times}}] \textcolor{Red2}{\textbf{No congestion awareness}}. ECMP assigns paths purely based on hash, not on current load. If one link is already overloaded, ECMP doesn't know $\rightarrow$ it may keep adding new flows there.
        \item[\textcolor{Red2}{\faIcon{times}}] \textcolor{Red2}{\textbf{Unfairness}}. Mice flows are fine, but a single elephant can dominate a link if unlucky with its hash. Other elephants hashed to that path suffer, while bandwidth on other links is wasted.
        \item[\textcolor{Red2}{\faIcon{times}}] \textcolor{Red2}{\textbf{Static behavior}}. Once a flow is mapped, it stays on that path until it finishes. ECMP doesn't migrate flows if conditions change.
    \end{itemize}
\end{itemize}
ECMP is simple, scalable, and TCP-friendly, which is why it's the default in datacenters. But it's also \textbf{static and oblivious to congestion}, so it can lead to \textbf{hotspots} when elephant flows collide.

\begin{deepeningbox}[: How ECMP Uses Hashing to Pick a Path]
    Three steps:
    \begin{enumerate}
        \item \important{Multiple Equal-Cost Paths Exist}. Imagine a datacenter topology (like a Fat-Tree). From server \textbf{S1} to server \textbf{S2}, the routing protocol (e.g., OSPF, IS-IS) discovers that there are \textbf{$k$ different next-hop paths} that all have the \textbf{same cost}. For example, 4 paths, so the next-hops are \texttt{\{N1, N2, N3, N4\}}. The routing table on the switch stores:
        \begin{equation*}
            \text{Destination S2} \rightarrow \text{Next-hops: N1, N2, N3, N4 (all cost 10)}
        \end{equation*}
        So, the switch knows it \emph{can choose} any of them.


        \item \important{Switch Computes a Hash of the Flow}. When a new packet arrives, the switch looks at the \textbf{flow identifier} (usually 5-tuple: source IP, destination IP, source port, destination port, protocol). It computes a \textbf{hash function}, for example:
        \begin{lstlisting}
hash = H(srcIP, dstIP, srcPort, dstPort, proto)\end{lstlisting}
        This gives an integer value.


        \item \important{Map the Hash to a Next-Hop}. Now comes the key: the switch takes the hash value \textbf{modulo the number of available next-hops ($k$)}:
        \begin{lstlisting}
path_index = hash % k\end{lstlisting}
        \begin{itemize}
            \item If \texttt{path\_index = 0} $\rightarrow$ send packet to N1.
            \item If \texttt{path\_index = 1} $\rightarrow$ send packet to N2.
            \item If \texttt{path\_index = 2} $\rightarrow$ send packet to N3.
            \item If \texttt{path\_index = 3} $\rightarrow$ send packet to N4.
        \end{itemize}
        All packets of the same flow have same 5-tuple, then same hash and same path. Instead, different flows have different hashes and likely different paths.
    \end{enumerate}
    
    \newpage

    \begin{flushleft}
        \textcolor{Green3}{\faIcon{question-circle} \textbf{Why This Works}}
    \end{flushleft}
    The \textbf{paths themselves aren't ``hashable''}. Instead, the switch maintains a \emph{list of possible next-hops} for the destination. The hash selects an \textbf{index} into that list. This is why the hash needs to be ``uniform'' $\rightarrow$ to spread flows across all next-hops evenly.

    In other words, ECMP doesn't hash the paths themselves. It hashes the \textbf{flow's header fields} and then uses the hash result to pick an \textbf{index} from the list of equal-cost paths in the routing table.

    \highspace
    \begin{flushleft}
        \textcolor{Green3}{\faIcon{question-circle} \textbf{Why ECMP Uses Modulo on the Hash Value}}
    \end{flushleft}
    We want to assign each flow to \textbf{one of the available next-hops}. Suppose there are $k$ equal-cost paths (say 4). The switch needs a simple way to map the \textbf{huge range of hash outputs} (e.g., 32-bit integer) down to just 4 choices.

    \textcolor{Red2}{\faIcon{exclamation-triangle} \textbf{The Problem.}} Hash functions produce large numbers (e.g., $0 \dots 2^{32-1}$). But the switch only has a small number of next-hops ($k$ paths). We need a consistent, deterministic way to map ``large space $\xrightarrow{to}$ small space''.

    \textcolor{Green3}{\faIcon{check-circle} \textbf{The Solution: Modulo.}} Compute:
    \begin{lstlisting}
path_index = hash(flow_id) % k\end{lstlisting}
    The result is guaranteed to be in the range $0 \dots k-1$. That matches exactly the \textbf{indices of the next-hop list}. For example, with 4 paths:
    \begin{lstlisting}[mathescape=true]
hash(flow A) = 57 $\rightarrow$ 57 % 4 = 1 $\rightarrow$ path 2
hash(flow B) = 134 $\rightarrow$ 134 % 4 = 2 $\rightarrow$ path 3
hash(flow C) = 29 $\rightarrow$ 29 % 4 = 1 $\rightarrow$ path 2        
    \end{lstlisting}
    \textcolor{Green3}{\faIcon{question-circle} \textbf{Why Modulo Works Well.}} There are three main reasons:
    \begin{enumerate}
        \item \textbf{Uniformity}: If the hash function is good, the outputs are ``random-looking'', so modulo spreads flows fairly evenly across paths.
        \item \textbf{Deterministic}: Same flow always hashes to the same path.
        \item \textbf{Simple in hardware}: Modulo is fast and easy for switches to implement.
    \end{enumerate}
    Modulo is used because it \textbf{compresses the large hash space into exactly the number of available paths}. That way, every flow gets assigned to one valid next-hop index.
\end{deepeningbox}

    %%%%%%%%%%%%%%%%%%%
    % Fancy pagestyle %
    %%%%%%%%%%%%%%%%%%%
    \pagestyle{fancy}
    \fancyhead{} % clear all header fields
    \fancyhead[R]{\nouppercase{\leftmark\hfill\rightmark}}

    %%%%%%%%%%%%%%%%%%%%%%%%%%
    % Bibliography and index %
    %%%%%%%%%%%%%%%%%%%%%%%%%%
    \pagestyle{fancy}
\fancyhead{} % clear all header fields
\fancyhead[R]{\nouppercase{\leftmark\hfill\rightmark}}

\pagestyle{fancy}
\fancyhead{} % clear all header fields
\fancyhead[R]{\nouppercase{\leftmark}}

\bibliography{bibtex}{}
\bibliographystyle{plain}

\newpage

\printindex
\end{document}
