\section{Data Structures}

\subsection{Introduction}

Modern network devices, particularly programmable switches (PISA, page \pageref{subsection: PISA and Compiler Pipeline Mapping}), implement a \textbf{packet processing pipeline} composed of three main blocks:
\begin{enumerate}
    \item \textbf{Parser}: Extracts relevant headers from incoming packets.
    \item \textbf{Match-Action Pipeline}:
    \begin{itemize}
        \item \textbf{Match}: Uses lookup tables to compare extracted headers against known values.
        \item \textbf{Action}: Applies logic (e.g., modify headers, make routing decisions).
    \end{itemize}
    \item \textbf{Deparser}: Reassembles the final packet for transmission.
\end{enumerate}
This flow is deterministic and must maintain constant processing latency per stage, as switches are often implemented as hardware pipelines (one packet per stage per clock cycle).

\highspace
\begin{flushleft}
    \textcolor{Green3}{\faIcon{question-circle} \textbf{Layer 3 (L3) Router}}
\end{flushleft}
The L3 router is a classic example used to explain the packet matching process:
\begin{itemize}
    \item \textbf{Input}: IP destination address from packet.
    \item \textbf{Match Logic}: \hl{Find the Longest Prefix Match (LPM) in a routing table.}
    \item \textbf{Action Logic}: Forward the packet to the correct output port, and adjust MAC address accordingly.
\end{itemize}
\definition{Longest Prefix match (LPM)} is a fundamental concept in IP routing, where the \textbf{goal is to find the most specific route} (i.e., the one with the longest matching prefix) \textbf{for a given IP destination address}.

\highspace
When a router receives a packet, it checks the \textbf{destination IP address} and compares it to entries in its \textbf{routing table}, which typically contain IP prefixes like:
\begin{itemize}
    \item \texttt{192.168.0.0/16}
    \item \texttt{192.168.1.0/24}
    \item \texttt{192.168.1.128/25}
\end{itemize}
The router selects the entry whose prefix \textbf{matches the destination address} and has the \textbf{longest subnet mask} (i.e., most specific match).

\highspace
In high-speed routers or programmable switches, \textbf{LPM must be done very quickly}, ideally in constant time. The naive solution for LPM is linear search over all routing entries. However, with thousands of entries, this is computationally infeasible at line rate. So the key questions in this section are:
\begin{itemize}
    \item How do we \textbf{efficiently implement LPM}?
    \item Which data structures allow \textbf{fast lookups in a predictable and limited time}?
\end{itemize}