\subsubsection{Count-Min Sketch}

The \definition{Count-Min Sketch} is a \textbf{probabilistic data structure} used to estimate the \textbf{frequency of elements} in a stream.
\begin{itemize}
    \item We don't store each element individually.
    \item Instead, we use a \textbf{compat structure to maintain approximate\break counts}.
    \item It's designed for efficiency, especially when tracking millions of elements would be too memory-intensive.
\end{itemize}

\highspace
\begin{flushleft}
    \textcolor{Green3}{\faIcon{tools} \textbf{How does it work?}}
\end{flushleft}
We create a 2D array of counters with:
\begin{itemize}
    \item $d$ rows (one per hash function)
    \item $w$ columns (size of each hash domain)
\end{itemize}
This gives a table of size $w \times d$, much smaller than a full hash table for all possible items. \textbf{Each row} has a \textbf{different hash function}.

\highspace
\begin{flushleft}
    \textcolor{Green3}{\faIcon{plus} \textbf{Insertion}}
\end{flushleft}
To insert an element (e.g. \texttt{ip.dest1}):
\begin{enumerate}
    \item \textbf{Hash the element} with each of the $d$ hash functions.
    \item \textbf{Each hash} gives we a \textbf{column index in its row}.
    \item \textbf{Increment} the corresponding \textbf{counters}.
\end{enumerate}
So we increment \textbf{1 counter per row}, total $d$ counters updated.

\highspace
\begin{flushleft}
    \textcolor{Green3}{\faIcon{search} \textbf{Querying the Frequency}}
\end{flushleft}
To estimate the count of an element:
\begin{enumerate}
    \item Hash it again with the same $d$ hash functions.
    \item Get the counter values from the same positions.
    \item \textbf{Return the minimum} of those $d$ counters.
\end{enumerate}
The \textbf{minimum} because:
\begin{itemize}
    \item Collisions with other elements can cause overestimation (counters get inflated).
    \item But the \textbf{minimum is never less than the true count}, so it's a \textbf{safe lower bound}.
\end{itemize}
That's where the name comes from: count, because it estimates the frequency, and min, because it takes the minimum over multiple counters.

\highspace
\begin{flushleft}
    \textcolor{Green3}{\faIcon{check-circle} \textbf{Advantages}}
\end{flushleft}
\begin{itemize}
    \item Sublinear space: uses \textbf{much less memory} than a full table.
    \item \textbf{Fast}: insertions and queries are both $O\left(d\right)$ time (constant if $d$ is fixed).
    \item Suitable for \textbf{high-speed data streams} (e.g., network flows, telemetry, monitoring).
\end{itemize}

\highspace
\begin{table}[!htp]
    \centering
    \begin{tabular}{@{} l | l @{}}
        \toprule
        \textbf{Feature} & \textbf{Value} \\
        \midrule
        Use case            & Approximate frequency counts \\ [.3em]
        Memory              & Sublinear ($ w \times d $)   \\ [.3em]
        Insertion time      & $ O(d) $                     \\ [.3em]
        Query time          & $ O(d) $, returns minimum    \\ [.3em]
        Overestimates       & Possible                     \\ [.3em]
        Underestimates      & Never                        \\ [.3em]
        Similar to          & Counting Bloom Filter        \\
        \bottomrule
    \end{tabular}
    \caption{Count-Min Sketch summary.}
\end{table}