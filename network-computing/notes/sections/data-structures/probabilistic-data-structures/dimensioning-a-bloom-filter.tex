\subsubsection{Dimensioning a Bloom Filter}

We want to design a Bloom Filter that:
\begin{itemize}
    \item Stores $N$ elements
    \item Uses $M$ bits (memory size)
    \item Applies $K$ hash functions
\end{itemize}
But we also to \textbf{control the False Positive Rate (FPR)} and avoid unnecessary computation.

\highspace
There are three parameters in play:
\begin{enumerate}
    \item \textbf{Memory} $M$: more bits $\Rightarrow$ lower FPR
    \item \textbf{Number of Hashes} $K$: more hashes $\Rightarrow$ lower FPR, but higher computational cost
    \item \textbf{False Positive Rate} (FPR): we want this to be as low as possible.
\end{enumerate}
Improving one usually worsens another. This is the classic \textbf{space/time/error trade-off}.

\highspace
\begin{flushleft}
    \textcolor{Green3}{\faIcon{check-circle} \textbf{Asymptotic Approximation for FPR}}
\end{flushleft}
In our case, the Asymptotic Approximation is a simplified mathematical expression that \textbf{estimates the False Positive Rate (FPR)} of a Bloom Filter when the number of \textbf{cells $M$ is large}. It's derived from the exact expression but uses limits and approximations that hold when $M \gg N$. It's much easier to work with and very accurate in practice.

\highspace
If we insert $N$ elements into a Bloom filter with $M$ bits and use $K$ hash functions, the \textbf{exact False Positive Rate (FPR)}:
\begin{equation}
    \text{Exact FPR } = \left(1 - \left(1 - \dfrac{1}{M}\right)^{\left(K \cdot N\right)}\right)^{K}
\end{equation}
This expression can be tedious to compute, especially for large values of $M$, $N$, and $K$. By using the approximation:
\begin{equation*}
    \left(1 - \dfrac{1}{M}\right)^{K \cdot N} \approx e^{\left(-K \cdot \frac{N}{M}\right)} \hspace{2em} \text{when } M \gg 1
\end{equation*}
The \definition{Asymptotic Approximation of False Positive Rate (FPR)} is:
\begin{equation}
    \text{FPR } \approx \left(1 - e^{\left(-K \cdot \frac{N}{M}\right)}\right)^{K}
\end{equation}
This approximation is easier to analyze and is widely used in practice.

\highspace
\begin{flushleft}
    \textcolor{Green3}{\faIcon{question-circle} \textbf{Finding the Optimal Number of Hash Functions}}
\end{flushleft}
The optimal number of hash functions $K$ \textbf{minimizes the FPR} for given $M$ and $N$. We can find it by minimizing the FPR formula:
\begin{equation}
    K_{\text{opt}} = \dfrac{M}{N} \cdot \ln \left(2\right)
\end{equation}