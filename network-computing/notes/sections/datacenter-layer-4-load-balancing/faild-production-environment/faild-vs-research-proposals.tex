\subsubsection{Faild vs Research Proposals}\label{sec:faild-vs-research-proposals}

Faild clearly illustrates the \textbf{mismatch between academic assumptions and production realities} in the design of Layer-4 load balancers. Many research proposals are developed under controlled conditions, where failures are rare, backend sets are relatively stable, and infrastructure is homogeneous. In such environments, it is reasonable to optimize for elegant properties such as strict statelessness, perfect uniformity, or mathematically minimal connection remapping.

\highspace
In production edge systems, however, these assumptions no longer hold. Failures, restarts, and reconfigurations are common events rather than exceptional cases. Traffic patterns are highly variable, and systems must remain operational even when parts of the infrastructure behave unpredictably. Under these conditions, designs that are optimal on paper can become fragile, hard to debug, and costly to operate.

\highspace
Faild embraces this reality by prioritizing \textbf{stability over optimality}. Instead of attempting to perfectly balance load at all times, it \hl{focuses on ensuring that the system behaves consistently under stress and that failures do not lead to widespread disruption}. Minor inefficiencies in load distribution are considered acceptable if they help contain damage and preserve user experience.

\highspace
Similarly, Faild values \textbf{predictability over perfect uniformity}. Deterministic behavior is easier to reason about during incidents, simplifies debugging, and reduces the risk of cascading failures. From an operational perspective, \hl{knowing how the system will react to changes is often more important than achieving ideal theoretical metrics}.

\highspace
Overall, \hl{Faild demonstrates that effective L4 load balancing in production} is not about achieving perfect theoretical guarantees, but \textbf{about making carefully engineered trade-offs}. By sacrificing some elegance and optimality, Faild gains robustness, operational simplicity, and resilience, qualities that are essential in large-scale edge deployments.
