\section{Datacenter Layer 4 Load Balancing}

\subsection{Introduction}

A \definition{Load Balancer (LB)} is a device/system that assigns incoming client requests to different servers. At \definitionWithSpecificIndex{Layer 4 (transport layer)}{Layer 4 Load Balancing}{}, this decision is made \textbf{based on transport-level headers} (IP addresses $+$ TCP/UPD ports). For example, packets belonging to the same TCP connection must go to the same server.

\highspace
The goal is to:
\begin{itemize}
    \item \textbf{Distribute traffic uniformly} across servers.
    \item \textbf{Preserve connection affinity} (all packets of a connection handled by the same backend).
    \item \textbf{Improve utilization} of compute, network, and storage resources.
\end{itemize}
In simple words, instead of one server handling all client requests, an L4 load balancer spreads them across many servers while keeping each connection consistent.

\highspace
\begin{flushleft}
    \textcolor{Green3}{\faIcon{question-circle} \textbf{Why is it needed in datacenters?}}
\end{flushleft}
\begin{enumerate}
    \item \textbf{User-facing traffic}. Datacenters host applications accessed by millions of users (websites, APIs, services). Incoming requests first hit a \textbf{Virtual IP (VIP)} exposed to the Internet. The load balancer decides which backend server should handle each request. Without load balancing, a single server would be overwhelmed.
    \item \textbf{Scalability}. Traffic volume is enormous: thousands/millions of requests per second. A load balancer enables \textbf{horizontal scaling} (adding more servers behind the VIP). Different approaches (TCP termination, NAT, Direct Server Return) improve scalability for asymmetric or heavy workloads.
    \item \textbf{Reliability \& performance}. If one server fails, the LB can redirect traffic to healthy servers. Balancing traffic avoids hotspots (some servers overloaded while others are idle). Helps achieve predictable performance and meet SLAs\footnote{%
        \textbf{SLA} stands for \textbf{Service Level Agreement}. It's a \textbf{formal contract} between a service provider (e.g., a cloud/datacenter operator) and a customer. It defines the \textbf{expected level of server} in measurable terms.
    } (latency, throughput).

    \textcolor{Green3}{\faIcon{question-circle} \textbf{Why do SLAs matter for load balancing?}} If all requests go to one overloaded server, latency spikes and SLA targets are violated. L4 Load Balancers spread traffic across servers to ensure:
    \begin{itemize}
        \item \textbf{No single server becomes a bottleneck}.
        \item \textbf{Latency and availability goals in SLAs are respected}.
    \end{itemize}
    SLAs drive the \textbf{design of resilient load balancers}: failover, redundancy, uniform distribution.
\end{enumerate}

\highspace
\begin{flushleft}
    \textcolor{Green3}{\faIcon{question-circle} \textbf{What is a VIP?}}
\end{flushleft}
A \definition{Virtual IP} is an \textbf{IP address not bound to a single physical server}, but instead represents a \textbf{service} (e.g. \href{http://www.example.com}{www.example.com}). Clients on the Internet only see the VIP when they connect. The \textbf{load balancer owns the VIP} and listens for incoming connections. When traffic arrives at the VIP:
\begin{enumerate}
    \item The load balancer decides \textbf{which backend server} should handle the request.
    \item It then forwards the packet/connection to that chosen server.
    \item From the client's perspective, it's always talking to the VIP, even though multiple servers are working behind it.
\end{enumerate}
For example \texttt{8.8.8.8} is Google's DNS VIP. Millions of users query \texttt{8.8.8.8}. Behind the scenes, Google's load balancers spread requests across many DNS servers worldwide. Users don't need to know which specific server they hit, they always use the same VIP.

\highspace
Layer 4 Load Balancing is fundamental in datacenters because it \textbf{efficiently distributes massive user traffic} across backend servers, ensures \textbf{connection consistency}, and supports \textbf{scalability and fault tolerance}. It's a key building block for serving large-scale Internet services.