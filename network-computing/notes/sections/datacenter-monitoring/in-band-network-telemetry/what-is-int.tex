\subsection{In-Band Network Telemetry (INT)}

\subsubsection{What is INT?}

\definition{In-Band Network Telemetry} is a framework where the \textbf{data plane itself collects telemetry information} as packets traverse the network. Instead of relying on mirrored copies or external probes (like Everflow or FlowRadar), INT \textbf{embeds telemetry instructions directly into packets}. This means that the \textbf{packet \emph{asks} switches along its path to record certain metadata} (e.g., delay, queue size, switch ID).

\highspace
INT demonstrated the power and usefulness of programmable switches. It was one of the first real-world use cases where P4 offered something no traditional switch could do.

\highspace
\begin{flushleft}
    \textcolor{Green3}{\faIcon{tools} \textbf{How it works?}}
\end{flushleft}
\begin{enumerate}
    \item Packets carry \textbf{INT headers}.
    \item INT-capable devices \textbf{read the instructions} in those headers.
    \item They \textbf{\emph{collect}} specific \textbf{network state} and \textbf{\emph{append} it to the packet} as it moves.
    \item The telemetry data is delivered \textbf{in-band}, alongside the normal traffic.
\end{enumerate}
The data collected cloud include: switch ID, input and output ports, queue occupancy, timestamp (arrival and departure), packet latency per hop.

\highspace
\begin{flushleft}
    \textcolor{Green3}{\faIcon{question-circle} \textbf{Why INT?}}
\end{flushleft}
INT \textbf{solves key limitations of older monitoring tools}:
\begin{itemize}[label=\textcolor{Green3}{\faIcon{check}}]
    \item \textbf{No} need for \textbf{mirrored} traffic (like Everflow).
    \item \textbf{Real-time per-packet} information.
    \item \textbf{High visibility} into what happens at every hop.
\end{itemize}
This is especially useful for: fine-grained performance monitoring; diagnostic congestion, jitter, and path problems; dynamic traffic engineering.