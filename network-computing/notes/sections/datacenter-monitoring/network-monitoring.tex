\subsection{Network Monitoring}

\definition{Network Monitoring} is the \textbf{continuous observation of a computer network to detect slowdowns or failures in components}. Its purpose is to detect, localize and respond to faults before they impact users.

\highspace
\begin{flushleft}
    \textcolor{Green3}{\faIcon{question-circle} \textbf{Monitoring Scope}}
\end{flushleft}
Monitoring spans the \textbf{entire network path}. Each router (or switch/server) is a point where failures or slowdowns can occur:
\begin{itemize}
    \item A switch could drop the packet silently.
    \item A routing issue could cause the packet to loop.
    \item A delay could occur due to congestion in queues.
\end{itemize}
\textbf{To effectively detect and diagnose problems}, the monitoring system must observe not just endpoints, but the entire path, or at least enough of it to: detect \emph{where} things go wrong, or understand \emph{why} a packet failed to reach its destination. This is why datacenter monitoring often tries to trace or mirror packets at different points in the network, to reconstruct the packet's journey and find anomalies.

\highspace
\begin{flushleft}
    \textcolor{Green3}{\faIcon{tools} \textbf{Monitoring Techniques}}
\end{flushleft}
There are many ways to monitor a network. It can be done:
\begin{enumerate}
    \item \textbf{From Switches}:
    \begin{enumerate}
        \item \textbf{Built-in Features}
        \begin{itemize}
            \item \emph{NetFlow}: collects IP traffic statistics.
            \item \emph{Mirroring}: duplicates selected packets for analysis.
            \item \emph{SNMP (Simple Network Management Protocol)}: polls device stats.
        \end{itemize}
        \item \textbf{Programmable Switches}
        \begin{itemize}
            \item Use \emph{data plane programmability} (e.g., P4 language) to define custom monitoring behaviors.
            \item Enables \emph{custom counters}, tagging, filtering, or tracing at wire speed.
        \end{itemize}
    \end{enumerate}
    
    \item \textbf{From Servers}:
    \begin{enumerate}
        \item \textbf{Standard Tools}
        \begin{itemize}
            \item \texttt{netstat}: network connections and stats.
            \item \texttt{tcpdump}: packet capture and inspection.
            \item \texttt{traceroute}: path tracing and latency.
        \end{itemize}
        \item \textbf{Ad-hoc Monitoring Services}
        \begin{itemize}
            \item Lightweight daemons or agents tailored for the datacenter.
            \item Export performance metrics or send alerts.
        \end{itemize}
    \end{enumerate}
\end{enumerate}
There is no single way to monitor, a \textbf{mix of passive and active, centralized and distributed methods is used}. Monitoring systems must collect data from multiple vantage points to build a full picture of the network's health.

\highspace
\begin{flushleft}
    \textcolor{Red2}{\faIcon{exclamation-triangle} \textbf{Why traditional monitoring isn't enough}}
\end{flushleft}
In large-scale datacenters many failures are subtle and effect only specific flows of packets:
\begin{itemize}
    \item \important{Silent packet drops}. Packets are dropped but not reported by switches. The causes are software bugs or faulty hardware.
    \item \important{Silent blackholes}. Traffic is blackholed without showing in forwarding tables. The causes are corrupted TCAM entries.
    \item \important{Inflated end-to-end latency}. Packet flow experiences unexpected delays. The causes are congestion or queuing.
    \item \important{Loops}. Packets circulate endlessly. The causes are middleboxes modify headers or breaking routing logic.
\end{itemize}
These failures are:
\begin{itemize}
    \item \textbf{Not visible} in flow-level stats.
    \item \textbf{Not logged} by switches.
    \item \textbf{Hard to localize} with only endpoint observations.
\end{itemize}
\textcolor{Green3}{\faIcon{check-circle}} We need \textcolor{Green3}{\textbf{per-packet visibility}} to detect and understand them.

\highspace
\begin{flushleft}
    \textcolor{Green3}{\faIcon{question-circle} \textbf{So can we monitor every packet on the network?}}
\end{flushleft}
Tracing all packets in large datacenters is not scalable:
\begin{itemize}
    \item Aggregate traffic can exceed 100 terabit per seconds.
    \item Microsoft estimated 3200 servers needed just to collect and analyze th data (in 2015).
\end{itemize}
To make packet-level telemetry practical, some strategies are required:
\begin{enumerate}
    \item Monitoring must be \textbf{selective and smart} (e.g., sample important flows).
    \item Diagnosing problems  often requires \textbf{correlating behaviors across multiple hops}.
    \item \textbf{Passive tracing alone is insufficient}:
    \begin{itemize}
        \item It may miss transient problems.
        \item It lacks the context to localize root causes effectively.
    \end{itemize}
\end{enumerate}