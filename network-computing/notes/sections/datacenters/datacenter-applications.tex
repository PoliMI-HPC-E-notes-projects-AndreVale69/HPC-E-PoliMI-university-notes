\subsection{Datacenter Applications}

Modern datacenters host a variety of applications that range from web services to large-scale data processing. These \textbf{applications can be classified based on their traffic patterns and computational needs}.  

\highspace
\begin{flushleft}
    \textcolor{Green3}{\faIcon{question-circle} \textbf{Customer-Facing Applications (North-South Traffic)}}
\end{flushleft}
Customer-facing applications involve direct interaction with users. This type of traffic follows a \definitionWithSpecificIndex{North-South communication model}{North-South traffic}{}, meaning that \textbf{data flows between external users and the datacenter}.

\begin{examplebox}[: North-South Traffic]
    Examples include:
    \begin{itemize}
        \item \textcolor{Green3}{\textbf{Web Search}} (e.g., Google, Bing)
        \begin{itemize}
            \item A user submits a query (e.g., ``Albert Einstein'').
            \item The request is routed through the datacenter's frontend servers.
            \item Backend database and indexing servers fetch relevant results.
            \item The response is assembled and sent back to the user.
        \end{itemize}
    
        \item \textcolor{Green3}{\textbf{Social Media Platforms}} (e.g., Facebook, Instagram, X (ex Twitter))
        \begin{itemize}
            \item Users interact with content hosted in the datacenter (e.g., loading a feed, liking posts).
            \item Each interaction requires queries to databases and caching systems.
            \item Content delivery is optimized using load balancers.
        \end{itemize}
    
        \item \textcolor{Green3}{\textbf{Cloud Services}} (e.g., Google Drive, Dropbox, OneDrive)
        \begin{itemize}
            \item Users upload, store, and retrieve files.
            \item Requests must be efficiently distributed across storage nodes.
        \end{itemize}
    \end{itemize}
\end{examplebox}

\highspace
\begin{flushleft}
    \textcolor{Green3}{\faIcon{question-circle} \textbf{Large-Scale Computation (East-West Traffic)}}
\end{flushleft}
Unlike customer-facing applications, backend computations do not involve direct interaction with external users. Instead, they focus on \textbf{processing massive datasets within the datacenter}. This type of traffic is known as \definition{East-West traffic} because it occurs \textbf{between servers inside the datacenter} \emph{rather than between the datacenter and the external world}.

\begin{examplebox}[: East-West Traffic]
    Examples include:
    \begin{itemize}
        \item \textcolor{Green3}{\textbf{Big Data Processing}} (e.g., MapReduce, Hadoop, Spark)
        \begin{itemize}
            \item Large datasets are distributed across multiple servers.
            \item Each server processes a portion of the data in parallel.
            \item Results are combined to generate insights (e.g., web indexing, analytics).
        \end{itemize}
        
        \item \textcolor{Green3}{\textbf{Machine Learning \& AI Training}} (e.g., Deep Learning Models)
        \begin{itemize}
            \item AI models are trained on massive datasets using clusters of GPUs/TPUs.
            \item The process requires high-bandwidth, low-latency communication.
            \item Synchronization between nodes is critical (e.g., gradient updates in distributed training).
        \end{itemize}

        \item \textcolor{Green3}{\textbf{Distributed Storage \& Backup Systems}} (e.g., Google File System, Amazon S3)
        \begin{itemize}
            \item Data is replicated across multiple locations for reliability.
            \item Servers frequently exchange data to ensure consistency and fault tolerance.
        \end{itemize}
    \end{itemize}
\end{examplebox}

\highspace
\begin{flushleft}
    \textcolor{Green3}{\faIcon{balance-scale} \textbf{Key differences between North-South and East-West traffic}}
\end{flushleft}
\begin{table}[!htp]
    \centering
    \begin{tabular}{@{} l l l @{}}
        \toprule
        \textbf{Feature} & \textbf{N-S traffic} & \textbf{E-W traffic} \\
        \midrule
        \textbf{Direction} & External users $\leftrightarrow$ Datacenter & Within datacenter \\ [.5em]
        \textbf{Examples} & File downloads & AI training \\ [.5em]
        \textbf{Bandwidth Needs} & Moderate & Very High \\ [.5em]
        \textbf{Latency Sensitivity} & High & Critical \\ [.5em]
        \textbf{Traffic Type} & Query-response & Bulk data transfer \\
        \bottomrule
    \end{tabular}
    \caption{Differences between North-South and East-West traffic.}
\end{table}
In terms of latency sensitivity, North-South traffic is high because user interactions must be fast. On the other hand, East-West traffic is critical because synchronization delays affect computation.

\newpage

\begin{flushleft}
    \textcolor{Red2}{\faIcon{book} \textbf{Traffic Patterns and Their Impact on Networking}}
\end{flushleft}
The way data moves within a datacenter \textbf{heavily influences network design}. The main \textbf{goal} is to \textbf{ensure high bandwidth, low latency, and efficient resource utilization}.
\begin{itemize}
    \item \important{Any-to-Any Communication Model}
    \begin{itemize}
        \item In \textbf{large-scale distributed applications}, \textbf{any server should be able to communicate with any other server at full bandwidth}.
        \item \textbf{Network congestion can severely degrade performance}, especially for AI/ML workloads and big data processing.
    \end{itemize}
    \item \important{High-Bandwidth Requirements}
    \begin{itemize}
        \item Applications like \textbf{MapReduce} and \textbf{deep learning} require \textbf{high data transfer rates}.
        \item If bandwidth is insufficient, \textbf{bottlenecks occur}, leading to delays.
    \end{itemize}
    \item \important{Latency is a Critical Factor}
    \begin{itemize}
        \item \textbf{Low-latency networking is essential} for interactive applications and distributed computing.
        \item AI training, for example, requires nodes to synchronize frequently; a delay in one node slows down the entire process.
    \end{itemize}
    \item \important{Worst-Case (Tail) Latency Matters}
    \begin{itemize}
        \item It's not enough for \textbf{most requests} to be fast; \textbf{the slowest request} can delay the entire computation.
        \item \textbf{Minimizing tail latency} is crucial for efficient AI model training and database queries.
    \end{itemize}
\end{itemize}

\highspace
\begin{flushleft}
    \textcolor{Green3}{\faIcon{exclamation-triangle} \textbf{Challenges in Datacenter Traffic Management}}
\end{flushleft}
The massive scale and complexity of modern datacenters introduce \textbf{several networking challenges}, including:  
\begin{itemize}
    \item \important{Network Congestion and Bottlenecks}. When multiple servers communicate simultaneously, \textbf{some network links become overloaded}, leading to congestion.

    For example, if many AI training jobs share the same network path, it can become a bottleneck, slowing down training.

    This can be a \textbf{critical issue for applications requiring real-time performance} (e.g., financial transactions, cloud gaming).


    \item \important{Load Balancing and Traffic Engineering}. How do we distribute traffic \emph{efficiently} across network links? The solutions are: \textbf{Equal-Cost Multipath Routing} (ECMP, \hl{spreads traffic across multiple paths}); \textbf{Dynamic Traffic Engineering} (\hl{adjusts paths in real time based on congestion levels}).


    \item \important{Avoiding Link Over-Subscription}. If too many servers send data over a single link, the available \textbf{bandwidth is divided}, leading to \textbf{slow performance}. Modern datacenters aim for \textbf{full-bisection bandwidth}, meaning \textbf{any server can talk to any other server at full capacity}.


    \item \important{Scaling Challenges}. Traditional datacenter network architectures do not scale well beyond a certain point. \textbf{New network topologies} (e.g., Fat Tree, Jellyfish) are being adopted to address these limitations.
\end{itemize}

\highspace
\begin{takeawaysbox}[: Datacenter Applications]
    \begin{itemize}
        \item Datacenters handle \textbf{two major types of applications}:
        \begin{enumerate}
            \item \textbf{Customer-facing applications (North-South traffic)} involve external users.
            \item \textbf{Large-scale computations (East-West traffic)} occur within the datacenter.
        \end{enumerate}  
        \item \textbf{Traffic patterns} affect \textbf{bandwidth, latency, and congestion control}.  
        \item \textbf{Managing congestion and ensuring high bandwidth} is critical for performance.  
        \item \textbf{New network topologies and routing techniques} help address scaling challenges.  
    \end{itemize}
\end{takeawaysbox}