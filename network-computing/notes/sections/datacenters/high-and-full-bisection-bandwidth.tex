\subsection{High and Full-Bisection Bandwidth}

\begin{flushleft}
    \textcolor{Green3}{\faIcon{question-circle} \textbf{Why is High-Bandwidth important in Datacenters?}}
\end{flushleft}
Modern datacenters \textbf{handle massive amounts of data} due to applications like AI training, cloud services, and big data processing. These workloads \emph{require}:
\begin{itemize}
    \item \textbf{High-bandwidth connections} to support fast data transfers.
    \item \textbf{Low latency} to ensure real-time performance.
    \item \textbf{Scalability} to accommodate increasing workloads.
\end{itemize}
Unlike traditional networks, where traffic primarily flows between users and servers (North-South), \textbf{datacenters experience heavy East-West traffic} (server-to-server communication). This shift \textbf{demands high-bandwidth and scalable network designs}.

\highspace
\begin{flushleft}
    \textcolor{Green3}{\faIcon{question-circle} \textbf{\emph{One step at a time}: What a Bisection Bandwidth is and why Full-Bisection Bandwidth is important}}
\end{flushleft}
\definition{Bisection Bandwidth} is a key metric that measures the \textbf{total bandwidth available between two halves of a network}.
\begin{definitionbox}[: Bisection Bandwidth]
    If a network is split into two equal halves, the \definition{Bisection Bandwidth} is the \textbf{total data transfer rate available between them}.
\end{definitionbox}
\begin{definitionbox}[: Full-Bisection Bandwidth]
    The \definition{Full-Bisection Bandwidth} is when every server can communicate with every other server at \textbf{full network speed}.
\end{definitionbox}

\noindent
In other words, bisection bandwidth can be thought of as cutting a data center network in half and measuring the total capacity of the links connecting the two halves. This tells us how much data can flow between the two sections simultaneously.

\begin{examplebox}[: Understand what bisection bandwidth is]
    Imagine a 1000-server datacenter, where 500 servers are processing data while 500 servers store the results. If the bisection bandwidth is \textbf{low}, the \textbf{data transfer between processing and storage nodes will be delayed}. This results in slow machine learning model training or delayed database queries.
\end{examplebox}

\noindent
As we can imagine, the full-bisection bandwidth is a real and critical aspect:
\begin{itemize}
    \item \important{Prevents bottlenecks}: Ensures high-throughput communication across racks and clusters.
    \item \important{Essential for AI/ML training}: AI models require massive parallel computations with continuous data exchanges.
    \item \important{Optimized for cloud computing}: Services like AWS, Google Cloud, and Azure depend on fast, reliable inter-server communication.
\end{itemize}

\highspace
\begin{flushleft}
    \textcolor{Red2}{\faIcon{exclamation-triangle} \textbf{Then try to get high-bandwidth all the time! Yes, but there are some challenges...}}
\end{flushleft}
Ideally, high-bandwidth should be the ultimate goal, but unfortunately, there are some problems with traditional three-based networks:
\begin{itemize}[label=\textcolor{Red2}{\faIcon{times}}]
    \item \textcolor{Red2}{\textbf{The Problem with Traditional Three-Based Networks}}. The standard \textbf{three-tier (core-aggregation-edge) topology} struggles to scale due to:
    \begin{enumerate}
        \item \textbf{Over-subscription} (definition on page \hqpageref{def: over-subscription problem}): The ratio of available bandwidth to required bandwidth is too high.
        \item \textbf{Core congestion}: Core routers become bottlenecks as traffic grows.
        \item \textbf{Single points of failure}: A failure in a core switch can affect a large portion of the datacenter.
    \end{enumerate}

    \item \textcolor{Red2}{\textbf{Over-Subscription and Its Impact on Network Performance}}. A naive solution would be to use over-subscription to solve these problems, but this limits performance. \definition{Over-Subscription} happens when the \textbf{network is provisioned with less bandwidth than needed} to cut costs.
    \begin{equation*}
        \text{Over-subscription} = \dfrac{\text{Total server bandwidth demand}}{\text{Available bandwidth at aggregation/core layer}}
    \end{equation*}
    Common over-subscription ratios are:
    \begin{itemize}
        \item 5:1, only 20\% of host bandwidth is available.
        \item 50:1, only 2\% of host bandwidth is available.
        \item 500:1, only 0.5\% of host bandwidth is available.
    \end{itemize}
    At 500:1 over subscription, congestion becomes severe, \textbf{limiting network efficiency}.
    
    \item \textcolor{Red2}{\textbf{The cost problem: scaling is expensive!}}
    \begin{itemize}
        \item Increasing bisection bandwidth requires \textbf{more high-performance network hardware}.
        \item \textbf{Scaling traditional networks} (adding more core switches) is extremely costly.
        \item \textbf{Energy consumption rises} with additional hardware.
    \end{itemize}
\end{itemize}
Thus, \textbf{alternative solutions} are needed to achieve high-bandwidth networking \textbf{without excessive costs}.

\newpage

\begin{flushleft}
    \textcolor{Green3}{\faIcon{check-circle} \textbf{Solutions to Achieve High and Full-Bisection Bandwidth}}
\end{flushleft}
To overcome these challenges, researchers and engineers have designed \textbf{new network architectures}.
\begin{itemize}[label=\textcolor{Green3}{\faIcon{check}}]
    \item \textcolor{Green3}{\textbf{Fat Tree (Clos Network) - The Scalable Solution}}. Unlike traditional three-based designs, Fat Tree provides \textbf{multiple paths} for traffic.
    
    \textcolor{Green3}{\faIcon{check-circle} \textbf{Advantages}}
    \begin{itemize}[label=\textcolor{Green3}{\faIcon{check}}]
        \item \textbf{Ensure full-bisection bandwidth} by allowing traffic to take alternative routes.
        \item \textbf{Eliminates single points of failure} using redundant paths.
        \item \textbf{Load balancing} optimizes network utilization.
    \end{itemize}


    \item \textcolor{Green3}{\textbf{Jellyfish - A More Flexible Approach}}. Uses a \textbf{randomized, non-hierarchical} topology instead of a fixed three structure.

    \textcolor{Green3}{\faIcon{check-circle} \textbf{Advantages}}
    \begin{itemize}[label=\textcolor{Green3}{\faIcon{check}}]
        \item \textbf{Better bandwidth scaling} as new servers are added.
        \item \textbf{More resilient to failures} (no single critical point of failure).
    \end{itemize}


    \item \textcolor{Green3}{\textbf{BCube - Optimized for Cloud Services}}. Designed for high-performance cloud environments with \textbf{massive inter-server communication}.
    
    \textcolor{Green3}{\faIcon{check-circle} \textbf{Advantages}}
    \begin{itemize}[label=\textcolor{Green3}{\faIcon{check}}]
        \item \textbf{Fast re-routing} in case of failures.
        \item \textbf{Low-latency communication for cloud applications}.
    \end{itemize}
\end{itemize}

\highspace
\begin{takeawaysbox}[: High and Full-Bisection Bandwidth]
    \begin{itemize}
        \item \textbf{High-bandwidth networking} is essential for modern datacenters.
        \item \textbf{Full-bisection bandwidth} ensures servers communicate at \textbf{full speed}.
        \item \textbf{Over-subscription} creates \textbf{bottlenecks}, limiting performance.
        \item \textbf{New network architectures} (Fat Tree, Jellyfish, BCube) solve scalability issues.
    \end{itemize}
\end{takeawaysbox}
