\section{Datacenters}

\subsection{What is a Datacenter?}
A \definition{Datacenter} is a specialized \textbf{facility that houses multiple computing resources}, including servers, networking equipment, and storage systems. These \textbf{resources are co-located} (placed together in the same physical location) to ensure \hl{efficient operations}, \hl{leverage shared environmental controls} (such as cooling and power), and \hl{maintain physical security}.

\highspace
So the main characteristics are:
\begin{itemize}
    \item \textbf{Centralized Infrastructure}: Unlike traditional computing models where resources are scattered, datacenters consolidate thousands to millions of machines in a single administrative domain.
    \item \textbf{Full Control over Network and Endpoints}: Datacenters operate under a single administrative entity, \hl{allowing customized configurations} beyond conventional network standards.
    \item \textbf{Traffic Management}: Unlike the open Internet, datacenter traffic is highly structured, and the \hl{organization can define routing, congestion control, and network security policies}.
\end{itemize}

\begin{table}[!htp]
    \centering
    \begin{tabular}{@{} l p{12.8em} p{12.8em} @{}}
        \toprule
        \textbf{Feature} & \textbf{Datacenter Networks} & \textbf{Traditional Networks} \\
        \midrule
        \textbf{Ownership}  & Fully controlled by a single organization & Usually spans multiple independent ISPs \\ [.5em]
        \textbf{Traffic}    & High-speed internal communication (east-west traffic) & Lower-speed, external client-based traffic (north-south) \\ [.5em]
        \textbf{Routing}    & Customizable (non standard protocols) & Uses standard internet protocols (BGP, OSPF, etc.) \\ [.5em]
        \textbf{Latency}    & Optimized for ultra-low latency & Variable latency, dependent on ISPs \\ [.5em]
        \textbf{Redundancy} & High redundancy to ensure failover and fault tolerance & Often limited by ISP policies \\
        \bottomrule
    \end{tabular}
    \caption{Difference between Datacenters and other networks (e.g., LANs).}
\end{table}

\highspace
\begin{flushleft}
    \textcolor{Green3}{\faIcon{question-circle} \textbf{Why are datacenters important?}}
\end{flushleft}
Datacenters are the backbone of modern cloud computing, large-scale data processing, and AI/ML workloads. They provide \textbf{high computational power and storage} for various applications, such as:
\begin{enumerate}
    \item \important{Web Search \& Content Delivery}. For example, when a user searches for ``Albert Einstein'' on Google, the request is processed in a datacenter where:
    \begin{enumerate}
        \item The query is parsed and sent to multiple servers.
        \item Indexed data is retrieved.
        \item A ranked list of results is generated and sent back to the user.
    \end{enumerate}

    \item \important{Cloud Computing}. Services like Amazon Web Services (AWS), Microsoft Azure, and Google Cloud offer computation, storage, and networking resources on-demand.
    \begin{itemize}
        \item Infrastructure as a Service (IaaS): Virtual machines, storage, and networking.
        \item Platform as a Service (PaaS): Databases, development tools, AI models.
        \item Software as a Service (SaaS): Google Drive, Microsoft Office 365.
    \end{itemize}

    \item \important{AI and Big Data Processing}. Large-scale computations like MapReduce and deep learning training rely on distributed datacenter resources.

    \item \important{Enterprise Applications}. Datacenters host internal IT infrastructure for businesses, including databases, ERP systems, and virtual desktops.
\end{enumerate}

\highspace
\begin{flushleft}
    \textcolor{Green3}{\faIcon{history} \textbf{Evolution of Datacenters}}
\end{flushleft}
While the concept of centralized computing dates back to the 1960s, the modern datacenter model emerged with cloud computing in the 2000s. Notable developments include:
\begin{itemize}
    \item 1970s: IBM mainframes operated in controlled environments similar to early datacenters.
    \item 1990s: Rise of client-server computing required dedicated server rooms.
    \item 2000s-Present: Hyperscale datacenters by Google, Microsoft, and Amazon revolutionized networking, storage, and scalability.
\end{itemize}

\highspace
\begin{flushleft}
    \textcolor{Green3}{\faIcon{\speedIcon} \textbf{What's new in Datacenters?}}
\end{flushleft}
Datacenters have been around for decades, but modern datacenters have undergone significant changes in scale, architecture, and service models. The primary \textbf{factors driving these changes} include:
\begin{enumerate}[label=\textcolor{Green3}{\faIcon{check}}]
    \item The exponential \textbf{growth of internet services} (Google, Facebook, Amazon, etc.).
    \item The \textbf{shift to cloud computing} and on-demand services.
    \item The need for \textbf{better network scalability, fault tolerance, and efficiency}.
\end{enumerate}

\newpage

\noindent
One of the most striking changes in modern data centers is their massive scale:
\begin{itemize}
    \item Companies like Google, Microsoft, Amazon, and Facebook operate \textbf{datacenters with over a million servers at a single site}.
    \item \textbf{Microsoft alone has more than 100,000 switches and routers} in some of its datacenters.
    \item \textbf{Google processes billions of queries per day}, requiring vast computational resources.
    \item \textbf{Facebook and Instagram serve billions of active users}, with every interaction generating requests to datacenters.
\end{itemize}
Another major change is the \textbf{shift from owning dedicated computing infrastructure} to \textbf{renting scalable cloud resources}. Datacenters no longer just host enterprise applications, \textbf{they now offer computing, storage, and network infrastructure as a service}. The most common cloud computing models are:
\begin{itemize}
    \item \textbf{Infrastructure as a Service (IaaS)}. User rent virtual machines (VMs), storage, and networking instead of maintaining their own physical servers (e.g., Amazon EC2).
    \item \textbf{Platform as a Service (PaaS)}. Provides a platform with pre-configured environments for software development (databases, frameworks, etc.).
    \item \textbf{Software as a Service (SaaS)}. Full software applications hosted in datacenters and delivered via the internet (e.g., Google Drive).
\end{itemize}
The move to cloud computing has fundamentally changed datacenters, shifting the focus to resource allocation, security, and performance guarantees. They are also moving from multi-tenancy to single-tenancy:
\begin{itemize}
    \item \definition{Single-Tenancy}. A client gets \textbf{dedicated infrastructure} for their services.
    \item \definition{Multi-Tenancy}. \textbf{Resources are shared among multiple clients while ensuring isolation}.
\end{itemize}
\textcolor{Red2}{\faIcon{times-circle} \textbf{Implications}}. But this massive scale brings new challenges:
\begin{itemize}
    \item \important{Scalability}: The need for \textbf{efficient network designs} to handle rapid growth.
    
    Traditional datacenter topologies, such as three-based architectures, are inefficient at scale. New designs, like \textbf{Clos-based networks (Fat Tree)} and \textbf{Jellyfish (random graphs)}, are being developed to:
    \begin{itemize}[label=\textcolor{Green3}{\faIcon{check}}]
        \item Ensure \textbf{high bisection bandwidth} (allow any-to-any communication efficiently).
        \item Provide \textbf{scalable and fault-tolerant networking}.
    \end{itemize}
    
    \newpage

    \item \important{Cost management}: More machines mean \textbf{higher power, cooling, and hardware costs}.
    
    Datacenters are \textbf{expensive to build and maintain}, requiring:
    \begin{itemize}
        \item \textbf{Efficient resource utilization} (prevent idle servers from wasting power).
        \item \textbf{Energy-efficient cooling solutions} (cooling accounts for a \emph{huge} portion of operational costs).
        \item \textbf{Automation to reduce human intervention} (e.g., AI-based network optimization).
    \end{itemize}
    
    
    \item \important{Reliability}: Hardware failures become \textbf{common at scale}, requiring \textbf{automated fault-tolerant solutions}.
    
    At the scale of modern datacenters, \textbf{hardware and software failures are common}. A key principle is: ``\emph{In large-scale systems, failures are the norm rather than the exception.}'' (Microsoft, ACM SIGCOMM 2015).

    Thus, new \textbf{automated failover mechanisms} are required to:
    \begin{itemize}
        \item Detect failures \textbf{quickly}.
        \item Redirect traffic \textbf{seamlessly}.
        \item Ensure \textbf{minimal service disruption}.
    \end{itemize}

    \item \important{Performance \& Isolation Guarantees}: In modern datacenters, \textbf{customers expect strict performance guarantees} for applications like: low-latency financial transactions, high-bandwidth video streaming, machine learning model training.

    To meet these demands, datacenters implement:
    \begin{itemize}[label=\textcolor{Green3}{\faIcon{check}}]
        \item \textcolor{Green3}{\textbf{Performance Guarantees}}: Allocating bandwidth and compute power dynamically.
        \item \textcolor{Green3}{\textbf{Isolation Guarantees}}: Ensuring one user's workload does not interfere with another's.
    \end{itemize}

    But this requires \textbf{advanced networking techniques}, such as:
    \begin{itemize}
        \item \textbf{Traffic engineering} to avoid congestion.
        \item \textbf{Load balancing} to distribute workloads efficiently.
        \item \textbf{Software-defined networking (SDN)} for centralized control over traffic flows.
    \end{itemize}
\end{itemize}

\newpage

\begin{takeawaysbox}[: What is a Datacenter?]
    \begin{itemize}
        \item \textbf{Datacenters centralize} computing resources for performance, security, and scalability.
        \item \textbf{They differ from traditional networks} by offering more control, lower latency, and higher redundancy.
        \item \textbf{Applications include cloud services, AI, and enterprise computing}.
        \item \textbf{Scalability is a key challenge}, with hyperscale datacenters hosting millions of machines.
        \item \textbf{Efficiency and cost containment are major concerns}, requiring innovative architectures.
    \end{itemize}
\end{takeawaysbox}
