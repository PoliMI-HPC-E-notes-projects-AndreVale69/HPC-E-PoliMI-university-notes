\subsubsection{Programmable NICs (SmartNICs)}

A \definition{SmartNIC} is a Network Interface Card (NIC) that includes \textbf{programmable compute resources}, allowing it to perform \textbf{packet processing and other tasks independently of the host CPU}. Unlike a traditional NIC that only transfers packets (DMA) and performs simple offloads (e.g., checksum offloading), a SmartNIC can execute programs directly on the NIC, process packets, run custom logic, and even make decisions about how to handle network traffic.

\highspace
\begin{flushleft}
    \textcolor{Green3}{\faIcon{question-circle} \textbf{What makes it ``smart''?}}
\end{flushleft}
A SmartNIC is considered ``smart'' because it typically includes:
\begin{itemize}
    \item \textbf{Onboard CPU}: A SmartNIC often has its own CPU (or multiple cores) that can run software independently of the host CPU. This allows it to perform complex processing tasks without involving the host.
    \item \textbf{FPGA}: Some SmartNICs use Field-Programmable Gate Arrays (FPGAs) to allow for highly customizable packet processing. FPGAs can be programmed to implement specific logic for handling network traffic, enabling very high performance and low latency.
    \item \textbf{ASIC with programmable pipeline}: Some SmartNICs are built on Application-Specific Integrated Circuits (ASICs) that include a program-\break{}mable pipeline for packet processing. This allows for efficient handling of network traffic while still providing flexibility in how packets are processed.
\end{itemize}
So the ``smartness'' of a SmartNIC comes from its ability to \textbf{perform complex processing tasks directly on the NIC}, reducing the load on the host CPU and improving overall network performance. It's not just a NIC anymore; it's a small computer that can handle network traffic in a more intelligent way.

\highspace
\begin{flushleft}
    \textcolor{Green3}{\faIcon{question-circle} \textbf{What are the benefits of using SmartNICs?}}
\end{flushleft}
The benefits of using SmartNICs include:
\begin{itemize}
    \item \textbf{Reduced CPU Load}: By offloading network processing tasks to the SmartNIC, the host CPU can focus on other tasks, improving overall system performance.
    \item \textbf{Lower Latency}: SmartNICs can process packets directly on the NIC, reducing the time it takes for packets to be processed and improving latency.
    \item \textbf{Enhanced Functionality}: SmartNICs can be programmed to perform specific tasks such as encryption, compression, or custom packet processing, providing enhanced functionality that may not be available in traditional NICs.
    \item \textbf{Improved Security}: SmartNICs can be used to implement security features such as firewalls, intrusion detection systems, or DDoS mitigation directly on the NIC, providing an additional layer of security.
    \item \textbf{Better Scalability}: In high-performance computing environments, the SmartNICs can help scale network performance by offloading tasks that would otherwise consume host CPU resources, allowing for better scalability as network traffic increases.
\end{itemize}
Overall, SmartNICs offer a powerful way to enhance network performance and functionality by leveraging programmable compute resources directly on the NIC, enabling a wide range of applications and use cases in modern data centers and high-performance computing environments.