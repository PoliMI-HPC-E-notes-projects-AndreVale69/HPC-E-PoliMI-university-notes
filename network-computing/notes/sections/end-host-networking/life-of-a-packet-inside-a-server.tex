\subsection{Life of a Packet Inside a Server}

When a packet arrives at a server, it does \textbf{not} go directly to the application. Instead, it must traverse a \textbf{layered processing pipeline}, which includes several steps:
\begin{itemize}
    \item Hardware boundaries (NIC, bus, memory).
    \item Protection domains (kernel, user space).
    \item Software abstractions (network stack, sockets, libraries).
\end{itemize}
Each step adds \textbf{latency}, \textbf{CPU overhead}, and \textbf{potential bottlenecks}. Understanding this pipeline is crucial for optimizing network performance.

\highspace
\begin{flushleft}
    \textcolor{Green3}{\faIcon{map-signs} \textbf{The baseline packet path}}
\end{flushleft}
At a very high level, an incoming packet follows this path:
\begin{enumerate}
    \item \important{Network} (i.e., the wire). It represents the external world where packets are transmitted.
    \item \important{NIC (Network Interface Card)}. The \textbf{NIC} is the hardware component connected to the network. Its responsibilities include:
    \begin{itemize}
        \item Receiving packets from the wire.
        \item Storing them temporarily in NIC memory.
        \item Performing basic checks (e.g., checksum offload).
        \item Initializing data transfer to host memory.
    \end{itemize}
    The NIC operates \textbf{independently of the CPU}, it cannot interpret application data. So it is the \textbf{first bottleneck} in the packet path: \hl{if it cannot sustain line rate, everything else is useless}.
    \item \important{PCIe bus} (Peripheral Component Interconnect Express, connecting NIC to memory). The NIC is \textbf{not directly connected} to the CPU or main memory. Packets must cross the \textbf{PCIe bus} using \textbf{DMA (Direct Memory Access)} to transfer data to host memory. The PCIe bus has its own \textbf{bandwidth and latency characteristics}, which can impact performance.
    \item \important{NIC Driver} (i.e., software managing the NIC). The \textbf{driver} is kernel code that manages the NIC, programs hardware registers, handles interrupts and moves packets into kernel data structures. It acts as the \textbf{software interface between hardware and kernel}. It runs in \textbf{kernel context}, it is executed frequently and bugs or inefficiencies here can have a large impact, since it is on the \textbf{critical path} of packet processing (i.e., every packet must go through it).
    \item \important{Kernel Networking Stack} (operating system's network processing). The \textbf{network stack} is a complex software layer that implements network protocols (e.g., TCP/IP). It is responsible for:
    \begin{itemize}
        \item Protocol handling (e.g., TCP state machine).
        \item Packet reassembly.
        \item Congestion control.
        \item Routing.
    \end{itemize}
    This is where security checks happen, congestion control lives, but also packet ordering is enforced. The kernel provides \textbf{generality and safety}, but at the cost of \textbf{performance overhead} due to context switches, data copies, and protocol processing.
    \item \important{User-space Application} (i.e., the server application). Eventually, the packet reaches a socket buffer or a user-space application via a system call (e.g., \texttt{recvfrom()}). Crossing from kernel to user space involves \textbf{context switches} and \textbf{data copies}, which add latency and CPU overhead. The user-kernel boundary crossing is expensive and unavoidable in the baseline model.
\end{enumerate}
This is the \textbf{baseline model}, all optimizations later in the notes aim to \textbf{reduce the cost of one or more of these steps}.

\begin{figure}[!htp]
    \centering
    \begin{tikzpicture}[
        font=\small,
        node distance=7mm,
        box/.style={
            draw,
            rounded corners=7pt,
            line width=0.6pt,
            minimum width=6.0cm,
            minimum height=1.05cm,
            align=center,
            inner sep=6pt
        },
        hw/.style={box, fill=gray!10},
        sw/.style={box, fill=blue!10},
        arrow/.style={-{Stealth[length=2.3mm]}, line width=0.75pt},
        region/.style={
            draw,
            dashed,
            line width=0.6pt,
            rounded corners=10pt,
            inner sep=10pt
        },
        regionLabel/.style={
            font=\footnotesize,
            fill=white,
            inner sep=2pt
        }
    ]

        % ---- Nodes ----
        \node[font=\footnotesize] (title) {Incoming packet from network};

        \node[hw, below=3mm of title] (nic) {Network Interface Card (NIC)};
        \node[hw, below=of nic] (pcie) {PCIe interconnect\\(DMA transfer)};

        \node[sw, below=of pcie] (drv) {NIC Driver};
        \node[sw, below=of drv] (kern) {Kernel networking stack\\(IP/TCP/UDP, routing, filtering)};

        \node[sw, below=of kern] (app) {User-space application\\(socket API)};

        % ---- Arrows ----
        \draw[arrow] (nic) -- (pcie);
        \draw[arrow] (pcie) -- (drv);
        \draw[arrow] (drv) -- (kern);
        \draw[arrow] (kern) -- (app);

        % ---- Regions (kernel + user space) ----
        \begin{pgfonlayer}{background}
            \node[region, fit=(drv) (kern)] (kregion) {};
            \node[region, fit=(app)] (uregion) {};
        \end{pgfonlayer}

        \node[regionLabel, anchor=west] at ([xshift=2mm]kregion.north west) {kernel space};
        \node[regionLabel, anchor=west] at ([xshift=2mm]uregion.north west) {user space};

    \end{tikzpicture}
    \caption{Baseline packet path inside a server.}
    \label{fig:baseline-endhost-path}
\end{figure}