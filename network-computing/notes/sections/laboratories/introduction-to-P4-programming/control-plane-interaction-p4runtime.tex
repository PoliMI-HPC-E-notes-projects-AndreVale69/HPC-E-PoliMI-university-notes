\subsubsection{Control Plane Interaction (\texttt{P4Runtime})}\label{subsection: Control Plane Interaction P4Runtime}

\begin{flushleft}
    \textcolor{Green3}{\faIcon{question-circle} \textbf{Why do we need a control plane at all?}}
\end{flushleft}
After compiling and loading a P4 program, the switch knows \textbf{how} to process packets, but it does \textbf{not know what rules to apply}. In other words, actions and tables are defined, but they are \textbf{empty} until the control plane installs entries. So we need a way for the control plane to interact with the switch at runtime, to install rules, update policies, and manage the network.

\highspace
\begin{flushleft}
    \textcolor{Green3}{\faIcon{book} \textbf{What is \texttt{P4Runtime}?}}
\end{flushleft}
\definition{P4Runtime} is a \textbf{standard API} that allows a control plane to communicate with a P4 target, configure its table at runtime, and read or update its state. It is target-independent, architecture-aware, and it is designed specifically for P4-programmable devices. It is like a \textbf{bridge} between the high-level network logic and low-level packet processing.
\begin{itemize}
    \item[\textcolor{Green3}{\faIcon{check}}] \textcolor{Green3}{\textbf{What it does}}. Using \texttt{P4Runtime}, the control plane can:
    \begin{itemize}
        \item Insert table entries (e.g., match-action rules) into the switch.
        \item Modify or delete entries as needed (e.g., for dynamic policies).
        \item Read counters and registers to get telemetry or state information from the switch.
        \item React to events (e.g., packet-in messages) generated by the switch.
    \end{itemize}
    \item[\textcolor{Red2}{\faIcon{times}}] \textcolor{Red2}{\textbf{What it does NOT do}}. \texttt{P4Runtime} does \textbf{not}:
    \begin{itemize}
        \item Change the P4 program logic (e.g., add new headers or tables); that requires recompilation and reloading.
        \item Modify the parser or the overall data plane architecture; those are fixed by the P4 program.
        \item Redefine tables or actions; it can only populate them with entries defined by the P4 program.
    \end{itemize}
    Those are \textbf{compile-time decisions}.
\end{itemize}