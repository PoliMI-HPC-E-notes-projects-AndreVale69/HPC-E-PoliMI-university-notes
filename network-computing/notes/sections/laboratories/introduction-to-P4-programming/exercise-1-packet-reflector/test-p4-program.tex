\paragraph{Test P4 program}

To test the P4 program, we can type the following command in the terminal:
\begin{lstlisting}
sudo python3 network_topo.py\end{lstlisting}
This command will start the Mininet network topology defined in the \texttt{network\_topo.py} file, which includes the P4 switch running the packet reflector program. Once the topology is up and running, we can use Mininet's CLI to interact with the hosts and test the packet reflection functionality.

\begin{figure}[!htp]
    \centering
    \includegraphics[width=\textwidth]{img/mininet_cli.png}
    \caption{Mininet CLI for testing the P4 program.}
\end{figure}

\noindent
For example, we can use the following commands in the Mininet CLI to test the connectivity between the host \texttt{h1} and the switch \texttt{s1}:
\begin{lstlisting}
mininet> h1 ping s1\end{lstlisting}
However, after running the network topology, we will write \texttt{xterm h1} to open a terminal for host \texttt{h1}. In this terminal, we can run the following command to test the packet reflection functionality:
\begin{lstlisting}
python3 test.py\end{lstlisting}
This command will execute the \texttt{test.py} script, which sends a packet from host \texttt{h1} to the switch \texttt{s1} and checks if the packet is correctly reflected back to the host. If the test is successful, we should see a message indicating that the packet was received and reflected correctly. If there are any issues with the P4 program, we may see error messages or the test may fail, indicating that the packet was not reflected as expected. A correct output will look like this:

\begin{figure}[!htp]
    \centering
    \includegraphics[width=\textwidth]{img/test_output.png}
    \caption{Output of the test script for the packet reflector P4 program.}
\end{figure}