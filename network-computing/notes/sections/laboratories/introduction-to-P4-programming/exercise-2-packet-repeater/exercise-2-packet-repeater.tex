\subsubsection{Exercise 2: Packet Repeater}\label{sec:exercise-2-packet-repeater}

The goal of \textbf{Exercise 2} is to build a small network with:
\begin{itemize}
    \item Two \textbf{hosts} (\texttt{h1}, \texttt{h2}).
    \item One P4 \textbf{switch} (\texttt{s1}).
    \item A \textbf{control switch} that programs the P4 switch at runtime using\break \texttt{P4Runtime}.
\end{itemize}
The P4 switch must behave as a \textbf{packet repeater}:
\begin{itemize}
    \item When a packet enters \textbf{port 1}, it must be forwarded to \textbf{port 2}.
    \item When a packet enters \textbf{port 2}, it must be forwarded to \textbf{port 1}.
\end{itemize}
So, unlike Exercise 1 (\autopageref{sec:exercise-1-packet-reflector}):
\begin{itemize}
    \item Forwarding is \textbf{not hard-coded} in the P4 program
    \item Forwarding decisions are installed \textbf{by the controller} using \textbf{P4Runtime}
\end{itemize}

\highspace
\begin{flushleft}
    \textcolor{Green3}{\faIcon{question-circle} \textbf{What files are given?}}
\end{flushleft}
The exercise provides the following files:
\begin{itemize}
    \item \texttt{network\_topo.py} (topology and deployment). This file is responsible for \textbf{creating and starting the network}. It will contain 1 P4 switch (\texttt{s1}), 2 hosts (\texttt{h1}, \texttt{h2}), and the necessary links.
    
    \item \texttt{packet\_repeater.p4} (data plane program). This file defines \textbf{what the switch is capable of doing}, not the actual forwarding rules.
    
    \item \texttt{control\_plane.py} (runtime control logic). This file is the \textbf{control plane} of the network. It connects to the P4 switch using \textbf{P4Runtime} and installs rules. Here is where you specify the actual forwarding policy (e.g., ``packets entering port 1 should be forwarded to port 2'').
    
    \item \texttt{send.py} (traffic generator). This file is used to \textbf{send packets from a host}. It runs on \texttt{h1}, crafts an Ethernet plus IP packet, and sends it to a destination IP (e.g., \texttt{h2}) using Scapy. It does \textbf{not} interact with P4 directly. Generates traffic to test the network. This file is \textbf{complete} and should not be modified.
    
    \item \texttt{receive.py} (traffic observer). This file is used to \textbf{observe packets arriving at a host}. It runs on \texttt{h2}, sniffs packets on \texttt{eth0}, and prints Ethernet, IP, and payload fields. It filters out packets sent by itself. Verifies whether forwarding worked correctly. This file is \textbf{complete} and should not be modified.
\end{itemize}