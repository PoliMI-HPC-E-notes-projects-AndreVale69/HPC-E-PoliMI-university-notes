\section{Laboratories}

\subsection{Introduction to P4 Programming}

The main goal of this laboratory is to \textbf{introduce the P4 programming language} and make us familiar with:
\begin{itemize}
    \item How a \textbf{programmable switch} is structured,
    \item How packets are \textbf{parsed, processed, and reconstructed},
    \item How to implement \textbf{simple packet-processing behaviors} in the data plane.
\end{itemize}
This is \textbf{not} about performance or optimization yet, but about \textbf{understanding the model}.

\highspace
\textcolor{Green3}{\faIcon[regular]{lightbulb} \textbf{Conceptual goal.}} Until now, from theory side, we learned that networks are no longer just ``\emph{dumb pipes}'' (i.e., simple forwarding devices), but rather \emph{programmable} entities that can be \textbf{customized} to implement a variety of functionalities. This lab makes that idea \emph{concrete}.

\highspace
\textcolor{Green3}{\faIcon{tools} \textbf{Practical goal.}} By the end of this lab, we should be able to:
\begin{itemize}
    \item Read a \textbf{P4 program} and understand its structure,
    \item Identify:
    \begin{itemize}
        \item Where headers are parsed,
        \item Where decisions are taken,
        \item Where packets are emitted,
    \end{itemize}
    \item Write \textbf{basic P4 programs} that reflect packets, repeat packets, and handle VLAN tags.
\end{itemize}
These exercises are intentionally simple so that \textbf{the focus is on the language and architecture}, not on complex networking logic.

