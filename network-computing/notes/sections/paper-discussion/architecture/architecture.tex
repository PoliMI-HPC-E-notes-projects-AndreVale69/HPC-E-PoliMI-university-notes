\subsection{Architecture}

At a high level, the architecture of Fastpass is quite simple.

\begin{figure}[!htp]
    \centering
    \includegraphics[width=.7\textwidth]{img/fastpass-arch-2.pdf}
    \caption{Architecture of Fastpass.\cite{10.1145/2740070.2626309}}
\end{figure}

\noindent
It has five main components:
\begin{itemize}
    \item \important{Endpoints} are normal servers that send and receive traffic, placed in the datacenter. However, unlike TCP:
    \begin{itemize}
        \item[\textcolor{Red2}{\faIcon{times}}] They \textbf{do not freely send packets} to the network.
        \item[\textcolor{Green3}{\faIcon{check}}] They must first request permission from the arbiter.
    \end{itemize}
    So the workflow is:
    \begin{enumerate}
        \item Endpoint has a packet to send.
        \item It sends a request to the arbiter.
        \item Arbiter assigns: a transmission timeslot and a path to the destination.
        \item Endpoint sends only in that assigned timeslot, and only on that assigned path.
    \end{enumerate}
    In this way, the endpoints are \textbf{obedient transmitter}, not congestion controllers. However, \hl{the endpoints must be modified (kernel modifications) to implement this behavior. Instead, the switches and the network are completely unmodified.} We will discuss this in more detail in the FCP section.
    
    
    \item \important{Central Arbiter} is the \textbf{brain} of Fastpass. Its responsibilities are:
    \begin{itemize}
        \item Maintain a global view of who wants to send what, and the network topology (i.e., the state of the network).
        \item Compute matchings for each timeslot, i.e., which endpoints can transmit at the same time without causing congestion.
        \item Assign paths to the endpoints, i.e., which path they should use to send their packets.
        \item Ensure no link conflict occurs, i.e., that no two endpoints are assigned to transmit on the same link at the same time.
    \end{itemize}
    It runs as a \textbf{software service} on a \textbf{commodity server} (i.e., it is not a specialized hardware component). The software is \textbf{highly optimized} to make fast decisions, and it is \textbf{multi-core scalable} to handle a large number of requests.
    
    
    \item \important{Timeslot Allocator}. The \hl{time is divided into fixed-size slots.} Each slot corresponds to one MTU (Maximum Transmission Unit)\footnote{
        MTU is the maximum size of a packet that can be transmitted on the network. For example, if the MTU is 1500 bytes, then each timeslot is long enough to transmit one 1500-byte packet.
    } transmission time on the slowest link in the network.
    
    For each timeslot, the arbiter chooses a \textbf{set of endpoints} (sender $\to$ receiver pairs) such that:
    \begin{itemize}
        \item Each sender sends at most 1 packet.
        \item Each receiver receives at most 1 packet.
    \end{itemize}
    For example, suppose we have 3 senders ($A$, $B$, $C$) and 3 receivers ($X$, $Y$, $Z$). In one timeslot, the arbiter might assign:
    \begin{itemize}
        \item $A$ wants to send to $X$.
        \item $B$ wants to send to $Z$.
        \item $C$ wants to send to $Y$.
    \end{itemize}
    Although this \textbf{constraint ensures there are no conflicts at the endpoints}, it does not guarantee there are no conflicts in the network (i.e., two paths may share a link). This is where the path selector comes in.
    
    
    \item \important{Path Selector} is \textbf{responsible for assigning a path to each sender-receiver pair selected by the timeslot allocator}. The path selector must ensure that \hl{no two packets use the same internal link in same timeslot}, otherwise there would be a conflict and congestion would occur. For example, if $A$ sends to $X$ and $B$ sends to $Z$, and both paths share a link, then there would be a conflict on that link.
    
    
    \item \important{Fastpass Control Protocol (FCP)} is the \textbf{protocol used by the endpoints to communicate with the arbiter} and vice versa. It is a \textbf{lightweight protocol} designed for low latency and high throughput. It is \hl{implemented in the kernel of the endpoints}, and it allows them to:
    \begin{itemize}
        \item Send requests to the arbiter when they have packets to send.
        \item Receive assignments from the arbiter (timeslot and path).
        \item Transmit packets according to the assignments.
    \end{itemize}
    The FCP is designed to be \textbf{simple and efficient}, with minimal overhead, to ensure that the communication between endpoints and arbiter does not become a bottleneck.
\end{itemize}