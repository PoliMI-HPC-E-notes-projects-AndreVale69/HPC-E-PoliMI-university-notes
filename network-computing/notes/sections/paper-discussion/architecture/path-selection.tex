\subsubsection{Path Selection}

So far we ensured one sender and one receiver per timeslot. But in a Clos (fat-tree, \autopageref{sec:fat-tree-network-architecture}) topology, the communication between a sender and a receiver can be performed through multiple paths (ToR $\to$ aggregation $\to$ core $\to$ aggregation $\to$ ToR). There are multiple possible core switches and if we are not careful, we might end up with multiple senders and receivers sharing the same core switch, which would cause congestion.

\begin{flushleft}
    \textcolor{Green3}{\faIcon{check-circle} \textbf{Path Selection: Edce Coloring}}
\end{flushleft}
We have a \textbf{bipartite graph} where the senders and receivers are the two sets of vertices, and the edges represent the communication between them. Now, each packet must traverse a core switch, and a network has $k$ core switches, which means that we can use at most $k$ edges (paths) that share the same core switch.

\highspace
So we can use \textbf{edge coloring algorithms} to \hl{assign a ``color'' (core switch) to each source-destination pair (edge) such that no two edges sharing a node (sender or receiver) have the same color (core switch).}

\begin{flushleft}
    \textcolor{Green3}{\faIcon{question-circle} \textbf{Why this constraint?}}
\end{flushleft}
Because if two packets share the same source rack and use the same core switch, they would compete for the same bandwidth on that core switch, which would lead to congestion. Similarly, if two packets share the same destination rack and use the same core switch, they would also compete for bandwidth on that core switch. So we must \hl{avoid two edges sharing a node using the same color.}

\highspace
Similar to sender-receiver matching, we want to ensure that \textbf{each source rack has at most one edge of each color} and that \textbf{each destination rack has at most one edge of each color}. This ensures that no two flows use the same ToR-core or core-ToR link, thus preventing congestion.

\highspace
This solution is elegant, because it decomposes a hard global scheduling problem into two simpler problems: matching (who talks to whom) and coloring (which path they use). There is a clear separation of concerns, and we can use well-known algorithms for both problems.

\begin{figure}[!htp]
    \centering
    \includegraphics[width=.65\textwidth]{img/fastpass-edge-coloring.pdf}
    \caption{Example of edge coloring in a bipartite graph representing senders and receivers. Each color represents a different core switch. This ensures that no two edges sharing a node (sender or receiver) have the same color (core switch), thus avoiding congestion.\cite{10.1145/2740070.2626309}}
\end{figure}