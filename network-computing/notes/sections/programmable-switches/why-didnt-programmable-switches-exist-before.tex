\subsection{Why didn't programmable switches exist before?}

In short, \textbf{programmable switches didn't make sense before} because we lacked the technical feasibility and practical justification. But now, due to advances in chip design and network complexity, it's finally possible, and necessary, to build them.

\highspace
\begin{flushleft}
    \textcolor{Red2}{\faIcon{dollar-sign} \textbf{In the past: Programmability was too expensive}}
\end{flushleft}
In the past, the trade-off between programmability and cost was too high:
\begin{enumerate}
    \item \textcolor{Red2}{\textbf{Performance was too low}}. Programmable hardware, like FPGAs or general-purpose CPUs, was much slower than fixed-function ASICs.
    \begin{itemize}
        \item[\textcolor{Green3}{\faIcon{check}}] A \textbf{fixed switch chip} could forward billions of packets per second.
        \item[\textcolor{Red2}{\faIcon{times}}] A \textbf{programmable one}? Too slow for line-rate performance.
    \end{itemize}
    So \textbf{if we wanted programmability, we had to sacrifice speed}. That was a deal-breaker for core network equipment.


    \item \textcolor{Red2}{\textbf{Chip area and power cost were too high}}. Fixed-function logic is compact and power-efficient. Programmable logic, by contrast, used to \textbf{take up more silicon and required more power}. Result: vendors and data center operators couldn't justify using programmable switches, they were too big, too hot, and too slow.
\end{enumerate}

\highspace
\begin{flushleft}
    \textcolor{Green3}{\faIcon{check-circle} \textbf{What changed?}}
\end{flushleft}
\textbf{Three technological trends} made programmable switches finally viable:
\begin{enumerate}
    \item[\textcolor{Green3}{\faIcon{\speedIcon}}] \textcolor{Green3}{\textbf{Chip speed caught up}}. We now have programmable switch chips (like Barefoot Tofino) that can \textbf{run at line rate}, just like fixed-function ones. In other words, programmability no longer costs us speed.
    \item[\textcolor{Red2}{\faIcon{exclamation-triangle}}] \textcolor{Red2}{\textbf{Network complexity exploded}}. There are now \textbf{too many protocols and features} to hard-code everything into silicon:
    \begin{itemize}
        \item New protocols, encapsulations (VXLAN, GTP, QUIC, etc.)
        \item Monitoring, load balancing, AI, security; all need custom, real-time logic.
    \end{itemize}
    Hard-coding all of this would take years, and would never be flexible enough.
    \item[\textcolor{Green3}{\faIcon{check}}] \textcolor{Green3}{\textbf{Moore's Law made logic ``free''}}. Thanks to Moore's Law:
    \begin{itemize}
        \item We can double the amount of logic in the same area every 2 years.
        \item The \textbf{cost} of programmability in terms \textbf{of chip area} and \textbf{power} has become \textbf{negligible}.
    \end{itemize}
    Now, the logic that makes a switch programmable barely takes up more space than a fixed-function design.
\end{enumerate}

\newpage

\begin{table}[!htp]
    \centering
    \begin{tabular}{@{} l | l | l @{}}
        \toprule
        \textbf{Factor} & \textbf{Before}               & \textbf{Now}              \\
        \midrule
        Chip speed      & Too slow for line-rate        & Equal to fixed-function   \\ [.3em]
        Logic cost      & Too expensive (area + power)  & Basically free            \\ [.3em]
        Protocols       & Few, stable                   & Too many to hard-code     \\ [.3em]
        Urgency         & Low                           & High (cloud, IoT, 5G, ML) \\
        \bottomrule
    \end{tabular}
    \caption{Why didn't programmable switches exist before?}
\end{table}