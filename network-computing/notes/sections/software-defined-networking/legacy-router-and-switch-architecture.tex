\subsection{Legacy Router \& Switch Architecture}

Legacy network devices, such as routers and switches, are \textbf{built with integrated control and data planes}, meaning \hl{each device independently makes forwarding decisions}. These devices consist of:
\begin{enumerate}
    \item \important{Hardware Components}
    \begin{itemize}
        \item \textbf{Application-Specific Integrated Circuits (ASICs)}, specialized chips for packet forwarding.
        \item \textbf{Memory (buffers \& TCAMs)}, stores forwarding tables and processing queues.
        \item \textbf{Network Interfaces (NICs, Ports)}, physical ports for connecting network cables.
    \end{itemize}
    \item \important{Software Components}
    \begin{itemize}
        \item \textbf{Router OS (Operating System)}, runs network protocols and management interfaces.
        \item \textbf{Routing Protocols (OSPF, BGP, RIP)}, determines paths for packet forwarding.
        \item \textbf{Forwarding Table}, maps destination addresses to outgoing ports.
    \end{itemize}
    \item \important{Management and Control Interfaces}
    \begin{itemize}
        \item \textbf{Command-Line Interface (CLI)}, used for configuring routers\break manually.
        \item \textbf{SNMP (Simple Network Management Protocol)}, enables\break monitoring and automation.
    \end{itemize}
\end{enumerate}

\begin{figure}[!htp]
    \centering
    \includegraphics[width=.7\textwidth]{img/router-switch-arc.pdf}
    \caption{Legacy router and switch architecture.}
\end{figure}

\noindent
Traditional network devices have two primary operational planes:
\begin{itemize}
    \item \textbf{Control Plane}: makes forwarding decisions based on routing protocols.
    \item \textbf{Data Plane}: physically forwards packets based on control plane decisions. For example MAC lookup and IP forwarding.
\end{itemize}
\hl{Each router operates autonomously}, \hl{using routing tables} built through protocols like OSPF and BGP. These protocols dynamically learn network paths and update the forwarding tables, ensuring efficient packet delivery.

\highspace
\begin{flushleft}
    \textcolor{Green3}{\faIcon{list-ol} \textbf{Packet Processing in a Legacy Router}}
\end{flushleft}
\begin{enumerate}
    \item \important{Lookup Destination IP} $\rightarrow$ Find matching entry in the forwarding table.
    \item \important{Update Header} $\rightarrow$ Modify packet headers if needed (e.g., TTL decrement).
    \item \important{Queue Packet} $\rightarrow$ Send packet to the appropriate output interface.
\end{enumerate}
\textcolor{Red2}{\faIcon{times-circle}} Since \textcolor{Red2}{\textbf{every device handles its own control and forwarding}}, \textbf{large-scale changes require individual device updates}, making traditional networking \hl{complex and inflexible}.

\highspace
\begin{flushleft}
    \textcolor{Red2}{\faIcon{times-circle} \textbf{Challenges in Traditional Network Management}}
\end{flushleft}
\begin{enumerate}[label=\textcolor{Red2}{\faIcon{times}}]
    \item \textcolor{Red2}{\textbf{Complex Configuration \& Management}}. \textbf{Each network device} has to be \textbf{configured individually}. Protocols like BGP and OSPF require manual tuning for optimal performance. Network engineers must interact with vendor-specific CLIs, which vary by manufacturer.
    
    \item \textcolor{Red2}{\textbf{Limited Innovation \& Vendor Lock-In}}. New network \textbf{features} require \textbf{firmware or software updates from vendors}. Custom networking solutions are \textbf{difficult to implement due to proprietary hardware and software}.
    
    \item \textcolor{Red2}{\textbf{Slow Response to Failures \& Traffic Changes}}. Routing adjustments depend on distributed algorithms that can take seconds to minutes to converge. Manual troubleshooting is often needed when failures occur.
    
    \item \textcolor{Red2}{\textbf{Scalability Issues}}. Growing networks require more hardware and manual configurations. \textbf{Updating} policies across multiple routers is \textbf{time-consuming and error-prone}.
\end{enumerate}

\highspace
\begin{takeawaysbox}[: Legacy Router and switch architecture]
    \begin{itemize}
        \item Legacy networking relies on \textbf{autonomous devices} with tightly integrated \textbf{control and data planes}.
        \item Routing is handled by protocols like OSPF and BGP, which operate \textbf{independently on each device}.
        \item Challenges include manual configuration, vendor lock-in, slow failure response, and scalability issues.
        \item These limitations paved the way for SDN, which offers centralized, programmable networking.
    \end{itemize}
\end{takeawaysbox}
