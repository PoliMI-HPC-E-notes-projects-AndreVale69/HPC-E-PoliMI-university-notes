\subsection{OpenFlow}

\definition{OpenFlow} is the \textbf{first and most widely adopted protocol} used in Software-Defined Networking (SDN) to \textbf{enable communication} between the \textbf{SDN Controller} (control plane) and the \textbf{data plane devices} (e.g., switches, routers, they are the forwarding engine). It allows the controller to \textbf{program flow tables} in the switches and \textbf{control how packets are forwarded}, enabling centralized management of traffic.

\highspace
In other words, OpenFlow is the practical implementation of SDN, standardizing how controllers manage packet forwarding.

\highspace
\begin{flushleft}
    \textcolor{Green3}{\faIcon{tools} \textbf{How OpenFlow works}}
\end{flushleft}
Each \textbf{OpenFlow switch} contains:
\begin{enumerate}
    \item \textbf{Flow Table}: Contains rules in the form of $Match \rightarrow Action$ pairs.
    \item \textbf{Communication Interface}: Connects to the SDN controller via the OpenFlow protocol.
    \item \textbf{Stats Module}: Collects statistics about packet flows.
\end{enumerate}

\highspace
\begin{examplebox}[: Flow Rules]
    \begin{enumerate}
        \item If $Header = p \Rightarrow$ send to port 5.
        \item If $Header = q \Rightarrow$ modify header to $r$, then send to ports 6 and 7.
        \item If $Header = p \Rightarrow$ send packet to the controller.
    \end{enumerate}
\end{examplebox}

\highspace
\textbf{Flow table operation}:
\begin{enumerate}
    \item Packet arrives at the switch.
    \item Switch checks for a \textbf{matching rule} in its flow table.
    \item If matched $\rightarrow$ \textbf{apply action} (e.g., forward, modify, drop).
    \item If no match $\rightarrow$ \textbf{send packet header to controller} for instructions.
\end{enumerate}

\highspace
\begin{examplebox}[: OpenFlow]
    \begin{enumerate}
        \item New packet arrives at switch.
        \item Match?
        \begin{itemize}
            \item Yes $\rightarrow$ forward according to rule.
            \item No $\rightarrow$ forward header to controller.
        \end{itemize}
        \item Controller analyzes packet and installs a new rule in switch.
        \item Next packets of same type $\rightarrow$ directly processed by switch using newly installed rule.
    \end{enumerate}
\end{examplebox}

\begin{figure}[!htp]
    \centering
    \includegraphics[width=.6\textwidth]{img/openflow.pdf}
\end{figure}

\highspace
\begin{flushleft}
    \textcolor{Green3}{\faIcon{list-ul} \textbf{Actions in OpenFlow}}
\end{flushleft}
OpenFlow supports many types of actions, such as:
\begin{itemize}
    \item \textbf{Forwarding} to one or multiple ports.
    \item \textbf{Dropping packets}.
    \item \textbf{Modifying headers} (e.g., VLAN tags, IP addresses).
    \item \textbf{Sending packets to controller}.
    \item \textbf{Statistics collection} for flow monitoring.
\end{itemize}
This \textbf{flexibility} allows SDN to implement advanced functions like load balancing, traffic shaping, and security filtering without specialized hardware.

\highspace
\begin{flushleft}
    \textcolor{Green3}{\faIcon{balance-scale} \textbf{Reactive vs. Proactive Flow Rules}}
\end{flushleft}
In OpenFlow, the \textbf{controller installs flow rules} into switches to determine how packets are processed. There are \textbf{two main modes} of operation for rule installation: Reactive and Proactive. These \textbf{modes define when and how flow entries are populated in the flow tables} of switches.
\begin{itemize}
    \item \definition{Reactive Mode}
    \begin{itemize}
        \item[\textcolor{Green3}{\faIcon{question-circle}}] \textcolor{Green3}{\textbf{How it works?}}
        \begin{enumerate}
            \item When a \textbf{new flow} (a new type of packet) arrives at a switch, and \textbf{no rule matches} it, the \textbf{packet header is sent to the controller}.
            \item The \textbf{controller analyzes the packet} and \textbf{decides what rule} should be installed in the switch to handle it.
            \item After the controller sends back a rule, the switch installs it and \textbf{forwards the packet} accordingly.
        \end{enumerate}
        \item[\textcolor{Green3}{\faIcon{list-ul}}] \textcolor{Green3}{\textbf{Key Characteristics}}
        \begin{itemize}
            \item \textbf{Efficient use of flow tables} - Only rules for \hl{active flows} are installed.
            \item \textbf{Every new flow} incurs \textbf{small setup time} (controller interaction delay).
            \item \textbf{Switch depends on the controller} for flow rule installation.
            \item If the \textbf{controller connection is lost}, the switch has \textbf{limited utility} for new flows.
        \end{itemize}
        \item[\textcolor{Green3}{\faIcon{check-circle}}] \textcolor{Green3}{\textbf{Advantages}}
        \begin{itemize}[label=\textcolor{Green3}{\faIcon{check}}]
            \item \textbf{Dynamic} and \textbf{adaptive} to real-time network traffic.
            \item \textbf{Minimizes unused flow entries}.
        \end{itemize}
        \item[\textcolor{Red2}{\faIcon{times-circle}}] \textcolor{Red2}{\textbf{Disadvantages}}
        \begin{itemize}[label=\textcolor{Red2}{\faIcon{times}}]
            \item Adds \textbf{latency} for the \textbf{first packet} of each flow.
            \item \textbf{High control plane load} in environments with many short flows.
        \end{itemize}
    \end{itemize}

    \item \definition{Proactive Mode}
    \begin{itemize}
        \item[\textcolor{Green3}{\faIcon{question-circle}}] \textcolor{Green3}{\textbf{How it works?}}
        \begin{enumerate}
            \item The \textbf{controller pre-installs flow rules} in the switches before any packet arrives.
            \item The switch \textbf{immediately processes packets} using pre-defined rules without contacting the controller.
        \end{enumerate}
        \item[\textcolor{Green3}{\faIcon{list-ul}}] \textcolor{Green3}{\textbf{Key Characteristics}}
        \begin{itemize}
            \item \textbf{Zero setup delay} for packet processing - packets are forwarded \textbf{immediately}.
            \item Requires \textbf{aggregated or wildcard rules} to efficiently use flow table space.
            \item \textbf{Independent of controller connectivity} - continues to operate even if controller is unreachable.
        \end{itemize}
        \item[\textcolor{Green3}{\faIcon{check-circle}}] \textcolor{Green3}{\textbf{Advantages}}
        \begin{itemize}[label=\textcolor{Green3}{\faIcon{check}}]
            \item \textbf{Fast packet forwarding} with \textbf{no initial delay}.
            \item \textbf{No dependency} on controller for flow rule installation during packet arrival.
            \item Ideal for \textbf{predictable traffic patterns} or \textbf{mission-critical environments}.
        \end{itemize}
        \item[\textcolor{Red2}{\faIcon{times-circle}}] \textcolor{Red2}{\textbf{Disadvantages}}
        \begin{itemize}[label=\textcolor{Red2}{\faIcon{times}}]
            \item Can \textbf{waste flow table space} if many pre-installed rules are unused.
            \item Requires \textbf{good planning} of rules; less flexible to dynamic traffic changes.
        \end{itemize}
    \end{itemize}
\end{itemize}
In summary, reactive mode is adaptive, but introduces latency and higher controller load; proactive mode is fast and resilient, but requires advance planning of rules.

\highspace
\begin{takeawaysbox}
    \begin{itemize}
        \item \textbf{OpenFlow} enables the SDN controller to manage \textbf{flow tables} in switches.
        \item \textbf{Flow rules} define how packets are handled, allowing \textbf{centralized, programmable networking}.
        \item OpenFlow supports \textbf{fine-grained traffic control} via a wide range of \textbf{match/action rules}.
        \item Two operation modes:
        \begin{itemize}
            \item Reactive (dynamic but with latency).
            \item Proactive (fast but needs good planning).
        \end{itemize}
    \end{itemize}
\end{takeawaysbox}