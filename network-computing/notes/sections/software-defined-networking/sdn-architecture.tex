\subsection{SDN Architecture}

The \textbf{core} concept of \definition{Software-Defined Networking (SDN)} is the \textbf{separation} of the \textbf{control plane} from the \textbf{data plane}:
\begin{itemize}
    \item In SDN, \textbf{network devices} (switches/routers) become \textbf{simple forwarding elements}, executing decisions made by a centralized \textbf{controller}.
    \item The \definition{SDN Controller} is a \textbf{software-based system} that \hl{manages, programs, and monitors the entire network}.
\end{itemize}
Key architecture:
\begin{itemize}
    \item \textbf{Data Plane (Forwarding Engine)} $\rightarrow$ Located on switches; handles packet forwarding.
    \item \textbf{Control Plane (SDN Controller)} $\rightarrow$ Runs on external servers; computes forwarding rules.
    \item \textbf{Communication Channel} $\rightarrow$ Allows the controller to instruct the data plane; typically uses OpenFlow.
\end{itemize}
But \emph{why decouple?} Enables centralized decision-making, consistent policy enforcement, and simplified management. Facilitates dynamic updates to the network without hardware changes.

\highspace
\begin{flushleft}
    \textcolor{Green3}{\faIcon{book} \textbf{The Role o the SDN Controller}}
\end{flushleft}
The SDN Controller is the \textbf{central brain} of the network. It performs:
\begin{enumerate}[label=\textcolor{Green3}{\faIcon{check}}]
    \item \textcolor{Green3}{\textbf{Network State Monitoring}}: Gathers real-time information from all forwarding devices.
    \item \textcolor{Green3}{\textbf{Decision-Making}}: Calculates the best routes, applies policies, and enforces security.
    \item \textcolor{Green3}{\textbf{Rule Installation}}: Pushes flow rules to switches, determining how packets should be handled.
\end{enumerate}
Controllers provide a \textbf{global, up-to-date view of the entire network}, enabling smarter control than traditional distributed routing.

\highspace
\begin{flushleft}
    \textcolor{Green3}{\faIcon{code} \textbf{Communication Interfaces}}
\end{flushleft}
SDN uses two types of APIs to manage communication between layers:
\begin{itemize}
    \item \important{Southbound Interface}: Connects the controller to the data plane devices (e.g., \textbf{OpenFlow protocol}). It instructs switches via OpenFlow or similar protocols, installing/removing flow rules and collecting stats.
    \item \important{Northbound Interface}: Allows \textbf{applications to interact with the controller} via APIs (e.g., REST APIs). Applications (e.g., security monitoring, load balancing) query and command the controller to implement network policies.
\end{itemize}

\highspace
\begin{flushleft}
    \textcolor{Green3}{\faIcon{book} \textbf{Network Operating System (Network OS)}}
\end{flushleft}
The controller runs a Network OS, providing:
\begin{itemize}
    \item \textbf{Abstractions} over the physical network (e.g., topology view, link status).
    \item \textbf{Programmatic Interfaces} for developing control programs.
    \item \textbf{Consistency \& Global View}: All decisions are made based on coherent, synchronized data.
\end{itemize}
The Network OS simplifies the task of writing network control logic by exposing standardized APIs.

\begin{figure}[!htp]
    \centering
    \includegraphics[width=\textwidth]{img/sdn-arch.pdf}
\end{figure}

\highspace
\begin{takeawaysbox}[: SDN Architecture]
    \begin{itemize}
        \item \textbf{Traditional networking} embeds the control plane within each device; SDN \textbf{centralizes control} in software.
        \item The \textbf{SDN Controller} dynamically manages the \textbf{data plane devices} using a \textbf{communication protocol}.
        \item \textbf{OpenFlow} is the primary protocol used to communicate between the controller and switches.
        \item \textbf{Network OS} provides an \textbf{abstraction layer} and \textbf{programming environment} for writing control logic.
    \end{itemize}
\end{takeawaysbox}

