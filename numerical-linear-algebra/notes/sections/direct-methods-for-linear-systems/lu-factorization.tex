\section{Direct Methods for Linear Systems}

This section provides a brief introduction to direct methods for linear systems of equations.

\subsection{LU Factorization}

\definition{LU Factorization} is a \textbf{method} used in numerical linear algebra \textbf{to decompose a square matrix $A$ into two factors}:
\begin{itemize}
    \item A lower unitary triangular matrix $L$;
    \item An upper triangular matrix $U$, such that $A = LU$.
\end{itemize}

\highspace
\begin{flushleft}
    \textcolor{Green3}{\faIcon{tools} \textbf{How does it work?}}
\end{flushleft}
\begin{enumerate}
    \item \textcolor{Green3}{\textbf{Principal Submatrices}}. Let $A_{p,i} \in \mathbb{R}^{i,i}, i = 1, \dots, n$ be the principal submatrices of $A$ obtained by considering the first $i$ rows and columns. These submatrices are \textbf{crucial in determining the existence of an LU factorization}.

    \item \textcolor{Green3}{\textbf{Determinant Condition}}. For LU factorization to be possible, the determinant of each principal submatrix must be non-zero:
    \begin{equation*}
        \det(A_{p,i}) \neq 0 \hspace{2em} \forall i = 1, \ldots, n-1
    \end{equation*}

    \item \textcolor{Green3}{\textbf{LU Factorization Theorem}}. \begin{theorem}
        If $A$ is invertible, then it admits an LU factorization if and only if all its leading principal minors are nonzero.
    \end{theorem}
\end{enumerate}
