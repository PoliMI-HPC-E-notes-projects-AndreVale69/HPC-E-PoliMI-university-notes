\subsubsection{The Schur Complement}

The \definition{Schur complement} is a powerful tool used in numerical linear algebra to \textbf{simplify the solution of linear systems involving block matrices}. When dealing with non-overlapping subdomains, the Schur complement allows us to \textbf{reduce the size of the system we need to solve directly by focusing on the interface unknowns}.

\highspace
\begin{flushleft}
    \textcolor{Green3}{\faIcon{question-circle} \textbf{How does it work?}}
\end{flushleft}
We introduce the Schur complement by the block LU factorization of a matrix $A$. The matrix $A$ is partitioned into blocks corresponding to the subdomains $\Omega_{1}$, $\Omega_{2}$, and the interface $\Gamma$:
\begin{equation*}
    A = \begin{pmatrix}
        A_{11}          & 0                  & A_{1\Gamma} \\
        0               & A_{22}             & A_{2\Gamma} \\
        A_{1\Gamma}^{T} & A_{2\Gamma}^{T}    & A_{\Gamma \Gamma}
    \end{pmatrix}
\end{equation*}
The factorization is expressed as:
\begin{equation*}
    A = \begin{pmatrix}
        I               & 0                 & 0 \\
        0               & I                 & 0 \\
        A_{1\Gamma}^{T} & A_{2\Gamma}^{T}   & I
    \end{pmatrix}
    \begin{pmatrix}
        A_{11}  & 0     & A_{1\Gamma} \\
        0       & A_{22}& A_{2\Gamma} \\
        0       & 0     & S
    \end{pmatrix}
\end{equation*}
Where $S$ is the \textbf{Schur complement}:
\begin{equation*}
    S = A_{\Gamma \Gamma} - A_{1\Gamma}^{T} A_{11}^{-1} A_{1\Gamma} - A_{2\Gamma}^{T} A_{22}^{-1} A_{2\Gamma}
\end{equation*}
To \textbf{solve for the interface unknowns} $\mathbf{x}_{\Gamma}$, we use the Schur complement system:
\begin{equation*}
    S \mathbf{x}_{\Gamma} = \widetilde{\mathbf{b}}_{\Gamma}
\end{equation*}
Where $\widetilde{\mathbf{b}}_{\Gamma}$ is defined as:
\begin{equation*}
    \widetilde{\mathbf{b}}_{\Gamma} = \mathbf{b}_{\Gamma} - A_{1\Gamma}^{T} A_{11}^{-1} \mathbf{b}_{1} - A_{2\Gamma}^{T} A_{22}^{-1} \mathbf{b}_{2}
\end{equation*}
Once $\mathbf{x}_{\Gamma}$ is determined, the \textbf{remaining unknowns} in $\Omega_{1}$ and $\Omega_{2}$ can be computed as follows:
\begin{equation*}
    \begin{array}{rcl}
        \mathbf{x}_{1} &=& A_{11}^{-1} (\mathbf{b}_{1} - A_{1\Gamma} \mathbf{x}_{\Gamma}) \\ [.8em]
        \mathbf{x}_{2} &=& A_{22}^{-1} (\mathbf{b}_{2} - A_{2\Gamma} \mathbf{x}_{\Gamma})
    \end{array}
\end{equation*}

\highspace
\begin{flushleft}
    \textcolor{Green3}{\faIcon{tools} \textbf{Key Characteristics of Schur Complement}}
\end{flushleft}
\begin{enumerate}
    \item[\textcolor{Red2}{\faIcon{times}}] \textcolor{Red2}{\textbf{Computational Expense}}. The Schur complement matrix $S$ is expensive to compute and is generally dense, even if $A$ is sparse.

    \item[\textcolor{Green3}{\faIcon{check}}] \textcolor{Green3}{\textbf{Iterative Solution}}. If the Schur complement \textbf{system} $S \mathbf{x}_{\Gamma} = \widetilde{\mathbf{b}}_{\Gamma}$ is \textbf{solved iteratively}, the Schur \textbf{complement} $S$ \textbf{does not need to be formed explicitly}.

    \item[\textcolor{Green3}{\faIcon{check}}] \textcolor{Green3}{\textbf{Matrix-Vector Multiplication}}. Matrix-vector multiplication by $S$ requires solving systems within each subdomain, implicitly involving $A_{11}^{-1}$ and $A_{22}^{-1}$, which can be done in parallel.

    \item \textcolor{Green3}{\textbf{Condition Number}}. The condition number\footnote{It is a measure of how sensitive the solution of a system of linear equations is to errors in the data or errors in the solution process.} of $S$ is generally \textbf{better than that of} $A$, typically $O\left(h^{-1}\right)$ instead of $O\left(h^{-2}\right)$ for a mesh size $h$.

    \item \textcolor{Green3}{\textbf{Preconditioning}}. In practice, suitable interface preconditioners are\break needed to accelerate convergence when solving the Schur complement system.
\end{enumerate}
The Schur complement allows us to \textbf{simplify the solution of linear systems involving block matrices by focusing on the interface unknowns}. This \textcolor{Green3}{\textbf{reduces the size of the system}} that needs to be solved directly. The process involves computing the Schur complement matrix $S$ and solving the Schur complement system for the interface unknowns. Although computing $S$ can be expensive, \textbf{iterative methods and parallel computation can help manage this complexity}.