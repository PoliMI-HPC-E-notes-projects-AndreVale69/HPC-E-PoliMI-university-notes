\subsubsection{Discretized Schwarz Methods}

The \definition{Discretized Schwarz Methods} involve \textbf{applying the Schwarz decomposition principles to a discretized version of the computational domain}. This results in an algebraic system that can be efficiently solved using the same iterative techniques.

\begin{itemize}
    \item \important{Discretization Process}. The \textbf{continuous domain is discretized into a grid or mesh}, resulting in a \textbf{finite set of points that represent the domain}. The PDE is then transformed into a system of linear equations:
    \begin{equation*}
        A\mathbf{x} = \mathbf{b}
    \end{equation*}
    \begin{itemize}
        \item $A$ matrix representing the \emph{discretized operator};
        \item $\mathbf{x}$ is the \emph{vector of unknowns};
        \item $\mathbf{b}$ represents the \emph{source terms} or \emph{boundary conditions}.
    \end{itemize}


    \item \important{Overlapping Subdomains}: Each subdomain $\Omega_{i}$ has a \textbf{set of grid points} ($n$) indexed by $S_{i}$ (with $n_{i} = \left|S_{i}\right|$). The \textbf{overlap between subdomains ensures that the indices in} $ S_1 $ and $ S_2 $ are \textbf{not disjoint} ($S_{1} \cap S_{2} \ne \emptyset$ and $n_{1} + n_{2} > n$), facilitating the exchange of boundary information.


    \item \important{Boolean Restriction Matrices}: Boolean restriction matrices $ R_{i}$ are used to \textbf{extract the relevant components of the solution vector} $\mathbf{v}$ \textbf{for each subdomain}. These matrices play a crucial role in the iterative solution process by ensuring that each subdomain only processes its relevant data.

    Formally, for $i = 1, 2$, let $R_{i}$ be an $n_{i} \times n$ boolean restriction matrix such that for any vector $\mathbf{v} \in \mathbb{R}^{n}$:
    \begin{equation*}
        \mathbf{v}_{i} = R_{i} \mathbf{v} \in \mathbb{R}^{n_{i}}
    \end{equation*}
    This means $\mathbf{v}_{i}$ contains the components of $\mathbf{v}$ corresponding to indices in $ S_{i}$ (i.e., those components associated with nodes in $\Omega_{i}$).


    \item \important{Extension Matrices}: Extension matrices are used to \textbf{expand the solution from subdomains back to the global domain}. The extension matrix $R_{i}^{T}$ maps the local solution $\mathbf{v}_{i}$ back to the global solution $\mathbf{v}$.

    Formally, for $i = 1, 2$:
    \begin{equation*}
        \mathbf{v} = R_{i}^{T} \mathbf{v}_{i}
    \end{equation*}
    This ensures that the components of $\mathbf{v}$ corresponding to indices in $S_{i}$ are the same as those of $\mathbf{v}_{i}$, while the remaining components are zero.


    \item \important{Principal Submatrices}: Principal submatrices are \textbf{submatrices} of the global matrix $\mathbf{A}$ \textbf{that correspond to the subdomains}. For each subdomain $\Omega_{i}$, the principal submatrix $\mathbf{A}_{i}$ is formed by restricting $\mathbf{A}$ to the indices in $S_{i}$.

    Formally, for subdomain $\Omega_{i}$, the principal submatrix $A_{i} \in \mathbb{R}^{n_{i} \times n_{i}}$ is given by:
    \begin{equation}\label{eq: principal submatrices - discretized schwarz methods}
        \mathbf{A}_{i} = R_{i} \mathbf{A} R_{i}^{T}
    \end{equation}
    These submatrices represent the \textbf{system of equations for the respective subdomains}, allowing them to be solved independently.
\end{itemize}

\begin{flushleft}
    \textcolor{Green3}{\faIcon{book} \textbf{Linking the Concepts}}
\end{flushleft}
\begin{itemize}
    \item \important{Extension Matrices and Boolean Restriction Matrices}. The extension matrices $R_i^T$ and the boolean restriction matrices $R_i$ are \textbf{complementary}.
    \begin{itemize}
    	\item Extension Matrices $R_{i}$ \textbf{extracts} the relevant \textbf{components of the solution} vector $\mathbf{v}$ \textbf{for each subdomain};
    	\item Boolean Restriction Matrices $R_{i}^{T}$ \textbf{extends} the \textbf{local solution} $\mathbf{v}_{i}$ \textbf{back to the global solution} $\mathbf{v}$.
    \end{itemize}

    \item \important{Principal Submatrices and Overlapping Subdomains}. The principal submatrices $\mathbf{A}_{i}$ are directly related to the overlapping subdomains. \textbf{Each subdomain $\Omega_{i}$ has its own principal submatrix $\mathbf{A}_{i}$, which represents the local system of equations}. The overlap between subdomains ensures that the principal submatrices share common indices, facilitating the exchange of boundary information and ensuring consistency in the global solution.
\end{itemize}

\highspace
The following pages introduce two types of Discretized Schwarz methods:
\begin{enumerate}
    \item \textbf{Multiplicative Schwarz method}, where subdomains are updated sequentially. The solution from each subdomain is used to update the boundary conditions for the next subdomain in the sequence.

    \item \textbf{Additive Schwarz method}, where subdomains are solved independently and their solutions are combined additively to form the global solution.
\end{enumerate}