\subsection{How it works}

The multigrid method is divided into \textbf{seven parts} that make the MG method work.
\begin{enumerate}
    \item \textbf{Coarse Grids} (page \pageref{subsubsection: Coarse Grids})
    \item \textbf{Correction} (page \pageref{subsubsection: Correction})
    \item \textbf{Interpolation Operator} (page \pageref{subsubsection: Interpolation Operator})
    \item \textbf{Restriction Operator} (page \pageref{subsubsection: Restriction Operator})
    \item \textbf{Two-Grid Scheme} (page \pageref{subsubsection: Two-Grid Scheme})
    \item \textbf{V-Cycle Scheme} (page \pageref{subsubsection: V-Cycle Scheme})
    \item \textbf{Full Multigrid (FMG) V-Cycle Scheme} (page \pageref{subsubsection: Full Multigrid (FMG) V-Cycle Scheme})
\end{enumerate}

\noindent
These elements \textbf{work together to handle errors at different scales}, making the method \textbf{highly effective for solving large and complex systems} of linear equations.

\highspace
Note that this \textbf{is \underline{not} an algorithm!} We can think of the MG method as a toolbox filled with powerful tools, each designed to address different aspects of solving complex problems efficiently.