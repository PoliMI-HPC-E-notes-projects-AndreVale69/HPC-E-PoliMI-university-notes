\subsubsection{Restriction Operator}\label{subsubsection: Restriction Operator}

\textbf{\underline{Purpose}}. The Restriction Operator \textbf{transfers data from a fine grid to a coarse grid}. It can be thought of as the opposite of the interpolation operator (page \pageref{subsubsection: Interpolation Operator}).

\highspace
Mathematically, the restriction operator is a linear operator and it is \textbf{denoted} as a matrix $I_{h}^{2h}$:
\begin{equation}
	\begin{array}{rcl}
		I^{2h}_{h}: \mathcal{F}_{h} & \longrightarrow & \mathcal{F}_{2h} \\ [.5em]
		\mathbb{R}^{n} & \longrightarrow & \mathbb{R}^{m}
	\end{array}
\end{equation}
Unlike the interpolation operator, we have more problems here. Because to apply this tool and to guarantee that the coarse grid construction reflects the fine grid problem, we cannot use a simple system. There are \textbf{two ways to apply the restriction operator}:
\begin{itemize}
	\item \underline{\textbf{Injection}}. It is the simple form of restriction where \textbf{coarse grid values are directly taken from the corresponding fine grid values}. It transfers the value without any averaging. Mathematically it is expressed with the equation:
	\begin{equation}
		I^{2h}_{h} \mathbf{w}_{h} = \mathbf{w}_{2h}
	\end{equation}
	\begin{flushleft}
		\textcolor{Green3}{\faIcon{check} \textbf{Pros}}
	\end{flushleft}
	\begin{itemize}
		\item Very \textbf{simple}.
		\item Computationally \textbf{inexpensive}.
	\end{itemize}
	\begin{flushleft}
		\textcolor{Red2}{\faIcon{times} \textbf{Cons}}
	\end{flushleft}
	\begin{itemize}
		\item It can \textbf{miss fundamental details} of the solution.
		\item \textbf{Less accurate} error correction.
		\item Very \textbf{poor overall results}.
	\end{itemize}
	
	\newpage
	
	\item \underline{\textbf{Galerkin condition}}. It ensures that the coarse grid operator $A_{2h}$ accurately represents the fine grid operator $A_{h}$. This means that the solution on the \textbf{coarse grid is a reasonable approximation of the solution on the fine grid}. The Galerkin condition is expressed by the following equation:
	\begin{equation}
		A_{2h} = I_{h}^{2h} A_{h} I_{2h}^{h}
	\end{equation}
	Where:
	\begin{itemize}
		\item $I_{h}^{2h}$ is the \emph{interpolation operator} (page \pageref{subsubsection: Interpolation Operator});
		\item $I_{2h}^{h}$ is the \emph{restriction operator};
		\item $A_{h}$ is the \emph{fine grid};
		\item $A_{2h}$ is the result \emph{coarse grid} that we obtain.
	\end{itemize}
	The Galerkin condition can be viewed as a scaled transpose of the interpolation operator:
	\begin{equation}
		I_{h}^{2h} = c \left( I_{2h}^{h} \right)^{T}
	\end{equation}
	Where $c$ is a scalar factor in the real numbers $\mathbb{R}$, it adjusts the magnitude of the values when the restriction operator is applied.
	\begin{flushleft}
		\textcolor{Green3}{\faIcon{check} \textbf{Pros}}
	\end{flushleft}
	\begin{itemize}
		\item Ensures mathematical consistency.
		\item Ensures accurate representation of the fine grid problem on the coarse grid.
		\item Leads to effective error correction and greater accuracy in solutions.
	\end{itemize}
	\begin{flushleft}
		\textcolor{Red2}{\faIcon{times} \textbf{Cons}}
	\end{flushleft}
	\begin{itemize}
		\item Slightly more complex and computationally intensive than injection.
	\end{itemize}
\end{itemize}