\subsection{Differential Models and PDEs}

\begin{definitionbox}[: Partial Differential Equation (PDE)]
    A \textbf{differential equation} (model) is an equation that involves \textbf{one or more derivatives of an unknown function}. In an \textbf{Ordinary Differential Equation} (ODE), \textbf{every derivative of the unknown solution is with respect to a single independent variable}. If instead, derivatives are partial, then we have a \definition{Partial Differential Equation (PDE)}.
\end{definitionbox}

\noindent
In other words, it is a differential equation where its derivatives are partial.

\highspace
There are different types of PDEs, and their nature depends on the conditions and their type. Mathematically, we can represent a \textbf{differential model} (equation) as follows:
\begin{equation}
    \mathcal{P}\left(u; g\right) = 0 \hspace{2em} \text{differential equation (mathematical problem)}
\end{equation}
Where:
\begin{itemize}
    \item $\mathcal{P}$ indicates the \emph{\textbf{model}};
    \item $u$ is the \emph{\textbf{exact solution}}, a function of one or more independent variables (space and/or time variables);
    \item $g$ indicates the \emph{\textbf{data}}.
\end{itemize}

\highspace
\begin{deepeningbox}[: PDE]
    A \textbf{Partial Differential Equation} is the mathematical language used to describe how \textbf{continuous physical quantities} (temperature, velocity, pressure, stress, concentration, electromagnetic field, etc.) evolve in \textbf{space and time}. They express: conservation laws (mass, momentum, energy), constitutive relations (material response), and external influences (forces, sources, boundaries). \textbf{Everything around us is governed by PDEs}:
    \begin{itemize}
        \item \textbf{Physics}: Maxwell's equations (electromagnetism), Schrödinger equation (quantum mechanics), Navier-Stokes equations (fluid dynamics).
        \item \textbf{Engineering}: Heat equation (thermal analysis), wave equation (vibrations, acoustics), elasticity equations (structural mechanics).
        \item \textbf{Biology}: Reaction-diffusion equations (pattern formation, population dynamics), diffusion equations (spread of substances).
        \item \textbf{Finance}: Black-Scholes equation (option pricing).
    \end{itemize}
    They are the \textbf{continuous models of reality}.

    \highspace
    \textcolor{Green3}{\faIcon{question-circle} \textbf{Why discretization of PDEs?}} Analytic solutions exist only for simple geometries or constant coefficients. Most real-world problems, like airflow around a car, stress in a bridge, or temperature in a turbine, are too complex. So we \textbf{discretize} the PDE to convert it into a \textbf{finite-dimensional algebraic system} that we can solve numerically:
    \begin{equation*}
        A u_{h} = b
    \end{equation*}
    That's the bridge from \textbf{physics} to \textbf{mathematics} to \textbf{computation}. Discretization (via FEM, FDM, FVM, etc.) transforms ``infinitely many unknowns'' into a \textbf{finite but huge system} that captures the essential physics at chosen resolution.

    \highspace
    \textcolor{Green3}{\faIcon{question-circle} \textbf{Why solve PDEs numerically? (the \emph{purpose})}} Because solving PDEs means \textbf{simulating the real-world without physically building it}. Instead of running an expensive or dangerous experiment, we compute the solution on a computer. Some examples:
    \begin{itemize}
        \item \textbf{Aerospace}: Simulate airflow over a new aircraft design to optimize aerodynamics before building prototypes.
        \item \textbf{Climate Science}: Model global climate systems to predict future climate change scenarios.
        \item \textbf{Biomedical Engineering}: Simulate blood flow in arteries to design medical devices.
        \item \textbf{Automotive}: Crash simulations to improve vehicle safety features.
        \item \textbf{Energy}: Model heat transfer in nuclear reactors for safety analysis.
    \end{itemize}
    Every time we \textbf{discretize and solve a PDE}, we create a \textbf{digital experiment}.
\end{deepeningbox}

\newpage

\subsubsection{ODEs}

\definition{Ordinary Differential Equation (ODE)} is also known as \textbf{initial value problem}.\index{Initial value problem}

\highspace
\begin{flushleft}
    \textcolor{Green3}{\faIcon{book} \textbf{I\textdegree ODE - Cauchy problem}}
\end{flushleft}
A \textbf{first order} ODE, a \textbf{Cauchy problem}, is a differential problem, whose:
\begin{itemize}
    \item \textbf{\emph{Solution}} $u = u\left(t\right)$ is a function of a single independent variable $t$, often interpreted as time.
    \item A \textbf{\emph{single condition}} is assigned on the solution, at a point (usually, the left end of the integration interval).
\end{itemize}
Its form is the following find $u : I \subset \mathbb{R} \rightarrow \mathbb{R}$ such that:
\begin{equation}
    \begin{cases}
        \dfrac{\mathrm{d}u}{\mathrm{d}t}\left(t\right) = f\left(t, u\left(t\right)\right) & t \in I \vspace{1em} \\
        u\left(t_{0}\right) = u_{0}
    \end{cases}
\end{equation}
Where:
\begin{itemize}
    \item $I = \left( t_{0}, t_{f} \right] \subset \mathbb{R}$ is a \emph{\textbf{time interval}};
    \item $u_{0}$ is the \emph{\textbf{initial value}} assigned at $t = t_{0}$;
    \item $f: I \times \mathbb{R} \rightarrow \mathbb{R}$
\end{itemize}
\textcolor{Green3}{\faIcon{question-circle} \textbf{Meaning}}. The equation describes the \textbf{evolution of a scalar quantity $u$ over time $t$}, \textbf{without distribution in space}.

\highspace
\textcolor{Green3}{\faIcon{question-circle} \textbf{Vectorial problems}}. In vectorial problems, the \textbf{unknown is a vector-valued function} $\mathbf{u} = \mathbf{u}\left(t\right)$, where $\mathbf{u} = \left(u_{1}, \dots, u_{m}\right) \in \mathbb{R}^{m}$, with $m \ge 1$. The first order Cauchy problem reads: find $\mathbf{u} : I \subset \mathbb{R} \rightarrow \mathbb{R}^{m}$ such that:
\begin{equation*}
    \begin{cases}
        \dfrac{\mathrm{d}\mathbf{u}}{\mathrm{d}t}\left(t\right) = \mathbf{f}\left(t, \mathbf{u}\left(t\right)\right) & t \in I \vspace{1em} \\
        \mathbf{u}\left(t_{0}\right) = \mathbf{u}_{0}
    \end{cases}
\end{equation*}
Where $\mathbf{u}_{0} \in \mathbb{R}^{m}$ is the initial datum and $\mathbf{f}: I \times \mathbb{R}^{m} \rightarrow \mathbb{R}^{m}$.

\highspace
\begin{flushleft}
    \textcolor{Green3}{\faIcon{book} \textbf{II\textdegree ODE - Cauchy problem}}
\end{flushleft}
A \textbf{second order Cauchy problem} sees second order time derivatives and two initial conditions. It reads as: find $u: I \subset \mathbb{R} \rightarrow \mathbb{R}$ such that:
\begin{equation}
    \begin{cases}
        \dfrac{\mathrm{d}^{2}u}{\mathrm{d}t^{2}}\left(t\right) = f\left(t, u\left(t\right), \dfrac{\mathrm{d}u}{\mathrm{d}t}\left(t\right)\right) & t \in I \vspace{1em} \\
        \dfrac{\mathrm{d}u}{\mathrm{d}t}\left(t_{0}\right) = v_{0} \vspace{1em} \\
        u\left(t_{0}\right) = u_{0}
    \end{cases}
\end{equation}
Where the initial data are $u_{0}$ and $v_{0}$, while $f: I \times \mathbb{R} \times \mathbb{R} \rightarrow \mathbb{R}$.