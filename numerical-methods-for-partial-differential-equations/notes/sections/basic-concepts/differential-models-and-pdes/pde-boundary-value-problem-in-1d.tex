\subsubsection{PDE, boundary value problem in 1D}

The \definition{Boundary value problem in 1D} is characterized by a \textbf{single independent variable} $x$, which represents the \textbf{space coordinate in an interval} $\Omega = \left(a,b\right) \in \mathbb{R}$ (1D).

\highspace
The problem involves \textbf{second order derivatives of the unknown solution} $u = u\left(x\right)$ with respect to $x$. The value of $u$, or the \textbf{value of its first derivate}, is a \textbf{set at the two boundaries of the domain} (interval) $\Omega$, that is at $x = a$ and $x = b$ (the domain boundary is $\partial\Omega = \left\{a,b\right\})$.

\highspace
Let us consider the following Poisson problem with (homogeneous) Dirichlet boundary conditions: find $u : \Omega \subset \mathbb{R} \rightarrow \mathbb{R}$ such that:
\begin{equation}
    \begin{cases}
        -\dfrac{\mathrm{d}^{2}u}{\mathrm{d}x^{2}}\left(x\right) = f\left(x\right) & x \in \Omega = \left(a,b\right) \vspace{1em} \\
        %
        u\left(a\right) = u\left(b\right) = 0
    \end{cases}
\end{equation}
This equation models a \textbf{stationary phenomenon} (the time variable doesn't appear in fact) and represent a \textbf{diffusion model}.

\highspace
\begin{examplebox}
    For example, the diffusion model models the diffusion of a pollutant along a 1D channel $\Omega = \left(a,b\right)$ or the vertical displacement of an \emph{elastic thread} fixed at its ends. In the first case, $f = f\left(x\right)$ indicates the source of the pollutant along the flow, while in the second case, $f$ is the traverse force acting on the elastic thread, in the hypothesis of negligible mass and small displacements of the thread.
\end{examplebox}

\noindent
\begin{flushleft}
    \textcolor{Green3}{\faIcon{balance-scale} \textbf{Boundary value problem in 1D vs ODE}}
\end{flushleft}
We remark that the \textbf{boundary value problem in 1D is a particular case of PDEs}, even if it involves only derivatives with respect to a single independent variable $x$. Indeed, even if apparently similar to a second order ODE, the boundary value problem is in reality substantially \textbf{different} from an ODE:
\begin{itemize}
    \item In ODE, two conditions are set at $t = t_{0}$;
    \item In the boundary value problem in 1D, one condition is set at $x = a$ and the other one at $x = b$.
\end{itemize}
The conditions in the boundary value problem determine to the so-called global nature of the model.