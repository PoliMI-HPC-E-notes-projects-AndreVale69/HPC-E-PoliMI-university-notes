\subsection{Numerical Methods}

Since in most cases of practical interest we \textbf{cannot solve a PDE analytically}, we need to use \textbf{numerical methods} that allow us to construct an \emph{approximation} $u_{h}$ of the \emph{exact solution} $u$, for which the corresponding \emph{error} $\left(u - u_{h}\right)$ can be quantified and/or estimated.
\begin{equation*}
    \begin{array}{cc}
        \mathcal{P}\left(u; g\right) = 0 \hspace{1em} & \text{PDE (mathematical problem)} \\ [.3em]
        \downarrow \hspace{1em} & \emph{numerical method} \\ [.3em]
        \mathcal{P}_{h}\left(u_{h}; g_{h}\right) = 0 \hspace{1em} & \text{approximate PDE (numerical problem)}
    \end{array}
\end{equation*}
Where:
\begin{itemize}
    \item $g_{h}$ is an approximation of the data $g$;
    \item $\mathcal{P}_{h}$ is a characterization of the approximate problem.
\end{itemize}
The subscript $h$ indicates a \textbf{discretization parameter} that characterizes the numerical approximation. Conventionally, the smaller is $h$, the better is the approximation of $u$ made by $u_{h}$. Furthermore, the error $\left(u - u_{h}\right)$ tends to zero as $h$ gets smaller and smaller. In this course, we will specifically introduce the FE method (page \pageref{definition: Finite Element Method (FEM)}) to build the numerical approximation of PDEs.

\highspace
\begin{flushleft}
    \textcolor{Green3}{\faIcon{bookmark} \textbf{Summary Notation}}
\end{flushleft}

\begin{table}[!htp]
    \centering
    \begin{tabular}{@{} c p{25em} @{}}
        \toprule
        \textbf{Notation} & \textbf{Description} \\
        \midrule
        $\mathcal{P}\left(u; g\right) = 0$ & PDE (mathematical problem) \\ [.5em]
        $u$ & \textbf{\emph{exact solution}} of a PDE \\ [.5em]
        $u_{h}$ & \textbf{\emph{approximate solution}} of a PDE \\ [.5em]
        $\left(u - u_{h}\right)$ & \textbf{\emph{error}} (quantified and/or estimated; tends to zero if $h$ is smaller) \\ [.5em]
        $h$ & \textbf{\emph{discretization parameter}} ($\downarrow$ smaller $h$, better approximation; $\uparrow$ higher $h$, poor approximation) \\ [.5em]
        $\mathcal{P}_{h}\left(u_{h}; g_{h}\right) = 0$ & approximate PDE (numerical problem) \\ [.5em]
        $g_{h}$ & \textbf{\emph{approximation}} of the \textbf{\emph{data}} $g$ \\ [.5em]
        $\mathcal{P}_{h}$ & \textbf{\emph{characterization}} of the approximate problem. \\
        \bottomrule
    \end{tabular}
    \caption{Notation used to approximate the PDE with numerical methods.}
\end{table}