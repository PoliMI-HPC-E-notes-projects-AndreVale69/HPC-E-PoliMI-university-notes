\subsubsection{Weak formulation}

We start from the \textbf{strong problem}:
\begin{equation*}
    \begin{cases}
        - u''(x) = f(x), & x \in (0,1), \\[.5em]
        u(0)=u(1)=0 &
    \end{cases}
\end{equation*}
To obtain the weak formulation, we use test functions.
\begin{enumerate}
    \item \important{Introduce test functions.} We don't try to force $u(x)$ to satisfy the equation pointwise. Instead, we ``test'' it against a set of functions $v(x)$ called \textbf{test functions}. They live in the same function space as $u$, with the same boundary conditions (so $v(0)=v(1)=0$). This space is called:
    \begin{equation}
        V = H^1_0(0,1) = \left\{ v \in L^2(0,1) \;\;|\;\; v' \in L^2(0,1),\; v(0)=v(1)=0 \right\}
    \end{equation}

    \begin{deepeningbox}[: Test function]
        A \definition{Test Function} is not the solution itself, but an \emph{arbitrary function} we use to ``probe'' whether the PDE is satisfied. Formally:
        \begin{itemize}
            \item A test function is usually called $v(x)$.
            \item It belongs to a certain function space $V$ (often the same as the solution space).
            \item It must satisfy the \textbf{same boundary conditions} as the solution if those are homogeneous (like Dirichlet $u=0$ on the boundary).
        \end{itemize}
        In our case:
        \begin{equation*}
            V = H^1_0(0,1) = \left\{ v \in L^2(0,1) \mid v' \in L^2(0,1),\; v(0)=v(1)=0 \right\}
        \end{equation*}

        \begin{flushleft}
            \textcolor{Green3}{\faIcon{question-circle} \textbf{Why do we need them?}}
        \end{flushleft}
        Because instead of requiring the PDE to hold \textbf{pointwise} (strong), we require it to hold \textbf{on average against all test functions}:
        \begin{equation*}
            a(u,v) = F(v) \quad \forall v \in V    
        \end{equation*}
        This means:
        \begin{itemize}
            \item If the equality holds for all possible test functions $v$, then the solution $u$ must encode the correct behavior of the PDE.
            \item Test functions are like ``magnifying glasses'': by choosing different $v$, we check the equation in different ways.
        \end{itemize}

        \highspace
        \begin{examplebox}[: Analogy]
            Imagine we don't know the exact shape of a curve, but we can test it with different weights.
            \begin{itemize}
                \item Multiplying by $v(x)$ and integrating is like asking: ``\emph{How does the error of my solution project onto this particular pattern $v(x)$?}''
                \item If the error is orthogonal to \emph{all} test functions, then the solution must be correct.
            \end{itemize}
        \end{examplebox}

        \begin{flushleft}
            \textcolor{Green3}{\faIcon{tools} \textbf{In practice (Finite Elements)}}
        \end{flushleft}
        Later, when we do the \textbf{Galerkin method}, we restrict test functions $v$ to be combinations of \textbf{basis functions} (hat functions, polynomials, ...). So instead of ``all possible test functions'', we only require the PDE to hold for a finite set of them. That's how we get a linear system $AU=f$.

        \highspace
        \begin{flushleft}
            \textcolor{Green3}{\faIcon{history} \textbf{Summary}}
        \end{flushleft}
        A \textbf{test function} $v$ is an arbitrary function from a suitable space (here $H^{1}_{0}(0,1)$). We multiply the PDE by $v$ and integrate, to weaken the formulation. The condition ``for all test functions'' ensures the weak solution is equivalent to the strong one (if enough regularity). In finite elements, test functions become the basis functions of the discrete space.
    \end{deepeningbox}


    \item \important{Multiply by a test function and integrate.} Multiply the PDE by $v(x)$ and integrate over $(0,1)$:
    \begin{equation}
        \displaystyle\int_0^1 \left(-u''(x)\right) \cdot v(x)\, \mathrm{d}x = \displaystyle\int_0^1 f(x) \cdot v(x)\, \mathrm{d}x
    \end{equation}
    This is already ``weaker'', because we're not asking the PDE to hold pointwise, only \textbf{in an averaged sense} against all test functions.
    \begin{itemize}
        \item[\textcolor{Green3}{\faIcon{question-circle}}] \textcolor{Green3}{\textbf{Why multiply by a test function?}} If we just write the residual of the PDE:
        \begin{equation}
            R(x) = -u''(x) - f(x)
        \end{equation}
        Then the strong form requires $R(x) = 0$ \textbf{at every point}. That's too strict. Instead, we say:
        \begin{itemize}
            \item ``We don't care if $R(x)$ is exactly 0 everywhere,
            \item We only care that $R(x)$ produces no effect when measured\break against any admissible test function $v(x)$''.
        \end{itemize}
        So, multiplying by $v(x)$ is like ``projecting the error'' onto a shape.
        \begin{itemize}
            \item If for every possible shape $v$, the projection vanishes, then the error must be zero.
            \item If the residual had any ``component'' left, some test function would catch it.
        \end{itemize}
        Great Analogy: Imagine we don't know if a sound is silent. We pass it through every possible frequency filter (test functions). If all filters output 0, then the sound is really zero.

        \item[\textcolor{Green3}{\faIcon{question-circle}}] \textcolor{Green3}{\textbf{Why integrate?}} Because multiplying alone just gives a function $R(x) \cdot v(x)$. To reduce it to a single \textbf{number} (a condition we can actually impose), we integrate over the domain:
        \begin{equation*}
            \displaystyle\int_0^1 R(x) \cdot v(x)\, \mathrm{d}x = 0
        \end{equation*}
        Integration turns the \textbf{pointwise condition} into a \textbf{global/average condition}. It's like asking: ``\emph{what's the net effect of the residual when weighted by $v(x)$ across the whole domain?}''. If this is 0 for all $v$, it forces the residual itself to be 0 in the weak sense.

        Analogy: If we want to check if water in a pipe is really at 0 pressure, we don't measure every molecule. We place sensors (test functions) that average pressure over regions. If all averages say 0, the whole pipe is at 0.
    \end{itemize}


    \item \important{Integration by parts.} We move one derivative away from $u$ (so we don't require $u''$ to exist, only $u'$):
    \begin{equation*}
        \displaystyle\int_0^1 -u''(x) \cdot v(x)\, \mathrm{d}x = \Big[ -u'(x) \cdot v(x) \Big]_0^1 + \displaystyle\int_0^1 u'(x) \cdot v'(x)\, \mathrm{d}x
    \end{equation*}
    The boundary term $\Big[-u'(x) v(x)\Big]_0^1$ vanishes, because $v(0)=v(1)=0$. We are left with:
    \begin{equation}
        \displaystyle\int_0^1 u'(x) \cdot v'(x)\, \mathrm{d}x = \displaystyle\int_0^1 f(x) \cdot v(x)\, \mathrm{d}x
    \end{equation}
    This is the \textbf{core weak formulation equation}.
    \begin{itemize}
        \item[\textcolor{Green3}{\faIcon{question-circle}}] \textcolor{Green3}{\textbf{Why do integration by parts?}} Because, before this step, the integral contains the second derivative, $u''(x)$. That means the solution $u$ must be \textbf{twice differentiable} (second derivate must exist in the classical sense). But in practice, with discontinuous sources or with finite element approximations, $u''$ might not exist everywhere.
        
        Integration by parts transfers one derivate from $u$ onto the test function $v$. That way, we only require $u'$ to exist (so $u$ just needs to be once differentiable) and $v$ is smooth enough so that $v'$ exists too. This relaxes the regularity (that's the whole point of the weak formulation).

        \begin{examplebox}[: Integration by parts - Analogy]
            Imagine we want to test if someone's handwriting is smooth:
            \begin{itemize}
                \item Checking \textbf{second derivatives} is like asking for very fine, strict smoothness (hard!).
                \item By integrating by parts, we only ask them to be ``once smooth'', not ``twice smooth''.
                \item That's much easier to satisfy, and still captures the essence.
            \end{itemize}
        \end{examplebox}

        So in other words, integration by parts is necessary because it removes the second derivative on $u$, replacing it with first derivatives on both $u$ and the test function. This reduces the regularity requirement, making the weak problem solvable in larger spaces (like $H^1$).
    \end{itemize}


    \item \important{Define bilinear and linear forms.} Writing integrals every time is messy. Mathematicians like to give names to these two operations:
    \begin{itemize}
        \item The \textbf{left-hand side} looks like a special product between $u$ and $v$. So we call it a \textbf{bilinear form} (because it's linear in $u$ and in $v$ separately):
        \begin{equation}
            a(u,v) \coloneq \int_0^1 u'(x) \cdot v'(x)\, \mathrm{d}x
        \end{equation}
        It is like the \textbf{energy inner product} between $u$ and $v$.

        \item The \textbf{right-hand side} is an integral that only depends on $v$. So we call it a \textbf{linear functional} (linear in $v$):
        \begin{equation}
            F(v) \coloneq \int_0^1 f(x) \cdot v(x)\, \mathrm{d}x
        \end{equation}
        It is like the \textbf{effect of the forcing} $f$ measured against $v$.
    \end{itemize}
    \textcolor{Green3}{\faIcon{question-circle} \textbf{Why do this?}} Because once we define these two objects, the weak problem looks \textbf{super compact}:
    \begin{equation}\label{eq: weak formulation}
        \text{Find } u \in V \text{ such that } a(u,v) = F(v) \quad \forall v \in V
    \end{equation}
    That's just a clean way of saying:
    \begin{equation*}
        \int_0^1 u'(x) \cdot v'(x)\, \mathrm{d}x = \int_0^1 f(x) \cdot v(x)\, \mathrm{d}x \quad \forall v \in V        
    \end{equation*}
    In simple terms, the weak problem says: ``Find $u$ such that, when tested against all possible $v$, the internal energy balance $a(u,v)$ equals the external forcing $F(v)$''.
\end{enumerate}
