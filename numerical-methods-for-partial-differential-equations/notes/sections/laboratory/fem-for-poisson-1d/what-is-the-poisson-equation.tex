\subsection{FEM for Poisson 1D}\label{subsection: FEM for Poisson 1D}

\subsubsection{What is the Poisson Equation?}

The \definition{Poisson equation} is one of the most fundamental Partial Differential Equations (PDEs). In general form (in multiple dimensions):
\begin{equation}
    - \Delta u(x) = f(x), \quad x \in \Omega
\end{equation}
With some boundary conditions on $\partial \Omega$. Where:
\begin{itemize}
    \item $u(x)$ is the \textbf{unknown function} (temperature, displacement, potential, etc.).
    \item $\Delta$ is the \textbf{Laplacian operator}, i.e. sum of second derivatives.
    \begin{equation}
        \Delta u = \dfrac{\partial^{2} u}{\partial x_{1}^{2}} + \dfrac{\partial^{2} u}{\partial x_{2}^{2}} + \cdots
    \end{equation}
    \item $f(x)$ is the \textbf{source term} (where heat is produced, where force acts, etc.).
\end{itemize}
So the Poisson equation says \textbf{the curvature of $u$ (second derivate) balances the source $f$}.

\highspace
\begin{flushleft}
    \textcolor{Green3}{\faIcon{question-circle} \textbf{What does ``1D'' mean?}}
\end{flushleft}
Normally the Poisson equation is written in 2D or 3D (for surfaces and volumes).
\begin{itemize}
    \item In \textbf{2D}, it's like heat distribution on a plate.
    \item In \textbf{3D}, it's like heat or potential in a cube.
\end{itemize}
In \textbf{1D}, the domain $\Omega = (0,1)$ is just a line (an interval). So the Laplacian reduces to an ordinary second derivative:
\begin{equation*}
    \Delta u(x) = \frac{d^2 u}{dx^2}
\end{equation*}
Therefore, in 1D, the Poisson equation looks like:
\begin{equation}
    - u''(x) = f(x)    
\end{equation}
If we allow a variable coefficient $\mu(x)$, it becomes:
\begin{equation*}
    - \left(\mu(x) u'(x)\right)' = f(x)    
\end{equation*}

\highspace
\begin{examplebox}[: Physical analogy]
    Imagine a \textbf{metal bar of length 1}. Fix both ends to zero temperature (they're in contact with ice). Apply a \textbf{heat source} (or sink, if negative) in some region of the bar. Then the \textbf{temperature distribution} inside the bar is described by the 1D Poisson equation.

    Another analogy. Think of a \textbf{stretched elastic string} fixed at both ends. Apply a \textbf{vertical load} (the forcing $f$) on some region. The resulting shape of the string $u(x)$ satisfies the Poisson equation.
\end{examplebox}

\highspace
\begin{flushleft}
    \textcolor{Green3}{\faIcon{book} \textbf{The core idea}}
\end{flushleft}
The Poisson equation links:
\begin{itemize}
    \item The \textbf{curvature} of the unknown function $u(x)$ (its second derivate)
    \item To the \textbf{source term} $f(x)$.
\end{itemize}
Formally:
\begin{equation*}
    - u''(x) = f(x) \quad \text{(in 1D)}
\end{equation*}
That means wherever $f(x)$ is nonzero, it \emph{forces} the function $u(x)$ to bend (curve). If $f(x) = 0$, the equation reduces to:
\begin{equation*}
    - u''(x) = 0 \quad \Longrightarrow \quad u''(x)=0
\end{equation*}
Whose solutions are straight lines. So \textbf{no sources}, \textbf{flat solution}.

\highspace
\begin{flushleft}
    \textcolor{Green3}{\faIcon{question-circle} \textbf{Physical interpretations}}
\end{flushleft}
We can understand Poisson in multiple ways, depending on the field:
\begin{enumerate}
    \item \textbf{Heat conduction}. The equation says: ``The way temperature curves along the bar is dictated by how much heat we add/remove locally''.
    \begin{itemize}
        \item $u(x)$: temperature.
        \item $f(x)$: heat sources (positive) or sinks (negative).
    \end{itemize}
    \item \textbf{Electrostatics}. The equation says: ``Charges create curvature in the potential''.
    \begin{itemize}
        \item $u(x)$: electric potential.
        \item $f(x)$: charge distribution (density).
    \end{itemize}
    \item \textbf{Elasticity}. The equation says: ``Where we press on the string, it bends''.
    \begin{itemize}
        \item $u(x)$: displacement of a string or membrane.
        \item $f(x)$: applied load (force per unit length).
    \end{itemize}
\end{enumerate}

\newpage

\begin{figure}[!htp]
    \centering
    \includegraphics[width=\textwidth]{img/poisson-1d/poisson-1d.pdf}
    \captionsetup{singlelinecheck=off}
    \caption[]{Visual explanation of the 1D Poisson problem. \textcolor{Red2}{\textbf{Top graph (red)}} is the forcing term $f(x)$. It is \textbf{zero everywhere}, except between $x = \dfrac{1}{8}$ and $x = \dfrac{1}{4}$, where it is negative ($-1$). That means only in that region we have a ``sink'' of heat. The \textcolor{Blue2}{\textbf{bottom graph (blue)}} is the solution $u(x)$, i.e. the \textbf{temperature profile} along the bar. The ends are fixed at $u(0) = u(1) = 0$. In the middle, the solution bends downward because of the negative forcing. Outside the forcing region, the solution is almost straight (since $f = 0$, then the curvature is $0$). So visually, the bar stays at 0 at the ends, but dips in the middle where we apply the negative forcing.
    
    \highspace
    For example, this graph could represent heat conduction in a bar.
    \begin{itemize}
        \item $u(x)$: \textbf{temperature} along a thin bar of length 1.
        \item Boundaries: both ends clamped to 0 \textdegree C (in contact with ice).
        \item $f(x) = -1$ between $x=\dfrac{1}{8}$ and $x=\dfrac{1}{4}$: this is like a \textbf{cooling region} (a heat sink).
        \item Physical picture: the bar stays cold at the ends and gets even colder in the middle where cooling is applied, producing the ``dip''.
    \end{itemize}}
    \label{fig: Poisson 1D}
\end{figure}

\newpage

\begin{flushleft}
    \textcolor{Green3}{\faIcon{question-circle} \textbf{What is the purpose of the Poisson equation?}}
\end{flushleft}
The Poisson equation gives us the \textbf{response of the system}, not just the input. Why? Because physical systems are not isolated points:
\begin{itemize}
    \item Temperature at a point depends on how heat flows along the entire bar.
    \item Displacement of a string at one point depends on forces applied nearby and the string's stiffness.
    \item Potential at a point depends on all surrounding charges.
\end{itemize}
The Poisson equation encodes that \textbf{interaction through curvature}:
\begin{equation*}
    - u''(x) = f(x)
\end{equation*}
\begin{itemize}
    \item The second derivate $u''$ measures how much the solution bends.
    \item The boundary conditions (ends fixed at 0) propagate constraints.
    \item The combination of $f$ and boundaries gives the actual \textbf{shape of the solution}.
\end{itemize}
For example, consider figure \ref{fig: Poisson 1D} on page \pageref{fig: Poisson 1D}. In particular, consider our forcing (the red curve in the top graph). It is localized, but the solution (the blue curve in the bottom graph) is \textbf{spread out}:
\begin{itemize}
    \item The dip is not only in the forcing region, it extends beyond, up to the ends.
    \item This spreading comes from the diffusion mechanism encoded by Poisson.
\end{itemize}
So if we only look at $f(x)$, we know \emph{where the cause is}. But if we solve Poisson, we know \emph{how the whole system reacts}.

\begin{examplebox}[: Real-world analogy]
    Imagine putting an ice cube (the sink) in one part of a metal bar. From the forcing alone, we know \emph{where} the cooling happens. But we don't know how the \textbf{temperature profile along the whole bar} looks; is the bar uniformly cold? Does the dip propagate? Solving Poisson tells us exactly the \textbf{temperature distribution everywhere}, considering both the sink and the fixed-zero boundary conditions.
\end{examplebox}