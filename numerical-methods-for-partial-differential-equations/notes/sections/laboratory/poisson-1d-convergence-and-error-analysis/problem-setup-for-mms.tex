\subsubsection{Problem Setup for MMS}\label{sec:problem-setup-for-mms}

\begin{flushleft}
    \textcolor{Green3}{\faIcon{question-circle} \textbf{How to choose a smooth exact solution}}
\end{flushleft}
The key idea is to pick something we like, as long as it is smooth enough. Typical choices are \textbf{polynomials}, \textbf{trigonometric functions} (e.g. sines and cosines), or \textbf{exponentials}, because their derivatives are easy to compute.

\highspace
However, the solution must satisfy three primary requirements:
\begin{enumerate}
    \item The exact solution $u_\text{ex}$ must satisfy the \textbf{boundary conditions} of the problem (e.g., homogeneous Dirichlet $u(0) = u(1) = 0$).
    \begin{examplebox}[: Choose a smooth exact solution]
        If we pick $u_\text{ex}(x) = \sin(2 \pi x)$, it automatically vanishes at both ends:
        \begin{equation*}
            u_\text{ex}(0) = \sin(0) = 0, \quad u_\text{ex}(1) = \sin(2 \pi) = 0
        \end{equation*}
    \end{examplebox}
    \setcounter{example}{\theexample-1}
    

    \item The exact solution $u_\text{ex}$ must be \textbf{sufficiently smooth} (at least $C^2$) so that plugging into $-u'' = f$ makes sens and the theory applies.
    \begin{examplebox}[ (continue): Choose a smooth exact solution]
        Taking $u_\text{ex}(x) = \sin(2 \pi x)$ is a good choice since it is infinitely differentiable:
        \begin{equation*}
            u_\text{ex}'(x) = 2 \pi \cos(2 \pi x), \quad u_\text{ex}''(x) = -4 \pi^2 \sin(2 \pi x) \implies u_\text{ex} \in C^\infty
        \end{equation*}
    \end{examplebox}
    \setcounter{example}{\theexample-1}


    \item It should \textbf{not belong to our FE space} (so that error is not identically zero). For example, avoid linear functions when using $r = 1$ elements.
    \begin{examplebox}[ (continue): Choose a smooth exact solution]
        The choice $u_\text{ex}(x) = \sin(2 \pi x)$ is suitable since it is not a polynomial and cannot be exactly represented by piecewise linear functions.
    \end{examplebox}
\end{enumerate}

\highspace
\begin{flushleft}
    \textcolor{Green3}{\faIcon{book} \textbf{Deriving the forcing term}}
\end{flushleft}
Once we have chosen a suitable exact solution $u_\text{ex}(x)$, we can derive the corresponding forcing term $f(x)$ by substituting $u_\text{ex}$ into the Poisson equation:
\begin{equation*}
    -u_\text{ex}''(x) = f(x)
\end{equation*}
That's our \textbf{manufactured forcing term}\footnote{%
    Note: the manufactured forcing term is not necessarily physically meaningful, but it serves our purpose for verifying the numerical method.
}.
\begin{examplebox}[: Deriving the forcing term]
    For our example $u_\text{ex}(x) = \sin(2 \pi x)$, we compute:
    \begin{equation*}
        u_\text{ex}''(x) = -4 \pi^2 \sin(2 \pi x) \implies f(x) = 4 \pi^2 \sin(2 \pi x)
    \end{equation*}
    Thus, the manufactured solution $u_\text{ex}(x) = \sin(2 \pi x)$ satisfies the Poisson equation with the forcing term $f(x) = 4 \pi^2 \sin(2 \pi x)$ and homogeneous Dirichlet boundary conditions.
\end{examplebox}

\highspace
\begin{flushleft}
    \textcolor{Green3}{\faIcon{book} \textbf{Defining the PDE problem}}
\end{flushleft}
We can now summarize the complete PDE problem for the Method of Manufactured Solutions (\textbf{manufactured PDE}):
\begin{equation*}
    \begin{cases}
        -u''(x) = 4 \pi^2 \sin(2 \pi x), & x \in (0, 1) \\
        u(0) = 0, \\
        u(1) = 0
    \end{cases}
\end{equation*}
with the exact solution:
\begin{equation*}
    u_\text{ex}(x) = \sin(2 \pi x)
\end{equation*}
This setup allows us to implement the finite element method, compute the numerical solution $u_h$, and then compare it to the exact solution $u_\text{ex}$ to analyze convergence and error. 

\highspace
\begin{flushleft}
    \textcolor{Green3}{\faIcon{question-circle} \textbf{What do we gain?}}
\end{flushleft}
By using the Method of Manufactured Solutions:
\begin{itemize}
    \item We know the \textbf{exact solution} $u_\text{ex}$, allowing us to compute the error $e_h = u_\text{ex} - u_h$ directly.
    \item We can \textbf{compute the error in different norms} (e.g., $L^2$, $H^1$) to assess the accuracy of our numerical method.
    \item We can \textbf{check if the solver exhibits the expected convergence rates}:
    \begin{itemize}
        \item $e_{L^2} = O\left(h^{r+1}\right)$, means the error in the $L^2$ norm decreases proportionally to $h^{r+1}$ as the mesh is refined.
        
        \textcolor{Green3}{\faIcon{question-circle} \textbf{Why?}} Because the $L^2$ error is related to the smoothness of the exact solution and the order of the finite element space. As we refine the mesh (reduce $h$), the approximation becomes more accurate, leading to a faster decrease in the error.

        It's obvious that more elements mean more mesh points and more computational work, but also a better approximation of the exact solution.


        \item $e_{H^1} = O\left(h^r\right)$, means the error in the $H^1$ norm decreases proportionally to $h^r$ as the mesh is refined.
        
        \textcolor{Green3}{\faIcon{question-circle} \textbf{Why?}} The reason behind the convergence rates is rooted in the mathematical properties of the finite element method. The $L^2$ norm measures the error in the function values, while the $H^1$ norm measures the error in both the function values and their first derivatives. As the mesh is refined, the finite element space can better approximate the exact solution, leading to a faster decrease in the $L^2$ error. However, since the $H^1$ norm also considers the derivative, which is generally less smooth than the function itself, the convergence rate is typically one order lower.
    \end{itemize}
\end{itemize}