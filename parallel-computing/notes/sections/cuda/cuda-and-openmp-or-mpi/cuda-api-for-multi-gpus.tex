\subsubsection{CUDA API for Multi-GPUs}

In high-performance computing, the use of multiple GPUs can significantly increase computational throughput and efficiency. \textbf{CUDA provides several API calls that allow developers to effectively manage and utilize multiple GPUs}. The most important of these are:
\begin{itemize}\label{code: cudaSetDevice}
    \item The \important{\texttt{cudaSetDevice()}} function is used to \textbf{specify which GPU device should be active for subsequent CUDA operations on the host thread}. By setting the desired GPU device, developers can control where kernel executions, memory allocations, and other device-specific calls take place. This feature is especially important when working with systems that have multiple GPUs, as it allows fine-grained control over which device is used for specific tasks.
    \begin{itemize}
        \item \important{Function}:
        \begin{lstlisting}[language=C++]
cudaSetDevice(int device)\end{lstlisting}
        \item \important{Description}:
        \begin{itemize}
            \item Sets the GPU device to use for subsequent CUDA operations on the active host thread.
            \item This function specifies which GPU to use for operations like kernel launches, memory allocations, and other device-specific calls.
        \end{itemize}
        \item \important{Parameters}:
        \begin{itemize}
            \item \texttt{device}: the integer ID of the GPU to be set as the active device.
        \end{itemize}
        \item \important{Usage}:
        \begin{itemize}
            \item Does not affect other host threads, meaning the device setting is local to the thread where this function is called.
            \item Does not affect previous asynchronous calls.
        \end{itemize}
        \item \important{Example}:
        \begin{lstlisting}[language=C++]
cudaSetDevice(0); // Sets GPU 0 as the active device\end{lstlisting}
    \end{itemize}
    \begin{flushleft}
      \textcolor{Green3}{\faIcon{tachometer-alt} \textbf{CUDA Runtime Calls Affected by \texttt{cudaSetDevice()}}}
    \end{flushleft}
    If \texttt{cudaSetDevice()} is \textbf{called before}:
    \begin{itemize}
      \item \important{Kernel Launch}. The kernel will execute on the specified active device.
      \item \important{Memory Allocations}. All memory allocations will be made on the specified active device.
      \item \important{Stream Creation}. The stream will be associated with the specified active device.
      \item \important{Synchronization Functions}. These functions will synchronize\break tasks on the specified active device and active host thread only.
    \end{itemize}

    \item The \important{\texttt{cudaGetDevice()}} function \textbf{retrieves the ID of the GPU currently being used by the active host thread}. This function is useful for confirming which GPU is set as active, especially in debugging scenarios. Knowing the active device ensures that operations are being performed on the intended GPU, preventing potential errors and resource conflicts.
    \begin{itemize}
        \item \important{Function}:
        \begin{lstlisting}[language=C++]
cudaGetDevice(int *device)\end{lstlisting}
        \item \important{Description}:
        \begin{itemize}
            \item Retrieves the ID of the GPU currently being used by the active host thread.
            \item Helps to confirm which GPU is currently set as active.
        \end{itemize}
        \item \important{Parameters}:
        \begin{itemize}
            \item \texttt{device}: a pointer to an integer where the ID of the active device will be stored.
        \end{itemize}
        \item \important{Usage}:
        \begin{itemize}
            \item This function is useful for debugging and ensuring that the correct GPU is being utilized.
        \end{itemize}
        \item \important{Example}:
        \begin{lstlisting}[language=C++]
int device;
// Gets the ID of the current active GPU
// and stores it in 'device'
cudaGetDevice(&device);\end{lstlisting}
    \end{itemize}
    
    \item The \important{\texttt{cudaGetDeviceCount()}} function is used to \textbf{determine the number of CUDA-capable GPUs available in the system}. This function is typically called during the initialization phase to understand the GPU resources available. By knowing the number of GPUs, developers can design their applications to efficiently distribute workloads across multiple devices, maximizing performance and resource utilization.
    \begin{itemize}
        \item \important{Function}:
        \begin{lstlisting}[language=C++]
cudaGetDeviceCount(int *count)\end{lstlisting}
        \item \important{Description}:
        \begin{itemize}
            \item Determines the number of CUDA-capable GPUs available in the system.
            \item Helps in discovering the total number of GPUs that can be utilized for computation.
        \end{itemize}
        \item \important{Parameters}:
        \begin{itemize}
            \item \texttt{count}: a pointer to an integer where the number of CUDA-capable devices will be stored.
        \end{itemize}
        \newpage
        \item \important{Usage}:
        \begin{itemize}
            \item This function is typically called during the initialization phase to understand the GPU resources available.
        \end{itemize}
        \item \important{Example}:
        \begin{lstlisting}[language=C++]
int deviceCount;
// Gets the number of CUDA-capable GPUs
// and stores it in 'deviceCount'
cudaGetDeviceCount(&deviceCount);\end{lstlisting}
    \end{itemize}
\end{itemize}
