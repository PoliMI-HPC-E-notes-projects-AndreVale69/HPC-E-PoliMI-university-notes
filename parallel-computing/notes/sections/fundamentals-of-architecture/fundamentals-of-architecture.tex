\section{Fundamentals of architecture}

\subsection{Introduction}

\subsubsection{Simplest processor}

Inside a computer, a processor executes instructions.
\begin{itemize}
    \item \textbf{Fetch/Decode}: Determine which instruction to run next;
    \item \textbf{ALU} (execution unit): Performs the operation described by an instruction, which may change values in the processor's registers or the computer's memory;
    \item \textbf{Registers}: maintain program state, store values of variables used as inputs and outputs to operations.
\end{itemize}
The simplest and most basic processor executes \textbf{one instruction per clock cycle}.
\begin{figure}[!htp]
    \centering
    \includegraphics[width=.3\textwidth]{img/simplest-prcoessor-1.pdf}
    \caption{The simplest and most basic processor.}
\end{figure}

\newpage

\subsubsection{Superscalar processor}
A more \dquotes{complex} and realistic model is the \definitionWithSpecificIndex{superscalar processor}{Superscalar Processor}. This \textbf{processor can decode and execute up to \underline{two instructions per clock}}. The execution is slightly different from the simplest processor. The \textbf{processor automatically finds independent instructions in an instruction sequence and can execute them in parallel on multiple execution units}.
\begin{figure}[!htp]
    \centering
    \includegraphics[width=.52\textwidth]{img/superscalar-prcoessor-1.pdf}
    \caption{The superscalar processor.}
\end{figure}

\noindent
The superscalar processor takes advantage of \definition{Instruction-Level Parallelism (ILP)}\footnote{Instruction-level parallelism (ILP) is the parallel or simultaneous execution of a sequence of instructions in a computer program. More specifically ILP refers to the average number of instructions run per step of this parallel execution.} within an instruction stream.
\begin{itemize}
    \item Processing \textbf{different instructions} from the same instruction stream \textbf{in parallel} (\textbf{within a core}).
    \item \textbf{Parallelism is automatically detected by the hardware during execution}.
\end{itemize}

\newpage

\subsubsection{Single Instruction, Multiple Data (SIMD) processor}

Adding execution units (ALUs) to the simplest processor can increase compute capability. Amortize the cost/complexity of managing an instruction stream across many ALUs using \definition{Single Instruction, Multiple Data (SIMD)} processing. Therefore, the \textbf{same instruction is sent to all ALUs}. This \textbf{operation is performed in parallel on all ALUs}.

\begin{flushleft}
    \textcolor{Green3}{\faIcon{check} \textbf{Advantages}}
\end{flushleft}
\begin{itemize}
    \item \textbf{Efficient for data-parallel workloads}: amortize control costs over many ALUs.
    \item Vectorization done by:
    \begin{itemize}
        \item Compiler (\textbf{\underline{explicit SIMD}}): parallelism is explicitly requested by the programmer through intrinsics, conveyed through parallel language semantics, and inferred through loop dependency analysis by the \dquotes{auto-vectorizing} compiler. In other words, the \textbf{SIMD parallelization is done at compile time, and when we inspect the program binary, we can see the SIMD instructions}.

        \item At runtime by hardware (\textbf{\underline{implicit SIMD}}): the \textbf{compiler generates a binary with scalar instructions}, but $n$ instances of the program are always executed together on the processor. The \textbf{hardware} (not the compiler) is \textbf{responsible for the simultaneous execution of the same instruction by multiple program instances on different data on SIMD ALUs}.
    \end{itemize}
\end{itemize}

\longline

\subsubsection{Multi-Core Processor}

A \definition{Multi-Core Processor (MCP)} is a \textbf{microprocessor} on a single integrated circuit (IC) with \textbf{two or more separate central processing units (CPUs)}, called \emph{cores} to emphasize their multiplicity (e.g., \emph{dual-core} or \emph{quad-core}). Each core reads and executes program instructions, specifically ordinary CPU instructions (such as \texttt{add}, \texttt{move data}, and \texttt{branch}). However, the MCP can \textbf{execute instructions on separate cores simultaneously}, \textbf{increasing overall speed for programs that support multithreading or other parallel computing techniques}.
\begin{flushleft}
    \textcolor{Green3}{\faIcon{check} \textbf{Advantages}}
\end{flushleft}
\begin{itemize}
    \item Provides \textbf{thread-level parallelism}: execute a completely different instruction stream on each core simultaneously.
    \item \textbf{Software creates threads to expose parallelism to hardware} (e.g., via threading API)
\end{itemize}