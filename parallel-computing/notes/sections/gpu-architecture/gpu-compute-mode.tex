\subsection{GPU compute mode}

\definition{GPU compute mode} refers to \textbf{GPU hardware that is optimized for general-purpose computing rather than graphics rendering}. This mode allows users to run non-graphics programs on the GPU's programmable cores, taking advantage of the GPU's parallel processing capabilities for tasks such as scientific simulations, data analysis, and machine learning.

\begin{examplebox}[: how to run code on a GPU (prior to 2007)]
    Now let us see how to run a simple code on a GPU. Suppose a user wants to draw a picture on a GPU:
    \begin{itemize}
        \item \textbf{GPU Shader Program Binaries}. The application, via the graphics driver, supplies the GPU with shader program binaries. These are compiled programs that the GPU will execute to perform rendering tasks.
        
        \item \textbf{Graphics Pipeline Parameters}. The application sets various parameters for the graphics pipeline, such as the output image size, to control how the rendering should be processed.
        
        \item \textbf{Vertex Buffer}. The application provides the GPU with a buffer of vertices. Vertices are data points that define the shape of the objects to be rendered.
        
        \item \textbf{Draw Command}. The application sends a draw command to the GPU using the function call \texttt{drawPrimitives(vertex\_buffer)}. This command instructs the GPU to start rendering using the provided vertex data.
    \end{itemize}
    The stages of the graphics pipeline:
    \begin{enumerate}
        \item \textbf{Input Vertex Buffer}: The initial stage where the vertex data is input to the pipeline.

        \item \textbf{Vertex Generation}: Vertices are generated or fetched from the vertex buffer.
        
        \item \textbf{Vertex Processing}: The vertices undergo various transformations and shading calculations.
        
        \item \textbf{Primitive Generation}: The processed vertices are used to generate geometric primitives (such as triangles).
        
        \item \textbf{Fragment Generation (Rasterization)}: The primitives are converted into fragments (potential pixels).
        
        \item \textbf{Fragment Processing}: Fragments undergo shading and texturing calculations to determine their final color and properties.
        
        \item \textbf{Pixel Operations}: Final operations are performed on the fragments, such as depth testing and blending.
        
        \item \textbf{Output Image Buffer}: The processed fragments are written to the output image buffer, resulting in the final rendered image.
    \end{enumerate}
\end{examplebox}

\begin{flushleft}
    \textcolor{Green3}{\faIcon{history} \textbf{Some history}}
\end{flushleft}
\textbf{Before 2007} the \textbf{only way to interface with GPU hardware was through the graphics pipeline}. This meant that GPUs were \emph{\textbf{designed and used specifically for tasks related to graphics rendering}}. The pipeline stages (such as vertex processing, fragment generation, and pixel operations) were all designed to transform vertex data into pixels displayed on the screen.

\highspace
Because they were optimized to handle parallel tasks associated with rendering images, their architecture and interfaces were tightly coupled with graphics APIs.

\highspace
\textbf{By 2007}, the concept of using GPUs for \definition{General-Purpose computing on Graphics Processing Units (GPGPU)} was emerging. Thanks mainly to the introduction of a new architecture signed NVIDIA called Tesla and CUDA, a parallel computing platform and programming model (also OpenCL was emerging).

\highspace
The \definition{NVIDIA Tesla architecture}, introduced with the \href{https://www.techpowerup.com/gpu-specs/geforce-8800-gt.c201}{GeForce 8800 GPU} in 2006, marked a significant shift in GPU design by unifying graphics and computing capabilities. This architecture featured a scalable parallel array of processors that could be programmed in \texttt{C} or via graphics \texttt{APIs2}. The Tesla architecture enabled \textbf{flexible}, \textbf{programmable graphics and high-performance computing}, making it possible to use GPUs for a wide range of applications beyond traditional graphics rendering. This unification enabled massive multithreading and parallel processing, dramatically improving performance for computationally intensive tasks.\cite{lindholm2008nvidia}

\highspace
From this point on, the programmable cores of the GPU are used:
\begin{itemize}
    \item Application could allocate buffers in GPU memory and copy data to/from buffers;
    \item Application (via the graphics driver) provides a single kernel program binary to the GPU;
    \item Application tells GPU to run kernel in SPMD fashion.
\end{itemize}