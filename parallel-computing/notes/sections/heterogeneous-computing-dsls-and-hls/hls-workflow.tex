\subsection{HLS Workflow}

This section explains \textbf{\emph{how} HLS converts high-level code} (\texttt{C}/\texttt{C++}/\texttt{Python}) \textbf{into hardware} (Verilog/VHDL) and the key steps in the HLS design flow.

\highspace
\begin{flushleft}
    \textcolor{Green3}{\faIcon{keyboard} \textbf{Inputs to an HLS Compiler}}
\end{flushleft}
To generate hardware from high-level code, an \textbf{HLS tool requires three main inputs}:
\begin{enumerate}
    \item \important{High-Level Code} (\texttt{C}, \texttt{C++}, or \texttt{Python})
    \begin{itemize}
        \item Describes the \textbf{algorithm's behavior}.
        \item Written similarly to software but \textbf{optimized for hardware}.
    \end{itemize}
    \item \important{Library of Characterized Modules}
    \begin{itemize}
        \item \textbf{Predefined hardware} building blocks (e.g., adders, multipliers, memory units).
        \item Helps the compiler understand available resources.
    \end{itemize}
    \item \important{Constraints \& Optimization Directives}
    \begin{itemize}
        \item Designer-defined constraints such as:
        \begin{itemize}
            \item \textbf{Area constraints} (how much hardware can be used).
            \item \textbf{Timing constraints} (desired clock speed).
            \item \textbf{Memory hierarchy} (external vs. on-chip memory).
        \end{itemize}
        \item Optional \textbf{HLS pragmas} (e.g., loop unrolling, pipelining) to fine-tune performance.
    \end{itemize}
\end{enumerate}

\highspace
\begin{flushleft}
    \textcolor{Green3}{\faIcon{question-circle} \textbf{HLS Objectives}}
\end{flushleft}
The goal of \textbf{HLS is to generate an efficient hardware design} based on the following objectives:
\begin{itemize}[label=\textcolor{Green3}{\faIcon{check}}]
    \item \textcolor{Green3}{\textbf{Minimize Area}}: Uses fewer functional units, registers, and interconnects.
    \item \textcolor{Green3}{\textbf{Maximize Speed}}: Reduces the number of clock cycles (latency) and increases throughput.
    \item \textcolor{Green3}{\textbf{Optimize Power}}: Reduces energy consumption for embedded systems.
\end{itemize}
However, the main trade-offs must be:
\begin{itemize}
    \item \textbf{Optimizing for speed} $\Rightarrow$ may increase hardware area.
    \item \textbf{Reducing hardware area} $\Rightarrow$ may increase execution time.
\end{itemize}

\newpage

\begin{flushleft}
    \textcolor{Green3}{\faIcon{microchip} \textbf{HLS Compilation Flow}}
\end{flushleft}
HLS follows \textbf{three main steps}:
\begin{itemize}
    \item \important{Front-End (Parsing and IR Generation)}. \textbf{Converts} \texttt{C++}/\texttt{Python} \textbf{code} into an \definition{Intermediate Representation (IR)}. IR is \textbf{similar to software compiler representations} (e.g., LLVM IR). In other words, it breaks the program down into basic operations.
    
    \item \important{Middle-End (Optimization and Scheduling)}. At this point, the HLS tools performed three important operations:
    \begin{enumerate}
        \item It \textbf{analyzes data dependencies} and \textbf{determines execution order}.
        \item It performs \textbf{scheduling} (decides when each operation runs).
        \item It allocates \textbf{hardware resources} (multipliers, memory, registers).
    \end{enumerate}
    \begin{examplebox}[: Scheduling Choices]
        \begin{center}
            \begin{tabular}{@{} l p{15em} @{}}
                \toprule
                \textbf{Scheduling Strategy} & \textbf{Impact} \\
                \midrule
                \textbf{As Soon As Possible} & Minimizes latency, but may use more hardware. \\ [.5em]
                \textbf{As Late As Possible} & Reduces hardware usage, but may increase delay. \\ [.5em]
                \textbf{Loop Unrolling}      & Increases parallelism, but requires more area. \\ [.5em]
                \textbf{Pipelining}          & Allows overlapping computations to improve throughput. \\
                \bottomrule
            \end{tabular}
        \end{center}
    \end{examplebox}

    \item \important{Back-End (Hardware Generation)}. Finally, the tools made two steps:
    \begin{enumerate}
        \item \textbf{Converts optimized IR into Verilog/VHDL}.
        \item \textbf{Generates a Finite State Machine with Datapath (FSMD)} representation.
    \end{enumerate}
    The result is a \textbf{synthesizable hardware description} ready for FPGA/ASIC implementation.

    Therefore, a \textbf{final output} of HLS tools is:
    \begin{itemize}
        \item Verilog/VHDL code (for FPGA/ASIC synthesis).
        \item Datapath and Controller Design.
        \item Simulation files for verification.
    \end{itemize}
\end{itemize}

\newpage

\begin{flushleft}
    \textcolor{Green3}{\faIcon{check} \textbf{Conclusion}}
\end{flushleft}
In conclusion, the key takeaways are:
\begin{itemize}
    \item \textbf{HLS automates hardware design} from high-level code (\texttt{C}/\texttt{C++}/\texttt{Python}).
    \item The workflow involves \textbf{parsing}, \textbf{optimization}, \textbf{scheduling}, and \textbf{hardware generation}.
    \item \textbf{Performance tuning} requires adjusting \textbf{scheduling}, \textbf{pipelining}, and \textbf{parallelism}.
    \item \textbf{HLS outputs Verilog/VHDL}, which can be synthesized on FPGAs/ASICs.
\end{itemize}
