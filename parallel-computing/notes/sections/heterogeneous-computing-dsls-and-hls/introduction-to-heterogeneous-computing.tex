\section{Heterogeneous Computing - DSLs and HLS}

\subsection{Introduction to Heterogeneous Computing}

\definition{Heterogeneous Computing} (or \definition{Heterogeneous Processing}) refers to \textbf{systems that use multiple types of processors or accelerators to handle different workloads more efficiently}.
\begin{itemize}
    \item In contrast to traditional homogeneous systems (which only use CPUs), heterogeneous systems combine different processing units such as CPUs, GPUs, DSPs, and FPGAs.
    \item The \textbf{goal is to match the right processor to the right task},  achieving higher performance and energy efficiency.  
\end{itemize}

\begin{examplebox}[: Heterogeneous Processing]
    A self-driving car requires CPUs for decision-making, GPUs for image recognition, and FPGAs for real-time sensor fusion.
\end{examplebox}

\highspace
\begin{flushleft}
    \textcolor{Green3}{\faIcon{\speedIcon} \textbf{Energy-Efficient Computing Strategies}}
\end{flushleft}
When designing a heterogeneous system, performance isn't the only goal; \textbf{energy efficiency is just as critical}. Given a fixed power budget, simply \textbf{increasing performance without considering power constraints is inefficient}. Specialized hardware (e.g., FPGAs, ASICs) achieves better performance per watt than general-purpose processors.

\highspace
There are two main strategies for improving energy efficiency:
\begin{enumerate}
    \item \important{Use Specialized Processors}. \textbf{CPUs are not energy-efficient} due to instruction decoding, branch handling, and pipeline management overhead. Specialized hardware (FPGAs, ASICs) reduces overhead, leading to more computations per joule.
    \begin{equation*}
        \text{Power} = \dfrac{\text{Op}}{\text{second}} \times \dfrac{\text{Joules}}{\text{Op}}
    \end{equation*} 
    \item \important{Minimize Data Movement}. \textbf{Memory access consumes more energy than computation!} Optimizing data locality reduces power consumption. For example, moving computation closer to memory (e.g., using tensor core inside GPUs) significantly reduces energy cost.
\end{enumerate}
