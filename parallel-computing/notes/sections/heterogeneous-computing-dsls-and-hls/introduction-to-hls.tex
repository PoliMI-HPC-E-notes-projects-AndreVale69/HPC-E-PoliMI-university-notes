\subsection{Introduction to HLS}

\begin{flushleft}
    \textcolor{Green3}{\faIcon{question-circle} \textbf{What is High-Level Synthesis (HLS)?}}
\end{flushleft}
\definition{High-Level Synthesis (HLS)} is a \textbf{process that converts high-level software code} (\texttt{C}/\texttt{C++}/\texttt{Python}) \textbf{into hardware designs} (Verilog/VHDL). It \textbf{automates hardware generation}, allowing developers to describe behavior in software-like code while the tool generates optimized circuits.

\highspace
\begin{flushleft}
    \textcolor{Red2}{\faIcon{book} \textbf{Traditional FPGA/ASIC Design Flow (Without HLS)}}
\end{flushleft}
The traditional FPGA or ASIC design flow is divided into three main steps:
\begin{enumerate}
    \item \textbf{Write Register-Transfer Level (RTL) code} in Verilog/VHDL.
    \item Manually optimize for \textbf{timing, area, power}.
    \item Run \textbf{logic synthesis}, \textbf{place \& route}, and \textbf{fabrication}.
\end{enumerate}
The main \textbf{problems} with this approach are:
\begin{itemize}[label=\textcolor{Red2}{\faIcon{times}}]
    \item \textcolor{Red2}{\textbf{Time-consuming}} (designing hardware manually takes months).
    \item \textcolor{Red2}{\textbf{Error-prone}} (low-level bugs are hard to debug).
    \item \textcolor{Red2}{\textbf{Difficult to modify}} (small changes require rewriting RTL code).
\end{itemize}

\begin{table}[!htp]
    \centering
    \begin{tabular}{@{} l p{11.3em} p{11.3em} @{}}
        \toprule
        \textbf{Aspect} & \textbf{Traditional RTL} & \textbf{High-Level Synthesis} \\
        \midrule
        \textbf{Design Level} & Low-level: gates, registers & High-level (\texttt{C++}, \texttt{Python}) \\ [.7em]
        \textbf{Productivity} & \textcolor{Red2}{\faIcon{times}} Time-consuming & \textcolor{Green3}{\faIcon{check}} Faster development \\ [.7em]
        \textbf{Optimizations} & Manual pipeline and control logic & Automated scheduling \\ [.7em]
        \textbf{Reusability} & \textcolor{Red2}{\faIcon{times}} Difficult to modify & \textcolor{Green3}{\faIcon{check}} Easily reusable code \\ [.7em]
        \textbf{Learning Curve} & Steep (hardware expertise needed) & Easier (similar to software programming) \\
        \bottomrule
    \end{tabular}
    \caption{\textcolor{Green3}{\faIcon{balance-scale}} Differences Between HLS and RTL-Based Hardware Design.}
\end{table}

\noindent
We can conclude that \textbf{HLS allows software engineers to efficiently design hardware without deep knowledge of Verilog/VHDL}.

\newpage

\begin{flushleft}
    \textcolor{Green3}{\faIcon{check-circle} \textbf{Why use HLS instead of traditional RTL design? Benefits of HLS}}
\end{flushleft}
\begin{enumerate}[label=\textcolor{Green3}{\faIcon{check}}]
    \item \textcolor{Green3}{\textbf{Faster Development Cycle}}. \textbf{Designers can write \texttt{C++}/\texttt{Python} instead of Verilog}, reducing design time.
    \item \textcolor{Green3}{\textbf{Automated Optimizations}}. \textbf{HLS compilers automatically optimize} \hl{parallelism}, \hl{pipelining}, and \hl{resource allocation}.
    \item \textcolor{Green3}{\textbf{Easier HW/SW Co-Design}}. Enables \textbf{rapid prototyping of hardware accelerators}.
    \item \textcolor{Green3}{\textbf{Technology Independence}}. The \textbf{same \texttt{C++} code can be compiled for different FPGA/ASIC platforms}.
\end{enumerate}

\highspace
\begin{flushleft}
    \textcolor{Red2}{\faIcon{times-circle} \textbf{Challenges of HLS}}
\end{flushleft}
\begin{itemize}
    \item \textbf{Not all software code can be efficiently translated into hardware}.
    \item \textbf{HLS tools must explore a huge design space} (parallelism, pipelining, resource constraints).
    \item \textbf{Requires careful optimization of memory access and data flow}.
\end{itemize}

\highspace
\begin{flushleft}
    \textcolor{Green3}{\faIcon{check} \textbf{Conclusion}}
\end{flushleft}
In conclusion, the key takeaways are:
\begin{itemize}
    \item HLS \textbf{translates high-level software} (\texttt{C}/\texttt{C++}/\texttt{Python}) \textbf{into hardware designs} (Verilog/VHDL).
    \item It automates the hardware design process, \textbf{reducing time and complexity}.
    \item HLS enables software engineers to \textbf{design hardware accelerators without deep RTL expertise}.
    \item Challenges include \textbf{optimizing memory access, parallelism, and\break scheduling}.
\end{itemize}