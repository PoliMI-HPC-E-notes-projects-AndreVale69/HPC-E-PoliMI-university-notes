\subsection{Scheduling in HLS}

This section explores how \textbf{scheduling} affects performance in HLS, including key \textbf{scheduling techniques} like \textbf{ASAP, ALAP, list-based scheduling, and constrained scheduling}.

\begin{flushleft}
    \textcolor{Green3}{\faIcon{question-circle} \textbf{What is Scheduling in HLS?}}
\end{flushleft}
Scheduling in HLS determines \textbf{when each operation in the high-level code is executed} while considering \textbf{hardware resource constraints}.

\highspace
The \textbf{goal} is to \textbf{minimize execution time (\emph{latency}) while using minimal hardware resources (\emph{area})}. Scheduling \textbf{maps operations to clock cycles}, ensuring dependencies are met.  

\highspace
\begin{flushleft}
    \textcolor{Green3}{\faIcon{\speedIcon} \textbf{Scheduling Techniques}}
\end{flushleft}
Different \textbf{scheduling algorithms} can be used based on constraints (area, speed, power):
\begin{itemize}
    \item \definition{ASAP (As Soon As Possible) Scheduling}. Tries to execute operations \textbf{as early as possible}. \hl{Minimizes latency}, but may \hl{increase hardware resource usage}.
    \begin{flushleft}
        \textcolor{Green3}{\faIcon{check-circle} \textbf{Pros}}
    \end{flushleft}
    \begin{itemize}[label=\textcolor{Green3}{\faIcon{check}}]
        \item Good for high-speed designs.
    \end{itemize}
    \begin{flushleft}
        \textcolor{Red2}{\faIcon{times-circle} \textbf{Cons}}
    \end{flushleft}
    \begin{itemize}[label=\textcolor{Red2}{\faIcon{times}}]
        \item Can use too many resources if multiple operations execute simultaneously.
    \end{itemize}
    \begin{examplebox}[: ASAP Scheduling]
        \begin{lstlisting}[language=c++]
Cycle 1: x1 = a * b;  // Multiplication
Cycle 2: x2 = c * d;
Cycle 3: y = x1 + x2; // Addition\end{lstlisting}
    \end{examplebox}


    \item \definition{ALAP (As Late As Possible) Scheduling}. \textbf{Delays operations until the latest possible cycle}. \hl{Reduces resource usage}, but may \hl{increase latency}.
    \begin{flushleft}
        \textcolor{Green3}{\faIcon{check-circle} \textbf{Pros}}
    \end{flushleft}
    \begin{itemize}[label=\textcolor{Green3}{\faIcon{check}}]
        \item Minimizes resource usage (fewer active units per cycle).
    \end{itemize}
    \begin{flushleft}
        \textcolor{Red2}{\faIcon{times-circle} \textbf{Cons}}
    \end{flushleft}
    \begin{itemize}[label=\textcolor{Red2}{\faIcon{times}}]
        \item May introduce unnecessary delays.
    \end{itemize}
    \begin{examplebox}[: ALAP Scheduling]
        \begin{lstlisting}[language=c++]
Cycle 1: x1 = a * b;  // Multiplication
Cycle 2: (idle)       // Delays execution
Cycle 3: x2 = c * d;
Cycle 4: y = x1 + x2; // Addition\end{lstlisting}
    \end{examplebox}


    \item \definition{List-Based Scheduling}. Uses \textbf{priority-based selection} of operations. \hl{Balances latency vs. resource constraints}.
    \begin{flushleft}
        \textcolor{Green3}{\faIcon{tools} \textbf{Steps}}
    \end{flushleft}
    \begin{enumerate}
        \item Construct a \textbf{priority list based on dependencies}.
        \item \textbf{Schedule operations by selecting the highest-priority} task that fits \hl{resource constraints}.
        \item \textbf{Continue until all operations are scheduled}.
    \end{enumerate}
    \begin{flushleft}
        \textcolor{Green3}{\faIcon{check-circle} \textbf{Pros}}
    \end{flushleft}
    \begin{itemize}[label=\textcolor{Green3}{\faIcon{check}}]
        \item \textbf{Efficient trade-off} between \textbf{area} and \textbf{speed}.
        \item Used in most HLS tools (e.g., Vitis HLS, Bambu HLS).
    \end{itemize}


    \item \definition{Constrained Scheduling}. Used when \textbf{hardware resources} (adders, multipliers, memory) \textbf{are limited}. Tries to \hl{minimize execution time given a fixed number of resources}.
    \begin{flushleft}
        \textcolor{Green3}{\faIcon{check-circle} \textbf{Pros}}
    \end{flushleft}
    \begin{itemize}[label=\textcolor{Green3}{\faIcon{check}}]
        \item Uses minimal hardware resources.
    \end{itemize}
    \begin{flushleft}
        \textcolor{Red2}{\faIcon{times-circle} \textbf{Cons}}
    \end{flushleft}
    \begin{itemize}[label=\textcolor{Red2}{\faIcon{times}}]
        \item May increase total latency due to resource sharing.
    \end{itemize}
    \begin{examplebox}[: Constrained Scheduling]
        \begin{center}
            \begin{tabular}{@{} c c c @{}}
                \toprule
                \textbf{Clock Cycle} & \textbf{Adder (1 avail.)} & \textbf{Multiplier (1 avail.)} \\
                \midrule
                1 & \texttt{-} & \texttt{x1 = a * b} \\ [.5em]
                2 & \texttt{-} & \texttt{x2 = c * d} \\ [.5em]
                3 & \texttt{y = x1 + x2} & \texttt{-} \\
                \bottomrule
            \end{tabular}
        \end{center}
    \end{examplebox}
\end{itemize}

\newpage

\begin{examplebox}[: Scheduling a Basic Block]
    Consider the following code:
    \begin{lstlisting}[language=c++]
tmp1 = a * b;
tmp2 = c * d;
x1 = tmp1 * tmp2;
x2 = a * d;
x3 = x2 - x1;
x4 = x2 + x3;\end{lstlisting}
    Choosing the right schedule depends on whether the goal is to minimize execution time or minimize resource usage. However, here we have different scheduling results grouped by the method chosen.
    \begin{center}
        \begin{tabular}{@{} l l l @{}}
            \toprule
            \textbf{Scheduling} & \textbf{Total Clock Cycles} & \textbf{Hardware Used} \\
            \midrule
            \textbf{ASAP}           & \textcolor{Green3}{\faIcon{\speedIcon} \textbf{4 Cycles}}   & High resource usage \\ [.5em]
            \textbf{ALAP}           & \textcolor{Green4}{\faIcon{bolt} \textbf{6 Cycles}}         & Low resource usage \\ [.5em]
            \textbf{List-Based}     & \faIcon{balance-scale} \textbf{Balanced}                    & Medium resource usage \\ [.5em]
            \textbf{Constrained}    & \textcolor{Red2}{\faIcon{hourglass-half} \textbf{Depends (constraints)}} & Minimum resources \\
            \bottomrule
        \end{tabular}
    \end{center}
\end{examplebox}

\highspace
\begin{flushleft}
    \textcolor{Green3}{\faIcon{check} \textbf{Conclusion}}
\end{flushleft}
In conclusion, the key takeaways are:
\begin{itemize}
    \item Scheduling determines \textbf{when operations execute}, impacting performance.
    \item \textbf{ASAP} scheduling \textbf{minimizes latency} but may use \textbf{more resources}.
    \item \textbf{ALAP} scheduling \textbf{minimizes resource} usage but may \textbf{increase latency}.
    \item \textbf{List-based} scheduling \textbf{balances speed and area} constraints.
    \item \textbf{Constrained} scheduling \textbf{optimizes execution within hardware limits}.
\end{itemize}
