\subsection{Challenges of heterogeneous designs}

The challenges of heterogeneous computing designs are:
\begin{itemize}
    \item \important{Choosing the Right Processor for Each Task}. In a heterogeneous system, one of the major challenges is \textbf{identifying which processor} (CPU, GPU, FPGA, DSP, etc.) \textbf{is best suited for a particular task}. Each type of processor has its strengths and optimal use cases, making the selection process essential for maximizing performance and efficiency.

    \item \important{Software Development Challenges}:
    \begin{itemize}
        \item \textbf{Mapping Applications to Resources}. Developers need to \hl{design algorithms that decompose into components and map effectively onto the heterogeneous mix of processing units}. This means picking the right tool for the job and ensuring each component works efficiently on its intended hardware.

        \item \textbf{Complex Scheduling}. The \hl{scheduling problem becomes more complex} in heterogeneous systems \hl{due to the need to dynamically allocate tasks across different processors}. This requires advanced scheduling algorithms to balance workload distribution and avoid bottlenecks.
        
        \item \textbf{Programming Complexity}. \hl{Heterogeneous systems often involve multiple programming languages and instruction set architectures (ISAs)}, increasing the complexity of development. Developers must be proficient in various programming models and tools to create optimized applications.
    \end{itemize}

    \item \important{Hardware Design Challenges}:
    \begin{itemize}
        \item \textbf{Resource Mixture Balance}. Hardware designers face the challenge of \hl{determining the appropriate mix of resources within the system}. Allocating too few throughput-oriented resources can lower peak throughput for parallel workloads, while too few sequential processing resources can limit performance for sequential tasks.

        \item \textbf{Specialized Function Allocation}. Designers must decide \hl{how much chip area should be dedicated to specific functions}, such as video processing, AI inference, or data encryption. This allocation impacts the system's overall performance and efficiency.
    \end{itemize}

    \item \important{Interoperability and Integration}:
    \begin{itemize}
        \item \textbf{Interfacing Different Components}. Ensuring \hl{seamless communication and interoperability among different processors within a heterogeneous system} can be challenging. Different processors may have distinct communication protocols and data formats, requiring sophisticated integration strategies.
        
        \item \textbf{Unified Memory Access}. Achieving \hl{efficient shared memory access across various processors is crucial}. Unified memory architectures help, but they require careful design to avoid performance penalties and ensure data consistency.
    \end{itemize}

    \newpage

    \item \important{Optimization and Tuning}:
    \begin{itemize}
        \item \textbf{Performance Tuning}. Optimizing the performance of a heterogeneous system involves \hl{fine-tuning each component and the overall system architecture}. This process can be time-consuming and requires deep knowledge of each processor's capabilities and limitations.

        \item \textbf{Energy Efficiency}. \hl{Balancing performance and energy efficiency} is essential, especially for battery-powered devices. Designers need to optimize workloads to minimize power consumption without sacrificing performance.
    \end{itemize}

    \item \important{Cost and Complexity}:
    \begin{itemize}
        \item \textbf{Development Costs}. The \hl{complexity of developing and maintaining heterogeneous systems can increase costs}. This includes the need for specialized hardware, development tools, and skilled personnel.

        \item \textbf{Increased Complexity}. The inherent complexity of heterogeneous systems can lead to \hl{longer development cycles and more potential points of failure}. Ensuring robust design and testing processes is critical to handle this complexity.
    \end{itemize}
\end{itemize}

\highspace
\begin{flushleft}
    \textcolor{Red2}{\faIcon{exclamation-triangle} \textbf{Pitfalls of Heterogeneous Designs}}
\end{flushleft}
Heterogeneous computing systems are designed to handle diverse and complex workloads by utilizing various specialized processors. While this architecture offers significant advantages in terms of performance, efficiency, and flexibility, it also presents several notable pitfalls:
\begin{itemize}
    \item \important{\underline{Under}-Provisioning Specialized Components}:
    \begin{itemize}
        \item \textbf{Resource Allocation}. \hl{Allocating too few resources to specialized components like rasterization units can lead to bottlenecks}.
        \begin{examplebox}
            For example, if the rasterizer in a graphics pipeline is under-provisioned, the throughput of pixel fragments may not meet the demand of the SIMD cores, causing them to run at lower efficiency.
        \end{examplebox}

        \item \textbf{Efficiency Loss}. Under-provisioning can result in significant efficiency losses.
        \begin{examplebox}
            For example, if the rasterizer throughput is only 1T when 1.2T is needed, programmable cores may only operate at 80\% efficiency, leaving a substantial portion of the chip idle.
        \end{examplebox}
    \end{itemize}

    \newpage

    \item \important{\underline{Over}-Provisioning Components}:
    \begin{itemize}
        \item \textbf{Resource Waste}. Over-provisioning fixed-function components to avoid bottlenecks can lead to resource waste. Over-allocating chip area for components like the rasterizer minimizes the advantage of fixed-function units, as resources could have been used more effectively elsewhere.
        
        \item \textbf{Reduced Performance Gains}. Over-provisioning of fixed-function components can reduce the overall performance gains of the heterogeneous system because the added resources may not result in proportional improvements in throughput or efficiency.
    \end{itemize}
 
    \item \important{Complexity in Load Balancing}:
    \begin{itemize}
        \item \textbf{Dynamic Workloads}. Balancing workloads dynamically across different processing units is complex. \hl{Changes in workload intensity can lead to uneven distribution, causing some units to become overburdened while others remain underutilized}.

        \item \textbf{Scheduling Challenges}. \hl{Efficiently scheduling tasks to match the strengths of each processing unit requires sophisticated algorithms and can be challenging to implement correctly}. Poor scheduling can result in suboptimal performance and increased latency.
    \end{itemize}

    \item \important{Integration and Interoperability Issues}:
    \begin{itemize}
        \item \textbf{Interfacing Different Processors}. Ensuring seamless communication between heterogeneous processors can be challenging. Different processors have unique communication protocols, data formats, and memory access patterns that require careful design to avoid interoperability issues.

        \item \textbf{Data Consistency}. Maintaining data consistency across multiple processing units involves managing data transfer and synchronization efficiently. Inadequate handling of these aspects can lead to data corruption or delays.
    \end{itemize}

    \item \important{Programming and Development Complexity}:
    \begin{itemize}
        \item \textbf{Multiple Programming Models}. Developing applications for heterogeneous systems often involves using various programming models and tools. This increases development complexity and requires specialized knowledge from developers.

        \item \textbf{Debugging and Optimization}. Debugging and optimizing heterogeneous systems can be more challenging due to the interactions between different processing units. Identifying performance bottlenecks and resolving issues requires deep insights into the behavior of each processor type.
    \end{itemize}
\end{itemize}