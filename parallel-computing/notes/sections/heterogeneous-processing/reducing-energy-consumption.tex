\subsection{Reducing energy consumption}

There are two main idea to reducing energy consumption in heterogeneous computing systems: \textbf{Specialized Processing} and \textbf{Minimizing Data Movement}.

\highspace
\begin{flushleft}
    \textcolor{Green3}{\faIcon{book} \textbf{Specialized Processing}}
\end{flushleft}
\definition{Specialized Processing} involves \textbf{using the most energy-efficient processor for a specific task}. This idea leverages the unique strengths of different processor types (CPUs, GPUs, FPGAs, DSPs, etc.) to optimize performance while minimizing energy consumption.

\highspace
A possible implementation might be:
\begin{itemize}
    \item Assign workloads to processors best suited for them.
    \begin{examplebox}
        For example, use:
        \begin{itemize}
            \item CPUs for general-purpose, sequential tasks.
            \item GPUs for parallel processing tasks like rendering and simulations.
            \item FPGAs for tasks requiring reconfigurable logic and custom hardware acceleration.
            \item DSPs for signal processing tasks.
        \end{itemize}
    \end{examplebox}

    \item Deploy fixed-function units for repetitive tasks such as audio processing, video decoding/encoding, and image processing. These units consume less power compared to general-purpose processors.
\end{itemize}
The main advantages are:
\begin{itemize}[label=\textcolor{Green3}{\faIcon{check}}]
    \item Using the right processor for specific tasks \textbf{reduces the overall energy consumption}.
    \item Specialized processors can perform tasks more efficiently, leading to \textbf{faster execution times} and lower power usage.
\end{itemize}

\newpage

\begin{flushleft}
    \textcolor{Green3}{\faIcon{book} \textbf{Minimizing Data Movement}}
\end{flushleft}
\definition{Data Movement} has a high energy cost. By \textbf{reducing the frequency and distance over which data is transferred}, significant energy savings can be achieved.

\highspace
A possible implementation might be:
\begin{itemize}
    \item \textbf{Minimize the amount of data moved} between memory and processors.
    \item \textbf{Structure algorithms to reuse data stored in local memory} as much as possible, reducing accesses to external memory.
    \item \textbf{Implement compression techniques} to reduce the volume of data transferred. This can involve using specialized hardware for efficient compression and decompression.
\end{itemize}
The main advantages are:
\begin{itemize}[label=\textcolor{Green3}{\faIcon{check}}]
    \item Keeping data transfers to a minimum \textbf{reduces the energy required for memory access}.
    \item Reduced data movement leads to \textbf{lower latency} and \textbf{faster processing times}.
\end{itemize}

\begin{examplebox}
    Reading 64 bits of data from local SRAM (1mm away on-chip) consumes about 26 picoJoules, whereas reading the same data from low-power mobile DRAM (LPDDR) consumes around 1200 picoJoules. Thus, minimizing data movement to external memory significantly saves energy (26 pJ vs 1200 pJ).
\end{examplebox}

\newpage

\begin{flushleft}
    \textcolor{Green3}{\faIcon{star} \textbf{Trends in Energy-Optimized Computing}}
\end{flushleft}
The following trends collectively \textbf{aim to address the pressing need for more sustainable computing solutions}. As the demand for computing power continues to grow, optimizing energy consumption helps to reduce environmental impact and ensure that technological advances do not come at the expense of the planet.
\begin{enumerate}
    \item \important{Compute Less!} Each computation costs energy, so by minimizing unnecessary operations, the energy efficiency of the system increases.

    The parallel algorithms may perform more work than sequential ones, which sometimes makes them less desirable despite potentially faster execution times. The goal is to balance the speed and energy consumption.

    \item \important{Specialize Compute Units}. This involves using specialized processing units tailored to specific tasks to reduce energy consumption:
    \begin{itemize}
        \item \textbf{Heterogeneous Processors}: Combining CPU-like cores (general-purpose) with GPU-like cores (optimized for parallel tasks) to use the best processor for each task.
        \item \textbf{Fixed-Function Units}: Utilizing units designed for specific tasks such as audio processing, video encoding/decoding, and image processing. These units consume less energy compared to general purpose processors for their respective tasks.
        \item \textbf{Specialized Instructions}: Incorporating specialized instructions like AVX for vector processing and AES-NI for encryption to perform tasks more efficiently.
        \item \textbf{Programmable Soft Logic (FPGAs)}: Using FPGAs, which can be reprogrammed for different tasks, providing flexibility and energy efficiency.
    \end{itemize}
    
    \item \important{Reduce Bandwidth Requirements}. This trend focuses on minimizing data transfer to save energy.

    Restructuring algorithms to maximize reuse of on-chip data, thereby reducing the need to transfer data from memory. Also, using additional computation to compress data before it is transferred to memory, reducing the amount of data moved. This approach helps reduce the power costs associated with data movement. For example, using fixed-function hardware to reduce the overhead of general-purpose data compression and decompression.
\end{enumerate}
