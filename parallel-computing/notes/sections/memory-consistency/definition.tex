\subsection{Definition}

After a brief discussion of the difference between coherence and consistency terms, and why it is important to understand these concepts, here we present a deeper view of memory consistency. This is because consistency deals with the broader and more complex issue of how all memory operations (across all memory locations) are ordered and observed in a parallel system. It is also necessary to understand the high-level rules that govern the interactions between processors and memory in a multiprocessor environment.

\begin{definitionbox}[: Memory Consistency]
    \definition{Memory Consistency} models define \textbf{how memory operations} (loads and stores) \textbf{performed by different processors are \underline{ordered} and become visible to one another in a multiprocessor system}.
\end{definitionbox}

\highspace
\begin{flushleft}
    \textcolor{Green3}{\faIcon{question-circle} \textbf{Why the order decided by the memory consistency model is important}}
\end{flushleft}
One main reason:
\begin{itemize}
    \item \textbf{Performance}. In multiprocessor systems, memory operations can be reordered to optimize performance. Unfortunately, this \textbf{reordering can result in behavior that seems counterintuitive or unusual from the perspective of a programmer} who expects sequential execution. However, it allows optimizations such as overlapping memory accesses with computations and reducing memory access latency.
\end{itemize}
Most application programmers don't have to deal directly with the effects of memory reordering, because higher-level constructs and synchronization mechanisms handle them. However, understanding memory consistency is critical to writing correct and efficient parallel programs. Developers of system software, such as operating systems and compilers, must deal with these issues to ensure that their low-level code conforms to the hardware's memory consistency model and maintains the correct order of memory operations.

\newpage

\begin{flushleft}
    \textcolor{Green3}{\faIcon{sort-amount-up-alt} \textbf{Memory Operation Ordering}}
\end{flushleft}
A program defines a sequence of loads and saves. There are \textbf{four types of memory operation sequences}:
\begin{enumerate}
    \item \important{$W_{X} \rightarrow R_{Y}$}: \textbf{Write to $X$ must commit before a subsequent read from $Y$}.

    This means that if a write to memory location $X$ occurs before a read from another memory location $Y$ in the program order, the write must be complete and visible before the read.


    \item \important{$R_{X} \rightarrow R_{Y}$}: \textbf{Read from $X$ must commit before a subsequent read from $Y$}.
   
    This order ensures that if a read from $X$ occurs before a read from $Y$ in the program, the first read must be completed before the second read occurs.


    \item \important{$R_{X} \rightarrow W_{Y}$}: \textbf{Read from $X$ must commit before a subsequent write to $Y$}.
   
    This means that if a read from $X$ occurs before a write to $Y$ in the program order, the read must be completed before the write is performed.


    \item \important{$W_{X} \rightarrow W_{Y}$}: \textbf{Write to $X$ must commit before a subsequent write to $Y$}.
   
    This ordering ensures that if a write to $X$ happens before a write to $Y$ in the program, the first write must be completed before the second write is performed.
\end{enumerate}
The word \dquotes{subsequent} means that each left operation must be completed and its result visible before the right operation can occur.
