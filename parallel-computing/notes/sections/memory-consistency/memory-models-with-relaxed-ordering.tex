\subsection{Memory Models with Relaxed Ordering}

The following models are \textbf{grouped based on the types of memory operation reordering} they allow. Each model applies the relaxing memory models, so they allow certain memory operation orders to be violated to improve performance.

\longline

\subsubsection{Allowing Reads to Move Ahead of Writes}

\begin{flushleft}
    \textcolor{Green3}{\faIcon{question-circle} \textbf{Which memory operation is relaxed?}}
\end{flushleft}
Given the four memory operation sequences (see detailed explanation on page \hqpageref{flushleft: Memory Operation Ordering}), the relaxed operations are marked as cancel:
\begin{itemize}
    \item[\textcolor{Green3}{\faIcon{check}}] \textcolor{Green3}{\textbf{\emph{Relaxed}} $\cancel{W_{X} \rightarrow R_{Y}}$}: \textbf{Write to $X$ must commit before a subsequent read from $Y$}.

    This means that if a write to memory location $X$ occurs before a read from another memory location $Y$ in the program order, the write must be complete and visible before the read.


    \item[\textcolor{Red2}{\faIcon{times}}] \important{$R_{X} \rightarrow R_{Y}$}: Read from $X$ must commit before a subsequent read from $Y$.
    \item[\textcolor{Red2}{\faIcon{times}}] \important{$R_{X} \rightarrow W_{Y}$}: Read from $X$ must commit before a subsequent write to $Y$.
    \item[\textcolor{Red2}{\faIcon{times}}] \important{$W_{X} \rightarrow W_{Y}$}: Write to $X$ must commit before a subsequent write to $Y$.
\end{itemize}
\begin{figure}[!htp]
    \centering
    \includegraphics[width=.4\textwidth]{img/tso-and-pc-1.pdf}
\end{figure}

\newpage

\begin{flushleft}
    \textcolor{Green3}{\faIcon{book} \textbf{Models}}
\end{flushleft}
\begin{itemize}
    \item \hqlabel{definition: Total Store Ordering (TSO)}{\definition{Total Store Ordering (TSO)}}
    \begin{flushleft}
        \textcolor{Green3}{\faIcon{tools} \textbf{How does it work?}}
    \end{flushleft}
    It allows a processor to read a variable (e.g., \texttt{B}) before its write to another variable (e.g., \texttt{A}) is visible to all processors ($\cancel{W_{\texttt{A}} \rightarrow R_{\texttt{B}}}$).

    This enables the processor to hide the latency of write operations by \textbf{allowing reads to proceed independently}.

    \begin{flushleft}
        \textcolor{Red2}{\faIcon{exclamation-triangle} \textbf{Implications}}
    \end{flushleft}
    Other processors cannot see the new value of \texttt{A} (the value written to register \texttt{A}) until the write to \texttt{A} is observed by all processors.

    Only $W_{\texttt{A}} \rightarrow R_{\texttt{B}}$ order is relaxed, while $W_{\texttt{A}} \rightarrow W_{\texttt{B}}$ constraints still exist, meaning writes by the same thread occur in program order. Other processors see these writes in the correct order, but there might be a delay before the writes are visible due to buffering.

    \item \hqlabel{definition: Processor Consistency (PC)}{\definition{Processor Consistency (PC)}}
    \begin{flushleft}
        \textcolor{Green3}{\faIcon{tools} \textbf{How does it work?}}
    \end{flushleft}
    Similar to TSO, but with slightly \textbf{more flexibility}.

    Any processor can read the new value of \texttt{A} \textbf{before the write is observed by all processors}, allowing reads to be even \textbf{more independent} of writes.

    \begin{flushleft}
        \textcolor{Red2}{\faIcon{exclamation-triangle} \textbf{Implications}}
    \end{flushleft}
    Like TSO, only $W_{\texttt{A}} \rightarrow R_{\texttt{B}}$ order is relaxed, and $W_{\texttt{A}} \rightarrow W_{\texttt{B}}$ constraints remain, ensuring that writes by the same thread occur in program order.
\end{itemize}

\highspace
\begin{flushleft}
    \textcolor{Green3}{\faIcon{balance-scale} \textbf{Main differences between TSO and PC}}
\end{flushleft}
\begin{enumerate}
    \item \important{Read Flexibility}
    \begin{itemize}
        \item \textbf{TSO}: Reads by a processor can move ahead of its own writes.
        \item \textbf{PC}: Reads by any processor can observe new values of writes \textbf{\emph{before they are globally visible}}, allowing even more flexibility.
    \end{itemize}

    \item \important{Write Order}
    \begin{itemize}
        \item \textbf{Both Models}: Maintain the order of writes within the same processor, ensuring that $W_{\texttt{A}} \rightarrow W_{\texttt{B}}$ order is preserved.
    \end{itemize}

    \item \important{Performance vs. Complexity}
    \begin{itemize}
        \item \textbf{TSO}: Strikes a \textbf{\emph{balance between performance and simplicity}} by allowing limited reordering of reads and writes.
        \item \textbf{PC}: Offers \emph{\textbf{greater flexibility in read operations}}, potentially improving performance but at the \textbf{\emph{cost of slightly increased complexity}}.
    \end{itemize}
\end{enumerate}

\newpage

\subsubsection{Allowing writes to be reordered}

\begin{flushleft}
    \textcolor{Green3}{\faIcon{question-circle} \textbf{Which memory operation is relaxed?}}
\end{flushleft}
Given the four memory operation sequences (see detailed explanation on page \hqpageref{flushleft: Memory Operation Ordering}), the relaxed operations are marked as cancel:
\begin{itemize}
    \item[\textcolor{Green3}{\faIcon{check}}] \textcolor{Green3}{\textbf{\emph{Relaxed}} $\cancel{W_{X} \rightarrow R_{Y}}$}: \textbf{Write to $X$ must commit before a subsequent read from $Y$}.

    This means that if a write to memory location $X$ occurs before a read from another memory location $Y$ in the program order, the write must be complete and visible before the read.


    \item[\textcolor{Red2}{\faIcon{times}}] \important{$R_{X} \rightarrow R_{Y}$}: Read from $X$ must commit before a subsequent read from $Y$.
    \item[\textcolor{Red2}{\faIcon{times}}] \important{$R_{X} \rightarrow W_{Y}$}: Read from $X$ must commit before a subsequent write to $Y$.
    \item[\textcolor{Green3}{\faIcon{check}}] \textcolor{Green3}{\textbf{\emph{Relaxed}} $\cancel{W_{X} \rightarrow W_{Y}}$}: \textbf{Write to $X$ must commit before a subsequent write to $Y$}.

    This ordering ensures that if a write to $X$ happens before a write to $Y$ in the program, the first write must be completed before the second write is performed.
\end{itemize}

\highspace
\begin{flushleft}
    \textcolor{Green3}{\faIcon{book} \textbf{Models}}
\end{flushleft}
\begin{itemize}
    \item \definition{Partial Store Ordering (PSO)}
    \begin{flushleft}
        \textcolor{Green3}{\faIcon{tools} \textbf{How does it work?}}
    \end{flushleft}
    PSO allows more \textbf{aggressive reordering of write operations} compared to TSO (page \hqpageref{definition: Total Store Ordering (TSO)}) and PC (page \hqpageref{definition: Processor Consistency (PC)}).

    \begin{flushleft}
        \textcolor{Red2}{\faIcon{exclamation-triangle} \textbf{Implications}}
    \end{flushleft}
    \begin{itemize}
        \item Thread 1 on Processor 1 (\texttt{P1})
        \begin{lstlisting}
A = 1;
flag = 1;\end{lstlisting}

        \item Thread 2 on Processor 2 (\texttt{P2})
        \begin{lstlisting}
while (flag == 0); 
print A;\end{lstlisting}
    \end{itemize}

    \texttt{P2} may observe the change to \texttt{flag} before the change to \texttt{A}. This shows that PSO allows writes to \texttt{A} and \texttt{flag} to be reordered, improving write performance but potentially complicating program reasoning.

    \newpage

    \begin{flushleft}
        \textcolor{Green3}{\faIcon{check-circle} \textbf{Benefits}}
    \end{flushleft}
    \begin{itemize}
        \item $W_{X} \rightarrow W_{Y}$. \textcolor{Green3}{\textbf{Write Buffering}}: \textbf{Processors can reorder write operations in a write buffer}. For example, one write might be a cache miss while another is a cache hit, and reordering them can optimize performance.
        
        \item  $R_{X} \rightarrow R_{Y}$, $R_{X} \rightarrow W_{Y}$. \textcolor{Green3}{\textbf{Instruction Reordering}}: \textbf{Processors can reorder independent read and write instructions within the instruction stream}, leveraging out-of-order execution to maximize efficiency.
    \end{itemize}
    These reorderings and optimizations are particularly \textbf{effective in single instruction streams}, where dependencies between instructions are well understood and managed.
\end{itemize}

\newpage

\subsubsection{Allowing all reorderings}

\begin{flushleft}
    \textcolor{Green3}{\faIcon{question-circle} \textbf{Which memory operation is relaxed?}}
\end{flushleft}
Given the four memory operation sequences (see detailed explanation on page \hqpageref{flushleft: Memory Operation Ordering}), the relaxed operations are marked as cancel:
\begin{itemize}
    \item[\textcolor{Green3}{\faIcon{check}}] \textcolor{Green3}{\textbf{\emph{Relaxed}} $\cancel{W_{X} \rightarrow R_{Y}}$}: \textbf{Write to $X$ must commit before a subsequent read from $Y$}.

    This means that if a write to memory location $X$ occurs before a read from another memory location $Y$ in the program order, the write must be complete and visible before the read.


    \item[\textcolor{Green3}{\faIcon{check}}] \textcolor{Green3}{\textbf{\emph{Relaxed}} $\cancel{R_{X} \rightarrow R_{Y}}$}: \textbf{Read from $X$ must commit before a subsequent read from $Y$}.

    This order ensures that if a read from $X$ occurs before a read from $Y$ in the program, the first read must be completed before the second read occurs.


    \item[\textcolor{Green3}{\faIcon{check}}] \textcolor{Green3}{\textbf{\emph{Relaxed}} $\cancel{R_{X} \rightarrow W_{Y}}$}: \textbf{Read from $X$ must commit before a subsequent write to $Y$}.

    This means that if a read from $X$ occurs before a write to $Y$ in the program order, the read must be completed before the write is performed.


    \item[\textcolor{Green3}{\faIcon{check}}] \textcolor{Green3}{\textbf{\emph{Relaxed}} $\cancel{W_{X} \rightarrow W_{Y}}$}: \textbf{Write to $X$ must commit before a subsequent write to $Y$}.

    This ordering ensures that if a write to $X$ happens before a write to $Y$ in the program, the first write must be completed before the second write is performed.
\end{itemize}

\highspace
\begin{flushleft}
    \textcolor{Green3}{\faIcon{book} \textbf{Models}}
\end{flushleft}
\begin{itemize}
    \item \definition{Weak Ordering (WO)}
    \begin{flushleft}
        \textcolor{Green3}{\faIcon{tools} \textbf{How does it work?}}
    \end{flushleft}
    Operations are divided into critical and non-critical sections. Reordering is allowed outside of critical sections to maximize performance.

    \example{Example}: Can reorder any non-critical operations to enhance efficiency, executing them independently.


    \item \definition{Release Consistency (RC)}
    \begin{flushleft}
        \textcolor{Green3}{\faIcon{tools} \textbf{How does it work?}}
    \end{flushleft}
    Divides operations into acquire and release categories, allowing extensive reordering within these categories.

    \example{Example}: Reorders operations within acquire or release phases, enabling significant performance gains.
\end{itemize}
In these models, there are \textbf{no strict guarantees about the order of operations on data}. Essentially, everything can be reordered.

\begin{flushleft}
    \textcolor{Green3}{\faIcon{question-circle} \textbf{Why allow all reorders?}}
\end{flushleft}
By overlapping multiple reads and writes, the \textbf{system can execute reads as early as possible and delay writes as late as possible}, effectively \textbf{hiding memory latency} and \textbf{increasing performance}.