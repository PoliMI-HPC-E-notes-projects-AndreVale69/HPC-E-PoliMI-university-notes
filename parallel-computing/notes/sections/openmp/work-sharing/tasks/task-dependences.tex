\paragraph{Task dependences}

By creating tasks instead of sections, each thread can execute any task as long as its input is ready. Since the internal scheduler decides how to manipulate the execution of tasks, there is a useful \textbf{clause to indicate a dependency between tasks}.
\marginpar{
    \href{https://www.openmp.org/spec-html/5.0/openmpsu99.html} {Doc. \faIcon{book}}
}
\begin{openmpbox}[: \texttt{depend}]
    \begin{lstlisting}[language=C++, mathescape=true]
#pragma omp task depend($\emph{dependence-type}$: variable)\end{lstlisting}
    The \texttt{depend} clause enforces additional constraints on the scheduling of tasks or loop iterations. These constraints establish dependencies only between sibling tasks or between loop iterations.

    The \emph{dependence-type} can be \texttt{in} or \texttt{out}.
\end{openmpbox}

\noindent
In addition to the depend clause, we can also use the priority clause to hint that more important tasks should be executed more frequently.

\highspace
The \texttt{priority} clause is a \textbf{hint for the priority of the generated task}. The priority-value is a \textbf{non-negative integer expression} that provides a hint for task execution order. Among all tasks ready to be executed, \textbf{higher priority tasks} (those with a higher numerical value in the priority clause expression) \textbf{are recommended to execute before lower priority ones}. The \textbf{default} priority-value when no priority clause is specified \textbf{is zero} (the \emph{lowest priority}).

\highspace
\textbf{If a value is specified in the priority clause that is higher than the \emph{max-task-priority-var}} ICV (internal control variables) then the implementation will \textbf{use the value of that ICV}. A program that relies on task execution order being determined by this priority-value may have unspecified behavior.
\begin{itemize}
    \item The \texttt{omp\_get\_max\_task\_priority} routine returns the \textbf{maximum value that can be specified in the \texttt{priority} clause}.
    \marginpar{
        \href{https://www.openmp.org/spec-html/5.0/openmpsu151.html\#x188-8880003.2.42} {Doc. \faIcon{book}}
    }
    \begin{openmpbox}[: \texttt{omp\_get\_max\_task\_priority}]
        \begin{lstlisting}[language=C++]
int omp_get_max_task_priority()\end{lstlisting}
    \end{openmpbox}

    \item The \texttt{OMP\_MAX\_TASK\_PRIORITY} environment variable \textbf{controls the use of task priorities} by setting the initial value of the \emph{max-task-priority-var} ICV. The value of this environment variable must be a non-negative integer.\marginpar{
        \href{https://www.openmp.org/spec-html/5.0/openmpse64.html\#x303-20810006.16} {Doc. \faIcon{book}}
    }
\end{itemize}