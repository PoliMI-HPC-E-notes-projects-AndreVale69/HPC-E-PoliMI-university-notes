\section{Parallel Patterns in OpenMP and CUDA}

\subsection{OpenMP}

OpenMP provides efficient implementations of many parallel programming patterns. Here are some of the key patterns that we have already discussed:
\begin{enumerate}
    \item \definition{Map Pattern on OpenMP}
    \begin{flushleft}
        \textcolor{Green3}{\faIcon{info-circle} \textbf{Characteristics}}
    \end{flushleft}
    \begin{itemize}
        \item \textbf{No dependencies} between operations.
        \item Simple and \textbf{ideal for SIMD} (Single Instruction, Multiple Data) execution.
        \item Executes independently on different data elements.
    \end{itemize}
    \begin{examplebox}[: Map Pattern on OpenMP]
        \begin{lstlisting}[language=c++]
#pragma omp parallel for
for (int i = 0; i < N; i++) {
    result[i] = operation(data[i]);
}\end{lstlisting}
    \end{examplebox}
    
    \item \definition{Reduction Pattern on OpenMP}
    \begin{flushleft}
        \textcolor{Green3}{\faIcon{info-circle} \textbf{Characteristics}}
    \end{flushleft}
    \begin{itemize}
        \item \textbf{Combines results from multiple threads into a single value} using a specified operation (e.g., sum, min, max).
        \item \textbf{Supported natively} in OpenMP (see \ref{paragraph: Reduction}, page \pageref{paragraph: Reduction}). It is an easy integration for operations that aggregate data across threads.
    \end{itemize}
    \begin{examplebox}[: Reduction Pattern on OpenMP]
        \begin{lstlisting}[language=c++]
#pragma omp parallel for reduction(+:sum)
for (int i = 0; i < N; i++) {
    sum += array[i];
}\end{lstlisting}
    \end{examplebox}
    
    \item \definition{Workpile Pattern on OpenMP}
    \begin{flushleft}
        \textcolor{Green3}{\faIcon{info-circle} \textbf{Characteristics}}
    \end{flushleft}
    \begin{itemize}
        \item \textbf{Handles irregular work distribution that may change dynamically} at runtime.
        \item Assumes \textbf{all tasks are independent}.
        \item Commonly used in applications like \textbf{tree searches} or \textbf{recursive computations}.
        \item Leverages OpenMP's tasking feature for dynamically generated workloads.
    \end{itemize}
    \begin{examplebox}[: Workpile Pattern on OpenMP]
        \begin{lstlisting}[language=c++]
#pragma omp task
void process_task() {
    // Perform a unit of work
}\end{lstlisting}
    \end{examplebox}

    \item \definition{Scan Pattern on OpenMP}
    \begin{flushleft}
        \textcolor{Green3}{\faIcon{info-circle} \textbf{Characteristics}}
    \end{flushleft}
    \begin{itemize}
        \item \textbf{Efficiently performs prefix-sum} or similar operations over an array.
        \item Widely \textbf{used for cumulative computations} and \textbf{inclusive/exclusive scans}.
        \item OpenMP 5.0 introduced \textbf{dedicated support} for this pattern.
        \item Three-Phase Approach:
        \begin{enumerate}
            \item \textbf{Build intermediate results} in parallel.
            \item \textbf{Combine intermediate results} in pairs.
            \item \textbf{Build final output} in parallel.
        \end{enumerate}
        \begin{figure}[!htp]
            \centering
            \includegraphics[width=.9\textwidth]{img/scan-pattern-2.pdf}
        \end{figure}
    \end{itemize}
    \begin{examplebox}[: Scan Pattern on OpenMP]
        Example with a SIMD reduction clause:
        \begin{lstlisting}[language=c++]
#pragma omp simd reduction(inscan, +:scan_a)
for (int i = 0; i < N; i++) {
    simd_scan[i] = scan_a; 
    #pragma omp scan exclusive(scan_a)
    scan_a += array[i];
}\end{lstlisting}
    \end{examplebox}
\end{enumerate}
