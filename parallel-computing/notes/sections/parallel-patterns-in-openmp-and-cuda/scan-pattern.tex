\subsection{Scan Pattern}

The scan pattern, also known as prefix sum, is a computational pattern where \textbf{each element in an output array is the sum of all previous elements up to the current index in the input array}.

\highspace
\begin{flushleft}
    \textcolor{Green3}{\faIcon{walking} \textbf{Sequential Implementation}}
\end{flushleft}
\begin{lstlisting}[language=c++]
y[0] = x[0];
for (int i = 1; i < Max_i; ++i)
    y[i] = y[i-1] + x[i];\end{lstlisting}
The complexity is $O(N)$ for $N$ elements. The \textcolor{Green3}{pros} are \hl{computationally efficient} and \hl{only one additional operation} compared to reduction. Unfortunately, the \textcolor{Red2}{limitation} is the \hl{sequential} execution; cannot fully utilize parallel resources.

\highspace
\begin{flushleft}
    \textcolor{Green3}{\faIcon{book} \textbf{Parallel Implementation with Interleaved Reduction Trees}}
\end{flushleft}
\begin{enumerate}
    \item \important{Input Reading}: Copy the input into shared memory.
    \item \important{Iterative Computation}: In each iteration, thread \texttt{j} adds elements \texttt{j} and \texttt{j - stride}, storing the result back in \texttt{j}. Finally, double the stride and repeat until the entire array is processed.
\end{enumerate}

\highspace
\begin{flushleft}
    \textcolor{Red2}{\faIcon{walking} \textbf{Work-inefficient CUDA implementation}}
\end{flushleft}
\begin{lstlisting}[language=c++]
__shared__ float XY[SECTION_SIZE];
int i = blockIdx.x * blockDim.x + threadIdx.x;

if (i < InputSize) {
    XY[threadIdx.x] = X[i];
}

// perform iterative scan on XY
for (
    unsigned int stride = 1;
    stride < blockDim.x;
    stride *= 2
) {
    __syncthreads();
    if (stride <= threadIdx.x) {
        XY[threadIdx.x] += XY[threadIdx.x - stride];
    }
}
\end{lstlisting}
The previous CUDA implementation is inefficient due to several factors:
\begin{enumerate}[label=\textcolor{Red2}{\faIcon{times}}]
    \item \textcolor{Red2}{\textbf{Work Inefficiency}}
    \begin{itemize}
        \item The implementation requires $\log_{2}(N)$ iterations for a data size of $N$, where each iteration involves progressively fewer additions (e.g., $(N-1)$, $(N-2)$, etc.).
        \item The total number of additions scales as $O(N \cdot \log_{2}(N))$, which is significantly higher than the sequential version's $O(N)$.
    \end{itemize}
    For example, with $N=1024$, this \hl{results in a 10x computational overhead} compared to the sequential version, as the hardware resources are not used optimally.

    \item \textcolor{Red2}{\textbf{High Energy Consumption}}. Due to the inefficiency of the work distribution, \hl{many hardware resources remain inactive during computation}. This leads to energy wastage, as resources that could be utilized for other tasks are idling.

    \item \textcolor{Red2}{\textbf{Shared Memory and Synchronization Overhead}}. The implementation relies on shared memory to store intermediate results for each block. \hl{Synchronization barriers} (\texttt{\_\_syncthreads()}) are \hl{required at each iteration} to ensure threads work correctly on shared data. This adds latency and further reduces efficiency.

    \item \textcolor{Red2}{\textbf{Unbalanced Load Distribution}}. As the number of elements to process reduces with each iteration, \hl{some threads become idle earlier than others}. This \hl{unbalanced workload} distribution leads to under-utilization of the GPU's parallel processing capabilities.

    \item \textcolor{Red2}{\textbf{Lack of Intermediate Result Reuse}}. \hl{Intermediate sums computed during the scan are not reused effectively in subsequent steps}. This leads to redundant computations, further increasing inefficiency.
\end{enumerate}

\newpage

\begin{flushleft}
    \textcolor{Green3}{\faIcon{\speedIcon} \textbf{Efficient CUDA implementation}}
\end{flushleft}
The improved CUDA implementation of the scan pattern involves two phases: \textbf{Reduction} and \textbf{Distribution}, both of which are designed to ensure work efficiency and minimize execution time.
\begin{itemize}
    \item \important{Reduction Phase}
    \begin{itemize}
        \item[\textcolor{Green3}{\faIcon{book}}] \textcolor{Green3}{\textbf{Goal}}. \textbf{Build} a prefix sum (or partial reduction) \textbf{tree in shared memory}.

        \item[\textcolor{Green3}{\faIcon{lightbulb}}] \textcolor{Green3}{\textbf{Key Idea}}. \textbf{Each thread works on shared memory}:
        \begin{equation*}
            \texttt{XY[2*BLOCK\_SIZE]}
        \end{equation*}
        And \textbf{accumulates results in a binary tree} fashion.

        \item[\textcolor{Green3}{\faIcon{tools}}] \textcolor{Green3}{\textbf{Process}}
        \begin{enumerate}
            \item The threads perform summation for strides increasing as powers of 2.
            
            For example, for \texttt{stride = 1}, thread at index \texttt{1} adds values from index \texttt{0}; for \texttt{stride = 2}, thread at index \texttt{3} adds values from index \texttt{1}, and so on.
            \item This builds a binary tree structure in memory, with the final prefix sum stored at the highest index.
        \end{enumerate}

        \item[\textcolor{Green3}{\faIcon{clock}}] \textcolor{Green3}{\textbf{Synchronization}}. The \texttt{\_\_syncthreads()} call \textbf{ensures all threads finish their stride operation} before the next begins.

        \item[\textcolor{Green3}{\faIcon{check-circle}}] \textcolor{Green3}{\textbf{Efficiency}}. It executes $\log_{2}\left(N\right)$ iterations (with $N$ being the input size). Total operations are $N - 1$.
    \end{itemize}

    \item \important{Distribution Phase}
    \begin{itemize}
        \item[\textcolor{Green3}{\faIcon{book}}] \textcolor{Green3}{\textbf{Goal}}. \textbf{Transform the prefix sum tree into the final scan result array}.

        \item[\textcolor{Green3}{\faIcon{lightbulb}}] \textcolor{Green3}{\textbf{Key Idea}}. \textbf{Threads traverse the binary tree} in reverse (top-down) to \textbf{propagate results} downwards.

        \item[\textcolor{Green3}{\faIcon{tools}}] \textcolor{Green3}{\textbf{Process}}
        \begin{enumerate}
            \item Strides start at half the block size (\texttt{BLOCK\_SIZE / 2}) and decrease by powers of 2.
            \item Each thread adds values from higher indices to lower indices in the tree to compute the final scan result.
            \item The results are written to the output array (\texttt{Y[i]}).
        \end{enumerate}

        \item[\textcolor{Green3}{\faIcon{clock}}] \textcolor{Green3}{\textbf{Synchronization}}. Threads use \texttt{\_\_syncthreads()} before and after each stride to ensure all updates are visible across threads.

        \item[\textcolor{Green3}{\faIcon{check-circle}}] \textcolor{Green3}{\textbf{Efficiency}}. It executes $\log_{2}\left(N\right)$ iterations with a reduced number of operations. Total operations are no more than $\left(N-1\right)$ in each phase.
    \end{itemize}
\end{itemize}

\newpage

\noindent
The reason why it is improved are:
\begin{itemize}[label=\textcolor{Green3}{\faIcon{check}}]
    \item \textcolor{Green3}{\textbf{Work Efficiency}}. Both \textbf{phases together execute a maximum of $2(N-1)$ operations}. This is much better compared to the naïve implementation, which can perform up to $O(N \cdot \log(N))$ operations.

    \item \textcolor{Green3}{\textbf{Parallelization}}. \textbf{Each thread handles fewer redundant operations}, and \textbf{shared memory minimizes memory latency}.

    \item \textcolor{Green3}{\textbf{Hardware Utilization}}. With enough CUDA cores and shared memory, the \textbf{overhead of extra work is compensated by reduced execution time}.
\end{itemize}
\begin{lstlisting}[language=c++]
/*******************
 * Reduction Phase *
 *******************/
// XY[2 * BLOCK_SIZE] is in shared memory

for (
    unsigned int stride = 1;
    stride <= BLOCK_SIZE;
    stride *= 2
) {
    int index = (threadIdx.x + 1) * stride * 2 - 1;
    if (index < 2 * BLOCK_SIZE) {
        XY[index] += XY[index - stride];
    }
    __syncthreads();
}

/**********************
 * Distribution Phase *
 **********************/
for (
    unsigned int stride = BLOCK_SIZE / 2;
    stride > 0;
    stride /= 2
) {
    __syncthreads();
    int index = (threadIdx.x + 1) * stride * 2 - 1;
    if (index + stride < 2 * BLOCK_SIZE) {
        XY[index + stride] += XY[index];
    }
}
__syncthreads();
if (i < InputSize) {
    Y[i] = XY[threadIdx.x];
}\end{lstlisting}