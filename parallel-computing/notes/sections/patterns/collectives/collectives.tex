\subsection{Collectives operations}

\definition{Collective operations} are fundamental to parallel computing. They \textbf{involve communication patterns that enable data to be distributed among multiple processes, combined, or transformed in various ways}. 

\highspace
In essence, \hl{collectives are operations where all processes in a parallel program participate}.

\highspace
\begin{flushleft}
    \textcolor{Green3}{\faIcon{stream} \textbf{Collective Patterns}}
\end{flushleft}
\begin{itemize}
    \item \important{Reduce}: Combines data from all processes into a single result.

    \item \important{Scan}: Computes cumulative operations (such as prefix sum) across processes.

    \item \important{Partition}: Divides data into subsets and distributes them among processes.

    \item \important{Scatter}: Distributes distinct pieces of data from one process to all other processes.

    \item \important{Gather}: Collects distinct pieces of data from all processes into one process.
\end{itemize}
Collectives are used in parallel algorithms to \textbf{ensure efficient and effective data handling across multiple processes}. They help in achieving tasks such as data aggregation, distribution, synchronization, and more.