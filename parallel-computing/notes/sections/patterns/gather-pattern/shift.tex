\subsubsection{Shift}

The \definition{Shift operation} is a specialized case of the gather pattern where \textbf{data is moved uniformly left or right in memory}. Unlike random gathers (already seen in section \ref{subsubsection: What is a Gather}, page \pageref{subsubsection: What is a Gather}), shifts involve \textbf{regular}, \textbf{predictable data movement} with \underline{fixed offsets}.

\highspace
\begin{flushleft}
    \textcolor{Green3}{\faIcon{star} \textbf{Key Features}}
\end{flushleft}
\begin{itemize}
    \item \important{Data Movement}. Elements are \hl{shifted by a fixed distance}.
    \item \important{Offset Access}. Data accesses are offset uniformly, making the operation predictable. 
\end{itemize}

\highspace
\begin{flushleft}
    \textcolor{Red2}{\faIcon{exclamation-triangle} \textbf{Boundary Conditions}}
\end{flushleft}
With the shift operation, the main problem we can encounter is mainly one: boundary conditions.

\highspace
When shifting data, \textbf{handling the edge elements} (out-of-bounds data) \textbf{is crucial}. There are several strategies:
\begin{itemize}[label=\textcolor{Green3}{\faIcon{check}}]
    \item \textcolor{Green3}{\textbf{Default Value}}. Fill the vacant position with a default (e.g., \texttt{0} or \texttt{null}).
    \begin{examplebox}[: Default Boundary Condition]
        Input array:
        \begin{equation*}
            \texttt{A = [1, 2, 3, 4, 5, 6, 7, 8, 9, 10]}
        \end{equation*}
        Shift with default boundary condition:
        \begin{itemize}
            \item Left shift:
            \begin{equation*}
                \texttt{A = [2, 3, 4, 5, 6, 7, 8, 9, 10, 0]}
            \end{equation*}
            \item Right shift:
            \begin{equation*}
                \texttt{A = [0, 1, 2, 3, 4, 5, 6, 7, 8, 9]}
            \end{equation*}
        \end{itemize}
    \end{examplebox}

    \item \textcolor{Green3}{\textbf{Duplicate}}. Replicate the edge element to fill the gap.
    \begin{examplebox}[: Duplicate Boundary Condition]
        Input array:
        \begin{equation*}
            \texttt{A = [1, 2, 3, 4, 5, 6, 7, 8, 9, 10]}
        \end{equation*}
        Shift with duplicate boundary condition:
        \begin{itemize}
            \item Left shift:
            \begin{equation*}
                \texttt{A = [2, 3, 4, 5, 6, 7, 8, 9, 10, 10]}
            \end{equation*}
            \item Right shift:
            \begin{equation*}
                \texttt{A = [1, 1, 2, 3, 4, 5, 6, 7, 8, 9]}
            \end{equation*}
        \end{itemize}
    \end{examplebox}

    \item \textcolor{Green3}{\textbf{Rotate}}. Wrap around the array, moving the last element to the first position (circular shift).
    \begin{examplebox}[: Rotate Boundary Condition]
        Input array:
        \begin{equation*}
            \texttt{A = [1, 2, 3, 4, 5, 6, 7, 8, 9, 10]}
        \end{equation*}
        Shift with rotate boundary condition:
        \begin{itemize}
            \item Left shift:
            \begin{equation*}
                \texttt{A = [2, 3, 4, 5, 6, 7, 8, 9, 10, 1]}
            \end{equation*}
            \item Right shift:
            \begin{equation*}
                \texttt{A = [10, 1, 2, 3, 4, 5, 6, 7, 8, 9]}
            \end{equation*}
        \end{itemize}
    \end{examplebox}
\end{itemize}

\highspace
\begin{flushleft}
    \textcolor{Green3}{\faIcon{tachometer-alt} \textbf{Advantages \& Benefits}}
\end{flushleft}
\begin{itemize}
    \item Shifts are highly efficient with \textbf{vector instructions} due to their regularity.
    \item They allow \textbf{parallel shifting} of multiple elements simultaneously.
    \item Good \textbf{data locality} enhances performance, as adjacent elements are accessed together.
\end{itemize}