\subsubsection{Unzip}

The \definition{Unzip operation} is essentially the \textbf{reverse of the zip operation}. While zip combines multiple arrays into one by interleaving their elements, \textbf{unzip separates a single interleaved array back into its original components}.

\highspace
\begin{flushleft}
    \textcolor{Green3}{\faIcon{tools} \textbf{How does it work?}}
\end{flushleft}
The operation \textbf{extracts sub-arrays at regular offsets} (strides) \textbf{to separate the elements into their original groups}. The input is a combined sequence (e.g., alternating real and imaginary parts).
\begin{examplebox}[: unzip operation]
    Consider the input array:
    \begin{equation*}
        \texttt{[Real0, Imag0, Real1, Imag1, Real2, Imag2, ...]}
    \end{equation*}
    The output are:
    \begin{itemize}
        \item Array of Real Parts: \texttt{[Real0, Real1, Real2]}
        \item Array of Imaginary Parts: \texttt{[Imag0, Imag1, Imag2]}
    \end{itemize}
    \begin{center}
        \includegraphics[width=.7\textwidth]{img/unzip-1.pdf}
    \end{center}
\end{examplebox}

\highspace
\begin{flushleft}
    \textcolor{Green3}{\faIcon{tachometer-alt} \textbf{Parallelism}}
\end{flushleft}
Each element extraction is independent. This allows for \textbf{parallel data access} because there's no dependency between different extractions.

\highspace
However, the unzip operation is an \textbf{efficient data extraction because it takes advantage of stride-based memory access}, which can be optimized for performance.