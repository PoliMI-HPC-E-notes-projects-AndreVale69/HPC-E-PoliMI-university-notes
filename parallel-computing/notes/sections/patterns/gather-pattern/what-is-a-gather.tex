\subsection{Gather Pattern}

\subsubsection{What is a Gather?}

\begin{flushleft}
    \textcolor{Green3}{\faIcon{question-circle} \textbf{Why is the need for the gather pattern born?}}
\end{flushleft}
There are two main reasons for this:
\begin{enumerate}
    \item \important{Data Movement}.

    \textcolor{Red2}{\faIcon{question} \textbf{Problem}}. Data movement is often more costly than computation, especially when transferring data across memory layers or networks. This includes factors like:
    \begin{itemize}
        \item Locality optimization. \hl{Reducing data access times} by keeping frequently accessed data close to the processing unit.
        \item Efficient data access. Improving performance by \hl{minimizing the cost of accessing scattered data locations}.
    \end{itemize}

    \textcolor{Green3}{\faIcon{check} \textbf{Solution}}. The gather pattern addresses these issues by efficiently \hl{collecting data from multiple, scattered memory locations into a contiguous structure}.


    \item \important{Parallel Data Reorganization}.
    
    \textcolor{Red2}{\faIcon{question} \textbf{Problem}}. To achieve performance gains in parallel algorithms, data often needs to be reorganized efficiently. This involves:
    \begin{itemize}
        \item Parallel data movement. \hl{Moving large datasets simultaneously} rather than sequentially.
        \item Consistency management. \hl{Ensuring data integrity} while reorganizing in parallel.
        \item Intermediate data structures. \hl{Holding temporary results during reorganization}.
    \end{itemize}

    \textcolor{Green3}{\faIcon{check} \textbf{Solution}}. The gather pattern is crucial for parallel data reorganization because it:
    \begin{itemize}[label=\textcolor{Green3}{\faIcon{check}}]
        \item \hl{Enables parallel data movement}, allowing multiple data points to be gathered simultaneously from different sources.
        \item \hl{Maintains data consistency}, ensuring that the reorganization process doesn't introduce errors.
        \item \hl{Supports intermediate structures}, making it easier to manage data during complex transformations.
    \end{itemize}
\end{enumerate}

\newpage

\begin{flushleft}
    \textcolor{Green3}{\faIcon{book} \textbf{What is a Gather Pattern?}}
\end{flushleft}
The \definition{Gather Pattern} is a data movement operation that \textbf{creates a new (\emph{source}) collection of data by reading elements from an existing (\emph{input}) data collection}. It is commonly used in parallel computing to efficiently \hl{reorganize} or \hl{extract data based on specific indices}. At first it might seem complicated, but with some simple examples it will be very easy.

\highspace
In general, it works like this:
\begin{enumerate}
    \item \textbf{Read} data from the input collection at positions specified.
    \item \textbf{Write} the gathered data to the output collection in the same order as the indexes specified in Input.
    \item Returns an output collection that has the same shape (or dimensionality) as the index collection.
\end{enumerate}
It can be seen as a \textbf{combination} of \texttt{map} and \texttt{random read} operations because it \textbf{performs multiple reads from non-contiguous memory locations}.

\highspace
\begin{flushleft}
    \textcolor{Green3}{\faIcon{tools} \textbf{\underline{Serial} Implementation}}
\end{flushleft}
The following pseudocode shows a serial implementation of the gather pattern:
\begin{lstlisting}[language=c++]
template<typename Data, typename Idx>
void gather(
    size_t n,    // number of elements in data collection
    size_t m,    // number of elements in index collection
    Data a[],    // input data collection (n elements)
    Data A[],    // output data collection (m elements)
                 // used to modify the output in place
    Idx idx[]    // input index collection (m elements)
) {
    // Iterate over index collection
    for (size_t i = 0; i < m; ++i) {
        // Get the i-th index
        size_t j = idx[i];
        // Ensure index is within bounds;
        // It avoids buffer overflow,
        // and ensure memory safety
        assert(0 <= j && j < n);
        // Perform random read and write to output
        A[i] = a[j];
    }
}
\end{lstlisting}
The pseudo-signature of the function is:
\begin{itemize}
    \item Input:
    \begin{itemize}
        \item \texttt{a[]}: original data collection (size \texttt{n}).
        \item \texttt{idx[]}: index collection specifying which elements to gather (size \texttt{m}).
    \end{itemize}

    \item Output:
    \begin{itemize}
        \item \texttt{A[]}: output collection to store gathered elements (size \texttt{m}, trivial since we iterate over the input indices \texttt{idx})
    \end{itemize}
\end{itemize}
The process performs the following operations:
\begin{enumerate}
    \item \textbf{Loop through each index} \texttt{i} in the index collection (\texttt{idx[]}).
    \item \textbf{Access the data} at position \texttt{idx[i]} in \texttt{a[]}.
    \item \textbf{Assertion} ensures that the index is valid (within the bounds of \texttt{a[]}).
    \item \textbf{Copy the data} from \texttt{a[idx[i]]} to the corresponding position \texttt{A[i]} in the output collection.
\end{enumerate}
It is interesting to note that there are good \textbf{opportunities to parallelize} the code thanks to the for loop. In fact, \hl{each iteration is independent} of the others (no data dependencies between iterations). More precisely, \hl{each thread writes to a unique position in the output} array (collection) \texttt{A}. Also, \hl{each position is valid} thanks to the assert boundary check.

\highspace
\begin{examplebox}[: Gather Pattern]
    Let the following arguments:
    \begin{itemize}
        \item Source Array, contains the original data elements:
        \begin{equation*}
            \texttt{A = [A, B, C, D, E, F, G, H]}
        \end{equation*}
        Indices \texttt{0} through \texttt{7}, so the \texttt{n} argument is \texttt{8}.

        \item Index Array, specifies which elements from the source array to retrieve:
        \begin{equation*}
            \texttt{idx = [1, 5, 0, 2, 2, 4]}
        \end{equation*}
        Indices \texttt{0} through \texttt{5}, so the \texttt{m} argument is \texttt{6}.

        \item Output Array, will store the gathered elements in the order of the indices \texttt{idx}:
        \begin{equation*}
            \texttt{B = []}
        \end{equation*}
        Empty initially, will be filled based on \texttt{idx}.
    \end{itemize}
    The function calculates:
    \begin{enumerate}
        \item First iteration:
        \begin{enumerate}
            \item Get \emph{0}-th index: \texttt{idx[0] = 1}
            \item Boundary check? \texttt{0 <= idx[0] $\land$ idx[0] < 8} \textcolor{Green3}{\faIcon{check}}
            \item Perform random read: \texttt{B[0] = A[idx[0]] = A[1] = B}
        \end{enumerate}
        \item Second iteration:
        \begin{enumerate}
            \item Get \emph{1}-th index: \texttt{idx[1] = 5}
            \item Boundary check? \texttt{0 <= idx[1] $\land$ idx[1] < 8} \textcolor{Green3}{\faIcon{check}}
            \item Perform random read: \texttt{B[1] = A[idx[1]] = A[5] = F}
        \end{enumerate}
        \newpage
        \item Third iteration:
        \begin{enumerate}
            \item Get \emph{2}-th index: \texttt{idx[2] = 0}
            \item Boundary check? \texttt{0 <= idx[2] $\land$ idx[2] < 8} \textcolor{Green3}{\faIcon{check}}
            \item Perform random read: \texttt{B[2] = A[idx[2]] = A[0] = A}
        \end{enumerate}
        \item Fourth iteration:
        \begin{enumerate}
            \item Get \emph{3}-th index: \texttt{idx[3] = 2}
            \item Boundary check? \texttt{0 <= idx[3] $\land$ idx[3] < 8} \textcolor{Green3}{\faIcon{check}}
            \item Perform random read: \texttt{B[3] = A[idx[3]] = A[2] = C}
        \end{enumerate}
        \item Fifth iteration:
        \begin{enumerate}
            \item Get \emph{4}-th index: \texttt{idx[4] = 2}
            \item Boundary check? \texttt{0 <= idx[4] $\land$ idx[4] < 8} \textcolor{Green3}{\faIcon{check}}
            \item Perform random read: \texttt{B[4] = A[idx[4]] = A[2] = C}
        \end{enumerate}
        \item Sixth iteration:
        \begin{enumerate}
            \item Get \emph{5}-th index: \texttt{idx[5] = 4}
            \item Boundary check? \texttt{0 <= idx[5] $\land$ idx[5] < 8} \textcolor{Green3}{\faIcon{check}}
            \item Perform random read: \texttt{B[5] = A[idx[5]] = A[4] = E}
        \end{enumerate}
    \end{enumerate}
    The returned output collection is:
    \begin{equation*}
        \texttt{B = [B, F, A, C, C, E]}
    \end{equation*}
\end{examplebox}

\noindent
Some interesting observations we can make are:
\begin{itemize}
    \item \textbf{Same dimensionality}: output data collection (array \texttt{B}) has the same number of elements as the number of indexes in the index collection (value \texttt{m}).
    \item \textbf{Same types}: elements of the output collection are of the same type as the input data collection.
\end{itemize}