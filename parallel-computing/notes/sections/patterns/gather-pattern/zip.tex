\subsubsection{Zip}

The \definition{Zip operation} is a special case of the gather pattern where \textbf{two (or more) arrays are combined by interleaving their elements}. It functions like a zipper, \textbf{taking one element from each array in sequence to form a new combined array}. It is important to note that it \textbf{works with different types}, so we can zip elements of different types, like integers and floats, or even complex objects.

\highspace
\begin{flushleft}
    \textcolor{Green3}{\faIcon{tools} \textbf{How does it work?}}
\end{flushleft}
The operation \textbf{takes an element from the first array}, then \textbf{one from the second} array, another from the third, and so on, and \textbf{repeats the process}. The \textbf{output is the combined sequence}.
\begin{examplebox}[: zip operation]
    Consider two arrays:
    \begin{enumerate}
        \item Real Parts: contains real numbers.
        \item Imaginary Parts: contains imaginary numbers.
    \end{enumerate}
    The output is a combined sequence like:
    \begin{equation*}
        \texttt{[Real0, Imag0, Real1, Imag1, Real2, Imag2, ...]}
    \end{equation*}
    \begin{center}
        \includegraphics[width=.7\textwidth]{img/zip-1.pdf}
    \end{center}
\end{examplebox}

\highspace
\begin{flushleft}
    \textcolor{Green3}{\faIcon{tachometer-alt} \textbf{Parallelism}}
\end{flushleft}
Each pair of elements (one from each array) can be \textbf{combined independently}. This independence allows \textbf{parallel execution since there's no dependency between the operations for different pairs}.
