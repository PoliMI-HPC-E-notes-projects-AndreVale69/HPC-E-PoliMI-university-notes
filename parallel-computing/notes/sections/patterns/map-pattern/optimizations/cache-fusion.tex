\paragraph{Cache Fusion}

\definition{Cache Fusion} is an optimization \textbf{technique} that improves the efficiency of memory access by \textbf{ensuring that data is accessed and processed in a way that takes full advantage of the CPU cache}. This is particularly important in parallel computing, where efficient use of cache can significantly enhance performance.

\highspace
\begin{flushleft}
    \textcolor{Green3}{\faIcon{bookmark} \textbf{Key Points}}
\end{flushleft}
\begin{itemize}
    \item \important{Optimizing Cache Usage}. Cache fusion involves \textbf{structuring code and data access patterns to maximize the cache's effectiveness}. This means \hl{minimizing cache misses} and \hl{ensuring} that \hl{data stays in the cache as long as needed}.
    
    \item \important{Reducing Memory Bandwidth Usage}. By optimizing how data is accessed, cache fusion \textbf{reduces the number of memory accesses\break needed}, thus \textbf{decreasing memory bandwidth usage}.
\end{itemize}

\highspace
\begin{flushleft}
    \textcolor{Green3}{\faIcon{question-circle} \textbf{How does it work?}}
\end{flushleft}
\begin{enumerate}
    \item \important{Breaking Operations into Smaller Blocs}. Instead of processing large blocks of data all at once, \textbf{cache fusion breaks these operations into smaller blocks that fit into the cache}.

    \textcolor{Green3}{\faIcon{check-circle}} This ensures that data accessed remains in the cache, reducing the need to repeatedly access main memory.


    \item \important{Efficient Data Movement}. The \textbf{smaller blocks are processed one at a time}, with each block being loaded into the cache, processed, and then written back to memory.

    \textcolor{Green3}{\faIcon{check-circle}} This minimizes the movement of data between the cache and main memory, improving overall performance.
\end{enumerate}

\begin{figure}[!htp]
    \centering
    \includegraphics[width=.4\textwidth]{img/cache-fusion-1.pdf}
    \caption{Graphical example of a cache fusion.}
\end{figure}