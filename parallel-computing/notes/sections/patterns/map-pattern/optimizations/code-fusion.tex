\paragraph{Code Fusion}

\definition{Code Fusion} is an \textbf{optimization technique} used to improve the performance of programs, particularly in parallel computing. It involves \textbf{combining multiple operations into a single}, \hl{more efficient operation}. By doing so, code fusion increases arithmetic intensity, reduces memory/cache usage, and enhances overall computational efficiency.

\highspace
\begin{flushleft}
    \textcolor{Green3}{\faIcon{bookmark} \textbf{Key Points}}
\end{flushleft}
\begin{itemize}
    \item \important{Combining Operations}. Code fusion \textbf{\emph{fuses} together multiple operations so that they are performed simultaneously}, rather than sequentially.
    \begin{examplebox}[: Code Fusion]
        Instead of performing a series of separate operations on an array, we combine them into one loop, reducing the number of passes over the data.
    \end{examplebox}


    \item \important{Increasing Arithmetic Intensity}. Arithmetic intensity refers to the \textbf{ratio of computational operations to memory operations}. By fusing operations, we increase this ratio, meaning \textbf{more calculations are done per memory access}.

    \textcolor{Green3}{\faIcon{check-circle} \textbf{Benefit}}. Higher arithmetic intensity leads to better utilization of the CPU and other computational resources.


    \item \important{Reducing Memory/Cache Usage}. By reducing the number of intermediate results that need to be stored in memory, code fusion \textbf{minimizes memory access and optimizes cache usage}.


    \item \important{Use of Registers}. Ideally, \textbf{operations can be performed using registers alone}, which are the fastest form of storage in a CPU. By keeping data in registers and reducing memory access, code fusion achieves maximum computational efficiency.
\end{itemize}

\begin{figure}[!htp]
    \centering
    \includegraphics[width=.45\textwidth]{img/code-fusion-1.pdf}
    \caption{Graphical example of a code fusion.}
\end{figure}
