\subsubsection{Optimizations}

\paragraph{Sequences of Maps}

\definition{Sequences of Maps} refer to a \textbf{series of map operations that are applied one after another in a pipeline}. \hl{Each operation in the sequence takes the output of the previous operation as its input}. This pattern is common in tasks involving vector math, where multiple small operations like additions and multiplications are applied in sequence to elements of an array or collection.

\highspace
\begin{flushleft}
    \textcolor{Green3}{\faIcon{bookmark} \textbf{Key Points}}
\end{flushleft}
\begin{itemize}
    \item \important{Independent Operations}. \textbf{Each map operation is independent and can be parallelized}. However, when several maps are applied in sequence, each intermediate result might be written to memory.

    \item \important{Vector Math}. Often, tasks in vector math consist of multiple small operations applied as maps, such as \texttt{a[i] = f(a[i])} followed by \texttt{a[i] = g(a[i])}, and so on.

    \item[\textcolor{red}{\faIcon{times}}] \important{Memory Inefficiency}. A naïve implementation that writes each intermediate result to memory can \textbf{waste memory bandwidth} and \textbf{potentially overwhelm the cache}.
    
    \item[\textcolor{Green3}{\faIcon{tachometer-alt}}] \textcolor{Green3}{\textbf{How to Improve Performance?}} In the following pages, we introduce two techniques, code fusion and cache fusion, that can be used to reduce memory bandwidth usage and improve cache efficiency.
\end{itemize}

\highspace
\begin{examplebox}[: Sequences of Maps]
    Image we have two map operations to apply to an array: one to double each element and another to add 3 to each element.
    \begin{lstlisting}[language=c]
for (int i = 0; i < n; ++i) {
    a[i] = a[i] * 2;
}

for (int i = 0; i < n; ++i) {
    a[i] = a[i] + 3;
}\end{lstlisting}
    In this example, each element of the array \texttt{a} is first doubled, and then \texttt{3} is added to each element.
\end{examplebox}
