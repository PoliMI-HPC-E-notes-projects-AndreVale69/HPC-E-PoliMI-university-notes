\subsubsection{Related Patterns}

\definition{Related Patterns} are \textbf{design patterns that share similarities in structure, function, or purpose}. They often \textbf{address similar problems or solutions in different ways}, complementing each other within a broader context of software or algorithm design. Understanding related patterns helps in selecting the most appropriate pattern for a specific task and recognizing how different patterns can be combined or interchanged to achieve optimal results.

\highspace
\begin{flushleft}
    \textcolor{Green3}{\faIcon{stream} \textbf{Patterns related to Map}}
\end{flushleft}
\begin{itemize}
    \item \definition{Stencil Pattern related to Map Pattern}
    \begin{itemize}
        \item[\textcolor{Red2}{\faIcon{book}}] \textcolor{Red2}{\textbf{Description}}. Each instance of the map function accesses neighbors of its input, offset from its usual input.

        In other words, it is used when we need to \textbf{apply a function to a collection of elements, but each function's input may also involve neighboring elements}.

        \item[\textcolor{Green3}{\faIcon{question}}] \textcolor{Green3}{\textbf{Usage}}. Common in tasks like image processing or solving partial differential equations (PDEs), where the value of each element depends not only on itself but also on its surrounding elements.
    \end{itemize}
    \begin{figure}[!htp]
        \centering
        \includegraphics[width=.8\textwidth]{img/stencil-pattern-2.pdf}
        \caption{Visual representation of the Stencil Pattern.}
    \end{figure}
    \begin{examplebox}[: Stencil Pattern related to Map Pattern]
        In a 2D grid, the function applied to each cell may also read values from its neighboring cells (e.g., for edge detection in image processing).
    \end{examplebox}


    \item \definition{Workpile Pattern related to Map Pattern}
    \begin{itemize}
        \item[\textcolor{Red2}{\faIcon{book}}] \textcolor{Red2}{\textbf{Description}}. Workpile pattern \textbf{allows work items to be added to the map while it is in progress, from inside map function instances}.

        \item[\textcolor{Green3}{\faIcon{question}}] \textcolor{Green3}{\textbf{Usage}}. The work grows and is consumed by the map. It terminates when no more work is available.
    \end{itemize}
    \begin{examplebox}[: Workpile Pattern related to Map Pattern]
        Useful in tasks like graph traversal or dynamic task generation where new tasks are created during the execution.
    \end{examplebox}


    \item \definition{Divide-and-Conquer Pattern related to Map Pattern}
    \begin{itemize}
        \item[\textcolor{Red2}{\faIcon{book}}] \textcolor{Red2}{\textbf{Description}}. This pattern applies if a \textbf{problem can be divided into smaller subproblems recursively until a base case is reached that can be solved serially}.

        \item[\textcolor{Green3}{\faIcon{question}}] \textcolor{Green3}{\textbf{Usage}}. Common in algorithms like mergesort, quicksort, and solving mathematical problems such as the Fibonacci sequence or matrix multiplication.
    \end{itemize}
    \begin{examplebox}[: Divide-and-Conquer Pattern related to Map Pattern]
        Divide an array into smaller subarrays, sort them individually, and then merge the sorted subarrays.
    \end{examplebox}
\end{itemize}
