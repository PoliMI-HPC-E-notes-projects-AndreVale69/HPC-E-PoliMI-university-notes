\subsection{Map Pattern}\label{subsection: Map Pattern}

\subsubsection{What is a Map?}

The \definition{Map Pattern} involves \textbf{applying a function to each element in a collection independently (without knowledge of neighbors), producing a new collection of results}. This pattern is \hl{highly parallelizable} because each element is processed independently of the others.

\highspace
\begin{flushleft}
    \textcolor{Green3}{\faIcon{bookmark} \textbf{Key Points}}
\end{flushleft}
\begin{itemize}
    \item \important{Independence}. \textbf{Each iteration of the map operation is independent of the others}, making it highly parallelizable.
    
    More precisely, assuming an infinite number of processors $\infty$ available, the operation takes constant time $O(1)$, but the overhead for managing these parallel tasks grows logarithmically with input size \hl{$O\left(\log n\right)$}; however, \hl{even with the implementation overhead, the time complexity remains very efficient and grows slowly with input size}.

    \textbf{Modifying shared state breaks parallelism independence}, and the violation can cause some problems: non-determinism, data races, undefined behavior, segmentation faults.

    \item \important{Unary Maps}. Maps can \textbf{operate over a single collection} (one input, one output).
    \begin{examplebox}[: Unary Map]
        \begin{lstlisting}[language=JavaScript]
// Unary Map takes a single value
// (a single position of the input array)
// and returns a single value
// (the new value of the output array)
function elementalFunction(val) {
    return val * 2;
}

// input
const inArr = [1, 2, 3, 4];

// apply unary map
const outArr = inArr.map((inVal) => elementalFunction(inVal));

// expected output: [2, 4, 6, 8]
console.log(outArr);\end{lstlisting}
    \end{examplebox}

    \newpage

    \item \important{N-ary Maps}. Maps can also \textbf{operate with multiple inputs and outputs}, such as performing element-wise operations on two arrays.
    \begin{examplebox}[: N-ary Map]
        \begin{center}
            \includegraphics[width=.7\textwidth]{img/n-ary-map-pattern-1.pdf}
        \end{center}
        \begin{lstlisting}[language=c]
int twoToOne(int x, int y) {
    return x + y;
}\end{lstlisting}
    \end{examplebox}
\end{itemize}
\begin{figure}[!htp]
    \centering
    \includegraphics[width=\textwidth]{img/map-pattern-2.pdf}
    \caption{A sequential Map on the left and a parallel Map on the right.}
\end{figure}

\highspace
\begin{flushleft}
    \textcolor{Green3}{\faIcon{check-circle} \textbf{Advantages}}
\end{flushleft}
\begin{itemize}[label=\textcolor{Green3}{\faIcon{check}}]
    \item \textcolor{Green3}{\textbf{Easy to Parallelize}}. Since each element is processed independently, map operations can be easily parallelized to improve performance.

    \item \textcolor{Green3}{\textbf{Local State}}. Each element has its own input and output, avoiding shared state issues.
\end{itemize}
