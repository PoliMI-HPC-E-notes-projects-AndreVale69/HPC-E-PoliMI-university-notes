\subsubsection{Bin}\label{subsubsection: Bin}

The \definition{Bin operation} is a \textbf{generalization of the Split operation}, where elements are categorized into \textbf{more than two groups (bins)} instead of just two.

\highspace
\begin{flushleft}
    \textcolor{Green3}{\faIcon{tools} \textbf{How does Split work?}}
\end{flushleft}
\begin{enumerate}
    \item Each element is assigned a category (bin number).
    \item Elements are grouped based on their bin numbers.
    \item The final arrangement collects all elements belonging to the same bin together.
\end{enumerate}

\begin{examplebox}[: Bin operation]
    We have:
    \begin{equation*}
        \begin{array}{rcl}
            \texttt{Elements} &=& \texttt{[A, B, C, D, E, F, G, H]} \\ [.5em]
            \texttt{Mask (Categories/Bins)} &=& \texttt{[2, 0, 1, 3, 3, 1, 1, 2]}
        \end{array}
    \end{equation*}
    What we do is:
    \begin{enumerate}
        \item Identify the Unique Bins: the mask contains four different categories (0, 1, 2, 3), so we create 4 bins.
        \item Distribute Elements into Bins:
        \begin{itemize}
            \item Bin (category) 0 $\rightarrow$ Elements Assigned: \texttt{B}
            \item Bin (category) 1 $\rightarrow$ Elements Assigned: \texttt{C, F, G}
            \item Bin (category) 2 $\rightarrow$ Elements Assigned: \texttt{A, H}
            \item Bin (category) 3 $\rightarrow$ Elements Assigned: \texttt{D, E}
        \end{itemize}
        \item Arrange output by Bins: elements are grouped together in bin order:
        \begin{equation*}
            \texttt{[B, C, F, G, A, H, D, E]}
        \end{equation*}
    \end{enumerate}
\end{examplebox}

\highspace
\begin{flushleft}
    \textcolor{Green3}{\faIcon{balance-scale} \textbf{Differences between Split and Bin}}
\end{flushleft}

\begin{table}[!htp]
    \centering
    \begin{tabular}{@{} l | c c @{}}
        \toprule
        \textbf{Feature} & \textbf{Split (Binary)} & \textbf{Bin (Generalized Split)} \\
        \midrule
        Categories & Only 2 & Greater than 2 (\underline{minimum 3}) \\
        Grouping & Two groups & Multiple bins \\
        \bottomrule
    \end{tabular}
\end{table}