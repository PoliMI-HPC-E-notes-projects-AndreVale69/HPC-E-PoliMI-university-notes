\subsubsection{Expand}

The \definition{Expand pattern} is a flexible generalization of the Pack pattern where \textbf{each input element can produce multiple output elements}, instead of at most one. In other words, it dynamically increases the number of outputs per input. It is widely used in parallel processing to efficiently handle variable-length results.

\highspace
\begin{flushleft}
    \textcolor{Green3}{\faIcon{star} \textbf{Key Idea}}
\end{flushleft}
\begin{enumerate}
    \item Unlike Pack, which filters elements, Expand \textbf{duplicates or expands elements}.
    \item \textbf{Each element can generate any number of outputs} (zero, one, or multiple).
    \item Useful when processing elements \textbf{produces variable-length results}.
\end{enumerate}

\begin{figure}[!htp]
    \centering
    \includegraphics[width=.6\textwidth]{img/expand-1.pdf}
\end{figure}

\begin{flushleft}
    \textcolor{Green3}{\faIcon{question-circle} \textbf{Why is this useful?}}
\end{flushleft}
\begin{itemize}
    \item \textbf{Adaptive Computation}: Expands based on conditions, useful in graph algorithms and database queries.
    \item \textbf{Efficient Storage}: Dynamically handles variable-length outputs without wasting memory.
    \item \textbf{Parallel Processing}: Enables dynamic parallel work distribution, where different inputs produce varying workloads.
\end{itemize}