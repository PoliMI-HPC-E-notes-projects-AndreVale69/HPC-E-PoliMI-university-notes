\subsubsection{Split}

The \definition{Split operation} is an \hl{extension of the Pack pattern}, but instead of removing elements, it \textbf{reorganizes them into \underline{exactly} two groups while preserving all data} (If there are more than two groups, it is a bin operation, see page \pageref{subsubsection: Bin}).

\highspace
\begin{flushleft}
    \textcolor{Green3}{\faIcon{tools} \textbf{How does Split work?}}
\end{flushleft}
\begin{enumerate}
    \item Each element has a corresponding state (0 or 1).
    \item Instead of discarding elements, as in the normal Pack operation, elements are \textbf{moved into two separate regions} in the output array.
    \item Elements with one state (e.g., 0) go to the first half, while elements with the other state (e.g., 1) go to the second half.
\end{enumerate}

\highspace
\begin{examplebox}[: Split operation]
    We have:
    \begin{equation*}
        \begin{array}{rcl}
            \texttt{Input elements} &=& \texttt{[A, B, C, D, E, F, G, H]} \\ [.5em]
            \texttt{Binary Condition (Mask)} &=& \texttt{[1, 0, 0, 1, 1, 0, 0, 0]}
        \end{array}
    \end{equation*}
    What we do is:
    \begin{enumerate}
        \item Classify elements:
        \begin{itemize}
            \item Elements corresponding to 0: \texttt{[B, C, F, G, H]}
            \item Elements corresponding to 1: \texttt{[A, D, E]}
        \end{itemize}
        \item Place Elements in two groups:
        \begin{itemize}
            \item Upper half (0s first): \texttt{[B, C, F, G, H]}
            \item Lower half (1s second): \texttt{[A, D, E]}
        \end{itemize}
    \end{enumerate}
    And the result is:
    \begin{equation*}
        \texttt{[B, C, F, G, H, A, D, E]}
    \end{equation*}
\end{examplebox}

\highspace
\begin{flushleft}
    \textcolor{Green3}{\faIcon{balance-scale} \textbf{Differences between Pack and Split}}
\end{flushleft}
\begin{table}[!htp]
    \centering
    \begin{tabular}{@{} l | c c @{}}
        \toprule
        \textbf{Feature} & \textbf{Pack} & \textbf{Split} \\
        \midrule
        Removes elements? & \textcolor{Green3}{\faIcon{check}} & \textcolor{Red2}{\faIcon{times}} \\
        Reorganizes elements? & \textcolor{Green3}{\faIcon{check}} & \textcolor{Green3}{\faIcon{check}} \\
        Maintains all data? & \textcolor{Red2}{\faIcon{times} \textbf{(discards elements)}} & \textcolor{Green3}{\faIcon{check} \textbf{(all data kept)}} \\
        \bottomrule
    \end{tabular}
\end{table}
