\subsubsection{Unsplit}

The \definition{Unsplit operation} is the inverse of Split. Instead of separating elements into two groups, it \textbf{reconstructs the original ordering from the split version}.

\highspace
\begin{flushleft}
    \textcolor{Green3}{\faIcon{tools} \textbf{How does Split work?}}
\end{flushleft}
\begin{enumerate}
    \item Start with the output formed by Split.
    \item Use the original mask (0s and 1s) to determine where elements should go.
    \item Reconstruct the original sequence by placing elements back in their original positions.
\end{enumerate}

\begin{examplebox}[: Unsplit operation]
    We have:
    \begin{equation*}
        \begin{array}{rcl}
            \texttt{Output from Split} &=& \texttt{[B, C, F, G, H, A, D, E]} \\ [.5em]
            \texttt{Binary Condition (Mask)} &=& \texttt{[1, 0, 0, 1, 1, 0, 0, 0]}
        \end{array}
    \end{equation*}
    What we do is:
    \begin{enumerate}
        \item Place elements back according to the mask:
        \begin{itemize}
            \item First position (\texttt{1} in the mask) $\rightarrow$ Take from Lower half: \texttt{A}
            \item Next (\texttt{0} in the mask) $\rightarrow$ Take from Upper half: \texttt{B}
            \item Next (\texttt{0} in the mask) $\rightarrow$ Take from Upper half: \texttt{C}
            \item Next (\texttt{1} in the mask) $\rightarrow$ Take from Lower half: \texttt{D}
            \item Next (\texttt{1} in the mask) $\rightarrow$ Take from Lower half: \texttt{E}
            \item Next (\texttt{1} in the mask) $\rightarrow$ Take from Lower half: \texttt{F}
            \item Next (\texttt{0} in the mask) $\rightarrow$ Take from Upper half: \texttt{G}
            \item Next (\texttt{0} in the mask) $\rightarrow$ Take from Upper half: \texttt{H}
        \end{itemize}
    \end{enumerate}
    And the result is (original order restored):
    \begin{equation*}
        \texttt{[A, B, C, D, E, F, G, H]}
    \end{equation*}
\end{examplebox}

\highspace
\begin{flushleft}
    \textcolor{Green3}{\faIcon{balance-scale} \textbf{Differences between Split and Unsplit}}
\end{flushleft}
\begin{table}[!htp]
    \centering
    \begin{tabular}{@{} l | c c @{}}
        \toprule
        \textbf{Feature} & \textbf{Split} & \textbf{Unsplit} \\
        \midrule
        Operation type & Reorders into two groups & Reconstructs original order \\
        Loses information? & \textcolor{Red2}{\faIcon{times}} & \textcolor{Red2}{\faIcon{times}} \\
        Involves a mask? & \textcolor{Green3}{\faIcon{check}} & \textcolor{Green3}{\faIcon{check}} \\
        Data organization & Elements grouped & Elements placed back \\
        \bottomrule
    \end{tabular}
\end{table}
