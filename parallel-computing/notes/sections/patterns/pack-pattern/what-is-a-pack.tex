\subsection{Pack Pattern}

\subsubsection{What is a Pack?}

The \definition{Pack Pattern} is used to \textbf{remove unused elements from a collection while keeping the retained elements contiguous in memory}. The goal is to \textbf{improve performance by reducing memory fragmentation and increasing data locality}, which benefits cache efficiency.

\highspace
\begin{flushleft}
    \textcolor{Green3}{\faIcon{tools} \textbf{How the Pack Algorithm works}}
\end{flushleft}
\begin{enumerate}
    \item \textbf{Convert the input array into Boolean values}. Each element is mapped to \texttt{1} (\emph{keep}) or \texttt{0} (\emph{discard}).
    \item \textbf{Perform an exclusive scan (prefix sum) on the Boolean array}. This step computes the new positions (offsets) of the retained elements.
    \item \textbf{Use the computed offsets to move retained elements to their new locations}. The elements corresponding to \texttt{1}'s in the Boolean array are copied to their new positions.
\end{enumerate}
\begin{examplebox}[: Pack Pattern]
    We have an input array and a mask that tells us which elements to keep (\texttt{1}) and which to remove (\texttt{0}):
    \begin{equation*}
        \begin{array}{rcl}
            \texttt{Index}          & = & \texttt{[0, 1, 2, 3, 4, 5, 6, 7]} \\ [.5em]
            \texttt{Input}          & = & \texttt{[A, B, C, D, E, F, G, H]} \\ [.5em]
            \texttt{Keep (Mask)}    & = & \texttt{[0, 1, 1, 0, 0, 1, 1, 1]}
        \end{array}
    \end{equation*}
    We want to create a new array with only the elements marked \texttt{1}, but we don't yet know \emph{where} to place them.

    The exclusive scan (prefix sum) on the \texttt{Keep} array calculates the new indices for the elements we want to keep.

    We compute the scan using addition (starting from \texttt{0}):
    \begin{equation*}
        \begin{array}{rcl}
            \texttt{Index}                      & = & \texttt{[0, 1, 2, 3, 4, 5, 6, 7]} \\ [.5em]
            \texttt{Keep (Mask)}                & = & \texttt{[0, 1, 1, 0, 0, 1, 1, 1]} \\ [.5em]
            \texttt{Exclusive Scan (New Index)} & = & \texttt{[0, 0, 1, 2, 2, 2, 3, 4]}
        \end{array}
    \end{equation*}
    The scan tells us:
    \begin{itemize}
        \item \texttt{B} goes to index 0
        \item \texttt{C} goes to index 1
        \item \texttt{F} goes to index 2
        \item \texttt{G} goes to index 3
        \item \texttt{H} goes to index 4
    \end{itemize}
    \begin{equation*}
        \texttt{[B, C, F, G, H]}
    \end{equation*}
\end{examplebox}

\highspace
\begin{flushleft}
    \textcolor{Green3}{\faIcon{question-circle} \textbf{Why is Packing Useful?}}
\end{flushleft}
\begin{itemize}
    \item \textcolor{Green3}{\textbf{Reduces memory waste}} by eliminating unnecessary elements.
    \item \textcolor{Green3}{\textbf{Improves data locality}} (sequential memory access is faster than scattered access).
    \item \textcolor{Green3}{\textbf{Boosts cache efficiency}}, reducing cache misses and improving parallel performance.
\end{itemize}

\highspace
\begin{flushleft}
    \textcolor{Red2}{\faIcon{book} \textcolor{Red2}{\textbf{What is the inverse of the Pack pattern?}}}
\end{flushleft}
The \definition{Unpack Pattern} is the inverse of packing. It \textbf{restores the packed elements to their original positions} in an expanded array, reintroducing gaps where necessary.

\highspace
\begin{flushleft}
    \textcolor{Green3}{\faIcon{tools} \textbf{How Unpacking Works}}
\end{flushleft}
Given the same Boolean mask used in packing, unpacking spreads elements back to their original locations, \textbf{leaving empty slots where elements were previously removed}.
\begin{enumerate}
    \item Start with the packed array and the original mask.
    \item Read the mask to decide where elements should go:
    \begin{itemize}
        \item If 1, we take the next packed element.
        \item If 0, we insert a placeholder (empty slot).
    \end{itemize}
    \item Reconstruct the original array.
\end{enumerate}

\begin{examplebox}[: Unpack Pattern]
    We have:
    \begin{enumerate}
        \item The packed array: \texttt{[B, C, F, G, H]}
        \item The original mask: \texttt{[0, 1, 1, 0, 0, 1, 1, 1]}
    \end{enumerate}
    Now we will restore elements using the mask:
    \begin{itemize}
        \item If \texttt{Keep (Mask) = 1}, we copy the next element from the packed array into that position.
        \item If \texttt{Keep (Mask) = 0}, we insert an empty slot (or default value).
    \end{itemize}
    Finally, we place elements back into their original positions using the mask:
    \begin{equation*}
        \begin{array}{rcl}
            \texttt{Index}          & = & \texttt{[0, 1, 2, 3, 4, 5, 6, 7]} \\ [.5em]
            \texttt{Restored Value} & = & \texttt{[\_, B, C, \_, \_, F, G, H]}
        \end{array}
    \end{equation*}
\end{examplebox}

\newpage

\begin{flushleft}
    \textcolor{Green3}{\faIcon{\speedIcon} \textbf{Fusion of Map and Pack: Efficient Filtering of Results}}
\end{flushleft}
The \textbf{fusion of map and pack patterns} is an optimization technique that \textbf{improves efficiency when many elements in a map operation are discarded} (see map pattern on page \pageref{subsection: Map Pattern})

\highspace
The key idea is that \hl{Map pattern applies a function to each element} (e.g., checking for collisions) and \hl{Pack pattern removes unnecessary elements}, storing only meaningful results.

\highspace
This fusion ensures that \textbf{only relevant results are stored and transmitted}, reducing unnecessary computations and memory usage.
\begin{enumerate}
    \item \important{Map checks for collisions}. \textbf{Each element is processed using a mapping function}, which determines whether the element should be kept or discarded. Some elements fail the check.
    \item \important{Pack stores only actual collisions}. \textbf{Only elements that passed the check} (e.g., \texttt{B, C, F, G, H}) \textbf{are stored in the output}. Elements that failed the check (e.g., \texttt{A, D, E}) are removed. The output contains only relevant elements, avoiding unnecessary storage.
    \item \important{Reduces output bandwidth}. The output size depends on valid results, not on the total number of inputs. This is \textbf{beneficial when most of the input elements are discarded}.
    \item \important{Selective output}. Unlike map-only operations, where each input produces an output, \textbf{this fusion allows selective output}.
\end{enumerate}
