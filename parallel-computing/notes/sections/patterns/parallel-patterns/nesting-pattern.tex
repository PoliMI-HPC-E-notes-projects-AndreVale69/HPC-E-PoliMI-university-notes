\subsubsection{Nesting Pattern}

The \definition{Nesting Pattern} is a compositional pattern\footnote{%
    \definition{Compositional Patterns} are \textbf{design patterns that allow us to compose complex systems from simpler, reusable components}. These patterns emphasize the importance of combining smaller pieces of functionality in a structured and predictable way. They are particularly useful in managing the complexity of large systems by breaking them down into manageable parts.
} that \textbf{allows for the hierarchical composition of patterns}. This is used in both serial and parallel algorithms. The main idea is to \textbf{use \dquotes{pattern diagrams}}\footnote{%
    \definition{Pattern Diagrams} are \textbf{visual tools used to represent and illustrate patterns}, particularly in the context of software engineering and algorithm design. They help in understanding how different components or tasks fit together to form a complete system or process.

    \begin{examplebox}[: Pattern Diagrams]
        Suppose we are designing a parallel algorithm to process large data sets. A pattern diagram can help us visualize the composition of different patterns, such as data partitioning, task scheduling, and data aggregation. Each of these patterns can be represented as a set of nodes and edges nested within a larger pattern that represents the entire algorithm.
    \end{examplebox}
} \textbf{to visually show how tasks can be organized and composed in a hierarchy}.

\highspace
\begin{flushleft}
    \textcolor{Green3}{\faIcon{book} \textbf{Key Points}}
\end{flushleft}
\begin{itemize}
    \item \important{Hierarchical Composition}: The Nesting Pattern \textbf{allows patterns to be nested within each other}, creating a hierarchy of patterns.
    \item \important{Task Blocks}: Each \textbf{\dquotes{task block} in a pattern diagram represents a location of general code in an algorithm}.
    \item \important{Replaceability}: \textbf{Any task block in the diagram can be replaced with a pattern} that has the same input/output and dependencies.
\end{itemize}

\highspace
In the following figure, we can see a diagram showing a multiple task blocks. Some of these blocks are highlighted in green to indicate that they can be replaced with other patterns. The goal is to illustrate how we can nest patterns to build more complex algorithms.
\begin{figure}[!htp]
    \centering
    \includegraphics[width=.7\textwidth]{img/nesting-pattern-1.pdf}
    \caption{Graphical example of a nesting pattern.}
\end{figure}