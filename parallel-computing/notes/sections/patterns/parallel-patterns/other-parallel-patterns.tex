\subsubsection{Other Parallel Patterns}

There are other important parallel patterns that we don't introduce. The following lists some of them with a brief description.
\begin{itemize}
    \item \definition{Superscalar Sequences Parallel Pattern}
    \begin{itemize}
        \item[\textcolor{Red2}{\faIcon{book}}] \textcolor{Red2}{\textbf{Description}}. Write a sequence of tasks, ordered only by dependencies.

        \item[\textcolor{Green3}{\faIcon{question}}] \textcolor{Green3}{\textbf{Usage}}. Helps in organizing tasks such that they can be executed out of order as long as dependencies are respected, increasing parallelism.
    \end{itemize}


    \item \definition{Futures Parallel Pattern}
    \begin{itemize}
        \item[\textcolor{Red2}{\faIcon{book}}] \textcolor{Red2}{\textbf{Description}}. Similar to fork-join, but tasks do not need to be nested hierarchically.

        \item[\textcolor{Green3}{\faIcon{question}}] \textcolor{Green3}{\textbf{Usage}}. Allows asynchronous execution of tasks, where a \dquotes{future} represents a value that will be computed and available in the future (a similar logic exists in Python: \href{https://docs.python.org/3/library/asyncio-future.html#asyncio.Future}{official documentation}).
    \end{itemize}


    \item \definition{Speculative Selection Parallel Pattern}
    \begin{itemize}
        \item[\textcolor{Red2}{\faIcon{book}}] \textcolor{Red2}{\textbf{Description}}. A general version of serial selection where the condition and both outcomes can all run in parallel.

        \item[\textcolor{Green3}{\faIcon{question}}] \textcolor{Green3}{\textbf{Usage}}. Used when it's beneficial to execute multiple branches in parallel and discard the results of the non-selected branch once the condition is evaluated.
    \end{itemize}


    \item \definition{Workpile Parallel Pattern}
    \begin{itemize}
        \item[\textcolor{Red2}{\faIcon{book}}] \textcolor{Red2}{\textbf{Description}}. A general map pattern where each instance of the elemental function can generate more instances, adding to the \dquotes{pile} of work.

        \item[\textcolor{Green3}{\faIcon{question}}] \textcolor{Green3}{\textbf{Usage}}. Efficient for dynamic task generation, where new tasks are created as the algorithm progresses.
    \end{itemize}


    \item \definition{Search Parallel Pattern}
    \begin{itemize}
        \item[\textcolor{Red2}{\faIcon{book}}] \textcolor{Red2}{\textbf{Description}}. Finds some data in a collection that meets specific criteria.

        \item[\textcolor{Green3}{\faIcon{question}}] \textcolor{Green3}{\textbf{Usage}}. Utilized in searching algorithms where multiple search operations can run in parallel to speed up the process.
    \end{itemize}


    \item \definition{Segmentation Parallel Pattern}
    \begin{itemize}
        \item[\textcolor{Red2}{\faIcon{book}}] \textcolor{Red2}{\textbf{Description}}. Operations on subdivided, non-overlapping, non-uni-\break formly sized partitions of 1D collections.

        \item[\textcolor{Green3}{\faIcon{question}}] \textcolor{Green3}{\textbf{Usage}}. Effective for processing large datasets by dividing them into manageable segments that can be processed in parallel.
    \end{itemize}


    \newpage


    \item \definition{Expand Parallel Pattern}
    \begin{itemize}
        \item[\textcolor{Red2}{\faIcon{book}}] \textcolor{Red2}{\textbf{Description}}. A combination of pack and map patterns.

        \item[\textcolor{Green3}{\faIcon{question}}] \textcolor{Green3}{\textbf{Usage}}. Allows elements to be expanded and processed in parallel, useful for operations that generate variable amounts of output from each input element.
    \end{itemize}


    \item \definition{Category Reduction Parallel Pattern}
    \begin{itemize}
        \item[\textcolor{Red2}{\faIcon{book}}] \textcolor{Red2}{\textbf{Description}}. Given a collection of elements each with a label, find all elements with the same label and reduce them.

        \item[\textcolor{Green3}{\faIcon{question}}] \textcolor{Green3}{\textbf{Usage}}. Common in data aggregation tasks where elements are\break grouped by a key (label) and then reduced (aggregated) by some operation.
    \end{itemize}
\end{itemize}
