\subsubsection{Parallel Control Patterns}

\definition{Parallel Control Patterns} are \textbf{design patterns used in parallel computing to manage the execution flow of tasks that can run simultaneously}. These patterns help in organizing and structuring parallel code to ensure efficient execution, resource management, and scalability. They are essential for leveraging the power of multi-core processors and distributed computing systems.

\highspace
In essence, parallel control patterns provide the blueprint for designing and implementing parallel algorithms, ensuring that tasks are executed efficiently and effectively.

\highspace
\begin{flushleft}
    \textcolor{Green3}{\faIcon{stream} \textbf{Types}}
\end{flushleft}
\begin{enumerate}
    \item \definition{Fork-Join Parallel Pattern}
    \begin{itemize}
        \item[\textcolor{Red2}{\faIcon{book}}] \textcolor{Red2}{\textbf{Definition}}. The Fork-Join pattern \hl{allows control flow to fork into multiple parallel flows and then rejoin later}.

        \item[\textcolor{Green3}{\faIcon{tools}}] \textcolor{Green3}{\textbf{Implementation}}. Languages like Cilk Plus use \texttt{spawn} and \texttt{sync} to implement this pattern. When a function is spawned, it runs in parallel with the caller. The caller then uses \texttt{sync} to wait for all spawned functions to complete before proceeding.

        \item[\textcolor{Green3}{\faIcon{not-equal}}] \textcolor{Green3}{\textbf{Difference from Barrier}}. A \dquotes{join} involves only one thread continuing, while a \dquotes{barrier} allows all threads to continue.
    \end{itemize}


    \item \definition{Map Parallel Pattern}
    \begin{itemize}
        \item[\textcolor{Red2}{\faIcon{book}}] \textcolor{Red2}{\textbf{Definition}}. The Map pattern \hl{applies a function to each element of a collection independently}, producing a new collection of results.

        \item[\textcolor{Green3}{\faIcon{tools}}] \textcolor{Green3}{\textbf{Usage}}. It replaces serial iterations where each iteration is independent.

        \item[\textcolor{Green3}{\faIcon{bookmark}}] \textcolor{Green3}{\textbf{Elemental Function}}. The function applied to each element is called an \dquotes{elemental function}.
    \end{itemize}
    \begin{figure}[!htp]
        \centering
        \includegraphics[width=.7\textwidth]{img/map-pattern-1.pdf}
        \caption{Visual representation of the Map Parallel Pattern.}
    \end{figure}

    \begin{examplebox}[: Map analogy]
        A good analogy is the \texttt{map} method used in Javascript. The difference is that the \texttt{map} in Javascript is done sequentially, instead the map pattern is done in parallel. However, as we can see from the example provided in the official documentation, a function is applied to each element of the input:
        \begin{lstlisting}[language=JavaScript]
// Elemental Function
function elementalFunction(num) {
  return num * 2;
}

// Example array
const array1 = [1, 4, 9, 16];

// Pass an elementalFunction to map
const map1 = array1.map((x) => elementalFunction(x));

console.log(map1);
// Expected output: Array [2, 8, 18, 32]\end{lstlisting}
        Here the official documentation:
        \begin{center}
            \qrcode{https://developer.mozilla.org/en-US/docs/Web/JavaScript/Reference/Global_Objects/Array/map}
        \end{center}
    \end{examplebox}


    \item \definition{Stencil Parallel Pattern}
    \begin{itemize}
        \item[\textcolor{Red2}{\faIcon{book}}] \textcolor{Red2}{\textbf{Definition}}. The Stencil pattern \hl{extends the Map pattern by allowing the elemental function to access neighboring elements}.

        \item[\textcolor{Green3}{\faIcon{tools}}] \textcolor{Green3}{\textbf{Usage}}. Commonly used in iterative solvers or simulations where the state of an element depends on its neighbors.

        \item[\textcolor{Green3}{\faIcon{question-circle}}] \textcolor{Green3}{\textbf{Consideration}}. Boundary conditions must be handled carefully.
    \end{itemize}
    \begin{figure}[!htp]
        \centering
        \includegraphics[width=.44\textwidth]{img/stencil-pattern-1.pdf}
        \caption{Visual representation of the Stencil Parallel Pattern.}
    \end{figure}


    \item \hqlabel{definition: Reduction Parallel Pattern}{\definition{Reduction Parallel Pattern}}
    \begin{itemize}
        \item[\textcolor{Red2}{\faIcon{book}}] \textcolor{Red2}{\textbf{Definition}}. The Reduction pattern \hl{combines elements of a collection using an associative ``combiner function''} (e.g., addition, multiplication, max, min).

        \item[\textcolor{Green3}{\faIcon{tachometer-alt}}] \textcolor{Green3}{\textbf{Parallelism}}. Different orderings of the reduction are possible due to the associative property, making it suitable for parallel execution.
    \end{itemize}
    \begin{figure}[!htp]
        \centering
        \includegraphics[width=.84\textwidth]{img/reduction-pattern-1.pdf}
        \caption{Visual representation of the Reduction Parallel Pattern.}
    \end{figure}
    \begin{examplebox}[: Reduction analogy]
        A good analogy is the \texttt{reduce} method used in Javascript. The difference is that the \texttt{reduce} in Javascript is done sequentially, instead the reduce pattern is done in parallel.
        \begin{lstlisting}[language=JavaScript]
function combinerFun(acc, curVal) {
    return acc + curVal;
}

// Example array
const arrayToReduce = [1, 2, 3, 4];

// 0 + 1 + 2 + 3 + 4
// steps:
// 0: 0 + 1
// 1: 1 + 2
// 2: 3 + 3
// 3: 6 + 4
// result = 10
const initialValue = 0;
const sumOfArray = arrayToReduce.reduce(
  (acc, curVal) => combinerFun(acc, curVal),
  initialValue
);

console.log(sumOfArray);
// Expected output: 10\end{lstlisting}
        Here the official documentation:
        \begin{center}
            \qrcode{https://developer.mozilla.org/en-US/docs/Web/JavaScript/Reference/Global_Objects/Array/reduce}
        \end{center}
    \end{examplebox}


    \item \definition{Scan Parallel Pattern}
    \begin{itemize}
        \item[\textcolor{Red2}{\faIcon{book}}] \textcolor{Red2}{\textbf{Definition}}. The Scan pattern \hl{computes all partial reductions of a collection}.

        \item[\textcolor{Green3}{\faIcon{tools}}] \textcolor{Green3}{\textbf{Usage}}. For each element in the output collection, it computes a reduction of the input elements up to that point.

        \item[\textcolor{Green3}{\faIcon{question-circle}}] \textcolor{Green3}{\textbf{Parallelization}}. Although it has dependencies, the associative nature of the function allows parallelization.
    \end{itemize}
    \begin{figure}[!htp]
        \centering
        \includegraphics[width=.84\textwidth]{img/scan-pattern-1.pdf}
        \caption{Visual representation of the Scan Parallel Pattern.}
    \end{figure}
    \begin{examplebox}[: Scan analogy]
        \begin{lstlisting}[language=JavaScript]
// JavaScript doesn't provide a native implementation of the scan,
// but we can build it manually
// * array - input
// * fn - reduction to apply to input values
// * initialValue - initial value of the accumulator
function scan(array, fn, initialValue) {
    // create the output array
    const result = [];
    // The scan method is a partial reduction,
    // so we need to apply the partial reduce
    array.reduce((acc, value, index) => {
        // combiner function to apply
        const newAcc = fn(acc, value);
        // save the calculated value
        // at the i-th iteration
        result[index] = newAcc;
        // returns the calculated value,
        // since it will be the new accumulator
        return newAcc;
    }, initialValue);
    // return the output
    return result;
}

// combiner function of the reduction pattern
function combinerFunc(accumulator, currentValue) {
    return accumulator + currentValue;
}

// Example array
const arr = [1, 2, 3, 4];
// Scan
const sumRes = scan(
    arr, (a, b) => combinerFunc(a, b), 0
);

console.log(sumRes);
// Expected output: [1, 3, 6, 10]\end{lstlisting}
        To test the execution, here there is a free console (or just open the browser console):
        \begin{center}
            \qrcode{https://www.programiz.com/online-compiler/74XgOz49pwsxt}
        \end{center}
    \end{examplebox}


    \item \definition{Recurrence Parallel Pattern}
    \begin{itemize}
        \item[\textcolor{Red2}{\faIcon{book}}] \textcolor{Red2}{\textbf{Definition}}. The Recurrence pattern is a more complex version of Map, where \hl{elements can depend on the outputs of adjacent elements}.

        \item[\textcolor{Green3}{\faIcon{tools}}] \textcolor{Green3}{\textbf{Usage}}. Similar to Map, but requires a serial ordering of elements for computability.
    \end{itemize}
\end{enumerate}
