\subsubsection{Parallel Data Management Patterns}

\definition{Parallel Data Management Patterns} are \textbf{strategies used to manage data in parallel computing environments}. These patterns ensure that data is properly managed, avoiding issues like race conditions and ensuring efficient data locality. They are crucial for optimizing the performance of parallel programs by managing how data is allocated, shared, accessed, and manipulated across multiple parallel workers.

\highspace
\begin{flushleft}
    \textcolor{Green3}{\faIcon{stream} \textbf{Types}}
\end{flushleft}
\begin{enumerate}
    \item \definition{Pack Parallel Pattern}
    \begin{itemize}
        \item[\textcolor{Red2}{\faIcon{book}}] \textcolor{Red2}{\textbf{Description}}. Pack is used to \textbf{eliminate unused space in a collection by discarding elements marked as false and placing the remaining elements in a contiguous sequence}.

        \item[\textcolor{Green3}{\faIcon{question}}] \textcolor{Green3}{\textbf{Usage}}. Commonly used with the Map pattern.

        \item[\textcolor{Green3}{\faIcon{bookmark}}] \textcolor{Green3}{\textbf{Inverse Operation}}. Unpack is the inverse operation, placing elements back in their original locations.
    \end{itemize}
    \begin{figure}[!htp]
        \centering
        \includegraphics[width=.6\textwidth]{img/pack-pattern-1.pdf}
        \caption{Visual representation of the Pack Parallel Pattern.}
    \end{figure}


    \item \definition{Pipeline Parallel Pattern}
    \begin{itemize}
        \item[\textcolor{Red2}{\faIcon{book}}] \textcolor{Red2}{\textbf{Description}}. \textbf{Pipeline connects tasks in a producer consumer manner}. Tasks are organized in a linear sequence where the output of one task becomes the input of the next.

        \item[\textcolor{Green3}{\faIcon{question}}] \textcolor{Green3}{\textbf{Usage}}. Useful when combined with other patterns to multiply available parallelism. Pipelines can also be structured as Directed Acyclic Graphs (DAGs).
    \end{itemize}
    \newpage
    \begin{figure}[!htp]
        \centering
        \includegraphics[width=.2\textwidth]{img/pipeline-pattern-1.pdf}
        \caption{Visual representation of the Pipeline Parallel Pattern.}
    \end{figure}


    \item \definition{Geometric Decomposition Parallel Pattern}
    \begin{itemize}
        \item[\textcolor{Red2}{\faIcon{book}}] \textcolor{Red2}{\textbf{Description}}. \textbf{Geometric Decomposition arranges data into subcollections}, which can be overlapping or non-overlapping.

        \item[\textcolor{Green3}{\faIcon{question}}] \textcolor{Green3}{\textbf{Usage}}. This pattern provides a different view of the data without necessarily moving it. It's commonly used in grid-based computations like finite element analysis or image processing.
    \end{itemize}
    \begin{figure}[!htp]
        \centering
        \includegraphics[width=\textwidth]{img/geometric-decomposition-pattern-1.pdf}
        \caption{Visual representation of the Geometric Decomposition Parallel Pattern.}
    \end{figure}


    \newpage
    \item \definition{Gather Parallel Pattern}
    \begin{itemize}
        \item[\textcolor{Red2}{\faIcon{book}}] \textcolor{Red2}{\textbf{Description}}. \textbf{Gather reads a collection of data given a collection of indices}, similar to a combination of Map and random serial reads.

        \item[\textcolor{Green3}{\faIcon{question}}] \textcolor{Green3}{\textbf{Usage}}. The output collection shares the same type as the input collection but has the same shape as the indices collection.
    \end{itemize}
    \begin{figure}[!htp]
        \centering
        \includegraphics[width=\textwidth]{img/gather-pattern-1.pdf}
        \caption{Visual representation of the Gather Parallel Pattern.}
    \end{figure}


    \item \definition{Scatter Parallel Pattern}
    \begin{itemize}
        \item[\textcolor{Red2}{\faIcon{book}}] \textcolor{Red2}{\textbf{Description}}. Scatter is the inverse of Gather. It \textbf{writes each element of the input to the output at the given index specified by a collection of indices}.

        \item[\textcolor{Green3}{\faIcon{question}}] \textcolor{Green3}{\textbf{Usage}}. Race conditions can occur when multiple writes target the same location, so care must be taken to avoid conflicts.
    \end{itemize}
    \begin{figure}[!htp]
        \centering
        \includegraphics[width=\textwidth]{img/scatter-pattern-1.pdf}
        \caption{Visual representation of the Scatter Parallel Pattern.}
    \end{figure}
\end{enumerate}
