\subsubsection{Serial Data Management Patterns}

\definition{Serial Data Management Patterns} refer to the various \textbf{methods and strategies used by serial programs to manage data}. These patterns deal with how data is allocated, shared, read, written, and copied in a program. Understanding these patterns is crucial for writing efficient and maintainable code.

\highspace
\begin{flushleft}
    \textcolor{Green3}{\faIcon{stream} \textbf{Types}}
\end{flushleft}
\begin{enumerate}
    \item \definition{Random Read and Write Pattern}
    \begin{itemize}
        \item[\textcolor{Red2}{\faIcon{book}}] \textcolor{Red2}{\textbf{Definition}}. \textbf{Memory locations are accessed using addresses}, often through pointers. This allows for random access to data, meaning any memory location can be read from or written to at any time.

        \item[\textcolor{Green3}{\faIcon{question}}] \textcolor{Green3}{\textbf{Challenges}}. \textbf{Aliasing} (uncertainty about whether two pointers refer to the same memory location) can cause problems when converting serial code to parallel code.
    \end{itemize}
    \begin{examplebox}[: Random Read and Write Pattern]
        \begin{lstlisting}[language=c++]
#include <iostream>

using namespace std;

int main() {
    int array[10];
    
    // Write to random positions
    array[3] = 5;
    array[7] = 10;
    
    // Read from random positions
    cout << "Value at position 3: " << array[3] << endl;
    cout << "Value at position 7: " << array[7] << endl;
    
    return 0;
}\end{lstlisting}
    \end{examplebox}


    \newpage
    \item \definition{Stack Allocation Pattern}
    \begin{itemize}
        \item[\textcolor{Red2}{\faIcon{book}}] \textcolor{Red2}{\textbf{Definition}}. \textbf{Stack allocation is used for dynamically allocating data} in a Last-In-First-Out (LIFO) manner. This is very efficient because allocation and deallocation can be done in constant time.

        \item[\textcolor{Green3}{\faIcon{check}}] \textcolor{Green3}{\textbf{Benefits}}. \textbf{Preserves locality, making data access faster}. In parallel programming, each thread usually has its own stack, preserving thread locality.
    \end{itemize}
    \begin{examplebox}[: Stack Allocation Pattern]
        \begin{lstlisting}[language=c++]
#include <iostream>

using namespace std;

void functionB() {
    int b = 10; // Stack allocation
    cout << "Function B, value of b: " << b << endl;
}

void functionA() {
    int a = 5; // Stack allocation
    cout << "Function A, value of a: " << a << endl;
    functionB();
}

int main() {
    functionA();
    return 0;
}\end{lstlisting}
    \end{examplebox}


    \item \definition{Heap Allocation Pattern}
    \begin{itemize}
        \item[\textcolor{Red2}{\faIcon{book}}] \textcolor{Red2}{\textbf{Definition}}. \textbf{Used when data cannot be allocated in a LIFO manner}. Heap allocation allows for dynamic memory allocation at any time.

        \item[\textcolor{Green3}{\faIcon{question}}] \textcolor{Green3}{\textbf{Challenges}}. \textbf{Slower and more complex than stack allocation}. In parallel programming, a parallelized heap allocator should be used to keep separate pools for each parallel worker.
    \end{itemize}
    \begin{examplebox}
        \begin{lstlisting}[language=c++]
#include <iostream>

using namespace std;

int main() {
    int* ptr = new int; // Heap allocation
    *ptr = 10;
    cout << "Value in heap: " << *ptr << endl;
    delete ptr; // Free the allocated memory
    return 0;
}\end{lstlisting}
    \end{examplebox}


    \newpage
    \item \definition{Objects Pattern}
    \begin{itemize}
        \item[\textcolor{Red2}{\faIcon{book}}] \textcolor{Red2}{\textbf{Definition}}. \textbf{Objects are language constructs that associate data with code to manipulate and manage that data}. They can have member functions and belong to a class of objects.

        \item[\textcolor{Green3}{\faIcon{question}}] \textcolor{Green3}{\textbf{Usage}}. In parallel programming, objects are often generalized in various ways to support concurrent execution.
    \end{itemize}
    \begin{examplebox}
        \begin{lstlisting}[language=c++]
#include <iostream>

using namespace std;

class MyObject {
public:
    int value;
    
    void display() {
        cout << "Value: " << value << endl;
    }
};

int main() {
    MyObject obj; // Stack allocation of an object
    obj.value = 15;
    obj.display();
    
    MyObject* objPtr = new MyObject(); // Heap allocation of an object
    objPtr->value = 20;
    objPtr->display();
    delete objPtr; // Free the allocated memory
    
    return 0;
}\end{lstlisting}
    \end{examplebox}
\end{enumerate}