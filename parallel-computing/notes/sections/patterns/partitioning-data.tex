\subsection{Partitioning Data}

\definition{Partitioning} is a \textbf{strategy used to parallelize computations by dividing data into independent, non-overlapping sections}. This allows different processing units (threads or cores) to work independently, avoiding conflicts and improving performance.

\highspace
\begin{flushleft}
    \textcolor{Green3}{\faIcon{tools} \textbf{How does partitioning work?}}
\end{flushleft}
\begin{enumerate}
    \item \important{Divide the Computational Domain}. Data is divided into non-over-\break lapping, equal-sized regions.
    
    \textbf{Importance of Non-Overlapping Regions}: \hl{Prevents race conditions} (simultaneous writes to the same memory); \hl{avoids unnecessary synchronization overhead}.


    \item \important{Work on Sections Individually (Parallel Execution)}. \textbf{Each section is processed independently by a separate thread or core}. Synchronization may be needed if sections share dependencies. Techniques include:
    \begin{itemize}
        \item Divide-and-Conquer (e.g., Merge Sort)
        \item Fork-Join Model (parallel task execution)
        \item Map-Reduce (parallel data processing)
    \end{itemize}


    \item \important{Combine the Results}. The computed results from each section are \textbf{merged into a final output}. Strategies depend on the problem:
    \begin{itemize}
        \item Reduction $\rightarrow$ Summing partial results (e.g., parallel prefix sum).
        \item Gathering $\rightarrow$ Collecting elements into a final array.
        \item Merging $\rightarrow$ Combining sorted sections (e.g., parallel merge sort).
    \end{itemize}
\end{enumerate}
To summarize:
\begin{enumerate}
    \item \important{Divide}: Break the data into chunks. 
    \item \important{Process}: Compute in parallel, ensuring minimal conflicts.
    \item \important{Combine}: Merge results efficiently.
\end{enumerate}

\highspace
\begin{flushleft}
    \textcolor{Green3}{\faIcon{question-circle} \textbf{Why is Partitioning important?}}
\end{flushleft}
\begin{itemize}[label=\textcolor{Green3}{\faIcon{check}}]
    \item \textcolor{Green3}{\textbf{Efficient Load Balancing}}: ensures each processing unit gets an equal workload.
    \item \textcolor{Green3}{\textbf{Avoids Conflicts}}: prevents threads from accessing the same memory location.
    \item \textcolor{Green3}{\textbf{Scalability}}: allows parallel execution without excessive synchronization overhead.
\end{itemize}
