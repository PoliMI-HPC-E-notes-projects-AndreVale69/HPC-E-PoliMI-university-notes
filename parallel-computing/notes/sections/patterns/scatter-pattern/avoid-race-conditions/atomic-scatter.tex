\subsubsection{Avoid race conditions}

\paragraph{Atomic Scatter}

\definition{Atomic Scatter} is a \textbf{collision resolution strategy} used when multiple threads attempt to write to the same memory location simultaneously. It is based on one of the most common and famous topic: \textbf{atomic operations}.

\highspace
\begin{flushleft}
    \textcolor{Green3}{\faIcon{question-circle} \textbf{How does it work?}}
\end{flushleft}
\begin{itemize}
    \item[\textcolor{Green3}{\faIcon{check}}] \textcolor{Green3}{\textbf{Atomic Writes}}. \textbf{Each write is atomic}, meaning it either \hl{completes fully or doesn't happen at all}. No partial writes.

    \item[\textcolor{Red2}{\faIcon{times}}] \textcolor{Red2}{\textbf{Non-Deterministic Outcome}}. Since it relies solely on atomic writes, this doesn't mean that there is a mechanism to check which thread should write before another. \hl{When a collision occurs, the outcome depends on which write completes first}. There's no predefined rule for determining which value to store.
\end{itemize}

\highspace
\begin{examplebox}[: Atomic Scatter]
    Consider this:
    \begin{itemize}
        \item Source Data:
        \begin{equation*}
            \texttt{A = [A, B, C, D, E, F]}
        \end{equation*}
        \item Index Array (write locations):
        \begin{equation*}
            \texttt{idx = [1, 5, 0, 2, 2, 4]}
        \end{equation*}
    \end{itemize}
    Notice that each thread is assigned to every position of the \texttt{idx} array. This means that threads \#4 and \#5 must write to the same position on the output array. However, we adopt the atomic strategies so that what happens depends on the order of writing.

    \highspace
    A possibles scenario:
    \begin{enumerate}
        \item Thread \#4 asks to do an atomic write to position 2 of the output array. The operation is marked atomic and can be done safely. The value written to \texttt{A[2]} is \texttt{D}.
        \item Then thread \#5 performs the same steps and overwrites the value written by thread \#4 and writes the value \texttt{E} to \texttt{A[2]}.
    \end{enumerate}
    This is non-deterministic, so this is a possible scenario, but it could also be the other way around and the final value could be \texttt{D}.
\end{examplebox}

\newpage

\begin{flushleft}
    \textcolor{Green3}{\faIcon{check-circle} \textbf{Advantages}}
\end{flushleft}
\begin{itemize}[label=\textcolor{Green3}{\faIcon{check}}]
    \item \textcolor{Green3}{\textbf{Simple to implement}}. No need for complex locking mechanisms.
    \item \textcolor{Green3}{\textbf{Fast}}. Atomic \hl{operations are optimized in hardware}, making them efficient for parallel execution.
\end{itemize}

\highspace
\begin{flushleft}
    \textcolor{Red2}{\faIcon{times-circle} \textbf{Disadvantages}}
\end{flushleft}
\begin{itemize}[label=\textcolor{Red2}{\faIcon{times}}]
    \item \textcolor{Red2}{\textbf{Unpredictable results}}. Non-determinism can be \hl{problematic for algorithms that require consistent outputs}.
    \item \textcolor{Red2}{\textbf{Data loss}}. \hl{One of the colliding values will be lost} without any mechanism to combine or preserve both.
\end{itemize}

\highspace
\begin{flushleft}
    \textcolor{Green3}{\faIcon{question-circle} \textbf{When to use Atomic Scatter?}}
\end{flushleft}
Atomic scatter is suitable when:
\begin{itemize}
    \item \textbf{Collisions are rare}: so the occasional data loss or non-determinism is acceptable.
    \item \textbf{We don't care which value is kept}: for example, in random sampling or approximate algorithms.
\end{itemize}