\subsection{Stencil Pattern}\label{subsection: Stencil Pattern}

\subsubsection{What is a Stencil?}

A \definition{Stencil Pattern} is a computational pattern used in parallel computing where \textbf{each point in a grid (or data structure) is updated based on a fixed set of neighboring points}. It's commonly used in scientific computing, numerical simulations, and image processing.

\highspace
\begin{flushleft}
    \textcolor{Green3}{\faIcon{tools} \textbf{How the Stencil works}}
\end{flushleft}
\begin{enumerate}
    \item \important{Define a Grid}. The data is usually structured in a 1D, 2D, or 3D grid (e.g., an image, matrix, or mesh).
    
    \item \important{Select a Stencil shape}. We choose a pattern defining how each point depends on its neighbors. Common shapes are:
    \begin{itemize}
        \item 1D: \texttt{[-1, 0, +1]} (left, center, right).
        \item 2D: \texttt{+} (cross) or \texttt{x} (diagonal) patterns.
        \item 3D: cube-based neighborhoods.
    \end{itemize}

    \item \important{Apply the Stencil to each grid point}. For each element, compute the new value based on its neighbors. For example, the average of neighboring values.

    \item (optional) \important{Handle Boundaries}. Ghost cells, mirroring, or skipping edges to prevent out-of-bounds errors.

    \item \important{Iterate (Time-Stepping, if needed)}. Some stencil computations run for multiple iterations, updating the grid repeatedly.
\end{enumerate}

\begin{examplebox}[: How Convolution Uses the Stencil Pattern]
    \begin{enumerate}
        \item \textbf{Grid Representation}. An image is a 2D grid of pixel values.
        \item \textbf{Stencil Shape}. A small matrix called kernel (or filter) moves over the image. A common kernel can be blur, or edge detection, or sharpening.
        \item \textbf{Applying the Stencil}. Each pixel's new value is computed using a weighted sum of its neighbors.
        \item \textbf{Boundary Handling}. Pixels on the edges require special treatment (padding, mirroring, or ignoring).
    \end{enumerate}
    For example, we can use the following kernel:
    \begin{equation*}
        \dfrac{1}{9} \begin{bmatrix}
            1 & 1 & 1 \\ 
            1 & 1 & 1 \\ 
            1 & 1 & 1 
        \end{bmatrix}
    \end{equation*}
    Where each pixel of the image is replaced by the average of itself and its 8 neighbors.
\end{examplebox}
