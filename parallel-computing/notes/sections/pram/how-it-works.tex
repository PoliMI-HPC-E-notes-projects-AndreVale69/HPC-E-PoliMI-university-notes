\subsection{How it works}

\subsubsection{Computation}

A single \textbf{processor} of the PRAM, at each computation, is \textbf{composed of 5 phases} (carried out in parallel by all the processors):
\begin{enumerate}
    \item \textbf{Reads a value from one of the cells} $X\left(1\right), \dots, X\left(N\right)$
    \item Reads one of the shared memory cells $A\left(1\right), A\left(2\right), \dots$
    \item Performs some internal computation
    \item \textbf{May write into one of the output cells} $Y\left(1\right), Y\left(2\right), \dots$
    \item May write into one of the shared memory cells $A\left(1\right), A\left(2\right), \dots$
\end{enumerate}

\longline

\subsubsection{PRAM Classificiation}

During execution, a subset of processors may remain idle. Also, some processors can read from the same cell at the same time (not really a problem), but they could also try to write to the same cell at the same time (\textbf{write conflict}). For these reasons, PRAMs are classified according to their read/write capabilities (realistic and useful):
\begin{itemize}
    \item \definition{Exclusive Read (ER)}. All processors can simultaneously read from distinct memory locations.
    
    \item \definition{Exclusive Write (EW)}. All processors can simultaneously write to distinct memory locations.
    
    \item \definition{Concurrent Read (CR)}. All processors can simultaneously read from any memory location.

    \item \definition{Concurrent Write (CW)}. All processors can write to any memory location.
    \begin{flushleft}
        \textcolor{Green3}{\faIcon{question-circle} \textbf{But what value is ultimately written?}}
    \end{flushleft}
    It depends on the mode we choose:
    \begin{itemize}
        \item \definition{Priority Concurrent Write}. Processors have priority based on which value is decided, the \textbf{highest priority is allowed to complete write}.

        \item \definition{Common Concurrent Write}. All processors are allowed to complete write \textbf{if and only if all the value to be written are equal}.
        
        Any \textbf{algorithm} for this model has to \textbf{make sure that this condition is satisfied}. \textbf{Otherwise}, the \textbf{algorithm is illegal} and the \textbf{machine state will be undefined}.

        \item \definitionWithSpecificIndex{Arbitrary/Random Concurrent Write}{Arbitrary Concurrent Write}{Random Concurrent Write}. One \textbf{randomly} chosen \textbf{processor is allowed to complete write}.
    \end{itemize}
\end{itemize}

\subsubsection{Strengths of PRAM}

PRAM is attractive and important model for designers of parallel algorithms because:
\begin{itemize}
    \item It is \textbf{natural}. The number of operations executed per one cycle on $P$ processors is at most $P$ (equal to $P$ is the ideal case).

    \item It is \textbf{strong}. Any processor can read/write any shared memory cell in unit time.

    \item It is \textbf{simple}. It abstracts from any communication or synchronization overhead, which makes the complexity and correctness of PRAM algorithm easier.

    \item It can be used as a \textbf{benchmark}. If a problem has no feasible/efficient solution on PRAM, it has no feasible/efficient solution for any parallel machine.
\end{itemize}

\longline

\subsubsection{How to compare PRAM models}

Consider two generic PRAMs, models $A$ and $B$. Model $A$ is \textbf{computationally stronger} than model $B$ ($A \ge B$) \textbf{if and only if} \textbf{any algorithm} written for model $B$ will \textbf{run unchanged} on model $A$ in the \textbf{same parallel time} and with the \textbf{same basic properties}.

\highspace
However, there are some useful metrics that can be used to compare models:
\begin{itemize}
    \item \textbf{Time to solve problem} of \textbf{input size} $n$ on \textbf{\underline{one} processor}, using \textbf{best \underline{sequential} algorithm}:
    \begin{equation}
        T^{*}\left(n\right)
    \end{equation}

    \item \textbf{Time} to solve problem of input size $n$ on \underline{$\mathbf{p}$} \textbf{processors}:
    \begin{equation}
        T_{p}\left(n\right)
    \end{equation}

    \item \textbf{Speedup on $\mathbf{p}$ processors}:
    \begin{equation}
        \mathrm{SU}_{p}\left(n\right) = \dfrac{T^{*}\left(n\right)}{T_{p}\left(n\right)}
    \end{equation}

    \item \textbf{Efficiency}, which is the work done by a processor to solve a problem of input size $n$ divided by the work done by $p$ processors:
    \begin{equation}
        E_{p}\left(n\right) = \dfrac{T_{1}\left(n\right)}{pT_{p}\left(n\right)}
    \end{equation}

    \item \textbf{Shortest run time} on any process $p$:
    \begin{equation}
        T_{\infty}\left(n\right)
    \end{equation}

    \item \textbf{Cost}, equal to processors and time:
    \begin{equation}
        C\left(n\right) = P\left(n\right) \cdot T\left(n\right)
    \end{equation}

    \item \textbf{Work}, equal to the total \textbf{number of operations}:
    \begin{equation}
        W\left(n\right)
    \end{equation}
\end{itemize}
Some properties on the metrics:
\begin{itemize}
    \item The time to solve a problem of input $n$ on a single processor using the best sequential algorithm \emph{is not equal to} the time to solve a problem of input $n$ in parallel using one of the $p$ processors available. In other words, \textbf{the problem should not be solvable on a single processor on a parallel machine} (otherwise, what would be the point of using a parallel model?)
    \begin{equation*}
        T^{*} \ne T_{1}
    \end{equation*}

    \item $\mathrm{SU}_{P} \le P$

    \item $\mathrm{SU}_{P} \le \dfrac{T_{1}}{T_{\infty}}$

    \item $E_{p} \le 1$

    \item $T_{1} \ge T^{*} \ge T_{p} \ge T_{\infty}$

    \item $T^{*} \approx T_{1} \Rightarrow E_{p} \approx \dfrac{T^{*}}{pT_{p}} = \dfrac{\mathrm{SU}_{p}}{p}$

    \item $E_{p} = \dfrac{T_{1}}{pT_{p}} \le \dfrac{T_{1}}{pT_{\infty}}$

    \item $T_{1} \in O\left(C\right),  T_{p} \in O\left(\dfrac{C}{p}\right)$

    \item $W \le C$

    \item $p \approx \text{AREA} \hspace{1em} W \approx \text{ENERGY} \hspace{1em} \dfrac{W}{T_{p}} \approx \text{POWER}$
\end{itemize}