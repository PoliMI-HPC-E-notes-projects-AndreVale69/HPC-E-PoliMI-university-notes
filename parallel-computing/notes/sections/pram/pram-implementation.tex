\subsection{PRAM implementation}

The PRAM is an ideal model for creating parallel algorithms. Now we look at \dquotes{\emph{is it really implementable?}} The short answer is yes.

\highspace
The longest answer is the following. There are already some examples of PRAM being converted to real machine models, such as \href{https://en.wikipedia.org/wiki/Explicit_multi-threading}{Explicit Multi-Threading (XMT)}, Rigel, Tilera, etc. If conversion is not easy or possible, the implementation can be \dquotes{\emph{direct}}:
\begin{itemize}
    \item The concurrent read is implemented as a detect-and-multicast technique.
    \item The concurrent write is implemented depending on the end result we want to achieve. Fetch-and-operate and prefix-sum are examples of serialized writing; otherwise, the CRCW technique is used:
    \begin{itemize}
        \item Common CRCW: detect and merge
        \item Priority CRCW: detect-and-priorities
        \item Arbitrary CRCW: arbitrary
    \end{itemize}
\end{itemize}

\begin{examplebox}[: Boolean DNF (sum of products) common CRCW]
    A logical formula is considered to be in DNF if it is a disjunction of one or more conjunctions of one or more literals.

    Consider $X$ as the sum of products of AND/OR operations:
    \begin{equation*}
        X = a_{1}b_{1} + a_{2}b_{2} + \dots
    \end{equation*}
    The PRAM code, with $X$ initialized to $0$ and task index equal to \$, is:
    \begin{center}
        \texttt{if ($a_{\$}b_{\$}$) X = 1;}
    \end{center}
    The common result is that not all processors write $X$ and those that do write 1. The time complexity is $O\left(1\right)$. It works on common, priority and arbitrary CRCW.
\end{examplebox}

\noindent
Despite the previous example, exists also the PRAM SoP for the concurrent write. Let boolean $X$ as:
\begin{equation*}
    X = a_{1}b_{1} + a_{2}b_{2} + \dots
\end{equation*}
The PRAM algorithm is:
\begin{center}
    \texttt{if ($a_{i}b_{i}$) X = 1;}
\end{center}
Where all cores which write into $X$, \textbf{write the same value}.

\newpage

\begin{flushleft}
    \textcolor{Green3}{\faIcon{check} \textbf{PRAM advantages}}
\end{flushleft}
\begin{itemize}
    \item Large body of algorithms.
    \item Easy to think about it.
    \item Sync version of shared memory. It eliminates sync and common issues, allows focus on algorithms, but allows adding these issues and allows conversion to async versions.
    \item Exists architectures for both synch (PRAM) and async (SM) model.
    \item PRAM algorithms can be mapped to other models.
\end{itemize}