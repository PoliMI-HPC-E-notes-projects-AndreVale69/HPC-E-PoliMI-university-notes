\subsection{PRAM variants and Lemmas}

The PRAM model presented here is one of the most commonly used. However, there are other important variants:
\begin{itemize}
    \item PRAM model with a \textbf{limited number of shared memory cells} (small memory PRAM). If the input data set exceeds the capacity of the shared memory, the I/O values can be evenly distributed among the processors.
    
    \item PRAM model with \textbf{limited number of processors} (small PRAM). If the number of execution threads is higher, processors can interleave multiple threads.

    \item PRAM model with \textbf{limited size of one machine word}.
    
    \item PRAM model with \textbf{access conflicts handling}. These are restrictions on simultaneous access to shared memory cells.
\end{itemize}

\begin{lemma}
    Assume $P' < P$ and same size of shared memory. Any problem that can be solved for a $P$ processor PRAM in $T$ steps can be solved in a $P'$ processor PRAM in:
    \begin{equation}
        T' = O\left(\dfrac{TP}{P'}\right)
    \end{equation}
\end{lemma}
\begin{proof}
    Partition $P$ is simulated processors into $P'$ groups of size $\frac{P}{P'}$ each. Associate each of the $P'$ simulating processors with one of these groups. Each of the simulating processors simulates one step of its group of processors by:
    \begin{itemize}
        \item Executing all their read and local computation substeps first;
        \item Executing their write substeps then.
    \end{itemize}
\end{proof}

\highspace
\begin{lemma}
    Assume $M' < M$. Any problem that can be solved for a $P$ processor and $M$-cell PRAM in $T$ steps can be solved on a $\max\left(P, M'\right)$-processors $M'$-cell PRAM in $O\left(\dfrac{TM}{M'}\right)$ steps.
\end{lemma}
\begin{proof}
    Partition $M$ simulated shared memory cells into $M'$ continuous segments $S$, of size $\frac{M}{M'}$ each. Each simulating processor $P_{I}'$ ($1 \le I \le P$), will simulate processor $P_{I}$ of the original PRAM. Each simulating processor $P_{I}'$ ($1 \le I \le M'$), stores the initial contents of $S_{I}$ into its local memory and will use $M'\left[I\right]$ as an auxiliary memory cell for simulation of accesses to cell of $S_{I}$.

    \noindent
    Simulation of one original read operation:
    \begin{lstlisting}[mathescape=true]
EACH $P_{I}'$ $\left(I = 1, \dots, \max\left(P, M'\right)\right)$ REPEATS FOR K = 1, ..., $\frac{M}{M'}$
    WRITE THE VALUE OF THE K-TH CELL OF $S_{I}$ INTO $M'\left[I\right]$ $\left(I = 1, \dots, M'\right)$
    READ THE VALUE WHICH THE SIMULATED PROCESSOR $P_{I}$ $\left(I = 1, \dots, P\right)$ WOULD READ IN THIS SIMULATED SUBSTEP, IF IT APPEARED IN THE SHARED MEMORY\end{lstlisting}
    The local computation substep of $P_{I}$ ($I = 1, \dots, P$) is simulated in one step by $P_{I}'$. SImulation of one original write operation is analogous to that of read.
\end{proof}
