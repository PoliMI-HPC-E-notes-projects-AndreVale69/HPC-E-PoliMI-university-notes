\subsection{Message passing model of communication}

In parallel computing, the \definition{message passing model} \textbf{allows} threads (or processes) to \textbf{communicate by sending and receiving messages}. This model is used to \textbf{facilitate data exchange between threads running in their own private address spaces}.

\highspace
Each thread operates within its private address space, meaning they \textbf{do not share memory directly}. When they need to communicate with a specific thread without using a \emph{shared address space model}, they use the \emph{message passing model} to send (or receive) the data.
\begin{itemize}
    \item Usually, the \textbf{sender} specifies
    \begin{itemize}
        \item the recipient;
        \item the buffer to be transmitted;
        \item an optional message identifier (a sort of tag).
    \end{itemize}
    \item Meanwhile, the \textbf{receiver} specifies:
    \begin{itemize}
        \item who the sender is;
        \item the buffer where to store data;
        \item an optional message identifier (again, a sort of tag).
    \end{itemize}
\end{itemize}
The message passing model is the \textbf{only way to exchange data between threads} because it guarantees three main advantages:
\begin{itemize}[label=\textcolor{Green3}{\faIcon{check-circle}}]
    \item \textbf{\underline{Data Isolation}}: Using message passing, \textbf{each thread maintains its address space}, reducing the risk of race conditions.

    \item \textbf{\underline{Scalability}}: Message passing \textbf{scales well with distributed systems and multi-core architectures}, making it suitable for large-scale parallel computing.
    
    \item \textbf{\underline{Flexibility}}: It \textbf{allows for explicit control over data exchange and synchronization, providing Flexibility in parallel program design}. Explicit control refers to the fact that in message passing, each communication action is explicitly defined by send and receive operations. This means we can precisely dictate which data is sent, when, and to whom it is sent. Furthermore, the synchronization is managed very well because message passing naturally incorporates synchronization. When a process sends a message, it can block until it is received, ensuring that the sender and receiver are synchronized.
\end{itemize}

\begin{flushleft}
    \textcolor{Green3}{\faIcon{question-circle} \textbf{Why message passing is preferable to the shared address space model}}
\end{flushleft}
Unlike shared memory systems that require complex hardware mechanisms to implement system-wide load and store operations, message passing systems do not need this capability. They \textbf{only need to communicate messages between nodes}. It can be considered as a great advantage for message passing models, which for this reason is very much used in the supercomputers and clusters. Finally, this model has the ability to \textbf{connect commodity hardware systems together to form a large parallel machine}. This means we can use off-the-shelf components to build powerful computing clusters.