\section{Introduction}

\subsection{Complex Numbers recap}

Complex Numbers play a fundamental role in quantum mechanics and quantum computing. For this reason, here is a brief summary of the most important concepts:
\begin{itemize}
    \item \important{Definition of a Complex Number}. A complex number $z$ is written as:
    \begin{equation*}
        z = x + iy
    \end{equation*}
    Where:
    \begin{itemize}
        \item $x$ is the \textbf{real part} ($\text{Re}(z) = x$).
        \item $y$ is the \textbf{imaginary part} ($\text{Im}(z) = y$).
        \item $i$ is the \textbf{imaginary unit}, satisfying $i^{2} = -1$.
    \end{itemize}
    A complex number can also be expressed in \textbf{polar form}:
    \begin{equation*}
        z = re^{i \varphi}
    \end{equation*}
    Where:
    \begin{itemize}
        \item $r = \left|z\right| = \sqrt{x^{2} + y^{2}}$ is the \definitionWithSpecificIndex{modulus}{Complex Numbers: Modulus}{} (also called \definitionWithSpecificIndex{magnitude}{Complex Numbers: Magnitude}{}).
        \item $\varphi = \arg(z) = \tan^{-1} \left( \frac{y}{x} \right)$ is the \definitionWithSpecificIndex{argument}{Complex Numbers: Argument}{} (also called \definitionWithSpecificIndex{phase angle}{Complex Numbers: Phase Angle}{}).
    \end{itemize}
    Using Euler's formula:
    \begin{equation*}
        e^{i \varphi} = \cos \varphi + i \sin \varphi
    \end{equation*}
    We can rewrite $z$ as:
    \begin{equation*}
        z = r \left(\cos \varphi + i \sin \varphi\right)
    \end{equation*}

    \item \important{Complex Conjugate}. The \definition{Complex Conjugate} of $z$ is:
    \begin{equation*}
        \bar{z} = x - iy = r e^{-i \varphi}
    \end{equation*}
    Properties:
    \begin{itemize}
        \item $ z \cdot \bar{z} = \left|z\right|^{2} = \left(\sqrt{x^{2} + y^{2}}\right)^{2} = x^{2} + y^{2} $
        \item The \textbf{conjugate reverses the sign of the imaginary part}.
    \end{itemize}

    \item \important{Operations on Complex Numbers}
    \begin{itemize}
        \item \textbf{Addition} and \textbf{Subtraction}:
        \begin{equation*}
            \begin{array}{rcl}
                \left(a + ib\right) + \left(c + id\right) &=& \left(a+c\right) + i\left(b+d\right) \\ [.5em]
                \left(a + ib\right) - \left(c + id\right) &=& \left(a-c\right) + i\left(b-d\right)
            \end{array}
        \end{equation*}
        \item \textbf{Multiplication}. Using the distributive property:
        \begin{equation*}
            \left(a + ib\right)\left(c + id\right) = ac + iad + ibc + i^2bd
        \end{equation*}
        Since $i^2 = -1$, we get:
        \begin{equation*}
            \left(ac - bd\right) + i\left(ad + bc\right)
        \end{equation*}
        \item \textbf{Division}. To divide $\dfrac{z_1}{z_2}$, multiply by the conjugate of the denominator:
        \begin{equation*}
            \frac{a + ib}{c + id} = \frac{(a+ib)(c-id)}{c^2 + d^2}
        \end{equation*}
        Expanding:
        \begin{equation*}
            \frac{(ac + bd) + i(bc - ad)}{c^2 + d^2}
        \end{equation*}
    \end{itemize}

    \item \important{Complex Exponentiation}. Using Euler's formula:
    \begin{equation*}
        e^{i\theta} = \cos \theta + i \sin \theta
    \end{equation*}
    For integer powers:
    \begin{equation*}
        (e^{i\theta})^n = e^{i \cdot n \cdot \theta}
    \end{equation*}
    For fractional exponents (roots):
    \begin{equation*}
        z^{\frac{1}{n}} = r^{\frac{1}{n}} e^{\frac{i\left(\varphi+2\pi k\right)}{n}}, \quad k = 0, 1, \dots, n-1
    \end{equation*}

    \item \important{Rotation Using Complex Numbers}. Multiplying by $e^{i\psi}$ rotates a complex number by an angle $\psi$:
    \begin{equation*}
        z' = z e^{i\psi}
    \end{equation*}
    Since:
    \begin{equation*}
        e^{i\psi} = \cos \psi + i \sin \psi
    \end{equation*}
    This means:
    \begin{equation*}
        \underbrace{\left(x + iy\right)}_{z}\underbrace{\left(\cos \psi + i \sin \psi\right)}_{e^{i\psi}}
    \end{equation*}
    Expanding:
    \begin{equation*}
        (x\cos \psi - y\sin \psi) + i (x\sin \psi + y\cos \psi)  
    \end{equation*}
    Thus, the new coordinates are:
    \begin{equation*}
        x' = x\cos \psi - y\sin \psi, \quad y' = x\sin \psi + y\cos \psi    
    \end{equation*}
    Which is a standard \definition{2D rotation matrix}:
    \begin{equation*}
        \begin{bmatrix}
            \bar{x} \\ \bar{y}
        \end{bmatrix}
        =
        \begin{bmatrix}
            \cos\psi & -\sin\psi \\
            \sin\psi & \cos\psi
        \end{bmatrix}
        \begin{bmatrix}
            x \\ y
        \end{bmatrix}
    \end{equation*}
    And it rotates a point counterclockwise by an angle $\psi$ in the 2D plane. This is important because rotations in the Bloch sphere (which represents qubits) are described by operations similar to this matrix.
    
    \item \important{Hermitian (Conjugate Transpose) of a Vector}. For a vector of complex numbers:
    \begin{equation*}
        \mathbf{z} = \begin{bmatrix} a \\ b \end{bmatrix}
    \end{equation*}
    The \definition{Hermitian conjugate} (denoted $\mathbf{z}^{\dagger}$ or $\mathbf{z}^{H}$) is:
    \begin{equation*}
        \mathbf{z}^H = \begin{bmatrix} \bar{a} & \bar{b} \end{bmatrix}  
    \end{equation*}
    Where $\bar{a}$ and $\bar{b}$ are the complex conjugates.
\end{itemize}
These concepts are fundamental because complex numbers describe quantum states. Also, Euler's formula provides a powerful tool for representing phase shifts. Finally, rotation and multiplication are key to quantum operations, and the hermitian conjugate is crucial in quantum mechanics.