\subsection{Dirac's Notation}

\definition{Dirac Notation}, also called \definitionWithSpecificIndex{bra-ket notation}{Bra-Ket Notation}{}, is a powerful mathematical framework used in quantum mechanics to \textbf{describe quantum states and their transformations}.

\highspace
\begin{flushleft}
    \textcolor{Green3}{\faIcon{question-circle} \textbf{What is a Ket?}}
\end{flushleft}
A \definition{Ket} $\quantumState{v}$ (is equal to the linear algebra annotation $\overrightarrow{v}$) is a \textbf{column vector} in a Hilbert space, that \textbf{represents a quantum state}.
\begin{equation*}
    \quantumState{v} = \begin{bmatrix}
        a \\ b
    \end{bmatrix}
\end{equation*}
Where $a$ and $b$ are complex numbers (amplitudes of the quantum state). In other words, a ket $\quantumState{v}$ is a \textbf{state vector}, and what we usually require is that it has \textbf{unit norm}, meaning:
\begin{equation*}
    \langle v | v \rangle = 1
\end{equation*}
This ensures that the total probability of measurement outcomes is 1. So, a ket is a normalized column vector in a Hilbert space, meaning it has unit norm.

\highspace
\begin{flushleft}
    \textcolor{Green3}{\faIcon{question-circle} \textbf{What is a Superposition and why is it related to the Ket?}}
\end{flushleft}
\definition{Superposition} is a fundamental principle of quantum mechanics that \textbf{applies to all quantum systems}, not just qubits.
\begin{definitionbox}[: Superposition]
    \textbf{Superposition} means that a \textbf{system can exist in several possible states at the same time until a measurement is made}.
\end{definitionbox}

\highspace
More in general, a quantum state $\quantumState{\psi}$ in a system with multiple possible states can exist in a \textbf{linear combination} of these states:
\begin{equation*}
    \quantumState{\psi} = c_{1} \quantumState{\psi_{1}} + c_{2} \quantumState{\psi_{2}} + \dots + c_{n} \quantumState{\psi_{n}}    
\end{equation*}
Where:
\begin{itemize}
    \item $\quantumState{\psi_{1}}, \quantumState{\psi_{2}}, \dots, \quantumState{\psi_{n}}$ are \textbf{basis states} of the system.
    
    \item $c_{1}, c_{2}, \dots, c_{n}$ are \textbf{complex probability amplitudes}.

    \item The system is in \textbf{all states at the same time}, with the probability of measuring each state given by $\left|c_{i}\right|^{2}$.

    \item The state is \textbf{normalized}, meaning:
    \begin{equation*}
        \left|c_{1}\right|^{2} + \left|c_{2}\right|^{2} + \dots + \left|c_{n}\right|^{2} = 1
    \end{equation*}
\end{itemize}

\newpage

\begin{examplebox}[: Single-Particle Systems (Quantum Mechanics)]
    In general quantum mechanics, a particle can exist in \textbf{multiple positions simultaneously} as a wave function $\Psi(x)$:
    \begin{equation*}
        \quantumState{\Psi} = \displaystyle\int \Psi(x) \: \quantumState{x} \: \mathrm{d}x  
    \end{equation*}
    \begin{itemize}
      \item The particle is in a \textbf{superposition of all possible positions} $\quantumState{x}$.
      \item Measurement collapses the wave function to a single position.
    \end{itemize}
\end{examplebox}

\noindent
\textcolor{Green3}{\faIcon{question-circle} \textbf{How Can a Particle Exist in Multiple States at Once?}} This question touches the heart of quantum mechanics, where our everyday intuition breaks down. The \textbf{key idea} is that a \textbf{quantum particle} is not just a tiny ball, it \textbf{is a wave function that spreads across multiple possibilities at once}.
\begin{enumerate}
  \item \important{Quantum Particles Are Waves, Not Just Points}. In classical physics, we think of particles as tiny, solid objects, like a small ball that always has a precise position and velocity.
  
  In quantum mechanics, however, particles behave more like waves. These waves are described by a wave function $\Psi(x)$, which represents the \textbf{probability of finding the particle at different locations}.
  
  The key idea is:
  \begin{itemize}
    \item The \textbf{wave function spreads across space}, meaning the \textbf{particle does not have a single location} before measurement.
    \item Instead, it exists in a superposition of all possible locations.
  \end{itemize}


  \item \important{The Double-Slit Experiment: Proof That a Particle Can Be in Two Places at Once}. The double-slit experiment demonstrates that light and matter can exhibit behavior of both classical particles and classical waves. In 1927, Davisson and Germer and, independently, George Paget Thomson and his research student Alexander Reid demonstrated that electrons show the same behavior, which was later extended to atoms and molecules.
  \begin{center}
    \href{https://youtu.be/A9tKncAdlHQ}{\emph{Double Slit Experiment explained! by Jim Al-Khalili}}\hspace{2em}\qrcode{https://youtu.be/A9tKncAdlHQ}
  \end{center}

  \begin{itemize}[label=\textcolor{Green3}{\faIcon{question-circle}}]
    \item \textcolor{Green3}{\textbf{What Happens in Classical Physics?}} If we throw tiny balls at a screen with two slits, each ball will pass through one slit or the other. After many throws, \textbf{we get two lines} behind the slits, corresponding to the \textbf{two possible paths}.

    \item \textcolor{Green3}{\textbf{What Happens in Quantum Mechanics?}} If we \textbf{send a single electron} (or photon) towards two slits, it \textbf{behaves like a wave}.

    It \textbf{passes through both slits at the same time} and interferes with itself, creating an interference pattern. This means the electron was in a \textbf{superposition of passing through both slits at once}.

    If we try to measure which slit the electron goes through, the \textbf{superposition collapses}, and it behaves like a classical particle!
  \end{itemize}


  \item \important{Quantum Superposition: More Than Just Probability}. A common misconception is that a particle in superposition is just \textbf{an unknown state}, like a coin that is either heads or tails, but we just don't know which. This is wrong, because quantum superposition is much deeper.

  A quantum state is a \textbf{combination of all possibilities}. \emph{Until measurement}, the system is in \textbf{all possible states at once}.
  
  Mathematically, for an electron in \textbf{two locations} $x_{1}$ and $x_{2}$:
  \begin{equation*}
    \quantumState{\psi} = a \quantumState{x_{1}} + b \quantumState{x_{2}}
  \end{equation*}
  \begin{itemize}
    \item The electron is \textbf{literally in both places simultaneously}.
    \item The coefficients $a$ and $b$ are \textbf{complex numbers representing the probability amplitudes}.
    \item \textbf{Interference} between these amplitudes \textbf{creates quantum effects that cannot be explained by classical probability}.
  \end{itemize}


  \item \important{Superposition in Quantum Computing}. Quantum computing \textbf{directly uses} the fact that a particle can be in multiple states at once.

  \begin{itemize}
    \item A \textbf{classical bit} can only be \textbf{0 or 1}.
    \item A \textbf{qubit (quantum bit)} (we will explain this later) can be in a superposition:
    \begin{equation*}
      \quantumState{\psi} = a \quantumState{0} + b \quantumState{1}
    \end{equation*}
  \end{itemize}
  This means a quantum computer can \textbf{perform many calculations simultaneously}. Therefore, \textbf{superposition allows quantum computers to process information exponentially faster than classical computers for certain tasks}.

  
  \item \important{Why Don't We See Superposition in Everyday Life?} In our daily experience, objects are \textbf{not in multiple states at once} because of a process called \textbf{quantum decoherence}.

  Quantum superposition is \textbf{fragile}. When a quantum system interacts with the environment (air, light, etc.), the superposition \textbf{collapses into one definite state}. This is why large objects (like humans or cars) do not appear in multiple places at once.
  
  However, experiments (like the Double-Slit experiment) confirm that superposition is real at microscopic scales (electrons, photons, atoms, and even molecules).
\end{enumerate}

\newpage

\noindent
The key \textbf{properties of Superposition} are:
\begin{enumerate}
    \item \textbf{Linearity}: Any combination of valid quantum states is also a valid quantum state.
    \item \textbf{Interference}: Quantum states in superposition can interfere, leading to constructive or destructive interference.
    \item \textbf{Measurement Collapse}: When measured, the superposition collapses into a single outcome.
    \item \textbf{Phase Information}: Unlike classical probabilities, quantum superpositions include complex phases that affect interference patterns.
\end{enumerate}

\begin{flushleft}
    \textcolor{Green3}{\faIcon{question-circle} \textbf{What is a Bra?}}
\end{flushleft}
A \definition{Bra} $\bra{v}$ (is equal to the linear algebra annotation $\overrightarrow{v}^{H}$) is the \textbf{conjugate transpose (Hermitian conjugate) of the ket} $\quantumState{v}$.

\highspace
Mathematically, if we start with the ket $\quantumState{v}$:
\begin{equation*}
    \quantumState{v} = \begin{bmatrix}
        a \\ b
    \end{bmatrix}
\end{equation*}
The bra is obtained by:
\begin{enumerate}
    \item \textbf{Transposing the ket} (switching it from column to a row).
    \item \textbf{Taking the complex conjugate of each element}.
\end{enumerate}
So the bra $\bra{v}$ is:
\begin{equation*}
    \bra{v} = \left(\bra{v}\right)^{\dagger} = \begin{bmatrix}
        \bar{a} & \bar{b}
    \end{bmatrix}
\end{equation*}
A bra is a \textbf{mathematical object that allows us to compute inner products and measure probabilities}.

\highspace
\begin{flushleft}
    \textcolor{Green3}{\faIcon{question-circle} \textbf{But why do we need another topic called ``Bra''?}}
\end{flushleft}
A \textbf{bra} $\bra{v}$ does \hl{not represent a physical} system by itself. Instead, it is a \textbf{mathematical tool} used to:
\begin{enumerate}
    \item \textbf{Extract} information from a quantum state.
    \item \textbf{Compute} inner products (which determine probabilities).
    \item \textbf{Define} quantum operators and measurements.
\end{enumerate}
The key idea of bras and kets working together is:
\begin{itemize}
    \item A ket $\quantumState{v}$ represents a quantum system.
    \item A bra $\bra{v}$ is like a \emph{test function} that helps us extract measurable information from a quantum system.
\end{itemize}
When they are combined as $\langle v | v \rangle = 1$, we obtain a \textbf{probability amplitude}.

\newpage

\begin{flushleft}
    \textcolor{Green3}{\faIcon{book} \textbf{Dirac's Notation: Multiplications}}
\end{flushleft}
The following list shows how Dirac's notation is used to describe how kets and bras interact through inner products, matrix-vector multiplications, and operator applications:
\begin{itemize}
    \item \definitionWithSpecificIndex{Inner (Scalar) Product: $\langle x | y \rangle$}{Dirac's Notation: Inner (Scalar) Product}{}

    The \textbf{inner product} (or scalar product) between two quantum states $\quantumState{x}$ and $\quantumState{x}$ is written as:
    \begin{equation*}
        \langle x | y \rangle
    \end{equation*}
    Where:
    \begin{itemize}
        \item The \textbf{bra} $\bra{x}$ is the \textbf{conjugate transpose} (row vector) of the ket $\quantumState{x}$.
        \item The \textbf{ket} $\quantumState{y}$ is a \textbf{column vector}.
        \item The inner product is the \textbf{dot product of these two vectors}, resulting in a \textbf{scalar (complex number)}.
    \end{itemize}

    The \hl{inner product tells us} \textbf{how much two quantum states ``overlap''}.
    \begin{itemize}
        \item If $\langle x | y \rangle = 0$, the states are \textbf{orthogonal} (\hl{completely different}).
        \item If $\langle x | y \rangle = 1$, the states are \textbf{identical}.
    \end{itemize}

    \begin{examplebox}[: Inner (Scalar) Product]
        Suppose we have two quantum states:
        \begin{equation*}
            \begin{array}{rcl}
                \quantumState{x} &=& \begin{bmatrix}
                    a \\ b
                \end{bmatrix} \\ [1em]
                \quantumState{y} &=& \begin{bmatrix}
                    c \\ d
                \end{bmatrix}
            \end{array}
        \end{equation*}
        Then:
        \begin{equation*}
            \langle x | y \rangle = \begin{bmatrix}
                \bar{a} & \bar{b}
            \end{bmatrix}
            \begin{bmatrix}
                c \\ d
            \end{bmatrix}
            = \bar{a}c + \bar{b}d
        \end{equation*}
    \end{examplebox}


    \item \definitionWithSpecificIndex{Matrix-Ket Multiplication: $M\quantumState{v}$}{Dirac's Notation: Matrix-Ket Multiplication}{}

    A quantum system evolves by \textbf{applying a matrix} (operator) $M$ \textbf{to a quantum state} (ket):
    \begin{equation*}
        M \quantumState{v}
    \end{equation*}
    Where:
    \begin{itemize}
        \item $\quantumState{v}$ is a \textbf{column vector} (a \hl{quantum state}).
        \item $M$ is a \textbf{matrix} (a \hl{quantum operator}).
    \end{itemize}
    The result is a \textbf{new quantum state} (a transformed column vector).

    Matrix-Ket Multiplication shows how \textbf{quantum gates (unitary matrices) transform quantum states}.

    \begin{examplebox}[: Matrix-Ket Multiplication]
        Let's take:
        \begin{equation*}
            M = \begin{bmatrix}
                0 & 1 \\ 1 & 0
            \end{bmatrix}
            \hspace{2em}
            \quantumState{v} = \begin{bmatrix}
                a \\ b
            \end{bmatrix}
        \end{equation*}
        Then:
        \begin{equation*}
            M \quantumState{v} = \begin{bmatrix}
                0 & 1 \\ 1 & 0
            \end{bmatrix}
            \begin{bmatrix}
                a \\ b
            \end{bmatrix}
            =
            \begin{bmatrix}
                b \\ a
            \end{bmatrix}
        \end{equation*}
    \end{examplebox}


    \item \definitionWithSpecificIndex{Concatenated Multiplications: $\langle x | M | y \rangle$}{Dirac's Notation: Concatenated Multiplications}{}

    A general quantum mechanical expression is:
    \begin{equation*}
        \langle x | M | y \rangle
    \end{equation*}  
    This is the \textbf{expected value or transition amplitude}, which means:
    \begin{enumerate}
        \item Apply the operator $M$ to $\quantumState{y}$ first:  
        \begin{equation*}
            M \quantumState{y}
        \end{equation*}
        Which gives a \textbf{new quantum state}.
        \item Take the inner product with $\bra{x}$:  
        \begin{equation*}
            \bra{x} \left(M \quantumState{y}\right)
        \end{equation*}
        Which results in a scalar (complex number).
    \end{enumerate}
    The result is a \textbf{scalar} that tells us the \textbf{probability amplitude of transitioning from} $\quantumState{y}$ \textbf{to} $\quantumState{x}$ \textbf{via} $M$.

    \begin{examplebox}
        Let's compute:
        \begin{equation*}
            \langle x | M | y \rangle
        \end{equation*}
        Using:
        \begin{equation*}
            \quantumState{x} =
            \begin{bmatrix} a \\ b \end{bmatrix} \hspace{2em}
            \quantumState{y} =
            \begin{bmatrix} c \\ d \end{bmatrix} \hspace{2em}
            M =
            \begin{bmatrix} 0 & 1 \\ 1 & 0 \end{bmatrix}
        \end{equation*}
        \begin{enumerate}
            \item Compute $M \quantumState{y}$:
            \begin{equation*}
                M \quantumState{y} =
                \begin{bmatrix} 0 & 1 \\ 1 & 0 \end{bmatrix}
                \begin{bmatrix} c \\ d \end{bmatrix}
                =
                \begin{bmatrix} d \\ c \end{bmatrix}
            \end{equation*}

            \item Compute: $\bra{x} \left(M \quantumState{y}\right)$:
            \begin{equation*}
                \bra{x} \left(M \quantumState{y}\right) =
                \begin{bmatrix} \bar{a} & \bar{b} \end{bmatrix}
                \begin{bmatrix} d \\ c \end{bmatrix}
                =
                \bar{a}d + \bar{b}c
            \end{equation*}
        \end{enumerate}
    \end{examplebox}
\end{itemize}