\subsection{Single Qubits}\label{subsection: Single Qubits}

A \definition{Qubit} is a \textbf{two-level quantum system}, meaning it \textbf{has only two basis states}:
\begin{equation*}
    \quantumState{0} = \begin{bmatrix}
        1 \\ 0
    \end{bmatrix}
    \hspace{2em}
    \quantumState{1} = \begin{bmatrix}
        0 \\ 1
    \end{bmatrix}
\end{equation*}
\textbf{Any quantum state} of a single qubit can be written as a \textbf{linear combination (\emph{superposition})} of these two basis states:
\begin{equation*}
    \quantumState{\psi} = \alpha \quantumState{0} + \beta \quantumState{1} \hspace{2em} \alpha \: \land \: \beta \ne 0
\end{equation*}
Where:
\begin{itemize}
    \item $\alpha, \beta$ are \textbf{complex numbers} called \definition{Probability Amplitudes}. If either $\alpha = 0$ or $\beta = 0$, the \textbf{state is NOT in superposition} (it is a pure basis state).
    \item The \textbf{normalization condition} holds:
    \begin{equation*}
        \left|\alpha\right|^{2} + \left|\beta\right|^{2} = 1
    \end{equation*}
    To ensure total probability is 1.
\end{itemize}
Therefore, a single qubit is described as a \textbf{2D complex vector} in a Hilbert space.

\begin{flushleft}
    \textcolor{Green3}{\faIcon{th} \textbf{Matrix Representation of Qubit States}}
\end{flushleft}
Quantum states can be expressed in \textbf{matrix form} as \textbf{column vectors}:
\begin{equation*}
    \quantumState{0} = \begin{bmatrix}
        1 \\ 0
    \end{bmatrix}
    \hspace{2em}
    \quantumState{1} = \begin{bmatrix}
        0 \\ 1
    \end{bmatrix}
\end{equation*}
And, a general qubit state is:
\begin{equation*}
    \quantumState{\psi} = \begin{bmatrix}
        \alpha \\ \beta
    \end{bmatrix}
\end{equation*}
Where $\alpha$ and $\beta$ are complex numbers.

\highspace
\begin{flushleft}
    \textcolor{Green3}{\faIcon{question-circle} \textbf{What is a Basis?}}
\end{flushleft}
A \definition{Basis} is a \textbf{set of vectors that define a coordinate system in which we describe quantum states}. Furthermore, basis should \textbf{always be orthonormal} (orthogonal and norm equal to one) because it ensures that quantum states are independent, complete, and allow meaningful probability calculations. For a \textbf{single qubit}, we typically use \textbf{two orthonormal basis states} $\quantumState{0}$ and $\quantumState{1}$, forming the \textbf{computational basis}:
\begin{equation*}
    \quantumState{0} = \begin{bmatrix}1\\0\end{bmatrix}
    \hspace{2em}
    \quantumState{1} = \begin{bmatrix}0\\1\end{bmatrix}
\end{equation*}
Any qubit state can be written as:
\begin{equation*}
    \quantumState{\psi} = \alpha \quantumState{0} + \beta \quantumState{1}
\end{equation*}
Where $\alpha$ and $\beta$ are complex numbers satisfying $\left|\alpha\right|^{2} + \left|\beta\right|^{2} = 1$.

\highspace
In other words, a \textbf{basis allows us to describe quantum states as linear combinations of simpler states}.

\highspace
\begin{flushleft}
    \textcolor{Green3}{\faIcon{question-circle} \textbf{Why do we have a choice of different bases and why should we choose them?}}
\end{flushleft}
While the \textbf{computational basis} $\left\{\quantumState{0}, \quantumState{1}\right\}$ is the \emph{standard}, we are \textbf{not forced} to use it! We can choose \textbf{other bases} depending on the situation, and they help in different computations. This is because a \textbf{different basis simply provides a new way to describe the same quantum state}.

\highspace
Another common basis is the \definition{Hadamard basis}, defined as:
\begin{equation*}
    \begin{array}{rcl}
        \quantumState{+} &=& \dfrac{1}{\sqrt{2}}\left(\quantumState{0} + \quantumState{1}\right) \\ [1em]
        \quantumState{-} &=& \dfrac{1}{\sqrt{2}}\left(\quantumState{0} - \quantumState{1}\right)
    \end{array}
\end{equation*}

\highspace
The basis is also important because the \textbf{choice of basis affects the measurement results}. For example, let's take the qubit state:
\begin{equation*}
    \quantumState{\psi} = \dfrac{1}{\sqrt{2}} \quantumState{0} + \dfrac{1}{\sqrt{2}} \quantumState{1}
\end{equation*}
\begin{itemize}
    \item If measured in the \textbf{computational basis} $\left\{\quantumState{0}, \quantumState{1}\right\}$, it has a 50\% chance of collapsing to $\quantumState{0}$ and 50\% to $\quantumState{1}$.
    \item If measured in the \textbf{Hadamard basis} $\left\{\quantumState{+}, \quantumState{-}\right\}$, it \textbf{always collapses to} $\quantumState{+}$. Then, the probability of collapsing into $\quantumState{+}$ is 100\%, and the probability of collapsing into $\quantumState{-}$ is 0\%.
    \begin{proof}
        We measure $\quantumState{\psi}$ in the Hadamard basis, therefore we must express it using $\quantumState{+}$ and $\quantumState{-}$.

        Since:
        \begin{equation*}
            \quantumState{+} = \dfrac{1}{\sqrt{2}} \left(\quantumState{0} + \quantumState{1}\right)
        \end{equation*}
        If we manipulate $\quantumState{\psi}$ a bit, we can see this:
        \begin{equation*}
            \begin{array}{rcl}
                \quantumState{\psi} &=& \dfrac{1}{\sqrt{2}} \quantumState{0} + \dfrac{1}{\sqrt{2}} \quantumState{1} \\ [1em]
                &=& \dfrac{1}{\sqrt{2}} \left(\quantumState{0} + \quantumState{1}\right) \\ [1em]
                &=& \quantumState{+}
            \end{array}
        \end{equation*}
        And there is no component of $\quantumState{-}$.
    \end{proof}
\end{itemize}

\newpage

\begin{flushleft}
    \textcolor{Green3}{\faIcon{question-circle} \textbf{What happens when we measure a Qubit?}}
\end{flushleft}
In classical computing, a bit is either 0 or 1. In quantum computing, a qubit exists in a \textbf{superposition} of $\quantumState{0}$ and $\quantumState{1}$, but \textbf{when measured, it collapses into one of these basis states}. This is because the \hl{measurement is probabilistic and destroys the superposition}.

\highspace
If a qubit is in state:
\begin{equation*}
    \quantumState{\psi} = \alpha \quantumState{0} + \beta \quantumState{1}
\end{equation*}
Measurement \textbf{forces the qubit to collapse} into either $\quantumState{0}$ or $\quantumState{1}$. The \textbf{probability of each outcome} is given by the \textbf{squared magnitudes} of the coefficients:
\begin{equation*}
    P\left(0\right) = \left|\alpha\right|^{2} \hspace{2em} P\left(1\right) = \left|\beta\right|^{2}
\end{equation*}
\textbf{After measurement}, the qubit \textbf{loses superposition} and remains in the measured state.

\highspace
\begin{examplebox}[: Qubit Measurement]
    Consider the qubit:
    \begin{equation*}
        \quantumState{\psi} = \dfrac{3}{5}\quantumState{0} + \dfrac{4}{5}\quantumState{1}
    \end{equation*}
    \begin{itemize}
        \item The probability of measuring $\quantumState{0}$ is:
        \begin{equation*}
            P\left(0\right) = \left(\dfrac{3}{5}\right)^{2} = 0.36
        \end{equation*}
        \item The probability of measuring $\quantumState{1}$ is:
        \begin{equation*}
            P\left(1\right) = \left(\dfrac{4}{5}\right)^{2} = 0.64
        \end{equation*}
    \end{itemize}
    If we \textbf{measure} the qubit:
    \begin{itemize}
        \item With 36\% probability, it collapses to $\quantumState{0}$.
        \item With 64\% probability, it collapses to $\quantumState{1}$.
    \end{itemize}
    After measurement, the qubit \textbf{remains in that state} until modified by another operation.
\end{examplebox}

\noindent
In the previous example, we can observe that the measurement \textbf{collapses} the quantum state \textbf{into one of the basis} states with a \textbf{probability determined by its amplitude}.

\newpage

\begin{flushleft}
    \textcolor{Green3}{\faIcon{question-circle} \textbf{What happens if we measure twice?}}
\end{flushleft}
If the qubit collapses to $\quantumState{0}$ in the first measurement, a second measurement in the same basis will return $\quantumState{0}$ with probability 1. This is because the qubit is already in $\quantumState{0}$ and has no component of $\quantumState{1}$ left. Therefore, \textbf{repeating a measurement in the same basis always gives the same result}.

\highspace
\begin{flushleft}
    \textcolor{Red2}{\faIcon{book} \textbf{Measurement as a Fundamental Axiom}}
\end{flushleft}
The behavior of quantum measurement is \textbf{not derived from other principles}, it is an \textbf{axiom of quantum mechanics}:
\begin{enumerate}
    \item Measurement collapses the quantum state.
    \item The probability of each outcome is given by the squared amplitude.
    \item A second measurement (in the same basis) gives the same result with probability 1.
\end{enumerate}
Therefore, the measurement is a \textbf{fundamental rule} of quantum mechanics.

\highspace
\begin{flushleft}
    \textcolor{Red2}{\faIcon{exclamation-triangle} \textbf{Fundamental limitation of Quantum Computing}}
\end{flushleft}
Unlike a classical bit, which can be only 0 or 1, a qubit can exist in any superposition:
\begin{equation*}
    \quantumState{\psi} = a \quantumState{0} + b \quantumState{1}
\end{equation*}
Where $a$ and $b$ are complex numbers that satisfy $\left|a\right|^{2} + \left|b\right|^{2} = 1$. Since $a$ and $b$ can take infinitely many values, \textbf{a single qubit \emph{theoretically} has an infinite number of possible states}.

\highspace
\textcolor{Green3}{\faIcon{question-circle} \textbf{Can a Qubit Store More than One Classical Bit?}} One might hope that because a qubit has infinitely many states
, it could \textbf{store and transmit more than one classical bit of information}. However, \textbf{this is not possible} because:
\begin{itemize}[label=\textcolor{Red2}{\faIcon{times}}]
    \item \textcolor{Red2}{\textbf{A single measurement only gives one classical bit}}. Measuring $\quantumState{\psi}$ forces it to collapse into $\quantumState{0}$ or $\quantumState{1}$. The outcome follows probabilities $P\left(0\right) = \left|a\right|^{2}$ and $P\left(1\right) = \left|b\right|^{2}$. Since the result is \textbf{just one binary outcome}, it \textbf{cannot reveal both $a$ and $b$ at the same time}.

    \item \textcolor{Red2}{\textbf{Measurement Destroys the Quantum State}}. Once we measure a qubit, its \textbf{original state is lost}. This means we cannot measure $a$ and $b$ separately, even if we repeat the measurement.
\end{itemize}
Unfortunately, this is a limitation, because a single qubit contains \textbf{infinite information theoretically}, but in practice, we can \textbf{only extract one classical bit per measurement}.

\newpage

\noindent
Furthermore, another problem is that we cannot copy a quantum state. The no-cloning theorem (explained later) states that it is \textbf{impossible to perfectly copy an arbitrary quantum state}. This has major consequences:
\begin{itemize}
    \item \important{We cannot measure a qubit's twice}. In classical computing, we can copy and measure a bit multiple times. In quantum computing, copying is not possible. Once a qubit is measured, the original superposition is destroyed.

    \item \emph{Why can't we just copy and measure?} Suppose we want to copy a qubit $\quantumState{\psi}$ and measure both copies. \textbf{Quantum mechanics forbids perfect duplication of unknown quantum states}. This prevents duplicating quantum information and extracting more than one bit of classical information per qubit. A proof will be provided later with a better explanation.
\end{itemize}

\highspace
\begin{flushleft}
    \textcolor{Green3}{\faIcon{book} \textbf{The state space of a Single Qubit}}
\end{flushleft}
A \textbf{single qubit exists in a two-dimensional complex Hilbert space}:
\begin{equation*}
    \quantumState{\psi} = \alpha\quantumState{0} + \beta\quantumState{1}
\end{equation*}
Where:
\begin{itemize}
    \item $\alpha$ and $\beta$ are complex numbers (probability amplitudes).
    \item The normalization condition ensures:
    \begin{equation*}
        \left|\alpha\right|^{2} + \left|\beta\right|^{2} = 1
    \end{equation*}
\end{itemize}
Since $\alpha$ and $\beta$ can take complex values, a \textbf{qubit is more than just a point} in 2D, it has \textbf{four real parameters} (\hl{two from each complex number}). However, \hl{due to normalization and global phase invariance}, \textbf{only two real parameters are needed to describe a qubit}.

\highspace
Therefore, the \textbf{space of all possible qubit states is a continuous space, not just discrete values like classical bits}.

\newpage

\begin{flushleft}
    \textcolor{Green3}{\faIcon{question-circle} \textbf{Block Sphere representation of the qubit (and why)}}
\end{flushleft}
The \textbf{bloch sphere} is a geometric representation of a qubit's state that helps visualize its properties. Since a general qubit state is:
\begin{equation*}
    \quantumState{\psi} = \alpha\quantumState{0} + \beta\quantumState{1}
\end{equation*}
We can rewrite it using two angles $\theta$ and $\phi$ (as \textbf{spherical coordinates}):
\begin{equation}\label{eq: superposition state}
    \quantumState{\psi} = \cos\left(\dfrac{\theta}{2}\right) \quantumState{0} + e^{i\phi} \sin\left(\dfrac{\theta}{2}\right) \quantumState{1}
\end{equation}
Where:
\begin{itemize}
    \item $\theta$ is called \textbf{polar angle} (latitude). It determines how much of $\quantumState{0}$ and $\quantumState{1}$ are mixed (the qubit is not just in one state, but in both simultaneously).
    \begin{itemize}
        \item When $\theta = 0 \rightarrow \quantumState{\psi} = \quantumState{0}$

        It is a pure state.
        \item When $\theta = \pi \rightarrow \quantumState{\psi} = \quantumState{1}$

        It is a pure state.
        \item When $\theta = \frac{\pi}{2} \rightarrow \quantumState{\psi} = \frac{1}
        {\sqrt{2}}\quantumState{0} + e^{i\phi} \frac{1}{\sqrt{2}}\quantumState{1}$

        It is called equal superposition.
    \end{itemize}
    \item $\phi$ is called \textbf{relative phase} (longitude). It \textbf{controls the phase relationship} between $\quantumState{0}$ and $\quantumState{1}$. Changing $\phi$ \textbf{does not affect measurement probabilities}, but it affects interference when qubits interacts with other qubits.
\end{itemize}
In general, on the block sphere:
\begin{itemize}
    \item The \textbf{north pole} $\left(0,0,1\right)$ is $\quantumState{0}$.
    \item The \textbf{south pole} $\left(0,0,-1\right)$ is $\quantumState{1}$.
    \item Any other point represents a \textbf{superposition state} (eq. \ref{eq: superposition state}, page \pageref{eq: superposition state}).

\end{itemize}
So, the Bloch sphere shows \textbf{how a qubit evolves} under quantum operations, making it easier to understand \textbf{rotations}, \textbf{phase shifts}, and \textbf{measurements}.

\highspace
\textcolor{Green3}{\faIcon{question-circle} \textbf{Why is the bloch sphere important?}}
The bloch sphere \textbf{helps us visualize} superposition, phase, and quantum operations intuitively.
\begin{enumerate}
    \item It gives a visual representation of qubit states. Classical bits are just points (0 or 1), whereas a qubit exists everywhere on the sphere.
    
    \item It shows quantum gates as rotations. As we will see in the following pages, quantum gates rotate the qubit around the sphere. For example, the Hadamard gate rotates $\quantumState{0}$ to $\quantumState{+}$, moving from the north pole to the equator.

    \item It helps understand measurement. Measuring a qubit collapses it to either $\quantumState{0}$ or $\quantumState{1}$, removing phase information.
\end{enumerate}

\highspace
\begin{flushleft}
    \textcolor{Green3}{\faIcon{question-circle} \textbf{Unfortunately, the $e^{i\phi}$ factor does not affect the global sphere?}}
\end{flushleft}
Wrong! Two quantum state vectors are considered \textbf{equivalent if they differ only by a global phase factor}. This means that if we have two quantum states:  
\begin{equation*}
    \begin{array}{rcl}
        \quantumState{\psi}  &=& \alpha\quantumState{0} + \beta\quantumState{1} \\[.5em]
        \quantumState{\psi'} &=& e^{i\gamma}\left(\alpha\quantumState{0} + \beta\quantumState{1}\right)
    \end{array}
\end{equation*}
Where $e^{i\gamma}$ is a \definition{global phase factor} (a complex number with magnitude 1), then \textbf{these two states are physically identical}.

\highspace
A \textbf{global phase} is a complex factor of the form:
\begin{equation*}
    e^{i\gamma} = \cos\gamma + i\sin\gamma
\end{equation*}
Which multiplies the entire quantum state but has no physical impact on measurement probabilities.

\highspace
Therefore, a qubit state is \textbf{not changed} by multiplying it by a global phase factor $e^{i\gamma}$, meaning that:
\begin{equation*}
    \quantumState{\psi} \sim e^{i \gamma} \quantumState{\psi}
\end{equation*}
This means that the \textbf{block sphere represents only unique qubit states}, since the global phase doesn't affect the measurement.

\begin{examplebox}[: Identical Qubit States]
    Suppose we have two quantum states:
    \begin{equation*}
        \quantumState{\psi} = \dfrac{1}{\sqrt{2}}\quantumState{0} + \dfrac{1}{\sqrt{2}}\quantumState{1}
    \end{equation*}
    And:
    \begin{equation*}
        \quantumState{\psi'} = e^{1\pi} \left(\dfrac{1}{\sqrt{2}}\quantumState{0} + \dfrac{1}{\sqrt{2}}\quantumState{1}\right)
    \end{equation*}
    Since $e^{i\pi} = -1$, we can simplify:
    \begin{equation*}
        \quantumState{\psi'} = -\dfrac{1}{\sqrt{2}}\quantumState{0} -\dfrac{1}{\sqrt{2}}\quantumState{1}
    \end{equation*}
    Even though $\quantumState{\psi}$ and $\quantumState{\psi'}$ look different mathematically, they are physically the same because they only differ by a global phase factor $e^{i\pi}$.

    The global phase does not affect the measurement results. In fact, the probabilities remain for both $\psi$ and $\psi'$:
    \begin{equation*}
        P\left(0\right) = \left|-\dfrac{1}{\sqrt{2}}\right|^{2} = \dfrac{1}{2}
        \hspace{2em}
        P\left(1\right) = \left|-\dfrac{1}{\sqrt{2}}\right|^{2} = \dfrac{1}{2}
    \end{equation*}
    Since measurement gives the same results for both states, we consider them physically identical.
\end{examplebox}

\begin{figure}[!htp]
    \centering
    \includegraphics[width=.7\textwidth]{img/bloch-sphere-qubit-1.pdf}
    \caption{Bloch sphere representation of qubit.}
    \label{fig: Bloch sphere representation of qubit}
\end{figure}