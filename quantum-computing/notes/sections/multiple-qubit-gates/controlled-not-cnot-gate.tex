\subsection{Controlled NOT (CNOT) Gate}

The \definitionWithSpecificIndex{Controlled-NOT (CNOT)}{Multiple Qubit Gates: Controlled-NOT (CNOT)}{} gate is one of the most fundamental and widely used \textbf{two-qubit gates} in quantum computing. It serves as a canonical example of a \textbf{quantum gate that cannot be decomposed into single-qubit operations}, and it plays a central role in the generation of entanglement, error correction, and universal gate constructions.

\highspace
\begin{flushleft}
    \textcolor{Green3}{\faIcon{tools} \textbf{How does it work?}}
\end{flushleft}
The CNOT gate operates on two qubits:
\begin{itemize}
    \item The \textbf{control qubit} \ket{c}
    \item The \textbf{target qubit} \ket{t}
\end{itemize}
Its defining action is \textbf{conditional flipping of the target qubit}. If the control qubit is in the state:
\begin{itemize}
    \item \ket{1}, the CNOT gate \textbf{applies the Pauli-X (NOT) gate} (page \pageref{subsubsection: Pauli-X NOT Gate}) to the target (so the target qubit flips and 0 becomes 1 and 1 becomes 0).
    \item \ket{0}, the \textbf{target is left unchanged}.
\end{itemize}
This behavior can be expressed algebraically as:
\begin{equation}
    \text{CNOT} \ket{c}\ket{t} = \ket{c} \ket{t \oplus c}    
\end{equation}
Where:
\begin{itemize}
    \item Qubit Input: two-qubit basis state \ket{c}\ket{t} ($\ket{c}\ket{t} = \ket{c \: t}$).
    \item CNOT operator: if $c=0$, leave $t$ unchanged, otherwise, flip $t$.
    \item XOR $\oplus$ symbol: denotes addition modulo 2.
    \begin{equation*}
        t \oplus c =
        \begin{cases}
            0 & \text{if } t = c \\
            1 & \text{if } t \neq c
        \end{cases}
    \end{equation*}
    It flips $t$ if and only if $c = 1$. The \definition{Addition modulo 2}, often denoted as $ \oplus $, is a simple binary operation where the result is the remainder after dividing the sum of two bits by 2. It's also known as the \textbf{exclusive OR (XOR)} operation.
    \begin{equation*}
        \begin{array}{c|c||c}
            a & b & a \oplus b \\
            \hline
            0 & 0 & 0 \\
            0 & 1 & 1 \\
            1 & 0 & 1 \\
            1 & 1 & 0 \\
        \end{array}
    \end{equation*}
    So, $t \oplus c$ means: ``flip the target qubit $t$ if and only if the control qubit $c$ is 1''.
    \item The control qubit \ket{c} stays the same.
    \item The target qubit \ket{t} becomes $\ket{t \oplus c}$ (flipped if $c = 1$).
\end{itemize}

\highspace
\begin{flushleft}
    \textcolor{Green3}{\faIcon{square-root-alt} \textbf{Matrix Representation}}
\end{flushleft}
The matrix representation of the CNOT gate is:
\begin{equation}
    \text{CNOT} =
    \begin{bmatrix}
        1 & 0 & 0 & 0 \\
        0 & 1 & 0 & 0 \\
        0 & 0 & 0 & 1 \\
        0 & 0 & 1 & 0
    \end{bmatrix}
\end{equation}
It operates on \textbf{two-qubit states} ordered as:
\begin{equation*}
    \left\{\ket{00}, \ket{01}, \ket{10}, \ket{11}\right\}
\end{equation*}
This matrix is clearly \textbf{unitary}, preserving the norm of quantum states, and it \textbf{interchanges the basis} vectors \ket{10} and \ket{11} while \textbf{leaving} \ket{00} and \ket{01} \textbf{unchanged}, consistent with its definition.

\highspace
For example, let a general two-qubit quantum state written as a linear combination of basis states:
\begin{equation*}
    \ket{\psi} = c_{0}\ket{00} + c_{1}\ket{01} + c_{2}\ket{10} + c_{3}\ket{11}
\end{equation*}
Where:
\begin{equation*}
    c_{0}, c_{1}, c_{2}, c_{3} \in \mathbb{C} \hspace{1em} \text{and} \hspace{1em} \sum\left|c_{i}\right|^{2} = 1
\end{equation*}
Now apply the CNOT gate:
\begin{equation*}
    \begin{bmatrix}
        1 & 0 & 0 & 0 \\
        0 & 1 & 0 & 0 \\
        0 & 0 & 0 & 1 \\
        0 & 0 & 1 & 0
    \end{bmatrix} \cdot
    \begin{bmatrix}
        c_{0} \\ c_{1} \\ c_{2} \\ c_{3}
    \end{bmatrix} =
    \begin{bmatrix}
        c_{0} \\ c_{1} \\ c_{3} \\ c_{2}
    \end{bmatrix}
\end{equation*}
The amplitudes of the \ket{10} and \ket{11} components have been swapped. This corresponds to flipping the target qubit only when the control is \ket{1}:
\begin{equation*}
    \ket{\psi} = c_{0}\ket{00} + c_{1}\ket{01} + c_{2}\ket{11} + c_{3}\ket{10}
\end{equation*}

\highspace
\begin{flushleft}
    \textcolor{Green3}{\faIcon{check-circle} \textbf{Who is the Control Qubit?}}
\end{flushleft}
We have discussed that the CNOT gate operates on two qubits: control \ket{c} and target \ket{t}. In the operations, \emph{how can we understand which is the control qubit or the target qubit?} By \textbf{convention}, the \textbf{control qubit is the most significant qubit}, i.e., the \textbf{leftmost one} in the tensor product \ket{c \: t}. So:
\begin{itemize}
    \item \ket{00}: control $= 0$, target $= 0$
    \item \ket{01}: control $= 0$, target $= 1$
    \item \ket{10}: control $= 1$, target $= 0$
    \item \ket{11}: control $= 1$, target $= 1$
\end{itemize}
Thus, in a state like:
\begin{equation*}
    \ket{\psi} = c_{0}\ket{00} + c_{1}\ket{01} + c_{2}\ket{10} + c_{3}\ket{11}
\end{equation*}
The \textbf{first qubit is the control}, and the \textbf{second is the target}.

\highspace
\begin{flushleft}
    \textcolor{Green3}{\faIcon{book} \textbf{Generate Entanglement}}
\end{flushleft}
An important aspect of the CNOT gate is its \textbf{ability to generate entanglement}. For instance, if the control qubit is prepared in a superposition state using a Hadamard gate $H$, such that the input to the CNOT is:
\begin{equation*}
    \left(H \otimes I\right)\ket{00} = \dfrac{1}{\sqrt{2}}\left(\ket{00} + \ket{10}\right)
\end{equation*}
Then applying the \textbf{CNOT produces the entangled Bell state}:
\begin{equation*}
    \text{CNOT} \left( \dfrac{1}{\sqrt{2}}\left(\ket{00} + \ket{10}\right) \right) = \dfrac{1}{\sqrt{2}}\left(\ket{00} + \ket{11}\right)
\end{equation*}
This outcome exemplifies how \textbf{CNOT combined with single-qubit gates can prepare maximally entangled states}, a prerequisite for many quantum protocols including teleportation and superdense coding.

\highspace
\begin{flushleft}
    \textcolor{Green3}{\faIcon{tools} \textbf{Visual representation of a CNOT}}
\end{flushleft}
Finally, in quantum circuit diagrams, the CNOT is represented with a solid dot on the control qubit and a \( \oplus \) symbol on the target qubit, connected by a vertical line. This visual notation helps emphasize the logical dependence between the two qubits.
\begin{center}
    \begin{quantikz}
        \lstick{\ket{c}} & \ctrl{1} & \rstick{\ket{c}} \\
        \lstick{\ket{t}} & \targ{}  & \rstick{\ket{c} $\otimes$ \ket{t}}
    \end{quantikz}
\end{center}