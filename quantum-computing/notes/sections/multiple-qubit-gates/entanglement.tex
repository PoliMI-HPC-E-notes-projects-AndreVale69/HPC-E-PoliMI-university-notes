\subsection{Entanglement}

In a multi-qubit quantum system, \definition{Entanglement} refers to the \textbf{phenomenon in which the quantum state of each qubit cannot be described independently of the state of the orders}. Entangled states are \textbf{non-separable}, meaning they cannot be expressed as the tensor product of individual qubit states.

\highspace
\begin{itemize}
    \item \important{The state of a $n$-qubit system is described by $2^{n}$ complex amplitudes.}

    \textcolor{Green3}{\faIcon{question-circle} \textbf{Why?}} Because a general quantum state is a \textbf{superposition of all basis states}. For $n$ qubits, there are $2^{n}$ possible basis states (e.g., \ket{000}, \ket{001}, $\dots$, \ket{111}). So a general state is:
    \begin{equation*}
        \ket{\psi} = c_{0} \ket{0 \dots 0} + c_{1} \ket{0 \dots 1} + \dots + c_{2^{n} - 1} \ket{1 \dots 1}
    \end{equation*}
    Where:
    \begin{itemize}
        \item $c_{0}, c_{1}, \dots, c_{2^{n} - 1} \in \mathbb{C}$.
        \item So we need $2^{n}$ complex numbers $=2 \cdot 2^{n} = 2^{n+1}$ real numbers.
    \end{itemize}


    %%%%%%%%%%%%%%%%%%%%


    \item \important{Accounting for normalization and the irrelevance of global phase, it takes $2^{n + 1} - 2$ real numbers to fully specify the state.}
    
    If we want to \textbf{completely describe the quantum state of an $n$-qubit system}, so that we can simulate it, reconstruct it, or predict its evolution, \textbf{we need to know $2^{n+1}-2$ real numbers}.

    \highspace
    In other words, if someone gives us:
    \begin{itemize}
        \item A list of $2^{n+1} - 2$ real values
        \item And tells us ``these specify the quantum state''
    \end{itemize}
    Then:
    \begin{enumerate}
        \item We can \textbf{reconstruct the full state vector}
        \item We know everything we need to calculate outcomes of \textbf{any measurable}
        \item We can simulate unitary evolution, or compute entanglement, etc.
    \end{enumerate}

    \begin{examplebox}[: $n=1$ qubit]
        The state is:
        \begin{equation*}
            \ket{\psi} = \alpha\ket{0} + \beta\ket{1} \quad \text{with } \alpha, \beta \in \mathbb{C}
        \end{equation*}
        That's 4 real numbers (2 complex), but minus:
        \begin{itemize}
            \item 1 for normalization: $\left|\alpha\right|^{2} + \left|\beta\right|^{2} = 1$
            \item 1 for global phase: $e^{i\theta}$ is physically irrelevant
        \end{itemize}
        So, to fully describe a single qubit, we need:
        \begin{equation*}
            2^{1+1} - 2 = 2 \text{ real numbers}
        \end{equation*}
        And this make sense, because we can describe any pure qubit with two real angles $\theta, \phi$ on the block sphere (page \pageref{fig: Bloch sphere representation of qubit}).
    \end{examplebox}

    \textcolor{Green3}{\faIcon{question-circle} \textbf{Why $2^{n+1}$?}}
    \begin{itemize}
        \item $2^{n}$ complex amplitudes for an $n$-qubit system (because there are $2^{n}$ basis states).
        \item Each complex number is made up of 2 real numbers: the real part and the imaginary part. So we have to add one to $n$.
    \end{itemize}

    \textcolor{Green3}{\faIcon{question-circle} \textbf{Why do we subtract 2?}}
    \begin{itemize}
        \item \important{Normalization}. \textbf{Quantum states must always be normalized}:
        \begin{equation*}
            \displaystyle\sum_{i=0}^{2^{n}-1} \left|c_{i}\right|^{2} = 1
        \end{equation*}
        This is \textbf{one constraint} on the set of amplitudes. Since it's a real-valuated equation, it removes 1 real degree of freedom from the total.

        \item \important{Global Phase}. In quantum mechanics, multiplying a quantum state by a global phase $e^{i\theta}$ doesn't change anything physically measurable:
        \begin{equation*}
            \ket{\psi} \quad \text{and} \quad e^{i\theta} \ket{\psi}
        \end{equation*}
        Are \textbf{physically indistinguishable}, they lead to the \textbf{same measurement outcomes}. So we treat two such states as equivalent. That means we're overcounting one extra complex degree of freedom, which corresponds to the global phase. This removes another real parameter (the angle $\theta$).
    \end{itemize}


    %%%%%%%%%%%%%%%%%%%%
    

    \item \important{In contrast, $n$ independent (separable, then not entangled)\break qubits would only require $2n$ real parameters.}
    
    \textcolor{Green3}{\faIcon{question-circle} \textbf{What about independent qubits?}} If the qubits are not entangled, each qubit can be written separately as:
    \begin{equation*}
        \ket{\psi_{i}} = \alpha_{i}\ket{0} + \beta_{i}\ket{1}
    \end{equation*}
    Each qubit needs:
    \begin{itemize}
        \item $\alpha_{i}, \beta_{i} \in \mathbb{C}$
        \item But again, we subtract 2 real numbers (normalization and global phase) for each qubit.
    \end{itemize}
    So one single qubit state needs \textbf{2 real parameters}.

    
    %%%%%%%%%%%%%%%%%%%%


    \item \important{Since $2^{n} \gg 2n$, most quantum states are entangled.}

    \textcolor{Green3}{\faIcon{question-circle} \textbf{What does this mean?}} If a state can be written as:
    \begin{equation*}
        v_{1} \otimes v_{2} \otimes \dots \otimes v_{n}
    \end{equation*}
    Then it only uses $2n$ real numbers. But a general $n$-qubit state needs $2^{n+1}-2 \gg 2n$ real numbers. So: \textbf{almost all quantum states cannot be written as tensor products of single-qubit states}. These are the entangled states.

    \highspace
    \textcolor{Green3}{\faIcon{question-circle} \textbf{How is this possible?}} The \textbf{full space} of quantum states is \textbf{exponentially larger} ($2^{n+1}-2$) than the subspace of separable states ($2n$). We can think of the space of all quantum states as a huge-dimensional sphere of radius 1 (because of normalization).
    \begin{itemize}
        \item The set of separable states is a tiny curved surface embedded inside that space.
        \item It's so tiny that if we pick a random point on the full sphere (a random quantum state), we will almost surely land outside the separable surface.
    \end{itemize}
    The probability of randomly choosing a separable state from the set of all quantum states is essentially \textbf{zero}.

    \highspace
    Note that this is not a mathematical trick! This is \textbf{experimentally observable}. If we:
    \begin{itemize}
        \item Generate random quantum states (e.g. using random circuits)
        \item Or simulate the uniform distribution over the Hilbert space
    \end{itemize}
    We'll find that:
    \begin{itemize}
        \item With high probability, the state cannot be written as a product of individual qubit states.
        \item In fact, the expected entanglement entropy of a randomly chosen pure state is very close to maximal.
    \end{itemize}
\end{itemize}

\begin{examplebox}[: Bell State]
    One of the canonical examples of an entangled state is the \textbf{Bell state}:
    \begin{equation}
        \ket{\Phi^{+}} = \dfrac{1}{\sqrt{2}}\left(\ket{00} + \ket{11}\right)
    \end{equation}
    Suppose we try to write this as product of two qubits:
    \begin{itemize}
        \item Let $v_{A} = a_{0} \ket{0} + a_{1}\ket{1}$
        \item Let $v_{A} = b_{0} \ket{0} + b_{1} \ket{1}$
    \end{itemize}
    Their tensor product becomes:
    \begin{equation*}
        v_{A} \otimes v_{B} = a_{0}b_{0}\ket{00} + a_{0}b_{1}\ket{01} + a_{1}b_{0}\ket{10} + a_{1}b_{1}\ket{11}
    \end{equation*}
    To match the Bell state, we would require:
    \begin{equation*}
        a_{0} b_{0} = \frac{1}{\sqrt{2}}, \quad a_{1} b_{1} = \frac{1}{\sqrt{2}}, \quad a_{0} b_{1} = 0, \quad a_{1} b_{0} = 0
    \end{equation*}
    But if $a_{0}b_{1} = 0$, then at least one of $a_{0}$ or $b_{1}$ must be zero. Similarly, if $a_{1}b_{0} = 0$, one of $a_{1}$ or $b_{0}$ must be zero. This leads to a contradiction, because we also need $a_{0}b_{0} \neq 0$ and $a_{1}b_{1} \neq 0$. Therefore, no such $a_{0}, a_{1}, b_{0}, b_{1}$ exist.

    Conclusion: the Bell state cannot be written as a tensor product of two individual qubit states, it is entangled.
\end{examplebox}

\highspace
\begin{flushleft}
    \textcolor{Green3}{\faIcon{book} \textbf{The Nature of Entanglement}}
\end{flushleft}
Entanglement is a \textbf{structural property} of the quantum state, \textbf{not dependent on the basis}. This \textbf{contrasts with superposition}, which is basis-dependent.
\begin{itemize}
    \item Whether a state is entangled depends on the decomposition of the system.
    \item A state can be entangled with respect to one partitioning of subsystems, and separable under another.
    \item In most contexts, when we say \textbf{a state is entangled}, we refer to its \textbf{entanglement with respect to the individual qubits} (standard tensor product structure).
\end{itemize}
So we can change the basis of the individual qubits (rotate them, use Hadamard basis), but we will \textbf{not} be able to write Bell state state as $\ket{\psi_{A}} \otimes \ket{\psi_{B}}$. No matter what basis we use, we will \textbf{not} be able to ``break'' the entanglement. Entanglement is about whether the state \textbf{can be written as a product of parts}, this is a property of the \textbf{whole structure}, not of how we write it.

\highspace
\begin{flushleft}
    \textcolor{Green3}{\faIcon{microchip} \textbf{Entangling Gates}}
\end{flushleft}
A gate is said to be \definitionWithSpecificIndex{entangling}{Entangling Gates}{} if it is \textbf{capable of producing entangled states from separable inputs}. Not all two-qubit gates are entangling.
\begin{itemize}
    \item A \textbf{generic 2-qubit gate} is a $4 \times 4$ unitary matrix, with 16 complex entries (minus constraint from unitarity).
    \item Two single-qubit gates acting independently on two qubits use only $2 \times 4 = 8$ complex parameters.
\end{itemize}
Thus, \textbf{most two-qubit gates are not tensor products} of single-qubit gates. If a gate cannot be written as $U_{1} \otimes U_{2}$, it is \textbf{non-separable} and \textbf{can generate entanglement}.

\begin{examplebox}[: CNOT as an Entangling Gate]
    The CNOT gate can transform separable states into entangled states, for example:
    \begin{equation*}
        \text{CNOT}\left( \dfrac{\ket{0} + \ket{1}}{\sqrt{2}} \otimes \ket{0} \right) = \dfrac{\ket{00} + \ket{11}}{\sqrt{2}} = \ket{\Phi^{+}} \quad \text{(Bell state)}
    \end{equation*}
    This demonstrates the key role of CNOT in quantum-algorithms, it \textbf{creates entanglement}, which is a resource for quantum advantage.
\end{examplebox}

\highspace
\begin{flushleft}
    \textcolor{Green3}{\faIcon{question-circle} \textbf{What are Bell States?}}
\end{flushleft}
\definition{Bell States} are \textbf{four special two-qubit states}:
\begin{equation}
    \begin{array}{rclcl}
        \ket{\Phi^{+}} & = & \dfrac{1}{\sqrt{2}} \left(\ket{00} + \ket{11} \right) \hspace{2.5em}
        \ket{\Phi^{-}} & = & \dfrac{1}{\sqrt{2}} \left(\ket{00} - \ket{11} \right) \\ [1em]
        \ket{\Psi^{+}} & = & \dfrac{1}{\sqrt{2}} \left(\ket{01} + \ket{10} \right) \hspace{2.5em}
        \ket{\Psi^{-}} & = & \dfrac{1}{\sqrt{2}} \left(\ket{01} - \ket{10} \right)
    \end{array}
\end{equation}
That are:
\begin{itemize}
    \item \important{Maximally entangled}: measuring one qubit fully determines the other.
    \item \important{Orthonormal}: they form a basis of the two-qubit Hilbert space.
    \item \important{Used in fundamental protocols} like teleportation, superdense coding, and entanglement-based quantum cryptography.
\end{itemize}
Bell states represent \textbf{all possible maximally entangles superpositions of two-qubit basis states}.

\highspace
\begin{flushleft}
    \textcolor{Red2}{\faIcon{exclamation-triangle} \textbf{Why are they important?}}
\end{flushleft}
Bell states are not just theoretical constructs, they are:
\begin{itemize}
    \item \textbf{Fundamental to quantum computing}: often used as building blocks for more complex states.
    \item \textbf{Hard to simulate classically}: their entanglement implies that classical representations cannot compress them efficiently.
    \item \textbf{Maximally entangled}: each individual qubit, when observed in isolation, appears \textbf{completely random}, even though the total state is pure and deterministic.
\end{itemize}
When a Bell state is measured on one qubit, the outcome of the second qubit is \textbf{fully correlated}, but until that measurement happens, each qubit looks like it is in a \textbf{completely mixed state} (i.e. maximum uncertainty). This \textbf{contrast with unentangled (separable)} states, where each qubit can be described independently and has less uncertainty when observed locally.

\highspace
\begin{flushleft}
    \textcolor{Green3}{\faIcon{question-circle} \textbf{How to create Bell states}}
\end{flushleft}
Bell states can be created using a \textbf{simple 2-gate quantum circuit}:
\begin{enumerate}
    \item Apply a \textbf{Hadamard gate to the first qubit}:
    \begin{equation*}
        H\ket{0} = \dfrac{1}{\sqrt{2}}\left(\ket{0} + \ket{1}\right)
    \end{equation*}
    \item Apply a \textbf{CNOT gate}, with:
    \begin{itemize}
        \item Control: qubit 1
        \item Target: qubit 2
    \end{itemize}
\end{enumerate}
\begin{center}
    \begin{quantikz}
        \lstick{\ket{0}} & \gate{H} & \ctrl{1} & \\
        \lstick{\ket{0}} &          & \targ{}  &
    \end{quantikz}
\end{center}

\noindent
The final output is:
\begin{equation*}
    \text{CNOT}\left(H \otimes I\right)\left(\ket{00}\right) = \dfrac{1}{\sqrt{2}}\left(\ket{00} + \ket{11}\right) = \ket{\Phi^{+}}
\end{equation*}
So this circuit creates the $\Phi^{+}$ Bell state starting from \ket{00}. Mathematical:
\begin{enumerate}
    \item Compute $H \otimes I$, i.e., apply Hadamard to the first qubit:
    \begin{equation*}
        \dfrac{1}{\sqrt{2}} \begin{bmatrix}
            1 & 0 & 1 & 0 \\
            0 & 1 & 0 & 1 \\
            1 & 0 & -1 & 0 \\
            0 & 1 & 0 & -1
        \end{bmatrix}
    \end{equation*}

    \item Then apply the CNOT operator to that result:
    \begin{equation*}
        \text{CNOT} =
        \begin{bmatrix}
            1 & 0 & 0 & 0 \\
            0 & 1 & 0 & 0 \\
            0 & 0 & 0 & 1 \\
            0 & 0 & 1 & 0
        \end{bmatrix}
    \end{equation*}

    \item Multiply:
    \begin{equation*}
        \text{CNOT} \cdot \left(H \otimes I\right) \ket{00} = \dfrac{1}{\sqrt{2}}\left(\ket{00} + \ket{11}\right)
    \end{equation*}
\end{enumerate}

\begin{examplebox}[: Bell States in the Hadamard Basis]
    In this exercise, we express the Bell state $\ket{\Phi^{+}}$ in the Hadamard basis instead of the computational basis.

    \begin{proof}[Bell state $\ket{\Phi^{+}}$ in the Hadamard basis]
        Given:
        \begin{equation*}
            \ket{\Phi^{+}} = \dfrac{1}{\sqrt{2}}\left(\ket{00} + \ket{11}\right)
        \end{equation*}
        We aim to re-express this state using the Hadamard basis $\left\{\ket{+}, \ket{-}\right\}$, where:
        \begin{itemize}
            \item $\ket{+} = \dfrac{1}{\sqrt{2}} \left(\ket{0} + \ket{1}\right)$
            \item $\ket{-} = \dfrac{1}{\sqrt{2}} \left(\ket{0} - \ket{1}\right)$
        \end{itemize}
        \begin{enumerate}
            \item \textbf{Invert the definitions}. To convert from the computational basis to the Hadamrd basis, we invert the above equations to express \ket{0} and \ket{1} in terms of \ket{+} and \ket{-}:
            \begin{itemize}
                \item $\ket{0} = \dfrac{1}{\sqrt{2}} \left(\ket{+} + \ket{-}\right)$
                \item $\ket{1} = \dfrac{1}{\sqrt{2}} \left(\ket{+} - \ket{-}\right)$
            \end{itemize}

            \item \textbf{Substitute into \ket{00} and \ket{{11}}}. We compute the tensor products using the expression above.
            \begin{itemize}
                \item Compute \ket{00}:
                \begin{equation*}
                    \begin{array}{rcl}
                        \ket{00} &=& \ket{0} \otimes \ket{0} \\ [.8em]
                        &=& \left( \dfrac{1}{\sqrt{2}} \left(\ket{+} + \ket{-}\right) \right)
                            \otimes
                            \left( \dfrac{1}{\sqrt{2}} \left(\ket{+} + \ket{-}\right) \right) \\ [1.5em]
                        &=& \dfrac{1}{2} \left(\ket{++} + \ket{+-} + \ket{-+} + \ket{--}\right)
                    \end{array}
                \end{equation*}
                \item Compute \ket{11}:
                \begin{equation*}
                    \begin{array}{rcl}
                        \ket{11} &=& \ket{1} \otimes \ket{1} \\ [.8em]
                        &=& \left( \dfrac{1}{\sqrt{2}} \left(\ket{+} - \ket{-}\right) \right)
                            \otimes
                            \left( \dfrac{1}{\sqrt{2}} \left(\ket{+} - \ket{-}\right) \right) \\ [1.5em]
                        &=& \dfrac{1}{2} \left(\ket{++} - \ket{+-} - \ket{-+} + \ket{--}\right)
                    \end{array}
                \end{equation*}
            \end{itemize}

            \item \textbf{Add the two to get $\ket{\Phi^{+}}$}. Now add the two results:
            \begin{equation*}
                \ket{\Phi^{+}} = \dfrac{1}{\sqrt{2}}\left(\ket{00} + \ket{11}\right)
            \end{equation*}
            Substituting:
            \begin{equation*}
                \begin{array}{rcl}
                    \ket{\Phi^{+}} &=& \dfrac{1}{\sqrt{2}} \left[
                        \dfrac{1}{2} \left( \ket{++} + \ket{+-} + \ket{-+} + \ket{--} \right)
                    \right. \\ [1.5em]
                    && \left.
                    \phantom{\dfrac{1}{2}}\hspace{1.2em} + \dfrac{1}{2} \left(\ket{++} - \ket{+-} - \ket{-+} + \ket{--}\right)
                    \right] \\ [1.5em]
                    &=& \dfrac{1}{\sqrt{2}} \cdot \dfrac{1}{2} \left( 2\ket{++} + 0\ket{+-} + 0\ket{-+} + 2\ket{--} \right) \\ [1.5em]
                    &=& \dfrac{1}{\sqrt{2}} \cdot \dfrac{1}{2} \left( 2\ket{++} + 2\ket{--} \right) \\ [1.5em]
                    &=& \dfrac{1}{\sqrt{2}} \left(\ket{++} + \ket{--}\right)
                \end{array}
            \end{equation*}
        \end{enumerate}
        Thus, in the Hadamard basis, we have:
        \begin{equation*}
            \ket{\Phi^{+}} = \dfrac{1}{\sqrt{2}}\left(\ket{++} + \ket{--}\right)
        \end{equation*}
        Which is a Bell-like state expressed in the Hadamard basis instead of the computational basis.
    \end{proof}
\end{examplebox}