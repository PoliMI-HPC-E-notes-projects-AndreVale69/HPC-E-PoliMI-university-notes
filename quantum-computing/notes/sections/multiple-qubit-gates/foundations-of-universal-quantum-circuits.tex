\subsection{Foundations of Universal Quantum Circuits}

To implement arbitrary quantum computations, we must answer two fundamental questions:
\begin{enumerate}
    \item \textbf{Which gates are sufficient} to build any quantum circuit?
    \item \textbf{How do we mathematically model} a circuit built from those gates?
\end{enumerate}
This section addresses both: it first introduces the concept of \textbf{universal gate sets}, and then explains how to formally represent quantum circuits as \textbf{matrices} acting on the full system state.

\highspace
\begin{flushleft}
    \textcolor{Green3}{\faIcon{book} \textbf{Universal Quantum Gate Sets}}
\end{flushleft}
Quantum computation occurs within the framework of unitary transformations on quantum states. However, since the space of all unitary operations is infinite (continuous), we cannot construct every possible unitary matrix exactly with a finite number of gates. Instead, we aim for \textbf{approximate universality}: the ability to approximate any unitary operation to arbitrary precision using a small set of elementary gates.

\highspace
A \definition{Universal Quantum Gate Set} is a finite collection of quantum gates from which \textbf{any unitary operation} on any number of qubits can be approximated arbitrarily well.

\begin{examplebox}
    One of the most commonly used universal sets includes:
    \begin{equation*}
        \left\{H, T, \text{CNOT}\right\}
    \end{equation*}
    \begin{itemize}
        \item $H$: Hadamard gate, creates superposition
        \item $T$: $\dfrac{\pi}{8}$ phase gate, a non-Clifford gate necessary for universality
        \item CNOT: a two-qubit entangling gate
    \end{itemize}
    With these gates alone, it is possible to approximate any quantum circuit. This is backed by the Solovay-Kitaev, which ensures efficient approximation of unitaries using such a discrete set.
\end{examplebox}

\noindent
The set of Clifford gates alone is \textbf{not universal}. To achieve universality, we must include \textbf{at least one non-Clifford gate}, like $T$ gate. Without it, the resulting circuits can be efficiently simulated classically, and therefore cannot exhibit true quantum advantage.

\newpage

\begin{flushleft}
    \textcolor{Green3}{\faIcon{book} \textbf{Modelling Quantum Circuits as Matrices}}
\end{flushleft}
Quantum circuits are not only visual tools; they correspond to precise mathematical transformations. Every complete circuit defines a \textbf{unitary matrix} acting on the system's joint state vector. Understanding how to construct this matrix from a sequence of gates is crucial for analysis and simulation.

\highspace
\textbf{When building the matrix of a circuit composed of multiple gates and qubits, we follow three formal rules}:
\begin{enumerate}
    \item \important{Tensor product across qubits}. Gates applied simultaneously on different qubits are composed using the \textbf{tensor product}. For example:
    \begin{equation*}
        A \text{ on qubit 1}, \quad B \text{ on qubit 2} \Rightarrow A \otimes B
    \end{equation*}
    
    \item \important{Matrix product along time (sequential composition)}. Gates applied \textbf{in sequence} (over time) are composed via \textbf{matrix multiplication}, applied \textbf{from right to left}. This reflects the order of function composition.
    
    \item \important{Identity padding for untouched qubits}. If a gate acts only on some of the qubits, we insert the \textbf{identity matrix} $I$ for those that are not affected, to preserve the tensor product structure.
\end{enumerate}