\subsection{Generic Controlled Gate}

After introducing the CNOT gate as controlled-X operation, we generalize the idea to any unitary gate. In quantum computing, we often want to \textbf{apply a certain gate} $U$ \textbf{to a target qubit} $v_{B}$, but \textbf{only if the control qubit} $v_{A}$ \textbf{is in state} \ket{1}. This leads us to the concept of a generic controlled gate, usually denoted $CU$ or $C_{U}$.

\highspace
The \definition{Generic Controlled Gate} has the matrix form:
\begin{equation}
    CU = C_{U} = \begin{bmatrix}
        I & 0 \\
        0 & U
    \end{bmatrix}
\end{equation}
This matrix acts on a two-qubit system, where:
\begin{itemize}
    \item $I$ is the $2 \times 2$ \textbf{identity} matrix
    \item $U$ is the \emph{any} \textbf{single-qubit unitary gate} (Pauli-X, Z, Y, ...)
    \item The \textbf{upper block} (top-left $2 \times 2$ block) acts when the \textbf{control qubit} is \ket{0}
    \item The \textbf{lower block} (bottom-right $2 \times 2$ block) applies $U$ when the \textbf{control qubit} is \ket{1}
\end{itemize}
This form is \textbf{not separable}, so the operator $C_{U}$ cannot be written as a tensor product of two individual single-qubit gates: $CU \neq A \otimes B$. The reason is simple:
\begin{itemize}
    \item A \textbf{tensor product} like $I \otimes U$ \textbf{always} applies $U$ to the second qubit \textbf{regardless} of the state of the first qubit.
    \item But $C_{U}$ applies $U$ to the second qubit \textbf{only if the first qubit is} \ket{1}.
\end{itemize}
So $C_{U}$ is a \textbf{conditional operation} that links the behavior of one qubit to the value of the other.

\highspace
\begin{flushleft}
    \textcolor{Green3}{\faIcon{tools} \textbf{How it works}}
\end{flushleft}
Suppose we have two qubits:
\begin{equation*}
    v_{A} = a_{0} \ket{0} + a_{1} \ket{1} \hspace{2em} v_{B} = b_{0} \ket{0} + b_{1} \ket{1}
\end{equation*}
Their joint state is:
\begin{equation*}
    v_{AB} = v_{A} \otimes v_{B} = \begin{bmatrix}
        a_0 b_0 \\
        a_0 b_1 \\
        a_1 b_0 \\
        a_1 b_1
    \end{bmatrix}
\end{equation*}
Then the result of applying the generic controlled gate is:
\begin{equation*}
    C_{U} \left( v_{A} \otimes v_{B} \right) =
    \begin{bmatrix}
        I & 0 \\
        0 & U
    \end{bmatrix}
    \begin{bmatrix}
        a_0 b_0 \\
        a_0 b_1 \\
        a_1 b_0 \\
        a_1 b_1
    \end{bmatrix}
\end{equation*}
This matrix selectively applies $U$ to the last two components (i.e., when $v_{A} = 1$), leaving the first two unchanged.
\begin{enumerate}
    \item \important{Control is \ket{0}}. If $v_{A} = \ket{0}$, then:
    \begin{itemize}
        \item $a_{0} = 1$
        \item $a_{1} = 0$
    \end{itemize}
    Consequently:
    \begin{equation*}
        v_{AB} = \begin{bmatrix}
            a_0 b_0 \\
            a_0 b_1 \\
            a_1 b_0 \\
            a_1 b_1
        \end{bmatrix}
        =
        \begin{bmatrix}
            1 \cdot b_0 \\
            1 \cdot b_1 \\
            0 \cdot b_0 \\
            0 \cdot b_1
        \end{bmatrix}
        =
        \begin{bmatrix}
            b_0 \\
            b_1 \\
              0 \\
              0
        \end{bmatrix}
    \end{equation*}
    Using the generic controlled gate, there is \textbf{no change}:
    \begin{equation*}
        C_{U}\left(v_{AB}\right) =
        \begin{bmatrix}
            I \cdot \begin{matrix} b_{0} \\ b_{1} \end{matrix} \\
            U \cdot \begin{matrix} 0 \\ 0 \end{matrix}
        \end{bmatrix}
    \end{equation*}
    Because the upper block is multiplied by the identity matrix. So the generic controlled gate \textbf{does nothing} when the control qubit is in state \ket{0}.

    \item \important{Control is \ket{1}}. If $v_{A} = \ket{1}$, then:
    \begin{itemize}
        \item $a_{0} = 0$
        \item $a_{1} = 1$
    \end{itemize}
    Now:
    \begin{equation*}
        v_{AB} = \begin{bmatrix}
            a_0 b_0 \\
            a_0 b_1 \\
            a_1 b_0 \\
            a_1 b_1
        \end{bmatrix}
        =
        \begin{bmatrix}
            0 \cdot b_0 \\
            0 \cdot b_1 \\
            1 \cdot b_0 \\
            1 \cdot b_1
        \end{bmatrix}
        =
        \begin{bmatrix}
              0 \\
              0 \\
            b_0 \\
            b_1
        \end{bmatrix}
    \end{equation*}
    Using the generic controlled gate:
    \begin{equation*}
        C_{U}\left(v_{AB}\right) =
        \begin{bmatrix}
            I \cdot \begin{matrix} 0 \\ 0 \end{matrix} \\
            U \cdot \begin{matrix} b_{0} \\ b_{1} \end{matrix}
        \end{bmatrix}
    \end{equation*}
    So the gate $U$ is \textbf{only applied to the target qubit when the control qubit is} \ket{1}.
\end{enumerate}
