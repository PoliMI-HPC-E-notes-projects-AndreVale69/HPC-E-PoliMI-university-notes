\subsection{Introduction to Multiple Qubit Gates}

In quantum computing, the transition from single-qubit to multi-qubit systems represents a fundamental leap in computational expressiveness. While single-qubit gates manipulate isolated quantum bits, the true power of quantum computation emerges only when we consider operations on multiple qubits. This is because \textbf{multi-qubit gates allow for the creation and manipulation of quantum entanglement}, a uniquely quantum phenomenon with no classical analog, and a critical resource for quantum advantage.

\highspace
The behavior of multi-qubit systems is governed by the \textbf{tensor product} structure of quantum mechanics. That is, the state space of a system composed of multiple qubits is described by the tensor product of the state spaces of the individual qubits. For example, a system of two qubits is represented by a vector in a four-dimensional Hilbert space $ \mathbb{C}^{2} \otimes \mathbb{C}^{2} = \mathbb{C}^{4} $, and similarly, an $n$-qubit system is described by a vector in $ \mathbb{C}^{2^{n}} $. This \textbf{exponential growth in state space dimensionality is one of the key reasons why quantum computers can}, in principle, \textbf{outperform classical ones for certain tasks}.

\highspace
\textbf{Multi-qubit gates} are therefore defined as \textbf{unitary transformations acting on these higher-dimensional vector spaces}. A simple extension of single-qubit operations to the multi-qubit context is given by applying gates in a factorized way (e.g., applying a Hadamard gate on the first qubit and an identity on the second corresponds to $ H \otimes I $). However, more powerful and interesting operations arise when gates do not decompose as tensor products of single-qubit gates. These non-separable gates, such as the Controlled-NOT (CNOT) gate, are capable of generating entangled states, and as such, are essential for universal quantum computation.

\highspace
Thus, the \textbf{study of multi-qubit gates} is not merely a generalization of single-qubit logic; it is the \textbf{gateway to fundamentally quantum phenomena}. Understanding these gates, their representations, and their implications is central to designing quantum algorithms and circuits that go beyond classical capabilities.