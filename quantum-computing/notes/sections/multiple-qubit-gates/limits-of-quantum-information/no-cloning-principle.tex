\subsection{Limits of Quantum Information}

\subsubsection{No-Cloning Principle}

The \definition{No-Cloning Theorem} states that it is \textbf{impossible to create an exact copy of an arbitrary unknown quantum state}. This is not a limitation of current technology, it is a fundamental result derived directly from the \textbf{linearity of quantum mechanics}.

\highspace
\begin{flushleft}
    \textcolor{Green3}{\faIcon{question-circle} \textbf{Why does it matter?}}
\end{flushleft}
The no-cloning principle is critical because it prevents:
\begin{enumerate}
    \item \important{Violations of quantum mechanics}: cloning would break the rules of \textbf{unitarity} and \textbf{linearity} of quantum evolution.

    \item \important{No-signaling violation} (section \ref{subsubsection: no-signaling principle} in depth, page \pageref{subsubsection: no-signaling principle}): if cloning were possible, we could \textbf{instantly send information} across large distances, violating causality.
    
    \item \important{Infinite data compression}: in theory, we could encode an arbitrarily long classical string into a quantum state and clone it infinitely to retrieve the data using quantum tomography.

    \item \important{Quantum error correction becomes harder}: many classical redundancy techniques rely on copying data, which is not possible in quantum mechanics.
\end{enumerate}

\highspace
\begin{flushleft}
    \textcolor{Green3}{\faIcon{check-circle} \textbf{Formal Proof}}
\end{flushleft}
Let's try to define a hypothetical \textbf{cloning unitary operator} $U$. We assume:
\begin{equation*}
    U \left( \ket{x} \otimes \ket{0} \right) = \ket{x} \otimes \ket{x}
\end{equation*}
This says: \emph{take any qubit \ket{x} and a blank \ket{0}, and get two \ket{x}}. Now take two arbitrary qubits \ket{x} and \ket{y}, and apply the same gate to both:
\begin{equation*}
    U\ket{x}\ket{0} = \ket{x}\ket{x} \quad U\ket{y}\ket{0} = \ket{y}\ket{y}
\end{equation*}
Since $U$ is \textbf{a unitary operator}, it \textbf{preserves inner products} and the following statement \underline{should} be true:
\begin{equation*}
    \left\langle x \otimes 0 | y \otimes 0 \right\rangle = \left\langle x \otimes x | y \otimes y \right\rangle
\end{equation*}
Let's compute both sides:
\begin{itemize}
    \item Left side, before cloning:
    \begin{equation*}
        \begin{array}{rcl}
            \left\langle x \otimes 0 | y \otimes 0 \right\rangle &=& \left\langle x | y \right\rangle \cdot \left\langle 0 | 0 \right\rangle \\ [.5em]
            &=& \left\langle x | y \right\rangle \cdot 1 \\ [.5em]
            &=& \left\langle x | y \right\rangle
        \end{array}
    \end{equation*}

    \item Right side, after cloning:
    \begin{equation*}
        \begin{array}{rcl}
            \left\langle x \otimes x | y \otimes y \right\rangle &=& \left\langle x | y \right\rangle \cdot \left\langle x | y \right\rangle \\ [.5em]
            &=& \left\langle x | y \right\rangle^{2}
        \end{array}
    \end{equation*}
\end{itemize}
So now're claiming:
\begin{equation*}
    \left\langle x | y \right\rangle = \left\langle x | y \right\rangle^{2}
\end{equation*}
Let's analyze this equation:
\begin{itemize}
    \item Let $\alpha = \left\langle x | y \right\rangle \in \mathbb{C}$
    \item Then $\alpha = \alpha^{2} \Rightarrow \alpha-\alpha^{2} = 0 \Rightarrow \alpha\left(\alpha - 1\right) = 0$
\end{itemize}
So this is only true if $\alpha = 0$ or $\alpha = 1$:
\begin{itemize}
    \item The states are \textbf{identical}; same direction, same vector:
    \begin{equation*}
        \left\langle x | y \right\rangle = 1
    \end{equation*}
    \item The states are \textbf{orthogonal}; completely distinguishable:
    \begin{equation*}
        \left\langle x | y \right\rangle = 0
    \end{equation*}
\end{itemize}
Unfortunately, for \textbf{arbitrary quantum states}, the inner product is \textbf{almost always} somewhere between 0 and 1. Only in \textbf{special cases} (perfect alignment or perfect orthogonality) is it exactly 0 or 1. That's why we say:
\begin{equation*}
    \left\langle x | y \right\rangle \ne 0,1 \text{ in general}
\end{equation*}
This is \textbf{exactly why the no-cloning proof works}, because it \textbf{fails for general} (non-orthogonal) \textbf{input states}, which are the rule rather than the exception.

\highspace
\begin{flushleft}
    \textcolor{Red2}{\faIcon{exclamation-triangle} \textbf{Apparent Violation via CNOT}}
\end{flushleft}
Let's explore a common misconception. We apply a CNOT gate with:
\begin{itemize}
    \item Control qubit $v_{A} = a_{0}\ket{0} + a_{1}\ket{1}$
    \item Target qubit $v_{B} = \ket{0}$
\end{itemize}
Naively, one might write:
\begin{equation*}
    \text{CNOT}\left(v_{A}, v_{B}\right) = v_{A} \otimes \left(v_{A} \oplus 0\right) = v_{A} \otimes v_{A}
\end{equation*}
It \textbf{looks like we cloned} $v_{A}$! But this is \underline{incorrect} because:
\begin{itemize}
    \item That informal rule $v_{A} \oplus v_{B}$ \textbf{only works for classical basis states} (\ket{0} or \ket{1}).
    \item When $v_{A}$ is in a \textbf{superposition}, this model breaks.
\end{itemize}

\newpage

\noindent
What really happens is:
\begin{enumerate}
    \item Initial state:
    \begin{equation*}
        v_{A} \otimes \ket{0} = a_{0}\ket{00} + a_{1}\ket{10}
    \end{equation*}
    \item Apply CNOT (flip second bit if control is 1):
    \begin{equation*}
        \Rightarrow a_{0}\ket{00} + a_{1}\ket{11}
    \end{equation*}
\end{enumerate}
This is \textbf{not} a clone! It's an \textbf{entangled state}, we cannot factor it as $\ket{\psi} \otimes \ket{\psi}$. Compare with actual cloned state:
\begin{equation*}
    \left(a_{0}\ket{0} + a_{1}\ket{1}\right) \otimes \left(a_{0}\ket{0} + a_{1}\ket{1}\right)
    =
    a_{0}^{2}\ket{00} + a_{0}a_{1}\ket{01} + a_{0}a_{1}\ket{10} + a_{1}^{2}\ket{11}
\end{equation*}
They're \textbf{clearly different}: cloning would produce 4 terms, but CNOT produces only 2. Thus CNOT \textbf{does not} clone a general qubit. It \textbf{creates entanglement}, not copies.