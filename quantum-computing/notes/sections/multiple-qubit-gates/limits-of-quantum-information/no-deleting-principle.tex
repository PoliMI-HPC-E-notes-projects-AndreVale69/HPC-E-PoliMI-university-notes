\subsubsection{No-Deleting Principle}

The \definition{No-Deleting Theorem} states that it is \textbf{impossible to delete an arbitrary unknown quantum state}. In other words, \textbf{there does not exist a unitary operator} $U$ that can take a pair of identical qubits $\ket{x} \otimes \ket{x}$ and produce:
\begin{equation*}
    U\left(\ket{x} \otimes \ket{x}\right) = \ket{0} \otimes \ket{x}
\end{equation*}
This operation would ``delete'' the first copy of \ket{x}, replacing it with a standard \ket{0}, while leaving the second untouched.

\highspace
\begin{flushleft}
    \textcolor{Green3}{\faIcon{question-circle} \textbf{Why is deletion a problem?}}
\end{flushleft}
If we could delete an arbitrary state, we could:
\begin{itemize}
    \item Circumvent information preservation in quantum systems (which must be \textbf{unitary and reversible}).
    \item Violate \textbf{quantum linearity}, much like cloning would.
    \item Enable undesirable operations that break the \textbf{symmetry and reversibility} of quantum evolution.
\end{itemize}

\highspace
\begin{flushleft}
    \textcolor{Green3}{\faIcon{check-circle} \textbf{Formal Proof}}
\end{flushleft}
Let's assume that such a deleting unitary $U$ exists. We define:
\begin{equation*}
    U\left(\ket{x} \otimes \ket{x}\right) = \ket{0} \otimes \ket{x} \quad U\left(\ket{y} \otimes \ket{y}\right) = \ket{0} \otimes \ket{y}
\end{equation*}
Take the inner product between both results:
\begin{itemize}
    \item Left side, \emph{before} deleting:
    \begin{equation*}
        \left\langle x | \left\langle x | \cdot U^{\dagger}U \cdot | y \right\rangle | y \right\rangle
    \end{equation*}
    Because $U^{\dagger} U = I$, and inner products distribute:
    \begin{equation*}
        = \left\langle x | y \right\rangle \cdot \left\langle x | y \right\rangle = \left\langle x | y \right\rangle^{2}
    \end{equation*}

    \item Right side, \emph{after} deleting:
    \begin{equation*}
        \left\langle x | 0 \right\rangle \cdot \left\langle y | 0 \right\rangle = \left\langle x | y \right\rangle
    \end{equation*}
    So now we're saying:
    \begin{equation*}
        \left\langle x | y \right\rangle^{2} = \left\langle x | y \right\rangle \quad \Rightarrow \quad \left\langle x | y \right\rangle \cdot \left(\left\langle x | y \right\rangle - 1\right) = 0
    \end{equation*}
    So the only possible solutions are:
    \begin{equation*}
        \left\langle x | y \right\rangle = 0 \quad \text{(orthogonal)} \quad \text{or} \quad \left\langle x | y \right\rangle = 1 \quad \text{(identical)}
    \end{equation*}
\end{itemize}
But for \textbf{general quantum states}, $\left\langle x | y \right\rangle \notin \left\{0,1\right\}$, so we've reached a \textbf{contradiction}. Therefore, \textbf{no such deletion operator $U$ can exist for arbitrary stats}.

\highspace
\begin{flushleft}
    \textcolor{Green3}{\faIcon{book} \textbf{Physical Meaning}}
\end{flushleft}
The no-deleting principle complements the \textbf{no-cloning principle}:
\begin{itemize}
    \item We can't \textbf{make} copies of unknown quantum states (no-cloning)
    \item We can't \textbf{erase} arbitrary copies either (no-deleting)
\end{itemize}
These theorems emphasize that \textbf{quantum information is conserved}:
\begin{itemize}
    \item It cannot be copied freely
    \item It cannot be erased freely
    \item It can only be \textbf{transferred} (e.g., through teleportation or interaction)
\end{itemize}