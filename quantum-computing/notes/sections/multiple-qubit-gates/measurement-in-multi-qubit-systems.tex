\subsection{Measurement in Multi-Qubit Systems}

We begin with a general two-qubit state:
\begin{equation*}
    \ket{v_{C}} = \ket{v_{A} v_{B}} = c_{0}\ket{00} + c_{1}\ket{01} + c_{2} \ket{10} + c_{3} \ket{11}
\end{equation*}
This is an entangled or general superposed state over the two-qubit basis. But the question is: \textbf{\emph{What happens to the state of the system if we measure \underline{only} qubit A?}}

\highspace
\begin{flushleft}
    \textcolor{Green3}{\faIcon{book} \textbf{Decomposition Strategy}}
\end{flushleft}
The idea is to \textbf{separate the global state} into parts conditioned on the state of qubit $A$:
\begin{equation*}
    \ket{v} = c_{01} \ket{0} \otimes \dfrac{c_{0}\ket{0} + c_{1}\ket{1}}{c_{01}} + c_{23}\ket{1} \otimes \dfrac{c_{2}\ket{0} + c_{3}\ket{1}}{c_{23}}
\end{equation*}
Where:
\begin{itemize}
    \item $c_{01} = \sqrt{\left|c_{0}\right|^{2} + \left|c_{1}\right|^{2}}$
    \item $c_{23} = \sqrt{\left|c_{2}\right|^{2} + \left|c_{3}\right|^{2}}$
\end{itemize}
This expression means: the total state is a \textbf{superposition of two conditional branches}:
\begin{itemize}
    \item One where qubit $A$ is \ket{0}, and qubit $B$ is in a normalized superposition of \ket{0}, \ket{1}.
    \item One where qubit $A$ is \ket{1}, and qubit $B$ is in another normalized state.
\end{itemize}

\highspace
\begin{flushleft}
    \textcolor{Green3}{\faIcon{question-circle} \textbf{What happens after a Measurement?}}
\end{flushleft}
\begin{enumerate}
    \item Case 1. \important{Qubit $A$ is measured as \ket{0}}:
    \begin{itemize}
        \item The outcome collapses the state: $c_{2} = 0$, $c_{3} = 0$
        \item The system is no longer a superposition, it's now in:
        \begin{equation*}
            \ket{v_{B}} = \dfrac{c_{0}\ket{0} + c_{1}\ket{1}}{c_{01}}
        \end{equation*}
        Which is a valid normalized single-qubit state (i.e., $\proj{v_{B}} = 1$)
    \end{itemize}

    \item Case 2. \important{Qubit $A$ is measured as \ket{1}}:
    \begin{itemize}
        \item Similarly, $c_{0} = 0$, $c_{1} = 0$
        \item New state of $B$ becomes:
        \begin{equation*}
            \ket{v_{B}} = \dfrac{c_{2} \ket{0} + c_{3}\ket{1}}{c_{23}}
        \end{equation*}
    \end{itemize}
\end{enumerate}
Measuring \textbf{one qubit} in an entangled system \textbf{instantaneously collapses the entire state}, even though \textbf{only one qubit} is directly observed.

\highspace
This feels strange because \textbf{measurement is local}, yet the \textbf{effect is nonlocal}, it determines the state of the other qubit, even if it's far away! This is a \textbf{manifestation of quantum entanglement} and is deeply \textbf{linked to} the phenomenon of \textbf{quantum nonlocality}.

\highspace
\begin{examplebox}[: Bell state]
    Let's apply this idea to the Bell state:
    \begin{equation*}
        \ket{v} = \dfrac{1}{\sqrt{2}} \left(\ket{00} + \ket{11}\right)
    \end{equation*}
    This can be rewritten as:
    \begin{equation*}
        \ket{v} = \dfrac{1}{\sqrt{2}} \left(\ket{0} \otimes \ket{0} + \ket{1} \otimes \ket{1}\right)
    \end{equation*}
    So we're in the entangled state $\ket{\Phi^{+}}$.

    \highspace
    Now, we measure qubit $A$. If we get \textbf{outcome}:
    \begin{itemize}
        \item \ket{0}, the global state collapses to \ket{00} $\Rightarrow$ qubit $B$ is now in state \ket{0}.
        \item \ket{1}, the state collapses to \ket{11} $\Rightarrow$ qubit $B$ is in state \ket{1}.
    \end{itemize}
\end{examplebox}