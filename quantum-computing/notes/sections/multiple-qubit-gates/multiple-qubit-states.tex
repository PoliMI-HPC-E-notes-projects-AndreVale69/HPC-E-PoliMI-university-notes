\section{Multiple Qubit Gates}

\subsection{Multiple Qubit States}

In quantum computing, the description of systems involving multiple qubits relies on a mathematical operation known as the \textbf{tensor product}. This operation enables us to \textbf{formally construct the joint state of two or more qubits starting from their individual states}.

\highspace
\begin{flushleft}
    \textcolor{Green3}{\faIcon{star} \textbf{The Tensor Product}}
\end{flushleft}
Suppose we have two qubits, each in its own superposition:
\begin{equation*}
    \ket{v_{A}} = a_{0} \ket{0} + a_{1} \ket{1} \hspace{2em} \text{and} \hspace{2em} \ket{v_{B}} = b_{0} \ket{0} + b_{1}\ket{1}
\end{equation*}
Our \textbf{goal} is to \textbf{describe the joint state of the system}, i.e., the probability amplitudes for:
\begin{itemize}
    \item Both qubits being in state \ket{0}
    \item Qubit $A$ in \ket{0} and qubit $B$ in \ket{1}
    \item Qubit $A$ in \ket{1} and qubit $B$ in \ket{0}
    \item Both in \ket{1}
\end{itemize}
To construct the combined state mathematically, we use the \definition{Tensor Product $\otimes$}. Give the vectors of $v_{A}$ and $v_{B}$:
\begin{equation*}
    \ket{v_{A}} = a_{0} \begin{bmatrix} 1 \\ 0 \end{bmatrix} + a_{1} \begin{bmatrix} 0 \\ 1 \end{bmatrix} = \begin{bmatrix} a_{0} \\ a_{1} \end{bmatrix}
    \hspace{2em}
    \ket{v_{B}} = b_{0} \begin{bmatrix} 1 \\ 0 \end{bmatrix} + b_{1} \begin{bmatrix} 0 \\ 1 \end{bmatrix} = \begin{bmatrix} b_{0} \\ b_{1} \end{bmatrix}
\end{equation*}
The tensor product $\ket{v_{A}} \otimes \ket{v_{B}}$ gives:
\begin{equation}
    \begin{array}{rcl}
        \ket{v_{A}} \otimes \ket{v_{B}} &=& \left( a_{0} \ket{0} + a_{1} \ket{1} \right) \otimes \left( b_{0} \ket{0} + b_{1} \ket{1} \right)   \\ [.8em]
                                        &=& \begin{bmatrix} a_{0} \\ a_{1} \end{bmatrix} \otimes \begin{bmatrix} b_{0} \\ b_{1} \end{bmatrix}   \\ [1.2em]
                                        &=& a_{0}b_{0} \ket{00} + a_{0}b_{1} \ket{01} + a_{1}b_{0} \ket{10} + a_{1}b_{1} \ket{11}               \\ [1em]
                                        &=& a_{0}b_{0} \begin{bmatrix} 1 \\ 0 \end{bmatrix} \otimes \begin{bmatrix} 1 \\ 0 \end{bmatrix} +
                                            a_{0}b_{1} \begin{bmatrix} 1 \\ 0 \end{bmatrix} \otimes \begin{bmatrix} 0 \\ 1 \end{bmatrix} +      \\ [1em]
                                        & & a_{1}b_{0} \begin{bmatrix} 0 \\ 1 \end{bmatrix} \otimes \begin{bmatrix} 1 \\ 0 \end{bmatrix} +
                                            a_{1}b_{1} \begin{bmatrix} 0 \\ 1 \end{bmatrix} \otimes \begin{bmatrix} 0 \\ 1 \end{bmatrix}        \\ [1.2em]
                                        &=& a_{0}b_{0} \begin{bmatrix} 1 \\ 0 \\ 0 \\ 0 \end{bmatrix}
                                            +
                                            a_{0}b_{1} \begin{bmatrix} 0 \\ 1 \\ 0 \\ 0 \end{bmatrix}
                                            +
                                            a_{1}b_{0} \begin{bmatrix} 0 \\ 0 \\ 1 \\ 0 \end{bmatrix}
                                            +
                                            a_{1}b_{1} \begin{bmatrix} 0 \\ 0 \\ 0 \\ 1 \end{bmatrix}                                           \\ [2.5em]
                                        &=& \begin{bmatrix} a_{0} b_{0} \\ a_{0} b_{1} \\ a_{1} b_{0} \\ a_{1} b_{1} \end{bmatrix}
    \end{array}
\end{equation}
This four-dimensional vector fully describes the joint system of two qubits in the computational basis. The tensor product \textbf{preserves the linearity} of the system and ensures that the \textbf{resulting state is still a valid quantum state}.

\highspace
About the notation, the ket \ket{00} can be explicitly rewritten as the tensor product $\ket{0} \otimes \ket{0}$. These two notations are the same.

\highspace
\begin{flushleft}
    \textcolor{Green3}{\faIcon{percent} \textbf{Measurement Probabilities}}
\end{flushleft}
Each coefficient in the final four-dimensional vector corresponds to an \textbf{amplitude for a specific measurement outcome}. For instance:
\begin{itemize}
    \item The probability of observing \ket{00} is $\left|a_{0}b_{0}\right|^{2}$
    \item The probability of observing \ket{01} is $\left|a_{0}b_{1}\right|^{2}$
    \item The probability of observing \ket{10} is $\left|a_{1}b_{0}\right|^{2}$
    \item The probability of observing \ket{11} is $\left|a_{1}b_{1}\right|^{2}$
\end{itemize}
This reflects the Born rule: the probability of an outcome is the square modulus of the corresponding amplitude (page \pageref{eq: Born rule}).

\highspace
\begin{flushleft}
    \textcolor{Green3}{\faIcon{book} \textbf{Normalization of Two-Qubit Quantum State}}
\end{flushleft}
The normalization of a quantum state means that the \textbf{total probability of all possible outcomes must sum to 1}. The reason we need normalization is due to the \definition{Born rule}: the probability of measuring a quantum state $\ket{\psi}$ and finding it in basis state $\ket{i}$ is $\left|c_{i}\right|^{2}$. Since total probability must equal 1, we must have:
\begin{equation*}
    \displaystyle\sum_{i} \left|c_{i}\right|^{2} = 1
\end{equation*}

\begin{proof}[Normalization of two-qubit quantum state]
    Given two normalized single-qubit states:
    \begin{equation*}
        \ket{v_{A}} = a_{0} \ket{0} + a_{1} \ket{1} \hspace{2em} \ket{v_{B}} = b_{0} \ket{0} + b_{1} \ket{1}
    \end{equation*}
    We know that (because they are normalized, and the total probability of all possible outcomes must sum to 1):
    \begin{equation*}
        \left|a_{0}\right|^{2} + \left|a_{1}\right|^{2} = 1 \hspace{2em} \left|b_{0}\right|^{2} + \left|b_{1}\right|^{2} = 1
    \end{equation*}
    The combined state, via the tensor product, is:
    \begin{equation*}
        \ket{v_{AB}} =  a_{0}b_{0}\ket{00} + a_{0}b_{1}\ket{01} + a_{1}b_{0}\ket{10} + a_{1}b_{1}\ket{11}
    \end{equation*}
    The question is: \textbf{does this state remain normalized?} We need to verify that the sum of the squared moduli (i.e., squared magnitudes) of the coefficients equals 1:
    \begin{equation*}
        \left|a_{0}b_{0}\right|^{2} + \left|a_{0}b_{1}\right|^{2} + \left|a_{1}b_{0}\right|^{2} + \left|a_{1}b_{1}\right|^{2} = 1
    \end{equation*}
    We use the property of modulus:
    \begin{equation*}
        \left|a_{i}b_{j}\right|^{2} = \left|a_{i}\right|^{2} \cdot \left|b_{j}\right|^{2}
    \end{equation*}

    \newpage

    \noindent
    Hence:
    \begin{itemize}
        \item $\left|a_{0} b_{0}\right|^{2} = \left|a_{0}\right|^{2} \left|b_{0}\right|^{2}$
        \item $\left|a_{0} b_{1}\right|^{2} = \left|a_{0}\right|^{2} \left|b_{1}\right|^{2}$
        \item $\left|a_{1} b_{0}\right|^{2} = \left|a_{1}\right|^{2} \left|b_{0}\right|^{2}$
        \item $\left|a_{1} b_{1}\right|^{2} = \left|a_{1}\right|^{2} \left|b_{1}\right|^{2}$
    \end{itemize}
    We regroup the terms:
    \begin{equation*}
        \left|a_{0}\right|^{2} \left(\left|b_{0}\right|^{2} + \left|b_{1}\right|^{2}\right) + \left|a_{1}\right|^{2} \left(\left|b_{0}\right|^{2} + \left|b_{1}\right|^{2}\right)
    \end{equation*}
    Factor out:
    \begin{equation*}
        \left(\left|a_{0}\right|^{2} + \left|a_{1}\right|^{2}\right) \cdot \left(\left|b_{0}\right|^{2} + \left|b_{1}\right|^{2}\right)
    \end{equation*}
    Since both $v_{A}$ and $v_{B}$ are normalized:
    \begin{equation*}
        \left|a_{0}\right|^{2} + \left|a_{1}\right|^{2} = 1 \hspace{2em} \left|b_{0}\right|^{2} + \left|b_{1}\right|^{2} = 1
    \end{equation*}
    So: $1 \cdot 1 = 1$.
\end{proof}

The combined two-qubit state $v_{AB} = v_{A} \otimes v_{B}$ is indeed \textbf{normalized}, provided the original qubits are normalized. This confirms that the \textbf{tensor product of two normalized states is itself a normalized state}, a crucial result that ensures consistency.