\subsection{SWAP Gate}

The \definition{SWAP gate} is a two-qubit gate that \textbf{exchanges the states of the two qubits}. That is, if qubit $A$ is in state $v_{A}$ and qubit $B$ in state $v_{B}$, then after applying SWAP, their roles are reversed.

\highspace
\begin{flushleft}
    \textcolor{Green3}{\faIcon{square-root-alt} \textbf{Matrix Representation}}
\end{flushleft}
The SWAP gate is defined by the following $4 \times 4$ unitary matrix:
\begin{equation}
    \begin{bmatrix}
        1 & 0 & 0 & 0 \\
        0 & 0 & 1 & 0 \\
        0 & 1 & 0 & 0 \\
        0 & 0 & 0 & 1
    \end{bmatrix}
\end{equation}
This matrix operates on the basis states in the usual order:
\begin{equation*}
    \left\{\ket{00}, \ket{01}, \ket{10}, \ket{11}\right\}
\end{equation*}
And swaps the coefficients of \ket{01} and \ket{10}, leaving \ket{00} and \ket{11} unchanged.

\highspace
\begin{flushleft}
    \textcolor{Green3}{\faIcon{tools} \textbf{How it works}}
\end{flushleft}
Suppose we have two qubits in superposition:
\begin{equation*}
    v_{A} = a_{0}\ket{0} + a_{1}\ket{1} \hspace{2em} v_{B} = b_{0}\ket{0} + b_{1}\ket{1}
\end{equation*}
Their joint state is:
\begin{equation*}
    v_{AB} = v_{A} \otimes v_{B} = a_{0}b_{0}\ket{00} + a_{0}b_{1}\ket{01} + a_{1}b_{0}\ket{10} + a_{1}b_{1}\ket{11}
\end{equation*}
After applying the SWAP gate:
\begin{equation*}
    \text{SWAP}\left(v_{AB}\right) = a_{0}b_{0}\ket{00} + a_{1}b_{0}\ket{01} + a_{0}b_{1}\ket{10} + a_{1}b_{1}\ket{11}
\end{equation*}
But this result can be obtained also doing $v_{B} \otimes v_{A}$:
\begin{equation*}
    v_{BA} = b_{0}a_{0}\ket{00} + b_{0}a_{1}\ket{01} + b_{1}a_{0}\ket{10} + b_{1}a_{1}\ket{11}
\end{equation*}
Note that the coefficients $a_{i}$ and $b_{j}$ are scalar: $a_{i} \cdot b_{j} = b_{j} \cdot a_{i}$. So the SWAP gate effectively implements:
\begin{equation}
    v_{A} \otimes v_{B} \xlongrightarrow{\text{SWAP gate}} v_{B} \otimes v_{A}
\end{equation}

\highspace
\begin{flushleft}
    \textcolor{Green3}{\faIcon{microchip} \textbf{Circuit Representation}}
\end{flushleft}
In circuit diagrams, the SWAP gate is often draw using \textbf{crossed lines with $\times$ symbols}:
\begin{center}
    \begin{quantikz}
        \lstick{$\ket{v_{A}}$} & \swap{1} & \rstick{$\ket{v_{B}}$} \\
        \lstick{$\ket{v_{B}}$} & \targX{} & \rstick{$\ket{v_{A}}$}
    \end{quantikz}
\end{center}

\highspace
\begin{flushleft}
    \textcolor{Green3}{\faIcon{star} \textbf{SWAP as Three CNOTs}}
\end{flushleft}
An important implementation detail is that \textbf{SWAP can be decomposed into three CNOT gates} (page \pageref{subsection: Controlled NOT Gate - CNOT}). This is crucial \textbf{because SWAP is not a native gate on most quantum hardware}, but CNOT often is.

\highspace
The decomposition is represented graphically as:
\begin{center}
    \begin{quantikz}
            \lstick{$\ket{v_{A}}$} & \ctrl{1}\gategroup[2,steps=3,style={dashed,rounded corners,fill=blue!20,inner xsep=2pt},background,label style={label position=below,anchor=north,yshift=-0.2cm}]{{\sc swap}} & \targ{} & \ctrl{1} & \rstick{$\ket{v_{B}}$} \\
            \lstick{$\ket{v_{B}}$} & \targ{} & \ctrl{-1} & \targ{} & \rstick{$\ket{v_{A}}$}
    \end{quantikz}
\end{center}
\begin{enumerate}
    \item First CNOT:
    \begin{itemize}
        \item Control qubit: $v_{A}$
        \item Target qubit: $v_{B}$
    \end{itemize}

    \item Second CNOT:
    \begin{itemize}
        \item Control qubit: $v_{B}$
        \item Target qubit: $v_{A}$
    \end{itemize}

    \item Third CNOT:
    \begin{itemize}
        \item Control qubit: $v_{A}$
        \item Target qubit: $v_{B}$
    \end{itemize}
\end{enumerate}

\begin{center}
    \begin{quantikz}
        \push{\ket{0}} & \ctrl{1} & \push{\ket{0}} & \targ{}   & \push{\ket{1}} & \ctrl{1} & \push{\ket{1}} \\
        \push{\ket{1}} & \targ{}  & \push{\ket{1}} & \ctrl{-1} & \push{\ket{1}} & \targ{}  & \push{\ket{0}}
    \end{quantikz}
\end{center}

\noindent
And:
\begin{center}
    \begin{quantikz}
        \push{\ket{1}} & \ctrl{1} & \push{\ket{1}} & \targ{}   & \push{\ket{0}} & \ctrl{1} & \push{\ket{0}} \\
        \push{\ket{0}} & \targ{}  & \push{\ket{1}} & \ctrl{-1} & \push{\ket{1}} & \targ{}  & \push{\ket{1}}
    \end{quantikz}
\end{center}