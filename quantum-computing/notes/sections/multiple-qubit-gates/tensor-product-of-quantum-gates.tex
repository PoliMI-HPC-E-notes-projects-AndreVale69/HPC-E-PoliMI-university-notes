\subsection{Tensor Product of Quantum Gates}

In a quantum system composed of multiple qubits, it is common to \textbf{apply gates independently and simultaneously to different qubits}. This operations is known as parallel execution, and it is described formally using the \definition{Tensor Product of Quantum Gates}.

\highspace
\begin{flushleft}
    \textcolor{Green3}{\faIcon{question-circle} \textbf{Motivation}}
\end{flushleft}
Suppose we have two qubits in individual states:
\begin{equation*}
    \ket{v_{A}} = a_{0}\ket{0} + a_{1}\ket{1} \hspace{2em} \ket{v_{B}} = b_{0}\ket{0} + b_{1}\ket{1}
\end{equation*}
The \textbf{joint state} of the system is represented using the tensor product (page \pageref{eq: Tensor Product}):
\begin{equation*}
    \ket{v_{AB}} = \ket{v_{A}} \otimes \ket{v_{B}}
\end{equation*}
Now assume we wish to apply:
\begin{itemize}
    \item A gate $A$ to the first qubit $\ket{v_{A}}$
    \item A gate $B$ to the second qubit $\ket{v_{B}}$
\end{itemize}
How do we describe this \emph{combined operation}?

\highspace
\begin{flushleft}
    \textcolor{Green3}{\faIcon{book} \textbf{Formal Definition}}
\end{flushleft}
The combined transformation on the system is represented by the \textbf{tensor product of the operators}:
\begin{equation*}
    \left(A \otimes B\right)\left(v_{A} \otimes v_{B}\right)
\end{equation*}
This is \textbf{not the same as applying} $A$ and $B$ \textbf{separately} and then multiplying the results. It's a structured, mathematically-defined operation where:
\begin{itemize}
    \item $A$ acts only on the first qubit
    \item $B$ acts only on the second qubit
\end{itemize}
The overall effect on the joint state $v_{AB}$ is captured by applying the composite operator $A \otimes B$.

\highspace
\begin{flushleft}
    \textcolor{Green3}{\faIcon{microchip} \textbf{Circuit Interpretation}}
\end{flushleft}
In a quantum circuit diagram, each horizontal line represent a qubit, and gates are applied along these lines. The parallel application of gates is captured mathematically by $A \otimes B$, and visually by \textbf{applying gates side-by-side in the circuit}.
\begin{center}
    \begin{quantikz}
        \lstick{\ket{v_{A}}} & \gate{A} & \rstick[2]{$\left(A \ket{v_{A}}\right) \otimes \left(B \ket{v_{B}}\right)$} \\
        \lstick{\ket{v_{B}}} & \gate{B} &
    \end{quantikz}
    =
    \begin{quantikz}
        \lstick{\ket{v_{A}}} & \gate[2]{A \otimes B} & \rstick[2]{$\left(A \otimes B\right)\ket{v_{AB}}$} \\
        \lstick{\ket{v_{B}}} &                       &
    \end{quantikz}
\end{center}

\highspace
\begin{flushleft}
    \textcolor{Green3}{\faIcon{book} \textbf{Applying a Gate to One Qubit}}
\end{flushleft}
Let's say we have two qubits:
\begin{equation*}
    \ket{v_{AB}} = v_{A} \otimes v_{B}
\end{equation*}
We want to apply:
\begin{itemize}
    \item No operation to qubit $A$ (i.e., we want it to stay as it is)
    \item Gate $B$ to qubit $B$
\end{itemize}
But since we're working in the \textbf{joint system}, we need to describe the total gate as acting on the full state vector in $\mathbb{C}^{4}$. We cannot apply gate $B$ directly on the 2nd qubit in isolation.

\highspace
To formalize ``\emph{do nothing}'' to qubit $A$, we use the \definition{Identity Gate}:
\begin{equation}
    I = \begin{bmatrix}
        1 & 0 \\
        0 & 1
    \end{bmatrix}
\end{equation}
Then, the correct total operation on the two-qubit state is:
\begin{equation*}
    \left(I \otimes B\right)\left(v_{A} \otimes v_{B}\right)
\end{equation*}
This notation ensures that:
\begin{itemize}
    \item $I$ acts on qubit $A$ and \textbf{leaves it unchanged}
    \item $B$ is applied to qubit $B$
    \item The full operator still acts on the \textbf{tensor product state space} $\mathbb{C}^{4}$
\end{itemize}
In the circuit diagram, this is shown as:
\begin{center}
    \begin{quantikz}
        \lstick{\ket{v_{A}}} & \ghost{I} & \rstick[2]{$\left(I \ket{v_{A}}\right) \otimes \left(B \ket{v_{B}}\right)$} \\
        \lstick{\ket{v_{B}}} & \gate{B} &
    \end{quantikz}
    =
    \begin{quantikz}
        \lstick{\ket{v_{A}}} & \gate[2]{I \otimes B} & \rstick[2]{$\left(I \otimes B\right)\ket{v_{AB}}$} \\
        \lstick{\ket{v_{B}}} &                       &
    \end{quantikz}
\end{center}
Even if we don't draw the identity gate, it is \textbf{implicitly applied} to all untouched qubits, because we are still working in the joint space.

\highspace
\begin{flushleft}
    \textcolor{Green3}{\faIcon{question-circle} \textbf{Hadamard Transform on Multiple Qubits}}
\end{flushleft}
In Section \ref{subsubsection: Hadamard Gate H} page \pageref{subsubsection: Hadamard Gate H}, we saw how to apply the Hadamard operation to a single qubit system. Now that we are on multiple qubit systems, we want to understand how to use the Hadamard operation.

\highspace
The \definition{Hadamard Gate (H)} creates superposition states. When applied to each qubit in a system, it \textbf{transforms a basis state into a uniform superposition of all possible states}. To apply the Hadamard gate to \textbf{two qubits in parallel}, we compute the \textbf{tensor product of two Hadamard matrices}:
\begin{equation*}
    H^{\otimes 2} = H \otimes H
\end{equation*}
This results in a $4 \times 4$ matrix:
\begin{equation*}
    H^{\otimes 2} = H \otimes H = \dfrac{1}{2}
    \begin{bmatrix}
        1 & 1 & 1 & 1 \\
        1 & -1 & 1 & -1 \\
        1 & 1 & -1 & -1 \\
        1 & -1 & -1 & 1
    \end{bmatrix}
\end{equation*}
For example, if we want to apply the Hadamard gate to the state \ket{00}, mathematically we have:
\begin{equation*}
    H^{\otimes 2} \ket{00}
    =
    \dfrac{1}{2} \begin{bmatrix} 1 \\ 1 \\ 1 \\ 1 \end{bmatrix}
    =
    \dfrac{1}{2} \left(\ket{00} + \ket{01} + \ket{10} + \ket{11}\right)
\end{equation*}
This is a \textbf{uniform superposition over all possible two-qubit basis states}. From a circuit point of view:
\begin{center}
    \begin{quantikz}
        \lstick{\ket{0}} & \gate{H} & \rstick[2]{$\left(H \otimes H\right)\ket{00}$} \\
        \lstick{\ket{0}} & \gate{H} & 
    \end{quantikz}
\end{center}

\noindent
The \textbf{Hadamard operation can be generalized} to $n$ qubits:
\begin{equation}
    H^{\otimes n} \ket{0}^{\otimes n} = \dfrac{1}{\sqrt{2^{n}}} \displaystyle\sum_{i=0}^{2^{n}-1} \ket{i}
\end{equation}
This creates a uniform quantum state across all $2^{n}$ possibilities. Something \textbf{impossible in classical computing} without explicitly listing all combinations.