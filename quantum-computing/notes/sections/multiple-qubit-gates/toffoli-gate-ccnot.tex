\subsection{Toffoli Gate (CCNOT)}

The \definition{Toffoli Gate}, or \definition{CCNOT (Controlled-Controlled-NOT)}, is a \textbf{three-qubit gate} where \textbf{two qubits act as controls} and \textbf{one as the target}. It generalizes the behavior of the CNOT gate by requiring \textbf{both control qubits} to be in the state \ket{1} before the NOT operation is applied to the target.

\highspace
\begin{flushleft}
    \textcolor{Green3}{\faIcon{square-root-alt} \textbf{Matrix Representation}}
\end{flushleft}
The matrix form of the Toffoli gate is an $8 \times 8$ identity matrix with only two elements swapped:
\begin{equation*}
    \text{CCNOT} = \begin{bmatrix}
        1 & 0 & 0 & 0 & 0 & 0 & 0 & 0 \\
        0 & 1 & 0 & 0 & 0 & 0 & 0 & 0 \\
        0 & 0 & 1 & 0 & 0 & 0 & 0 & 0 \\
        0 & 0 & 0 & 1 & 0 & 0 & 0 & 0 \\
        0 & 0 & 0 & 0 & 1 & 0 & 0 & 0 \\
        0 & 0 & 0 & 0 & 0 & 1 & 0 & 0 \\
        0 & 0 & 0 & 0 & 0 & 0 & 0 & 1 \\
        0 & 0 & 0 & 0 & 0 & 0 & 1 & 0
    \end{bmatrix}
\end{equation*}
This matrix swpas the \ket{110} $\leftrightarrow$ \ket{111} entries. In other words, all basis states remain unchanged except:
\begin{itemize}
    \item \ket{110} becomes \ket{111}
    \item \ket{111} becomes \ket{110}
\end{itemize}

\highspace
\begin{flushleft}
    \textcolor{Green3}{\faIcon{question-circle} \textbf{Gate Behavior}}
\end{flushleft}
The CCNOT gate can be described as follows:
\begin{itemize}
    \item \textbf{Inputs}: $v_{0}, v_{1}$ as \emph{control qubits}, and $v_{2}$ as \emph{target qubit}.
    \item \textbf{Output}:
    \begin{equation*}
        v_{2} \rightarrow v_{2} \oplus \left(v_{0} \land v_{1}\right)
    \end{equation*}
\end{itemize}
This means:
\begin{itemize}
    \item If $v_{0}$ logically combined with $v_{1}$ returns true, then flip $v_{2}$. In other words, flip if and only if $v_{0}$ and $v_{1}$ are 1.
    \item Otherwise, leave $v_{2}$ unchanged.
\end{itemize}
This behavior is \textbf{nonlinear} in the classic sense, and it's what makes Toffoli especially powerful: it can \textbf{simulate universal classical logic} within a quantum system. So any logic circuit that we can build using AND, OR and NOT gates, can also be implemented using only Toffoli gates.

\newpage

\begin{flushleft}
    \textcolor{Green3}{\faIcon{microchip} \textbf{Circuit Notation}}
\end{flushleft}
In quantum circuit diagrams, the CCNOT gate is drawn with:
\begin{itemize}
    \item Two control dots on the top two lines (for $v_{0}, v_{1}$)
    \item A $\oplus$ symbol on the third line (for the target qubit $v_{2}$)
    \item Vertical lines connecting the three
\end{itemize}
This shows the conditional behavior clearly: the \textbf{NOT is triggered only when both controls are active}.

\begin{center}
    \begin{quantikz}
        \lstick{$\ket{v_{0}}$} & \ctrl{1} & \rstick{$\ket{v_{0}}$} \\
        \lstick{$\ket{v_{1}}$} & \ctrl{1} & \rstick{$\ket{v_{1}}$} \\
        \lstick{$\ket{v_{2}}$} & \targ{}  & \rstick{$\ket{v_{2}} \oplus \ket{v_{0}} \land \ket{v_{1}}$}
    \end{quantikz}
\end{center}