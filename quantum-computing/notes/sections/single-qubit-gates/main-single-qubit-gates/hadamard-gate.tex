\subsubsection{Hadamard Gate (H)}\label{subsubsection: Hadamard Gate H}

The \definitionWithSpecificIndex{Hadamard Gate (H)}{Single Qubit Gates: Hadamard Gate (S)}{} \textbf{transforms a qubit from a definite state} ($\ket{0}$ or $\ket{1}$) \textbf{into a superposition of both}. It is the \textbf{gateway to quantum parallelism}, allowing quantum computers to \textbf{process multiple possibilities simultaneously}.

\highspace
\begin{flushleft}
    \textcolor{Green3}{\faIcon{square-root-alt} \textbf{Matrix Representation}}
\end{flushleft}
\begin{equation}
    H = \dfrac{1}{\sqrt{2}} 
    \begin{pmatrix}
        1 & 1 \\ 1 & -1
    \end{pmatrix}
\end{equation}
This matrix is \textbf{unitary} and \textbf{Hermitian}: $H^{\dagger} = H$ and $H^{2} = I$.

\highspace
\begin{flushleft}
    \textcolor{Green3}{\faIcon{book} \textbf{Action on Basis States}}
\end{flushleft}
\begin{equation*}
    \begin{array}{rclcl}
        H \ket{0} &=& \dfrac{1}{\sqrt{2}}\left(\ket{0} + \ket{1}\right) &=& \ket{+} \hspace{2em} \text{(superposition state)} \\ [1em]
        H \ket{1} &=& \dfrac{1}{\sqrt{2}}\left(\ket{0} - \ket{1}\right) &=& \ket{-}
    \end{array}
\end{equation*}
The $\ket{+}$ and $\ket{-}$ \textbf{states} are known as the \definition{Hadamard basis}, or \textbf{superposition states}, with \textbf{equal probability amplitudes} for $\ket{0}$ and $\ket{1}$.

\highspace
\begin{flushleft}
    \textcolor{Green3}{\faIcon{question-circle} \textbf{Bloch Sphere Interpretation}}
\end{flushleft}
On the Bloch sphere, the \textbf{Hadamard gate performs two sequential rotations}:
\begin{enumerate}
    \item Rotation around the $y$-axis by $\frac{\pi}{2}$ radians ($90\degree$).
    \item Followed by a rotation around the $x$-axis by $\pi$ radians ($180\degree$).
\end{enumerate}
This brings $\ket{0}$ to the $+x$-axis ($\ket{+}$) and $\ket{1}$ to the $-x$-axis ($\ket{-}$). As a result, \textbf{qubit moves from pole to equator}, entering \textbf{maximum superposition}.

\highspace
\begin{flushleft}
    \textcolor{Green3}{\faIcon{book} \textbf{Superposition and Measurement}}
\end{flushleft}
After applying $H$ to $\ket{0}$, the resulting qubit is:
\begin{equation*}
    \ket{\psi} = \dfrac{1}{\sqrt{2}}\left(\ket{0} + \ket{1}\right)
\end{equation*}
Upon measurement:
\begin{itemize}
    \item \textbf{Probability of 0}:
    \begin{equation*}
        \left|\dfrac{1}{\sqrt{2}}\right|^{2} = \dfrac{1}{2}
    \end{equation*}
    \item \textbf{Probability of 1}:
    \begin{equation*}
        \dfrac{1}{2}
    \end{equation*}
\end{itemize}
This \textbf{equal probability} reflects \textbf{maximum uncertainty}, a hallmark of \textbf{quantum superposition}.

\highspace
\begin{flushleft}
    \textcolor{Green3}{\faIcon{list-ul} \textbf{Key Properties of H Gate}}
\end{flushleft}
\begin{enumerate}
    \item \important{Involution}: $H^{2} = I \rightarrow$ applying $H$ \textbf{twice restores the original state}.
    \item \important{Self-adjoint}: $H = H^{\dagger} \rightarrow$ H is its own \textbf{inverse and adjoint}.
    \item \important{Basis Change}: H converts between the \textbf{computational basis} $\left\{\ket{0}, \ket{1}\right\}$ and the \textbf{Hadamard basis} $\left\{\ket{+}, \ket{-}\right\}$.
\end{enumerate}

\highspace
\begin{flushleft}
    \textcolor{Green3}{\faIcon{check-circle} \textbf{Applications}}
\end{flushleft}
\begin{itemize}
    \item \important{Quantum Parallelism}: Prepares qubits for \textbf{simultaneous computation paths}.
    \item \important{Quantum Algorithms}: Found in Grover's Search, Quantum Fourier Transform, Shor's Algorithm.
    \item \important{Entanglement Creation}: Combined with CNOT, H creates Bell states.
    \item \important{Interference}: Enables \textbf{constructive and destructive interference} in algorithms.
\end{itemize}
