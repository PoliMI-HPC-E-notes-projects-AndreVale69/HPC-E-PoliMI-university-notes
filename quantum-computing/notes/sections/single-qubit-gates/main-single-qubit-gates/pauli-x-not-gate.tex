\subsubsection{Pauli-X (NOT) Gate}\label{subsubsection: Pauli-X NOT Gate}

The \definitionWithSpecificIndex{Pauli-X Gate}{Single Qubit Gates: Pauli-X Gate}{}, often called the \textbf{quantum NOT gate}, is one of the \emph{fundamental} single-qubit quantum gates. It is denoted $X$ and \textbf{corresponds to the} $\sigma_{X}$ \textbf{Pauli matrix} in quantum mechanics. This gate \textbf{flips the state} of a qubit, \textbf{similar to how a classical NOT gate inverts a bit}.

\highspace
\begin{remarkbox}[: Pauli Matrix]
   In quantum mechanics, the \definition{Pauli matrices} are \textbf{a set of three $2 \times 2$ Hermitian and unitary matrices} that represent \textbf{spin operators} for spin-$\frac{1}{2}$ particles and are widely used in quantum computing to describe single-qubit gates. They are:
   \begin{equation*}
      \begin{array}{rcl}
         \sigma_{X} &=&
         \begin{pmatrix}
            0 & 1 \\
            1 & 0
         \end{pmatrix} \\ [1em]
         \sigma_{Y} &=&
         \begin{pmatrix}
            0 & -i \\
            i & 0
         \end{pmatrix} \\ [1em]
         \sigma_{Z} &=&
         \begin{pmatrix}
            1 & 0 \\
            0 & -1
         \end{pmatrix}
      \end{array}
   \end{equation*}
   Key properties:
   \begin{itemize}
      \item Hermitian: $\sigma_{i}^{\dagger} = \sigma_{i}$ (observable-related)
      \item Unitary: $\sigma_{i}^{\dagger} \sigma_{i} = I$ (reversible transformations)
      \item Involution: $\sigma_{i}^{2} = I$
      \item Non-commuting: $\left[\sigma_{i}, \sigma_{j}\right] = 2i \varepsilon_{ijk} \sigma_{k}$ (with $\varepsilon$ the Levi-Civita symbol)
   \end{itemize}
\end{remarkbox}

\highspace
\begin{flushleft}
   \textcolor{Green3}{\faIcon{square-root-alt} \textbf{Matrix Representation}}
\end{flushleft}
\begin{equation}
   X = \sigma_{X} =
   \begin{pmatrix}
      0 & 1 \\ 1 & 0
   \end{pmatrix}
\end{equation}
This \textbf{matrix swaps} the $\ket{0}$ and $\ket{1}$ \textbf{basis states}:
\begin{equation}
   X\ket{0} = \ket{1} \hspace{2em} \text{and} \hspace{2em} X\ket{1} = \ket{0}
\end{equation}

\highspace
\begin{flushleft}
   \textcolor{Green3}{\faIcon{book} \textbf{Action on a General Qubit}}
\end{flushleft}
Let's consider a qubit in the general state:
\begin{equation*}
   \ket{\psi} = a\ket{0} + b\ket{1}
\end{equation*}
Applying the $X$ gate:
\begin{equation*}
   X\ket{\psi} = a X \ket{0} + b X \ket{1} = a \ket{1} + b \ket{0} = b \ket{0} + a \ket{1}
\end{equation*}
As a result, the amplitudes $a$ and $b$ are \textbf{swapped}.

\highspace
\begin{flushleft}
   \textcolor{Green3}{\faIcon{question-circle} \textbf{Geometric Interpretation: Rotation on the Bloch Sphere}}
\end{flushleft}
The Pauli-X gate corresponds to a \textbf{rotation around the $x$-axis by $\pi$ radians ($180\degree$)} on the Bloch sphere.

\highspace
\textbf{To understand} this, recall the \textbf{general qubit state in spherical coordinates}:
\begin{equation*}
   \ket{\psi} = \cos\left(\dfrac{\theta}{2}\right) \ket{0} + e^{i \phi} \sin\left(\dfrac{\theta}{2}\right) \ket{1}
\end{equation*}
After a $\pi$ rotation around the $x$-axis:
\begin{itemize}
   \item $\theta \rightarrow \pi - \theta$
   \item $\phi \rightarrow -\phi$ 
\end{itemize}
This \textbf{rotation mirrors the qubit across the $x$-axis}, flipping its position between $\ket{0}$ and $\ket{1}$, and adjusting the phase accordingly.

\highspace
\begin{flushleft}
   \textcolor{Green3}{\faIcon{list-ul} \textbf{Key Properties of X Gate}}
\end{flushleft}
\begin{enumerate}
   \item \important{Involution}. Applying X twice returns the original state:
   \begin{equation*}
      X X = X^{2} = I
   \end{equation*}
   This \textbf{reflects reversibility} and \textbf{matches the intuition of two $180\degree$ rotations} around $x$-axis equaling no rotation.

   \item \important{Entanglement Tool}. The X gate is crucial in \textbf{entangling qubits} when combined with CNOT in multi-qubit circuits.
   \item \important{Measurement Preparation}. Used to \textbf{flip the measurement outcome}, e.g., preparing $\ket{1}$ from $\ket{0}$.
\end{enumerate}