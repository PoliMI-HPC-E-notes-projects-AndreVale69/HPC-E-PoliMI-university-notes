\subsubsection{Pauli-Y Gate}

The \definitionWithSpecificIndex{Pauli-Y Gate}{Single Qubit Gates: Pauli-Y Gate}{}, denoted $Y$, is one of the three fundamental \textbf{Pauli matrices} in quantum mechanics. It is less intuitive than Pauli-X and Z, but equally important, especially for its role in complex rotations and quantum interference.

\highspace
\begin{flushleft}
    \textcolor{Green3}{\faIcon{square-root-alt} \textbf{Matrix Representation}}
\end{flushleft}
\begin{equation*}
    Y = \sigma_{Y} = \begin{pmatrix}
        0 & -i \\ i & 0
    \end{pmatrix}
\end{equation*}
Like the other Pauli gates, $Y$ is \textbf{Hermitian} and \textbf{unitary}, which guarantees that it is \textbf{observable} (measurable) and \textbf{reversible}.

\highspace
\begin{flushleft}
    \textcolor{Green3}{\faIcon{book}}
\end{flushleft}
Given a qubit in state:
\begin{equation*}
	\quantumState{\psi} = a\quantumState{0} + b\quantumState{1}
\end{equation*}
Apply the Y gate:
\begin{equation*}
	Y\quantumState{\psi} = aY\quantumState{0} + bY\quantumState{1} = a\left(-i \quantumState{1}\right) + b\left(i \quantumState{0}\right) = ib\quantumState{0} - ia\quantumState{1}
\end{equation*}
So the \textbf{amplitudes are swapped and multiplied by $\pm i$}. This shows Pauli-Y \textbf{introduces a complex phase shift} alongside the swap.

\highspace
\begin{flushleft}
    \textcolor{Green3}{\faIcon{question-circle} \textbf{Geometric Interpretation: Rotation on the Bloch Sphere}}
\end{flushleft}
The Pauli-Y gate corresponds to a \textbf{rotation around the $y$-axis by $\pi$ radians ($180 \degree$)}. The effect on Bloch sphere coordinates:
\begin{itemize}
    \item $\theta \rightarrow \pi - \theta$
    \item $\psi \rightarrow \pi - \psi$
\end{itemize}
This rotates the qubit across the $y$-axis, \textbf{altering both amplitudes and their phases}.

\highspace
\begin{flushleft}
    \textcolor{Green3}{\faIcon{book} \textbf{Pauli-Y as a Composite Gate}}
\end{flushleft}
Interestingly, $Y = i X Z$. Let's verify this:
\begin{equation*}
    XZ = \begin{pmatrix}
        0 & 1 \\ 1 & 0
    \end{pmatrix}
    \begin{pmatrix}
        1 & 0 \\ 0 & -1
    \end{pmatrix}
    =
    \begin{pmatrix}
        0 & -1 \\ 1 & 0
    \end{pmatrix}
\end{equation*}
Now, multiply by $i$:
\begin{equation*}
    iXZ = i \begin{pmatrix}
        0 & -1 \\ 1 & 0
    \end{pmatrix}
    =
    \begin{pmatrix}
        0 & -i \\ i & 0
    \end{pmatrix}
    = Y
\end{equation*}
This identity confirms that \textbf{Pauli-Y can be viewed as a combination of Pauli-X and Pauli-Z} (page \pageref{subsubsection: Pauli-Z (Phase Flip) Gate}) \textbf{with a global phase factor $i$}, which does not affect measurement outcomes.

\highspace
\begin{flushleft}
    \textcolor{Green3}{\faIcon{list-ul} \textbf{Key Properties of Y Gate}}
\end{flushleft}
\begin{enumerate}
    \item Involution: $Y^{2} = I$
    \item Anticommutation: $XY = -YX$
    \item Phase Sensitive: introduces imaginary coefficients, unlike $X$ and $Z$.
\end{enumerate}
