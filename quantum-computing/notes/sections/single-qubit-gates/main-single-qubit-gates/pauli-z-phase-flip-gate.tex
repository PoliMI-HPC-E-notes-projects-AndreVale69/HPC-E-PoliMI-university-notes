\subsubsection{Pauli-Z (Phase Flip) Gate}\label{subsubsection: Pauli-Z Phase Flip Gate}

The \definitionWithSpecificIndex{Pauli-Z Gate}{Single Qubit Gates: Pauli-Z Gate}{}, also called the \textbf{Phase Flip Gate}, is one of the core \textbf{Pauli operators} used in quantum computing. It acts \textbf{only on the phase} of the qubit, \textbf{leaving the $\ket{0}$ amplitude unchanged} but \textbf{flipping the sign of the $\ket{1}$ amplitude}.

\highspace
\begin{flushleft}
    \textcolor{Green3}{\faIcon{square-root-alt} \textbf{Matrix Representation}}
\end{flushleft}
\begin{equation}
    Z = \sigma_{Z} = 
    \begin{pmatrix}
        1 & 0 \\
        0 & -1
    \end{pmatrix}
\end{equation}

\highspace
\begin{flushleft}
    \textcolor{Green3}{\faIcon{book} \textbf{Action on a General Qubit}}
\end{flushleft}
For a general state:
\begin{equation*}
    \ket{\psi} = a\ket{0} + b\ket{1}
\end{equation*}
Applying $Z$:
\begin{equation*}
    Z\ket{\psi} = aZ\ket{0} + bZ\ket{1} = a\ket{0} - b\ket{1}
\end{equation*}
As a result, $Z$ \textbf{flips the sign of the $\ket{1}$ amplitude}, this is why it's called a \textbf{phase flip}. Finally, it's trivial to show that:
\begin{equation*}
    Z\ket{0} = \ket{0} \hspace{1em} \text{(unchanged)}, \hspace{2em} Z\ket{1} = -\ket{1} \hspace{1em} \text{(sign flip)}
\end{equation*}

\highspace
\begin{flushleft}
    \textcolor{Green3}{\faIcon{question-circle} \textbf{Geometric Interpretation: Rotation around the $z$-axis}}
\end{flushleft}
On the Bloch sphere, the Pauli-Z gate performs a \textbf{$\pi$ radians ($180\degree$) rotation around the $z$-axis}. In spherical coordinates:
\begin{equation*}
    \ket{\psi} = \cos\left(\dfrac{\theta}{2}\right)\ket{0} + e^{i \phi} \sin\left(\dfrac{\theta}{2}\right)\ket{1}
\end{equation*}
After Z rotation:
\begin{itemize}
    \item The angle $\phi \rightarrow \phi + \pi$
    \item This phase shift \textbf{changes the sign of the complex amplitude for} $\ket{1}$
\end{itemize}
Since \textbf{measurement probabilities depend on} $\left|a\right|^{2}$ and $\left|b\right|^{2}$, which \hl{remain unchanged}, the Z gate \textbf{does not affect measurement outcomes} in the standard basis. It only affects \textbf{interference and future gate interactions}.

\highspace
\begin{flushleft}
    \textcolor{Green3}{\faIcon{book} \textbf{Interpretation and Use Cases}}
\end{flushleft}
\begin{itemize}
    \item \important{Phase Shift}: $Z$ is a special case of a phase shift gate, shifting the phase by $\pi$.
    \item \important{No effect on $\ket{0}$}: Useful for controlled operations, where only $\ket{1}$ paths acquire a phase.
    \item \important{Circuit Simplification}: Often used in \textbf{error correction} and \textbf{phase-sensitive algorithms}.
\end{itemize}

\highspace
\begin{flushleft}
    \textcolor{Green3}{\faIcon{list-ul} \textbf{Key Properties}}
\end{flushleft}
\begin{enumerate}
    \item \textbf{Involution}: $Z^{2} = I$
    \item \textbf{Diagonal Matrix}: Easy to compute and simulate.
    \item \textbf{Commutes with Z and Phase gates}, but \textbf{anticommutes with X and Y}.
\end{enumerate}