\subsubsection{Phase Gate (S)}

The \definitionWithSpecificIndex{Phase Gate (S)}{Single Qubit Gates: Phase Gate (S)}{} is a single-qubit gate that \textbf{adds a phase shift of $\frac{\pi}{2}$ ($90\degree$)} \textbf{to the amplitude of the $\quantumState{1}$ state}. It \textbf{leaves $\quantumState{0}$ unchanged}, like the Pauli-Z gate, but instead of a sign flip ($\pi$ phase), it \textbf{introduces a complex phase factor $i$}.

\highspace
\begin{flushleft}
    \textcolor{Green3}{\faIcon{square-root-alt} \textbf{Matrix Representation}}
\end{flushleft}
\begin{equation}
    S = \begin{pmatrix}
        1 & 0 \\ 0 & i
    \end{pmatrix}
\end{equation}

\highspace
\begin{flushleft}
    \textcolor{Green3}{\faIcon{book} \textbf{Action on a General Qubit}}
\end{flushleft}
For a state:
\begin{equation*}
    \quantumState{\psi} = a\quantumState{0} + b\quantumState{1}
\end{equation*}
Apply $S$:
\begin{equation*}
    S\quantumState{\psi} = a\quantumState{0} + ib\quantumState{1}
\end{equation*}
Only the $\quantumState{1}$ \textbf{amplitude gains a phase of} $i$, affecting interference but \textbf{not measurement probabilities} in the computational basis.
\begin{equation*}
    S\quantumState{0} = \quantumState{0} \hspace{1em} \text{(unchanged)}, \hspace{2em} S\quantumState{1} = i\quantumState{1} \hspace{1em} \left(\text{phase } + \frac{\pi}{2}\right)
\end{equation*}

\highspace
\begin{flushleft}
    \textcolor{Green3}{\faIcon{question-circle} \textbf{Geometric Interpretation: Rotation around $z$-axis $\left(\frac{\pi}{2}\right)$}}
\end{flushleft}
On the Bloch sphere, $S$ performs a rotation of $\frac{\pi}{2}$ radians around the $z$-axis. This rotation \textbf{shifts the phase angle} $\phi \rightarrow \phi + \frac{\pi}{2}$, altering \textbf{how the qubit interferes} in later operations, especially when \textbf{Hadamard gates} are involved.

\highspace
\begin{flushleft}
    \textcolor{Green3}{\faIcon{exchange-alt} \textbf{Relationship to Pauli-Z}}
\end{flushleft}
Two applications of $S$ equal $Z$:
\begin{equation*}
    S \cdot S = 
    \begin{pmatrix}
        1 & 0 \\ 0 & i
    \end{pmatrix}
    \cdot
    \begin{pmatrix}
        1 & 0 \\ 0 & i
    \end{pmatrix}
    =
    \begin{pmatrix}
        1 & 0 \\ 0 & i^{2}
    \end{pmatrix}
    =
    \begin{pmatrix}
        1 & 0 \\ 0 & -1
    \end{pmatrix}
    = Z
\end{equation*}
Thus, $S^{2} = Z$, this is a \textbf{useful identity} in circuit simplifications.

\highspace
\begin{flushleft}
    \textcolor{Green3}{\faIcon{check-circle} \textbf{Applications and Significance}}
\end{flushleft}
\begin{itemize}
    \item \important{Superposition Control}: $S$ \textbf{adjusts relative phases}, crucial in algorithms like Quantum Fourier Transform and Grover's Search.
    \item \important{Building Block}: $S$ helps construct \textbf{higher-order phase gates}, such as the T gate, which introduces a $\frac{\pi}{4}$ phase.
    \item \important{Error Correction}: Used in \textbf{stabilizer codes} for \textbf{fault-tolerant quantum computing}.
\end{itemize}

\highspace
\begin{flushleft}
    \textcolor{Green3}{\faIcon{list-ul} \textbf{Key Properties}}
\end{flushleft}
\begin{enumerate}
    \item \important{Unitary}: $S^{\dagger} S = I$
    \item \important{Not Hermitian}: Unlike Pauli gates, $S \ne S^{\dagger}$
    \item \important{Diagonal Matrix}: Efficient in simulation.
\end{enumerate}