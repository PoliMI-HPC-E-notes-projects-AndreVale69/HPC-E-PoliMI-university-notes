\subsection{Measurement}

In quantum computing, \textbf{measurement} is a special type of operation that differs fundamentally from unitary gates. While gates perform reversible transformations, \textbf{measurement is irreversible}. It allows us to \textbf{gain information about the state of a qubit}, but at the price of \textbf{collapsing the quantum state}.

\highspace
\begin{flushleft}
  \textcolor{Green3}{\faIcon{square-root-alt} \textbf{Measurement Operator: Matrix Form}}
\end{flushleft}
A \textbf{measurement operator} is a \textbf{non-unitary}, \textbf{non-invertible} matrix. For a measurement \textbf{along a specific direction}, represented by a ket \ket{k}, the associated \textbf{projector} is:
\begin{equation}
  M_{k} = \proj{k}
\end{equation}
This is called a \definition{Projection Operator}, it projects any vector \textbf{onto the direction of} \ket{k}.

\highspace
\begin{flushleft}
  \textcolor{Red2}{\faIcon{exclamation-triangle} \textbf{Effect of Measurement}}
\end{flushleft}
Given a qubit \ket{\psi}, applying $M_{k}$ yields the \textbf{(unnormalized) projected vector}:
\begin{equation*}
  \ket{\psi_{k}} = M_{k} \ket{\psi} = \ket{k} \braket{k}{\psi} = \braket{k}{\psi} \ket{k}
\end{equation*}
To obtain the \textbf{new normalized state}, we \textbf{divide by the square root of the probability}:
\begin{equation*}
  \ket{\psi_{k}^{\text{norm}}} =
  \dfrac{M_{k} \cdot \ket{\psi}}{\sqrt{\left\langle \psi \left| M_{k} \right| \psi \right\rangle}} =
  \dfrac{\braket{k}{\psi} \cdot \ket{k}}{\sqrt{\left|\braket{k}{\psi}\right|^{2}}} =
  \ket{k}
\end{equation*}
So after measurement, the qubit \textbf{collapses} to \ket{k}, the \textbf{eigenstate corresponding to the measurement outcome}.

\begin{remarkbox}[: $\left\langle \psi \left| M_{k} \right| \psi \right\rangle$]
  This is called a quadratic form in linear algebra, and in quantum mechanics it represents the \textbf{probability of the measurement outcome} $k$ when \textbf{measuring the qubit} \ket{\psi} with \textbf{measurement operator} $M_{k}$.

  Suppose a qubit $\psi$:
  \begin{equation*}
    \ket{\psi} = \begin{pmatrix}
      a \\
      b
    \end{pmatrix}
    \hspace{2em}
    \text{with } \left|a\right|^{2} + \left|b\right|^{2} = 1
  \end{equation*}
  If we apply the measurement operator $M_{0}$:
  \begin{equation*}
    M_{0} \ket{\psi} =
    \begin{pmatrix}
      1 & 0 \\ 0 & 0
    \end{pmatrix}
    \begin{pmatrix}
      a \\ b
    \end{pmatrix}
    =
    \begin{pmatrix}
      a \\ 0
    \end{pmatrix}
  \end{equation*}
  And we finally calculate the bra \bra{\psi}:
  \begin{equation*}
    \bra{\psi} \begin{pmatrix}
      a \\ 0
    \end{pmatrix}
    =
    \begin{pmatrix}
      \overline{a} & \overline{b}
    \end{pmatrix}
    \begin{pmatrix}
      a \\ 0
    \end{pmatrix}
    = \overline{a} \cdot a = \left|a\right|^{2}
  \end{equation*}
  In other words, this is the probability of measuring 0: $p_{0}$.
\end{remarkbox}

\highspace
\begin{flushleft}
  \textcolor{Green3}{\faIcon{percentage} \textbf{Probability of Outcome}}
\end{flushleft}
The \textbf{probability of measuring the qubit in the state} \ket{k} is:
\begin{equation}\label{eq: Born rule}
  p_{k} = \left\langle \psi \left| M_{k} \right| \psi \right\rangle = \left|\braket{k}{\psi}\right|^{2} =
  \left| \overline{k}_{1} \cdot \psi_{1} + \overline{k_{2}} \cdot \psi_{2} \right|^{2}
\end{equation}
This is the \definition{Born rule}, a fundamental postulate of quantum mechanics. In quantum computing we have:
\begin{equation*}
  \begin{array}{rcl}
    p_{0} &=& \left\langle \psi \left|M_{0}\right| \psi \right\rangle = \left|\braket{0}{\psi}\right|^{2} = \left|1 \cdot \psi_{1} + \cancel{0 \cdot \psi_{2}} \right| \\ [.8em]
    %
    p_{1} &=& \left\langle \psi \left|M_{1}\right| \psi \right\rangle = \left|\braket{1}{\psi}\right|^{2} = \left|\cancel{0 \cdot \psi_{1}} + 1 \cdot \psi_{2} \right|
  \end{array}
\end{equation*}

\highspace
\begin{flushleft}
  \textcolor{Green3}{\faIcon{book} \textbf{Standard Basis Measurement}}
\end{flushleft}
In quantum computing, we usually measure in the \textbf{computational basis}\break $\left\{\ket{0}, \ket{1}\right\}$. The \textbf{projectors} are:
\begin{equation}
  \begin{array}{rclcl}
    M_{0} &=& \ket{0}\bra{0} &=& \begin{pmatrix}
      1 & 0 \\
      0 & 0
    \end{pmatrix} \\ [1em]
    M_{1} &=& \ket{1}\bra{1} &=& \begin{pmatrix}
      0 & 0 \\
      0 & 1
    \end{pmatrix}
  \end{array}
\end{equation}
For a \textbf{general qubit} $\ket{\psi} = a\ket{0} + b\ket{1}$, the \textbf{probabilities} are:
\begin{equation}
  p_{0} = \left|a\right|^{2} \hspace{2em} p_{1} = \left|b\right|^{2}
\end{equation}
After measurement, the qubit \textbf{collapses to \ket{0} with probability $\left|a\right|^{2}$}, or to \textbf{\ket{1} with probability $\left|b\right|^{2}$}.

\highspace
\begin{flushleft}
  \textcolor{Green3}{\faIcon{star} \textbf{Key Properties of Measurement Operator}}
\end{flushleft}
\begin{itemize}
  \item \important{Idempotent}: once applied, \textbf{applying it again does nothing}:
  \begin{equation*}
    M_{k}^{2} = M_{k}
  \end{equation*}
  A simple proof:
  \begin{equation*}
    M_{k}^{2} = \ket{k} \proj{k} \bra{k} = \ket{k}\bra{k} = M_{k}
  \end{equation*}
  This reflects that \textbf{after projection, the qubit is already in the state} \ket{k}.

  \item \important{Non-unitary} and \important{Non-reversible}: measurement \textbf{destroys superposition} and \textbf{cannot be undone}.
\end{itemize}