\section{Single Qubit Gates}

\subsection{Operations on Qubits}

In quantum computing, \textbf{operations on qubits} are fundamental to information processing and computation. These operations can be broadly \textbf{classified} into:
\begin{itemize}
    \item \textbf{Logic Gates}
    \item \textbf{Measurements}
    \item \textbf{Initialization Procedures}
\end{itemize}
Each of these plays a distinct role in the manipulation and evolution of quantum states, which are represented as \textbf{superpositions} of basis states $\ket{0}$ and $\ket{1}$.

\highspace
\begin{flushleft}
    \textcolor{Green3}{\faIcon{book} \textbf{Logic Gates: Reversible Transformations}}
\end{flushleft}
Quantum \definitionWithSpecificIndex{Logic Gates}{Single Qubit Gates: Logic Gates}{} serve as the \textbf{building blocks of quantum circuits}. Unlike classical logic gates, quantum gates operate under the principles of quantum mechanics, notably \textbf{unitarity} and \textbf{reversibility}. These gates can act on a single qubit or on multiple qubits simultaneously, and their \textbf{primary function is to transform the state of the qubit(s)} involved.

\highspace
\textbf{Key properties} of quantum logic gates include:
\begin{itemize}
    \item They are \textbf{reversible}: this implies that \hl{any operation performed by a gate can be undone by applying its inverse}. In quantum mechanics, reversibility is linked to the \textbf{requirement that gates be unitary matrices}.

    \item They \textbf{preserve the norm} of the quantum state, a \hl{direct consequence of unitarity}.
    \item Gates \textbf{change the qubit's state} in a way that aligns with the \hl{no-cloning} and \hl{no-deleting theorems}; in other words, \hl{information is neither duplicated nor destroyed, only transformed}.
\end{itemize}

\highspace
\begin{flushleft}
    \textcolor{Green3}{\faIcon{book} \textbf{Measurement: Irreversible Collapse of State}}
\end{flushleft}
\definitionWithSpecificIndex{Measurement}{Single Qubit Gates: Measurement}{} is a fundamentally different operation from logic gates because it is \textbf{irreversible}. When a qubit is measured, \textbf{information about its state is extracted}, and as a consequence, the \textbf{qubit's quantum state collapses} to one of the basis states $\ket{0}$ or $\ket{1}$. This collapse causes the \textbf{loss of superposition} and, if entanglement is involved, the \textbf{loss of entanglement} as well (explained later).

\highspace
Important points:
\begin{itemize}
    \item Measurement provides \textbf{probabilistic outcomes based on the amplitudes} of the quantum state.
    \item After measurement, the qubit is \textbf{no longer in a superposition}; it is deterministically in $\ket{0}$ or $\ket{1}$.
    \item This operation is \textbf{non-unitary} and hence \textbf{non-reversible}.
\end{itemize}

\highspace
\begin{flushleft}
    \textcolor{Green3}{\faIcon{book} \textbf{Initialization: Preparing Known States}}
\end{flushleft}
The \definitionWithSpecificIndex{Initialization}{Single Qubit Gates: Initialization}{} of qubits is the process of \textbf{setting them into a known, well-defined state}, most commonly $\ket{0}$ or $\ket{1}$. \hl{Initialization is essential because quantum algorithms require precisely defined input states}.

\highspace
Key facts:
\begin{itemize}
    \item Initialization can \emph{often} be \textbf{implemented via measurement}: by measuring a qubit and then applying a gate if necessary, we can prepare it in the desired state.
    \item As with measurement, initialization \textbf{collapses the quantum state}, making it \textbf{irreversible}.
    \item While measurement discards the quantum coherence, \textbf{initialization ensures a clean starting point} for computations.
\end{itemize}

\highspace
\begin{flushleft}
    \textcolor{Green3}{\faIcon{balance-scale} \textbf{Comparison between Initialization and Measurement}}
\end{flushleft}
\begin{itemize}
    \item \textcolor{Green3}{\textbf{Purpose}}
    \begin{itemize}
        \item \textbf{Measurement}: To extract information about the qubit's state.
        \item \textbf{Initialization}: To prepare the qubit in a known state ($\ket{0}$ or $\ket{1}$).
    \end{itemize}
    \item \textcolor{Green3}{\textbf{Reversibility}}
    \begin{itemize}
        \item \textbf{Measurement}: Irreversible, collapses superposition.
        \item \textbf{Initialization}: Irreversible, collapses superposition to set a state.
    \end{itemize}
    \item \textcolor{Green3}{\textbf{Effect on Qubit}}
    \begin{itemize}
        \item \textbf{Measurement}: Collapses to $\ket{0}$ or $\ket{1}$ randomly (probabilistic).
        \item \textbf{Initialization}: Collapses to a specific state, often $\ket{0}$ (deterministic or controlled).
    \end{itemize}
    \item \textcolor{Green3}{\textbf{Information Gained}}
    \begin{itemize}
        \item \textbf{Measurement}: Yes, outcome of measurement is known.
        \item \textbf{Initialization}: No, goal is not to gain information, just set state.
    \end{itemize}
    \item \textcolor{Green3}{\textbf{How Implemented}}
    \begin{itemize}
        \item \textbf{Measurement}: Via a measurement operator.
        \item \textbf{Initialization}: Often via measurement \texttt{+} gate correction.
    \end{itemize}

    \newpage

    \item \textcolor{Green3}{\textbf{Post-operation Use}}
    \begin{itemize}
        \item \textbf{Measurement}: Qubit may be discarded or reused depending on outcome.
        \item \textbf{Initialization}: Qubit is now ready for computation.
    \end{itemize}
\end{itemize}
In other words:
\begin{itemize}
    \item \textbf{Measurement} tells we \textbf{what state the qubit is} in.
    \item \textbf{Initialization sets the qubit} to a desired state so we can \textbf{start a computation}.
\end{itemize}