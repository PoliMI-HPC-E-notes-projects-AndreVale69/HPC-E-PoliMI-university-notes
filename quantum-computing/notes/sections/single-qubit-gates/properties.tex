\subsection{Properties}

Single-qubit gates exhibit a rich set of mathematical and geometric properties, all of which stem from their nature as unitary transformations on a two-level quantum system. Understanding these properties is key to \textbf{predicting gate behavior}, \textbf{designing quantum circuits}, and \textbf{analyzing quantum algorithms}.

\highspace
\begin{flushleft}
    \textcolor{Green3}{\textbf{1. All Quantum Gates Are Equivalent to Rotations}}
\end{flushleft}
A profound insight in quantum computing is that \textbf{every single-qubit unitary gate corresponds to a rotation of the qubit's state vector on the Bloch sphere}.
\begin{itemize}
    \item \important{Bloch Sphere} is a geometric representation of a qubit's state, where any point on the sphere represents a valid pure state.
    \item A \important{unitary operation} on qubit results in \textbf{rotating the vector} on this sphere \textbf{without changing its length} (preserving probability).
\end{itemize}
Some examples of rotations:
\begin{itemize}
    \item \texttt{X Gate} $\rightarrow$ Rotation $\pi$ around x-axis (section~\ref{subsubsection: Pauli-X NOT Gate}, page~\pageref{subsubsection: Pauli-X NOT Gate})
    \item \texttt{Z Gate} $\rightarrow$ Rotation $\pi$ around z-axis (section~\ref{subsubsection: Pauli-Z Phase Flip Gate}, page~\pageref{subsubsection: Pauli-Z Phase Flip Gate})
    \item \texttt{Y Gate} $\rightarrow$ Rotation $\pi$ around y-axis (section~\ref{subsubsection: Pauli-Y Gate}, page~\pageref{subsubsection: Pauli-Y Gate})
    \item \texttt{S Gate} $\rightarrow$ Rotation $\frac{\pi}{2}$ around z-axis (section~\ref{subsubsection: Phase Gate S}, page~\pageref{subsubsection: Phase Gate S})
    \item \texttt{H Gate} $\rightarrow$ Rotation $\frac{\pi}{2}$ around y, then $\pi$ around x (section~\ref{subsubsection: Hadamard Gate H}, page~\pageref{subsubsection: Hadamard Gate H})
\end{itemize}
\textcolor{Green3}{\faIcon{question-circle} \textbf{Why is this important?}} Quantum computation is fundamentally about \textbf{state rotations}, understanding gates as rotations helps visualize \textbf{interference}, \textbf{phase shifts}, and \textbf{entanglement creation}.

\highspace
\begin{flushleft}
    \textcolor{Green3}{\textbf{2. Reversibility: Applying a Gate Twice}}
\end{flushleft}
Many single-qubit gates, especially the Pauli gates, exhibit the property of \textbf{involution}: applying the same gate twice \textbf{returns the qubit to its original state}. Mathematically it can be expressed as:
\begin{equation*}
    X^{2} = Y^{2} = Z^{2} = I
\end{equation*}
This is due to the fact that \textbf{two $180\degree$ rotations around the same axis bring the vector back} to its original position on the Bloch sphere.

\begin{proof}[Proof that X Gate has involution]
    \begin{equation*}
        X^{2} =
        \begin{pmatrix}
            0 & 1 \\ 1 & 0
        \end{pmatrix}
        \begin{pmatrix}
            0 & 1 \\ 1 & 0
        \end{pmatrix}
        =
        \begin{pmatrix}
            1 & 0 \\ 0 & 1
        \end{pmatrix}
        =
        I
    \end{equation*}
    These identities confirm that \textbf{Pauli gates are their own inverses}, and \textbf{all unitary gates are reversible}.
\end{proof}

\highspace
\begin{flushleft}
    \textcolor{Green3}{\textbf{3. Global Phases Do Not Matter}}
\end{flushleft}
Quantum gates may \textbf{introduce a global phase} (a constant complex factor $e^{i\phi}$) to a state. Physically, \textbf{global phases are unobservable}, meaning they do \textbf{not affect measurement probabilities}. For example:
\begin{itemize}
    \item States $\quantumState{\psi}$ and $e^{i\phi}\quantumState{\psi}$ are physically indistinguishable.
    \item The Y gate $= iXZ$, where $i$ is a global phase that can be ignored in practice. 
\end{itemize}

\highspace
\begin{flushleft}
    \textcolor{Green3}{\textbf{4. Composite Gates and Commutativity}}
\end{flushleft}
\begin{itemize}
    \item Gates can be \textbf{combined} by \textbf{matrix multiplication} (note: non commutative in general).
    \item \textbf{Order matters}: $AB \ne BA$ for most gates.
    \item Exception: \textbf{Gate Z and Gate S} are \textbf{commuted} because they are both \textbf{diagonal phase gates}.
\end{itemize}

\highspace
\begin{table}[!htp]
    \begin{adjustbox}{width={\textwidth},totalheight={\textheight},keepaspectratio}%
        \centering
        \begin{tabular}{@{} l | c | c | c @{}}
            \toprule
            \textbf{Property}                       & Pauli Gates (X, Y, Z)                             & Phase Gate (S)                                & Hadamard (H) \\
            \midrule
            \textbf{Rotation Axis}                  & x, y, z ($\pi$ radians)                           & z $\left(\frac{\pi}{2} \text{radians}\right)$ & y $\left(\frac{\pi}{2}\right) + $ x $\left(\pi\right)$ \\ [.3em]
            \textbf{Involution} $G^{2} = I$         & \textcolor{Green3}{\faIcon{check}}                & \textcolor{Red2}{\faIcon{times}} (S² = Z)     & \textcolor{Green3}{\faIcon{check}}                     \\ [.3em]
            \textbf{Self-adjoint} $G = G^{\dagger}$ & \textcolor{Green3}{\faIcon{check}}                & \textcolor{Red2}{\faIcon{times}}              & \textcolor{Green3}{\faIcon{check}}                     \\ [.3em]
            \textbf{Unitary}                        & \textcolor{Green3}{\faIcon{check}}                & \textcolor{Green3}{\faIcon{check}}            & \textcolor{Green3}{\faIcon{check}}                     \\ [.3em]
            \textbf{Phase Introduction}             & Z, S                                              & $i$ for $\quantumState{1}$                    & \textcolor{Green3}{\faIcon{check}} (interference)      \\ [.3em]
            \textbf{Creates Superposition?}         & \textcolor{Red2}{\faIcon{times}} (X, Y, Z alone)  & \textcolor{Red2}{\faIcon{times}}              & \textcolor{Green3}{\faIcon{check}}                     \\
            \bottomrule
        \end{tabular}
    \end{adjustbox}
    \caption{Summary of Properties.}
\end{table}