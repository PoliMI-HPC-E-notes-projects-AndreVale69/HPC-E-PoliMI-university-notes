\subsection{Quantum Logic Gates Overview}

In quantum computing, quantum logic gates are the primary tools used to \textbf{manipulate qubits}. These gates perform \textbf{unitary transformations} on the state of one or more qubits, meaning they \textbf{preserve the norm} of the quantum state and are \textbf{reversible}. The \hl{design} of \hl{quantum algorithms} and the \hl{execution of quantum circuits rely entirely on the sequential application of these gates}.

\highspace
\begin{flushleft}
    \textcolor{Green3}{\faIcon{list-ul} \textbf{Types of Quantum Gates}}
\end{flushleft}
\definition{Quantum Gates} can be classified into:
\begin{itemize}
    \item \definition{Single-Qubit Gates}: These gates operate on \textbf{individual qubits}, modifying their state in isolation.
    \item \textbf{Multiple-Qubit Gates}: These gates act on \textbf{two or more qubits simultaneously}, allowing for entanglement and complex correlations between qubits.
\end{itemize}
In this section we will focus on single qubit gates.

\highspace
\begin{flushleft}
    \textcolor{Green3}{\faIcon{square-root-alt} \textbf{Mathematical Framework: Qubit as Vectors, Gates as Matrices}}
\end{flushleft}
A key distinction of quantum logic compared to classical logic is that \textbf{qubits are vectors}, while \textbf{quantum gates are matrices}.
\begin{itemize}
    \item A \textbf{qubit's state}, as we discussed on page \pageref{subsection: Single Qubits}, is a \textbf{vector in a 2-dimensional complex vector space}, often written as:
    \begin{equation*}
        \quantumState{\psi} = a\quantumState{0} + b\quantumState{1} \hspace{2em} \text{with } \left|a\right|^{2} + \left|b\right|^{2} = 1
    \end{equation*}

    \item A \textbf{Quantum Gate} that acts on a single qubit is a $2 \times 2$ \textbf{complex matrix}, specifically a \textbf{unitary matrix}, denoted by $U$.

    The \textbf{action of a gate} $U$ on a qubit $\quantumState{\psi}$ is simply \textbf{matrix-vector multiplication}:
    \begin{equation}
        \quantumState{\psi'} = U\quantumState{\psi}
    \end{equation}
    This transforms the qubit into a \textbf{new quantum state}, while \hl{ensuring} that \textbf{probabilities remain normalized}.
\end{itemize}

\highspace
\begin{flushleft}
    \textcolor{Green3}{\faIcon{bookmark} \textbf{Unitarity: The Fundamental Constraint}}
\end{flushleft}
A quantum gate \textbf{\underline{must be unitary}}, which means:
\begin{equation}
    U^{\dagger} U = U U^{\dagger} = I
\end{equation}
Where:
\begin{itemize}
    \item $U^{\dagger}$ is the \textbf{Hermitian transpose} (conjugate transpose) of $U$.
    \item $I$ is the \textbf{identity matrix}.
\end{itemize}
\textcolor{Green3}{\faIcon{question-circle} \textbf{Why is it important?}} Because \textbf{unitarity ensures reversibility}: every gate has an inverse, and information is never lost, only transformed. This is in contrast to measurement, which is non-unitary and irreversible.

\highspace
\begin{flushleft}
    \textcolor{Red2}{\faIcon{question-circle} \textbf{Why must gates be unitarity?}}
\end{flushleft}
\begin{enumerate}
    \item \definition{No-Cloning Theorem}: It is impossible to create an identical copy of an unknown qubit. Unitary transformations respect this by not duplicating information.
    \item \definition{No-Deleting Theorem}: We cannot delete quantum information arbitrarily. Since unitarity gates are invertible, they do not erase information.
    \item \textbf{Preservation of Probability}: The normalization condition $\left|a\right|^{2} + \left|b\right|^{2} = 1$ must hold after any operation, and only unitarity matrices preserve this norm.
\end{enumerate}

\highspace
\begin{flushleft}
    \textcolor{Green3}{\faIcon{book} \textbf{Physical Intuition}}
\end{flushleft}
In classical circuits, logic gates like AND, OR, NOT process bits deterministically. In quantum circuits:
\begin{itemize}
    \item Gates \textbf{rotate} the quantum state on the \textbf{bloch sphere}.
    \item These rotations correspond to \textbf{unitary transformations}.
    \item The \textbf{physical implementation} of quantum gates may vary across hardware (e.g., superconducting qubits, trapped ions), but \textbf{mathematically}, the \textbf{same unitary matrix describes the gate}.
\end{itemize}
\textbf{Quantum gates do not correspond to physical gates} in the classical sense. They are \textbf{abstract operations} realized by \textbf{controlled interactions} in quantum systems.
