\subsection{Single-Qubit Quantum Circuits}

A \definition{Quantum Circuit} is a \textbf{mathematical model} for quantum computation, \textbf{composed of sequential operations applied to qubits}. These operations include initializations, unitary gates, and measurements. In this section, we focus on \hl{circuits involving a single qubit}, which form the foundation for understanding multi-qubit systems.

\highspace
\begin{flushleft}
    \textcolor{Green3}{\faIcon{list-ul} \textbf{Circuit Elements}}
\end{flushleft}
\begin{enumerate}
    \item \important{Initialization}
    \begin{itemize}
        \item Qubit is prepared in a \textbf{known state}, typically $\ket{0}$ or $\ket{1}$.
        \item Essential \textbf{starting point} for any computation.
    \end{itemize}

    \item \important{Gates}
    \begin{itemize}
        \item Single-qubit \textbf{unitary operations} (e.g., X, Z, H, S).
        \item Transform the qubit's state \textbf{reversibly}.
    \end{itemize}

    \item \important{Measurement}
    \begin{itemize}
        \item \textbf{Irreversible process}.
        \item Collapses the qubit to $\ket{0}$ or $\ket{1}$ with probabilities dictated by the quantum state's amplitudes.
    \end{itemize}
\end{enumerate}

\highspace
\begin{flushleft}
    \textcolor{Green3}{\faIcon{tools} \textbf{Quantum Circuit Diagram: Visual Language}}
\end{flushleft}
\begin{itemize}
    \item \textbf{Horizontal lines} represent the \textbf{timeline of a single qubit}.
    \begin{center}
        \begin{quantikz}
            \lstick{$\ket{v_{\text{in}}}$}
            & & & & & & % #6
            \rstick{$\ket{v_{\text{out}}}$}
        \end{quantikz}
    \end{center}
    \item \textbf{Boxes} on the line represent \textbf{gates} (letters) or \textbf{measurements} (tachometer icon).
    \begin{center}
        \begin{quantikz}
            \lstick{$\ket{v_{\text{in}}}$}
            & \gate{A} & \gate{B} & \gate{\cdots} & \gate{N} & \meter{}
        \end{quantikz}
    \end{center}
    \item \textbf{Double lines} (if present) represent \textbf{classical information} (e.g., measurement result).
    \begin{center}
        \begin{quantikz}
            \lstick{$\ket{v_{\text{in}}}$}
            & \gate{A} & \gate{B} & \meter{} & \setwiretype{c}
            \rstick{$\ket{v_{\text{out}}}$}
        \end{quantikz}
    \end{center}
    \item Left to right: The \textbf{circuit evolves over time from left (input) to right (output)}.
\end{itemize}

\newpage

\begin{flushleft}
    \textcolor{Green3}{\faIcon{book} \textbf{Gate Sequence: Serial Gates}}
\end{flushleft}
Consider two gates A and B applied in sequence:
\begin{center}
    \begin{quantikz}
        \lstick{$\ket{v_{\text{in}}}$}
        & \gate{A} & \gate{B} &
        \rstick{$\ket{v_{\text{out}}}$}
    \end{quantikz}
\end{center}
\begin{enumerate}
    \item First apply A, then B.
    \item Mathematically, this is expressed as:
    \begin{equation*}
        v_{\text{out}} = B A v_{text{in}}
    \end{equation*}
    \item The \textbf{combined effect} of the two gates is equivalent to a single gate C = BA:
    \begin{center}
        \begin{quantikz}
            \lstick{$\ket{v_{\text{in}}}$}
            & \gate{BA} &
            \rstick{$\ket{v_{\text{out}}}$}
        \end{quantikz}
    \end{center}
    \begin{equation*}
        v_{\text{out}} = C v_{\text{in}} = BA v_{\text{in}}
    \end{equation*}
\end{enumerate}
\textbf{Matrix multiplication is from right to left}, which may seem counterintuitive. The \textbf{last gate applied} (B) is \textbf{on the left} of the product \textbf{BA}.

\highspace
\begin{examplebox}[: Hadamard followed by Hadamard]
    Let's compute H followed by H to $\ket{0}$:
    
    \begin{center}
        \begin{quantikz}
            \lstick{$\ket{0}$}
            & \gate{H} & \gate{H} &
            \rstick{$\ket{v_{\text{out}}}$}
        \end{quantikz}
    \end{center}

    \begin{enumerate}
        \item Apply H:
        \begin{equation*}
            \begin{array}{rcl}
                H\ket{0} &=&
                \dfrac{1}{\sqrt{2}}
                \cdot
                \begin{pmatrix}
                    1 & 1  \\
                    1 & -1
                \end{pmatrix}
                \cdot
                \begin{pmatrix}
                    1 \\
                    0
                \end{pmatrix}
                \\ [1.5em]
                &=&
                \dfrac{1}{\sqrt{2}}
                \cdot
                \begin{pmatrix}
                    1 \\ 1
                \end{pmatrix}
                \\ [1.5em]
                &=&
                \dfrac{1}{\sqrt{2}}
                \cdot
                \left(
                \begin{pmatrix}
                    1 \\ 0
                \end{pmatrix}
                \cdot
                \begin{pmatrix}
                    0 \\ 1
                \end{pmatrix}
                \right)
                \\ [1.5em]
                &=&
                \dfrac{1}{\sqrt{2}}\left(\ket{0} + \ket{1}\right) = \ket{+}
            \end{array}
        \end{equation*}
        \item Applying H again, we \textbf{restore the original state due to the involution properties} ($H^{2} = I$):
        \begin{equation*}
            H\left(H\ket{0}\right) =
            \dfrac{1}{\sqrt{2}}
            \begin{pmatrix}
                1 & 1 \\
                1 & -1
            \end{pmatrix}
            \cdot
            \dfrac{1}{\sqrt{2}}
            \begin{pmatrix}
                1 \\
                1
            \end{pmatrix}
            =
            \dfrac{1}{2}
            \begin{pmatrix}
                2 \\
                0
            \end{pmatrix}
            =
            \begin{pmatrix}
                1 \\
                0
            \end{pmatrix}
            =
            \ket{0}
        \end{equation*}
    \end{enumerate}
\end{examplebox}
