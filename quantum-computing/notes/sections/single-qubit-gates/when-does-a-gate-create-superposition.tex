\subsection{When Does a Gate Create Superposition?}

In this section, our goal is to determine \textbf{what structural properties a quantum gate must have to create superposition} when applied to a \textbf{basis state} ($\ket{0}$ or $\ket{1}$).

\highspace
Let's consider a generic $2 \times 2$ gate A, given by:
\begin{equation*}
    A =
    \begin{pmatrix}
        a_{11} & a_{12} \\ 0 & a_{22}
    \end{pmatrix}
\end{equation*}
This matrix has a zero in the lower-left corner, suggesting no mixing from $\ket{0}$ to $\ket{1}$ via the first column. In other words:
\begin{equation*}
    \ket{0}
    =
    \begin{pmatrix}
        1 \\ 0
    \end{pmatrix}
    \Rightarrow
    A\ket{0}
    =
    \begin{pmatrix}
        a_{11} & a_{12} \\ 0 & a_{22}
    \end{pmatrix}
    \begin{pmatrix}
        1 \\ 0
    \end{pmatrix}
    =
    \begin{pmatrix}
        a_{11} \\ 0
    \end{pmatrix}
\end{equation*}
That zero blocks (in the lower-left corner) blocks any $\ket{1}$ component from arising when we apply $A$ to $\ket{0}$. So no superposition is created from $\ket{0}$, it stays ``pure''. Note that to create a superposition, the output of $A\ket{0}$ must contain both $\ket{0}$ and $\ket{1}$ components, so both entries are non-zero.

\highspace
\begin{flushleft}
    \textcolor{Green3}{\textbf{Step 1: Apply $A$ to $\ket{0}$}}
\end{flushleft}
\begin{equation*}
    A\ket{0}
    =
    \begin{pmatrix}
        a_{11} & a_{12} \\ 0 & a_{22}
    \end{pmatrix}
    \begin{pmatrix}
        1 \\ 0
    \end{pmatrix}
    =
    \begin{pmatrix}
        a_{11} \\ 0
    \end{pmatrix}
\end{equation*}
The output is purely $\ket{0}$, \textbf{no superposition}.

\highspace
\begin{flushleft}
    \textcolor{Green3}{\textbf{Step 2: Apply $A$ to $\ket{1}$}}
\end{flushleft}
\begin{equation*}
    A\ket{1}
    =
    \begin{pmatrix}
        a_{11} & a_{12} \\ 0 & a_{22}
    \end{pmatrix}
    \begin{pmatrix}
        0 \\ 1
    \end{pmatrix}
    =
    \begin{pmatrix}
        a_{12} \\ a_{22}
    \end{pmatrix}
\end{equation*}
We have both $\ket{0}$ and $\ket{1}$ components, \textbf{potential superposition}. But now the key insight: \textbf{superposition requires that both amplitudes (coefficients) are non-zero}. This only happens if $a_{12} \ne 0$ and $a_{22} \ne 0$.

\highspace
\begin{flushleft}
    \textcolor{Green3}{\textbf{Proof that Gate A cannot create a superposition}}
\end{flushleft}
\begin{proof}[Proof that Gate A cannot create a superposition]
    Quantum gates operate according to the principles of quantum mechanics, unitarity and reversibility. Therefore, Gate A must satisfy the following unitarity condition:
    \begin{equation*}
        A^{\dagger} A = I
    \end{equation*}
    Let's simplify the discussion by assuming $A$ is real, so $A^{\dagger} = A^{T}$. We compute $A^{T}A$:
    \begin{gather*}
        A^{T} =
        \begin{pmatrix}
            a_{11} & 0      \\
            a_{12} & a_{22}
        \end{pmatrix}
        \hspace{2em}
        A =
        \begin{pmatrix}
            a_{11}  & a_{12} \\
            0       & a_{22}
        \end{pmatrix}
        \\
        A^{T} A =
        \begin{pmatrix}
            a_{11}^{2}      & a_{11}a_{12}           \\
            a_{11}a_{12}    & a_{12}^{2} + a_{22}^{2}
        \end{pmatrix}
    \end{gather*}
    Set equal to identity:
    \begin{equation*}
        A^T A =
        \begin{pmatrix}
            1 & 0 \\
            0 & 1
        \end{pmatrix}
    \end{equation*}
    From this, we get:
    \begin{enumerate}
        \item $a_{11}^{2} = 1 \Rightarrow a_{11} = \pm 1$
        \item $a_{11}a_{12} = 0 \Rightarrow a_{12} = 0$
        \item $a_{12}^{2} + a_{22}^{2} = 1 \Rightarrow a_{22} = \pm 1 $
    \end{enumerate}
    The superposition is eliminated by $a_{12} = 0$, because both columns must have non-zero elements.
\end{proof}

\noindent
In conclusion, \important{to create superposition}, \important{a gate must have non-zero elements in both columns}. In fact, gates like Hadamard H, with all non-zero elements, can create superposition. However, diagonal gates like Z or S cannot produce superposition.
