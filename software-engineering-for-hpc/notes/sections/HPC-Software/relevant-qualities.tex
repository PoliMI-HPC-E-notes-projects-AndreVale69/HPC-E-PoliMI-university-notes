\subsection{Relevant Qualities}

For the two categories explained, there are some important characteristics:
\begin{itemize}
    \item For \definition{Compute-intensive applications}:
    \begin{itemize}
        \item \underline{\definition{Correctness}}: the \textbf{software is correct if it satisfies the specifications}, but be careful! Sometimes, modelling reality into a model (using the specifications) isn't the bigger problem. Instead, it is difficult or impossible to show actual correctness concerning reality. For \example{example}, imagine you are building a simulator of a planet lander before you have ever visited it.
        
        \emph{How can you fix this issue?} We can \textbf{check that the software output fulfils the important desired properties} and \textbf{identify and apply a measure of accuracy}.

        
        \item \underline{\definition{Performance}}: it is the \textbf{efficient use of resources}. Again, be careful! Is it a good idea? Is performance improvement always a good idea? Because it is \textbf{not necessarily if}:
        \begin{itemize}
            \item It makes software \textbf{more difficult to read} and \textbf{maintain}
            \item It \textbf{reduces the portability} of software
        \end{itemize}
        
        
        \item \underline{\definition{Portability}}
        
        
        \item \underline{\definition{Maintainability}}: 
    \end{itemize}

    \item For \definition{Data-intensive applications}:
    \begin{itemize}
        \item \underline{\definition{Reliability}}: can be mathematically defined as \textbf{probability of absence of failures for a certain period}. The typical expectations are:
        \begin{itemize}
            \item The application \textbf{performs the expected function}
            \item It can \textbf{tolerate mistakes by users}
            \item It \textbf{prevents unauthorized access} and \textbf{abuse}
        \end{itemize}
        
        
        \item \underline{\definition{Scalability}}: 
        
        
        \item \underline{\definition{Maintainability}}: 
    \end{itemize}
\end{itemize}
In the software, there can be some errors, but a software engineer should be able to recognize the type of failure, faults or defects:
\begin{itemize}
    \item A \definition{defect} is an \textbf{imperfection or deficiency in a work product} where that work product does \underline{not meet its requirements or specifications} and needs to be either repaired or replaced.

    \item A \textbf{defect encountered during software execution} is a \definition{fault} (a fault is a subtype of defect).
    
    \item A \definition{system failure} can be:
    \begin{itemize}
        \item Termination of the ability of a product to perform a required function or its inability to perform within previously specified limits.

        \item An event in which a system or system component does not perform a required function within specified limits.
    \end{itemize}
\end{itemize}