\subsection{Microservice architectural style}

The microservice architectural style is an approach to developing a single application as a suite of \textbf{small services}, each running in its own process and communicating \textbf{lightweight mechanisms}, often an HTTP resource API.

\begin{flushleft}
    \textcolor{Green3}{\textbf{\faIcon{check} Benefits}}
\end{flushleft}
There are two main benefits:
\begin{itemize}
    \item \textbf{Technology heterogeneity}. \textbf{Each service uses its own technology stack}. The technology stack can be selected to fit the task best (e.g. data analysis vs video streaming). The teams can experiment with new technologies within a single microservice (e.g. we can deploy two versions and do A/B testing). Also, no unnecessary dependencies or libraries for each service.
    
    \item \textbf{Scaling}. \textbf{Each microservice can be scaled independently}. Also, identified bottlenecks can be addressed directly. Parts of the system that do not represent bottlenecks can remain simple and unscaled.
\end{itemize}