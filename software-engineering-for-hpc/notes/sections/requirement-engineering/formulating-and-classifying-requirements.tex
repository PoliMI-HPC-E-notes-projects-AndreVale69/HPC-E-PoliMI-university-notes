\subsection{Formulating and classifying requirements}

The requirements can be of three types:
\begin{itemize}
    \item \definition{Functional requirements}: describe the interactions between the system and its environment (independent from implementation). In other words, are the \textbf{main} (functional) \textbf{goals the software has to realize}.
    
    For \emph{\example{example}}: \dquotes{\emph{the word processor shall allow users to search for strings in the text}}; \dquotes{\emph{the system shall allow users to reserve taxis}}.


    \item \definition{Non-functional requirements (NFRs)}: further characterization of user-visible aspects of the system not directly related to functions.
    
    For \emph{\example{example}}: \dquotes{\emph{the response time must be less than 1 second}}; \dquotes{\emph{the server must be available 24 hours a day}}.
    

    \item \definition{Constraints requirements}: imposed by the customer or the environment in which the system operates.
    
    For \emph{\example{example}}: \dquotes{\emph{the implementation language must be Java}}; \dquotes{\emph{the credit card payment system must be able to be dynamically invoked by other systems relying on it}}.
\end{itemize}
We make some observations about non-functional requirements. NFRs predicate on \textbf{external} non-functional qualities, and these qualities must be \textbf{measurable by metrics}. NFRs have some \underline{characteristics}:
\begin{itemize}
    \item Constraints on \textbf{how functionality must be provided to the end user}.

    \item The \textbf{application domain determines} their \textbf{relevance} and their \textbf{prioritization}.
    
    \item Have a \textbf{strong influence on the structure of the system to be built}. For \example{example}, a system may require 24/7 availability. As a result, it is likely to be designed as a replicated system (with redundant components).
\end{itemize}

\begin{examplebox}[: are these requirements?]
    \begin{enumerate}
        \item \dquotes{\emph{The user should insert correct information in the enrolment form}}.
    \end{enumerate}
    This is not a requirement! How can the software prevent a user from entering incorrect information? Specifically, is a domain assumption!

    \begin{enumerate}
        \setcounter{enumi}{1}
        \item \dquotes{\emph{The system should check whether fiscal code are well formed}}.
    \end{enumerate}
    Yes, the software can do this! So it is a requirement.
\end{examplebox}

\begin{examplebox}[: types of requirements]
    Example of \textbf{functional requirements}:
    \begin{itemize}
        \item \dquotes{\emph{The system shall allow users to reserve taxis}}.
        \item \dquotes{\emph{The system should never allowe non-registered users to see the list of other users willing to share a taxi}}.
        \item \dquotes{\emph{The system should guarantee that the reserved taxi picks the user up}}.
    \end{itemize}
    But attention! There is \textbf{unfeasible} (from the perspective of the software to be) \textbf{functional requirements}:
    \begin{itemize}
        \item \dquotes{\emph{The system should guarantee that the reserved taxi picks the user up}}.
    \end{itemize}
    This is because the software cannot guarantee this feature!

    \noindent
    Example of \textbf{non-functional requirements}:
    \begin{itemize}
        \item \dquotes{\emph{The system has to provide a feedback in 5 seconds}}.
        \item \dquotes{\emph{The system should be available 24/7}}.
    \end{itemize}
    Example of \textbf{technical requirements}:
    \begin{itemize}
        \item \dquotes{\emph{The system should be implemented in Java}}.
        \item \dquotes{\emph{The search for the available taxi should be implemented in class \texttt{Controller}}}.
    \end{itemize}
\end{examplebox}

\begin{examplebox}[: bad requirements]
    \begin{enumerate}
        \item \dquotes{\emph{The system shall process all mouse clicks very fast to ensure users do not have to wait}}.
    \end{enumerate}
    The problem here is that it \textbf{cannot be verified (tested)}, because what does \dquotes{fast} mean? Do we have a metric? Can you quantify it?

    \begin{enumerate}
        \setcounter{enumi}{1}
        \item \dquotes{\emph{The user must have Adobe Acrobat installed}}.
    \end{enumerate}
    The problem here is that it \textbf{cannot be achieved by the software system itself}. It is not something that the system has to do. \underline{But} it could be expressed as a domain assumption, so it is not a functional requirement for our software.
\end{examplebox}