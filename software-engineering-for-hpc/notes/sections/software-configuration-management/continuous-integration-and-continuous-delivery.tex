\subsection{Continuous Integration and Continuous Delivery}

\definition{Continuous Integration (CI)} is the \textbf{practice of integrating changes into the main codebase \underline{as often as possible}}. It relies on automation to ensure that proper checks are performed during integration. Typically, commits or pull requests trigger compilation or build, unit and integration testing, code quality checking, and merging into the main repository.

\highspace
However, \definition{Continuous Delivery (CD)} is different from continuous deployment. The former is the \textbf{automatic preparation and tracking of a release into production}. Instead, continuous deployment automatically deploys into the operational environment.

\highspace
This theory can be put into practice using \href{https://docs.github.com/en/actions}{GitHub Actions}. A \textbf{workflow} is a configurable automated process that runs one or more jobs. An \textbf{event} triggers the execution of workflows, and a \textbf{job} is a set of steps in a workflow. A \textbf{step} can be an action (reusable in different contexts) or a shell script. The \textbf{runner} is a \href{https://en.wikipedia.org/wiki/Virtual_machine}{Virtual Machine} in which a job is executed.

\highspace
To create the workflow, we'll need to create a directory called \texttt{.github} and a folder called \texttt{workflows} within it. Some \href{https://github.com/actions/starter-workflows/tree/main/ci}{pre-written examples} are:
\begin{itemize}
    \item \example{Example} of CI workflow using \texttt{cmake}: \href{https://github.com/actions/starter-workflows/blob/main/ci/cmake-single-platform.yml}{cmake-single-platform.yml}

    \item \example{Example} of CI workflow using \texttt{Maven}: \href{https://github.com/actions/starter-workflows/blob/main/ci/maven.yml}{maven.yml}
\end{itemize}
A note about YAML. YAML (YAML Ain't Markup Language) is a data serialization language designed to be human friendly.