\section{Software Configuration Management}

\subsection{Introduction}

\href{https://www.atlassian.com/microservices/microservices-architecture/configuration-management}{Configuration Management (CM)} is a systems engineering process for establishing consistency of a product's attributes throughout its life. In the technology world, configuration management is an IT management process that tracks individual configuration items of an IT system. IT systems are composed of IT assets that vary in granularity. An IT asset may represent a piece of software, or a server, or a cluster of servers. The following focuses on configuration management as it directly applies to IT software assets and software asset CI/CD.

\highspace
\definition{Software Configuration Management} is a \textbf{systems engineering process that tracks and monitors changes to a software systems configuration metadata}. In software development, configuration management is commonly used alongside version control and CI/CD infrastructure (explained later).

\highspace
The basic \textbf{approach to using a decentralized CM} is to have a repository (project) on the server side. 
\begin{itemize}
    \item When we want to work on the project, we clone the repository on the local PC. This workflow is used because we can work offline and on the local project without making critical changes to the repository server side.
    
    \item The local changes can be saved using the commit command and when we are ready to publish our changes, we use the push command to update the repository on the server side.
    
    \item After a push, anyone who has a local copy should make a pull command to update the local project.
\end{itemize}
This workflow can be done using git commands, and a good cheat sheet can be found \href{https://education.github.com/git-cheat-sheet-education.pdf}{here}.