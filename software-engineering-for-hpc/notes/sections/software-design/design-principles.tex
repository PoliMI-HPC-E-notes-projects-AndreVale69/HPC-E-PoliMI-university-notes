\subsection{Design principles}

The following is a list of important design principles in software engineering.

\begin{definitionbox}
    \begin{enumerate}
        \item \textbf{Divide et impera} (also called \emph{divide and conquer})
        
        \item \textbf{Keep the level of abstraction as high as possible}

        \item \textbf{Increase cohesion where possible}

        \item \textbf{Reduce coupling where possible}

        \item \textbf{Design for reusability}

        \item \textbf{Reuse existing designs and code}

        \item \textbf{Design for flexibility}

        \item \textbf{Anticipate obsolescence}

        \item \textbf{Design for portability}

        \item \textbf{Design for testability}

        \item \textbf{Design defensively}
    \end{enumerate}
\end{definitionbox}

\begin{flushleft}
    \large
    \textcolor{Red3}{\textbf{Divide et impera} (\emph{divide and conquer})}
\end{flushleft}
\textbf{Divide and Conquer} is a problem-solving strategy that involves breaking down a complex problem into smaller, more manageable parts, solving each part individually, and then combining the solutions to solve the original problem.

\begin{flushleft}
    \large
    \textcolor{Red3}{\textbf{Keep the level of abstraction as high as possible}}
\end{flushleft}
Ensure that your designs allow you to hide or defer consideration of details, thus reducing complexity. A good abstraction is said to provide \emph{information hiding}. Abstractions allow you to understand the essence of a subsystem without having to know unnecessary details.