\subsection{Test case generation}

\subsubsection{Introduction}

\definition{Testing Workflow} is a \textbf{type of software testing that verifies that each software workflow accurately reflects the given business process}. A workflow is a series of tasks to produce a desired result, usually involving several stages or steps. For any business process, testing these sequential steps is defined as \dquotes{workflow testing}.

\begin{figure}[!htp]
    \centering
    %\includegraphics[width=\textwidth]{img/}
    \caption{Testing Workflow.}
    \label{fig: Testing Workflow}
\end{figure}

\noindent
As we can see in Figure \ref{fig: Testing Workflow}, test cases can be generated in a \textbf{black-box} or \textbf{white-box} manner. The \definition{White Box} is a \textbf{generation based on code features}. Meanwhile, the \definition{Black Box} is a generation \textbf{based on specification features}.

\highspace
Test case generation can be done manually (no need to explain) or automatically. Automatic generation can be done in several ways:
\begin{itemize}
    \item \definition{Combinatorial} testing. It enumerates all possible inputs according to some policy (e.g. smaller to larger).
    
    \item \definition{Concolic Execution} (analyzed in the section \ref{subsubsection: Concolic Execution} on page \pageref{subsubsection: Concolic Execution}). It's a pseudo-random generation of inputs guided by symbolic path properties.
    
    \item \definition{Fuzz} testing (\emph{fuzzing}). It's a pseudo-random generation of inputs, including invalid, unexpected inputs.
    
    \item \definition{Search-based} testing. It explores the space of valid inputs, looking for those that improve some metric (e.g. coverage, diversity, fault inducing capability).
    
    \item \definition{Metamorphic} testing. Generates new test cases based on some metamorphic relationships and other previously defined test cases.
\end{itemize}

\newpage

\subsubsection{Concolic Execution}\label{subsubsection: Concolic Execution}

\definition{Concolic Execution} (concrete-symbolic execution) is an automatic generation of test cases. It's a pseudo-random generation of inputs guided by symbolic path properties.

\highspace
In other words, the concolic execution performs symbolic execution (definition on page \pageref{def: symbolic execution}) \textbf{alongside} concrete execution (concrete inputs). Under the hood, in concolic execution a \textbf{state} combines a symbolic part and a concrete part, used as needed to make progress in the exploration.

\highspace
The steps are then as follows:
\begin{itemize}
    \item Concrete $\xrightarrow{\text{to}}$ Symbolic, derive conditions to explore new paths.
    
    \item Symbolic $\xrightarrow{\text{to}}$ Concrete, simplifying conditions to generate concrete inputs.
\end{itemize}
Let's take an example to clarify the explanation.

\begin{examplebox}
    See the code below:
    \lstinputlisting[language=python]{code/concolic-execution-1.py}
    Let's explore all the paths of the \texttt{m2} method, starting with \textbf{a (random) concrete input} and at the \emph{same time} building the \textbf{symbolic condition of the explored path}. Unfortunately, in some cases we will not be able to solve the symbolic execution. For example, the behavior of the first if-condition (\texttt{z == x}) is unknown in the code. For this reason, we execute it with the identified input cases: given \texttt{y = 7}, run \texttt{bb(7)} and return \texttt{14}. With this arrangement, the condition can be solved.

    The annotation used will be the same as in the Symbolic Execution example on page \pageref{example: Symbolic Execution}.

    \begin{enumerate}
        \item Let's start with defined parameters and no condition state:
        \begin{center}
            \includegraphics[width=.5\textwidth]{img/concolic-execution-1.pdf}
        \end{center}

        \newpage

        \item Call the function with the parameter \texttt{y}, i.e. value \texttt{7}. The return value of the called function will be \texttt{14}.
        \begin{center}
            \includegraphics[width=.5\textwidth]{img/concolic-execution-2.pdf}
        \end{center}

        \item The if condition \texttt{14 == 22} is obviously false, so the path will be \texttt{6} and \texttt{7}. By the way, the condition can also be seen as a logical condition $\lnot \left(\texttt{bb}\left(\texttt{Y}\right) \ne \texttt{X}\right)$. But it can't be solved, so the concolic execution goes from symbolic to concrete $\lnot \left(14 \ne \texttt{X}\right) \equiv \texttt{14} = \texttt{X}$.
        \begin{center}
            \includegraphics[width=.7\textwidth]{img/concolic-execution-3.pdf}
        \end{center}
    \end{enumerate}

    \longline

    \begin{enumerate}
        \setcounter{enumi}{3}
        \item After a random concrete input, we can solve the constraint and start a new exploration. The parameter value of \texttt{X} is now \texttt{14}. This is because we do not want the condition from the previous step to be \emph{not equal}.
        \begin{center}
            \includegraphics[width=.5\textwidth]{img/concolic-execution-4.pdf}
        \end{center}

        \item The result of the function \texttt{bb(7)} is \texttt{14}.
        \begin{center}
            \includegraphics[width=.5\textwidth]{img/concolic-execution-5.pdf}
        \end{center}

        \item Unlike before, the \texttt{if} condition is now true!
        \begin{center}
            \includegraphics[width=.6\textwidth]{img/concolic-execution-6.pdf}
        \end{center}

        \item The evaluation of the third line of code needs no explanation.
        \begin{center}
            \includegraphics[width=.7\textwidth]{img/concolic-execution-7.pdf}
        \end{center}

        \item The path just explored explores the whole code. To try another path, we can negate the condition, but be careful! We only need to negate the second condition, because we already negated the first condition in the first exploration.
        \begin{equation*}
            \texttt{bb}\left(\texttt{Y}\right) = \texttt{X} \land \texttt{X} \le \texttt{Y+10} \: \Longrightarrow \: \texttt{bb}\left(\texttt{Y}\right) = \texttt{X} \land \texttt{X} > \texttt{Y+10}
        \end{equation*}
        \begin{center}
            \includegraphics[width=.8\textwidth]{img/concolic-execution-8.pdf}
        \end{center}
    \end{enumerate}

    \longline

    \begin{enumerate}
        \setcounter{enumi}{8}
        \item Explore the new path using two (random) values that satisfy the previous logical condition.
        \begin{center}
            \includegraphics[width=.5\textwidth]{img/concolic-execution-9.pdf}
        \end{center}

        \item The result of the \texttt{bb} function is \texttt{34}.
        \begin{center}
            \includegraphics[width=.5\textwidth]{img/concolic-execution-10.pdf}
        \end{center}

        \item The \texttt{if} condition is obviously true.
        \begin{center}
            \includegraphics[width=.7\textwidth]{img/concolic-execution-11.pdf}
        \end{center}

        \newpage

        \item We add \texttt{10} to the variable \texttt{Y}.
        \begin{center}
            \includegraphics[width=.7\textwidth]{img/concolic-execution-12.pdf}
        \end{center}

        \item In this case the \texttt{if} condition is false and we complete the exercise because we have explored every possible path!
        \begin{center}
            \includegraphics[width=.8\textwidth]{img/concolic-execution-13.pdf}
        \end{center}
    \end{enumerate}
\end{examplebox}